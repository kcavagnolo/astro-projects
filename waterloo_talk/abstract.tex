From Cosmology to Star Formation: An Hour with Galaxy Clusters
Ken Cavagnolo, Michigan State University

Adiabatic models of hierarchical structure formation predict clusters
of galaxies which should be scaled versions of each other. These
models also predict massive galaxy formation should be continuous
through redshift, resulting in present-day galaxies rich with young
stellar populations. However, observations have long shown that 1)
clusters do not obey simple low-scatter scaling relations, 2) that
massive galaxies are ``red and dead'', and 3) that these galaxies are
less massive than models predict. In this talk I will discuss these
discrepancies as they relate to cosmological and galaxy formation
studies. By focusing on the processes of cluster relaxation and
feedback (e.g. from star formation and active galactic nuclei), I will
draw attention to how a better understanding of intracluster medium
temperature inhomogeneity could lead to better cosmology studies with
clusters, and why ICM entropy may be integral to understanding massive
galaxy formation.
