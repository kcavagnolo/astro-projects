\documentclass[letterpaper,12pt]{article}
\usepackage{graphics,graphicx}
\setlength{\textwidth}{6.5in} 
\setlength{\textheight}{9in}
\setlength{\topmargin}{-0.0625in} 
\setlength{\oddsidemargin}{0in}
\setlength{\evensidemargin}{0in} 
\setlength{\headheight}{0in}
\setlength{\headsep}{0in} 
\setlength{\hoffset}{0in}
\setlength{\voffset}{0in}
\makeatletter
\renewcommand{\section}{\@startsection%
{section}{1}{0mm}{-\baselineskip}%
{0.5\baselineskip}{\normalfont\Large\bfseries}}%
\makeatother
\begin{document}
\pagestyle{plain}
\pagenumbering{arabic}
\begin{center}
\bfseries\uppercase{The Hyperluminous Infrared Galaxy IRAS 09104+4109: An Extreme Brightest Cluster Galaxy}
\end{center}
\section{Abstract}
We propose a study of the cluster environment of IRAS 09104+4109 via
temperature, entropy, and pressure. IRAS
09104+4109 is a rare combination hyperluminous infrared
brightest cluster galaxy (BCG) in a rich cluster, MACS J0913.7+4056. The
environment surrounding the BCG is best described as extreme with
canabalized companion galaxies, stripped gas, and the most powerful radio
source of any IRAS object which is likely blowing
bubbles. Understanding the relationship of the AGN feedback mechanism active in the BCG with the extreme environment
surrounding IRAS 09104+4109 will allow us to fit this unique and
extreme object into the feedback framework and may tell us about a very short-lived but
highly active stage of cluster formation and of the formation of the
central galaxy.
\end{document}
