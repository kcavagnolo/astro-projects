\documentclass[preprint]{aastex}
\usepackage{lscape}
\bibliographystyle{apj}
\shorttitle{README}
\shortauthors{Cavagnolo}
\begin{document}

\title{README}
\author{Kenneth W. Cavagnolo\altaffilmark{1,2}}
\altaffiltext{1}{Department of Physics and Astronomy, Michigan State University, BPS Building, East Lansing, MI 48824}
\altaffiltext{2}{cavagnolo@pa.msu.edu}

%%%%%%%%%%%%%%%%%%%%%%%%%%
\section{Sample Selection}
%%%%%%%%%%%%%%%%%%%%%%%%%%

Our Mathiesen-Evrard (ME) sample, to which we applied the supplied
filter criteria of $0.15 \leq z \leq 0.3$, T$_X \gtrsim 6.5$keV, and
dec $\gtrsim -20\deg$, comes exclusively from the Chandra X-Ray
Telescope's Data Archive (CDA). We initially drew our sample from the
{\textit{ROSAT}} Brightest Cluster Sample (RBC,
\cite{1998MNRAS.301..881E}), RBC Extended Sample (RBCE,
\cite{2000MNRAS.318..333E}), and {\textit{ROSAT}} Brightest 55 Sample
(B55, \cite{1990MNRAS.245..559E}, \cite{1998MNRAS.298..416P}). While
these samples provide a complete, flux-limited sample down to
$1.7\times10^{-11}$ ergs cm$^{-2}$ sec$^{-1}$, the volume-limit of each
sample is not as deep as we desire (RBC z $\leq 0.3$, RBCE z
$\leq 0.41$ and B55 z $\leq 0.15$) to explore a broad dynamical
range. Thus we incorporated clusters from the CDA at a redshift large enough
to where a minimum of $r_{2500}$ ($r_{\Delta_c}$ being the radius at which the
average cluster density is $\Delta_c$ times the critical density of the
Universe, $\rho_c$) is within the Chandra ACIS-S or ACIS-I field of
view. The portion of our sample at z $\gtrsim 0.4$ can also be found in a
combination of the {\textit{Einstein}} Extended Medium Sensitivity Survey
(\cite{1990ApJS...72..567G}), North Ecliptic Pole Survey
(\cite{2006ApJS..162..304H}), {\textit{ROSAT}} Deep Cluster Survey
(\cite{1995ApJ...445L..11R}), and {\textit{ROSAT}} Serendipitous Survey
(\cite{1998ApJ...502..558V}).

%%%%%%%%%%%%%%%%%%%%%%%%%%
\section{Filtering Result}
%%%%%%%%%%%%%%%%%%%%%%%%%%

The result of our CDA search is a total of 229 observations of which
176 are used in the ME-project. Of those 176 observations, 35 meet the
filter criteria and are included in the FITS files. Filtering was
performed using the best-fit values of the r$_{2500}$ minus central
70kpc aperture (apertures explained below). All the data for each
observation are included in the ``RealOnly'' extension of the FITS files.

Seven clusters, Abell 586, Abell 781, Abell 1423, Abell 2409, Abell
2537, RXJ1720, and RXJ2129, were added because the upper bound of the
90\% confidence interval on the best-fit T$_X$ was within 10\% the 6.5keV cut.

A comparsion between literature values and our best-fit values for
T$_X$ finds agreement with the exception of A2537, A2294, and A781. We
have found published values which "book-end" or outright agree with
our temperatures to within 20\% of the 1.6-sigma errors.
The three discrepant clusters are not totally unbelievable:
\begin{description}
\item[A781]:  ours, ~7.0 keV (upper bound); \cite{2005MNRAS.359.1481B}, ~10.8 keV
\item[A2294]: ours, ~8.5 keV (lower bound); \cite{2005MNRAS.359.1481B}, ~7.10 keV
\item[A2537]: ours, ~7.5 keV (lower bound); Don Horner's Thesis, ~6.08 keV
\end{description}

%%%%%%%%%%%%%%%%%%%
\section{Apertures}
%%%%%%%%%%%%%%%%%%%

Spectra were extracted from six different apertures. The circular apertures
were defined using the X-Ray surface brightness peak as the center with
the following radii:
\begin{description}
\item[r$_{max}$]: radius reaching to the vignetted edge of the aimpoint chip.
\item[r$_{break}$]: radius at which the radial surface brightness
profile is $\le 1\sigma$ above the background surface brightness.
\item[r$_{2500}$]: radius at which the average cluster density is
$2500$ times the critical density of the Universe.
\item[r$_{2500}-70$kpc]: the same as r$_{2500}$ except the central
70kpc is excluded resulting in an annulus.
\item[r$_{5000}$]: radius at which the average cluster density is
$5000$ times the critical density of the Universe.
\item[r$_{5000}-70$kpc]: the same as r$_{5000}$ except the central
70kpc is excluded resulting in an annulus.
\end{description}

The naming scheme of the FITS files follows the region used for
spectral extraction. The file names also specify that N$_H$ was fixed (nhfro)
and Fe was free (fefree) during fitting. The file names are as such:
\begin{description}
\item[*rmax-70*.fits] = maximum radius aperture minus central 70kpc
\item[*rbreak-70*.fits] = break radius aperture minus central 70kpc
\item[*r2500-70*.fits] = r$_{2500}$ radius aperture minus central 70kpc
\item[*r2500*.fits] = r$_{2500}$ radius aperture, no central exclusion
\item[*r5000-70*.fits] = r$_{5000}$ radius aperture minus central 70kpc
\item[*r5000*.fits] = r$_{5000}$ radius aperture, no central exclusion
\end{description}

The header for each extension contains information regarding all the
columns included and the corresponding units (where
applicable). Within the FITS files '77' refers to the [0.7-7]keV
bandpass and '27' refers to the [2.0/(1+z)-7.0]keV bandpass. This is
noted in the FITS header.

For our calculations of radii we assumed a flat cosmology of
$\Omega_{M} = 0.3$, $\Lambda = 0.7$, and $H_{0} = 70$ km s$^{-1}$ Mpc$^{-1}$
and adopt the relation from \cite{2002A&A...389....1A}:

\begin{eqnarray}
r_{\Delta_c} &=& 2.71
\cdot \beta_T
\cdot \Delta_z^{-1/2}
\cdot (1+z)^{3/2}
\cdot (\frac{kT_X}{10keV})^{1/2}\\
\Delta_z &=& \frac{\Delta_c \Omega_M}{18\pi^2\Omega_z} \nonumber \\
\Omega_z &=& \frac{\Omega_M (1+z)^3}{[\Omega_M (1+z)^3]+[(1-\Omega_M-\Lambda)(1+z)^2]+\Lambda} \nonumber
\end{eqnarray}

where $r_{\Delta_c}$ is in units of Mpc h$_{70}^{-1}$, $\Delta_c$ is
the assumed density contrast of the cluster at $r_{\Delta_c}$, and
$\beta_T$ is a numerically determined, cosmology-independent ($\lesssim \pm 20\%$)
normalization for the virial relation
$GM/2R = \beta_TkT$. We use $\beta_T = 1.05$ taken from \cite{1996ApJ...469..494E}.

For some extraction regions the central $70$kpc of the cluster was
excised. The central region of our sample
clusters were excluded because this area is dominated by radiative cooling which
greatly impacts the global temperature of a cluster if not
ignored. This radiative cooling is not a part of the self-similar
gravitational heating we are interested in measuring and thus must be
removed.

%%%%%%%%%%%%%%%%%%%%%%%%%%%%%%%%
\section{Additional Information}
%%%%%%%%%%%%%%%%%%%%%%%%%%%%%%%%

The X-Ray luminosities, L$_X$,
for the ME-sample were estimated using $r_{2500}$ 
and the photons between $[0.3-2.0]$keV. These L$_X$ values
should be considered as lower limits given the limited field of view
used for calculation.

The fiducial X-Ray temperature and metallicity were taken from the Ph.D. thesis of
Don Horner\footnote{D. Horner's catalog and Ph.D. thesis are on-line at
http://lheawww.gsfc.nasa.gov/horner/thesis.html}. For clusters not
observed with {\textit{ASCA}} and thus not listed in Horner's thesis,
we used a literature search to locate values. If there were no
published values for a cluster, we approximated a temperature (for the sole purpose of
computing extraction regions as is discussed below)
by recursively extracting a spectrum in the region $0.1-0.2r_{500}$
(based upon an initial guess of T$_{X}$), fitting a temperature, and
recalculating $r_{500}$. This process was repeated until a convergant
temperature was reached. This method of temperature determination has
been employed in other studies, see \cite{2006MNRAS.tmp.1068S} and
\cite{2006ApJS..162..304H} as examples.

The simulations of ME01 and MZ04 used the energy range $[0.3-10.0]$keV
for their spectral analyses, but to make a reliable comparison with
{\textit{Chandra}} data we restrict our study focus to the spectral properties of
clusters in the energy bands $[0.7-7.0]$keV and
$[2.0/(1+z)-7.0]$keV. We exclude data below $0.7$keV to avoid the
effective area and quantum efficiency variations of the ACIS detectors, and
also exclude energies above $7.0$keV where diffuse emission is
dominated by background and the effective area is small. The spectoscopic, emission measure, and
emission-weighted measure temperatures of the simulated
clusters from ME01 and MZ04 were calculated beginning at $0.3$ keV and
$2.0$keV in the cluster rest frame, hence in our study we focus on the observational
rest frame energy $2.0/(1+z)$keV where the factor $1+z$ is due
to cosmological redshift.

Spectra were fit with {\tt XSPEC 11.3.3} (\cite{1996ASPC..101...17A}) (this is recorded in the FITS
header under the 'XSPEC' keyword along with the CIAO and CALDB
versions used in data reduction with the 'CIAO' and 'CALDB' keywords)
using a projected, single-temperature {\tt MEKAL} model
(\cite{1985A&AS...62..197M}, \cite{1986A&AS...65..511M},
\cite{1992SRON}, \cite{1995ApJ...438L.115L})
in combination with the photoelectric
absorption model {\tt WABS} (\cite{1983ApJ...270..119M}). Galactic
absorption values, $N_{H}$, are taken from
\cite{1990ARA&A..28..215D}.

As a gauge of how dense an accreting cool subclump must be to tip the
temperature relation for the two bandpasses of interest, we simulated
spectra for each cluster using the {\tt fakeit} command within {\tt
XSPEC}. Our goal in using simulated spectra is to see how adding gas
of varying temperature and density to the best-fit model for each
observation effects the single-temperature best-fit values.

We began by convolving the observation specific background, ARF, and RMF
with a {\tt WABS(MEKAL$_{1}$+MEKAL$_{2}$)} model for a period equal to
the observation exposure time. In addition, counting statistics are
added into the model. The final output represents the same model
used in fitting the real spectra with the exception of adding a second
component, {\tt MEKAL$_{2}$}, to reflect the fake gas.

We fix $N_H$ to the Galactic value and leave ($Z/Z_{\sun}$) free. The
temperature for the {\tt MEKAL$_{1}$} component was taken from the
best-fit to the real observation over the matching extraction
region. The final integrated count rate for a simulated spectrum
is a function of the normalization for both {\tt MEKAL} components,
and we require the count rate in our simulated spectrum to equal
the count rate in the real spectrum. Thus the normalization, $K_1$, for the {\tt
MEKAL$_{1}$} component must be incremented from the best-fit value by a
prescribed amount, $\xi$, which then allows for a wide variety of
normalizations, $K_2$, for the {\tt MEKAL$_2$} component. We use
values of $\xi = 0.5, 0.6, 0.7, 0.8, 0.85, 0.9, 0.95, 0.96, 0.97,
0.98, 0.99,$ and $1.0$ where the case of $\xi=1.0$ is our
control sample with the normalization and temperature for {\tt
MEKAL$_2$} set to zero. Recall that
$K_i \propto \rho_{i}^2$ so our incrementing of $\xi$ is directly
correlated with probing varying gas densities. For the {\tt MEKAL$_2$} component we vary
the temperature using values T$_2 = 0.5, 0.75, $ and $1.0$keV,
and the metallicity is tied to the value of {\tt MEKAL$_1$}.

The faked spectra were fit using the same method as the real data. The
fake spectra are included in the FITS extension named ``RealandFake''.

\bibliography{cavagnolo}
\end{document}
