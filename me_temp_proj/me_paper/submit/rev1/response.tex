\documentclass[11pt]{article}
\setlength{\topmargin}{-.3in}
\setlength{\oddsidemargin}{-0.1in}
\setlength{\evensidemargin}{-0.1in}
\setlength{\textwidth}{6.7in}
\setlength{\headheight}{0in}
\setlength{\headsep}{0in}
\setlength{\topskip}{0.5in}
\setlength{\textheight}{9.25in}
\setlength{\parindent}{0.0in}
\setlength{\parskip}{1em}
\usepackage{common,graphicx,hyperref,epsfig}

\pagestyle{empty}

\begin{document}

To the editor:

Below is our reply to Referee's Report of ApJ MS \#73823. The referee's
comments are in quotes and \textit{italicized}, while our replies are
in regular font. We have found the referee's comments to be very
helpful in making the focus and discussion of our paper more concise
and thorough.

TITLE: Bandpass Dependence of X-Ray Temperatures in Galaxy Clusters

AUTHORS: Kenneth W. Cavagnolo, Megan Donahue, G. Mark Voit, and Ming
Sun

---------------------------------------------------------------------

\textit{
``General Comments:\\
The Author never discuss the influence of the metalicity in
their measurements. In Table 4 and 5 there are some clusters
with extremely low metalicity values, a significant fraction
has metallicity equal to 0. I would like to see a discussion in
Sec. 6.2 about the dependence of $T_{HBR}$ from the metalicity. If
it is significant I would show it in a new panel of Figure 4.''
}

We have addressed the concern of a $T_{HBR}$ abundance dependence by
including an additional figure, Figure 6, which plots $T_{HBR}$ versus
abundance in Solar units. We find no significant trends for either
$R_{2500-\mathrm{CORE}}$, $R_{5000-\mathrm{CORE}}$, or for the control
sample of simulated spectra.

---------------------------------------------------------------------

\textit{
``Listening the simulated spectra results, the second point states that
to obtained a $T_{HBR}$ greater than 1.1 one need the emission of the
second component greater than 10\% the total emission (this phrase is
present in the abstract). I suppose that this number - 10 \% - depends
on the choice of the second component temperature, $T_2$. To perform
their test the Authors fixed $T_2$ equal to 0.5, 0.7, 1, assuming
implicitly that it is only the very cold gas that causes the increase
of $T_{HBR}$. I believe that $T_{HBR}$ is greater than 1 whenever the
lower component is lower than $T_2$ = 2-3 keV and there is a big
difference with the first component $T_1$. Even if, in the case of $T_2$ =
3 the required emission of the second component could be of order of
20-30\%. I would like to see a larger discussion on this point. My
suggestion is to plot the minimum emission required for the 2nd
component in order to have $T_{BHR}>1.1$ versus the Delta T=(T1-T2)
once fixed $T_2$ = 0.5 or 0.7 or 1.''
}

The referee is correct, the 10\% quoted in the abstract and text was
for the case of $T_2 = 0.75$ keV. The purpose of for using the simulated
spectra was to demonstrate that cold gas contributing a reasonable
proportion of the emission could in fact produce the large values and
dispersion of $T_{HBR}$ we measured. After reading the referee's
comments we have revised the way in which we both produce and analyze
the ensemble of simulated spectra.

The following is a step-by-step explanation of the new procedure:\\
1. We take the best-fit $R_{2500-\mathrm{CORE}}$ temperature (or
$R_{5000-\mathrm{CORE}}$ in the cases where no
$R_{2500-\mathrm{CORE}}$ analysis was performed) and construct a two
temperature component absorbed spectral model: $T_1 + T_2$.\\
2. $T_2$ is incremented over values of 0.5, 0.75, 1.0, 2.0, and 3.0
keV.\\
3. The normalizations of each component, $N_1$ and $N_2$, are
adjusted such that the total spectral normalization is $N = N_1 + \xi
\cdot N_2$.\\
4. $\xi$ is prescribed to be 0.4, 0.3, 0.2, 0.15, 0.1, 0.05, and
$N_1$ is determined iteratively such that the real and simulated
spectral count rates are equal.\\
5. Each simulated spectrum represents the best-fit cluster spectrum
plus a prescribed cool component of some known strength.\\
6. We then take these simulated spectra and fit them following the
exact same routine outlined in the ``Fitting'' section.\\
7. With $T_{2.0-7.0}$ and $T_{0.7-7.0}$ in hand for each simulated
spectrum, we then calculate $T_{HBR}$.\\

The new simulations are used to more thoroughly explore the question:
For a given cluster and for each $T_2$, what is the minimum $\xi$
which produces $T_{HBR} \geq 1.1$ with 90\% confidence? We
recognize that the referee suggested a plot of $\Delta T$ versus
$\xi_{min}$, but we found this figure to not be informative as a
consequence of considerable statistical scatter. We therefore opted to
define for each $T_2$ what mean $\xi_{min}$ was required to produce
a $T_{HBR}$ significantly greater than 1.1. These values are listed in
Table 2.

We also simulated a new set of two-temperature spectra
representing an idealized observation for a spectral model with
$N_{HI} = 3.0\times10^{20}$ cm$^{-2}$, $T_1 = 5$ keV, $Z/Z_{\odot} =
0.3$ and $z = 0.1$. For these spectra we used a finer temperature and
$\xi$ grid of $T_2 = 0.5 \rightarrow 3.0$ in steps of 0.25 keV, and
and $\xi = 0.02 \rightarrow 0.4$ in steps of 0.02 and explored spectra
with counts of 15K, 30K, 60K, and 120K. We use these idealized spectra
as a means of testing the effect of only adding cool gas
(e.g. neglecting background and variations in CCD response epoch,
$N_{HI}$, redshift, abundance, et cetera).

The added discussion answers not only the question
``Are we sensitive to the presence of cool gas?'', but also the
question, ``What general emission-measure strength is required to
change the temperature ratio?'' The entire ``Simulated
Spectra'' section has been revised and expanded as a result. A table
has been added which summarizes the results of this analysis.

---------------------------------------------------------------------

\textit{
``Again on this point, I would like to know why the Authors did not
retrieve $T_2$ directly from the spectra using a 2 MEKAL models''
}

For many of the clusters in our sample, the addition of a separate
second temperature component, $T_2$, when fitting the observational
spectra is not necessary as, based on a comparison of $\chi^2$, a two
temperature model does not produce a better fit for any cluster in our
sample. The addition of a second component also presumes the accreting
gas is isothermal, while it is more likely multi-phase. The motivation
from ME01 is that the cool, accreting subclumps which are biasing the
temperature are not spectroscopically resolved and thus do not make
their presence known by producing poor fits when a single-component
thermal model is used in fitting. The trend here of a common soft
component sufficient to change the temperature measurement in a
single-temperature model is statistical, a result that comes from an
aggregate view of the sample rather than any individual fit.

---------------------------------------------------------------------

\textit{
``In Figure 1, in Figure 4 (1st, 2nd, 3rd, and 5th panel), and in Figure
6 I suggest to differentiate the points representing $R_{2500-CORE}$
from those relative to $R_{5000-CORE}$ (ad ex. using empty and filled
symbols)''
}

We have addressed this lack of clarity by adding additional
plots and panels which are specific to $R_{2500-\mathrm{CORE}}$ and
$R_{5000-\mathrm{CORE}}$. We believe this to be a more transparent
solution and should make discerning between apertures obvious to the
reader.

---------------------------------------------------------------------

\textit{
``During the analysis, some cluster have been excluded. I suggest to
evidence them using a bold font in Table 4 and 5. Moreover, I will
state clearly what is the final number of clusters analyzed after all
the exclusions.''
}

We have italicized the excluded clusters in Table 1, and the excluded
clusters should not have been present in Tables 4 and 5. This was an
oversight on our part and we thank the referee for being so thorough
as to catch this mistake.

We have added mention of the initial and final number of clusters in
our sample in several places:\\
1. the abstract\\
2. last paragraph of the ``Sample Selection'' section\\
3. caption for Fig. 1\\
4. caption for Fig. 4\\
5. caption for Fig. 5\\
6. first paragraph of the ``Summary and Conclusions'' section

---------------------------------------------------------------------

\textit{
``It is clear to me that some clusters present in Table 5 are not
present in Table 4, but I do not understand why the vice-versa should
be true. For ex. why ABELL 0781 is present in Table 4 and not in Table
5?''
}

As stated above, this was an oversight on our part. The clusters which
were in Table 4 and not Table 5 were clusters which were excluded from
our analysis (Abell 781, Abell 1682, CL J1213+0253, CL J1641+4001,
IRAS 09104+4109, Lynx E, MACS J1824.3+4309, MS 0302.7+1658, and RX
J1053+5735) but had not been removed from the table. This was a result
of automatically generating tables and not incorporating them into the
ms.tex file properly. This error has been corrected.

---------------------------------------------------------------------

\textit{
``{INTRODUCTION}\\
The Authors refer to the Mathiesen \& Evrard 2001 as the inspiring
paper for this project. There, it was predicted a 20\% of difference
between temperatures measured in the broad band and in the hard
band. This value is larger than what found from the Authors in this
work. The most probable solution of this discrepancy is due to the
fact that the simulations used in Mathiensen \& Evrard were performed
with a simple treatment of the Intra Cluster Medium physics
comprehensive of only no-radiative description. We know that the
presence, the distribution and the evolution of cold blobs during the
merging phase is strictly connected to the physics involved in
simulations. With only no-radiative physics the cold blobs are not
destroyed during a merging event and they conserve their identity for
longer. This behavior, among others, tell us that no-radiative
simulations are not suited to describe the true universe. I think is
worth to add this comment in the Introduction Section.''
}

We have added a paragraph in the introduction which addresses this
point. We note that radiative cooling is important if simulations are
to properly replicate the observables of clusters, but that one might
still expect to observe the temperature skewing predicted by ME01, but
at a different level than they find from their non-radiative
simulations.

---------------------------------------------------------------------

\textit{
``{SAMPLE SECTION}\\
I will add in the text some general description of the sample as
redshift range, luminosity and temperature range.''
}

We have included a sentence in paragraph three of the ``Sample
Selection'' section which highlights the redshift, luminosity, and
temperature ranges of our sample:\\
1. $z = 0.045-1.24$\\
2. $L_{bol.} = 0.12-100.4\times10^{44} \mathrm{~ergs~s}^{-1}$ ($E = 0.1-100$ keV)\\
3. $T_X = 2.56-19.2$ keV

---------------------------------------------------------------------

\textit{
``{SPECTRAL ANALYSIS: FITTING}\\
Specify the used statistic.''
}

The second to last sentence of the first paragraph in the ``Fitting''
section has been added to clearly specify which statistic was used
during spectral fitting.

---------------------------------------------------------------------

\textit{
``In the spectroscopic analysis the Authors fit their spectra using the
MEKAL model. Is there any reason for which they do not use the APEC
model?''
}

The optically-thin collisional plasma models MEKAL, APEC, and SMITH
have been shown to produce spectral fits which are not significantly
different (e.g. de Plaa, et al. 2006, A\&A, 465, 345; Kim, D. \&
Fabbiano, G. 2004, ApJ, 613, 933). There is arguably a difference
between APEC and MEKAL models when fitting for individual element
abundances (e.g. Sanders, J. S. \& Fabian, A. C.  2006, MNRAS, 371
1483), but as we have shown in this paper, $T_{HBR}$ is insensitive to
the abundance parameter and thus any differences which could arise
from the different spectral models is negligible at best.

---------------------------------------------------------------------

\textit{
``The Authors should clarify the meaning of the dagger symbol used in
Table 4 and Table 5.''
}

We have expanded the discussion in the notes of Tables 4 and 5, in
addition to adding a paragraph of discussion in section ``Fitting'',
to make it more clear for the reader what is meant by ``fit
simultaneously''.

---------------------------------------------------------------------

\textit{
{SPECTRAL ANALYSIS: SIMULATED SPECTRA}\\
In Table 4 and Table 5 there are clusters with 0 metalicity. When you
computed the simulated spectra did you fix the metalicity equal to the
0 value?''
}

Yes, the metallicity is fixed to zero to match the best-fit
metallicity to the observed spectrum for these clusters. While this is
clearly not physical, we nevertheless fixed metallicity to zero to
accurately reconstruct the best-fit spectrum so that we could properly
explore the parameter space for our ensemble of simulated
spectra. While zero is a boundary of the parameter space, it is still
a valid quantity and thus must be included in the analysis. In the
end, the metallicity makes little difference as we show in Figure 6.

---------------------------------------------------------------------

\textit{
``The control sample in Figure 3 (right panels) is not referred and
commented in the text.  Add a reference and explain which control
sample is used (single temperature or double temperature)''
}

The control sample is a single-temperature simulation of the best-fit
spectral model for each cluster. The explanation of what the control
sample is and how it was generated have been re-worded and expanded
for more clarity. We have also added a sentence in the ``Simulated
Spectra'' section which references the far right panels of Figure
3. As discussed above, we have revised our treatment and analysis of
the simulated spectra, but the use of the control sample remains the
same. The text now makes it clear that there is only one control
sample.

---------------------------------------------------------------------

\textit{
``Last point of Section 5.2. I would also mention that for clusters at
high redshifts the difference between broad and hard band decreases
since the lower energy limit is redshift-dependent. At z=0.6, 2/(1+z)
$\approx$ 1.2 and at z=1.37 (the highest redshift consider)
2/(1+z)$\approx$ 0.8''
}

We have added sentences to point \#4 in section ``Simulated Spectra''
which discuss the effect of redshift on $T_{HBR}$.

---------------------------------------------------------------------

\textit{
``{RESULTS AND DISCUSSION}\\
Figure 3. I am surprise to see that there is an object at redshift around
1.1 which seems to have a $T_{HBR}$ around 1.3-1.40. It seems
improbable to me that the temperature in the band [0.7-7] keV is
30-40\% different from that one in the [0.95-7] keV (I am assuming
that the object is at 1.1)''
}

The best-fit value for this point, while impressive, is not
statistically distinguishable from unity. The uncertainty on this
point (cluster RX0910+5422) is large owing to the large upper-limit on
the $T_{2-7}$ temperature.

---------------------------------------------------------------------

\textit{
``Figure 3. I strongly suggest to add two other panels for the
$R_{2500-CORE}$ and $R_{500-CORE}$ where to plot
$T_{HBR}/\sigma$. This would make clear the significancy of the
result.''
}

This was a very good suggestion and we have added a panel each to
Figures 4 and 5 which show the number of standard deviations of
$T_{HBR}$ from unity.

---------------------------------------------------------------------

\textit{
``Section 6.3.1, Paragraph 4. Presenting Figure 5 the Authors state:
"the number of CC clusters falls off more rapidly than the number of
NCC clusters. This effect is dramatically reduced -as expected- if the
core is included." Is this clear from Figure 5? I understood the all
the points in Figure 5 are core-excluded.''
}

This statement, ``...the number of CC clusters falls off more rapidly
than the number of NCC clusters. This effect is dramatically reduced,
as expected, if the core is included...'' is not evidenced by the
figures presented. We find this statement extraneous to the discussion
and so have omitted it from the revision.

---------------------------------------------------------------------

\textit{
``Figure 5. Say in the caption what is the total number of clusters
here considered.''
}

We have added a comment in the caption which notes the number of
clusters plotted for each aperture.

---------------------------------------------------------------------

\textit{
``Section 6.3.2. Figure 6. Same comment of Figure 3: I will split the
Figure in two panel and plot in the second one $T_{HBR}/\sigma$.
Furthermore, at the end of Section 6.3.2 there is a list of clusters:
six are CC and ten unclassified, I will label these objects with
numbers and letters in Figure 6 to make clear their position in
the plot.''
}

We have expanded Figure 5 (which is now Fig. 8 in the revision) to
include panels for each aperture. We have also added a comment in the
caption which notes the number of total clusters and types plotted for
each aperture. Addition of letters and or numbers for the clusters
mentioned at the end of section 6.3.2, while a good idea, resulted in
a figure which was too cluttered, and we find the merit of the figure to
be somewhat compromised as a result. Interested readers can quickly
find the value of $T_{HBR}$ and $T_X$ for each of these clusters via
Tables 3, 4, and 5.

---------------------------------------------------------------------

\textit{
``{SUMMARY}\\
The fact that all the clusters have a $N_{counts}$ greater than 1500
should have been declared also before in Section 5.1''
}

A sentence about the minimum number of counts in the cluster spectra
was added at the end of the first paragraph in section ``Fitting''.

\end{document}
