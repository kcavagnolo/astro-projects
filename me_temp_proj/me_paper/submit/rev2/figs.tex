\clearpage
\begin{figure}
\begin{center}
\includegraphics*[width=\textwidth, trim=0mm 0mm 0mm 0mm, clip]{f1.eps}
\caption{
Bolometric luminosity ($E = 0.1-100$ keV) plotted as a function of
redshift for the 202 clusters which make up the initial
sample. $L_{bol}$ values are limited to the region of spectral
extraction, $R=R_{2500-\mathrm{CORE}}$. For clusters without
$R_{2500-\mathrm{CORE}}$ fits, $R=R_{5000-\mathrm{CORE}}$ fits were
used and are denoted in the figure by empty stars. Dotted lines
represent constant fluxes of $3.0\times10^{-15}$, $10^{-14}$,
$10^{-13}$, and $10^{-12}$ ergs sec$^{-1}$ cm$^{-2}$.
}
\label{fig:lx_z}
\end{center}
\end{figure}
\clearpage

\clearpage
\begin{figure}
\begin{center}
\includegraphics*[width=\textwidth, trim=5mm 0mm 0mm 0mm, clip]{f2.eps}
\caption{
Ratio of target field and blank-sky field count rates in the 9.5-12.0
keV band for all 244 observations in our initial sample. Vertical
dashed lines represent $\pm 20\%$ of unity. Despite the good agreement
between the blank-sky background and observation count rates for most
observations, all backgrounds are normalized.
}
\label{fig:bgd}
\end{center}
\end{figure}
\clearpage

\clearpage
\begin{figure}
\begin{center}
\includegraphics*[width=\textwidth, trim=0mm 0mm 0mm 0mm, clip]{f3.eps}
\caption{
Best-fit temperatures for the hard-band, $T_{2.0-7.0}$, divided by the
broad-band, $T_{0.7-7.0}$, and plotted against the broad-band
temperature. For binned data, each bin contains 25 clusters, with the
exception of the highest temperature bins which contain 16 and 17 for
$R_{2500-\mathrm{CORE}}$ and $R_{5000-\mathrm{CORE}}$, respectively. The
simulated data bins contain 1000 clusters with the last bin having 780
clusters. The line of equality is shown as a dashed line and the
weighted mean for the full sample is shown as a dashed-dotted
line. Error bars are omitted in the unbinned data for clarity. Note
the net skewing of $T_{HBR}$ to greater than unity for both apertures
with no such trend existing in the simulated data. The dispersion of
$T_{HBR}$ for the real data is also much larger than the dispersion of
the simulated data.
}
\label{fig:ftx}
\end{center}
\end{figure}
\clearpage

\clearpage
\begin{figure}
\begin{center}
\includegraphics*[width=\textwidth, trim=0mm 0mm 0mm 0mm, clip]{f4.eps}
\caption{
Plotted here are a few possible sources of systematic uncertainty
versus $T_{HBR}$ calculated for the $R_{2500-\mathrm{CORE}}$
apertures (166 clusters). Error bars have been omitted in several
plots for clarity. The line of equality is shown as a dashed line in
all panels.
{\bfseries\em{(Upper-left:)}} $T_{HBR}$ versus redshift for the
entire sample. The trend in $T_{HBR}$ with redshift is expected as the
$T_{2.0-7.0}$ lower boundary nears convergence with the $T_{0.7-7.0}$
lower boundary at $z \approx 1.85$. Weighted values of $T_{HBR}$ are
consistent with unity starting at $z \sim
0.6$.
{\bfseries\em{(Upper-right:)}} $T_{HBR}$ versus percentage of
spectrum flux which is attributed to the source. We find no trend with
signal-to-noise which suggests calibration uncertainty not is playing
a major role in our results.
{\bfseries\em{(Middle-left:)}} $T_{HBR}$
versus Galactic column density. We find no trend in absorption which
would result if $N_{HI}$ values are inaccurate or if we had improperly
accounted for local soft
contamination.
{\bfseries\em{(Middle-right:)}} $T_{HBR}$ versus the
deviation from unity in units of measurement uncertainty. Recall that
we have used 90\% confidence ($1.6\sigma$) for our analysis.
{\bfseries\em{(Lower-left:)}} $T_{HBR}$ plotted versus
observation start date. The plotted points are culled from the full
sample and represent only clusters which have a single observation and
where the spectral flux is $> 75\%$ from the source. We note no
systematic trend with time.
{\bfseries\em{(Lower-right:)}} Ratio of {\it Chandra}
temperatures derived in this work to {\it ASCA} temperatures taken
from Don Horner's thesis. We note a trend of comparatively hotter {\it
Chandra} temperatures for clusters $> 10$ keV, otherwise our derived
temperatures are in good agreement with those of {\it ASCA}.
}
\label{fig:sysr25}
\end{center}
\end{figure}
\clearpage

\clearpage
\begin{figure}
\begin{center}
\includegraphics*[width=\textwidth, trim=0mm 0mm 0mm 0mm, clip]{f5.eps}
\caption{
Plotted here are a few possible sources of systematic uncertainty
versus $T_{HBR}$ calculated for the $R_{5000-\mathrm{CORE}}$
apertures (192 clusters). Error bars have been omitted in several plots for
clarity. The line of equality is shown as a dashed line in all
panels.
{\bfseries\em{(Upper-left:)}} $T_{HBR}$ versus redshift for the
entire sample. The trend in $T_{HBR}$ with redshift is expected as the
$T_{2.0-7.0}$ lower boundary nears convergence with the $T_{0.7-7.0}$
lower boundary at $z \approx 1.85$. Weighted values of $T_{HBR}$ are
consistent with unity starting at $z \sim
0.6$.
{\bfseries\em{(Upper-right:)}} $T_{HBR}$ versus percentage of
spectrum flux which is attributed to the source. We find no trend with
signal-to-noise which suggests calibration uncertainty is not playing
a major role in our results.
{\bfseries\em{(Middle-left:)}} $T_{HBR}$
versus Galactic column density. We find no trend in absorption which
would result if $N_{HI}$ values are inaccurate or if we had improperly
accounted for local soft
contamination.
{\bfseries\em{(Middle-right:)}} $T_{HBR}$ versus the
deviation from unity in units of measurement uncertainty. Recall that
we have used 90\% confidence ($1.6\sigma$) for our analysis.
{\bfseries\em{(Lower-left:)}} $T_{HBR}$ plotted versus
observation start date. The plotted points are culled from the full
sample and represent only clusters which have a single observation and
where the spectral flux is $> 75\%$ from the source. We note no
systematic trend with time.
{\bfseries\em{(Lower-right:)}} Ratio of {\it Chandra}
temperatures derived in this work to {\it ASCA} temperatures taken
from Don Horner's thesis. We note a trend of comparatively hotter {\it
Chandra} temperatures for clusters $> 10$ keV, otherwise our derived
temperatures are in good agreement with those of {\it ASCA}.
}
\label{fig:sysr50}
\end{center}
\end{figure}
\clearpage

\clearpage
\begin{figure}
\begin{center}
\includegraphics*[width=\textwidth, trim=0mm 0mm 0mm 0mm, clip]{f6.eps}
\caption{
Plotted here is $T_{HBR}$ as a function of metal abundance for 
$R_{2500-\mathrm{CORE}}$, $R_{5000-\mathrm{CORE}}$, and the Control
sample (see discussion of control sample in
\S\ref{sec:simulated}). Error bars are omitted for clarity. The
dashed-line represents the linear best-fit using the bivariate
correlated error and intrinsic scatter (BCES) method of
\cite{1996ApJ...470..706A} which takes into consideration errors on
both $T_{HBR}$ and abundance when performing the fit. We note no trend
in $T_{HBR}$ with metallicity (the apparent trend in the top panel is
not significant) and also note the low dispersion in the control
sample relative to the observations. The striation of abundance arises
from our use of two decimal places in recording the best-fit values
from {\textsc{XSPEC}}.
}
\label{fig:metal}
\end{center}
\end{figure}
\clearpage

\clearpage
\begin{figure}
\begin{center}
\includegraphics*[width=\textwidth, trim=15mm 10mm 0mm 0mm, clip]{f7.eps}
\caption{
Plotted here is the normalized number of cool core (CC) and non-cool core
(NCC) clusters as a function of cuts in $T_{HBR}$. There are 166
clusters plotted in the top panel and 192 in the bottom panel. We have
defined a cluster as having a cool core (CC) when the temperature for
the 50 kpc region around the cluster center divided by the temperature
for $R_{2500-\mathrm{CORE}}$, or $R_{5000-\mathrm{CORE}}$, was less
than one at the $2\sigma$ level. We then take cuts in $T_{HBR}$ at the
$1\sigma$ level and ask how many CC and NCC clusters are above these
cuts. The number of CC clusters falls off more rapidly than NCC
clusters in this classification scheme suggesting higher values of
$T_{HBR}$ prefer less relaxed systems which do not have cool
cores. This result is insensitive to our choice of significance level
in both the core classification and $T_{HBR}$ cuts.
}
\label{fig:cc_ncc_bin}
\end{center}
\end{figure}
\clearpage

\clearpage
\begin{figure}
\begin{center}
\includegraphics*[width=\textwidth, trim=15mm 10mm 0mm 0mm, clip]{f8.eps}
\caption{
$T_{HBR}$ plotted against $T_{0.7-7.0}$ for the
$R_{2500-\mathrm{CORE}}$ and $R_{5000-\mathrm{CORE}}$ apertures. Note
the vertical scales for both panels are not the same. The top and
bottom panes contain 166 and 192 clusters respectively. Only two
clusters -- Abell 697 and MACS J2049.9-3217 -- do not have a $T_{HBR}
> 1.1$ in one aperture and not the other. In both cases though, it was
a result of narrowly missing the cut. The dashed lines are the lines
of equivalence. Symbols and color coding are based on two criteria: 1)
presence of a cool core (CC) and 2) value of $T_{HBR}$. Black stars (6
top, 7 bottom) are clusters with a CC and $T_{HBR}$ significantly
greater than 1.1. Green upright-triangles (21 top, 27 bottom) are NCC
clusters with $T_{HBR}$ significantly greater than 1.1. Blue
down-facing triangles (49 top, 60 bottom) are CC clusters and red
squares (90 top, 98 bottom) are NCC clusters. We have found most, if
not all, of the clusters with $T_{HBR} \gtrsim 1.1$ are merger
systems. Note that the cut at $T_{HBR} > 1.1$ is arbitrary and there
are more merger systems in our sample then just those highlighted in
this figure. However it is rather suggestive that clusters with the
highest values of $T_{HBR}$ appear to be merging systems.
}
\label{fig:ftx_tx}
\end{center}
\end{figure}
\clearpage
