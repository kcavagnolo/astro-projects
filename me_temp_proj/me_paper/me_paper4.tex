%%%%%%%%%%
% Header %
%%%%%%%%%%
\documentclass{emulateapj}
\usepackage{apjfonts,graphicx,here,lscape}
\shorttitle{X-ray Band Dependent Temperature}
\shortauthors{Cavagnolo et al.}
\bibliographystyle{apj}
\newcommand{\tf}{T$_{HFR}$ }
\newcommand{\hard}{T$_{2.0-7.0}$ }
\newcommand{\full}{T$_{0.7-7.0}$ }
\newcommand{\tx}{T$_{X}$}
\newcommand{\rtwf}{R$_{2500-\text{CORE}}$ }
\newcommand{\rfif}{R$_{5000-\text{CORE}}$ }
\newcommand{\chan}{{\textit{Chandra }}}
\newcommand{\asca}{{\textit{ASCA }}}
\newcommand{\xmm}{{\textit{XMM-Newton }}}

%%%%%%%%%%%%%%%%%%%%%
% Title and Authors %
%%%%%%%%%%%%%%%%%%%%%

\begin{document}
\title{Bandpass Dependence of X-ray Temperatures in Galaxy Clusters}
\author{Kenneth W. Cavagnolo\altaffilmark{1,2}, Megan
Donahue\altaffilmark{1}, G. Mark Voit\altaffilmark{1}, and Ming
Sun\altaffilmark{1}}
\altaffiltext{1}{Department of Physics and Astronomy, Michigan State
University, BPS Building, East Lansing, MI 48824}
\altaffiltext{2}{cavagnolo@pa.msu.edu}

%%%%%%%%%%%%
% Abstract %
%%%%%%%%%%%%

\begin{abstract}

We explore the band-dependence of the inferred X-ray temperature of
the intracluster medium (ICM) for 193 well-observed ($N_{photons} >
1500$) clusters of galaxies selected from the \chan Data Archive. If
the hot gas in a cluster is nearly isothermal in the projected region
of interest, the  X-ray temperature inferred from a broad-band
(0.7-7.0 keV) spectrum should be identical to the X-ray temperature
inferred from a hard-band (2.0-7.0 keV) spectrum. However,  if there are
excess soft X-ray photons contributed by cooler lumps of gas or by
non-thermal processes, the temperature of a best-fit 
single-temperature thermal model will be cooler for a broad-band
spectrum than for a hard-band spectrum. Such a diagnostic may
indicate soft contamination even when the X-ray spectrum itself may
not have sufficient signal-to-noise to resolve multiple temperature
components. To test this possible diagnostic, we extract X-ray spectra
from annular regions between $R=70$ kpc and $R=R_{2500}$, $R_{5000}$ for
each cluster in our archival sample. We compare the X-ray temperatures
inferred for single-temperature fits of global spectra when the energy
range of the fit is 0.7-7.0 keV (full) and when the energy range is
2.0/(1+z)-7.0 keV (hard). We find, on average, the hard-band
temperature is significantly higher than the full-band 
temperature. Upon further exploration, we find the ratio
\tf = \hard/\full is enhanced preferentially for clusters which are
known merger systems and for clusters which are nearly isothermal. Clusters
with temperature decrements in their cores (known as cool-core
clusters) tend to have best-fit hard-band temperatures that are
statistically consistent with their best fit full-band
temperatures. We show, using simulated spectra, that this test is
sensitive to cool components with emission measures $> 10\%$ of the
hot component for $2.0 \leq $\tx$ \leq 15.0$ keV and for $z < 0.6$.

We show this test is relatively insensitive to second components
in an individual cluster when the total counts in the spectrum are $<
2500$, but that investigation of a sample of low-count clusters may
still reveal interesting trends. A comparison to the predicted
distribution of temperature ratios and their relationship to putative
cool lumps and/or non-thermal soft X-ray emission in cluster
simulations would be a very useful next step.
\end{abstract}

%%%%%%%%%%%%
% Keywords %
%%%%%%%%%%%%

\keywords{catalogs -- galaxies: clusters: general -- X-rays: galaxies:
clusters -- cosmology: observations -- methods: data analysis}

%%%%%%%%%%%%%%%%%%%%%%%%%%%%%%%%%%%%%%%
\section{Introduction}\label{sec:intro}
%%%%%%%%%%%%%%%%%%%%%%%%%%%%%%%%%%%%%%%

Cluster mass functions and the evolution of the cluster mass function
are useful for measuring cosmological parameters
\citep{1989ApJ...341L..71E, 1998ApJ...508..483W, 2001ApJ...553..545H,
2003PhRvD..67h1304H, 2004PhRvD..70l3008W}. Cluster evolution tests the
effect of dark matter and dark energy on the evolution of dark matter
halos, and therefore provide a complementary and distinct constraint
on cosmological parameters to those tests which constrain
them geometrically (e.g. supernovae \citep{1998AJ....116.1009R,
2007ApJ...659...98R} and baryon acoustic oscillations
\citep{2005ApJ...633..560E}).

However, clusters are a useful cosmological tool only if we can infer
cluster masses -- the fundamental cluster property inferred from
cosmological simulations \citep{1990ApJ...363..349E} -- from
observable properties such as X-ray luminosity, X-ray temperature,
lensing shear, optical luminosity, and galaxy velocity
dispersion. Empirically, the relationship of mass and these observable
properties is well-established \citep{2005RvMP...77..207V}. However,
if we could identify a ``3rd parameter" -- possibly reflecting the
degree of relaxation in the cluster -- we could improve the utility of
clusters as cosmological probes.

Toward this end, we desire to understand the dynamical state of a
cluster beyond identifying which clusters appear to be relaxed and
those which appear to be unrelaxed. More likely, clusters have a
dynamical state which is somewhere in between
\citep{2006ApJ...639...64O, 2006ApJ...650..128K}. The degree to which
a cluster is virialized must be quantified within simulations and then
observationally calibrated with an unbiased statistical sample of
clusters.

One such study to quantify the dynamical state of
clusters was performed by \cite{2001ApJ...546..100M} (hereafter ME01)
using the ensemble of simulations by \cite{1997ApJ...491...38M}. ME01
found clusters which had experienced a recent merger were much cooler
than the cluster mass-observable scaling relations predict. They
attribute this to the presence of cool, spectroscopically unresolved
accreting subclusters which introduce energy into the ICM which
requires a long timescale to dissipate. The consequence being an
under-prediction of cluster binding masses of $15-30\%$
\citep{2001ApJ...546..100M}.

One method of quantifying cluster substructure, which results in the  
underestimate of cluster temperatures (and therefore cluster mass),
employs the ratios of X-ray surface brightness moments to quantify the
degree of relaxation \citep{1995ApJ...452..522B, 1996ApJ...458...27B,
2005ApJ...624..606J}. Although an excellent tool, power ratio suffers
from being aspect dependent, much like other substructure measures
such as axial ratio or centroid variation. ME01 found an auxiliary
measure of substructure which does not depend on perspective and could
be combined with power ratio, axial ratio, and centroid variation to
yield a more robust metric for quantifying a cluster's degree of
relaxation. They found hard-band (2.0$_{rest}$-9.0 keV) temperatures
were $\sim 20\%$ hotter than broad-band (0.5-9.0 keV)
temperatures. The cooler broad-band temperature being caused by
unresolved accreting cool subclusters contributing significant amounts
of line emission to the soft-band ($E<2$ keV). Work by
\cite{2004MNRAS.354...10M} and \cite{2006ApJ...640..710V} have
confirmed this effect in simulated \chan and
{\textit{XMM-Newton}} spectra.

ME01 proposed this temperature skewing, and consequently the
fingerprint of accretion, could be detected utilizing the energy
resolution and soft-band sensitivity of \chan. They
proposed comparing single-phase temperature fits to a hard band and
full band for a sufficiently large sample of clusters covering a broad
dynamical range then checking for a net skew in the ratio of hard and
full temperatures above unity.

In this paper we present our findings of just such a temperature ratio
test using \chan archival data. We find, on average, the
hard-band temperature exceeds the broad-band temperature by $\sim15\%$
in multiple flux-limited samples of X-ray clusters from the \chan
archive. This mean excess is weaker than the $20\%$
predicted by ME01, but is significant at the $11\sigma$ level
nonetheless. Hereafter, we refer to the hard-band to broad-band
temperature ratio as \tf. We also find increasing values of \tf favor
non-cool core systems and mergers. With this study we find reason to
believe \tf is an indicator of a cluster's temporal proximity to
the most recent merger event.

This paper proceeds in the following manner. In \S\ref{sec:selection}
we outline sample selection criteria and \chan
observations selected under these criteria. Data reduction and
handling of the X-ray background is discussed in
\S\ref{sec:data}. Spectral extraction is discussed in
\S\ref{sec:extraction} while fitting and simulated spectra are
discussed in \S\ref{sec:specan}. Results and discussion of our
analysis are presented in \S\ref{sec:r&d}. A final summary of our work
is presented in \S\ref{sec:summary}. For this work we have assumed a
flat $\Lambda$CDM Universe with cosmology $\Omega_{M} = 0.3$,
$\Omega_{\Lambda} = 0.7$, and $H_{0} = 70$ km s$^{-1}$ Mpc$^{-1}$. All
quoted uncertainties are at the 1.6$\sigma$ level (90\% confidence).

%%%%%%%%%%%%%%%%%%%%%%%%%%%%%%%%%%%%%%%%%%%%%%%%
\section{Sample Selection} \label{sec:selection}
%%%%%%%%%%%%%%%%%%%%%%%%%%%%%%%%%%%%%%%%%%%%%%%%

Our sample is selected from observations publicly available in the
\chan X-ray Telescope's Data Archive (CDA). Our initial
selection pass came from the {\textit{ROSAT}} Brightest Cluster Sample
\citep{1998MNRAS.301..881E}, RBC Extended Sample
\citep{2000MNRAS.318..333E}, and {\textit{ROSAT}} Brightest 55 Sample
\citep{1990MNRAS.245..559E, 1998MNRAS.298..416P}. The portion of our
sample at $z \gtrsim 0.4$ can also be found in a combination of the
{\textit{Einstein}} Extended Medium Sensitivity Survey
\citep{1990ApJS...72..567G}, North Ecliptic Pole Survey
\citep{2006ApJS..162..304H}, {\textit{ROSAT}} Deep Cluster Survey
\citep{1995ApJ...445L..11R}, {\textit{ROSAT}} Serendipitous Survey
\citep{1998ApJ...502..558V}, and Massive Cluster Survey
\citep{2001ApJ...553..668E}. We later extended our sample to include
clusters found in the REFLEX Survey \citep{2004A&A...425..367B}. Once
we had a master list of possible targets, we cross-referenced this
list with the CDA and gathered observations where a minimum of
R$_{5000}$ (defined below) is fully within the CCD field of
view.

R$_{\Delta_c}$ is defined as the radius at which the average cluster
density is $\Delta_c$ times the critical density of the Universe,
$\rho_c=3H_{0}^2/8\pi G$. For our calculations of R$_{\Delta_c}$ we
adopt the relation from \cite{2002A&A...389....1A}:

\begin{eqnarray}
R_{\Delta_c} &=& 2.71 \text{ Mpc }
\beta_T
\Delta_{\text{z}}^{-1/2}
(1+z)^{3/2}
(\frac{kT_X}{10 \text{ keV}})^{1/2}\\
\Delta_z &=& \frac{\Delta_c \Omega_M}{18\pi^2\Omega_z} \nonumber \\
\Omega_z &=& \frac{\Omega_M (1+z)^3}{[\Omega_M
(1+z)^3]+[(1-\Omega_M-\Omega_{\Lambda})(1+z)^2]+\Omega_{\Lambda}} \nonumber
\end{eqnarray}
where R$_{\Delta_c}$ is in units of h$_{70}^{-1}$, $\Delta_c$ is
the assumed density contrast of the cluster at R$_{\Delta_c}$, and
$\beta_T$ is a numerically determined, cosmology-independent
($\lesssim \pm 20\%$) normalization for the virial relation $GM/2R =
\beta_TkT$. We use $\beta_T = 1.05$ taken from
\cite{1996ApJ...469..494E}.

The result of our CDA search is a total of 374 observations of which
we use 244 for 193 clusters. The bolometric ($E \sim 0.1-100$ keV)
luminosities, L$_{bol}$, for our sample clusters plotted as a function of
redshift are shown in Figure \ref{fig:lx_z}. These L$_{bol}$ values
are calculated from our best-fit spectral models and are limited to
the region of the spectral extraction (from $R=70$ kpc to
$R=R_{2500}$).

Basic properties of our sample are listed in Table
\ref{tab:sample}. For the sole purpose of defining extraction regions
based on fixed overdensities as discussed in \S\ref{sec:extraction},
fiducial temperatures and redshifts were taken from the Ph.D. thesis
of Don Horner\footnote{Available at
http://asd.gsfc.nasa.gov/Donald.Horner/thesis.html} (all redshifts
confirmed with NED\footnote{http://nedwww.ipac.caltech.edu/}).
We will show later the \asca temperatures are
sufficiently close to the \chan temperatures that R$_{500}$
is reliably estimated within 20\%. Note that R$_{500}$ is only
proportional to T$^{1/2}$, so a 20\% error in the temperature leads to
only a 10\% error in R$_{500}$, which in turn affects our final
results imperceptibly. For clusters not listed in Horner's thesis, we
used a literature search to find previously measured temperatures. If
no published value could be located, we measured the global temperature by
recursively extracting a spectrum in the region 0.1 $<$ r $<$ 0.2
R$_{500}$ fitting a temperature and recalculating R$_{500}$. This
process was repeated until three consecutive iterations produced
R$_{500}$ values which differed by $\leq 1\sigma$. This method of
temperature determination has been employed in other studies, see
\cite{2006MNRAS.tmp.1068S} and \cite{2006ApJS..162..304H} as
examples.

%%%%%%%%%%%%%%%%%%%%%%%%%%%%%%%%%%%%%%%%%%%%%%%%
\section{\chan Data}\label{sec:data}
%%%%%%%%%%%%%%%%%%%%%%%%%%%%%%%%%%%%%%%%%%%%%%%%

%%%%%%%%%%%%%%%%%%%%%%%%%%%%%%%%%%%%%%%%%%%%%%%%%%%%%%%%%%%%%%%
\subsection{Reprocessing and Reduction}\label{sec:reprocessing}
%%%%%%%%%%%%%%%%%%%%%%%%%%%%%%%%%%%%%%%%%%%%%%%%%%%%%%%%%%%%%%%

All datasets were reduced utilizing the \chan Interactive Analysis of
Observations package ({\tt CIAO}) and accompanying Calibration
Database ({\tt CALDB}). Using {\tt CIAO v3.3.0.1} and {\tt
CALDB v3.2.2}, standard data analysis was followed for each
observation to apply the most up-to-date time-dependent gain
correction and charge transfer inefficiency (CTI) correction (when
appropriate) \citep{2000ApJ...534L.139T}.

We visually identified point sources on the image. To automatically
define apertures around most point sources we used the adaptive wavelet 
tool {\tt wavdetect} \citep{2002ApJS..138..185F} on events files
filtered in the energy range $0.3-9.0$ keV as recommended in
\citep{2000SPIE.4012...17J}. We added regions for point
sources which were missed by wavdetect and deleted regions for
spuriously detected ``sources''. We used these regions to define a
point source mask. Spurious sources are typically CCD features (chip
gaps and chip edges) not fully removed when divided by the exposure
map. Each remaining source was then masked out using $2\sigma$
ellipses as calculated from {\tt wavdetect}. This process results in
an events file (at "level 2") that has been cleaned of point sources.

To check for contamination from background flares or periods of
excessively high background, light curve analyses were performed using
Maxim Markevitch's contributed {\tt CIAO} script {\tt
lc\_clean.sl}\footnote{Available at
http://cxc.harvard.edu/contrib/maxim/acisbg/}. Periods with count
rates $\geq 3\sigma$ and/or a factor $\geq 1.2$ of the mean background
level of the observation were removed from the GTI file. As prescribed
by Markevitch's
cookbook\footnote{http://cxc.harvard.edu/contrib/maxim/acisbg/COOKBOOK},
ACIS front-illuminated (FI) chips were analyzed in the $0.3-12.0$ keV
range with time bins of $259.28$ sec in length, and for the ACIS
back-illuminated (BI) chips, $2.5-7.0$ keV energy range with time bins
of $1037.12$ sec.

When a FI and BI chip were both active during an observation, we
compared light curves from both chips to detect long duration,
soft-flares which can go undetected on the FI chips but show up on the
BI chips. While rare, this class of flare must be filtered out of the
data as it introduces a spectral component which artificially
increases the best-fit temperature via a high energy tail. We find
evidence for long duration soft flares in the observations of Abell
1758 \citep{2004ApJ...613..831D}, CL J2302.8+0844, and IRAS
09104+4109. These flares were handled by removing the time period of
the flare from the GTI file.

Determining a cluster ``center'' is essential for the later purpose of
removing cool cores from our spectral analysis (see
\S\ref{sec:extraction}). To determine the cluster center, we
calculated the centroid of the clean, point-source
free level-2 events file filtered to include only photons in the
$0.7-7.0$ keV range. Before centroiding, the events file is exposure
corrected and ``holes'' created by excluding point sources are filled
using interpolated values taken from a narrow annular region just
outside the hole (``holes'' are not filled during spectral extraction
discussed in \S\ref{sec:extraction}). Prior to centroiding we define the emission peak by
heavily binning the image, finding the peak value within a circular
region extending from the peak to the chip edge (defined by the
radius, R$_{max}$), reducing R$_{max}$ by 5\%, reducing the binning by a
factor of two, and finding the peak again. This process is repeated
until the image is unbinned (binning factor of one). We then return to
an unbinned image with an aperture centered on the emission peak with
a radius R$_{max}$ and find the centroid using {\tt CIAO}'s {\tt
dmstat}. The centroid, (x$_c$,y$_c$), for a distribution of $N$ good
pixels with coordinates (x$_i$,y$_j$) and values f(x$_i$,y$_j$) is
defined as:

\begin{eqnarray}
Q &=& \sum_{i,j=1}^N f(x_i,y_i) \\
x_c &=& \frac{\sum_{i,j=1}^N x_i \cdot f(x_i,y_i)}{Q} \nonumber \\
y_c &=& \frac{\sum_{i,j=1}^N y_i \cdot f(x_i,y_i)}{Q} \nonumber
\end{eqnarray}

If the centroid was within 70 kpc of the emission peak the emission
peak is selected as the center, otherwise the centroid is used
as the center. This selection was made to ensure all ``peaky'' cool
cores are at the cluster center thus maximizing their exclusion later
in our analysis. All cluster centers are verified by-eye.

%%%%%%%%%%%%%%%%%%%%%%%%%%%%%%%%%%%%%%%%%%%%%%%%%%%%
\subsection{X-ray Background} \label{sec:background}
%%%%%%%%%%%%%%%%%%%%%%%%%%%%%%%%%%%%%%%%%%%%%%%%%%%%

Because we are attempting to measure a global cluster temperature,
specifically looking for a temperature ratio shift in energy bands
which can be contaminated by the high-energy particle background or
the soft local background, it is important to carefully analyze the
background and subtract it from our resulting spectra.

We use blank-sky observations of the X-ray background from
\cite{2001ApJ...562L.153M} supplied within the CXC {\tt CALDB}. First, we
compare the flux from the diffuse soft X-ray background of the
{\textit{ROSAT}} All-Sky Survey ({\textit{RASS}}) combined
bands R12, R45, and R67 to the 0.7-7.0 keV flux in each extraction
aperture for each observation. RASS combined bands give fluxes for
energy ranges of 0.12-0.28 keV, 0.47-1.21 keV, and 0.76-2.04 keV
respectively corresponding to R12, R45, and R67. For the purpose of
simplifying the subsequent analysis, we discarded observations with an
R45 flux $\geq 10\%$ of the total cluster X-ray flux.

The appropriate blank-sky dataset for each observation was
selected from the {\tt CALDB}, reprocessed exactly as the
observation, and then reprojected using the aspect solutions provided
with each observation. For observations on the ACIS-I array, we
reprojected blank-sky backgrounds for chips I0-I3 plus chips S2 and/or
S3. For ACIS-S observations, we created blank-sky backgrounds only for
the S3 chip plus chips I2 and/or I3. The additional off-aimpoint
chips were included only if they were active during the observation
and had available blank-sky data sets for the observation time
period. For observations which did not have a matching off-aimpoint
blank-sky background in the CALDB a source-free region of the active
chips is located and used for background normalization. Off-aimpoint
chips were cleaned for point sources and diffuse sources using the
method outlined in \S\ref{sec:reprocessing}.

The additional off-aimpoint chips were included in data reduction
since they contain data which is farther from the cluster center and
are therefore more useful in analyzing the observation background. To
normalize the hard particle component we measured fluxes for identical
chips in the blank-sky field and target field in the 9.5-12.0 keV
range. The effective area of the ACIS arrays above 9.5 keV is zero and
thus the collected photons there are exclusively from the particle
background. 

A histogram of the ratios of the 9.5-12.0 keV count rate from an
observation's off-aimpoint chip to that of the observation specific
blank-sky background are presented in Figure \ref{fig:bgd}. The
majority of the observations are in agreement to $\lesssim 20\%$ of
the blank-sky background rate, which is small enough to not affect our
analysis. Even so, we re-normalize all blank-sky backgrounds to match
the observed background at large radii.

Normalization brings the observation background and blank-sky
background into agreement for energies $> 2$ keV, but even after
normalization, typically, there still may exist a soft excess/deficit
associated with the spatially varying soft Galactic
background. Following the technique detailed in
\cite{2005ApJ...628..655V} we construct and fit soft residuals for
this component. For each observation we first compare the 0.3-2.0 keV
flux of the blank-sky field and off-aimpoint field. We then
subtract a spectrum of the blank-sky field from a spectrum of the
off-aimpoint field to create a residual. The residual is fit
with a solar abundance, zero redshift {\tt MeKaL} model
\citep{1985A&AS...62..197M, 1986A&AS...65..511M, 1992SRON,
1995ApJ...438L.115L} where the normalization is allowed to be
negative. The resulting best-fit temperatures for all of the soft
residuals identified here were between 0.2-1.0 keV, which is in
agreement with results of \cite{2005ApJ...628..655V}. The
normalization of this background component was then scaled to the
cluster sky area. The re-scaled component is included as a fixed
background component during fitting of a cluster's spectra.

%%%%%%%%%%%%%%%%%%%%%%%%%%%%%%%%%%%%%%%%%%%%%%%%%%%%
\section{Spectral Extraction} \label{sec:extraction}
%%%%%%%%%%%%%%%%%%%%%%%%%%%%%%%%%%%%%%%%%%%%%%%%%%%%

The simulated spectra calculated by ME01 were analyzed in the energy
ranges $0.5-9.0$ keV and $2.0_{rest}-9.0$ keV, but to make a reliable
comparison with \chan data we restrict our focus to a
full energy band, $0.7-7.0$ keV, and a hard energy band,
$2.0_{rest}-7.0$ keV. We exclude data below $0.7$ keV to avoid the
effective area and quantum efficiency variations of the ACIS
detectors, and exclude energies above $7.0$ keV in which diffuse
emission is dominated by the background and \chan's
effective area is small. We also account for cosmic redshift by
shifting the low-energy boundary of the hard-band from 2.0 keV to
$2.0/(1+z)$ keV (henceforth, assume the 2.0 keV cut is in the rest
frame).

ME01 calculated the relation between T$_{0.5-9.0}$ and T$_{2.0-9.0}$
using apertures of R$_{200}$ and R$_{500}$ in size. While it is
trivial to calculate a temperature out to R$_{200}$ or R$_{500}$ from
a simulation, such a measurement at these scales is extremely
difficult with \chan observations (see
\cite{2005ApJ...628..655V} for a detailed explanation).  However, in a
typical \chan observation, apertures with $R=R_{200}$ or even
$R_{500}$ can not be fully included in the field of view. Thus we
chose to extract spectra from regions with radius R$_{2500}$ when
possible and R$_{5000}$ otherwise. Clusters analyzed only within
R$_{5000}$ are denoted in Table \ref{tab:sample} by a dagger.

The cores of some clusters are dominated by cool cores which affect
the global temperature, therefore we excise the central 70 kpc of each
aperture. These excised apertures are denoted by ``-CORE'' in the
text. Recent work by \cite{2007astro.ph..3504M} has shown excising
0.15 R$_{500}$ rather than 70 kpc reduces scatter in mass-observable
scaling relations. But such a reduction does not effect this work as
the conclusions drawn from our spectral analysis are strongly related
to the uncertainties of \hard and not on the best-fit value of \full
where the effect of a cool core is the strongest.

Although some clusters are not circular in projection, but rather are
elliptical or asymmetric, we find that extracting from a circular
annulus does not significantly change the best-fit values. For another
such example see \cite{2005MNRAS.359.1481B}.

After defining annular apertures, we extracted source spectra from the
target cluster and background spectra from the corresponding
normalized blank-sky dataset. By standard {\tt CIAO} means we created
effective area functions (ARF files) and redistribution matrices (RMF
files) for each cluster using a flux-weighted map (WMAP) across the
entire extraction region. The WMAP was calculated over the energy
range 0.3-2.0 keV to weight calibrations that vary as a function of
position on the chip. Each spectrum was then binned to contain a
minimum of 25 counts per channel.

%%%%%%%%%%%%%%%%%%%%%%%%%%%%%%%%%%%%%%%%%%%%%%%
\section{Spectral Analysis} \label{sec:specan}
%%%%%%%%%%%%%%%%%%%%%%%%%%%%%%%%%%%%%%%%%%%%%%%

%%%%%%%%%%%%%%%%%%%%%%%%%%%%%%%%%%%%%%%%%%%%%%%
\subsection{Fitting} \label{sec:fitting}
%%%%%%%%%%%%%%%%%%%%%%%%%%%%%%%%%%%%%%%%%%%%%%%

Spectra were fit with {\tt XSPEC 11.3.2ag} \citep{1996ASPC..101...17A}
using a single-temperature {\tt MeKaL} model in combination with the
photoelectric absorption model {\tt WABS} \citep{1983ApJ...270..119M}
for Galactic absorption. Galactic absorption values, $N_{HI}$, are taken from
\cite{1990ARA&A..28..215D}. The potential free parameters of the
absorbed thermal model ({\tt WABS(MeKaL)}) are $N_{HI}$, X-ray
temperature (\tx), metal abundance normalized to Solar (elemental ratios taken
from \cite{1989GeCoA..53..197A}), and a normalization
proportional to the integrated emission measure of the cluster. The
spectra from clusters with multiple observations were fit
simultaneously. Results from the fitting are presented in Table
\ref{tab:r2500specfits} and Table \ref{tab:r5000specfits}. No
systematic error is added during fitting and thus all quoted errors
are statistical only.

Additional statistical error is introduced because of the uncertainty
of the soft local background component discussed in
\S\ref{sec:background}. To estimate the sensitivity of our best-fit
temperatures to this uncertainty, we use the differences between the
best-fit \tx based on a model using our best estimate for the background
normalization and best-fit \tx that results when we assume a
background normalization to $\pm1\sigma$ of the best estimate for the
background normalization. The statistical uncertainty of the original
fit and the additional uncertainty inferred
from the range of normalizations to the soft X-ray background
component are then added in quadrature to produce a final error. In
all cases this additional background error on the temperature is less
than 10\% of the total statistical error and therefore represents a
minor inflation of the error budget.

In all fits the metal abundance is a free parameter. When comparing
fits with Galactic column $N_{HI}$ fixed with those where it
was a free parameter, we found neither the goodness of fit per
free parameter nor the best fit \tx were significantly different. Thus
$N_{HI}$ is fixed at the Galactic value with the exception of three
cases: Abell 399 \citep{2004MNRAS.351.1439S}, Abell 520, and Hercules
A. For these three clusters $N_{HI}$ is a free parameter.

After fitting we reject several datasets as their best-fit \hard
had no upper bound in the 90\% confidence interval and thus are
insufficient for our analysis. All fits for the clusters Abell 781,
Abell 1682, CL J1213+0253, CL J1641+4001, IRAS 09104+4109, Lynx E,
MACS J1824.3+4309, MS 0302.7+1658, and RX J1053+5735 are rejected. We
also remove Abell 2550 from our sample after finding it to be a
an anomalously cool (\tx $\sim 2$ keV) ``cluster''. In fact it is a line
of sight set of groups as discussed by \cite{2004cgpc.sympE..31M}.

%%%%%%%%%%%%%%%%%%%%%%%%%%%%%%%%%%%%%%%%%%%%%%%%%%%
\subsection{Simulated Spectra}\label{sec:simulated}
%%%%%%%%%%%%%%%%%%%%%%%%%%%%%%%%%%%%%%%%%%%%%%%%%%%

To quantify the effect a second, cooler gas component has on the fit
of a single-phase spectral model, we create an ensemble of simulated
spectra for our entire sample using the {\tt fakeit} command within
{\tt XSPEC}. With these simulated spectra we are attempting to answer the
question: Given the count level in each observation of our sample, how
bright must a second temperature component be to see it in the 2.0-7.0
keV band over the 0.7-7.0 keV band?  Put another way, we are asking at
what flux ratio a second gas phase skews \tf to greater than unity at
the 1$\sigma$ level.

We began by adding the observation-specific background to a convolved,
absorbed thermal model with two temperature components ({\tt
WABS(MeKaL$_1$+MeKaL$_2$)})for a time period equal to the observation
exposure time and then added Poisson noise. For each realization of an
observation's simulated spectrum we define the {\tt MeKaL$_1$}
component to have the best-fit temperature and metallicity of the
\rtwf, 0.7-7.0 keV fit, and we step the {\tt
MeKaL$_2$} component temperature over the values 0.5, 0.75, and 1.0
keV. The {\tt MeKaL$_2$} metallicity is fixed and set equal to the
metallicity of the {\tt MeKaL$_1$} component.

We adjust the normalization of the simulated two-component spectra to
achieve equivalent count rates to that in the real spectra. We
set the {\tt MeKaL$_1$} normalization to $K_1 = \xi \cdot K_{bf}$
where $K_{bf}$ is the best-fit normalization of the
\rtwf, 0.7-7.0 keV fit and $\xi$ is a preset factor
taking the values 0.8, 0.85, 0.9, 0.95, 0.96, 0.97, 0.98, and
0.99. The {\tt MeKaL$_2$} normalization is determined through an
iterative process to make real and simulated spectral count rates match.

We also simulate a control sample of single-temperature models. The
control sample is a simulated version of the best-fit model. This
control provides us with a check of how often a hard component
temperature might differ from a broad-band temperature statistically
(i.e. if calibration effects are under control).

For each observation, we have 59 total simulated spectra: 35
single-temperature control spectra and 24 two-component simulated
spectra (three second temperatures, each with 8 different $\xi$). Our
resulting simulated spectra ensemble contains 11269 spectra. After
generating all the spectra we follow the same fitting routine detailed
in \S\ref{sec:fitting}.

There are three important results taken away from the analysis of these
simulated spectra:\\
1. Using the control sample, a single temperature component rarely
(frequency $\sim 2\%$) gives a significantly different \full and \hard
temperature. The weighted-mean value for the control sample is 1.00
with a standard deviation of $\pm0.06$ and an unbiased deviation of
the mean of $\pm0.002$. There appears to be additional instrinsic
width of the \tf distribution which is likely associted with
statistical noise of fitting in {\tt XSPEC} (Dupke, private
communication). This result indicates our remaining set of
observations is statistically sound: that a result where \tf
significantly differs from 1.0 cannot result from statistical
fluctuations alone. 
2. A significantly bright second temperature component -- meaning $\xi
\leq 0.90$ with T$_2 = 0.5, 0.75, 1.0 kev$ -- must be present in order
to get significant deviations of \tf from 1.0 as large as 1.1-1.2 as
seen in the real data.
3. As redshift increases gas cooler than 1.0 keV is slowly redshifted
out of the observable X-ray band. As expected, we find from our
simulated spectra for z $> 0.6$ \tf is no longer significantly greater
than unity and a net skew in \tf only shows up under weight averaging
making a cluster by cluster detection of the coolest gas
difficult. For the 14 clusters with z $\geq 0.6$ in our real sample we
can safely conclude we are not overestimating the contribution of cool
gas to the spectra and we are most likely underestimating temperature
inhomogeneity.

%%%%%%%%%%%%%%%%%%%%%%%%%%%%%%%%%%%%%%%%%%%%%%%%
\section{Results and Discussion} \label{sec:r&d}
%%%%%%%%%%%%%%%%%%%%%%%%%%%%%%%%%%%%%%%%%%%%%%%%

%%%%%%%%%%%%%%%%%%%%%%%%%%%%%%%%%%%%%%%%%%%%%%%%%%%%%%%%
\subsection{Temperature Ratios} \label{sec:tfresults}
%%%%%%%%%%%%%%%%%%%%%%%%%%%%%%%%%%%%%%%%%%%%%%%%%%%%%%%%

We estimate a temperature ratio \tf = \hard/\full for each
cluster. We find a clear departure of the mean \tf from unity for
our entire sample at greater than $11\sigma$. The weighted mean values
for our sample are shown in Table \ref{tab:wavg}. Presented in Figure
\ref{fig:ftx} are the binned weighted-means and raw \tf values of
\rtwf and \rfif. Each bin contains 25 clusters with the exception of
the highest temperature bins which have 6 and 20 for the \rtwf and
\rfif, respectively. The peculiar points with \tf $<$ 1 are all
statistically consistent with \tf = 1. In addition, ratios less than
one in the real data occur at nearly the same frequency in the
simulated data, $\sim 0.3\%$, suggesting that these ratios are
consistent with statistical uncertainties in the data. The presence of
clusters where \tf = 1 suggests that the calibration is not the sole
reason for deviations of \tf from 1. We also do not find the
temperature ratio depends on the best-fit full-band temperature.

The uncertainty associated with each value of \tf is dominated by
the larger error in \hard, and on average, $\Delta$\hard$\approx
2.3\Delta$\full. This error interval discrepancy naturally results
from excluding the bulk of a cluster's emission occurring below 2
keV. While choosing a temperature-sensitive cut-off energy for the
hard-band (other than $2.0_{rest}$ keV) might maintain a more consistent
error budget across our sample, we do not find any systematic trend in
the ratio with cluster temperature.

%%%%%%%%%%%%%%%%%%%%%%%%%%%%%%%%%%%%%%%%
\subsection{Systematics} \label{sec:sys}
%%%%%%%%%%%%%%%%%%%%%%%%%%%%%%%%%%%%%%%%

The disagreement between \xmm and \chan cluster temperatures has been
noted in several independent studies,
i.e. \cite{2005ApJ...628..655V,chanxmmdis}. But the source of this
discrepancy is not well understood and efforts to perform
cross-calibration between \xmm and \chan have thus far not been
conclusive. One possible explanation is poor soft calibration of \chan
which may arise from hydrocarbon contamination of the HRMA similar in
nature to the contaminant on the ACIS detectors
\citep{aciscontaminant}. We have considered this possibility in our
study. Soft calibration uncertainty from a contaminant may result in
various systematic trends in \tf, such as trends with full-band
temperature or observation time.

As noted in \S\ref{sec:tfresults} and
seen in Figure \ref{fig:ftx} we find no systematic trend with
temperature either for the full sample or for a sub-sample of single
observation clusters with source percent $> 75\%$. Plotted in Figure
\ref{fig:sys} is \tf versus time for clusters which have a single
observation (multiple observations are fit simultaneously and any time
effect would be washed out) and where the spectral flux is $> 75\%$
from the source (higher S/N clusters will be more effected by
calibration uncertainty). We note no systematic trend with time which
suggests any contamination of \chan's HRMA may not be changing with
time. Our conclusion on this matter is the soft calibration
uncertainty is not playing a dominant role in our results.

Recall ME01 gave us physical motivation for this study and we were not spurred on
by existing literature of temperature discrepancies between
bandpasses. As discussed in \S\ref{sec:relax} we find a correlation
between the dynamical state of a cluster and the value of \tf. Even if
the uncertainty in \full were to drive \tf down by 10\%, the
correlation between clusters with the highest significant values of \tf
and their relaxation state would persist. It may be the case that
we are observing a superposition of effects: 1) physics within the
cluster associated with mergers and 2) uncalibrated soft emission
uncertainty.

Aside from instrumental and calibration effects there may be
additional systematics in our analysis which do not present themselves
on an individual basis but emerge as underlying trends in the sample
as a whole. Three possible sources of these systematics
are signal-to-noise (S/N), redshift selection, and Galactic
absorption. Presented in Figure \ref{fig:sys} are these three
parameters versus \tf. The trend in \tf with redshift is expected
as the 2.0/(1+z) keV hard-band lower boundary nears convergence with
the 0.7 keV full-band lower boundary which occurs at z $\sim 1.85$. We
find no systematic trends with S/N or Galactic absorption, which might
occur if the ratio skew were a consequence of poor count statistics,
inaccurate Galactic absorption, or very poor calibration. Also
shown in panel four of Fig. \ref{fig:sys} is a comparison of the
best-fit results for \rtwf and \rfif (error bars are omitted for
clarity as they all cross the line of equality). For every cluster in
our sample both apertures result in \tf values which are not
significantly different from one another. Our results are robust to
changes in aperture size.

Shown in Figure \ref{fig:sys} is the ratio of \asca
temperatures taken from Don Horner's thesis to \chan
temperatures derived in this work. The spurious point below 0.5 with
very large error bars is MS 2053.7-0449 which has a poorly
constrained \asca temperature of $10.03^{+8.73}_{-3.52}$. Our value of
$\sim 3.5 keV$ for this cluster is in agreement with the recent work of
\cite{2007astro.ph..3156M}. Not all our sample clusters have an \asca
temperature, but a sufficient number (53) are available to make this
comparison reliable. Apertures used in the extraction of \asca spectra
had no core region removed and were substantially larger than
R$_{2500}$. \asca spectra were also fit over a broader energy range
(0.6-10 keV) than we use here. Thus we make comparison with the unexcised
R$_{2500}$, 0.7-7.0 keV temperatures as these get us
the closest to replicating the \asca apertures and energy range. Our
temperatures are in good agreement with those from \asca, but we do
note a trend of comparatively hotter \chan temperatures for T$_{\chan}
> 10$ keV. The clusters with T$_{\chan} > 10$ keV are Abell 1758,
Abell 2163, Abell 2255, and RX J1347.5-1145.

Based on this trend, we test excluding the hottest clusters
(T$_{\chan} > 10$ keV where \asca and \chan disagree) from our sample.
The mean temperature ratio for the sample remains $1.16$ and the error
of the mean increases from $\pm 0.014$ to $\pm 0.015$ for
\rtwf; \tf increases by $0.9\%$ for \rfif from $1.14\pm 0.013$ to
$1.15\pm 0.014$. Our results are not being influenced by the inclusion
of hot clusters.

The concern surrounding disagreement between \asca and \chan is that
we have extracted spectra from apertures which are much too large and
are thus sampling more of the background than the source. This of
course is only a problem for clusters which do not fill the aperture
with emission or have very low surface brightness. As a test case of this
scenario, we reanalyzed MS 2053.7-0449 using apertures defined with a
temperature from the iterative method outlined at the end of
\S\ref{sec:selection}. Recall, aperture radius is changing as
T$^{1/2}$ and even in the case of an extreme temperature mismatch for
a low surface brightness cluster, such as MS 2053.7-0449, we find no
significant change in any of the best-fit temperatures. As we
demonstrated earlier in this section, aperture size is not playing a
significant role in temperature determination in either the hard or full
bands.

%%%%%%%%%%%%%%%%%%%%%%%%%%%%%%%%%%%%%%%%%%%%%%%%%%%%%%%%%%%%%%%%%%%
\subsection{Using \tf to Select for Relaxation} \label{sec:relax}
%%%%%%%%%%%%%%%%%%%%%%%%%%%%%%%%%%%%%%%%%%%%%%%%%%%%%%%%%%%%%%%%%%%

\subsubsection{Cool Core Versus Non-Cool Core}\label{sec:ccncc}

The process of virialization may robustly result in the formation of a
cool core \citep{2006ApJ...640..673O}. Flux-limited surveys have found
the prevalence of CCs to be $34-60\%$ \citep{1997MNRAS.292..419W,
1998MNRAS.298..416P, 2005MNRAS.359.1481B, 2007A&A...466..805C}
depending upon classification criteria, completeness, and possible
selection biases. As discussed in \S\ref{sec:intro}, ME01 give us
reason to believe the observed skewing of \tf to greater than unity is
related to the dynamic state of a cluster. We thus seek to identify
which clusters in our sample have cool cores (CC), which have (NCC),
and if the presence or absence of a cool core is correlated to \tf. We
also ask about the number of cool core (CC) and non-cool core (NCC)
clusters as a function of \tf. Recall that we exclude the cool core
itself during spectral extraction.

To identify a CC, we extract spectra for a 50 kpc region
surrounding the cluster center, we then define a temperature
decrement 

\begin{equation}
T_{dec} = T_{50}/T_{cluster}
\end{equation}

where T$_{50}$ is the temperature of the inner 50 kpc
and T$_{cluster}$ is the \rtwf temperature
(clusters without \rtwf fits are excluded from this
analysis). If T$_{dec}$ is 2$\sigma$ less than one, we define the
cluster as having a CC, otherwise the cluster is defined as NCC. We
find CCs in 35\% of our sample and if we lessen the significance
needed for CC classification from 2$\sigma$ to 1$\sigma$, we find 46\%
of our sample clusters have CCs. The frequency of CCs in our study is
consistent with other more detailed studies of CC/NCC populations.

When fitting for T$_{50}$ we alter the method outlined
in \S\ref{sec:fitting} to use {\tt XSPEC}'s modified Cash statistic
\citep{1979ApJ...228..939C}, {\tt cstat}, on ungrouped
spectra. This choice is made because the distribution of counts per
bin in low count spectra is not Gaussian but instead
Poisson. As a result, the best-fit temperature using $\chi^2$ is
typically cooler \citep{1989ApJ...342.1207N, 2007A&A...462..429B}. We
have explored this systematic in {\bfseries\em{all}} our fits and
only found it to be significant in the lowest count spectra of the
inner 50 kpc apertures discussed here. But for consistency, we fit all
inner 50 kpc spectra using the modified Cash statistic.

With each cluster classified, we then take cuts in \tf 
and ask how many CC and NCC clusters are above these cuts. 
Figure \ref{fig:cc_ncc_bin} shows the normalized number of CC and NCC
clusters as a function of cuts in \tf. If \tf were insensitive to
the dynamical state of a cluster we would expect, for normally
distributed \tf values, to see the number of CC and NCC clusters
monotonically decreasing. However, the number of CC clusters falls off
more rapidly than the number of NCC clusters. This effect
is dramatically reduced -- as expected -- if the core is included. All
curves demonstrate linear behavior in the region $0.9 \leq
$T$_{HFR,cut} \leq 1.2$. The slope of the curve's linear segment for
\rtwf CC is $-3.50\pm 0.06$ and for NCC is $-2.23\pm
0.10$. For \rfif the CC slope is $-2.70\pm 0.10$ and
for NCC is $-2.08\pm 0.23$. All fits have excellent goodness of fit.
If the presence of a CC is indicative of a cluster's advancement
towards complete virialization, then the significantly steeper
fall-off of CC clusters as a function of \tf we observe indicates
higher values of \tf are associated with a less virialized state. This
result is insensitive to our choice of significance level in the core
classification.

Because of the CC/NCC definition we've selected, our identification of
CCs and NCCs is only as robust as the errors on T$_{50}$ allow. One can
thus ask the question, does our loose definition bias us towards
finding more NCCs than CCs? To explore this question we simulated 20
spectra for each observation following the method outlined in
\S\ref{sec:simulated} for the control sample but using the inner 50
kpc spectral best-fit values as input. For each simulated spectrum we
then calculate the temperature decrement as described above and
re-classify the cluster as having a CC or NCC. Using the new set of
mock classifications we can assign a reliability factor to each real
classification which is simply the fraction of mock classifications
which agree with the real classification. A value of 1.0 indicates
complete agreement and 0.0 meaning no agreement.

When we remove clusters with reliability factors less than 0.9 and
repeat the analysis above, we find no significant change in the number
of CC clusters as a function of \tf.

Recall for z $\geq 0.6$ cluster gas below 1 keV is being shifted out
of the observable band (\S\ref{sec:simulated}). Thus we are likely not
detecting ``weak'' CCs in the highest redshift clusters of our
sample and thus they would be classified as NCCs. When we exclude
the 14 clusters at z $\geq 0.6$ from this core classification analysis
we find no significant change in our result.

\subsubsection{Mergers Versus Non-Mergers}\label{sec:merge}

We further define a subclass of clusters as known mergers. Known
mergers are clusters which have been identified as mergers in the
cluster literature. From Figure \ref{fig:ftx_tx} we can see clusters
exhibiting the highest significant values of \tf are all ongoing or
recent mergers. At the 2$\sigma$ level, we find increasing values of
\tf favor merger systems with NCCs over relaxed, CC clusters. Mergers
have left a spectroscopic imprint on the ICM which is predicted by
ME01 and which we observe in our sample.

To further investigate \tf we list the identities of 
clusters with T$_{frac} > 1.1$ at the 1$\sigma$ level. These are
presented in Table \ref{tab:tf11}. Of the 29 clusters, only six have
CCs. Three of those, MKW3S, 3C 28.0, and RX J1720.1+2638 have their
apertures centered on the bright, dense cores in confirmed
mergers. Two more clusters, Abell 2384 and RX J1525+0958, while
unconfirmed mergers have morphologies which are consistent with
powerful ongoing mergers. Abell 2384 has a long gas tail extending out
to a gaseous clump which has presumably passed through the cluster
recently. RXJ1525 has a core which is shaped like an arrowhead and is
reminiscent of the bow shock seen in 1E0657-56. Abell 907 is the
remaining CC of the six listed. It has no signs of being a merger
system but the highly compressed surface brightness contours to the
west of the core are indicative of a prominent cold
front which are tell-tale signs of a subcluster merger event
\citep{2007PhR...443....1M}.

The unclassified systems -- RX J0439.0+0715, MACS J2243.3-0935, MACS
J0547.0-3904, ZWCL 1215, MACS J2311+0338, and Abell 267 -- have NCCs and
X-ray morphologies consistent with an ongoing or post-merger
scenario. Two clusters, Abell 1204 and MACS J1427.6-2521, show no
signs of recent or ongoing merger activity, however, they reside at
the bottom of the arbitrary \tf cut, and as evidenced by Abell 401 and
Abell 1689, exceptional spherical symmetry is no guarantee of
relaxation.

Returning to the discussion in \S\ref{sec:ccncc}, we find no cluster
with \tf $> 1.1$ has a reliability factor $< 0.9$. The correlation of
\tf and a cluster's evolutionary state is robust.

%%%%%%%%%%%%%%%%%%%%%%%%%%%%%%%%%%%%%%%%%%%%%%%%%%%%%
\section{Summary and Conclusions}\label{sec:summary}
%%%%%%%%%%%%%%%%%%%%%%%%%%%%%%%%%%%%%%%%%%%%%%%%%%%%%

We have explored the band-dependence of the inferred X-ray temperature
of the intracluster medium (ICM) for 193 well-observed ($N_{counts} >
1500$) clusters of galaxies selected from the \chan data archive.

We extracted spectra from the annulus between $R=70$ kpc and
$R=R_{2500}$, R$_{5000}$ for each cluster. We compare the X-ray
temperatures inferred for single-temperature fits to global spectra
when the energy range of the fit is 0.7-7.0 keV (full) and when the
energy range is $2.0/(1+z)$-7.0 keV (hard). We find that, on
average, the hard-band temperature is significantly higher than
the full-band temperature. Upon further exploration, we find that
the ratio \tf is enhanced preferentially for clusters which are known
merger systems and for clusters which are isothermal. Clusters with
temperature decrements in their cores (known as as cool-core clusters)
tend to have best-fit hard-band temperatures that are statistically
consistent with their best fit full-band temperatures.

As was the original motivation from ME01, we suggest \tf can be
utilized as an indicator for the degree of relaxation/virialization. A
test for this prediction can be made with simulations by tracking \tf
during hierarchical assembly of clusters.

If \tf is correlated with a cluster's degree of relaxation then along
with other methods of substructure measure, such as power ratio, axial
ratio, and centroid shift, we can assemble a powerful metric for
measuring a cluster's expected deviation from mean scaling relations
such as L$_X$-\tx and L$_X$-M. Because \tf is aspect independent
this metric will have two components, spatial and spectroscopic,
by adding spectroscopic information to our understanding of cluster
virialization we may be able to reduce the scatter introduced into
observational studies of scaling relations which will in turn yield
smaller uncertainties in cosmological studies -- specifically for
those studies of dark energy which require use of a mass
threshold. However, to reach such a lofty goal we need to expand our
observational sample and begin getting information about the
underlying distribution of \tf and the intrinsic scatter in \tf which
was not possible with this sample.

%%%%%%%%%%%%%%%%%%%
% Acknowledgments %
%%%%%%%%%%%%%%%%%%%

\acknowledgements
This work was supported through \chan grant
XXX-YYY-123456. We thank the \chan X-Ray Observatory Science Center,
operated for the National Aeronautics and Space Administration (NASA)
by the Smithsonian Astrophysical Observatory (SAO), for maintaining
the \chan Data Archive (CDA) and for supplying and supporting the
\chan Interactive Analysis of Observations (CIAO) software and
Calibration Database (CALDB). This project also utilized the NASA/IPAC
Extragalactic Database (NED), which is operated by the Jet Propulsion
Laboratory, California Institute of Technology, under contract with
NASA, and of NASA's Astrophysical Data System Bibliographic
Services. We also extend a special thanks to Keith Arnaud for
personally providing support for {\tt XSPEC}, obtained from the High
Energy Astrophysics Science Archive Research Center (HEASARC) and
supported by NASA's Goddard Space Flight Center.

%%%%%%%%%%%%%%%%
% Bibliography %
%%%%%%%%%%%%%%%%

\bibliography{cavagnolo}

%%%%%%%%%%%%%%%%%%%%%
% Figures and Tables%
%%%%%%%%%%%%%%%%%%%%%

\clearpage
\begin{figure}[htp]
  \begin{center}
    \begin{minipage}[htp]{0.9\linewidth}
      \includegraphics*[width=\textwidth, trim=15mm 10mm 10mm 10mm, clip]{beta.eps}
      \caption{Surface brightness profiles for clusters requiring a
        $\beta$-model fit for deprojection (discussed in
        \S\ref{sec:beta}). The best-fit $\beta$-model for each cluster
        is overplotted as a dashed line. The discrepancy between the
        data and best-fit model for some clusters results from the
        presence of a compact X-ray source at the center of the
        cluster. These cases are discussed in Appendix
        \ref{app:beta}.}
      \label{fig:betamods}
    \end{minipage}
  \end{center}
\end{figure}
\clearpage
\begin{figure}[htp]
  \begin{center}
    \begin{minipage}[htp]{0.9\linewidth}
      \includegraphics*[width=\textwidth, trim=5mm 0mm 5mm 5mm, clip]{itplflat_rat.eps}
      \caption{Ratio of best-fit \kna\ for the two treatments of
        central temperature interpolation (see \S\ref{sec:temppr}):
        (1) temperature is free to decline across the central density
        bins ($\Delta T_{center} \ne 0$), and (2) the temperature
        across the central density bins is isothermal ($\Delta
        T_{center} = 0$). Filled black squares are clusters for which
        the \kna\ ratio is inconsistent with unity.}
      \label{fig:kcomp}
    \end{minipage}
  \end{center}
\end{figure}
\clearpage
\begin{figure}[htp]
  \begin{center}
    \begin{minipage}[htp]{0.9\linewidth}
      \includegraphics*[width=\textwidth, trim=5mm 0mm 5mm 5mm, clip]{k0res.eps}
      \caption{Best-fit \kna\ vs. redshift. Some clusters have
        \kna\ error bars smaller than the point. The clusters with
        upper-limits ({\it{black points with downward arrows}}) are:
        A2151, AS0405, MS 0116.3-0115, and RX J1347.5-1145. The
        numerically labeled clusters are: (1) M87, (2) Centaurus
        Cluster, (3) RBS 533, (4) HCG 42, (5) HCG 62, (6) SS2B153, (7)
        A1991, (8) MACS0744.8+3927, and (9) CL J1226.9+3332. For
        CLJ1226, \cite{2007ApJ...659.1125M} found best-fit $\kna = 132
        \pm 24 \ent$ which is not significantly different from our
        value of $\kna = 166 \pm 45 \ent$. The lack of $\kna < 10
        \ent$ clusters at $z > 0.1$ is most likely the result of
        insufficient angular resolution (see \S\ref{sec:angres}).}
      \label{fig:k0res}
    \end{minipage}
  \end{center}
\end{figure}
\clearpage
\begin{center}
  \begin{figure}[htp]
    \begin{minipage}[htp]{0.5\linewidth}
      \includegraphics*[width=\textwidth, trim=28mm 7mm 30mm 17mm, clip]{curvk0.eps}
    \end{minipage}
    \begin{minipage}[htp]{0.5\linewidth}
      \includegraphics*[width=\textwidth, trim=28mm 7mm 30mm 17mm, clip]{nbins_k0.eps}
    \end{minipage}
    \begin{minipage}[htp]{0.5\linewidth}
      \includegraphics*[width=\textwidth, trim=28mm 7mm 30mm 17mm, clip]{texpk0.eps}
    \end{minipage}
    \begin{minipage}[htp]{0.5\linewidth}
      \includegraphics*[width=\textwidth, trim=28mm 7mm 30mm 17mm, clip]{ntxbins_k0.eps}
    \end{minipage}
    \caption{Plots of possible systematics versus best-fit \kna.
      {\it{Top left:}} Best-fit \kna\ plotted versus average curvature
      of the corresponding entropy profile (see eq. \ref{eqn:avgcurv})
      There is no trend between these two quantities suggesting that
      \kna\ is not heavily influenced by the total shape of the
      entropy profile. {\it{Top right:}} Best-fit \kna\ plotted versus
      number of bins in the entropy profile which were used during
      fitting. Again, no trend is found. {\it{Bottom left:}} Best-fit
      \kna\ plotted versus the total used exposure time for each
      cluster. No trend is found. {\it{Bottom right:}} Best-fit
      \kna\ plotted versus the number of bins in the temperature
      profile for each cluster. As expected, fewer $\Tx(r)$ does not
      correlate with \kna.}
    \label{fig:sys}
  \end{figure}
\end{center}
\clearpage
\begin{center}
  \begin{figure}[htp]
    \begin{minipage}[htp]{0.5\linewidth}
      \includegraphics*[width=\textwidth, trim=28mm 7mm 30mm 17mm, clip]{splots_allt.eps}
    \end{minipage}
    \begin{minipage}[htp]{0.5\linewidth}
      \includegraphics*[width=\textwidth, trim=28mm 7mm 30mm 17mm, clip]{splots_tle4.eps}
    \end{minipage}
    \begin{minipage}[htp]{0.5\linewidth}
      \includegraphics*[width=\textwidth, trim=28mm 7mm 30mm 17mm, clip]{splots_gt4tle8.eps}
    \end{minipage}
    \begin{minipage}[htp]{0.5\linewidth}
      \includegraphics*[width=\textwidth, trim=28mm 7mm 30mm 17mm, clip]{splots_tgt8.eps}
    \end{minipage}
    \caption{Composite plots of entropy profiles for varying cluster
      temperature ranges. Profiles are color-coded based on average
      cluster temperature. Units of the color bars are keV. The solid
      line is the pure-cooling model of \cite{voitbryan}, the dashed
      line is the mean profile for clusters with $\kna \le 50 \ent$,
      and the dashed-dotted line is the mean profile for clusters with
      $\kna > 50 \ent$. {\it{Top left:}} This panel contains all the
      entropy profiles in our study. {\it{Top right:}} Clusters with
      $kT_X < 4$ keV. {\it{Bottom left:}} Clusters with $4\keV < kT_X
      < 8\keV$. {\it{Bottom right:}} Clusters with $kT_X > 8$
      keV. Note that while the dispersion of core entropy for each
      temperature range is large, as the $kT_X$ range increases so to
      does the mean core entropy.}
    \label{fig:splots}
  \end{figure}
\end{center}
\clearpage
\begin{figure}[htp]
  \begin{center}
    \begin{minipage}[htp]{0.9\linewidth}
      \includegraphics*[width=\textwidth, trim=20mm 10mm 10mm 10mm, clip]{k0hist.eps}
      \caption{{\it{Top panel:}} Histogram of best-fit \kna\ for all
        the clusters in \accept. Bin widths are 0.15 in log space.
        {\it{Bottom panel:}} Cumulative distribution of \kna\ values
        for the full sample. The distinct bimodality in \kna\ is
        present in both distributions, which would not be seen if it
        were an artifact of the histogram binning. A KMM test finds
        the \kna\ distribution cannot arise from a simple unimodal
        Gaussian.}
      \label{fig:k0hist}
    \end{minipage}
  \end{center}
\end{figure}
\clearpage
\begin{figure}[htp]
  \begin{center}
    \begin{minipage}[htp]{0.9\linewidth}
      \includegraphics*[width=\textwidth, trim=20mm 10mm 10mm 10mm, clip]{hifl_k0hist.eps}
      \caption{{\it{Top panel:}} Histogram of best-fit \kna\ values
        for the primary \hifl\ sample. Bin widths are 0.15 in log
        space.  {\it{Bottom panel:}} Cumulative distribution of
        best-fit \kna\ values. The distinct bimodality seen in the
        full \accept\ sample (Fig. \ref{fig:k0hist}) is also present
        in the \hifl\ subsample and shares the same gap between the
        low-entropy peak at 10-20 \ent\ and the high-entropy peak at
        100-200 \ent. That bimodality is present in both samples is
        strong evidence it is not a result of an unknown archival
        bias.}
      \label{fig:hiflk0}
    \end{minipage}
  \end{center}
\end{figure}
\clearpage
\begin{figure}[htp]
  \begin{center}
    \begin{minipage}[htp]{0.8\linewidth}
      \includegraphics*[width=\textwidth, trim=20mm 10mm 10mm 10mm, clip]{t0.eps}
    \end{minipage}
    \begin{minipage}[htp]{0.8\linewidth}
      \includegraphics*[width=\textwidth, trim=20mm 10mm 10mm 10mm, clip]{k0cool.eps}
    \end{minipage}
    \caption{{\it{Top panel:}} Log-binned histogram and cumulative
      distribution of best-fit core cooling times, $t_{c0}$
      (eqn. \ref{eqn:tc0}), for all the clusters in \accept. Histogram
      bin widths are 0.2 in log space. {\it{Bottom panel:}} Log-binned
      histogram and cumulative distribution of core cooling times
      calculated from best-fit \kna\ values, $t_{c0}(\kna)$
      (eqn. \ref{eqn:tck0}), for all the clusters in
      \accept. Histogram bin widths are 0.2 in log space. The
      bimodality we observe in the \kna\ distribution is also present
      in best-fit $t_{c0}$. However, the gaps between the two
      populations of $t_{c0}$ and $t_{c0}(\kna)$ differ by $\sim 0.3$
      Gyrs which may be an artifact of the binning.}
    \label{fig:t0}
  \end{center}
\end{figure}



%%%%%%%%%%
% Tables %
%%%%%%%%%%

\clearpage
\LongTables
\begin{deluxetable}{lcccccccc}
\tablewidth{0pt}
\tabletypesize{\scriptsize}
\tablecaption{Summary of Sample\label{tab:sample}}
\tablehead{\colhead{Cluster} & \colhead{Obs.ID} & \colhead{R.A.} & \colhead{Dec.} & \colhead{ExpT} & \colhead{Mode} & \colhead{ACIS} & \colhead{$z$} & \colhead{$L_{bol.}$}\\
\colhead{ } & \colhead{ } & \colhead{hr:min:sec} & \colhead{$\degr:\arcmin:\arcsec$} & \colhead{ksec} & \colhead{ } & \colhead{ } & \colhead{ } & \colhead{$10^{44}$ ergs s$^{-1}$}\\
\colhead{{(1)}} & \colhead{{(2)}} & \colhead{{(3)}} & \colhead{{(4)}} & \colhead{{(5)}} & \colhead{{(6)}} & \colhead{{(7)}} & \colhead{{(8)}} & \colhead{{(9)}}
}
\startdata
1E0657 56 & \dataset [ADS/Sa.CXO\#obs/03184] {3184} & 06:58:29.627 & -55:56:39.79 & 87.5 & VF & I3 & 0.296 & 52.48\\
1E0657 56 & \dataset [ADS/Sa.CXO\#obs/05356] {5356} & 06:58:29.619 & -55:56:39.35 & 97.2 & VF & I2 & 0.296 & 52.48\\
1E0657 56 & \dataset [ADS/Sa.CXO\#obs/05361] {5361} & 06:58:29.670 & -55:56:39.80 & 82.6 & VF & I3 & 0.296 & 52.48\\
1RXS J2129.4-0741 & \dataset [ADS/Sa.CXO\#obs/03199] {3199} & 21:29:26.274 & -07:41:29.18 & 19.9 & VF & I3 & 0.570 & 20.58\\
1RXS J2129.4-0741 & \dataset [ADS/Sa.CXO\#obs/03595] {3595} & 21:29:26.281 & -07:41:29.36 & 19.9 & VF & I3 & 0.570 & 20.58\\
2PIGG J0011.5-2850 & \dataset [ADS/Sa.CXO\#obs/05797] {5797} & 00:11:21.623 & -28:51:14.44 & 19.9 & VF & I3 & 0.075 &  2.15\\
2PIGG J0311.8-2655 $\dagger$ & \dataset [ADS/Sa.CXO\#obs/05799] {5799} & 03:11:33.904 & -26:54:16.48 & 39.6 & VF & I3 & 0.062 &  0.25\\
2PIGG J2227.0-3041 & \dataset [ADS/Sa.CXO\#obs/05798] {5798} & 22:27:54.560 & -30:34:34.84 & 22.3 & VF & I2 & 0.073 &  0.81\\
3C 220.1 & \dataset [ADS/Sa.CXO\#obs/00839] {839} & 09:32:40.218 & +79:06:29.46 & 18.9 &  F & S3 & 0.610 &  3.25\\
3C 28.0 & \dataset [ADS/Sa.CXO\#obs/03233] {3233} & 00:55:50.401 & +26:24:36.47 & 49.7 & VF & I3 & 0.195 &  4.78\\
3C 295 & \dataset [ADS/Sa.CXO\#obs/02254] {2254} & 14:11:20.280 & +52:12:10.55 & 90.9 & VF & I3 & 0.464 &  6.92\\
3C 388 & \dataset [ADS/Sa.CXO\#obs/05295] {5295} & 18:44:02.365 & +45:33:29.31 & 30.7 & VF & I3 & 0.092 &  0.52\\
4C 55.16 & \dataset [ADS/Sa.CXO\#obs/04940] {4940} & 08:34:54.923 & +55:34:21.15 & 96.0 & VF & S3 & 0.242 &  5.90\\
ABELL 0013 $\dagger$ & \dataset [ADS/Sa.CXO\#obs/04945] {4945} & 00:13:37.883 & -19:30:09.10 & 55.3 & VF & S3 & 0.094 &  1.41\\
ABELL 0068 & \dataset [ADS/Sa.CXO\#obs/03250] {3250} & 00:37:06.309 & +09:09:32.28 & 10.0 & VF & I3 & 0.255 & 12.70\\
ABELL 0119 $\dagger$ & \dataset [ADS/Sa.CXO\#obs/04180] {4180} & 00:56:15.150 & -01:14:59.70 & 11.9 & VF & I3 & 0.044 &  1.39\\
ABELL 0168 & \dataset [ADS/Sa.CXO\#obs/03203] {3203} & 01:14:57.909 & +00:24:42.55 & 40.6 & VF & I3 & 0.045 &  0.23\\
ABELL 0168 & \dataset [ADS/Sa.CXO\#obs/03204] {3204} & 01:14:57.925 & +00:24:42.73 & 37.6 & VF & I3 & 0.045 &  0.23\\
ABELL 0209 & \dataset [ADS/Sa.CXO\#obs/03579] {3579} & 01:31:52.565 & -13:36:39.29 & 10.0 & VF & I3 & 0.206 & 10.96\\
ABELL 0209 & \dataset [ADS/Sa.CXO\#obs/00522] {522} & 01:31:52.595 & -13:36:39.25 & 10.0 & VF & I3 & 0.206 & 10.96\\
ABELL 0267 & \dataset [ADS/Sa.CXO\#obs/01448] {1448} & 01:52:29.181 & +00:57:34.43 & 7.9 &  F & I3 & 0.230 &  8.62\\
ABELL 0267 & \dataset [ADS/Sa.CXO\#obs/03580] {3580} & 01:52:29.180 & +00:57:34.23 & 19.9 & VF & I3 & 0.230 &  8.62\\
ABELL 0370 & \dataset [ADS/Sa.CXO\#obs/00515] {515} & 02:39:53.169 & -01:34:36.96 & 88.0 &  F & S3 & 0.375 & 11.95\\
ABELL 0383 & \dataset [ADS/Sa.CXO\#obs/02321] {2321} & 02:48:03.364 & -03:31:44.69 & 19.5 &  F & S3 & 0.187 &  5.32\\
ABELL 0399 & \dataset [ADS/Sa.CXO\#obs/03230] {3230} & 02:57:54.931 & +13:01:58.41 & 48.6 & VF & I0 & 0.072 &  4.37\\
ABELL 0401 & \dataset [ADS/Sa.CXO\#obs/00518] {518} & 02:58:56.896 & +13:34:14.48 & 18.0 &  F & I3 & 0.074 &  8.39\\
ABELL 0478 & \dataset [ADS/Sa.CXO\#obs/06102] {6102} & 04:13:25.347 & +10:27:55.62 & 10.0 & VF & I3 & 0.088 & 16.39\\
ABELL 0514 & \dataset [ADS/Sa.CXO\#obs/03578] {3578} & 04:48:19.229 & -20:30:28.79 & 44.5 & VF & I3 & 0.072 &  0.66\\
ABELL 0520 & \dataset [ADS/Sa.CXO\#obs/04215] {4215} & 04:54:09.711 & +02:55:23.69 & 66.3 & VF & I3 & 0.202 & 12.97\\
ABELL 0521 & \dataset [ADS/Sa.CXO\#obs/00430] {430} & 04:54:07.004 & -10:13:26.72 & 39.1 & VF & S3 & 0.253 &  9.77\\
ABELL 0586 & \dataset [ADS/Sa.CXO\#obs/00530] {530} & 07:32:20.339 & +31:37:58.59 & 10.0 & VF & I3 & 0.171 &  8.54\\
ABELL 0611 & \dataset [ADS/Sa.CXO\#obs/03194] {3194} & 08:00:56.832 & +36:03:24.09 & 36.1 & VF & S3 & 0.288 & 10.78\\
ABELL 0644 $\dagger$ & \dataset [ADS/Sa.CXO\#obs/02211] {2211} & 08:17:25.225 & -07:30:40.03 & 29.7 & VF & I3 & 0.070 &  6.95\\
ABELL 0665 & \dataset [ADS/Sa.CXO\#obs/03586] {3586} & 08:30:59.231 & +65:50:37.78 & 29.7 & VF & I3 & 0.181 & 13.37\\
ABELL 0697 & \dataset [ADS/Sa.CXO\#obs/04217] {4217} & 08:42:57.549 & +36:21:57.65 & 19.5 & VF & I3 & 0.282 & 26.10\\
ABELL 0773 & \dataset [ADS/Sa.CXO\#obs/05006] {5006} & 09:17:52.566 & +51:43:38.18 & 19.8 & VF & I3 & 0.217 & 12.87\\
ABELL 0781 & \dataset [ADS/Sa.CXO\#obs/00534] {534} & 09:20:25.431 & +30:30:07.56 & 9.9 & VF & I3 & 0.298 &  0.00\\
ABELL 0907 & \dataset [ADS/Sa.CXO\#obs/03185] {3185} & 09:58:21.880 & -11:03:52.20 & 48.0 & VF & I3 & 0.153 &  6.19\\
ABELL 0963 & \dataset [ADS/Sa.CXO\#obs/00903] {903} & 10:17:03.744 & +39:02:49.17 & 36.3 &  F & S3 & 0.206 & 10.65\\
ABELL 1063S & \dataset [ADS/Sa.CXO\#obs/04966] {4966} & 22:48:44.294 & -44:31:48.37 & 26.7 & VF & I3 & 0.354 & 71.09\\
ABELL 1068 $\dagger$ & \dataset [ADS/Sa.CXO\#obs/01652] {1652} & 10:40:44.520 & +39:57:10.28 & 26.8 &  F & S3 & 0.138 &  4.19\\
ABELL 1201 $\dagger$ & \dataset [ADS/Sa.CXO\#obs/04216] {4216} & 11:12:54.489 & +13:26:08.76 & 39.7 & VF & S3 & 0.169 &  3.52\\
ABELL 1204 & \dataset [ADS/Sa.CXO\#obs/02205] {2205} & 11:13:20.419 & +17:35:38.45 & 23.6 & VF & I3 & 0.171 &  3.92\\
ABELL 1361 $\dagger$ & \dataset [ADS/Sa.CXO\#obs/02200] {2200} & 11:43:39.827 & +46:21:21.40 & 16.7 &  F & S3 & 0.117 &  2.16\\
ABELL 1423 & \dataset [ADS/Sa.CXO\#obs/00538] {538} & 11:57:17.026 & +33:36:37.44 & 9.8 & VF & I3 & 0.213 &  7.01\\
ABELL 1651 & \dataset [ADS/Sa.CXO\#obs/04185] {4185} & 12:59:22.830 & -04:11:45.86 & 9.6 & VF & I3 & 0.084 &  6.66\\
ABELL 1664 $\dagger$ & \dataset [ADS/Sa.CXO\#obs/01648] {1648} & 13:03:42.478 & -24:14:44.55 & 9.8 & VF & S3 & 0.128 &  2.59\\
ABELL 1682 & \dataset [ADS/Sa.CXO\#obs/03244] {3244} & 13:06:50.764 & +46:33:19.86 & 9.8 & VF & I3 & 0.226 &  0.00\\
ABELL 1689 & \dataset [ADS/Sa.CXO\#obs/01663] {1663} & 13:11:29.612 & -01:20:28.69 & 10.7 &  F & I3 & 0.184 & 24.71\\
ABELL 1689 & \dataset [ADS/Sa.CXO\#obs/05004] {5004} & 13:11:29.606 & -01:20:28.61 & 19.9 & VF & I3 & 0.184 & 24.71\\
ABELL 1689 & \dataset [ADS/Sa.CXO\#obs/00540] {540} & 13:11:29.595 & -01:20:28.47 & 10.3 &  F & I3 & 0.184 & 24.71\\
ABELL 1758 & \dataset [ADS/Sa.CXO\#obs/02213] {2213} & 13:32:42.978 & +50:32:44.83 & 58.3 & VF & S3 & 0.279 & 21.01\\
ABELL 1763 & \dataset [ADS/Sa.CXO\#obs/03591] {3591} & 13:35:17.957 & +40:59:55.80 & 19.6 & VF & I3 & 0.187 &  9.26\\
ABELL 1795 $\dagger$ & \dataset [ADS/Sa.CXO\#obs/05289] {5289} & 13:48:52.829 & +26:35:24.01 & 15.0 & VF & I3 & 0.062 &  7.59\\
ABELL 1835 & \dataset [ADS/Sa.CXO\#obs/00495] {495} & 14:01:01.951 & +02:52:43.18 & 19.5 &  F & S3 & 0.253 & 39.38\\
ABELL 1914 & \dataset [ADS/Sa.CXO\#obs/03593] {3593} & 14:26:01.399 & +37:49:27.83 & 18.9 & VF & I3 & 0.171 & 26.25\\
ABELL 1942 & \dataset [ADS/Sa.CXO\#obs/03290] {3290} & 14:38:21.878 & +03:40:12.97 & 57.6 & VF & I2 & 0.224 &  2.27\\
ABELL 1995 & \dataset [ADS/Sa.CXO\#obs/00906] {906} & 14:52:57.758 & +58:02:51.34 & 0.0 &  F & S3 & 0.319 & 10.19\\
ABELL 2029 $\dagger$ & \dataset [ADS/Sa.CXO\#obs/06101] {6101} & 15:10:56.163 & +05:44:40.89 & 9.9 & VF & I3 & 0.076 & 13.90\\
ABELL 2034 & \dataset [ADS/Sa.CXO\#obs/02204] {2204} & 15:10:11.003 & +33:30:46.46 & 53.9 & VF & I3 & 0.113 &  6.45\\
ABELL 2065 $\dagger$ & \dataset [ADS/Sa.CXO\#obs/031821] {31821} & 15:22:29.220 & +27:42:46.54 & 0.0 & VF & I3 & 0.073 &  2.92\\
ABELL 2069 & \dataset [ADS/Sa.CXO\#obs/04965] {4965} & 15:24:09.181 & +29:53:18.05 & 55.4 & VF & I2 & 0.116 &  3.82\\
ABELL 2111 & \dataset [ADS/Sa.CXO\#obs/00544] {544} & 15:39:41.432 & +34:25:12.26 & 10.3 &  F & I3 & 0.230 &  7.45\\
ABELL 2125 & \dataset [ADS/Sa.CXO\#obs/02207] {2207} & 15:41:14.154 & +66:15:57.20 & 81.5 & VF & I3 & 0.246 &  0.77\\
ABELL 2163 & \dataset [ADS/Sa.CXO\#obs/01653] {1653} & 16:15:45.705 & -06:09:00.62 & 71.1 & VF & I1 & 0.170 & 49.11\\
ABELL 2204 $\dagger$ & \dataset [ADS/Sa.CXO\#obs/0499] {499} & 16:32:45.437 & +05:34:21.05 & 10.1 &  F & S3 & 0.152 & 20.77\\
ABELL 2204 & \dataset [ADS/Sa.CXO\#obs/06104] {6104} & 16:32:45.428 & +05:34:20.89 & 9.6 & VF & I3 & 0.152 & 22.03\\
ABELL 2218 & \dataset [ADS/Sa.CXO\#obs/01666] {1666} & 16:35:50.831 & +66:12:42.31 & 48.6 & VF & I0 & 0.171 &  8.39\\
ABELL 2219 $\dagger$ & \dataset [ADS/Sa.CXO\#obs/0896] {896} & 16:40:21.069 & +46:42:29.07 & 42.3 &  F & S3 & 0.226 & 33.15\\
ABELL 2255 & \dataset [ADS/Sa.CXO\#obs/00894] {894} & 17:12:40.385 & +64:03:50.63 & 39.4 &  F & I3 & 0.081 &  3.67\\
ABELL 2256 $\dagger$ & \dataset [ADS/Sa.CXO\#obs/01386] {1386} & 17:03:44.567 & +78:38:11.51 & 12.4 &  F & I3 & 0.058 &  4.65\\
ABELL 2259 & \dataset [ADS/Sa.CXO\#obs/03245] {3245} & 17:20:08.299 & +27:40:11.53 & 10.0 & VF & I3 & 0.164 &  5.37\\
ABELL 2261 & \dataset [ADS/Sa.CXO\#obs/05007] {5007} & 17:22:27.254 & +32:07:58.60 & 24.3 & VF & I3 & 0.224 & 17.49\\
ABELL 2294 & \dataset [ADS/Sa.CXO\#obs/03246] {3246} & 17:24:10.149 & +85:53:09.77 & 10.0 & VF & I3 & 0.178 & 10.35\\
ABELL 2384 & \dataset [ADS/Sa.CXO\#obs/04202] {4202} & 21:52:21.178 & -19:32:51.90 & 31.5 & VF & I3 & 0.095 &  1.95\\
ABELL 2390 $\dagger$ & \dataset [ADS/Sa.CXO\#obs/04193] {4193} & 21:53:36.825 & +17:41:44.38 & 95.1 & VF & S3 & 0.230 & 31.02\\
ABELL 2409 & \dataset [ADS/Sa.CXO\#obs/03247] {3247} & 22:00:52.567 & +20:58:34.11 & 10.2 & VF & I3 & 0.148 &  7.01\\
ABELL 2537 & \dataset [ADS/Sa.CXO\#obs/04962] {4962} & 23:08:22.313 & -02:11:29.88 & 36.2 & VF & S3 & 0.295 & 10.16\\
ABELL 2550 & \dataset [ADS/Sa.CXO\#obs/02225] {2225} & 23:11:35.806 & -21:44:46.70 & 59.0 & VF & S3 & 0.154 &  0.58\\
ABELL 2554 $\dagger$ & \dataset [ADS/Sa.CXO\#obs/01696] {1696} & 23:12:19.939 & -21:30:09.84 & 19.9 & VF & S3 & 0.110 &  1.57\\
ABELL 2556 $\dagger$ & \dataset [ADS/Sa.CXO\#obs/02226] {2226} & 23:13:01.413 & -21:38:04.47 & 19.9 & VF & S3 & 0.086 &  1.43\\
ABELL 2631 & \dataset [ADS/Sa.CXO\#obs/03248] {3248} & 23:37:38.560 & +00:16:28.64 & 9.2 & VF & I3 & 0.278 & 12.59\\
ABELL 2667 & \dataset [ADS/Sa.CXO\#obs/02214] {2214} & 23:51:39.395 & -26:05:02.75 & 9.6 & VF & S3 & 0.230 & 19.91\\
ABELL 2670 & \dataset [ADS/Sa.CXO\#obs/04959] {4959} & 23:54:13.687 & -10:25:08.85 & 39.6 & VF & I3 & 0.076 &  1.39\\
ABELL 2717 & \dataset [ADS/Sa.CXO\#obs/06974] {6974} & 00:03:11.996 & -35:56:08.01 & 19.8 & VF & I3 & 0.048 &  0.26\\
ABELL 2744 & \dataset [ADS/Sa.CXO\#obs/02212] {2212} & 00:14:14.396 & -30:22:40.04 & 24.8 & VF & S3 & 0.308 & 29.00\\
ABELL 3128 $\dagger$ & \dataset [ADS/Sa.CXO\#obs/00893] {893} & 03:29:50.918 & -52:34:51.04 & 19.6 &  F & I3 & 0.062 &  0.35\\
ABELL 3158 $\dagger$ & \dataset [ADS/Sa.CXO\#obs/03201] {3201} & 03:42:54.675 & -53:37:24.36 & 24.8 & VF & I3 & 0.059 &  3.01\\
ABELL 3158 $\dagger$ & \dataset [ADS/Sa.CXO\#obs/03712] {3712} & 03:42:54.683 & -53:37:24.37 & 30.9 & VF & I3 & 0.059 &  3.01\\
ABELL 3164 & \dataset [ADS/Sa.CXO\#obs/06955] {6955} & 03:46:16.839 & -57:02:11.38 & 13.5 & VF & I3 & 0.057 &  0.19\\
ABELL 3376 & \dataset [ADS/Sa.CXO\#obs/03202] {3202} & 06:02:05.122 & -39:57:42.82 & 44.3 & VF & I3 & 0.046 &  0.75\\
ABELL 3376 & \dataset [ADS/Sa.CXO\#obs/03450] {3450} & 06:02:05.162 & -39:57:42.87 & 19.8 & VF & I3 & 0.046 &  0.75\\
ABELL 3391 $\dagger$ & \dataset [ADS/Sa.CXO\#obs/04943] {4943} & 06:26:21.511 & -53:41:44.81 & 18.4 & VF & I3 & 0.056 &  1.44\\
ABELL 3921 & \dataset [ADS/Sa.CXO\#obs/04973] {4973} & 22:49:57.829 & -64:25:42.17 & 29.4 & VF & I3 & 0.093 &  3.37\\
AC 114 & \dataset [ADS/Sa.CXO\#obs/01562] {1562} & 22:58:48.196 & -34:47:56.89 & 72.5 &  F & S3 & 0.312 & 10.90\\
CL 0024+17 & \dataset [ADS/Sa.CXO\#obs/00929] {929} & 00:26:35.996 & +17:09:45.37 & 39.8 &  F & S3 & 0.394 &  2.88\\
CL 1221+4918 & \dataset [ADS/Sa.CXO\#obs/01662] {1662} & 12:21:26.709 & +49:18:21.60 & 79.1 & VF & I3 & 0.700 &  8.65\\
CL J0030+2618 & \dataset [ADS/Sa.CXO\#obs/05762] {5762} & 00:30:34.339 & +26:18:01.58 & 17.9 & VF & I3 & 0.500 &  3.41\\
CL J0152-1357 & \dataset [ADS/Sa.CXO\#obs/00913] {913} & 01:52:42.141 & -13:57:59.71 & 36.5 &  F & I3 & 0.831 & 13.30\\
CL J0542.8-4100 & \dataset [ADS/Sa.CXO\#obs/00914] {914} & 05:42:49.994 & -40:59:58.50 & 50.4 &  F & I3 & 0.630 &  6.18\\
CL J0848+4456 & \dataset [ADS/Sa.CXO\#obs/01708] {1708} & 08:48:48.235 & +44:56:17.11 & 61.4 & VF & I1 & 0.574 &  0.62\\
CL J0848+4456 & \dataset [ADS/Sa.CXO\#obs/00927] {927} & 08:48:48.252 & +44:56:17.13 & 125.1 & VF & I1 & 0.574 &  0.62\\
CL J1113.1-2615 & \dataset [ADS/Sa.CXO\#obs/00915] {915} & 11:13:05.167 & -26:15:40.43 & 104.6 &  F & I3 & 0.730 &  2.22\\
CL J1213+0253 & \dataset [ADS/Sa.CXO\#obs/04934] {4934} & 12:13:34.948 & +02:53:45.45 & 18.9 & VF & I3 & 0.409 &  0.00\\
CL J1226.9+3332 & \dataset [ADS/Sa.CXO\#obs/03180] {3180} & 12:26:58.373 & +33:32:47.36 & 31.7 & VF & I3 & 0.890 & 30.76\\
CL J1226.9+3332 & \dataset [ADS/Sa.CXO\#obs/05014] {5014} & 12:26:58.372 & +33:32:47.18 & 32.7 & VF & I3 & 0.890 & 30.76\\
CL J1641+4001 & \dataset [ADS/Sa.CXO\#obs/03575] {3575} & 16:41:53.704 & +40:01:44.40 & 46.5 & VF & I3 & 0.464 &  0.00\\
CL J2302.8+0844 & \dataset [ADS/Sa.CXO\#obs/00918] {918} & 23:02:48.156 & +08:43:52.74 & 108.6 &  F & I3 & 0.730 &  2.93\\
DLS J0514-4904 & \dataset [ADS/Sa.CXO\#obs/04980] {4980} & 05:14:40.037 & -49:03:15.07 & 19.9 & VF & I3 & 0.091 &  0.68\\
EXO 0422-086 $\dagger$ & \dataset [ADS/Sa.CXO\#obs/04183] {4183} & 04:25:51.271 & -08:33:36.42 & 10.0 & VF & I3 & 0.040 &  0.65\\
HERCULES A $\dagger$ & \dataset [ADS/Sa.CXO\#obs/01625] {1625} & 16:51:08.161 & +04:59:32.44 & 14.8 & VF & S3 & 0.154 &  3.27\\
IRAS 09104+4109 & \dataset [ADS/Sa.CXO\#obs/00509] {509} & 09:13:45.481 & +40:56:27.49 & 9.1 &  F & S3 & 0.442 &  0.00\\
LYNX E & \dataset [ADS/Sa.CXO\#obs/017081] {17081} & 08:48:58.851 & +44:51:51.44 & 61.4 & VF & I2 & 1.260 &  0.00\\
LYNX E & \dataset [ADS/Sa.CXO\#obs/09271] {9271} & 08:48:58.858 & +44:51:51.46 & 125.1 & VF & I2 & 1.260 &  0.00\\
MACS J0011.7-1523 & \dataset [ADS/Sa.CXO\#obs/03261] {3261} & 00:11:42.965 & -15:23:20.79 & 21.6 & VF & I3 & 0.360 & 10.75\\
MACS J0011.7-1523 & \dataset [ADS/Sa.CXO\#obs/06105] {6105} & 00:11:42.957 & -15:23:20.76 & 37.3 & VF & I3 & 0.360 & 10.75\\
MACS J0025.4-1222 & \dataset [ADS/Sa.CXO\#obs/03251] {3251} & 00:25:29.368 & -12:22:38.05 & 19.3 & VF & I3 & 0.584 & 13.00\\
MACS J0025.4-1222 & \dataset [ADS/Sa.CXO\#obs/05010] {5010} & 00:25:29.399 & -12:22:38.10 & 24.8 & VF & I3 & 0.584 & 13.00\\
MACS J0035.4-2015 & \dataset [ADS/Sa.CXO\#obs/03262] {3262} & 00:35:26.573 & -20:15:46.06 & 21.4 & VF & I3 & 0.364 & 19.79\\
MACS J0111.5+0855 & \dataset [ADS/Sa.CXO\#obs/03256] {3256} & 01:11:31.515 & +08:55:39.21 & 19.4 & VF & I3 & 0.263 &  0.64\\
MACS J0152.5-2852 & \dataset [ADS/Sa.CXO\#obs/03264] {3264} & 01:52:34.479 & -28:53:38.01 & 17.5 & VF & I3 & 0.341 &  6.33\\
MACS J0159.0-3412 & \dataset [ADS/Sa.CXO\#obs/05818] {5818} & 01:59:00.366 & -34:13:00.23 & 9.4 & VF & I3 & 0.458 & 18.92\\
MACS J0159.8-0849 & \dataset [ADS/Sa.CXO\#obs/03265] {3265} & 01:59:49.453 & -08:50:00.90 & 17.9 & VF & I3 & 0.405 & 26.31\\
MACS J0159.8-0849 & \dataset [ADS/Sa.CXO\#obs/06106] {6106} & 01:59:49.422 & -08:50:00.42 & 35.3 & VF & I3 & 0.405 & 26.31\\
MACS J0242.5-2132 & \dataset [ADS/Sa.CXO\#obs/03266] {3266} & 02:42:35.906 & -21:32:26.30 & 11.9 & VF & I3 & 0.314 & 12.74\\
MACS J0257.1-2325 & \dataset [ADS/Sa.CXO\#obs/01654] {1654} & 02:57:09.130 & -23:26:06.25 & 19.8 &  F & I3 & 0.505 & 21.72\\
MACS J0257.1-2325 & \dataset [ADS/Sa.CXO\#obs/03581] {3581} & 02:57:09.152 & -23:26:06.21 & 18.5 & VF & I3 & 0.505 & 21.72\\
MACS J0257.6-2209 & \dataset [ADS/Sa.CXO\#obs/03267] {3267} & 02:57:41.024 & -22:09:11.12 & 20.5 & VF & I3 & 0.322 & 10.77\\
MACS J0308.9+2645 & \dataset [ADS/Sa.CXO\#obs/03268] {3268} & 03:08:55.927 & +26:45:38.34 & 24.4 & VF & I3 & 0.324 & 20.42\\
MACS J0329.6-0211 & \dataset [ADS/Sa.CXO\#obs/03257] {3257} & 03:29:41.681 & -02:11:47.67 & 9.9 & VF & I3 & 0.450 & 12.82\\
MACS J0329.6-0211 & \dataset [ADS/Sa.CXO\#obs/03582] {3582} & 03:29:41.688 & -02:11:47.81 & 19.9 & VF & I3 & 0.450 & 12.82\\
MACS J0329.6-0211 & \dataset [ADS/Sa.CXO\#obs/06108] {6108} & 03:29:41.681 & -02:11:47.57 & 39.6 & VF & I3 & 0.450 & 12.82\\
MACS J0404.6+1109 & \dataset [ADS/Sa.CXO\#obs/03269] {3269} & 04:04:32.491 & +11:08:02.10 & 21.8 & VF & I3 & 0.355 &  3.90\\
MACS J0417.5-1154 & \dataset [ADS/Sa.CXO\#obs/03270] {3270} & 04:17:34.686 & -11:54:32.71 & 12.0 & VF & I3 & 0.440 & 37.99\\
MACS J0429.6-0253 & \dataset [ADS/Sa.CXO\#obs/03271] {3271} & 04:29:36.088 & -02:53:09.02 & 23.2 & VF & I3 & 0.399 & 11.58\\
MACS J0451.9+0006 & \dataset [ADS/Sa.CXO\#obs/05815] {5815} & 04:51:54.291 & +00:06:20.20 & 10.2 & VF & I3 & 0.430 &  8.20\\
MACS J0455.2+0657 & \dataset [ADS/Sa.CXO\#obs/05812] {5812} & 04:55:17.426 & +06:57:47.15 & 9.9 & VF & I3 & 0.425 &  9.77\\
MACS J0520.7-1328 & \dataset [ADS/Sa.CXO\#obs/03272] {3272} & 05:20:42.052 & -13:28:49.38 & 19.2 & VF & I3 & 0.340 &  9.63\\
MACS J0547.0-3904 & \dataset [ADS/Sa.CXO\#obs/03273] {3273} & 05:47:01.582 & -39:04:28.24 & 21.7 & VF & I3 & 0.210 &  1.59\\
MACS J0553.4-3342 & \dataset [ADS/Sa.CXO\#obs/05813] {5813} & 05:53:27.200 & -33:42:53.02 & 9.9 & VF & I3 & 0.407 & 32.68\\
MACS J0717.5+3745 & \dataset [ADS/Sa.CXO\#obs/01655] {1655} & 07:17:31.654 & +37:45:18.52 & 19.9 &  F & I3 & 0.548 & 46.58\\
MACS J0717.5+3745 & \dataset [ADS/Sa.CXO\#obs/04200] {4200} & 07:17:31.651 & +37:45:18.46 & 59.2 & VF & I3 & 0.548 & 46.58\\
MACS J0744.8+3927 & \dataset [ADS/Sa.CXO\#obs/03197] {3197} & 07:44:52.802 & +39:27:24.41 & 20.2 & VF & I3 & 0.686 & 24.67\\
MACS J0744.8+3927 & \dataset [ADS/Sa.CXO\#obs/03585] {3585} & 07:44:52.779 & +39:27:24.41 & 19.9 & VF & I3 & 0.686 & 24.67\\
MACS J0744.8+3927 & \dataset [ADS/Sa.CXO\#obs/06111] {6111} & 07:44:52.800 & +39:27:24.41 & 49.5 & VF & I3 & 0.686 & 24.67\\
MACS J0911.2+1746 & \dataset [ADS/Sa.CXO\#obs/03587] {3587} & 09:11:11.325 & +17:46:31.06 & 17.9 & VF & I3 & 0.541 & 10.52\\
MACS J0911.2+1746 & \dataset [ADS/Sa.CXO\#obs/05012] {5012} & 09:11:11.309 & +17:46:30.92 & 23.8 & VF & I3 & 0.541 & 10.52\\
MACS J0949+1708   & \dataset [ADS/Sa.CXO\#obs/03274] {3274} & 09:49:51.824 & +17:07:05.62 & 14.3 & VF & I3 & 0.382 & 19.19\\
MACS J1006.9+3200 & \dataset [ADS/Sa.CXO\#obs/05819] {5819} & 10:06:54.668 & +32:01:34.61 & 10.9 & VF & I3 & 0.359 &  6.06\\
MACS J1105.7-1014 & \dataset [ADS/Sa.CXO\#obs/05817] {5817} & 11:05:46.462 & -10:14:37.20 & 10.3 & VF & I3 & 0.466 & 11.29\\
MACS J1108.8+0906 & \dataset [ADS/Sa.CXO\#obs/03252] {3252} & 11:08:55.393 & +09:05:51.16 & 9.9 & VF & I3 & 0.449 &  8.96\\
MACS J1108.8+0906 & \dataset [ADS/Sa.CXO\#obs/05009] {5009} & 11:08:55.402 & +09:05:51.14 & 24.5 & VF & I3 & 0.449 &  8.96\\
MACS J1115.2+5320 & \dataset [ADS/Sa.CXO\#obs/03253] {3253} & 11:15:15.632 & +53:20:03.71 & 8.8 & VF & I3 & 0.439 & 14.29\\
MACS J1115.2+5320 & \dataset [ADS/Sa.CXO\#obs/05008] {5008} & 11:15:15.646 & +53:20:03.74 & 18.0 & VF & I3 & 0.439 & 14.29\\
MACS J1115.2+5320 & \dataset [ADS/Sa.CXO\#obs/05350] {5350} & 11:15:15.632 & +53:20:03.37 & 6.9 & VF & I3 & 0.439 & 14.29\\
MACS J1115.8+0129 & \dataset [ADS/Sa.CXO\#obs/03275] {3275} & 11:15:52.048 & +01:29:56.56 & 15.9 & VF & I3 & 0.120 &  1.47\\
MACS J1131.8-1955 & \dataset [ADS/Sa.CXO\#obs/03276] {3276} & 11:31:56.011 & -19:55:55.85 & 13.9 & VF & I3 & 0.307 & 17.45\\
MACS J1149.5+2223 & \dataset [ADS/Sa.CXO\#obs/01656] {1656} & 11:49:35.856 & +22:23:55.02 & 18.5 & VF & I3 & 0.544 & 21.60\\
MACS J1149.5+2223 & \dataset [ADS/Sa.CXO\#obs/03589] {3589} & 11:49:35.848 & +22:23:55.05 & 20.0 & VF & I3 & 0.544 & 21.60\\
MACS J1206.2-0847 & \dataset [ADS/Sa.CXO\#obs/03277] {3277} & 12:06:12.276 & -08:48:02.40 & 23.5 & VF & I3 & 0.440 & 37.02\\
MACS J1226.8+2153 & \dataset [ADS/Sa.CXO\#obs/03590] {3590} & 12:26:51.207 & +21:49:55.22 & 19.0 & VF & I3 & 0.370 &  2.63\\
MACS J1311.0-0310 & \dataset [ADS/Sa.CXO\#obs/03258] {3258} & 13:11:01.665 & -03:10:39.50 & 14.9 & VF & I3 & 0.494 & 10.03\\
MACS J1311.0-0310 & \dataset [ADS/Sa.CXO\#obs/06110] {6110} & 13:11:01.680 & -03:10:39.75 & 63.2 & VF & I3 & 0.494 & 10.03\\
MACS J1319+7003   & \dataset [ADS/Sa.CXO\#obs/03278] {3278} & 13:20:08.370 & +70:04:33.81 & 21.6 & VF & I3 & 0.328 &  7.03\\
MACS J1427.2+4407 & \dataset [ADS/Sa.CXO\#obs/06112] {6112} & 14:27:16.175 & +44:07:30.33 & 9.4 & VF & I3 & 0.477 & 14.18\\
MACS J1427.6-2521 & \dataset [ADS/Sa.CXO\#obs/03279] {3279} & 14:27:39.389 & -25:21:04.66 & 16.9 & VF & I3 & 0.220 &  1.55\\
MACS J1621.3+3810 & \dataset [ADS/Sa.CXO\#obs/03254] {3254} & 16:21:25.552 & +38:09:43.56 & 9.8 & VF & I3 & 0.461 & 11.49\\
MACS J1621.3+3810 & \dataset [ADS/Sa.CXO\#obs/03594] {3594} & 16:21:25.558 & +38:09:43.44 & 19.7 & VF & I3 & 0.461 & 11.49\\
MACS J1621.3+3810 & \dataset [ADS/Sa.CXO\#obs/06109] {6109} & 16:21:25.535 & +38:09:43.34 & 37.5 & VF & I3 & 0.461 & 11.49\\
MACS J1621.3+3810 & \dataset [ADS/Sa.CXO\#obs/06172] {6172} & 16:21:25.559 & +38:09:43.63 & 29.8 & VF & I3 & 0.461 & 11.49\\
MACS J1731.6+2252 & \dataset [ADS/Sa.CXO\#obs/03281] {3281} & 17:31:39.902 & +22:52:00.55 & 20.5 & VF & I3 & 0.366 &  9.32\\
MACS J1824.3+4309 & \dataset [ADS/Sa.CXO\#obs/03255] {3255} & 18:24:18.444 & +43:09:43.39 & 14.9 & VF & I3 & 0.487 &  0.00\\
MACS J1931.8-2634 & \dataset [ADS/Sa.CXO\#obs/03282] {3282} & 19:31:49.656 & -26:34:33.99 & 13.6 & VF & I3 & 0.352 & 23.14\\
MACS J2046.0-3430 & \dataset [ADS/Sa.CXO\#obs/05816] {5816} & 20:46:00.522 & -34:30:15.50 & 10.0 & VF & I3 & 0.413 &  5.79\\
MACS J2049.9-3217 & \dataset [ADS/Sa.CXO\#obs/03283] {3283} & 20:49:56.245 & -32:16:52.30 & 23.8 & VF & I3 & 0.325 &  8.71\\
MACS J2211.7-0349 & \dataset [ADS/Sa.CXO\#obs/03284] {3284} & 22:11:45.856 & -03:49:37.24 & 17.7 & VF & I3 & 0.270 & 22.11\\
MACS J2214.9-1359 & \dataset [ADS/Sa.CXO\#obs/03259] {3259} & 22:14:57.467 & -14:00:09.35 & 19.5 & VF & I3 & 0.503 & 24.05\\
MACS J2214.9-1359 & \dataset [ADS/Sa.CXO\#obs/05011] {5011} & 22:14:57.481 & -14:00:09.39 & 18.5 & VF & I3 & 0.503 & 24.05\\
MACS J2228+2036   & \dataset [ADS/Sa.CXO\#obs/03285] {3285} & 22:28:33.241 & +20:37:11.42 & 19.9 & VF & I3 & 0.412 & 17.92\\
MACS J2229.7-2755 & \dataset [ADS/Sa.CXO\#obs/03286] {3286} & 22:29:45.358 & -27:55:38.41 & 16.4 & VF & I3 & 0.324 &  9.49\\
MACS J2243.3-0935 & \dataset [ADS/Sa.CXO\#obs/03260] {3260} & 22:43:21.537 & -09:35:44.30 & 20.5 & VF & I3 & 0.101 &  0.78\\
MACS J2245.0+2637 & \dataset [ADS/Sa.CXO\#obs/03287] {3287} & 22:45:04.547 & +26:38:07.88 & 16.9 & VF & I3 & 0.304 &  9.36\\
MACS J2311+0338   & \dataset [ADS/Sa.CXO\#obs/03288] {3288} & 23:11:33.213 & +03:38:06.51 & 13.6 & VF & I3 & 0.300 & 10.98\\
MKW3S & \dataset [ADS/Sa.CXO\#obs/0900] {900} & 15:21:51.930 & +07:42:31.97 & 57.3 & VF & I3 & 0.045 &  1.14\\
MS 0016.9+1609 & \dataset [ADS/Sa.CXO\#obs/00520] {520} & 00:18:33.503 & +16:26:12.99 & 67.4 & VF & I3 & 0.541 & 32.94\\
MS 0302.7+1658 & \dataset [ADS/Sa.CXO\#obs/00525] {525} & 03:05:31.614 & +17:10:02.06 & 10.0 & VF & I3 & 0.424 &  0.00\\
MS 0440.5+0204 $\dagger$ & \dataset [ADS/Sa.CXO\#obs/04196] {4196} & 04:43:09.952 & +02:10:18.70 & 59.4 & VF & S3 & 0.190 &  2.17\\
MS 0451.6-0305 & \dataset [ADS/Sa.CXO\#obs/00902] {902} & 04:54:11.004 & -03:00:52.19 & 44.2 &  F & S3 & 0.539 & 33.32\\
MS 0735.6+7421 & \dataset [ADS/Sa.CXO\#obs/04197] {4197} & 07:41:44.245 & +74:14:38.23 & 45.5 & VF & S3 & 0.216 &  7.57\\
MS 0839.8+2938 & \dataset [ADS/Sa.CXO\#obs/02224] {2224} & 08:42:55.969 & +29:27:26.97 & 29.8 &  F & S3 & 0.194 &  3.10\\
MS 0906.5+1110 & \dataset [ADS/Sa.CXO\#obs/00924] {924} & 09:09:12.753 & +10:58:32.00 & 29.7 & VF & I3 & 0.163 &  4.64\\
MS 1006.0+1202 & \dataset [ADS/Sa.CXO\#obs/00925] {925} & 10:08:47.194 & +11:47:55.99 & 29.4 & VF & I3 & 0.221 &  4.75\\
MS 1008.1-1224 & \dataset [ADS/Sa.CXO\#obs/00926] {926} & 10:10:32.312 & -12:39:56.80 & 44.2 & VF & I3 & 0.301 &  6.44\\
MS 1054.5-0321 & \dataset [ADS/Sa.CXO\#obs/00512] {512} & 10:56:58.499 & -03:37:32.76 & 89.1 &  F & S3 & 0.830 & 27.22\\
MS 1455.0+2232 & \dataset [ADS/Sa.CXO\#obs/04192] {4192} & 14:57:15.088 & +22:20:32.49 & 91.9 & VF & I3 & 0.259 & 10.25\\
MS 1621.5+2640 & \dataset [ADS/Sa.CXO\#obs/00546] {546} & 16:23:35.522 & +26:34:25.67 & 30.1 &  F & I3 & 0.426 &  6.49\\
MS 2053.7-0449 & \dataset [ADS/Sa.CXO\#obs/01667] {1667} & 20:56:21.295 & -04:37:46.81 & 44.5 & VF & I3 & 0.583 &  2.96\\
MS 2053.7-0449 & \dataset [ADS/Sa.CXO\#obs/00551] {551} & 20:56:21.297 & -04:37:46.80 & 44.3 &  F & I3 & 0.583 &  2.96\\
MS 2137.3-2353 & \dataset [ADS/Sa.CXO\#obs/04974] {4974} & 21:40:15.178 & -23:39:40.71 & 57.4 & VF & S3 & 0.313 & 11.28\\
MS J1157.3+5531 $\dagger$ & \dataset [ADS/Sa.CXO\#obs/04964] {4964} & 11:59:52.295 & +55:32:05.61 & 75.1 & VF & S3 & 0.081 &  0.12\\
NGC 6338 $\dagger$ & \dataset [ADS/Sa.CXO\#obs/04194] {4194} & 17:15:23.036 & +57:24:40.29 & 47.3 & VF & I3 & 0.028 &  0.13\\
PKS 0745-191 & \dataset [ADS/Sa.CXO\#obs/06103] {6103} & 07:47:31.469 & -19:17:40.01 & 10.3 & VF & I3 & 0.103 & 18.41\\
RBS 0797 & \dataset [ADS/Sa.CXO\#obs/02202] {2202} & 09:47:12.971 & +76:23:13.90 & 11.7 & VF & I3 & 0.354 & 26.07\\
RDCS 1252-29    & \dataset [ADS/Sa.CXO\#obs/04198] {4198} & 12:52:54.221 & -29:27:21.01 & 163.4 & VF & I3 & 1.237 &  2.28\\
RX J0232.2-4420 & \dataset [ADS/Sa.CXO\#obs/04993] {4993} & 02:32:18.771 & -44:20:46.68 & 23.4 & VF & I3 & 0.284 & 18.17\\
RX J0340-4542   & \dataset [ADS/Sa.CXO\#obs/06954] {6954} & 03:40:44.765 & -45:41:18.41 & 17.9 & VF & I3 & 0.082 &  0.33\\
RX J0439+0520   & \dataset [ADS/Sa.CXO\#obs/00527] {527} & 04:39:02.218 & +05:20:43.11 & 9.6 & VF & I3 & 0.208 &  3.57\\
RX J0439.0+0715 & \dataset [ADS/Sa.CXO\#obs/01449] {1449} & 04:39:00.710 & +07:16:07.65 & 6.3 &  F & I3 & 0.230 &  9.44\\
RX J0439.0+0715 & \dataset [ADS/Sa.CXO\#obs/03583] {3583} & 04:39:00.710 & +07:16:07.63 & 19.2 & VF & I3 & 0.230 &  9.44\\
RX J0528.9-3927 & \dataset [ADS/Sa.CXO\#obs/04994] {4994} & 05:28:53.039 & -39:28:15.53 & 22.5 & VF & I3 & 0.263 & 12.99\\
RX J0647.7+7015 & \dataset [ADS/Sa.CXO\#obs/03196] {3196} & 06:47:50.029 & +70:14:49.66 & 19.3 & VF & I3 & 0.584 & 26.48\\
RX J0647.7+7015 & \dataset [ADS/Sa.CXO\#obs/03584] {3584} & 06:47:50.014 & +70:14:49.69 & 20.0 & VF & I3 & 0.584 & 26.48\\
RX J0819.6+6336 $\dagger$ & \dataset [ADS/Sa.CXO\#obs/02199] {2199} & 08:19:26.007 & +63:37:26.53 & 14.9 &  F & S3 & 0.119 &  0.98\\
RX J0910+5422   & \dataset [ADS/Sa.CXO\#obs/02452] {2452} & 09:10:44.478 & +54:22:03.77 & 65.3 & VF & I3 & 1.100 &  1.33\\
RX J1053+5735   & \dataset [ADS/Sa.CXO\#obs/04936] {4936} & 10:53:39.844 & +57:35:18.42 & 92.2 &  F & S3 & 1.140 &  0.00\\
RX J1347.5-1145 & \dataset [ADS/Sa.CXO\#obs/03592] {3592} & 13:47:30.593 & -11:45:10.05 & 57.7 & VF & I3 & 0.451 & 100.36\\
RX J1347.5-1145 & \dataset [ADS/Sa.CXO\#obs/00507] {507} & 13:47:30.598 & -11:45:10.27 & 10.0 &  F & S3 & 0.451 & 100.36\\
RX J1350+6007   & \dataset [ADS/Sa.CXO\#obs/02229] {2229} & 13:50:48.038 & +60:07:08.39 & 58.3 & VF & I3 & 0.804 &  2.19\\
RX J1423.8+2404 & \dataset [ADS/Sa.CXO\#obs/01657] {1657} & 14:23:47.759 & +24:04:40.45 & 18.5 & VF & I3 & 0.545 & 15.84\\
RX J1423.8+2404 & \dataset [ADS/Sa.CXO\#obs/04195] {4195} & 14:23:47.763 & +24:04:40.63 & 115.6 & VF & S3 & 0.545 & 15.84\\
RX J1504.1-0248 & \dataset [ADS/Sa.CXO\#obs/05793] {5793} & 15:04:07.415 & -02:48:15.70 & 39.2 & VF & I3 & 0.215 & 34.64\\
RX J1525+0958   & \dataset [ADS/Sa.CXO\#obs/01664] {1664} & 15:24:39.729 & +09:57:44.42 & 50.9 & VF & I3 & 0.516 &  3.29\\
RX J1532.9+3021 & \dataset [ADS/Sa.CXO\#obs/01649] {1649} & 15:32:55.642 & +30:18:57.69 & 9.4 & VF & S3 & 0.345 & 20.77\\
RX J1532.9+3021 & \dataset [ADS/Sa.CXO\#obs/01665] {1665} & 15:32:55.641 & +30:18:57.31 & 10.0 & VF & I3 & 0.345 & 20.77\\
RX J1716.9+6708 & \dataset [ADS/Sa.CXO\#obs/00548] {548} & 17:16:49.015 & +67:08:25.80 & 51.7 &  F & I3 & 0.810 &  8.04\\
RX J1720.1+2638 & \dataset [ADS/Sa.CXO\#obs/04361] {4361} & 17:20:09.941 & +26:37:29.11 & 25.7 & VF & I3 & 0.164 & 11.39\\
RX J1720.2+3536 & \dataset [ADS/Sa.CXO\#obs/03280] {3280} & 17:20:16.953 & +35:36:23.63 & 20.8 & VF & I3 & 0.391 & 13.02\\
RX J1720.2+3536 & \dataset [ADS/Sa.CXO\#obs/06107] {6107} & 17:20:16.949 & +35:36:23.68 & 33.9 & VF & I3 & 0.391 & 13.02\\
RX J1720.2+3536 & \dataset [ADS/Sa.CXO\#obs/07225] {7225} & 17:20:16.947 & +35:36:23.69 & 2.0 & VF & I3 & 0.391 & 13.02\\
RX J2011.3-5725 & \dataset [ADS/Sa.CXO\#obs/04995] {4995} & 20:11:26.889 & -57:25:09.08 & 24.0 & VF & I3 & 0.279 &  2.77\\
RX J2129.6+0005 & \dataset [ADS/Sa.CXO\#obs/00552] {552} & 21:29:39.944 & +00:05:18.83 & 10.0 & VF & I3 & 0.235 & 12.56\\
S0463 & \dataset [ADS/Sa.CXO\#obs/06956] {6956} & 04:29:07.040 & -53:49:38.02 & 29.3 & VF & I3 & 0.099 & 22.19\\
S0463 & \dataset [ADS/Sa.CXO\#obs/07250] {7250} & 04:29:07.063 & -53:49:38.11 & 29.1 & VF & I3 & 0.099 & 22.19\\
TRIANG AUSTR $\dagger$ & \dataset [ADS/Sa.CXO\#obs/01281] {1281} & 16:38:22.712 & -64:21:19.70 & 11.4 &  F & I3 & 0.051 &  9.41\\
V 1121.0+2327 & \dataset [ADS/Sa.CXO\#obs/01660] {1660} & 11:20:57.195 & +23:26:27.60 & 71.3 & VF & I3 & 0.560 &  3.28\\
ZWCL 1215 & \dataset [ADS/Sa.CXO\#obs/04184] {4184} & 12:17:40.787 & +03:39:39.42 & 12.1 & VF & I3 & 0.075 &  3.49\\
ZWCL 1358+6245 & \dataset [ADS/Sa.CXO\#obs/00516] {516} & 13:59:50.526 & +62:31:04.57 & 54.1 &  F & S3 & 0.328 & 12.42\\
ZWCL 1953 & \dataset [ADS/Sa.CXO\#obs/01659] {1659} & 08:50:06.677 & +36:04:16.16 & 24.9 &  F & I3 & 0.380 & 17.11\\
ZWCL 3146 & \dataset [ADS/Sa.CXO\#obs/00909] {909} & 10:23:39.735 & +04:11:08.05 & 46.0 &  F & I3 & 0.290 & 29.59\\
ZWCL 5247 & \dataset [ADS/Sa.CXO\#obs/00539] {539} & 12:34:21.928 & +09:47:02.83 & 9.3 & VF & I3 & 0.229 &  4.87\\
ZWCL 7160 & \dataset [ADS/Sa.CXO\#obs/00543] {543} & 14:57:15.158 & +22:20:33.85 & 9.9 &  F & I3 & 0.258 & 10.14\\
ZWICKY 2701 & \dataset [ADS/Sa.CXO\#obs/03195] {3195} & 09:52:49.183 & +51:53:05.27 & 26.9 & VF & S3 & 0.210 &  5.19\\
ZwCL 1332.8+5043 & \dataset [ADS/Sa.CXO\#obs/05772] {5772} & 13:34:20.698 & +50:31:04.64 & 19.5 & VF & I3 & 0.620 &  4.46\\
ZwCl 0848.5+3341 & \dataset [ADS/Sa.CXO\#obs/04205] {4205} & 08:51:38.873 & +33:31:08.00 & 11.4 & VF & S3 & 0.371 &  4.58
\enddata
\tablecomments{(1) Cluster name, (2) CDA observation identification number, (3) R.A. of cluster center, (4) Dec. of cluster center, (5) nominal exposure time, (6) observing mode, (7) CCD location of centroid, (8) redshift, (9) NRAO absorbing Galactic neutral hydrogen column density, (10) bolometric luminosity. $\dagger$ indicates clusters analyzed within R$_{5000}$ only.}
\end{deluxetable}


%%%%%%%%%%%%%%%%%%%%%%%%%%%%%%%%
% Comparison of THFR for wavg %
%%%%%%%%%%%%%%%%%%%%%%%%%%%%%%%%

\clearpage
\clearpage
\begin{deluxetable}{c|ccc|ccc}
\tabletypesize{\scriptsize}
\tablecaption{Weighted averages for various apertures\label{tab:wavg}}
\tablewidth{0pt}
\tablehead{
\colhead{Aperture} & 
\colhead{[0.7-7.0]} & \colhead{[2.0-7.0]} & \colhead{$T_{HBR}$} & 
\colhead{[0.7-7.0]} & \colhead{[2.0-7.0]} & \colhead{$T_{HBR}$}\\
\colhead{ } & \colhead{keV} & \colhead{keV} &
\colhead{ } & \colhead{keV} & \colhead{keV} & \colhead{ }\\
\multicolumn{1}{c}{} & \multicolumn{3}{l}{\dotfill Without Core\dotfill} & \multicolumn{3}{l}{\dotfill With Core\dotfill}}
\startdata
R$_{2500}$ & 4.94$\pm 0.02$   & 6.26$\pm 0.07$   & 1.16$\pm 0.01$   & 4.47$\pm 0.02$ & 5.45$\pm 0.05$ & 1.13$\pm 0.01$\\ 
R$_{5000}$ & 4.83$\pm 0.02$   & 6.11$\pm 0.07$   & 1.14$\pm 0.01$   & 4.27$\pm 0.02$ & 5.29$\pm 0.05$ & 1.14$\pm 0.01$\\
Simulated  & 3.853$\pm 0.004$ & 4.457$\pm 0.009$ & 1.131$\pm 0.002$ & \dotfill       & \dotfill       & \dotfill\\
Control    & 4.782$\pm 0.004$ & 5.143$\pm 0.008$ & 1.001$\pm 0.002$ & \dotfill       & \dotfill       & \dotfill
\enddata
\tablecomments{
Quoted errors are standard deviation of the mean calculated using an
unbiased estimator for weighted samples. Simulated sample has been
culled to include only T$_2$=0.75 keV.
}
\end{deluxetable}

%%%%%%%%%%%%%%%%%%%%%%%%%%%%
% Clusters with thfr > 1.1 %
%%%%%%%%%%%%%%%%%%%%%%%%%%%%

\begin{deluxetable}{lcccclc}
\tabletypesize{\scriptsize}
\tablecaption{Clusters with $T_{HBR}$ $> 1.1$ with 1$\sigma$ significance.\label{tab:tf11}}
\tablewidth{0pt}
\tablehead{
\colhead{Name}       & \colhead{$T_{HBR}$} & \colhead{Merger?} &
\colhead{Core Class} & \colhead{$T_{dec}$} & \colhead{X-ray Morphology} & 
\colhead{Ref.}}
\startdata
RX J1525+0958       \dotfill & 1.86$^{+0.83}_{-0.51}$ & Y       &  CC & 0.42$^{+0.14}_{-0.08}$ & Arrowhead shape \& no discernible core & [27]\\
MS 1008.1-1224      \dotfill & 1.59$^{+0.37}_{-0.27}$ & Y       & NCC & 0.93$^{+0.19}_{-0.14}$ & Wide gas tail extending $\approx$550 kpc north & [1]\\
ABELL 2034          \dotfill & 1.40$^{+0.14}_{-0.11}$ & Y       & NCC & 1.07$^{+0.11}_{-0.09}$ & Prominent cold front \& gas tail extending south & [2]\\
ABELL 401           \dotfill & 1.37$^{+0.12}_{-0.10}$ & Y       & NCC & 1.13$^{+0.12}_{-0.10}$ & Highly spherical \& possible cold front to north & [3]\\
ABELL 1689          \dotfill & 1.36$^{+0.14}_{-0.12}$ & Y       & NCC & 0.95$^{+0.09}_{-0.07}$ & Exceptionally spherical \& bright central core & [6],[7]\\
RX J0439.0+0715     \dotfill & 1.42$^{+0.24}_{-0.18}$ & Unknown & NCC & 0.98$^{+0.11}_{-0.09}$ & Bright core \& possible cold front to north & [27]\\
ABELL 3376          \dotfill & 1.33$^{+0.11}_{-0.10}$ & Y       & NCC & 0.97$^{+0.07}_{-0.07}$ & Highly disturbed \& broad gas tail to west & [4],[5]\\
ABELL 2255          \dotfill & 1.32$^{+0.12}_{-0.10}$ & Y       & NCC & 1.48$^{+0.32}_{-0.23}$ & Spherical \& compressed isophotes west of core & [8],[9]\\
ABELL 2218          \dotfill & 1.36$^{+0.19}_{-0.15}$ & Y       & NCC & 1.39$^{+0.23}_{-0.19}$ & Spherical, core of cluster elongated NW-SE & [10]\\
ABELL 1763          \dotfill & 1.48$^{+0.39}_{-0.26}$ & Y       & NCC & 0.83$^{+0.17}_{-0.13}$ & Elongated ENE-SSW \& cold front to west of core & [11],[12]\\
MACS J2243.3-0935   \dotfill & 1.76$^{+0.81}_{-0.55}$ & Unknown & NCC & 1.73$^{+0.44}_{-0.32}$ & No core \& highly flattened along WNW-ESE axis & [27]\\
ABELL 2069          \dotfill & 1.32$^{+0.17}_{-0.14}$ & Y       & NCC & 1.00$^{+0.18}_{-0.14}$ & No core \& highly elongated NNW-SSE & [13]\\
ABELL 2384          \dotfill & 1.31$^{+0.16}_{-0.14}$ & Unknown &  CC & 0.59$^{+0.03}_{-0.03}$ & Gas tail extending 1.1 Mpc from core & [27]\\
ABELL 168           \dotfill & 1.31$^{+0.16}_{-0.14}$ & Y       & NCC & 1.16$^{+0.14}_{-0.10}$ & Highly disrupted \& irregular & [14],[15]\\
ABELL 209           \dotfill & 1.38$^{+0.28}_{-0.22}$ & Y       & NCC & 1.08$^{+0.22}_{-0.17}$ & Asymmetric core structure \& possible cold front & [16]\\
ABELL 665           \dotfill & 1.29$^{+0.15}_{-0.13}$ & Y       & NCC & 1.14$^{+0.19}_{-0.15}$ & Wide, broad gas tail to north \& cold front & [17]\\
1E0657-56           \dotfill & 1.21$^{+0.06}_{-0.05}$ & Y       & NCC & 1.04$^{+0.10}_{-0.08}$ & The famous ``Bullet Cluster'' & [18]\\
MACS J0547.0-3904   \dotfill & 1.51$^{+0.50}_{-0.36}$ & Unknown & NCC & 0.79$^{+0.11}_{-0.09}$ & Bright core \& gas spur extending NW & [27]\\
ZWCL 1215           \dotfill & 1.31$^{+0.21}_{-0.18}$ & Unknown & NCC & 0.95$^{+0.15}_{-0.12}$ & No core, flattened along NE-SW axis & [27]\\
ABELL 1204          \dotfill & 1.26$^{+0.17}_{-0.14}$ & Unknown & NCC & 0.96$^{+0.05}_{-0.05}$ & Highly spherical \& bright centralized core & [27]\\
MKW3S               \dotfill & 1.17$^{+0.05}_{-0.05}$ & Y       &  CC & 0.87$^{+0.02}_{-0.02}$ & High mass group, egg shaped \& bright core & [19]\\
MACS J2311+0338     \dotfill & 1.54$^{+0.68}_{-0.42}$ & Unknown & NCC & 0.69$^{+0.20}_{-0.15}$ & Elongated N-S \& disc-like core & [27]\\
ABELL 267           \dotfill & 1.33$^{+0.27}_{-0.21}$ & Unknown & NCC & 1.09$^{+0.20}_{-0.16}$ & Elongated NNE-SSW \& cold front to north & [27]\\
RX J1720.1+2638     \dotfill & 1.22$^{+0.12}_{-0.11}$ & Y       &  CC & 0.73$^{+0.04}_{-0.04}$ & Very spherical, bright peaky core, \& cold front & [20]\\
ABELL 907           \dotfill & 1.20$^{+0.09}_{-0.08}$ & Unknown &  CC & 0.77$^{+0.03}_{-0.03}$ & NW-SW elongation \& western cold front & [27]\\
ABELL 514           \dotfill & 1.26$^{+0.19}_{-0.15}$ & Y       & NCC & 1.56$^{+1.07}_{-0.40}$ & Very diffuse \& disrupted & [21]\\
ABELL 1651          \dotfill & 1.24$^{+0.16}_{-0.13}$ & Y       & NCC & 1.07$^{+0.10}_{-0.08}$ & Spherical \& compressed isophotes to SW & [22]\\
3C 28.0             \dotfill & 1.23$^{+0.14}_{-0.12}$ & Y       &  CC & 0.54$^{+0.03}_{-0.03}$ & Obvious merger \& $\sim$1 Mpc gas tail & [23]\\[0.5cm]
\hline
\multicolumn{7}{c}{$R_{5000-\mathrm{CORE}}$ Only}\\
\hline
TRIANG AUSTR        \dotfill & 1.42$^{+0.14}_{-0.14}$ &    Y &   CC &  0.86$^{+0.07}_{-0.06}$ &     descrip & [24]\\
ABELL 3158          \dotfill & 1.23$^{+0.05}_{-0.05}$ &    Y &  NCC &  1.15$^{+0.05}_{-0.05}$ &     descrip & [25]\\
ABELL 2256          \dotfill & 1.29$^{+0.13}_{-0.12}$ &    Y &  NCC &  1.40$^{+0.15}_{-0.12}$ &     descrip & [26]\\
NGC 6338            \dotfill & 1.22$^{+0.12}_{-0.10}$ & Unknown &  NCC &  0.96$^{+0.04}_{-0.03}$ &     descrip & [27]\\
ABELL 2029          \dotfill & 1.21$^{+0.12}_{-0.10}$ &    Y &   CC &  0.86$^{+0.04}_{-0.04}$ &     descrip & [27]\\
\enddata
\tablecomments{Clusters ordered by lower limit of $T_{HBR}$.
[1]  \cite{1994ApJS...94..583G}, %[M1008]
[2]  \cite{2003ApJ...593..291K}, %[A2034]
[3]  \cite{2005ChJAA...5..126Y}, %[A0401]
[4]  \cite{1998ApJ...503...77M}, %[A3376]
[5]  \cite{2006Sci...314..791B}, %[A3376]
[6]  \cite{1990ApJS...72..715T}, %[A1689]
[7]  \cite{2004ApJ...607..190A}, %[A1689]
[8]  \cite{1995ApJ...446..583B}, %[A2255]
[9]  \cite{1997A&A...317..432F}, %[A2255]
[10] \cite{1997ApJ...490...56G}, %[A2218]
[11] \cite{2002ApJS..139..313D}, %[A1763]
[12] \cite{2005MNRAS.359..417S}, %[A1763]
[13] \cite{1982ApJ...255L..17G}, %[A2069]
[14] \cite{2004ApJ...610L..81H}, %[A0168]
[15] \cite{2004ApJ...614..692Y}, %[A0168]
[16] \cite{2003A&A...408...57M}, %[A0209]
[17] \cite{2000ApJ...540..726G}, %[A0665]
[18] \cite{1998ApJ...496L...5T}, %[1E065]
[19] \cite{1999AcA....49..403K}, %[MKW3s]
[20] \cite{2001ApJ...555..205M}, %[R1720]
[21] \cite{2001A&A...379..807G}, %[A0514]
[22] \cite{1998MNRAS.301..609B}, %[A1651]
[23] \cite{2005ApJ...619..161G}, %[3C280]
[24] \cite{1996ApJ...472L..17M}, %[TRIAS]
[25] \cite{2001ASPC..251..474O}, %[A3158]
[26] \cite{2000ApJ...534L..43M}, %[A2256]
[27] this work.
}
\end{deluxetable}


%%%%%%%%%%%%%%%%%%%%%%%%%%%%%%
% R2500 Spectral fit results %
%%%%%%%%%%%%%%%%%%%%%%%%%%%%%%

\clearpage
\begin{deluxetable}{lcccccccccc}
\tablewidth{0pt}
\tabletypesize{\scriptsize}
\tablecaption{Summary of Excised R$_{2500}$ Spectral Fits\label{tab:r2500specfits}}
\tablehead{\colhead{Cluster} & \colhead{R$_{\mathrm{CORE}}$} & \colhead{R$_{2500}$ } & \colhead{N$_{HI}$} & \colhead{T$_{77}$} & \colhead{T$_{27}$} & \colhead{T$_{HBR}$} & \colhead{Z$_{77}$} & \colhead{$\chi^2_{red,77}$} & \colhead{$\chi^2_{red,27}$} & \colhead{\% Source}\\
\colhead{ } & \colhead{kpc} & \colhead{kpc} & \colhead{$10^{20}$ cm$^{-2}$} & \colhead{keV} & \colhead{keV} & \colhead{ } & \colhead{Z$_{\sun}$} & \colhead{ } & \colhead{ } & \colhead{ }\\
\colhead{{(1)}} & \colhead{{(2)}} & \colhead{{(3)}} & \colhead{{(4)}} & \colhead{{(5)}} & \colhead{{(6)}} & \colhead{{(7)}} & \colhead{{(8)}} & \colhead{{(9)}} & \colhead{{(10)}} & \colhead{{(11)}}
}
\startdata
1E0657 56 $\ddagger$ &    69 &   688 & 6.53  & 11.99  $^{+0.27   }_{-0.26   }$  & 14.54  $^{+0.67   }_{-0.53   }$  & 1.21   $^{+0.06   }_{-0.05   }$  & 0.29$^{+0.03   }_{-0.02   }$  & 1.24 & 1.11 &  92\\
1RXS J2129.4-0741 $\ddagger$ &    71 &   526 & 4.36  & 8.22   $^{+1.18   }_{-0.95   }$  & 8.10   $^{+1.47   }_{-1.10   }$  & 0.99   $^{+0.23   }_{-0.18   }$  & 0.43$^{+0.18   }_{-0.17   }$  & 1.07 & 1.05 &  80\\
2PIGG J0011.5-2850 &    69 &   547 & 2.18  & 5.15   $^{+0.25   }_{-0.24   }$  & 6.20   $^{+0.79   }_{-0.65   }$  & 1.20   $^{+0.16   }_{-0.14   }$  & 0.26$^{+0.09   }_{-0.08   }$  & 1.09 & 1.00 &  70\\
2PIGG J2227.0-3041 &    69 &   378 & 1.11  & 2.80   $^{+0.15   }_{-0.14   }$  & 2.97   $^{+0.34   }_{-0.27   }$  & 1.06   $^{+0.13   }_{-0.11   }$  & 0.35$^{+0.09   }_{-0.08   }$  & 1.16 & 1.15 &  69\\
3C 220.1 &    73 &   456 & 1.91  & 9.26   $^{+14.71  }_{-3.98   }$  & 8.00   $^{+17.66  }_{-4.03   }$  & 0.86   $^{+2.35   }_{-0.57   }$  & 0.00$^{+0.59   }_{-0.00   }$  & 1.20 & 1.40 &  30\\
3C 28.0 &    70 &   420 & 5.71  & 5.53   $^{+0.29   }_{-0.27   }$  & 6.81   $^{+0.71   }_{-0.60   }$  & 1.23   $^{+0.14   }_{-0.12   }$  & 0.30$^{+0.08   }_{-0.07   }$  & 0.98 & 0.88 &  87\\
3C 295 &    69 &   465 & 1.35  & 5.16   $^{+0.42   }_{-0.38   }$  & 5.93   $^{+0.84   }_{-0.69   }$  & 1.15   $^{+0.19   }_{-0.16   }$  & 0.38$^{+0.12   }_{-0.11   }$  & 0.91 & 0.93 &  79\\
3C 388 &    69 &   420 & 6.11  & 3.23   $^{+0.23   }_{-0.21   }$  & 3.26   $^{+0.49   }_{-0.37   }$  & 1.01   $^{+0.17   }_{-0.13   }$  & 0.51$^{+0.16   }_{-0.14   }$  & 0.95 & 0.95 &  68\\
4C 55.16 &    69 &   426 & 4.00  & 4.98   $^{+0.17   }_{-0.17   }$  & 5.54   $^{+0.40   }_{-0.36   }$  & 1.11   $^{+0.09   }_{-0.08   }$  & 0.49$^{+0.07   }_{-0.07   }$  & 0.89 & 0.80 &  58\\
ABELL 0068 &    72 &   680 & 4.60  & 9.01   $^{+1.53   }_{-1.14   }$  & 9.13   $^{+2.60   }_{-1.71   }$  & 1.01   $^{+0.34   }_{-0.23   }$  & 0.46$^{+0.24   }_{-0.22   }$  & 1.15 & 1.13 &  79\\
ABELL 0168 $\ddagger$ &    72 &   398 & 3.27  & 2.56   $^{+0.11   }_{-0.08   }$  & 3.36   $^{+0.37   }_{-0.35   }$  & 1.31   $^{+0.16   }_{-0.14   }$  & 0.29$^{+0.06   }_{-0.04   }$  & 1.07 & 1.03 &  40\\
ABELL 0209 $\ddagger$ &    70 &   609 & 1.68  & 7.30   $^{+0.59   }_{-0.51   }$  & 10.07  $^{+1.91   }_{-1.41   }$  & 1.38   $^{+0.28   }_{-0.22   }$  & 0.23$^{+0.10   }_{-0.09   }$  & 1.12 & 1.11 &  82\\
ABELL 0267 $\ddagger$ &    70 &   545 & 2.74  & 6.70   $^{+0.56   }_{-0.47   }$  & 8.88   $^{+1.68   }_{-1.27   }$  & 1.33   $^{+0.27   }_{-0.21   }$  & 0.32$^{+0.11   }_{-0.11   }$  & 1.18 & 1.15 &  82\\
ABELL 0370 &    69 &   516 & 3.37  & 7.35   $^{+0.72   }_{-0.84   }$  & 10.35  $^{+1.89   }_{-2.27   }$  & 1.41   $^{+0.29   }_{-0.35   }$  & 0.45$^{+0.06   }_{-0.23   }$  & 1.08 & 1.04 &  39\\
ABELL 0383 &    69 &   423 & 4.07  & 4.91   $^{+0.29   }_{-0.27   }$  & 5.42   $^{+0.74   }_{-0.59   }$  & 1.10   $^{+0.16   }_{-0.13   }$  & 0.44$^{+0.11   }_{-0.11   }$  & 0.97 & 0.90 &  64\\
ABELL 0399 &    69 &   546 & 7.57$^{+0.71   }_{-0.71   }$  & 7.95   $^{+0.35   }_{-0.31   }$  & 8.87   $^{+0.55   }_{-0.50   }$  & 1.12   $^{+0.08   }_{-0.08   }$  & 0.30$^{+0.05   }_{-0.05   }$  & 1.12 & 0.99 &  82\\
ABELL 0401 &    69 &   643 & 12.48 & 6.37   $^{+0.19   }_{-0.19   }$  & 8.71   $^{+0.72   }_{-0.61   }$  & 1.37   $^{+0.12   }_{-0.10   }$  & 0.26$^{+0.06   }_{-0.06   }$  & 1.44 & 1.05 &  78\\
ABELL 0478 &    69 &   598 & 30.90 & 7.30   $^{+0.26   }_{-0.24   }$  & 8.62   $^{+0.58   }_{-0.54   }$  & 1.18   $^{+0.09   }_{-0.08   }$  & 0.45$^{+0.06   }_{-0.05   }$  & 1.05 & 0.95 &  91\\
ABELL 0514 &    73 &   516 & 3.14  & 3.33   $^{+0.16   }_{-0.16   }$  & 4.02   $^{+0.54   }_{-0.46   }$  & 1.21   $^{+0.17   }_{-0.15   }$  & 0.25$^{+0.08   }_{-0.06   }$  & 1.07 & 0.97 &  53\\
ABELL 0520 &    70 &   576 & 1.06$^{+1.06   }_{-1.05   }$  & 9.29   $^{+0.67   }_{-0.60   }$  & 9.88   $^{+0.85   }_{-0.73   }$  & 1.06   $^{+0.12   }_{-0.10   }$  & 0.37$^{+0.07   }_{-0.07   }$  & 1.11 & 1.04 &  87\\
ABELL 0521 &    72 &   558 & 6.17  & 7.03   $^{+0.59   }_{-0.53   }$  & 8.39   $^{+1.62   }_{-1.22   }$  & 1.19   $^{+0.25   }_{-0.20   }$  & 0.39$^{+0.13   }_{-0.12   }$  & 1.10 & 1.15 &  49\\
ABELL 0586 &    70 &   635 & 4.71  & 6.47   $^{+0.55   }_{-0.47   }$  & 8.06   $^{+1.46   }_{-1.11   }$  & 1.25   $^{+0.25   }_{-0.19   }$  & 0.56$^{+0.17   }_{-0.16   }$  & 0.91 & 0.81 &  82\\
ABELL 0611 &    70 &   523 & 4.99  & 7.06   $^{+0.55   }_{-0.48   }$  & 7.97   $^{+1.09   }_{-0.91   }$  & 1.13   $^{+0.18   }_{-0.15   }$  & 0.35$^{+0.11   }_{-0.10   }$  & 0.97 & 0.98 &  54\\
ABELL 0665 &    69 &   617 & 4.24  & 7.45   $^{+0.38   }_{-0.34   }$  & 9.61   $^{+1.02   }_{-0.85   }$  & 1.29   $^{+0.15   }_{-0.13   }$  & 0.31$^{+0.06   }_{-0.07   }$  & 1.02 & 0.93 &  87\\
ABELL 0697 &    69 &   612 & 3.34  & 9.52   $^{+0.87   }_{-0.76   }$  & 12.24  $^{+2.05   }_{-1.63   }$  & 1.29   $^{+0.25   }_{-0.20   }$  & 0.37$^{+0.12   }_{-0.11   }$  & 1.08 & 1.02 &  89\\
ABELL 0773 &    69 &   615 & 1.46  & 7.83   $^{+0.66   }_{-0.57   }$  & 9.75   $^{+1.65   }_{-1.27   }$  & 1.25   $^{+0.24   }_{-0.19   }$  & 0.44$^{+0.12   }_{-0.12   }$  & 1.06 & 1.09 &  84\\
ABELL 0907 &    69 &   488 & 5.69  & 5.62   $^{+0.18   }_{-0.17   }$  & 6.78   $^{+0.49   }_{-0.43   }$  & 1.21   $^{+0.10   }_{-0.08   }$  & 0.42$^{+0.06   }_{-0.05   }$  & 1.13 & 1.00 &  88\\
ABELL 0963 &    69 &   543 & 1.39  & 6.73   $^{+0.32   }_{-0.30   }$  & 6.98   $^{+0.66   }_{-0.57   }$  & 1.04   $^{+0.11   }_{-0.10   }$  & 0.29$^{+0.07   }_{-0.08   }$  & 1.06 & 1.02 &  64\\
ABELL 1063S &    69 &   648 & 1.77  & 11.96  $^{+0.88   }_{-0.79   }$  & 13.70  $^{+1.68   }_{-1.38   }$  & 1.15   $^{+0.16   }_{-0.14   }$  & 0.38$^{+0.09   }_{-0.09   }$  & 1.02 & 0.98 &  90\\
ABELL 1204 &    70 &   419 & 1.44  & 3.63   $^{+0.18   }_{-0.16   }$  & 4.58   $^{+0.57   }_{-0.45   }$  & 1.26   $^{+0.17   }_{-0.14   }$  & 0.31$^{+0.09   }_{-0.09   }$  & 1.06 & 0.90 &  88\\
ABELL 1423 &    70 &   614 & 1.60  & 6.01   $^{+0.75   }_{-0.64   }$  & 7.53   $^{+2.35   }_{-1.55   }$  & 1.25   $^{+0.42   }_{-0.29   }$  & 0.30$^{+0.18   }_{-0.17   }$  & 0.87 & 0.65 &  78\\
ABELL 1651 &    70 &   596 & 2.02  & 6.26   $^{+0.30   }_{-0.27   }$  & 7.78   $^{+0.90   }_{-0.76   }$  & 1.24   $^{+0.16   }_{-0.13   }$  & 0.42$^{+0.09   }_{-0.09   }$  & 1.19 & 1.20 &  86\\
ABELL 1689 $\ddagger$ &    70 &   679 & 1.87  & 9.48   $^{+0.38   }_{-0.35   }$  & 12.89  $^{+1.23   }_{-1.01   }$  & 1.36   $^{+0.14   }_{-0.12   }$  & 0.36$^{+0.06   }_{-0.05   }$  & 1.13 & 1.02 &  91\\
ABELL 1758 &    69 &   574 & 1.09  & 12.14  $^{+1.15   }_{-0.92   }$  & 11.16  $^{+3.08   }_{-2.14   }$  & 0.92   $^{+0.27   }_{-0.19   }$  & 0.56$^{+0.13   }_{-0.13   }$  & 1.21 & 1.09 &  58\\
ABELL 1763 &    69 &   561 & 0.82  & 7.78   $^{+0.67   }_{-0.60   }$  & 11.49  $^{+2.89   }_{-1.84   }$  & 1.48   $^{+0.39   }_{-0.26   }$  & 0.25$^{+0.11   }_{-0.10   }$  & 1.12 & 0.92 &  84\\
ABELL 1835 &    70 &   570 & 2.36  & 9.77   $^{+0.57   }_{-0.52   }$  & 11.00  $^{+1.23   }_{-1.03   }$  & 1.13   $^{+0.14   }_{-0.12   }$  & 0.31$^{+0.08   }_{-0.07   }$  & 0.98 & 1.02 &  86\\
ABELL 1914 &    70 &   698 & 0.97  & 9.62   $^{+0.55   }_{-0.49   }$  & 11.42  $^{+1.26   }_{-1.06   }$  & 1.19   $^{+0.15   }_{-0.13   }$  & 0.30$^{+0.08   }_{-0.07   }$  & 1.07 & 1.03 &  92\\
ABELL 1942 &    69 &   473 & 2.75  & 4.77   $^{+0.38   }_{-0.35   }$  & 5.49   $^{+0.98   }_{-0.74   }$  & 1.15   $^{+0.22   }_{-0.18   }$  & 0.33$^{+0.12   }_{-0.14   }$  & 1.06 & 1.04 &  70\\
ABELL 1995 &    73 &   381 & 1.44  & 8.37   $^{+0.70   }_{-0.61   }$  & 9.23   $^{+1.44   }_{-1.13   }$  & 1.10   $^{+0.20   }_{-0.16   }$  & 0.39$^{+0.12   }_{-0.11   }$  & 1.02 & 0.96 &  74\\
ABELL 2034 &    69 &   594 & 1.58  & 7.15   $^{+0.23   }_{-0.22   }$  & 10.02  $^{+0.92   }_{-0.75   }$  & 1.40   $^{+0.14   }_{-0.11   }$  & 0.32$^{+0.05   }_{-0.05   }$  & 1.22 & 1.00 &  84\\
ABELL 2069 &    70 &   623 & 1.97  & 6.50   $^{+0.33   }_{-0.29   }$  & 8.61   $^{+1.02   }_{-0.84   }$  & 1.32   $^{+0.17   }_{-0.14   }$  & 0.26$^{+0.08   }_{-0.07   }$  & 1.04 & 0.96 &  71\\
ABELL 2111 &    70 &   592 & 2.20  & 7.13   $^{+1.29   }_{-0.95   }$  & 11.10  $^{+4.67   }_{-3.05   }$  & 1.56   $^{+0.71   }_{-0.48   }$  & 0.13$^{+0.19   }_{-0.13   }$  & 1.06 & 0.88 &  76\\
ABELL 2125 &    70 &   371 & 2.75  & 2.88   $^{+0.30   }_{-0.27   }$  & 3.76   $^{+0.98   }_{-0.65   }$  & 1.31   $^{+0.37   }_{-0.26   }$  & 0.31$^{+0.18   }_{-0.16   }$  & 1.26 & 1.30 &  61\\
ABELL 2163 &    69 &   751 & 12.04 & 19.20  $^{+0.87   }_{-0.80   }$  & 21.30  $^{+1.77   }_{-1.47   }$  & 1.11   $^{+0.11   }_{-0.09   }$  & 0.10$^{+0.06   }_{-0.06   }$  & 1.37 & 1.26 &  90\\
ABELL 2204 &    70 &   575 & 5.84  & 8.65   $^{+0.58   }_{-0.52   }$  & 10.57  $^{+1.48   }_{-1.23   }$  & 1.22   $^{+0.19   }_{-0.16   }$  & 0.37$^{+0.10   }_{-0.09   }$  & 0.95 & 1.00 &  90\\
ABELL 2218 &    70 &   558 & 3.12  & 7.35   $^{+0.39   }_{-0.35   }$  & 10.03  $^{+1.26   }_{-0.98   }$  & 1.36   $^{+0.19   }_{-0.15   }$  & 0.22$^{+0.07   }_{-0.06   }$  & 1.01 & 0.90 &  87\\
ABELL 2255 &    73 &   596 & 2.53  & 6.12   $^{+0.20   }_{-0.19   }$  & 8.10   $^{+0.66   }_{-0.58   }$  & 1.32   $^{+0.12   }_{-0.10   }$  & 0.30$^{+0.06   }_{-0.06   }$  & 1.13 & 0.95 &  76\\
ABELL 2259 &    69 &   480 & 3.70  & 5.18   $^{+0.46   }_{-0.39   }$  & 6.40   $^{+1.33   }_{-0.95   }$  & 1.24   $^{+0.28   }_{-0.21   }$  & 0.41$^{+0.14   }_{-0.14   }$  & 1.05 & 1.01 &  85\\
ABELL 2261 &    69 &   576 & 3.31  & 7.63   $^{+0.47   }_{-0.43   }$  & 9.30   $^{+1.21   }_{-0.91   }$  & 1.22   $^{+0.18   }_{-0.14   }$  & 0.36$^{+0.08   }_{-0.08   }$  & 0.99 & 0.95 &  90\\
ABELL 2294 &    69 &   572 & 6.10  & 9.98   $^{+1.43   }_{-1.12   }$  & 11.07  $^{+3.19   }_{-2.11   }$  & 1.11   $^{+0.36   }_{-0.25   }$  & 0.53$^{+0.21   }_{-0.21   }$  & 1.07 & 0.95 &  82\\
ABELL 2384 &    70 &   436 & 2.99  & 4.75   $^{+0.22   }_{-0.20   }$  & 6.22   $^{+0.72   }_{-0.60   }$  & 1.31   $^{+0.16   }_{-0.14   }$  & 0.23$^{+0.07   }_{-0.07   }$  & 1.06 & 0.92 &  81\\
ABELL 2409 &    70 &   511 & 6.72  & 5.94   $^{+0.43   }_{-0.38   }$  & 6.77   $^{+0.99   }_{-0.82   }$  & 1.14   $^{+0.19   }_{-0.16   }$  & 0.37$^{+0.13   }_{-0.11   }$  & 1.13 & 0.96 &  88\\
ABELL 2537 &    69 &   497 & 4.26  & 8.40   $^{+0.76   }_{-0.68   }$  & 7.81   $^{+1.15   }_{-0.93   }$  & 0.93   $^{+0.16   }_{-0.13   }$  & 0.40$^{+0.13   }_{-0.13   }$  & 0.91 & 0.84 &  46\\
ABELL 2631 &    68 &   631 & 3.74  & 7.06   $^{+1.06   }_{-0.84   }$  & 7.83   $^{+2.18   }_{-1.45   }$  & 1.11   $^{+0.35   }_{-0.24   }$  & 0.34$^{+0.19   }_{-0.18   }$  & 0.97 & 0.88 &  83\\
ABELL 2667 &    70 &   525 & 1.64  & 6.75   $^{+0.48   }_{-0.43   }$  & 7.45   $^{+1.06   }_{-0.88   }$  & 1.10   $^{+0.18   }_{-0.15   }$  & 0.36$^{+0.11   }_{-0.11   }$  & 1.17 & 1.08 &  76\\
ABELL 2670 &    69 &   451 & 2.88  & 3.95   $^{+0.14   }_{-0.12   }$  & 4.65   $^{+0.42   }_{-0.36   }$  & 1.18   $^{+0.11   }_{-0.10   }$  & 0.42$^{+0.08   }_{-0.06   }$  & 1.13 & 1.07 &  70\\
ABELL 2717 &    70 &   298 & 1.12  & 2.63   $^{+0.17   }_{-0.16   }$  & 3.17   $^{+0.58   }_{-0.43   }$  & 1.21   $^{+0.23   }_{-0.18   }$  & 0.48$^{+0.13   }_{-0.10   }$  & 0.88 & 0.87 &  55\\
ABELL 2744 &    71 &   647 & 1.82  & 9.18   $^{+0.68   }_{-0.60   }$  & 10.20  $^{+1.38   }_{-1.10   }$  & 1.11   $^{+0.17   }_{-0.14   }$  & 0.24$^{+0.10   }_{-0.09   }$  & 0.99 & 0.90 &  67\\
ABELL 3164 &    72 &   451 & 2.55  & 2.83   $^{+0.53   }_{-0.26   }$  & 3.81   $^{+3.56   }_{-1.42   }$  & 1.35   $^{+1.28   }_{-0.52   }$  & 0.39$^{+0.33   }_{-0.21   }$  & 0.88 & 0.94 &  29\\
ABELL 3376 $\ddagger$ &    70 &   463 & 5.21  & 4.48   $^{+0.11   }_{-0.12   }$  & 5.95   $^{+0.47   }_{-0.42   }$  & 1.33   $^{+0.11   }_{-0.10   }$  & 0.39$^{+0.05   }_{-0.08   }$  & 1.16 & 1.09 &  63\\
ABELL 3921 &    69 &   535 & 3.07  & 5.70   $^{+0.24   }_{-0.23   }$  & 6.65   $^{+0.65   }_{-0.54   }$  & 1.17   $^{+0.12   }_{-0.11   }$  & 0.31$^{+0.08   }_{-0.07   }$  & 1.02 & 0.96 &  77\\
AC 114 &    72 &   550 & 1.44  & 7.53   $^{+0.49   }_{-0.44   }$  & 8.30   $^{+1.03   }_{-0.85   }$  & 1.10   $^{+0.15   }_{-0.13   }$  & 0.26$^{+0.08   }_{-0.09   }$  & 1.07 & 1.06 &  55\\
CL 0024+17 &    73 &   435 & 4.36  & 6.03   $^{+1.66   }_{-1.10   }$  & 7.18   $^{+7.91   }_{-3.16   }$  & 1.19   $^{+1.35   }_{-0.57   }$  & 0.60$^{+0.37   }_{-0.33   }$  & 1.00 & 1.44 &  37\\
CL 1221+4918 &    71 &   445 & 1.44  & 6.62   $^{+1.24   }_{-0.99   }$  & 7.11   $^{+1.73   }_{-1.31   }$  & 1.07   $^{+0.33   }_{-0.25   }$  & 0.34$^{+0.20   }_{-0.18   }$  & 0.94 & 0.93 &  62\\
CL J0030+2618 &    72 &   786 & 4.10  & 4.63   $^{+2.72   }_{-1.32   }$  & 5.18   $^{+8.29   }_{-1.96   }$  & 1.12   $^{+1.91   }_{-0.53   }$  & 0.26$^{+0.75   }_{-0.26   }$  & 1.00 & 1.23 &  37\\
CL J0152-1357 &    72 &   391 & 1.45  & 7.33   $^{+2.78   }_{-1.77   }$  & 7.31   $^{+3.43   }_{-2.02   }$  & 1.00   $^{+0.60   }_{-0.37   }$  & 0.00$^{+0.24   }_{-0.00   }$  & 0.89 & 1.00 &  36\\
CL J0542.8-4100 &    71 &   446 & 3.59  & 6.07   $^{+1.47   }_{-1.05   }$  & 6.29   $^{+2.14   }_{-1.41   }$  & 1.04   $^{+0.43   }_{-0.29   }$  & 0.16$^{+0.23   }_{-0.16   }$  & 1.04 & 0.91 &  66\\
CL J0848+4456 $\ddagger$ &    71 &   319 & 2.53  & 4.53   $^{+1.57   }_{-1.13   }$  & 5.52   $^{+3.28   }_{-1.74   }$  & 1.22   $^{+0.84   }_{-0.49   }$  & 0.00$^{+0.45   }_{-0.00   }$  & 0.92 & 0.93 &  58\\
CL J1113.1-2615 &    72 &   435 & 5.51  & 4.19   $^{+1.61   }_{-1.02   }$  & 4.10   $^{+2.47   }_{-1.44   }$  & 0.98   $^{+0.70   }_{-0.42   }$  & 0.46$^{+0.63   }_{-0.44   }$  & 1.01 & 1.08 &  23\\
CL J1226.9+3332 $\ddagger$ &    69 &   450 & 1.37  & 11.81  $^{+2.25   }_{-1.70   }$  & 11.29  $^{+2.45   }_{-1.77   }$  & 0.96   $^{+0.28   }_{-0.20   }$  & 0.21$^{+0.21   }_{-0.21   }$  & 0.81 & 0.86 &  86\\
CL J2302.8+0844 &    72 &   514 & 5.05  & 4.25   $^{+1.17   }_{-1.32   }$  & 4.67   $^{+2.00   }_{-1.80   }$  & 1.10   $^{+0.56   }_{-0.54   }$  & 0.13$^{+0.33   }_{-0.13   }$  & 0.89 & 0.97 &  50\\
DLS J0514-4904 &    72 &   507 & 2.52  & 4.62   $^{+0.53   }_{-0.47   }$  & 6.14   $^{+2.08   }_{-1.34   }$  & 1.33   $^{+0.48   }_{-0.32   }$  & 0.37$^{+0.24   }_{-0.20   }$  & 1.04 & 1.12 &  54\\
MACS J0011.7-1523 $\ddagger$ &    69 &   451 & 2.08  & 6.49   $^{+0.48   }_{-0.43   }$  & 6.76   $^{+0.81   }_{-0.66   }$  & 1.04   $^{+0.15   }_{-0.12   }$  & 0.30$^{+0.10   }_{-0.09   }$  & 0.86 & 0.90 &  87\\
MACS J0025.4-1222 $\ddagger$ &    72 &   473 & 2.72  & 6.33   $^{+0.85   }_{-0.70   }$  & 6.01   $^{+1.05   }_{-0.85   }$  & 0.95   $^{+0.21   }_{-0.17   }$  & 0.37$^{+0.16   }_{-0.15   }$  & 0.90 & 0.92 &  80\\
MACS J0035.4-2015 &    70 &   527 & 1.55  & 7.46   $^{+0.79   }_{-0.66   }$  & 9.31   $^{+1.75   }_{-1.29   }$  & 1.25   $^{+0.27   }_{-0.21   }$  & 0.33$^{+0.12   }_{-0.12   }$  & 0.94 & 0.93 &  90\\
MACS J0111.5+0855 &    72 &   435 & 4.18  & 4.11   $^{+1.61   }_{-1.05   }$  & 3.72   $^{+3.08   }_{-1.29   }$  & 0.91   $^{+0.83   }_{-0.39   }$  & 0.11$^{+0.59   }_{-0.11   }$  & 0.68 & 0.65 &  49\\
MACS J0152.5-2852 &    72 &   459 & 1.46  & 5.64   $^{+0.89   }_{-0.70   }$  & 7.24   $^{+2.57   }_{-1.59   }$  & 1.28   $^{+0.50   }_{-0.32   }$  & 0.22$^{+0.17   }_{-0.17   }$  & 1.10 & 1.02 &  84\\
MACS J0159.0-3412 &    72 &   572 & 1.54  & 10.90  $^{+4.77   }_{-2.53   }$  & 14.65  $^{+12.31  }_{-5.39   }$  & 1.34   $^{+1.27   }_{-0.58   }$  & 0.26$^{+0.35   }_{-0.26   }$  & 0.87 & 0.92 &  81\\
MACS J0159.8-0849 $\ddagger$ &    69 &   585 & 2.01  & 9.16   $^{+0.71   }_{-0.63   }$  & 9.83   $^{+1.13   }_{-0.96   }$  & 1.07   $^{+0.15   }_{-0.13   }$  & 0.30$^{+0.09   }_{-0.09   }$  & 1.08 & 1.09 &  90\\
MACS J0242.5-2132 &    70 &   498 & 2.71  & 5.58   $^{+0.63   }_{-0.52   }$  & 6.26   $^{+1.38   }_{-0.99   }$  & 1.12   $^{+0.28   }_{-0.21   }$  & 0.34$^{+0.16   }_{-0.15   }$  & 1.03 & 0.83 &  87\\
MACS J0257.1-2325 $\ddagger$ &    70 &   579 & 2.09  & 9.25   $^{+1.28   }_{-1.01   }$  & 10.16  $^{+1.95   }_{-1.54   }$  & 1.10   $^{+0.26   }_{-0.21   }$  & 0.14$^{+0.12   }_{-0.12   }$  & 0.99 & 1.08 &  84\\
MACS J0257.6-2209 &    69 &   540 & 2.02  & 8.02   $^{+1.12   }_{-0.88   }$  & 8.17   $^{+1.92   }_{-1.30   }$  & 1.02   $^{+0.28   }_{-0.20   }$  & 0.30$^{+0.16   }_{-0.17   }$  & 1.12 & 1.26 &  84\\
MACS J0308.9+2645 &    69 &   539 & 11.88 & 10.54  $^{+1.28   }_{-1.07   }$  & 11.38  $^{+2.16   }_{-1.66   }$  & 1.08   $^{+0.24   }_{-0.19   }$  & 0.28$^{+0.13   }_{-0.14   }$  & 0.97 & 1.01 &  87\\
MACS J0329.6-0211 $\ddagger$ &    68 &   420 & 6.21  & 6.30   $^{+0.47   }_{-0.41   }$  & 7.50   $^{+0.83   }_{-0.69   }$  & 1.19   $^{+0.16   }_{-0.13   }$  & 0.41$^{+0.10   }_{-0.09   }$  & 1.10 & 1.17 &  86\\
MACS J0404.6+1109 &    74 &   494 & 14.96 & 5.77   $^{+1.14   }_{-0.88   }$  & 6.15   $^{+2.00   }_{-1.30   }$  & 1.07   $^{+0.41   }_{-0.28   }$  & 0.24$^{+0.22   }_{-0.20   }$  & 0.85 & 0.78 &  73\\
MACS J0417.5-1154 &    70 &   429 & 4.00  & 11.07  $^{+1.98   }_{-1.49   }$  & 14.90  $^{+5.03   }_{-3.24   }$  & 1.35   $^{+0.51   }_{-0.34   }$  & 0.33$^{+0.19   }_{-0.19   }$  & 1.07 & 0.97 &  94\\
MACS J0429.6-0253 &    69 &   495 & 5.70  & 5.66   $^{+0.64   }_{-0.54   }$  & 6.71   $^{+1.26   }_{-0.98   }$  & 1.19   $^{+0.26   }_{-0.21   }$  & 0.35$^{+0.14   }_{-0.13   }$  & 1.21 & 1.12 &  82\\
MACS J0451.9+0006 &    72 &   459 & 7.65  & 5.80   $^{+1.46   }_{-1.03   }$  & 7.02   $^{+3.29   }_{-1.80   }$  & 1.21   $^{+0.64   }_{-0.38   }$  & 0.51$^{+0.33   }_{-0.29   }$  & 1.25 & 1.35 &  83\\
MACS J0455.2+0657 &    71 &   481 & 10.45 & 7.25   $^{+2.04   }_{-1.33   }$  & 8.25   $^{+3.98   }_{-2.10   }$  & 1.14   $^{+0.64   }_{-0.36   }$  & 0.56$^{+0.37   }_{-0.33   }$  & 0.83 & 0.94 &  82\\
MACS J0520.7-1328 &    69 &   492 & 8.88  & 6.35   $^{+0.81   }_{-0.67   }$  & 8.22   $^{+2.18   }_{-1.45   }$  & 1.29   $^{+0.38   }_{-0.27   }$  & 0.43$^{+0.17   }_{-0.16   }$  & 1.23 & 1.38 &  86\\
MACS J0547.0-3904 &    69 &   364 & 4.08  & 3.58   $^{+0.44   }_{-0.37   }$  & 5.41   $^{+1.67   }_{-1.18   }$  & 1.51   $^{+0.50   }_{-0.36   }$  & 0.09$^{+0.15   }_{-0.09   }$  & 1.16 & 1.42 &  75\\
MACS J0553.4-3342 &    72 &   692 & 2.88  & 13.14  $^{+3.82   }_{-2.50   }$  & 13.86  $^{+6.45   }_{-3.44   }$  & 1.05   $^{+0.58   }_{-0.33   }$  & 0.57$^{+0.35   }_{-0.33   }$  & 0.80 & 0.76 &  87\\
MACS J0717.5+3745 $\ddagger$ &    70 &   563 & 6.75  & 12.77  $^{+1.16   }_{-1.00   }$  & 13.21  $^{+1.58   }_{-1.29   }$  & 1.03   $^{+0.16   }_{-0.13   }$  & 0.30$^{+0.10   }_{-0.11   }$  & 0.93 & 0.90 &  88\\
MACS J0744.8+3927 $\ddagger$ &    70 &   537 & 4.66  & 8.09   $^{+0.77   }_{-0.66   }$  & 8.77   $^{+1.04   }_{-0.87   }$  & 1.08   $^{+0.16   }_{-0.14   }$  & 0.32$^{+0.10   }_{-0.10   }$  & 1.14 & 1.18 &  82\\
MACS J0911.2+1746 $\ddagger$ &    72 &   541 & 3.55  & 7.51   $^{+1.27   }_{-0.99   }$  & 7.17   $^{+1.60   }_{-1.20   }$  & 0.95   $^{+0.27   }_{-0.20   }$  & 0.21$^{+0.17   }_{-0.16   }$  & 0.93 & 0.84 &  78\\
MACS J0949+1708 &    72 &   580 & 3.17  & 9.16   $^{+1.53   }_{-1.18   }$  & 9.11   $^{+2.27   }_{-1.55   }$  & 0.99   $^{+0.30   }_{-0.21   }$  & 0.37$^{+0.20   }_{-0.20   }$  & 0.89 & 0.84 &  89\\
MACS J1006.9+3200 &    72 &   512 & 1.83  & 7.89   $^{+2.78   }_{-1.74   }$  & 8.05   $^{+5.70   }_{-2.45   }$  & 1.02   $^{+0.81   }_{-0.38   }$  & 0.15$^{+0.35   }_{-0.15   }$  & 1.84 & 1.15 &  76\\
MACS J1105.7-1014 &    73 &   502 & 4.58  & 7.54   $^{+2.29   }_{-1.51   }$  & 7.78   $^{+3.93   }_{-1.97   }$  & 1.03   $^{+0.61   }_{-0.33   }$  & 0.22$^{+0.29   }_{-0.22   }$  & 1.17 & 1.27 &  81\\
MACS J1108.8+0906 $\ddagger$ &    74 &   491 & 2.52  & 6.52   $^{+0.94   }_{-0.82   }$  & 7.31   $^{+1.89   }_{-1.29   }$  & 1.12   $^{+0.33   }_{-0.24   }$  & 0.29$^{+0.18   }_{-0.17   }$  & 0.95 & 0.80 &  80\\
MACS J1115.2+5320 $\ddagger$ &    70 &   527 & 0.98  & 8.91   $^{+1.42   }_{-1.12   }$  & 9.58   $^{+2.36   }_{-1.62   }$  & 1.08   $^{+0.32   }_{-0.23   }$  & 0.37$^{+0.20   }_{-0.18   }$  & 0.93 & 0.88 &  75\\
MACS J1115.8+0129 &    70 &   448 & 4.36  & 6.78   $^{+1.17   }_{-0.91   }$  & 8.27   $^{+3.27   }_{-2.16   }$  & 1.22   $^{+0.53   }_{-0.36   }$  & 0.07$^{+0.21   }_{-0.07   }$  & 1.00 & 0.97 &  65\\
MACS J1131.8-1955 &    69 &   576 & 4.49  & 8.64   $^{+1.23   }_{-0.97   }$  & 11.01  $^{+3.61   }_{-2.10   }$  & 1.27   $^{+0.46   }_{-0.28   }$  & 0.42$^{+0.17   }_{-0.17   }$  & 1.00 & 1.00 &  87\\
MACS J1149.5+2223 $\ddagger$ &    69 &   504 & 2.32  & 7.65   $^{+0.89   }_{-0.75   }$  & 8.13   $^{+1.36   }_{-1.04   }$  & 1.06   $^{+0.22   }_{-0.17   }$  & 0.20$^{+0.12   }_{-0.11   }$  & 1.00 & 1.09 &  87\\
MACS J1206.2-0847 &    70 &   522 & 4.15  & 10.21  $^{+1.19   }_{-0.97   }$  & 12.51  $^{+2.44   }_{-1.87   }$  & 1.23   $^{+0.28   }_{-0.22   }$  & 0.33$^{+0.13   }_{-0.13   }$  & 0.96 & 1.05 &  93\\
MACS J1226.8+2153 &    73 &   489 & 1.82  & 4.21   $^{+1.07   }_{-0.80   }$  & 5.02   $^{+3.29   }_{-1.52   }$  & 1.19   $^{+0.84   }_{-0.43   }$  & 0.23$^{+0.38   }_{-0.23   }$  & 1.02 & 0.81 &  67\\
MACS J1311.0-0310 $\ddagger$ &    69 &   425 & 2.18  & 5.76   $^{+0.48   }_{-0.42   }$  & 5.91   $^{+0.73   }_{-0.62   }$  & 1.03   $^{+0.15   }_{-0.13   }$  & 0.39$^{+0.13   }_{-0.11   }$  & 0.96 & 0.98 &  72\\
MACS J1319+7003 &    72 &   496 & 1.53  & 7.99   $^{+2.08   }_{-1.43   }$  & 10.62  $^{+7.35   }_{-3.22   }$  & 1.33   $^{+0.98   }_{-0.47   }$  & 0.30$^{+0.29   }_{-0.28   }$  & 1.25 & 1.24 &  74\\
MACS J1427.2+4407 &    73 &   488 & 1.41  & 9.80   $^{+3.87   }_{-2.53   }$  & 10.35  $^{+6.30   }_{-3.26   }$  & 1.06   $^{+0.77   }_{-0.43   }$  & 0.00$^{+0.34   }_{-0.00   }$  & 0.67 & 0.50 &  84\\
MACS J1427.6-2521 &    73 &   426 & 6.11  & 4.65   $^{+0.92   }_{-0.72   }$  & 8.11   $^{+5.04   }_{-2.77   }$  & 1.74   $^{+1.14   }_{-0.65   }$  & 0.18$^{+0.26   }_{-0.18   }$  & 1.19 & 1.40 &  68\\
MACS J1621.3+3810 $\ddagger$ &    69 &   504 & 1.07  & 7.12   $^{+0.66   }_{-0.55   }$  & 7.09   $^{+0.92   }_{-0.75   }$  & 1.00   $^{+0.16   }_{-0.13   }$  & 0.34$^{+0.11   }_{-0.11   }$  & 0.93 & 0.86 &  73\\
MACS J1731.6+2252 &    73 &   521 & 6.48  & 7.45   $^{+1.32   }_{-0.99   }$  & 10.99  $^{+4.67   }_{-2.46   }$  & 1.48   $^{+0.68   }_{-0.38   }$  & 0.35$^{+0.19   }_{-0.17   }$  & 1.20 & 1.07 &  84\\
MACS J1931.8-2634 &    68 &   535 & 9.13  & 6.97   $^{+0.72   }_{-0.61   }$  & 7.72   $^{+1.31   }_{-0.99   }$  & 1.11   $^{+0.22   }_{-0.17   }$  & 0.27$^{+0.11   }_{-0.12   }$  & 0.95 & 0.86 &  90\\
MACS J2046.0-3430 &    73 &   386 & 4.98  & 4.64   $^{+1.18   }_{-0.82   }$  & 5.49   $^{+2.29   }_{-1.47   }$  & 1.18   $^{+0.58   }_{-0.38   }$  & 0.20$^{+0.32   }_{-0.20   }$  & 0.89 & 1.11 &  82\\
MACS J2049.9-3217 &    69 &   524 & 5.99  & 6.83   $^{+0.84   }_{-0.69   }$  & 8.94   $^{+2.08   }_{-1.48   }$  & 1.31   $^{+0.34   }_{-0.25   }$  & 0.43$^{+0.17   }_{-0.15   }$  & 0.99 & 0.92 &  83\\
MACS J2211.7-0349 &    69 &   663 & 5.86  & 11.30  $^{+1.46   }_{-1.17   }$  & 13.82  $^{+3.54   }_{-2.41   }$  & 1.22   $^{+0.35   }_{-0.25   }$  & 0.15$^{+0.13   }_{-0.14   }$  & 1.24 & 1.26 &  88\\
MACS J2214.9-1359 $\ddagger$ &    70 &   529 & 3.32  & 9.78   $^{+1.38   }_{-1.09   }$  & 10.45  $^{+2.19   }_{-1.56   }$  & 1.07   $^{+0.27   }_{-0.20   }$  & 0.23$^{+0.14   }_{-0.14   }$  & 0.99 & 1.06 &  87\\
MACS J2228+2036 &    70 &   545 & 4.52  & 7.86   $^{+1.08   }_{-0.85   }$  & 9.17   $^{+2.05   }_{-1.46   }$  & 1.17   $^{+0.31   }_{-0.22   }$  & 0.39$^{+0.16   }_{-0.15   }$  & 0.99 & 1.00 &  88\\
MACS J2229.7-2755 &    69 &   465 & 1.34  & 5.01   $^{+0.50   }_{-0.43   }$  & 5.79   $^{+1.11   }_{-0.86   }$  & 1.16   $^{+0.25   }_{-0.20   }$  & 0.55$^{+0.19   }_{-0.18   }$  & 1.05 & 1.08 &  85\\
MACS J2243.3-0935 &    73 &   574 & 4.31  & 4.09   $^{+0.51   }_{-0.45   }$  & 7.20   $^{+3.17   }_{-2.12   }$  & 1.76   $^{+0.81   }_{-0.55   }$  & 0.03$^{+0.15   }_{-0.03   }$  & 1.17 & 0.92 &  51\\
MACS J2245.0+2637 &    69 &   454 & 5.50  & 6.06   $^{+0.63   }_{-0.54   }$  & 6.76   $^{+1.24   }_{-0.93   }$  & 1.12   $^{+0.24   }_{-0.18   }$  & 0.60$^{+0.20   }_{-0.18   }$  & 0.94 & 1.09 &  88\\
MACS J2311+0338 &    72 &   363 & 5.23  & 8.12   $^{+1.44   }_{-1.16   }$  & 12.40  $^{+5.12   }_{-2.88   }$  & 1.53   $^{+0.69   }_{-0.42   }$  & 0.46$^{+0.22   }_{-0.20   }$  & 1.07 & 1.15 &  88\\
MKW3S &    70 &   339 & 3.05  & 3.91   $^{+0.06   }_{-0.06   }$  & 4.58   $^{+0.18   }_{-0.18   }$  & 1.17   $^{+0.05   }_{-0.05   }$  & 0.34$^{+0.03   }_{-0.04   }$  & 1.38 & 0.97 &  86\\
MS 0016.9+1609 &    69 &   550 & 4.06  & 8.94   $^{+0.71   }_{-0.62   }$  & 9.78   $^{+1.09   }_{-0.90   }$  & 1.09   $^{+0.15   }_{-0.13   }$  & 0.29$^{+0.09   }_{-0.08   }$  & 0.91 & 0.88 &  83\\
MS 0451.6-0305 &    72 &   536 & 5.68  & 8.90   $^{+0.85   }_{-0.72   }$  & 10.43  $^{+1.59   }_{-1.26   }$  & 1.17   $^{+0.21   }_{-0.17   }$  & 0.37$^{+0.11   }_{-0.11   }$  & 1.00 & 0.93 &  60\\
MS 0735.6+7421 &    69 &   491 & 3.40  & 5.55   $^{+0.24   }_{-0.22   }$  & 6.34   $^{+0.57   }_{-0.50   }$  & 1.14   $^{+0.11   }_{-0.10   }$  & 0.35$^{+0.07   }_{-0.06   }$  & 1.05 & 1.05 &  62\\
MS 0839.8+2938 &    70 &   415 & 3.92  & 4.68   $^{+0.32   }_{-0.29   }$  & 5.05   $^{+0.82   }_{-0.65   }$  & 1.08   $^{+0.19   }_{-0.15   }$  & 0.46$^{+0.13   }_{-0.12   }$  & 0.90 & 0.87 &  60\\
MS 0906.5+1110 &    70 &   616 & 3.60  & 5.38   $^{+0.33   }_{-0.29   }$  & 6.76   $^{+0.92   }_{-0.77   }$  & 1.26   $^{+0.19   }_{-0.16   }$  & 0.27$^{+0.09   }_{-0.09   }$  & 1.21 & 1.08 &  75\\
MS 1006.0+1202 &    70 &   556 & 3.63  & 5.61   $^{+0.51   }_{-0.43   }$  & 7.48   $^{+1.66   }_{-1.22   }$  & 1.33   $^{+0.32   }_{-0.24   }$  & 0.24$^{+0.11   }_{-0.12   }$  & 1.30 & 1.34 &  75\\
MS 1008.1-1224 &    70 &   548 & 6.71  & 5.65   $^{+0.49   }_{-0.43   }$  & 9.01   $^{+1.95   }_{-1.38   }$  & 1.59   $^{+0.37   }_{-0.27   }$  & 0.26$^{+0.11   }_{-0.10   }$  & 1.21 & 0.98 &  78\\
MS 1054.5-0321 &    72 &   558 & 3.69  & 9.38   $^{+1.72   }_{-1.34   }$  & 9.91   $^{+2.66   }_{-1.77   }$  & 1.06   $^{+0.34   }_{-0.24   }$  & 0.13$^{+0.17   }_{-0.13   }$  & 1.02 & 1.03 &  41\\
MS 1455.0+2232 &    69 &   436 & 3.35  & 4.77   $^{+0.13   }_{-0.13   }$  & 5.37   $^{+0.36   }_{-0.22   }$  & 1.13   $^{+0.08   }_{-0.06   }$  & 0.44$^{+0.05   }_{-0.05   }$  & 1.29 & 1.10 &  90\\
MS 1621.5+2640 &    72 &   537 & 3.59  & 6.11   $^{+0.95   }_{-0.76   }$  & 6.22   $^{+1.56   }_{-1.10   }$  & 1.02   $^{+0.30   }_{-0.22   }$  & 0.40$^{+0.23   }_{-0.21   }$  & 1.02 & 1.21 &  68\\
MS 2053.7-0449 $\ddagger$ &    72 &   561 & 5.16  & 3.66   $^{+0.81   }_{-0.60   }$  & 4.07   $^{+1.23   }_{-0.83   }$  & 1.11   $^{+0.42   }_{-0.29   }$  & 0.39$^{+0.38   }_{-0.33   }$  & 0.97 & 1.07 &  58\\
MS 2137.3-2353 &    70 &   502 & 3.40  & 6.01   $^{+0.52   }_{-0.46   }$  & 7.48   $^{+1.68   }_{-1.09   }$  & 1.24   $^{+0.30   }_{-0.20   }$  & 0.45$^{+0.13   }_{-0.14   }$  & 1.12 & 1.25 &  55\\
PKS 0745-191 &    69 &   651 & 40.80 & 8.13   $^{+0.37   }_{-0.34   }$  & 9.68   $^{+0.83   }_{-0.72   }$  & 1.19   $^{+0.12   }_{-0.10   }$  & 0.38$^{+0.06   }_{-0.06   }$  & 1.02 & 0.98 &  89\\
RBS 0797 &    69 &   493 & 2.22  & 7.68   $^{+0.92   }_{-0.77   }$  & 9.05   $^{+1.80   }_{-1.33   }$  & 1.18   $^{+0.27   }_{-0.21   }$  & 0.32$^{+0.14   }_{-0.13   }$  & 1.07 & 1.06 &  89\\
RDCS 1252-29 &    71 &   276 & 6.06  & 4.25   $^{+1.82   }_{-1.14   }$  & 4.47   $^{+2.16   }_{-1.29   }$  & 1.05   $^{+0.68   }_{-0.41   }$  & 0.79$^{+1.01   }_{-0.62   }$  & 1.07 & 1.17 &  50\\
RX J0232.2-4420 &    69 &   568 & 2.53  & 7.83   $^{+0.77   }_{-0.68   }$  & 9.92   $^{+2.11   }_{-1.44   }$  & 1.27   $^{+0.30   }_{-0.21   }$  & 0.36$^{+0.12   }_{-0.13   }$  & 1.13 & 1.09 &  85\\
RX J0340-4542 &    72 &   412 & 1.63  & 3.16   $^{+0.38   }_{-0.35   }$  & 2.80   $^{+0.94   }_{-0.57   }$  & 0.89   $^{+0.32   }_{-0.21   }$  & 0.62$^{+0.31   }_{-0.25   }$  & 1.27 & 1.22 &  43\\
RX J0439+0520 &    72 &   474 & 10.02 & 4.60   $^{+0.64   }_{-0.59   }$  & 4.95   $^{+1.28   }_{-0.88   }$  & 1.08   $^{+0.32   }_{-0.24   }$  & 0.44$^{+0.29   }_{-0.24   }$  & 1.03 & 1.14 &  77\\
RX J0439.0+0715 $\ddagger$ &    70 &   532 & 11.16 & 5.63   $^{+0.36   }_{-0.32   }$  & 8.02   $^{+1.25   }_{-0.93   }$  & 1.42   $^{+0.24   }_{-0.18   }$  & 0.32$^{+0.10   }_{-0.08   }$  & 1.28 & 1.16 &  82\\
RX J0528.9-3927 &    70 &   640 & 2.36  & 7.89   $^{+0.96   }_{-0.76   }$  & 8.91   $^{+2.30   }_{-1.42   }$  & 1.13   $^{+0.32   }_{-0.21   }$  & 0.27$^{+0.14   }_{-0.14   }$  & 0.92 & 0.93 &  83\\
RX J0647.7+7015 $\ddagger$ &    69 &   512 & 5.18  & 11.28  $^{+1.85   }_{-1.45   }$  & 11.01  $^{+2.17   }_{-1.63   }$  & 0.98   $^{+0.25   }_{-0.19   }$  & 0.20$^{+0.17   }_{-0.17   }$  & 1.02 & 1.00 &  80\\
RX J0910+5422 $\ddagger$ &    73 &   246 & 2.07  & 4.53   $^{+3.02   }_{-1.70   }$  & 5.98   $^{+5.30   }_{-2.49   }$  & 1.32   $^{+1.46   }_{-0.74   }$  & 0.00$^{+0.73   }_{-0.00   }$  & 0.90 & 0.71 &  31\\
RX J1347.5-1145 $\ddagger$ &    68 &   607 & 4.89  & 14.62  $^{+0.97   }_{-0.79   }$  & 16.62  $^{+1.54   }_{-1.24   }$  & 1.14   $^{+0.13   }_{-0.10   }$  & 0.32$^{+0.08   }_{-0.07   }$  & 1.12 & 1.12 &  93\\
RX J1350+6007 &    71 &   334 & 1.77  & 4.48   $^{+2.32   }_{-1.49   }$  & 5.31   $^{+3.02   }_{-2.07   }$  & 1.19   $^{+0.91   }_{-0.61   }$  & 0.13$^{+1.23   }_{-0.13   }$  & 0.82 & 0.72 &  57\\
RX J1423.8+2404 $\ddagger$ &    73 &   441 & 2.65  & 6.64   $^{+0.38   }_{-0.34   }$  & 7.01   $^{+0.59   }_{-0.51   }$  & 1.06   $^{+0.11   }_{-0.09   }$  & 0.37$^{+0.07   }_{-0.07   }$  & 1.02 & 0.98 &  86\\
RX J1504.1-0248 &    70 &   628 & 6.27  & 8.00   $^{+0.27   }_{-0.24   }$  & 8.92   $^{+0.52   }_{-0.46   }$  & 1.11   $^{+0.08   }_{-0.07   }$  & 0.40$^{+0.04   }_{-0.05   }$  & 1.29 & 1.25 &  91\\
RX J1525+0958 &    74 &   416 & 2.96  & 3.74   $^{+0.63   }_{-0.45   }$  & 6.96   $^{+2.88   }_{-1.73   }$  & 1.86   $^{+0.83   }_{-0.51   }$  & 0.67$^{+0.36   }_{-0.29   }$  & 1.29 & 0.93 &  79\\
RX J1532.9+3021 $\ddagger$ &    70 &   458 & 2.21  & 6.03   $^{+0.42   }_{-0.38   }$  & 6.95   $^{+0.88   }_{-0.72   }$  & 1.15   $^{+0.17   }_{-0.14   }$  & 0.42$^{+0.11   }_{-0.10   }$  & 0.94 & 1.05 &  73\\
RX J1716.9+6708 &    71 &   486 & 3.71  & 5.71   $^{+1.47   }_{-1.06   }$  & 5.77   $^{+1.88   }_{-1.28   }$  & 1.01   $^{+0.42   }_{-0.29   }$  & 0.68$^{+0.42   }_{-0.35   }$  & 0.79 & 0.74 &  55\\
RX J1720.1+2638 &    69 &   510 & 4.02  & 6.37   $^{+0.28   }_{-0.26   }$  & 7.78   $^{+0.69   }_{-0.61   }$  & 1.22   $^{+0.12   }_{-0.11   }$  & 0.35$^{+0.07   }_{-0.06   }$  & 1.10 & 1.02 &  90\\
RX J1720.2+3536 $\ddagger$ &    71 &   455 & 3.35  & 7.21   $^{+0.53   }_{-0.46   }$  & 6.97   $^{+0.76   }_{-0.59   }$  & 0.97   $^{+0.13   }_{-0.10   }$  & 0.41$^{+0.10   }_{-0.10   }$  & 1.12 & 1.09 &  85\\
RX J2011.3-5725 &    73 &   416 & 4.76  & 3.94   $^{+0.45   }_{-0.37   }$  & 4.40   $^{+1.20   }_{-0.81   }$  & 1.12   $^{+0.33   }_{-0.23   }$  & 0.34$^{+0.21   }_{-0.18   }$  & 0.94 & 1.09 &  76\\
RX J2129.6+0005 &    70 &   690 & 4.30  & 5.91   $^{+0.54   }_{-0.47   }$  & 7.02   $^{+1.30   }_{-0.99   }$  & 1.19   $^{+0.25   }_{-0.19   }$  & 0.45$^{+0.15   }_{-0.15   }$  & 1.21 & 1.07 &  80\\
S0463 $\ddagger$ &    72 &   433 & 1.06  & 3.10   $^{+0.29   }_{-0.25   }$  & 3.10   $^{+0.66   }_{-0.53   }$  & 1.00   $^{+0.23   }_{-0.19   }$  & 0.24$^{+0.14   }_{-0.11   }$  & 1.10 & 1.07 &  47\\
V 1121.0+2327 &    74 &   444 & 1.30  & 3.60   $^{+0.62   }_{-0.46   }$  & 4.08   $^{+1.09   }_{-0.80   }$  & 1.13   $^{+0.36   }_{-0.27   }$  & 0.36$^{+0.29   }_{-0.24   }$  & 1.21 & 1.19 &  66\\
ZWCL 1215 &    70 &   392 & 1.76  & 6.64   $^{+0.40   }_{-0.35   }$  & 8.72   $^{+1.30   }_{-1.07   }$  & 1.31   $^{+0.21   }_{-0.18   }$  & 0.29$^{+0.09   }_{-0.09   }$  & 1.17 & 1.04 &  88\\
ZWCL 1358+6245 &    72 &   553 & 1.94  & 10.66  $^{+1.48   }_{-1.13   }$  & 10.19  $^{+4.83   }_{-2.24   }$  & 0.96   $^{+0.47   }_{-0.23   }$  & 0.47$^{+0.19   }_{-0.19   }$  & 1.08 & 1.04 &  55\\
ZWCL 1953 &    69 &   730 & 3.10  & 7.37   $^{+1.00   }_{-0.78   }$  & 10.44  $^{+3.25   }_{-2.20   }$  & 1.42   $^{+0.48   }_{-0.33   }$  & 0.19$^{+0.13   }_{-0.13   }$  & 0.84 & 0.78 &  74\\
ZWCL 3146 &    68 &   723 & 2.70  & 7.48   $^{+0.32   }_{-0.30   }$  & 8.61   $^{+0.66   }_{-0.58   }$  & 1.15   $^{+0.10   }_{-0.09   }$  & 0.31$^{+0.05   }_{-0.06   }$  & 1.03 & 0.98 &  86\\
ZWCL 5247 &    72 &   635 & 1.70  & 5.06   $^{+0.85   }_{-0.64   }$  & 5.91   $^{+2.09   }_{-1.30   }$  & 1.17   $^{+0.46   }_{-0.30   }$  & 0.22$^{+0.21   }_{-0.19   }$  & 0.83 & 0.72 &  74\\
ZWCL 7160 &    69 &   637 & 3.10  & 4.53   $^{+0.40   }_{-0.35   }$  & 5.16   $^{+1.01   }_{-0.77   }$  & 1.14   $^{+0.24   }_{-0.19   }$  & 0.40$^{+0.15   }_{-0.14   }$  & 0.94 & 0.92 &  80\\
ZWICKY 2701 &    69 &   445 & 0.83  & 5.21   $^{+0.34   }_{-0.30   }$  & 5.68   $^{+0.85   }_{-0.66   }$  & 1.09   $^{+0.18   }_{-0.14   }$  & 0.43$^{+0.13   }_{-0.11   }$  & 0.89 & 0.94 &  57\\
ZwCL 1332.8+5043 &    74 &   642 & 1.10  & 3.62   $^{+3.46   }_{-1.20   }$  & 3.84   $^{+5.93   }_{-1.48   }$  & 1.06   $^{+1.93   }_{-0.54   }$  & 0.76$^{+12.45  }_{-0.76   }$  & 0.24 & 0.29 &  48\\
ZwCl 0848.5+3341 &    73 &   518 & 1.12  & 6.83   $^{+2.18   }_{-1.33   }$  & 7.24   $^{+5.11   }_{-2.26   }$  & 1.06   $^{+0.82   }_{-0.39   }$  & 0.56$^{+0.54   }_{-0.45   }$  & 0.82 & 0.93 &  37\\
\enddata
\tablecomments{Note: \"77\" refers to 0.7-7.0 keV band and \"27\" refers to 2.0-7.0 keV band. (1) Cluster name, (2) size of excluded core region in kpc, (3) $R_{2500}$ in kpc, (4) absorbing Galactic neutral hydrogen column density, (5,6) best-fit {\textsc{MeKaL}} temperatures, (7) best-fit 77 {\textsc{MeKaL}} abundance, (8) $T_{0.7-7.0}$/$T_{2.0-7.0}$ also called $T_{HBR}$, (9,10) respective reduced $\chi^2$ statistics, and (11) percent of emission attributable to source. A ($\ddagger$) indicates a cluster which has multiple observations. Each observation has an independent spectrum extracted along with an associated WARF, WRMF, and normalized background spectrum. Each independent spectrum is then fit simultaneously with the same spectral model to produce the final.}
\end{deluxetable}


%%%%%%%%%%%%%%%%%%%%%%%%%%%%%%
% R5000 Spectral fit results %
%%%%%%%%%%%%%%%%%%%%%%%%%%%%%%

\clearpage
\begin{deluxetable}{lcccccccccc}
\tablewidth{0pt}
\tabletypesize{\scriptsize}
\tablecaption{Summary of Excised $R_{5000}$ Spectral Fits\label{tab:r5000specfits}}
\tablehead{\colhead{Cluster} & \colhead{$R_{CORE}$} & \colhead{$R_{5000}$ } & \colhead{$N_{HI}$} & \colhead{$T_{77}$} & \colhead{$T_{27}$} & \colhead{$T_{HBR}$} & \colhead{$Z_{77}$} & \colhead{$\chi^2_{red,77}$} & \colhead{$\chi^2_{red,27}$} & \colhead{\% Source}\\
\colhead{ } & \colhead{kpc} & \colhead{kpc} & \colhead{$10^{20}$ cm$^{-2}$} & \colhead{keV} & \colhead{keV} & \colhead{ } & \colhead{$Z_{\sun}$} & \colhead{ } & \colhead{ } & \colhead{ }\\
\colhead{{(1)}} & \colhead{{(2)}} & \colhead{{(3)}} & \colhead{{(4)}} & \colhead{{(5)}} & \colhead{{(6)}} & \colhead{{(7)}} & \colhead{{(8)}} & \colhead{{(9)}} & \colhead{{(10)}} & \colhead{{(11)}}
}
\startdata
1E0657 56 $\dagger$ &    69 &   486 & 6.53  & 11.81  $^{+0.29   }_{-0.27   }$  & 14.13  $^{+0.58   }_{-0.53   }$  & 1.20   $^{+0.06   }_{-0.05   }$  & 0.29$^{+0.03   }_{-0.03   }$  & 1.22 & 1.10 &  95\\
1RXS J2129.4-0741 $\dagger$ &    68 &   358 & 4.36  & 8.47   $^{+1.31   }_{-1.04   }$  & 8.57   $^{+1.73   }_{-1.27   }$  & 1.01   $^{+0.26   }_{-0.19   }$  & 0.51$^{+0.20   }_{-0.19   }$  & 1.16 & 1.27 &  87\\
2PIGG J0011.5-2850 &    70 &   387 & 2.18  & 5.25   $^{+0.29   }_{-0.27   }$  & 6.21   $^{+0.83   }_{-0.68   }$  & 1.18   $^{+0.17   }_{-0.14   }$  & 0.23$^{+0.09   }_{-0.08   }$  & 1.08 & 1.01 &  78\\
2PIGG J0311.8-2655 &    69 &   320 & 1.46  & 3.35   $^{+0.25   }_{-0.22   }$  & 3.67   $^{+0.71   }_{-0.54   }$  & 1.10   $^{+0.23   }_{-0.18   }$  & 0.33$^{+0.13   }_{-0.11   }$  & 1.03 & 1.10 &  51\\
2PIGG J2227.0-3041 &    70 &   267 & 1.11  & 2.81   $^{+0.16   }_{-0.15   }$  & 2.99   $^{+0.36   }_{-0.28   }$  & 1.06   $^{+0.14   }_{-0.11   }$  & 0.35$^{+0.11   }_{-0.08   }$  & 1.14 & 1.10 &  77\\
3C 220.1 &    70 &   308 & 1.91  & 7.81   $^{+7.50   }_{-3.00   }$  & 7.49   $^{+11.53  }_{-3.50   }$  & 0.96   $^{+0.77   }_{-0.38   }$  & 0.00$^{+0.56   }_{-0.00   }$  & 0.79 & 0.76 &  64\\
3C 28.0 &    70 &   297 & 5.71  & 5.18   $^{+0.28   }_{-0.27   }$  & 7.11   $^{+1.15   }_{-0.90   }$  & 1.37   $^{+0.23   }_{-0.19   }$  & 0.30$^{+0.09   }_{-0.07   }$  & 0.96 & 0.77 &  90\\
3C 295 &    69 &   329 & 1.35  & 5.35   $^{+0.48   }_{-0.41   }$  & 8.06   $^{+2.35   }_{-1.52   }$  & 1.51   $^{+0.46   }_{-0.31   }$  & 0.29$^{+0.12   }_{-0.10   }$  & 1.01 & 1.07 &  86\\
3C 388 &    70 &   297 & 6.11  & 3.27   $^{+0.24   }_{-0.21   }$  & 3.44   $^{+0.73   }_{-0.51   }$  & 1.05   $^{+0.24   }_{-0.17   }$  & 0.43$^{+0.16   }_{-0.13   }$  & 1.09 & 1.04 &  76\\
4C 55.16 &    70 &   300 & 4.00  & 4.88   $^{+0.16   }_{-0.16   }$  & 5.11   $^{+0.44   }_{-0.39   }$  & 1.05   $^{+0.10   }_{-0.09   }$  & 0.52$^{+0.07   }_{-0.07   }$  & 0.93 & 0.85 &  71\\
ABELL 0013 &    70 &   404 & 2.03  & 6.84   $^{+0.36   }_{-0.36   }$  & 9.29   $^{+1.37   }_{-1.08   }$  & 1.36   $^{+0.21   }_{-0.17   }$  & 0.55$^{+0.10   }_{-0.12   }$  & 1.14 & 1.10 &  56\\
ABELL 0068 &    69 &   460 & 4.60  & 9.72   $^{+1.82   }_{-1.36   }$  & 10.89  $^{+5.21   }_{-2.85   }$  & 1.12   $^{+0.58   }_{-0.33   }$  & 0.41$^{+0.24   }_{-0.23   }$  & 1.08 & 1.03 &  87\\
ABELL 0119 &    69 &   399 & 3.30  & 5.86   $^{+0.28   }_{-0.27   }$  & 6.20   $^{+0.74   }_{-0.59   }$  & 1.06   $^{+0.14   }_{-0.11   }$  & 0.44$^{+0.10   }_{-0.10   }$  & 0.98 & 0.89 &  75\\
ABELL 0168 $\dagger$ &    69 &   270 & 3.27  & 2.56   $^{+0.13   }_{-0.10   }$  & 3.37   $^{+0.48   }_{-0.41   }$  & 1.32   $^{+0.20   }_{-0.17   }$  & 0.32$^{+0.07   }_{-0.05   }$  & 1.03 & 0.97 &  44\\
ABELL 0209 $\dagger$ &    70 &   430 & 1.68  & 7.32   $^{+0.65   }_{-0.56   }$  & 10.05  $^{+2.33   }_{-1.58   }$  & 1.37   $^{+0.34   }_{-0.24   }$  & 0.21$^{+0.11   }_{-0.10   }$  & 1.07 & 1.15 &  88\\
ABELL 0267 $\dagger$ &    69 &   385 & 2.74  & 6.46   $^{+0.51   }_{-0.45   }$  & 7.46   $^{+1.22   }_{-0.91   }$  & 1.15   $^{+0.21   }_{-0.16   }$  & 0.37$^{+0.12   }_{-0.11   }$  & 1.18 & 1.29 &  88\\
ABELL 0370 &    71 &   365 & 3.37  & 8.74   $^{+0.98   }_{-0.83   }$  & 10.15  $^{+2.17   }_{-1.52   }$  & 1.16   $^{+0.28   }_{-0.21   }$  & 0.37$^{+0.14   }_{-0.13   }$  & 1.05 & 1.02 &  50\\
ABELL 0383 &    69 &   299 & 4.07  & 4.95   $^{+0.30   }_{-0.28   }$  & 5.92   $^{+1.05   }_{-0.85   }$  & 1.20   $^{+0.22   }_{-0.18   }$  & 0.43$^{+0.12   }_{-0.11   }$  & 1.12 & 1.10 &  75\\
ABELL 0399 &    70 &   386 & 8.33$^{+0.82   }_{-0.80   }$  & 7.93   $^{+0.38   }_{-0.35   }$  & 8.86   $^{+0.67   }_{-0.59   }$  & 1.12   $^{+0.10   }_{-0.09   }$  & 0.32$^{+0.06   }_{-0.05   }$  & 1.06 & 0.96 &  87\\
ABELL 0401 &    70 &   454 & 12.48 & 6.54   $^{+0.22   }_{-0.20   }$  & 9.37   $^{+0.91   }_{-0.74   }$  & 1.43   $^{+0.15   }_{-0.12   }$  & 0.29$^{+0.07   }_{-0.06   }$  & 1.53 & 1.10 &  85\\
ABELL 0478 &    70 &   422 & 30.90 & 7.27   $^{+0.26   }_{-0.25   }$  & 8.19   $^{+0.56   }_{-0.50   }$  & 1.13   $^{+0.09   }_{-0.08   }$  & 0.47$^{+0.06   }_{-0.06   }$  & 1.02 & 0.93 &  95\\
ABELL 0514 &    70 &   350 & 3.14  & 3.85   $^{+0.27   }_{-0.25   }$  & 4.92   $^{+1.02   }_{-0.76   }$  & 1.28   $^{+0.28   }_{-0.21   }$  & 0.29$^{+0.12   }_{-0.11   }$  & 1.00 & 1.01 &  58\\
ABELL 0520 &    69 &   408 & 1.14$^{+1.16   }_{-1.14   }$  & 9.15   $^{+0.73   }_{-0.63   }$  & 10.43  $^{+1.41   }_{-1.06   }$  & 1.14   $^{+0.18   }_{-0.14   }$  & 0.36$^{+0.07   }_{-0.07   }$  & 1.12 & 1.01 &  91\\
ABELL 0521 &    69 &   378 & 6.17  & 7.31   $^{+0.79   }_{-0.64   }$  & 9.01   $^{+3.73   }_{-1.87   }$  & 1.23   $^{+0.53   }_{-0.28   }$  & 0.48$^{+0.17   }_{-0.16   }$  & 1.11 & 0.95 &  55\\
ABELL 0586 &    70 &   449 & 4.71  & 6.43   $^{+0.55   }_{-0.49   }$  & 8.06   $^{+1.51   }_{-1.14   }$  & 1.25   $^{+0.26   }_{-0.20   }$  & 0.50$^{+0.15   }_{-0.15   }$  & 0.88 & 0.81 &  87\\
ABELL 0611 &    69 &   369 & 4.99  & 6.79   $^{+0.51   }_{-0.46   }$  & 6.88   $^{+1.23   }_{-0.95   }$  & 1.01   $^{+0.20   }_{-0.16   }$  & 0.32$^{+0.10   }_{-0.10   }$  & 1.04 & 1.07 &  67\\
ABELL 0644 &    69 &   411 & 6.31  & 7.81   $^{+0.20   }_{-0.19   }$  & 8.08   $^{+0.44   }_{-0.39   }$  & 1.03   $^{+0.06   }_{-0.06   }$  & 0.42$^{+0.05   }_{-0.04   }$  & 1.15 & 1.05 &  92\\
ABELL 0665 &    69 &   437 & 4.24  & 7.35   $^{+0.40   }_{-0.37   }$  & 10.43  $^{+1.76   }_{-1.31   }$  & 1.42   $^{+0.25   }_{-0.19   }$  & 0.29$^{+0.07   }_{-0.07   }$  & 1.07 & 0.94 &  91\\
ABELL 0697 &    70 &   433 & 3.34  & 9.78   $^{+0.99   }_{-0.85   }$  & 14.71  $^{+4.47   }_{-2.90   }$  & 1.50   $^{+0.48   }_{-0.32   }$  & 0.48$^{+0.13   }_{-0.13   }$  & 1.06 & 0.95 &  93\\
ABELL 0773 &    69 &   434 & 1.46  & 8.08   $^{+0.74   }_{-0.65   }$  & 11.24  $^{+2.84   }_{-1.94   }$  & 1.39   $^{+0.37   }_{-0.26   }$  & 0.37$^{+0.12   }_{-0.12   }$  & 1.03 & 0.96 &  89\\
ABELL 0907 &    70 &   345 & 5.69  & 5.60   $^{+0.19   }_{-0.18   }$  & 6.26   $^{+0.49   }_{-0.44   }$  & 1.12   $^{+0.10   }_{-0.09   }$  & 0.46$^{+0.06   }_{-0.06   }$  & 1.17 & 1.02 &  92\\
ABELL 0963 &    70 &   385 & 1.39  & 6.97   $^{+0.35   }_{-0.32   }$  & 7.65   $^{+1.00   }_{-0.82   }$  & 1.10   $^{+0.15   }_{-0.13   }$  & 0.29$^{+0.08   }_{-0.07   }$  & 1.13 & 1.12 &  74\\
ABELL 1063S &    70 &   458 & 1.77  & 11.94  $^{+0.91   }_{-0.80   }$  & 14.04  $^{+1.83   }_{-1.47   }$  & 1.18   $^{+0.18   }_{-0.15   }$  & 0.38$^{+0.10   }_{-0.09   }$  & 1.01 & 0.98 &  94\\
ABELL 1068 &    70 &   304 & 0.71  & 4.67   $^{+0.18   }_{-0.18   }$  & 5.49   $^{+0.71   }_{-0.58   }$  & 1.18   $^{+0.16   }_{-0.13   }$  & 0.37$^{+0.06   }_{-0.07   }$  & 0.92 & 0.91 &  77\\
ABELL 1201 &    69 &   401 & 1.85  & 5.74   $^{+0.44   }_{-0.40   }$  & 5.99   $^{+1.39   }_{-0.95   }$  & 1.04   $^{+0.26   }_{-0.18   }$  & 0.35$^{+0.13   }_{-0.11   }$  & 1.06 & 1.10 &  50\\
ABELL 1204 &    70 &   295 & 1.44  & 3.67   $^{+0.18   }_{-0.16   }$  & 4.72   $^{+0.75   }_{-0.57   }$  & 1.29   $^{+0.21   }_{-0.17   }$  & 0.32$^{+0.09   }_{-0.09   }$  & 1.11 & 0.92 &  92\\
ABELL 1361 &    70 &   316 & 2.18  & 5.14   $^{+1.00   }_{-0.74   }$  & 7.24   $^{+8.23   }_{-2.78   }$  & 1.41   $^{+1.62   }_{-0.58   }$  & 0.29$^{+0.31   }_{-0.27   }$  & 1.10 & 0.82 &  61\\
ABELL 1423 &    70 &   435 & 1.60  & 6.04   $^{+0.82   }_{-0.68   }$  & 7.93   $^{+4.09   }_{-2.20   }$  & 1.31   $^{+0.70   }_{-0.39   }$  & 0.33$^{+0.20   }_{-0.17   }$  & 0.95 & 0.91 &  84\\
ABELL 1651 &    70 &   421 & 2.02  & 6.30   $^{+0.32   }_{-0.28   }$  & 7.46   $^{+0.99   }_{-0.81   }$  & 1.18   $^{+0.17   }_{-0.14   }$  & 0.44$^{+0.09   }_{-0.09   }$  & 1.13 & 1.17 &  91\\
ABELL 1664 &    70 &   291 & 8.47  & 4.26   $^{+0.30   }_{-0.26   }$  & 4.91   $^{+1.05   }_{-0.80   }$  & 1.15   $^{+0.26   }_{-0.20   }$  & 0.31$^{+0.12   }_{-0.11   }$  & 1.07 & 1.08 &  70\\
ABELL 1689 $\dagger$ &    70 &   480 & 1.87  & 9.76   $^{+0.40   }_{-0.38   }$  & 12.97  $^{+1.25   }_{-1.05   }$  & 1.33   $^{+0.14   }_{-0.12   }$  & 0.35$^{+0.06   }_{-0.05   }$  & 1.14 & 1.04 &  94\\
ABELL 1758 &    70 &   406 & 1.09  & 9.66   $^{+0.75   }_{-0.64   }$  & 9.90   $^{+1.22   }_{-1.89   }$  & 1.02   $^{+0.15   }_{-0.21   }$  & 0.48$^{+0.11   }_{-0.11   }$  & 1.03 & 0.96 &  68\\
ABELL 1763 &    69 &   396 & 0.82  & 7.74   $^{+0.73   }_{-0.64   }$  & 12.56  $^{+6.70   }_{-3.12   }$  & 1.62   $^{+0.88   }_{-0.42   }$  & 0.22$^{+0.11   }_{-0.12   }$  & 1.16 & 1.02 &  89\\
ABELL 1795 &    69 &   449 & 1.22  & 6.05   $^{+0.15   }_{-0.15   }$  & 6.85   $^{+0.42   }_{-0.38   }$  & 1.13   $^{+0.07   }_{-0.07   }$  & 0.33$^{+0.04   }_{-0.05   }$  & 1.19 & 1.03 &  93\\
ABELL 1835 &    69 &   404 & 2.36  & 9.55   $^{+0.55   }_{-0.51   }$  & 11.99  $^{+1.96   }_{-1.44   }$  & 1.26   $^{+0.22   }_{-0.17   }$  & 0.35$^{+0.07   }_{-0.08   }$  & 0.91 & 0.88 &  91\\
ABELL 1914 &    70 &   494 & 0.97  & 9.73   $^{+0.58   }_{-0.51   }$  & 11.97  $^{+1.90   }_{-1.40   }$  & 1.23   $^{+0.21   }_{-0.16   }$  & 0.32$^{+0.08   }_{-0.07   }$  & 1.11 & 1.03 &  95\\
ABELL 1942 &    70 &   334 & 2.75  & 4.96   $^{+0.45   }_{-0.39   }$  & 5.94   $^{+2.24   }_{-0.99   }$  & 1.20   $^{+0.46   }_{-0.22   }$  & 0.37$^{+0.15   }_{-0.14   }$  & 1.04 & 0.87 &  77\\
ABELL 1995 &    70 &   260 & 1.44  & 8.50   $^{+0.83   }_{-0.71   }$  & 9.41   $^{+1.87   }_{-1.32   }$  & 1.11   $^{+0.25   }_{-0.18   }$  & 0.33$^{+0.12   }_{-0.12   }$  & 1.05 & 1.02 &  81\\
ABELL 2029 &    69 &   434 & 3.26  & 8.20   $^{+0.32   }_{-0.29   }$  & 9.90   $^{+0.90   }_{-0.73   }$  & 1.21   $^{+0.12   }_{-0.10   }$  & 0.40$^{+0.06   }_{-0.06   }$  & 1.07 & 1.03 &  94\\
ABELL 2034 &    69 &   420 & 1.58  & 7.35   $^{+0.26   }_{-0.24   }$  & 9.96   $^{+1.09   }_{-0.84   }$  & 1.36   $^{+0.16   }_{-0.12   }$  & 0.34$^{+0.05   }_{-0.05   }$  & 1.17 & 1.02 &  90\\
ABELL 2065 &    70 &   370 & 2.96  & 5.75   $^{+0.19   }_{-0.17   }$  & 6.39   $^{+0.46   }_{-0.41   }$  & 1.11   $^{+0.09   }_{-0.08   }$  & 0.28$^{+0.05   }_{-0.05   }$  & 1.11 & 1.01 &  89\\
ABELL 2069 &    69 &   441 & 1.97  & 6.33   $^{+0.36   }_{-0.32   }$  & 8.29   $^{+1.36   }_{-1.02   }$  & 1.31   $^{+0.23   }_{-0.17   }$  & 0.24$^{+0.08   }_{-0.08   }$  & 1.14 & 1.15 &  78\\
ABELL 2111 &    69 &   418 & 2.20  & 5.74   $^{+1.43   }_{-0.97   }$  & 7.18   $^{+6.73   }_{-2.52   }$  & 1.25   $^{+1.21   }_{-0.49   }$  & 0.16$^{+0.30   }_{-0.16   }$  & 1.06 & 0.97 &  74\\
ABELL 2125 &    69 &   262 & 2.75  & 3.09   $^{+0.37   }_{-0.31   }$  & 3.69   $^{+1.99   }_{-0.81   }$  & 1.19   $^{+0.66   }_{-0.29   }$  & 0.36$^{+0.25   }_{-0.20   }$  & 1.25 & 1.22 &  68\\
ABELL 2163 &    69 &   531 & 12.04 & 18.78  $^{+0.89   }_{-0.83   }$  & 19.49  $^{+2.03   }_{-1.86   }$  & 1.04   $^{+0.12   }_{-0.11   }$  & 0.09$^{+0.06   }_{-0.05   }$  & 1.33 & 1.25 &  93\\
ABELL 2204 $\dagger$ &    70 &   406 & 5.84  & 9.35   $^{+0.43   }_{-0.41   }$  & 10.18  $^{+0.95   }_{-0.77   }$  & 1.09   $^{+0.11   }_{-0.10   }$  & 0.37$^{+0.07   }_{-0.07   }$  & 0.95 & 0.97 &  86\\
ABELL 2218 &    70 &   394 & 3.12  & 7.37   $^{+0.40   }_{-0.37   }$  & 9.36   $^{+1.42   }_{-1.07   }$  & 1.27   $^{+0.20   }_{-0.16   }$  & 0.22$^{+0.07   }_{-0.06   }$  & 1.00 & 0.91 &  91\\
ABELL 2219 &    70 &   463 & 1.76  & 12.60  $^{+0.65   }_{-0.61   }$  & 12.54  $^{+1.52   }_{-1.21   }$  & 1.00   $^{+0.13   }_{-0.11   }$  & 0.31$^{+0.07   }_{-0.07   }$  & 1.02 & 0.98 &  81\\
ABELL 2255 &    70 &   404 & 2.53  & 6.37   $^{+0.24   }_{-0.23   }$  & 7.70   $^{+0.79   }_{-0.69   }$  & 1.21   $^{+0.13   }_{-0.12   }$  & 0.34$^{+0.06   }_{-0.07   }$  & 0.93 & 0.84 &  81\\
ABELL 2256 &    69 &   423 & 4.05  & 5.66   $^{+0.19   }_{-0.17   }$  & 7.30   $^{+0.69   }_{-0.63   }$  & 1.29   $^{+0.13   }_{-0.12   }$  & 0.31$^{+0.07   }_{-0.07   }$  & 1.61 & 1.44 &  79\\
ABELL 2259 &    70 &   339 & 3.70  & 5.07   $^{+0.46   }_{-0.40   }$  & 5.49   $^{+1.29   }_{-0.91   }$  & 1.08   $^{+0.27   }_{-0.20   }$  & 0.40$^{+0.16   }_{-0.14   }$  & 0.92 & 0.92 &  90\\
ABELL 2261 &    70 &   408 & 3.31  & 7.86   $^{+0.51   }_{-0.47   }$  & 9.84   $^{+1.94   }_{-1.30   }$  & 1.25   $^{+0.26   }_{-0.18   }$  & 0.40$^{+0.09   }_{-0.09   }$  & 0.98 & 0.95 &  94\\
ABELL 2294 &    69 &   404 & 6.10  & 10.49  $^{+1.75   }_{-1.30   }$  & 12.33  $^{+5.72   }_{-3.05   }$  & 1.18   $^{+0.58   }_{-0.33   }$  & 0.57$^{+0.25   }_{-0.24   }$  & 1.16 & 1.08 &  88\\
ABELL 2384 &    69 &   309 & 2.99  & 4.53   $^{+0.22   }_{-0.21   }$  & 6.78   $^{+1.13   }_{-0.89   }$  & 1.50   $^{+0.26   }_{-0.21   }$  & 0.15$^{+0.07   }_{-0.06   }$  & 0.99 & 0.88 &  86\\
ABELL 2390 &    69 &   447 & 6.71  & 10.85  $^{+0.34   }_{-0.31   }$  & 10.53  $^{+0.62   }_{-0.53   }$  & 0.97   $^{+0.06   }_{-0.06   }$  & 0.35$^{+0.05   }_{-0.04   }$  & 1.15 & 1.03 &  81\\
ABELL 2409 &    69 &   362 & 6.72  & 5.93   $^{+0.45   }_{-0.39   }$  & 5.87   $^{+0.95   }_{-0.76   }$  & 0.99   $^{+0.18   }_{-0.14   }$  & 0.35$^{+0.13   }_{-0.11   }$  & 1.05 & 0.76 &  92\\
ABELL 2537 &    68 &   351 & 4.26  & 8.83   $^{+0.87   }_{-0.74   }$  & 7.83   $^{+1.54   }_{-1.16   }$  & 0.89   $^{+0.20   }_{-0.15   }$  & 0.39$^{+0.14   }_{-0.14   }$  & 0.93 & 0.83 &  59\\
ABELL 2550 &    69 &   247 & 2.03  & 2.12   $^{+0.11   }_{-0.11   }$  & 2.56   $^{+0.69   }_{-0.49   }$  & 1.21   $^{+0.33   }_{-0.24   }$  & 0.36$^{+0.10   }_{-0.08   }$  & 1.34 & 1.14 &  47\\
ABELL 2554 &    70 &   398 & 2.04  & 5.35   $^{+0.45   }_{-0.40   }$  & 6.46   $^{+1.93   }_{-1.24   }$  & 1.21   $^{+0.37   }_{-0.25   }$  & 0.35$^{+0.15   }_{-0.13   }$  & 0.93 & 0.79 &  40\\
ABELL 2556 &    70 &   323 & 2.02  & 3.57   $^{+0.16   }_{-0.15   }$  & 4.07   $^{+0.56   }_{-0.46   }$  & 1.14   $^{+0.16   }_{-0.14   }$  & 0.36$^{+0.07   }_{-0.07   }$  & 0.99 & 0.95 &  58\\
ABELL 2631 &    70 &   446 & 3.74  & 7.18   $^{+1.18   }_{-0.94   }$  & 9.18   $^{+3.17   }_{-1.96   }$  & 1.28   $^{+0.49   }_{-0.32   }$  & 0.34$^{+0.20   }_{-0.19   }$  & 1.03 & 0.99 &  89\\
ABELL 2667 &    69 &   371 & 1.64  & 6.68   $^{+0.48   }_{-0.43   }$  & 7.35   $^{+1.27   }_{-1.05   }$  & 1.10   $^{+0.21   }_{-0.17   }$  & 0.41$^{+0.12   }_{-0.12   }$  & 1.05 & 0.95 &  84\\
ABELL 2670 &    69 &   319 & 2.88  & 3.96   $^{+0.13   }_{-0.13   }$  & 4.75   $^{+0.50   }_{-0.41   }$  & 1.20   $^{+0.13   }_{-0.11   }$  & 0.45$^{+0.08   }_{-0.07   }$  & 1.16 & 1.09 &  80\\
ABELL 2717 &    69 &   210 & 1.12  & 2.59   $^{+0.17   }_{-0.16   }$  & 3.18   $^{+0.59   }_{-0.44   }$  & 1.23   $^{+0.24   }_{-0.19   }$  & 0.53$^{+0.14   }_{-0.12   }$  & 0.90 & 0.95 &  67\\
ABELL 2744 &    68 &   439 & 1.82  & 9.82   $^{+0.89   }_{-0.77   }$  & 11.21  $^{+2.76   }_{-1.81   }$  & 1.14   $^{+0.30   }_{-0.20   }$  & 0.30$^{+0.12   }_{-0.12   }$  & 0.88 & 0.73 &  74\\
ABELL 3128 &    69 &   305 & 1.59  & 3.04   $^{+0.23   }_{-0.21   }$  & 3.48   $^{+0.73   }_{-0.54   }$  & 1.14   $^{+0.26   }_{-0.19   }$  & 0.33$^{+0.13   }_{-0.10   }$  & 1.05 & 1.13 &  64\\
ABELL 3158 $\dagger$ &    70 &   381 & 1.60  & 5.08   $^{+0.08   }_{-0.08   }$  & 6.26   $^{+0.26   }_{-0.24   }$  & 1.23   $^{+0.05   }_{-0.05   }$  & 0.40$^{+0.03   }_{-0.03   }$  & 1.15 & 0.97 &  89\\
ABELL 3164 &    69 &   306 & 2.55  & 2.40   $^{+0.65   }_{-0.48   }$  & 3.19   $^{+5.68   }_{-1.41   }$  & 1.33   $^{+2.39   }_{-0.64   }$  & 0.23$^{+0.32   }_{-0.19   }$  & 1.29 & 1.59 &  30\\
ABELL 3376 $\dagger$ &    69 &   327 & 5.21  & 4.44   $^{+0.14   }_{-0.13   }$  & 5.94   $^{+0.55   }_{-0.47   }$  & 1.34   $^{+0.13   }_{-0.11   }$  & 0.36$^{+0.06   }_{-0.06   }$  & 1.18 & 1.13 &  65\\
ABELL 3391 &    69 &   397 & 5.46  & 5.72   $^{+0.31   }_{-0.28   }$  & 6.44   $^{+0.80   }_{-0.66   }$  & 1.13   $^{+0.15   }_{-0.13   }$  & 0.11$^{+0.08   }_{-0.07   }$  & 1.00 & 0.97 &  67\\
ABELL 3921 &    70 &   378 & 3.07  & 5.69   $^{+0.25   }_{-0.24   }$  & 6.74   $^{+0.71   }_{-0.58   }$  & 1.18   $^{+0.14   }_{-0.11   }$  & 0.34$^{+0.08   }_{-0.07   }$  & 0.93 & 0.85 &  84\\
AC 114 &    69 &   373 & 1.44  & 7.75   $^{+0.56   }_{-0.50   }$  & 9.76   $^{+2.28   }_{-1.55   }$  & 1.26   $^{+0.31   }_{-0.22   }$  & 0.36$^{+0.11   }_{-0.10   }$  & 1.01 & 0.95 &  63\\
CL 0024+17 &    70 &   296 & 4.36  & 4.75   $^{+1.07   }_{-0.76   }$  & 7.14   $^{+5.42   }_{-2.83   }$  & 1.50   $^{+1.19   }_{-0.64   }$  & 0.58$^{+0.35   }_{-0.30   }$  & 1.07 & 0.97 &  44\\
CL 1221+4918 &    68 &   300 & 1.44  & 6.73   $^{+1.29   }_{-1.02   }$  & 7.60   $^{+4.33   }_{-2.01   }$  & 1.13   $^{+0.68   }_{-0.34   }$  & 0.32$^{+0.20   }_{-0.19   }$  & 0.92 & 0.69 &  73\\
CL J0030+2618 &    69 &   532 & 4.10  & 4.48   $^{+2.43   }_{-1.40   }$  & 3.77   $^{+9.73   }_{-1.96   }$  & 0.84   $^{+2.22   }_{-0.51   }$  & 0.00$^{+0.37   }_{-0.00   }$  & 1.01 & 0.85 &  51\\
CL J0152-1357 &    69 &   266 & 1.45  & 7.20   $^{+7.14   }_{-2.48   }$  & 6.07   $^{+6.16   }_{-2.51   }$  & 0.84   $^{+1.20   }_{-0.45   }$  & 0.00$^{+0.63   }_{-0.00   }$  & 2.97 & 3.26 &  49\\
CL J0542.8-4100 &    68 &   300 & 3.59  & 5.65   $^{+1.21   }_{-0.90   }$  & 5.93   $^{+3.52   }_{-1.76   }$  & 1.05   $^{+0.66   }_{-0.35   }$  & 0.25$^{+0.24   }_{-0.22   }$  & 0.67 & 0.58 &  72\\
CL J0848+4456 $\dagger$ &    68 &   215 & 2.53  & 3.73   $^{+1.47   }_{-0.85   }$  & 4.96   $^{+2.82   }_{-1.81   }$  & 1.33   $^{+0.92   }_{-0.57   }$  & 0.17$^{+0.98   }_{-0.17   }$  & 0.87 & 0.82 &  64\\
CL J1113.1-2615 &    69 &   295 & 5.51  & 4.74   $^{+1.52   }_{-0.98   }$  & 4.79   $^{+1.15   }_{-1.26   }$  & 1.01   $^{+0.40   }_{-0.34   }$  & 0.53$^{+0.52   }_{-0.37   }$  & 1.02 & 1.01 &  32\\
CL J1226.9+3332 $\dagger$ &    70 &   316 & 1.37  & 13.02  $^{+2.69   }_{-2.00   }$  & 12.33  $^{+2.78   }_{-2.13   }$  & 0.95   $^{+0.29   }_{-0.22   }$  & 0.18$^{+0.23   }_{-0.18   }$  & 0.75 & 0.80 &  91\\
CL J2302.8+0844 &    69 &   347 & 5.05  & 5.94   $^{+1.73   }_{-1.86   }$  & 6.58   $^{+8.08   }_{-2.67   }$  & 1.11   $^{+1.40   }_{-0.57   }$  & 0.10$^{+0.29   }_{-0.10   }$  & 0.94 & 1.01 &  56\\
DLS J0514-4904 &    69 &   344 & 2.52  & 4.94   $^{+0.61   }_{-0.55   }$  & 6.26   $^{+2.33   }_{-1.30   }$  & 1.27   $^{+0.50   }_{-0.30   }$  & 0.35$^{+0.27   }_{-0.23   }$  & 0.86 & 1.03 &  63\\
EXO 0422-086 &    70 &   294 & 6.22  & 3.41   $^{+0.14   }_{-0.13   }$  & 3.44   $^{+0.37   }_{-0.31   }$  & 1.01   $^{+0.12   }_{-0.10   }$  & 0.37$^{+0.08   }_{-0.08   }$  & 0.96 & 0.93 &  80\\
HERCULES A &    69 &   313 & 1.49$^{+2.01   }_{-1.49   }$  & 5.28   $^{+0.60   }_{-0.50   }$  & 4.50   $^{+0.88   }_{-0.65   }$  & 0.85   $^{+0.19   }_{-0.15   }$  & 0.42$^{+0.15   }_{-0.14   }$  & 0.98 & 0.98 &  70\\
LYNX E $\dagger$ &    68 &   185 & 2.53  & 4.14   $^{+5.16   }_{-1.38   }$  & 3.06   $^{+3.86   }_{-0.70   }$  & 0.74   $^{+1.31   }_{-0.30   }$  & 3.10$^{+33.81  }_{-2.93   }$  & 0.27 & 0.44 &  66\\
MACS J0011.7-1523 $\dagger$ &    69 &   320 & 2.08  & 6.73   $^{+0.55   }_{-0.47   }$  & 7.27   $^{+0.99   }_{-0.74   }$  & 1.08   $^{+0.17   }_{-0.13   }$  & 0.27$^{+0.10   }_{-0.09   }$  & 0.90 & 0.95 &  92\\
MACS J0025.4-1222 $\dagger$ &    69 &   321 & 2.72  & 6.65   $^{+1.07   }_{-0.85   }$  & 6.31   $^{+1.38   }_{-1.02   }$  & 0.95   $^{+0.26   }_{-0.20   }$  & 0.39$^{+0.22   }_{-0.19   }$  & 0.66 & 0.75 &  86\\
MACS J0035.4-2015 &    69 &   373 & 1.55  & 7.72   $^{+0.88   }_{-0.74   }$  & 9.39   $^{+1.91   }_{-1.35   }$  & 1.22   $^{+0.28   }_{-0.21   }$  & 0.39$^{+0.14   }_{-0.13   }$  & 1.02 & 1.05 &  94\\
MACS J0111.5+0855 &    69 &   294 & 4.18  & 4.12   $^{+1.60   }_{-1.04   }$  & 4.16   $^{+2.96   }_{-1.44   }$  & 1.01   $^{+0.82   }_{-0.43   }$  & 0.00$^{+0.43   }_{-0.00   }$  & 0.79 & 1.23 &  62\\
MACS J0152.5-2852 &    69 &   311 & 1.46  & 5.75   $^{+1.05   }_{-0.78   }$  & 7.70   $^{+3.21   }_{-1.89   }$  & 1.34   $^{+0.61   }_{-0.38   }$  & 0.28$^{+0.22   }_{-0.21   }$  & 0.84 & 0.58 &  90\\
MACS J0159.0-3412 &    69 &   388 & 1.54  & 10.99  $^{+5.87   }_{-2.95   }$  & 12.74  $^{+12.45  }_{-4.72   }$  & 1.16   $^{+1.29   }_{-0.53   }$  & 0.50$^{+0.52   }_{-0.50   }$  & 1.35 & 1.34 &  85\\
MACS J0159.8-0849 $\dagger$ &    69 &   414 & 2.01  & 9.36   $^{+0.77   }_{-0.67   }$  & 10.37  $^{+1.29   }_{-1.04   }$  & 1.11   $^{+0.17   }_{-0.14   }$  & 0.29$^{+0.09   }_{-0.09   }$  & 1.05 & 1.01 &  94\\
MACS J0242.5-2132 &    69 &   351 & 2.71  & 5.48   $^{+0.62   }_{-0.51   }$  & 5.99   $^{+2.04   }_{-1.19   }$  & 1.09   $^{+0.39   }_{-0.24   }$  & 0.32$^{+0.16   }_{-0.15   }$  & 1.08 & 1.06 &  92\\
MACS J0257.1-2325 $\dagger$ &    70 &   409 & 2.09  & 9.42   $^{+1.37   }_{-1.05   }$  & 10.76  $^{+2.05   }_{-1.69   }$  & 1.14   $^{+0.27   }_{-0.22   }$  & 0.14$^{+0.13   }_{-0.13   }$  & 1.03 & 1.13 &  90\\
MACS J0257.6-2209 &    68 &   382 & 2.02  & 8.09   $^{+1.10   }_{-0.88   }$  & 7.90   $^{+1.64   }_{-1.20   }$  & 0.98   $^{+0.24   }_{-0.18   }$  & 0.41$^{+0.19   }_{-0.18   }$  & 1.13 & 1.24 &  90\\
MACS J0308.9+2645 &    69 &   381 & 11.88 & 10.64  $^{+1.38   }_{-1.14   }$  & 11.12  $^{+2.23   }_{-1.68   }$  & 1.05   $^{+0.25   }_{-0.19   }$  & 0.37$^{+0.15   }_{-0.15   }$  & 0.96 & 0.97 &  92\\
MACS J0329.6-0211 $\dagger$ &    71 &   296 & 6.21  & 6.44   $^{+0.50   }_{-0.45   }$  & 7.55   $^{+0.88   }_{-0.73   }$  & 1.17   $^{+0.16   }_{-0.14   }$  & 0.40$^{+0.10   }_{-0.09   }$  & 1.12 & 1.16 &  91\\
MACS J0404.6+1109 &    71 &   334 & 14.96 & 6.90   $^{+2.01   }_{-1.29   }$  & 7.40   $^{+3.63   }_{-1.93   }$  & 1.07   $^{+0.61   }_{-0.34   }$  & 0.22$^{+0.27   }_{-0.22   }$  & 0.96 & 0.92 &  80\\
MACS J0417.5-1154 &    70 &   303 & 4.00  & 10.44  $^{+2.08   }_{-1.56   }$  & 14.46  $^{+5.92   }_{-3.41   }$  & 1.39   $^{+0.63   }_{-0.39   }$  & 0.41$^{+0.23   }_{-0.21   }$  & 1.10 & 1.17 &  96\\
MACS J0429.6-0253 &    68 &   349 & 5.70  & 5.96   $^{+0.72   }_{-0.60   }$  & 7.48   $^{+2.65   }_{-1.64   }$  & 1.26   $^{+0.47   }_{-0.30   }$  & 0.34$^{+0.15   }_{-0.14   }$  & 1.02 & 0.78 &  89\\
MACS J0451.9+0006 &    69 &   312 & 7.65  & 5.76   $^{+1.77   }_{-1.11   }$  & 6.68   $^{+4.50   }_{-1.94   }$  & 1.16   $^{+0.86   }_{-0.40   }$  & 0.47$^{+0.46   }_{-0.38   }$  & 1.03 & 1.33 &  89\\
MACS J0455.2+0657 &    68 &   326 & 10.45 & 6.99   $^{+2.27   }_{-1.44   }$  & 8.35   $^{+5.66   }_{-2.49   }$  & 1.19   $^{+0.90   }_{-0.43   }$  & 0.48$^{+0.35   }_{-0.31   }$  & 1.04 & 1.24 &  88\\
MACS J0520.7-1328 &    69 &   347 & 8.88  & 6.77   $^{+1.01   }_{-0.79   }$  & 9.41   $^{+3.38   }_{-1.91   }$  & 1.39   $^{+0.54   }_{-0.33   }$  & 0.33$^{+0.16   }_{-0.16   }$  & 1.22 & 1.33 &  91\\
MACS J0547.0-3904 &    69 &   256 & 4.08  & 3.70   $^{+0.44   }_{-0.37   }$  & 5.82   $^{+2.97   }_{-1.66   }$  & 1.57   $^{+0.82   }_{-0.48   }$  & 0.24$^{+0.21   }_{-0.17   }$  & 1.14 & 1.21 &  83\\
MACS J0553.4-3342 &    69 &   470 & 2.88  & 13.90  $^{+5.89   }_{-3.28   }$  & 14.59  $^{+11.16  }_{-4.72   }$  & 1.05   $^{+0.92   }_{-0.42   }$  & 0.38$^{+0.39   }_{-0.38   }$  & 1.22 & 1.10 &  91\\
MACS J0717.5+3745 $\dagger$ &    70 &   400 & 6.75  & 13.30  $^{+1.44   }_{-1.21   }$  & 12.82  $^{+1.70   }_{-1.39   }$  & 0.96   $^{+0.17   }_{-0.14   }$  & 0.32$^{+0.12   }_{-0.13   }$  & 0.91 & 0.87 &  91\\
MACS J0744.8+3927 $\dagger$ &    71 &   379 & 4.66  & 8.58   $^{+0.85   }_{-0.73   }$  & 9.32   $^{+1.20   }_{-0.96   }$  & 1.09   $^{+0.18   }_{-0.15   }$  & 0.30$^{+0.11   }_{-0.11   }$  & 1.14 & 1.19 &  89\\
MACS J0911.2+1746 $\dagger$ &    69 &   367 & 3.55  & 7.71   $^{+1.55   }_{-1.16   }$  & 7.88   $^{+2.11   }_{-1.44   }$  & 1.02   $^{+0.34   }_{-0.24   }$  & 0.22$^{+0.20   }_{-0.20   }$  & 0.77 & 0.77 &  85\\
MACS J0949+1708 &    69 &   395 & 3.17  & 8.94   $^{+1.57   }_{-1.20   }$  & 10.29  $^{+5.60   }_{-2.41   }$  & 1.15   $^{+0.66   }_{-0.31   }$  & 0.48$^{+0.23   }_{-0.22   }$  & 0.74 & 0.58 &  93\\
MACS J1006.9+3200 &    69 &   348 & 1.83  & 7.03   $^{+2.66   }_{-1.64   }$  & 6.53   $^{+4.61   }_{-2.11   }$  & 0.93   $^{+0.74   }_{-0.37   }$  & 0.18$^{+0.45   }_{-0.18   }$  & 1.64 & 1.53 &  81\\
MACS J1105.7-1014 &    70 &   341 & 4.58  & 7.73   $^{+2.85   }_{-1.73   }$  & 6.61   $^{+3.02   }_{-1.79   }$  & 0.86   $^{+0.50   }_{-0.30   }$  & 0.20$^{+0.32   }_{-0.20   }$  & 1.27 & 1.08 &  87\\
MACS J1108.8+0906 $\dagger$ &    71 &   331 & 2.52  & 6.80   $^{+1.21   }_{-0.93   }$  & 7.52   $^{+2.39   }_{-1.53   }$  & 1.11   $^{+0.40   }_{-0.27   }$  & 0.24$^{+0.20   }_{-0.19   }$  & 1.08 & 1.01 &  86\\
MACS J1115.2+5320 $\dagger$ &    70 &   373 & 0.98  & 9.58   $^{+1.85   }_{-1.37   }$  & 9.80   $^{+2.74   }_{-1.81   }$  & 1.02   $^{+0.35   }_{-0.24   }$  & 0.37$^{+0.22   }_{-0.21   }$  & 0.94 & 0.91 &  82\\
MACS J1115.8+0129 &    69 &   317 & 4.36  & 6.82   $^{+1.15   }_{-0.88   }$  & 9.39   $^{+4.77   }_{-2.84   }$  & 1.38   $^{+0.74   }_{-0.45   }$  & 0.07$^{+0.19   }_{-0.07   }$  & 0.94 & 0.85 &  77\\
MACS J1131.8-1955 &    70 &   408 & 4.49  & 8.64   $^{+1.32   }_{-1.03   }$  & 9.45   $^{+2.52   }_{-1.68   }$  & 1.09   $^{+0.34   }_{-0.23   }$  & 0.49$^{+0.19   }_{-0.19   }$  & 1.07 & 1.02 &  91\\
MACS J1149.5+2223 $\dagger$ &    69 &   359 & 2.32  & 7.72   $^{+0.94   }_{-0.79   }$  & 8.36   $^{+1.51   }_{-1.14   }$  & 1.08   $^{+0.24   }_{-0.18   }$  & 0.25$^{+0.12   }_{-0.13   }$  & 0.87 & 0.94 &  75\\
MACS J1206.2-0847 &    70 &   368 & 4.15  & 9.98   $^{+1.27   }_{-1.01   }$  & 11.93  $^{+2.56   }_{-1.88   }$  & 1.20   $^{+0.30   }_{-0.22   }$  & 0.32$^{+0.13   }_{-0.14   }$  & 1.02 & 1.15 &  95\\
MACS J1226.8+2153 &    70 &   333 & 1.82  & 4.86   $^{+1.58   }_{-1.08   }$  & 5.84   $^{+3.45   }_{-2.14   }$  & 1.20   $^{+0.81   }_{-0.51   }$  & 0.00$^{+0.28   }_{-0.00   }$  & 1.32 & 1.36 &  78\\
MACS J1311.0-0310 $\dagger$ &    69 &   300 & 2.18  & 5.73   $^{+0.46   }_{-0.40   }$  & 5.92   $^{+0.70   }_{-0.60   }$  & 1.03   $^{+0.15   }_{-0.13   }$  & 0.44$^{+0.12   }_{-0.12   }$  & 0.93 & 1.00 &  83\\
MACS J1319+7003 &    69 &   337 & 1.53  & 8.08   $^{+2.14   }_{-1.56   }$  & 10.12  $^{+5.50   }_{-2.78   }$  & 1.25   $^{+0.76   }_{-0.42   }$  & 0.10$^{+0.25   }_{-0.10   }$  & 1.00 & 1.07 &  82\\
MACS J1427.2+4407 &    70 &   331 & 1.41  & 8.61   $^{+4.04   }_{-2.23   }$  & 8.83   $^{+5.55   }_{-2.81   }$  & 1.03   $^{+0.80   }_{-0.42   }$  & 0.14$^{+0.36   }_{-0.14   }$  & 0.68 & 0.58 &  90\\
MACS J1427.6-2521 &    70 &   289 & 6.11  & 4.43   $^{+0.86   }_{-0.64   }$  & 5.54   $^{+3.81   }_{-1.77   }$  & 1.25   $^{+0.89   }_{-0.44   }$  & 0.21$^{+0.26   }_{-0.21   }$  & 1.08 & 1.15 &  79\\
MACS J1621.3+3810 $\dagger$ &    69 &   357 & 1.07  & 7.49   $^{+0.73   }_{-0.63   }$  & 7.75   $^{+1.12   }_{-0.89   }$  & 1.03   $^{+0.18   }_{-0.15   }$  & 0.35$^{+0.13   }_{-0.12   }$  & 0.98 & 0.92 &  82\\
MACS J1731.6+2252 &    70 &   353 & 6.48  & 8.19   $^{+1.88   }_{-1.31   }$  & 10.50  $^{+4.76   }_{-2.46   }$  & 1.28   $^{+0.65   }_{-0.36   }$  & 0.49$^{+0.27   }_{-0.25   }$  & 1.16 & 0.98 &  87\\
MACS J1824.3+4309 &    68 &   246 & 4.52  & 4.13   $^{+1.87   }_{-1.01   }$  & 4.76   $^{+10.39  }_{-1.86   }$  & 1.15   $^{+2.57   }_{-0.53   }$  & 0.76$^{+2.14   }_{-0.75   }$  & 0.52 & 0.13 &  78\\
MACS J1931.8-2634 &    70 &   377 & 9.13  & 6.85   $^{+0.73   }_{-0.61   }$  & 6.86   $^{+1.58   }_{-1.15   }$  & 1.00   $^{+0.25   }_{-0.19   }$  & 0.23$^{+0.12   }_{-0.11   }$  & 1.02 & 1.07 &  94\\
MACS J2046.0-3430 &    70 &   263 & 4.98  & 5.02   $^{+1.95   }_{-1.04   }$  & 6.23   $^{+2.57   }_{-2.30   }$  & 1.24   $^{+0.70   }_{-0.53   }$  & 0.23$^{+0.55   }_{-0.23   }$  & 1.10 & 1.14 &  89\\
MACS J2049.9-3217 &    69 &   371 & 5.99  & 7.88   $^{+1.22   }_{-0.98   }$  & 11.48  $^{+4.02   }_{-2.42   }$  & 1.46   $^{+0.56   }_{-0.36   }$  & 0.37$^{+0.18   }_{-0.16   }$  & 0.94 & 0.90 &  89\\
MACS J2211.7-0349 &    70 &   469 & 5.86  & 11.13  $^{+1.45   }_{-1.15   }$  & 13.77  $^{+3.49   }_{-2.40   }$  & 1.24   $^{+0.35   }_{-0.25   }$  & 0.18$^{+0.14   }_{-0.14   }$  & 1.33 & 1.34 &  93\\
MACS J2214.9-1359 $\dagger$ &    70 &   376 & 3.32  & 9.87   $^{+1.54   }_{-1.17   }$  & 9.97   $^{+2.17   }_{-1.50   }$  & 1.01   $^{+0.27   }_{-0.19   }$  & 0.31$^{+0.17   }_{-0.17   }$  & 1.03 & 1.01 &  92\\
MACS J2228+2036 &    70 &   387 & 4.52  & 7.79   $^{+1.14   }_{-0.90   }$  & 10.04  $^{+3.96   }_{-2.25   }$  & 1.29   $^{+0.54   }_{-0.32   }$  & 0.41$^{+0.18   }_{-0.17   }$  & 0.84 & 0.96 &  92\\
MACS J2229.7-2755 &    69 &   330 & 1.34  & 5.25   $^{+0.54   }_{-0.46   }$  & 6.07   $^{+1.76   }_{-1.18   }$  & 1.16   $^{+0.36   }_{-0.25   }$  & 0.59$^{+0.20   }_{-0.19   }$  & 0.98 & 1.02 &  91\\
MACS J2243.3-0935 &    70 &   389 & 4.31  & 5.15   $^{+0.65   }_{-0.54   }$  & 8.81   $^{+4.31   }_{-2.67   }$  & 1.71   $^{+0.86   }_{-0.55   }$  & 0.05$^{+0.17   }_{-0.05   }$  & 1.38 & 1.27 &  66\\
MACS J2245.0+2637 &    70 &   320 & 5.50  & 6.05   $^{+0.66   }_{-0.56   }$  & 7.05   $^{+1.31   }_{-1.08   }$  & 1.17   $^{+0.25   }_{-0.21   }$  & 0.64$^{+0.21   }_{-0.20   }$  & 0.78 & 0.95 &  92\\
MACS J2311+0338 &    69 &   247 & 5.23  & 7.60   $^{+1.60   }_{-1.19   }$  & 11.76  $^{+8.05   }_{-3.81   }$  & 1.55   $^{+1.11   }_{-0.56   }$  & 0.44$^{+0.24   }_{-0.22   }$  & 1.21 & 0.96 &  92\\
MKW3S &    69 &   239 & 3.05  & 3.93   $^{+0.06   }_{-0.06   }$  & 4.58   $^{+0.19   }_{-0.17   }$  & 1.17   $^{+0.05   }_{-0.05   }$  & 0.35$^{+0.02   }_{-0.03   }$  & 1.28 & 0.93 &  88\\
MS 0016.9+1609 &    69 &   388 & 4.06  & 9.11   $^{+0.79   }_{-0.68   }$  & 11.73  $^{+2.98   }_{-1.84   }$  & 1.29   $^{+0.35   }_{-0.22   }$  & 0.32$^{+0.10   }_{-0.09   }$  & 0.91 & 0.92 &  88\\
MS 0440.5+0204 &    70 &   477 & 9.10  & 5.99   $^{+0.91   }_{-0.73   }$  & 4.45   $^{+1.61   }_{-1.37   }$  & 0.74   $^{+0.29   }_{-0.25   }$  & 0.66$^{+0.32   }_{-0.29   }$  & 0.89 & 0.74 &  28\\
MS 0451.6-0305 &    69 &   363 & 5.68  & 9.25   $^{+0.89   }_{-0.77   }$  & 11.55  $^{+2.88   }_{-1.91   }$  & 1.25   $^{+0.33   }_{-0.23   }$  & 0.42$^{+0.12   }_{-0.11   }$  & 0.95 & 0.94 &  71\\
MS 0735.6+7421 &    69 &   347 & 3.40  & 5.54   $^{+0.24   }_{-0.23   }$  & 6.47   $^{+0.75   }_{-0.65   }$  & 1.17   $^{+0.14   }_{-0.13   }$  & 0.35$^{+0.07   }_{-0.07   }$  & 1.09 & 1.08 &  74\\
MS 0839.8+2938 &    70 &   293 & 3.92  & 4.63   $^{+0.30   }_{-0.28   }$  & 4.64   $^{+0.94   }_{-0.71   }$  & 1.00   $^{+0.21   }_{-0.16   }$  & 0.49$^{+0.13   }_{-0.13   }$  & 0.97 & 0.91 &  69\\
MS 0906.5+1110 &    70 &   436 & 3.60  & 5.56   $^{+0.34   }_{-0.31   }$  & 6.94   $^{+1.23   }_{-0.92   }$  & 1.25   $^{+0.23   }_{-0.18   }$  & 0.34$^{+0.10   }_{-0.10   }$  & 1.20 & 0.97 &  82\\
MS 1006.0+1202 &    69 &   393 & 3.63  & 5.79   $^{+0.54   }_{-0.46   }$  & 7.76   $^{+2.25   }_{-1.56   }$  & 1.34   $^{+0.41   }_{-0.29   }$  & 0.28$^{+0.12   }_{-0.12   }$  & 1.22 & 1.24 &  82\\
MS 1008.1-1224 &    69 &   388 & 6.71  & 5.76   $^{+0.56   }_{-0.47   }$  & 9.88   $^{+2.54   }_{-1.70   }$  & 1.72   $^{+0.47   }_{-0.33   }$  & 0.24$^{+0.11   }_{-0.11   }$  & 1.29 & 1.08 &  83\\
MS 1054.5-0321 &    69 &   378 & 3.69  & 9.75   $^{+1.69   }_{-1.28   }$  & 14.17  $^{+12.06  }_{-4.93   }$  & 1.45   $^{+1.26   }_{-0.54   }$  & 0.16$^{+0.16   }_{-0.16   }$  & 1.05 & 0.85 &  51\\
MS 1455.0+2232 &    70 &   308 & 3.35  & 4.81   $^{+0.13   }_{-0.13   }$  & 5.81   $^{+0.42   }_{-0.36   }$  & 1.21   $^{+0.09   }_{-0.08   }$  & 0.46$^{+0.05   }_{-0.05   }$  & 1.34 & 1.12 &  94\\
MS 1621.5+2640 &    69 &   363 & 3.59  & 5.72   $^{+0.90   }_{-0.72   }$  & 5.10   $^{+2.04   }_{-1.27   }$  & 0.89   $^{+0.38   }_{-0.25   }$  & 0.37$^{+0.23   }_{-0.21   }$  & 1.00 & 0.98 &  74\\
MS 2053.7-0449 $\dagger$ &    69 &   381 & 5.16  & 4.68   $^{+1.04   }_{-0.75   }$  & 5.37   $^{+1.73   }_{-1.19   }$  & 1.15   $^{+0.45   }_{-0.31   }$  & 0.26$^{+0.26   }_{-0.24   }$  & 0.99 & 0.94 &  65\\
MS 2137.3-2353 &    69 &   355 & 3.40  & 6.00   $^{+0.55   }_{-0.47   }$  & 7.56   $^{+2.79   }_{-1.46   }$  & 1.26   $^{+0.48   }_{-0.26   }$  & 0.35$^{+0.13   }_{-0.12   }$  & 1.08 & 1.28 &  69\\
MS J1157.3+5531 &    70 &   272 & 1.22  & 3.28   $^{+0.36   }_{-0.32   }$  & 10.57  $^{+6.42   }_{-3.33   }$  & 3.22   $^{+1.99   }_{-1.06   }$  & 0.76$^{+0.30   }_{-0.19   }$  & 1.22 & 1.15 &  37\\
NGC 6338 &    70 &   254 & 2.60  & 2.34   $^{+0.08   }_{-0.07   }$  & 3.03   $^{+0.30   }_{-0.24   }$  & 1.29   $^{+0.14   }_{-0.11   }$  & 0.24$^{+0.04   }_{-0.03   }$  & 1.06 & 1.00 &  54\\
PKS 0745-191 &    69 &   460 & 40.80 & 8.30   $^{+0.39   }_{-0.36   }$  & 9.69   $^{+0.84   }_{-0.73   }$  & 1.17   $^{+0.12   }_{-0.10   }$  & 0.42$^{+0.06   }_{-0.07   }$  & 1.01 & 0.97 &  93\\
RBS 0797 &    70 &   350 & 2.22  & 7.63   $^{+0.94   }_{-0.77   }$  & 8.62   $^{+2.60   }_{-1.69   }$  & 1.13   $^{+0.37   }_{-0.25   }$  & 0.25$^{+0.13   }_{-0.13   }$  & 1.06 & 0.83 &  93\\
RDCS 1252-29 &    68 &   188 & 6.06  & 4.63   $^{+2.39   }_{-1.41   }$  & 4.94   $^{+9.84   }_{-2.82   }$  & 1.07   $^{+2.20   }_{-0.69   }$  & 1.14$^{+2.11   }_{-0.83   }$  & 1.36 & 0.28 &  60\\
RX J0232.2-4420 &    69 &   402 & 2.53  & 7.92   $^{+0.85   }_{-0.74   }$  & 10.54  $^{+2.53   }_{-1.74   }$  & 1.33   $^{+0.35   }_{-0.25   }$  & 0.38$^{+0.13   }_{-0.13   }$  & 1.05 & 0.98 &  91\\
RX J0340-4542 &    69 &   279 & 1.63  & 3.10   $^{+0.43   }_{-0.38   }$  & 2.75   $^{+1.15   }_{-0.67   }$  & 0.89   $^{+0.39   }_{-0.24   }$  & 0.63$^{+0.39   }_{-0.28   }$  & 1.22 & 1.30 &  48\\
RX J0439+0520 &    69 &   322 & 10.02 & 4.67   $^{+0.58   }_{-0.47   }$  & 5.37   $^{+2.03   }_{-1.24   }$  & 1.15   $^{+0.46   }_{-0.29   }$  & 0.36$^{+0.22   }_{-0.20   }$  & 0.91 & 0.81 &  85\\
RX J0439.0+0715 $\dagger$ &    69 &   376 & 11.16 & 5.65   $^{+0.38   }_{-0.34   }$  & 8.21   $^{+1.29   }_{-0.96   }$  & 1.45   $^{+0.25   }_{-0.19   }$  & 0.34$^{+0.09   }_{-0.09   }$  & 1.32 & 1.14 &  87\\
RX J0528.9-3927 &    69 &   452 & 2.36  & 7.96   $^{+1.01   }_{-0.81   }$  & 9.84   $^{+2.92   }_{-1.81   }$  & 1.24   $^{+0.40   }_{-0.26   }$  & 0.26$^{+0.14   }_{-0.15   }$  & 0.96 & 1.04 &  88\\
RX J0647.7+7015 $\dagger$ &    69 &   362 & 5.18  & 11.46  $^{+2.05   }_{-1.58   }$  & 11.18  $^{+2.46   }_{-1.77   }$  & 0.98   $^{+0.28   }_{-0.20   }$  & 0.24$^{+0.18   }_{-0.20   }$  & 1.00 & 0.92 &  88\\
RX J0819.6+6336 &    70 &   308 & 4.11  & 3.92   $^{+0.46   }_{-0.40   }$  & 3.24   $^{+1.26   }_{-0.66   }$  & 0.83   $^{+0.34   }_{-0.19   }$  & 0.16$^{+0.17   }_{-0.14   }$  & 1.00 & 1.00 &  50\\
RX J0910+5422 $\dagger$ &    70 &   165 & 2.07  & 4.08   $^{+3.11   }_{-1.34   }$  & 5.00   $^{+5.09   }_{-2.03   }$  & 1.23   $^{+1.56   }_{-0.64   }$  & 0.43$^{+1.89   }_{-0.43   }$  & 0.64 & 0.56 &  42\\
RX J1347.5-1145 $\dagger$ &    68 &   431 & 4.89  & 15.12  $^{+1.03   }_{-0.86   }$  & 17.32  $^{+1.73   }_{-1.40   }$  & 1.15   $^{+0.14   }_{-0.11   }$  & 0.33$^{+0.07   }_{-0.08   }$  & 1.12 & 1.11 &  96\\
RX J1350+6007 &    68 &   227 & 1.77  & 4.22   $^{+3.13   }_{-1.53   }$  & 3.29   $^{+10.52  }_{-1.93   }$  & 0.78   $^{+2.56   }_{-0.54   }$  & 0.63$^{+5.75   }_{-0.63   }$  & 1.00 & 0.14 &  66\\
RX J1423.8+2404 $\dagger$ &    70 &   301 & 2.65  & 6.90   $^{+0.39   }_{-0.37   }$  & 7.19   $^{+0.59   }_{-0.52   }$  & 1.04   $^{+0.10   }_{-0.09   }$  & 0.38$^{+0.07   }_{-0.08   }$  & 0.94 & 0.90 &  90\\
RX J1504.1-0248 &    69 &   445 & 6.27  & 8.02   $^{+0.26   }_{-0.25   }$  & 8.52   $^{+0.58   }_{-0.50   }$  & 1.06   $^{+0.08   }_{-0.07   }$  & 0.39$^{+0.04   }_{-0.05   }$  & 1.25 & 1.17 &  95\\
RX J1532.9+3021 $\dagger$ &    69 &   323 & 2.21  & 6.06   $^{+0.43   }_{-0.39   }$  & 7.20   $^{+0.94   }_{-0.77   }$  & 1.19   $^{+0.18   }_{-0.15   }$  & 0.46$^{+0.10   }_{-0.11   }$  & 0.92 & 1.02 &  83\\
RX J1716.9+6708 &    68 &   328 & 3.71  & 6.51   $^{+1.79   }_{-1.24   }$  & 6.21   $^{+4.03   }_{-2.26   }$  & 0.95   $^{+0.67   }_{-0.39   }$  & 0.56$^{+0.39   }_{-0.32   }$  & 0.84 & 0.92 &  63\\
RX J1720.1+2638 &    70 &   361 & 4.02  & 6.33   $^{+0.29   }_{-0.25   }$  & 7.71   $^{+0.84   }_{-0.72   }$  & 1.22   $^{+0.14   }_{-0.12   }$  & 0.37$^{+0.07   }_{-0.07   }$  & 1.04 & 0.96 &  94\\
RX J1720.2+3536 $\dagger$ &    70 &   320 & 3.35  & 7.34   $^{+0.59   }_{-0.50   }$  & 7.40   $^{+0.86   }_{-0.71   }$  & 1.01   $^{+0.14   }_{-0.12   }$  & 0.43$^{+0.11   }_{-0.11   }$  & 1.03 & 0.94 &  91\\
RX J2011.3-5725 &    70 &   283 & 4.76  & 4.10   $^{+0.47   }_{-0.39   }$  & 3.93   $^{+0.98   }_{-0.70   }$  & 0.96   $^{+0.26   }_{-0.19   }$  & 0.41$^{+0.24   }_{-0.20   }$  & 0.95 & 1.08 &  84\\
RX J2129.6+0005 &    70 &   488 & 4.30  & 6.01   $^{+0.55   }_{-0.46   }$  & 7.19   $^{+1.68   }_{-1.21   }$  & 1.20   $^{+0.30   }_{-0.22   }$  & 0.51$^{+0.16   }_{-0.15   }$  & 1.29 & 1.34 &  87\\
S0463 $\dagger$ &    69 &   294 & 1.06  & 3.26   $^{+0.33   }_{-0.38   }$  & 3.92   $^{+1.16   }_{-0.94   }$  & 1.20   $^{+0.38   }_{-0.32   }$  & 0.23$^{+0.18   }_{-0.15   }$  & 1.08 & 1.08 &  54\\
TRIANG AUSTR &    70 &   516 & 13.27 & 9.37   $^{+0.32   }_{-0.30   }$  & 14.68  $^{+1.18   }_{-1.08   }$  & 1.57   $^{+0.14   }_{-0.13   }$  & 0.09$^{+0.04   }_{-0.04   }$  & 0.01 & 1.89 &  88\\
V 1121.0+2327 &    71 &   302 & 1.30  & 4.17   $^{+0.78   }_{-0.60   }$  & 4.70   $^{+3.00   }_{-1.17   }$  & 1.13   $^{+0.75   }_{-0.32   }$  & 0.46$^{+0.36   }_{-0.28   }$  & 1.09 & 0.87 &  74\\
ZWCL 1215 &    69 &   277 & 1.76  & 6.63   $^{+0.45   }_{-0.40   }$  & 7.64   $^{+1.34   }_{-1.00   }$  & 1.15   $^{+0.22   }_{-0.17   }$  & 0.37$^{+0.11   }_{-0.11   }$  & 1.10 & 1.04 &  91\\
ZWCL 1358+6245 &    69 &   375 & 1.94  & 9.70   $^{+1.16   }_{-0.94   }$  & 9.04   $^{+2.09   }_{-1.46   }$  & 0.93   $^{+0.24   }_{-0.18   }$  & 0.57$^{+0.19   }_{-0.19   }$  & 1.03 & 0.90 &  65\\
ZWCL 1953 &    69 &   517 & 3.10  & 8.28   $^{+1.25   }_{-0.96   }$  & 15.12  $^{+11.45  }_{-5.45   }$  & 1.83   $^{+1.41   }_{-0.69   }$  & 0.20$^{+0.14   }_{-0.15   }$  & 0.87 & 0.74 &  82\\
ZWCL 3146 &    70 &   512 & 2.70  & 7.46   $^{+0.32   }_{-0.30   }$  & 8.99   $^{+0.94   }_{-0.78   }$  & 1.21   $^{+0.14   }_{-0.12   }$  & 0.31$^{+0.06   }_{-0.05   }$  & 1.06 & 0.97 &  91\\
ZWCL 5247 &    69 &   430 & 1.70  & 4.89   $^{+0.86   }_{-0.65   }$  & 4.39   $^{+2.30   }_{-1.21   }$  & 0.90   $^{+0.50   }_{-0.27   }$  & 0.37$^{+0.30   }_{-0.25   }$  & 1.09 & 0.93 &  78\\
ZWCL 7160 &    70 &   451 & 3.10  & 4.64   $^{+0.41   }_{-0.37   }$  & 6.82   $^{+2.26   }_{-1.44   }$  & 1.47   $^{+0.50   }_{-0.33   }$  & 0.36$^{+0.14   }_{-0.14   }$  & 0.94 & 0.76 &  87\\
ZWICKY 2701 &    69 &   315 & 0.83  & 5.08   $^{+0.32   }_{-0.30   }$  & 4.96   $^{+0.87   }_{-0.69   }$  & 0.98   $^{+0.18   }_{-0.15   }$  & 0.45$^{+0.13   }_{-0.11   }$  & 0.95 & 0.76 &  70\\
ZwCL 1332.8+5043 &    71 &   434 & 1.10  & 3.82   $^{+3.34   }_{-1.42   }$  & 2.86   $^{+3.96   }_{-1.21   }$  & 0.75   $^{+1.23   }_{-0.42   }$  & 0.16$^{+4.75   }_{-0.16   }$  & 0.71 & 0.95 &  60\\
ZwCl 0848.5+3341 &    70 &   350 & 1.12  & 6.54   $^{+2.04   }_{-1.27   }$  & 6.41   $^{+3.79   }_{-1.88   }$  & 0.98   $^{+0.66   }_{-0.34   }$  & 0.59$^{+0.59   }_{-0.48   }$  & 0.89 & 1.01 &  47
\enddata
\tablecomments{Note: '77' refers to 0.7-7.0 keV band, '27' refers to 2.0-7.0 keV band. (1) Cluster name, (2) excluded core region in kpc, (3) $R_{5000}$ in kpc, (3) absorbing, Galactic neutral hydrogen column density, (4,5) best-fit MeKaL temperatures, (6) best-fit 77 MeKaL abundance, (7) $T_{0.7-7.0}$/$T_{2.0-7.0}$ also called $T_{HBR}$, (8,9) reduced $\chi^2$ statistics, (10) percent of emission attributable to source. $\dagger$ indicates cluster with multiple independent, simultaneously fit spectra.}
\end{deluxetable}


%%%%%%%%%%%%%%%%%%%%
% End the document %
%%%%%%%%%%%%%%%%%%%%

\end{document}
