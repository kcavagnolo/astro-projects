\documentclass[12pt]{cv}
\usepackage[colorlinks=true,linkcolor=blue,urlcolor=blue]{hyperref}
\usepackage[T1]{fontenc}
\usepackage{common,subfig,epsfig,colortbl,graphics,graphicx,wrapfig,amssymb,common,mathptmx,multicol}
\pagestyle{empty}
\parindent 0pt
\parskip
\baselineskip
\setlength{\topmargin}{-0.30in}
\setlength{\oddsidemargin}{-0.30in}
\setlength{\evensidemargin}{-0.30in}
\setlength{\headheight}{0in}
\setlength{\headsep}{0.25in}
\setlength{\topskip}{0.25in}
\setlength{\textwidth}{6.9in}
\setlength{\textheight}{9.25in}
\pagestyle{myheadings}
\begin{document}

\begin{center}
{\large \textbf{Kenneth W. Cavagnolo\\Ph.D. Dissertation Summary}}\\
\rule{17.35cm}{2pt}
\normalsize
\end{center}

Presented in the dissertation is an analysis of the X-ray emission
from the intracluster medium (ICM) in clusters of galaxies observed
with the \cxo. The dynamic state for a sample of clusters is
investigated via ICM temperature inhomogeneity, and ICM entropy is
used to evaluate the thermodynamics of cluster cores. The key results
presented in the dissertation are that ICM temperature inhomogeneity
correlates well with the process of cluster relaxation, galaxy
clusters have isentropic cores whose distribution is bimodal, and that
the processes of active galactic nucleus (AGN) feedback and star
formation are highly sensitive to the entropy state of a cluster's
core region. Also provided is a descriptive outline for {\it{CORP}} --
the robust, extensible suite of X-ray data reduction and analysis
tools written to complete the dissertation.

{\bfseries{ICM Temperature Inhomogeneity}}

To more accurately weigh galaxy clusters, how secondary dynamical
processes (\eg\ mergers and AGN feedback) alter cluster observables
must first be quantified if cluster temperature or luminosity are to
serve as accurate mass proxies. It has been demonstrated that spatial
cluster substructure correlates well with dynamical state, and that
the most relaxed clusters have the smallest deviations from mean
mass-observable relations (\eg\ Ventimiglia et al. 2008). But spatial
analysis is at the mercy of perspective. If equally robust
aspect-independent measures of dynamical state could be found, then
quantifying deviation from mean mass-scaling relations would be
improved and the uncertainty of inferred cluster masses could be
further reduced. The Cavagnolo dissertation confronts this difficulty
via temperature inhomogeneity.

If the hot ICM is nearly isothermal in the projected region of
interest, the X-ray temperature inferred from a broadband (0.7-7.0
keV) spectrum should be identical to the X-ray temperature inferred
from a hard-band (2.0-7.0 keV) spectrum. However, if unresolved cool
lumps of gas are contributing soft X-ray emission, the temperature of
a best-fit single-component thermal model will be cooler for the
broadband spectrum than for the hard-band spectrum. Using this
difference as a diagnostic, the ratio of best-fitting hard-band and
broadband temperatures may indicate the presence of cooler gas even
when the X-ray spectrum itself may not have sufficient signal-to-noise
ratio to resolve multiple temperature components (Mathiesen \& Evrard
2001).

Building on the Mathiesen \& Evrard (2001) simulation results, the
dissertation investigates the band dependence of the inferred X-ray
temperature of the ICM for 192 well-observed galaxy clusters selected
from the \chandra\ Data Archive. X-ray spectra from core-excised
annular regions of fixed fractions of the virial radius, $R_{2500}$
and $R_{5000}$, are extracted for each cluster in the archival
sample. A comparison is made of the X-ray temperatures inferred from
single-temperature fits when the energy range of the fit is 0.7-7.0
keV (broad) and when the energy range is 2.0/(1+$z$)-7.0 keV
(hard). On average, the hard-band temperature is found to be
significantly higher than the broadband temperature, and the ratio of
the temperatures is quantified as $T_{HBR} = T_{2.0-7.0}/T_{0.7-7.0}$,
shown in Figure \ref{fig:thbr}. On further exploration, it is found
that the temperature ratio $T_{HBR}$ is enhanced preferentially for
clusters which are known merging systems. In addition, cool-core
clusters tend to have best-fit hard-band temperatures that are in
closer agreement with their best-fit broadband temperatures, shown
using symbols in Figure \ref{fig:thbr}. Presuming cool cores and
mergers are good indicators of dynamical state, the dissertation
concludes that $T_{HBR}$ is a useful metric for further assessing the
process of cluster relaxation. The work associated with this part of
the dissertation is published in Cavagnolo et al. (2008a).

{\bfseries{ICM Entropy}}
\markright{K.W. Cavagnolo, Summary}

ICM temperature and density alone primarily reflect the shape and
depth of the cluster dark matter potential, but it is the specific
entropy of a gas parcel which governs the density at a given pressure
(Voit et al. 2002). In addition, the ICM is convectively stable when,
without dramatic perturbation, the lowest entropy gas is near the core
and high entropy gas has buoyantly risen to large radii. ICM entropy
can also only be changed by addition or subtraction of heat, thus the
entropy of the ICM reflects most of the cluster thermal
history. Therefore, properties of the ICM can be viewed as a
manifestation of the dark matter potential and cluster thermal history
- which is encoded in the entropy structure (\eg\ Voit et
al. 2002). ICM Entropy is therefore a useful quantity for studying the
effects of feedback on the cluster environment and investigating the
breakdown of cluster self-similarity.

The dissertation studies feedback using radial entropy profiles of the
ICM for a collection of 239 clusters taken from the \chandra Data
Archive, presented in Figure \ref{fig:splots}. It is found that most
ICM entropy profiles are well-fit by a model which is a power-law at
large radii and approaches a constant entropy value at small radii:
$K(r) = \kna + \khun (r/100 \kpc)^{\alpha}$, where \kna\ quantifies
the typical excess of core entropy above the best fitting power-law
found at larger radii and \khun\ is the entropy normalization at 100
kpc. Discussion is presented in relation to theoretical models
(\eg\ Voit \& Donahue 2005) explaining why non-zero \kna\ values are
consistent with the process of energy injection from AGN
feedback. Further, it is shown that the \kna\ distributions of both
the full archival sample and the flux-limited, unbiased primary
{\it{HIFLUGCS}} sample of Reiprich (2001) are bimodal with a distinct
gap centered at $\kna \approx 40 \ent$ and population peaks at $\kna
\sim 15 \ent$ and $\kna \sim 150 \ent$ (Figure \ref{fig:k0hist}). It
is suggested that the bimodal distribution may result from the effects
of ICM thermal conduction and cluster-cluster mergers. The results
from this work are presented in Cavagnolo et al. (2008b).

Also of interest is how cluster core entropy state is associated with
AGN feedback and star formation in the galaxy which resides at the
center of a cluster. As an extension of the radial entropy analysis,
the dissertation delves into exploring the relationship between some
expected by-products of ICM cooling -- \eg\ gaseous instabilities,
star formation, and AGN activity -- and the \kna\ values of
clusters. To determine the activity level of feedback in cluster
cores, the readily available observables \halpha\ and radio emission
are selected as tracers.

Utilizing the results of the archival study of intracluster entropy,
the dissertation goes on to show that \halpha\ and radio emission from
central cluster galaxies are much more pronounced when the cluster's
core gas entropy is $\la 30 \ent$. The prevalence of \halpha\ emission
below this threshold indicates that it marks a dichotomy between
clusters that can harbor multiphase gas and star formation in their
cores and those that cannot. The fact that strong central radio
emission also appears below this boundary suggests that feedback from
an AGN turns on when the ICM starts to condense, strengthening the
case for AGN feedback as the mechanism that limits star formation in
the Universe's most luminous galaxies. The results of this work are
presented in Cavagnolo et al. (2009). The dissertation results also
suggest that the sharp entropy threshold for the formation of thermal
instabilities in the ICM and initiation of processes such as star
formation and AGN activity arises from thermal conduction. A
discussion of this topic is presented in Voit et al. (2008).

\markright{K.W. Cavagnolo, Summary}

{\textbf{References}}\\
Cavagnolo, K.W., Donahue, M., Voit, G.M., \& Sun, M. 2008a, ApJ, 682, 821\\
Cavagnolo, K.W., Donahue, M., Voit, G.M., \& Sun, M. 2008b, ApJS, 182, 12\\
Cavagnolo, K.W., Donahue, M., Voit, G.M., \& Sun, M. 2009, ApJ, 683, L107\\
Reiprich, T.H., 2001, Ph.D., Max-Planck-Institut f{\"u}r extraterrestrische Physik\\
Ventimiglia, D.A., Voit, G.M., Donahue, M., \& Ameglio, S., 2008, ApJ, 685, 118\\
Voit, G.M., Bryan, G.L., Balogh, M.L., \& Bower, R.G., 2002, ApJ, 576, 601\\
Voit, G.M \& Donahue, M., ApJ, 634, 955\\
Voit, G.M., Cavagnolo, K.W., Donahue, M., Rafferty, D.A., McNamara, B.R., Nulsen, P.E.J., 2008, ApJ, 681, L5

\clearpage
\markright{K.W. Cavagnolo, Summary}
\begin{figure}[t]
    \begin{minipage}[t]{0.5\linewidth}
        \centering
        \includegraphics*[width=\textwidth, trim=17mm 3mm 10mm 11mm, clip]{thbr.eps}
        \caption{\footnotesize $T_{HBR}$ vs. $T_{0.7-7.0}$. The dashed
          line is the line of equivalence. Symbols and color coding
          are based on two criteria: 1) presence of a cool core (CC)
          and 2) value of $T_{HBR}$. Black stars are clusters with a
          CC and $T_{HBR}$ significantly greater than 1.1. Green
          upright-triangles are NCC clusters with $T_{HBR}$
          significantly greater than 1.1. Blue down-facing triangles
          are CC clusters and red squares are NCC clusters. It is
          found that most, if not all, of the clusters with $T_{HBR}
          \gtrsim 1.1$ are merger systems.}
        \label{fig:thbr}
    \end{minipage}
    \hspace{0.1in}
    \begin{minipage}[t]{0.5\linewidth}
        \centering
        \includegraphics*[width=\textwidth, trim=28mm 7mm 30mm 17mm, clip]{splots_allt.eps}
        \caption{\footnotesize Composite plot of entropy profiles for
          archival sample. Profiles are color-coded based on average
          cluster temperature; units of the color bar are keV. The
          solid line is the pure-cooling model of Voit et al. (2002),
          the dashed line is the mean profile for clusters with $\kna
          \le 50 \ent$, and the dashed-dotted line is the mean profile
          for clusters with $\kna > 50 \ent$.}
        \label{fig:splots}
    \end{minipage}
    \hspace{0.1in}
    \begin{minipage}[t]{0.5\linewidth}
        \centering
        \includegraphics*[width=\textwidth, trim=32mm 8mm 32mm 18mm, clip]{k0hist.eps}
        \caption{\footnotesize Histogram of best-fit \kna\ for all the
          clusters in the archival study. Bin widths are 0.15 in log
          space. The distinct bimodality in \kna is bracketed by the
          vertical dashed lines.}
        \label{fig:k0hist}
    \end{minipage}
    \hspace{0.1cm}
    \begin{minipage}[t]{0.5\linewidth}
        \centering
        \includegraphics*[width=\columnwidth, trim=28mm 7mm 40mm 17mm, clip]{ha.eps}
        \caption{\footnotesize Central entropy
          vs. \halpha\ luminosity. Orange circles represent
          \halpha\ detections, black circles are non-detection upper
          limits, and blue squares with inset red stars or orange
          circles are peculiar clusters which do not adhere to the
          observed trend. The vertical dashed line marks $\kna = 30
          \ent$. Note the presence of a sharp \halpha\ detection
          dichotomy beginning at $\kna \la 30 \ent$.}
        \label{fig:ha}
    \end{minipage}
\end{figure}
 
\end{document}
