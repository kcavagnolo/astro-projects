\documentclass[12pt]{plan}
\usepackage{psfig}
\usepackage{macros_desai}
\setlength{\topmargin}{-0.25in}
\setlength{\oddsidemargin}{-0.1in}
\setlength{\evensidemargin}{0in}
\setlength{\textwidth}{6.7in}
\setlength{\headheight}{0in}
\setlength{\headsep}{0in}
\setlength{\topskip}{0.55in}
\setlength{\textheight}{9.25in}
\pagestyle{myheadings}
\markright{\hspace*{\fill}{\it Eight Years of Science with Chandra Abstract}\hspace{10mm}}

\begin{document}
\begin{center}
\vspace{1.5mm}
{\bf The Entropy-Feedback Connection and Quantifying Cluster Virialization}\\
Kenneth Cavagnolo, Megan Donahue, Mark Voit, Ming Sun \textit{(Michigan State Univ.)}\\
\vspace{1.5mm}
\end{center}

Understanding the entropy of intracluster gas is the key to
understanding 1) the feedback mechanisms active within clusters and 2)
the role of the cluster environment on galaxy formation. Our presented
work focuses on examining feedback mechanisms together with the
breaking of self-similar relations expected in cluster and galaxy
formation models with star formation and AGN. In this poster, we
present and describe radial profiles of the entropy distribution in
cluster gas. We also examine a metric proposed to quantify the degree
of cluster virialization which may in turn reduce scatter in scaling
relations, thus increasing clusters utility in cosmological studies.

We have assembled a library of entropy profiles for a sample of
100 clusters in the \textit{Chandra} Data Archive (CDA) covering broad
mass and morphological ranges, together with the radio and optical
properties of the central galaxy. We will discuss the interconnection
of central entropy with radio luminosity and H$\alpha$ emission. We
will describe the distribution of central entropy levels in our sample
and briefly discuss what can be learned about the range of central
heating mechanisms and the timescale of feedback mechanisms from this
distribution.

We will also present recently completed work for which we explore the
band-dependence of the inferred X-ray temperature of the ICM for 179
clusters selected from the \textit{Chandra} archive. We compare the X-ray temperatures
inferred for single-temperature fits of global spectra (with the
central R=70 kpc excluded and an outer radius of R$_{2500}$) when the
energy range of the fit is 0.7-7.0 keV (full) and when the energy range is
2.0/(1+z)-7.0 keV (hard). We find, on average, the hard-band
temperature is significantly higher than the full-band
temperature. Upon further exploration, we find the ratio T$_{HFR}$ =
T$_{2.0-7.0}$/T$_{0.7-7.0}$ is enhanced preferentially for clusters
which are known merger systems and for clusters which do not have
detectable cool cores. Annular regions surrounding cool core clusters
tend to have best-fit hard-band temperatures that are statistically
consistent with their best-fit full-band temperatures. 

\end{document}
