\documentclass[12pt]{plan}
\usepackage{psfig}
\usepackage{macros_desai}
\setlength{\topmargin}{-0.25in}
\setlength{\oddsidemargin}{-0.1in}
\setlength{\evensidemargin}{0in}
\setlength{\textwidth}{6.7in}
\setlength{\headheight}{0in}
\setlength{\headsep}{0in}
\setlength{\topskip}{0.55in}
\setlength{\textheight}{9.25in}
\pagestyle{myheadings}
\markright{\hspace*{\fill}{\it Eight Years of Science with Chandra Abstract}\hspace{10mm}}

\begin{document}
\begin{center}
\vspace{1.5mm}
{\bf The Entropy-Feedback Connection and Quantifying Cluster Virialization}\\
Kenneth Cavagnolo, Megan Donahue, Mark Voit, Ming Sun \textit{(Michigan State Univ.)}\\
\vspace{1.5mm}
\end{center}

Entropy has been shown to be of vital importance in 1) understanding the
feedback mechanisms active within clusters and 2) the role of the cluster
environment on galaxy formation. Our presented work focuses on tying
together feedback mechanisms with the breaking of self-similar relations
expected in cluster and galaxy formation models. We also examine a metric
to quantify the degree of cluster virialization which may in turn
reduce scatter in scaling relations, thus increasing clusters utility
in cosmological studies.

We have assembled a library of entropy profiles for $> 80$ clusters in
the \textit{Chandra} Data Archive (CDA) covering a broad mass and
morphological range. We will be presenting these profiles and
discussing the interconnection of central entropy with radio
luminosity and H$\alpha$ emission. We will describe the distribution of
central entropy for our sample and briefly discuss what can be learned
about the timescale of feedback mechanisms from this distribution.

We will also present recently completed work for which we explore the
band-dependence of the inferred X-ray temperature of the ICM for 179
clusters selected from the CDA. We compare the X-ray temperatures
inferred for single-temperature fits of global spectra when the energy
range of the fit is 0.7-7.0 keV (full) and when the energy range is
2.0/(1+z)-7.0 keV (hard). We find, on average, the hard-band
temperature is significantly higher than the full-band
temperature. Upon further exploration, we find the ratio T$_{HFR}$ =
T$_{2.0-7.0}$/T$_{0.7-7.0}$ is enhanced preferentially for clusters
which are known merger systems and for clusters which are
isothermal. Cool core clusters tend to have best-fit hard-band
temperatures that are statistically consistent with their best fit
full-band temperatures.

\end{document}
