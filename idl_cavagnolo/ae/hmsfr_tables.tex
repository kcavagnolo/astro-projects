% $Id: hmsfr_tables.tex,v 1.2 2008-10-15 19:37:03 cavagnolo Exp $
\documentclass[preprint]{aastex}



\begin{document}
Prototype tables for HMSFR point sources:


%\begin{deluxetable}{ccrccccc}
%\centering 
%%\rotate
%\tabletypesize{\tiny} \tablewidth{0pt}
%\tablecolumns{8}
%
%
%\tablecaption{ Log of {\em Chandra} Observations 
% \label{tbl:obslog}}
%
%\tablehead{
%\colhead{Target} & 
%\colhead{Obs ID} & 
%\colhead{Start Time} & 
%\colhead{Exposure Time} & 
%\multicolumn{2}{c}{Aimpoint} & 
%\colhead{Roll Angle} & 
%\colhead{Mode\tablenotemark{a}} \\
%\cline{5-6}
%
%\colhead{} & 
%\colhead{} & 
%\colhead{(UT)} & 
%\colhead{(s)} & 
%\colhead{$\alpha_{\rm J2000}$} & 
%\colhead{$\delta_{\rm J2000}$} & 
%\colhead{(deg)} & 
%\colhead{} 
%}
%
%\startdata
%Field 1 & 4395 & 2003 Jan 08 20:58 & 39440 & 18 20 30.39 & +04 55 42.0 & 311 & VF \\
%
%\enddata
%
%\tablenotetext{a}{ The observing mode:  (F) Faint, (VF) Very Faint.}
%
%\tablecomments{  Exposure times are the net usable times after various
%filtering steps are applied in the data reduction process. The
%aimpoints and roll angles are obtained from the satellite aspect
%solution before astrometric correction is applied.  Units of right
%ascension are hours, minutes, and seconds; units of declination are
%degrees, arcminutes, and arcseconds.}
%
%\end{deluxetable}
%

%-----------------------------------------------------------------------------
%\clearpage

TEMPLATE src_properties
\begin{deluxetable}{rcrrrrrrrrrrrccrr}
\centering \rotate \tabletypesize{\tiny} \tablewidth{0pt}
\tablecolumns{17}

\tablecaption{Primary {\em Chandra} Catalog:  Basic Source Properties \label{tbl:src_properties_main}}

\tablehead{
\multicolumn{2}{c}{Source} &
\multicolumn{4}{c}{Position} &
\multicolumn{5}{c}{Extraction} &
\multicolumn{6}{c}{Characteristics} \\
                                
\multicolumn{2}{c}{\hrulefill} &  
\multicolumn{4}{c}{\hrulefill} &
\multicolumn{5}{c}{\hrulefill} &
\multicolumn{6}{c}{\hrulefill} \\

\colhead{Seq. No.} & \colhead{CXOU J} &
\colhead{$\alpha$ (J2000.0)} & \colhead{$\delta$ (J2000.0)} & \colhead{Error} & \colhead{$\theta$} &
\colhead{$C_{t,net}$} & \colhead{$\sigma_{t,net}$} & \colhead{$B_{t}$} & \colhead{$C_{h,net}$} & \colhead{PSF Frac.} &   
\colhead{Signif.} & \colhead{$\log P_B$} & \colhead{Anom.} & \colhead{Var.} &\colhead{Eff. Exp.} & \colhead{$E_{median}$}  \\



\colhead{} & \colhead{} &
\colhead{(deg)} & \colhead{(deg)} & \colhead{(arcsec)} & \colhead{(arcmin)} &
\colhead{(counts)} & \colhead{(counts)} & \colhead{(counts)} & \colhead{(counts)} & \colhead{} &
\colhead{} & \colhead{} & \colhead{} & \colhead{} & \colhead{(ks)} & \colhead{(keV)}
 \\


\colhead{(1)} & \colhead{(2)} &
\colhead{(3)} & \colhead{(4)} & \colhead{(5)} & \colhead{(6)} &
\colhead{(7)} & \colhead{(8)} & \colhead{(9)} & \colhead{(10)} & \colhead{(11)} &
\colhead{(12)} & \colhead{(13)} & \colhead{(14)} & \colhead{(15)} & \colhead{(16)} & \colhead{(17)}  }


\startdata
(%"%4d & %s &
%10.6f & %10.6f & %4.1f & %4.1f &
%7.1f & %5.1f & %5.1f & %7.1f & %4.2f &
%5.1f & %s & %s & %s & %6.1f & %3.1f \\\\")

\enddata

\tablecomments{Table~\ref{tbl:src_properties_main} is available in its entirety in the electronic edition of the \apjs.  
Interesting sources mentioned in the text are shown here for guidance regarding its form and content. 
\\Col.\ (1): X-ray catalog sequence number, sorted by RA.
\\Col.\ (2): IAU designation.
\\Cols.\ (3) and (4): Right ascension and declination for epoch (J2000.0).
\\Col.\ (5): Estimated standard deviation of the random component of the position error, $\sqrt{\sigma_x^2 + \sigma_y^2}$.  The single-axis position errors, $\sigma_x$ and $\sigma_y$, are estimated from the single-axis standard deviations of the PSF inside the extraction region and the number of counts extracted.
\\Col.\ (6): Off-axis angle.
\\Cols.\ (7) and (8): Net counts extracted in the total energy band (0.5--8~keV); average of the upper and lower $1\sigma$ errors on col.\ (7).
\\Col.\ (9): Background counts expected in the source extraction region (total band).
\\Col.\ (10): Net counts extracted in the hard energy band (2--8~keV).
\\Col.\ (11): Fraction of the PSF (at 1.497 keV) enclosed within the extraction region. A reduced PSF fraction (significantly below 90\%) may indicate that the source is in a crowded region. 
\\Col.\ (12): Photometric significance computed as net counts divided by the upper error on net counts. 
\\Col.\ (13): Logarithmic probability that extracted counts (total band) are solely from background.  Some sources have $P_B$ values above the 1\% threshold that defines the catalog because local background estimates can rise during the final extraction iteration after sources are removed from the catalog.
\\Col.\ (14):  Source anomalies: (g) fractional time that source was on a detector (FRACEXPO from {\em mkarf}) is $<$0.9.
%; (e) source on field edge; (p) source piled up; (s) source on readout streak.
\\Col.\ (15): Variability characterization based on K-S statistic (total band): (a) no evidence for variability ($0.05<P_{KS}$); (b) possibly variable ($0.005<P_{KS}<0.05$); (c) definitely variable ($P_{KS}<0.005$).  No value is reported for sources with fewer than four counts or for sources in chip gaps or on field edges.
\\Col.\ (16): Effective exposure time: approximate time the source would have to be observed on-axis (no telescope vignetting) on a nominal region of the detector (no dithering over insensitive regions of the detector) to obtain the reported number of counts. 
\\Col.\ (17): Background-corrected median photon energy (total band).}

\end{deluxetable}
END

%-----------------------------------------------------------------------------
%\clearpage

TEMPLATE thermal_spectroscopy
\begin{deluxetable}{rcrrlllrrrrrc}
\centering \rotate \tabletypesize{\scriptsize} \tablewidth{0pt}
\tablecolumns{13}

\tablecaption{X-ray Spectroscopy for Photometrically Selected Sources:  Thermal Plasma Fits
\label{tbl:thermal_spectroscopy}}


\tablehead{
\multicolumn{4}{c}{Source\tablenotemark{a}} &
\multicolumn{3}{c}{Spectral Fit\tablenotemark{b}} &
\multicolumn{5}{c}{X-ray Luminosities\tablenotemark{c}} &
\colhead{Notes\tablenotemark{d}} \\ 

\multicolumn{4}{c}{\hrulefill} &
\multicolumn{3}{c}{\hrulefill} &
\multicolumn{5}{c}{\hrulefill} \\

\colhead{Seq. No.} & \colhead{CXOU J} & \colhead{$C_{t,net}$} & \colhead{Signif.} &
\colhead{$\log N_H$} & \colhead{$kT$} & \colhead{$\log EM$} &  
\colhead{$\log L_s$} & \colhead{$\log L_h$} & \colhead{$\log L_{h,c}$} & \colhead{$\log L_t$} & \colhead{$\log L_{t,c}$} &
\colhead{}  \\


\colhead{} & \colhead{} & \colhead{} & \colhead{} &
\colhead{(cm$^{-2}$)} & \colhead{(keV)} & \colhead{(cm$^{-3}$)} & 
\multicolumn{5}{c}{(ergs s$^{-1}$)} &
\colhead{} \\


\colhead{(1)} & \colhead{(2)} & \colhead{(3)} & \colhead{(4)} &
\colhead{(5)} & \colhead{(6)} & \colhead{(7)} &
\colhead{(8)} & \colhead{(9)} & \colhead{(10)} &\colhead{(11)} & \colhead{(12)} &
\colhead{(13)}
}

% NOTE:  the \phantom calls in this table are used to make the numbers line
% up properly.

% You can NOT simply put pairs of '$' characters around each field in the format string below to get math mode for all the columns because, unfortunately, the AASTEX /nodata command fails in math mode.
% The table generator will substitute (via a 'sed' process) /nodata for each NaN printed by the format string below.
% We COULD carefully craft these format strings to produce '$/nodata$' in such cases and then have the table generator's sed look for that exact string.

\startdata
(%"%4d & %s & %7.1f & %5.1f &
%s & %s & %s &
%5.2f & %5.2f & %5.2f & %5.2f & %5.2f &
%s \\\\")

\enddata

\tablecomments{Table~\ref{tbl:thermal_spectroscopy} is available in its entirety in the electronic edition of the \apjs.  
Interesting sources mentioned in the text are shown here for guidance regarding its form and content. }

\tablenotetext{a}{ 
For convenience, cols.\ (1)--(4) reproduce the source identification, net counts, and photometric significance data from Table~\ref{tbl:src_properties_main}.
}

\tablenotetext{b}{
All fits used the ``wabs(apec)'' model in XSPEC and assumed 0.3$Z_{\odot}$ abundances \citep{Imanishi01,Feigelson02}.  
Cols.\ (5) and (6) present the best-fit values for the extinction column density and plasma temperature parameters.
Col.\ (7) presents the emission measure derived from the model spectrum, assuming a distance of XXXX.xxxx~kpc. 
Quantities marked with an asterisk (*) were frozen in the fit.  
Uncertainties represent 90\% confidence intervals.
More significant digits are used for uncertainties $<$0.1 in order to avoid large rounding errors; for consistency, the same number of significant digits is used for both lower and upper uncertainties.
Uncertainties are missing when XSPEC was unable to compute them or when their values were so large that the parameter is effectively unconstrained.  
Fits lacking uncertainties, fits with large uncertainties, and fits with frozen parameters should be viewed merely as splines to the data to obtain rough estimates of luminosities; the listed parameter values are unreliable.  
}

\tablenotetext{c}{ X-ray luminosities derived from the model spectrum are presented in cols.\ (8)--(12): (s) soft band (0.5--2 keV); (h) hard band (2--8 keV); (t) total band (0.5--8 keV).  
Absorption-corrected luminosities are subscripted with a $c$.
Cols. (8) and (12) are omitted when $\log N_H > 22.5$~cm$^{-2}$ since the soft band emission is essentially unmeasurable.  
}

\tablenotetext{d}{``2T'' means a two-temperature model was used; 
the second temperature is shown in parentheses.   
 } 



\end{deluxetable}
END


%-----------------------------------------------------------------------------
%\clearpage

TEMPLATE powerlaw_spectroscopy
\begin{deluxetable}{rcrrlllrrrrrc}
\centering \rotate \tabletypesize{\scriptsize} \tablewidth{0pt}
\tablecolumns{13}

\tablecaption{X-ray Spectroscopy for Photometrically Selected Sources:  Power Law Fits
\label{tbl:powerlaw_spectroscopy}}


\tablehead{
\multicolumn{4}{c}{Source\tablenotemark{a}} &
\multicolumn{3}{c}{Spectral Fit\tablenotemark{b}} &
\multicolumn{5}{c}{X-ray Luminosities\tablenotemark{c}} &
\colhead{Notes\tablenotemark{d}} \\ 

\multicolumn{4}{c}{\hrulefill} &
\multicolumn{3}{c}{\hrulefill} &
\multicolumn{5}{c}{\hrulefill} \\

\colhead{Seq. No.} & \colhead{CXOU J} & \colhead{$C_{t,net}$} & \colhead{Signif.} &
\colhead{$\log N_H$} & \colhead{$\Gamma$} & \colhead{$\log N_{\Gamma}$} &  
\colhead{$\log L_s$} & \colhead{$\log L_h$} & \colhead{$\log L_{h,c}$} & \colhead{$\log L_t$} & \colhead{$\log L_{t,c}$} &
\colhead{}  \\


\colhead{} & \colhead{} & \colhead{} & \colhead{} &
\colhead{(cm$^{-2}$)} & \colhead{} & \colhead{} & 
\multicolumn{5}{c}{(ergs s$^{-1}$)} &
\colhead{} \\


\colhead{(1)} & \colhead{(2)} & \colhead{(3)} & \colhead{(4)} &
\colhead{(5)} & \colhead{(6)} & \colhead{(7)} & 
\colhead{(8)} & \colhead{(9)} &\colhead{(10)} & \colhead{(11)} & \colhead{(12)} &
\colhead{(13)}
}

% NOTE:  the \phantom calls in this table are used to make the numbers line
% up properly.

% You can NOT simply put pairs of '$' characters around each field in the format string below to get math mode for all the columns because, unfortunately, the AASTEX /nodata command fails in math mode.
% The table generator will substitute (via a 'sed' process) /nodata for each NaN printed by the format string below.
% We COULD carefully craft these format strings to produce '$/nodata$' in such cases and then have the table generator's sed look for that exact string.

\startdata
(%"%4d & %s & %7.1f & %5.1f &
%s & %s & %s &
%6.2f & %6.2f & %6.2f & %6.2f & %6.2f &
%s \\\\")

\enddata

\tablecomments{Table~\ref{tbl:powerlaw_spectroscopy} is available in its entirety in the electronic edition of the \apjs.  
Interesting sources mentioned in the text are shown here for guidance regarding its form and content. }

\tablenotetext{a}{ 
For convenience, cols.\ (1)--(4) reproduce the source identification, net counts, and photometric significance data from Table~\ref{tbl:src_properties_main}.
}

\tablenotetext{b}{
All fits used the ``wabs(powerlaw)'' model in XSPEC.  
Cols.\ (5) and (6) present the best-fit values for the extinction column density and power law photon index parameters.
Col.\ (7) presents the power law normalization for the model spectrum. 
Quantities marked with an asterisk (*) were frozen in the fit.  
Uncertainties represent 90\% confidence intervals.
More significant digits are used for uncertainties $<$0.1 in order to avoid large rounding errors; for consistency, the same number of significant digits is used for both lower and upper uncertainties.
Uncertainties are missing when XSPEC was unable to compute them or when their values were so large that the parameter is effectively unconstrained.  
Fits lacking uncertainties, fits with large uncertainties, and fits with frozen parameters should be viewed merely as splines to the data to obtain rough estimates of luminosities; the listed parameter values are unreliable.  
}

\tablenotetext{c}{ X-ray luminosities derived from the model spectrum, assuming a distance of XXXX.xxxx~kpc, are presented in cols.\ (8)--(12): (s) = soft band (0.5--2 keV); (h) hard band (2--8 keV); (t) total band (0.5--8 keV).  
Absorption-corrected luminosities are subscripted with a $c$.
Cols. (8) and (12) are omitted when $\log N_H > 22.5$~cm$^{-2}$ since the soft band emission is essentially unmeasurable.  
}

\tablenotetext{d}{} 


\end{deluxetable}
END


%=============================================================================
\end{document}
%=============================================================================

