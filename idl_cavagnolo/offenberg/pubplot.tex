%		PUBPLOT.TEX
%
%	This is a summary of how to use the IDL routine PUBPLOT.PRO.
%	
%	This file is a plain TeX (NOT LaTeX) file--to read it, you should
%	use the ``tex'' command to convert it to a DVI file, then use T2 to 
%	print it out (VAX) or use ``dvi2ps'' to convert it to PostScript and 
%	use ``lpr'' to print it out on a PostScript printer. (SUN)
%		
%	History:
%		Written by Joel D. Offenberg, HSTX, 25 May 1993
%		Last modified: Joel D. Offenberg, HSTX, 25 Aug 1993
%

\magnification=1000  % Char. size * 1.0
\parindent=10mm      % Paragraph indentation
\hsize=6.5truein    % Horizontal size of window on page in which to write
\vsize=8.9truein     % Vertical size of window on page in which to write
\lineskiplimit=2pt   % Minimum limit on space between lines
\lineskip=3pt        % If line-space less than limit then separate by this
% font definitions
% roman text
%
\font\rmlil=cmr5
\font\rmsm=cmr8
\font\rm=cmr10
\font\rmtwelve=cmr12
\font\rmseventeen=cmr17
%
% bold face extended
%
\font\bflil=cmbx6
\font\bfsm=cmbx8
\font\bf=cmbx10 
\font\bftwelve=cmbx12 
\font\bflarge=cmbx12 scaled\magstep2
%
%slanted roman
%
\font\slsm=cmsl8
\font\sl=cmsl10
\font\sltwelve=cmsl12
%
%text italic
%
\font\itsm=cmti8
\font\it=cmti10
\font\itwelve=cmti12
%
% narrow bold font
%
\font\nb=cmb10
%
% slanted bold font
%
\font\slab=cmbxsl10 
%
% 1 inch bold font
%
\font\big=cminch
%
% sans serif font
%
\font\sansm=cmss8
\font\san=cmss10 
\font\santwelve=cmss12
\font\sanseventeen=cmss17
%
% sans serif italic font
%
\font\sani=cmssi10
\font\sanitwelve=cmssi12
\font\saniseventeen=cmssi17
%
% sans serif bold extended
%
\font\sanb=cmssbx10
\font\sanbtwelve=cmssbx10 scaled\magstep2
%
% unslanted italic
%
\font\usit=cmu10
%
% dunhill style
%
\font\dun=cmdunh10
%
% typewriter style
%
\font\ttsmall=cmtt8
\font\tt=cmtt10
\font\tttwelve=cmtt12
%
%Math Symbol fonts.
%
\font\mathitalics=cmmi10
\font\mathsymbol=cmsy10
\font\mathextended=cmex10

\rm                                             % Def. font = roman
%
%			Macros
%
\def\double{\baselineskip=24pt \parskip=0pt}   % Double spacing def.
\def\single{\baselineskip=12pt \parskip=6pt}    % Single spacing def.
\single                                         % Single space the document.
%						  C++ macro.
\def\cpp{C\kern -.07em\raise .25ex\hbox{+}\kern -.1em\raise .25ex\hbox{+}}
\def\bcc{{\it Borland \cpp\ 2.0}}		% Borland C++ name.
\def\tasm{{\it Turbo Assembler 2.5}}		% Turbo Assembler name.
\def\fpfmt#1#2{$#1\times 10^{#2}$}		% Scientific notation.
%
%				Multiple "item" macro.  
%				syntax:  \nitem{2}{string}text....
%				The first expected parameter is the number of 
%				indents.  The second is the string placed 
%				inside the hanging indent.  This macro requires
%				a counter called \itemcount; this counter is
%				defined here. 
%
\newcount\itemcount
\def\nitem#1{\itemcount=#1
\par\loop\advance\itemcount by-1 \ifnum\itemcount>0 \indent\repeat
\hangindent#1\parindent \textindent}

\rm
\def\apj {\underbar\bf Ap. J.}
\def\idl {\hfill\break IDL$>$}
\def\pubplot {\bf PubPlot \rm}

\centerline {\bflarge PubPlot}
\vskip 2mm \centerline {\rmtwelve Joel D. Offenberg}
\vskip 2mm \centerline {\rmtwelve 25 August 1993}

\vskip 5mm \centerline {\rmtwelve Introduction}


{\bf PubPlot} is now available in the MOUSSE library and the IDLASTRO library. 
This 
routine is intended to allow users to adjust plots for publications such as
the \apj\ easily
and quickly by acting as a widget based interface with many of the plotting
keywords.  

{\bf Pubplot's} major features include:
\item{$\bullet$} Plot range settings, allowing the use of both sliders
	and keyboard.
\item{$\bullet$} Settings for plotting symbols and/or line styles.
\item{$\bullet$} Capability to add, modify and remove annotation and comments
	easily.
\item{$\bullet$} Axis styles and types are easily set.
\item{$\bullet$} Addition of a legend, for improved annotation.
\item{$\bullet$} The capability for adding error bars to data.

This memo is intended to give a summary of what PubPlot does and how to 
begin to go about using it.  

\vskip 3mm \centerline{\rmtwelve How to Start \bftwelve PubPlot}
First, enter an IDL session on a workstation (Widgets must be 
present).  Load in 
or create your data.  Then issue the one following IDL commands:

\idl\ PUBPLOT, X, Y	
\idl\ PUBPLOT, X, Y, XERROR = XERR, YERROR = YERR 
\idl\ PUBPLOT, X, Y, OUTPUT = F	
\idl\ PUBPLOT, X, Y, F, OUTPUT=FF	

In the first example, PubPlot will create plots using the data in X and Y.  In
the second, PubPlot will create plots using the data in X and Y and also allow
the user to use the data in XERR and YERR for error bars.  In the third 
example, a record of the plot will be put into the variable F, which can then
be used to recreate your plot.  The last example shows hot to recreate the 
plot saved in F, with the new output being recorded into FF.

In all cases, the ``Publication Plotting'' widget will appear when you start it, along with
a separate IDL window for plots (if there isn't one already).

\vskip 3mm \centerline{\rmtwelve How to Use on Different Machine Types}

\pubplot was written and tested by the author on a VAX workstation running 
DECWindows Motif, and on a Sun SPARCstation running X11R5.  The \pubplot widget
layouts may not look pretty if run on different workstations and/or with 
different windowing systems.  If this is the case, you 
should modify the variables set in the routine SET\_ARCH (which is in the file
PUBPLOT.PRO) to suit your own system.  

\pubplot is written entirely in standard IDL.  Aside from the widget layout
problem mentioned in the above paragraph, \pubplot should be completely 
portable between systems.  

\vskip 3mm \centerline{\rmtwelve More Examples}

For what follows, any items surrounded by brackets [\ ] represent inputs or 
keywords that are optional for the given task.  Items that are surrounded by
curly brackets $\{\} $ are English sentences describing the task rather than 
giving IDL commands.

\noindent To save the results from your \pubplot session for later use:
\idl\ PUBPLOT, X, Y, [XERROR = xerr, YERROR = yerr,] OUTPUT = F
\idl\ SAVE, /VARIABLES, FILE = ``MyFile.sav''

\noindent To recreate a plot from a previous session of \pubplot:
\idl\ RESTORE, ``MyFile.sav''
\idl\ MAKEPUBPLOT, X, Y, F [,XERROR = xerr, YERROR = Yerr, /POSTSCRIPT]

\noindent To recreate a plot from a previous session of \pubplot and print it out on your
default laser printer:
\idl\ RESTORE, ``MyFile.sav''
\idl\ MAKEPUBPLOT, X, Y, F, /POSTSCRIPT [,XERROR = xerr, Yerror = yerr]
\idl\ PSPLOT

\noindent To restore and use the results from a previous \pubplot session:
\idl\ RESTORE, ``MyFile.sav''
\idl\ PUBPLOT, X, Y, F [,XERROR = xerr, YERROR = yerr, OUTPUT = FF]

\noindent The results from one PubPlot session may be used again with a 
different data set entirely.

\noindent To add extra data to a plot on your workstation screen:
\idl\ $\{ $Use \pubplot to create your plot. $\} $
\idl\ OPLOT, XNEW, YNEW

\noindent To add extra data to a plot sent to a PostScript file:
\idl\ $\{ $Use \pubplot to create your plot and copy it to a PostScript file.$\} $
\idl\ SETPLOT,'PS'
\idl\ OPLOT, XNEW, YNEW

\noindent Please note that a PostScript file can not be changed by IDL once it 
has 
been sent to the printer.

\vskip 3mm \centerline{\rmtwelve Using PubPlot}

When you start PubPlot, a large widget will appear.  What follows is a 
quick description of the purpose of each section:

\item{---} DONE -- This button causes the program to exit
\item{---} PLOT to PostScript -- This button causes a ``PostScript Plotting''
 	widget to appear.  You can use this widget to send a copy of your plot
	to a disk PostScript file, to your default PostScript printer or both.
\item{---} LINEAR/LOG -- These buttons will change the plotting type.  
	Initally, your plot will be linear in both axes.
\item{---} AXIS RANGES -- There are 4 widget sets, each one controlling the 
	value at a given edge of the plot (e.g. ``Bottom'' controls the value
	at the lower end of the Y-axis).  You can select a value by either
	moving the slider or by entering a value in the text widgets. 

\itemitem{$\triangleright $} SLIDERS -- Moving the sliders will cause the plot's
	range to change and also will update the text widget when you are done.
\itemitem{$\triangleright $} TEXT -- Your entry in the text window may be in
	any format, decimal or scientific notation.  If you enter a number in
	text widget that is outside of the range of the slider, the value that
	you enter will be used.  If you enter
	a non-numeric value in the text widget, the program should not crash 
	(although it will give you an error message on your terminal window).
\item{---} TITLES -- You can add titles to the top of the plot, or to the X and
	Y axes.  The four buttons to the right will make the text of the
	titles larger, smaller, thicker or thinner by small increments.
\item{---} ANNOTATION -- You may add annotation text or change it with these
	buttons (see Annotation, below).  
\item{---} ERROR BARS -- Clicking here causes a separate widget for handling 
	error bars to
	appear (see Error Bars, below).
\item{---} LEGEND -- This button creates a separate widget for creating and
	modifying a legend (see Legend, below).
\item{---} ADD PLOT DATE -- Puts a small date and user notice at the bottom
	right of your plot when selected (Calls PLOTDATE.PRO).
\item{---} AXIS STYLES -- These options correspond to IDL's plotting keywords
	XSTYLE and YSTYLE.  The left column of buttons corresponds to the X 
	axis, the
	right to the Y axis.  
\item{---} SYMBOLS \& LINES -- There is one set of lists and buttons for each
	set of data.  The initial setting is ``none'' for the symbol and 
	``solid'' for the lines.  You may select any combination of line 
	and/or symbol for
	each set.  You may also change the size and thickness of the lines and
	symbols using the buttons provided.  If needed, there will be a 
	scroll-bar you 
	may use to find the proper set.
 
\item{\ } If you have already created a legend for your plot, changing the 
	symbol or line selections will also be updated in the legend.

\vskip 3mm \centerline{\rmtwelve Annotation}

Clicking on the ``ADD ANNOTATION'' button will cause a message widget to 
appear,
telling you to select a position for your text with the mouse on your plot.
Once you have done that, another widget will appear, asking you for the text.
The annotation text will now appear on your plot.  You may add as many
text notations as you like.

To change the annotation, click on the ``CUSTOM ANNOTATION'' button on the
main widget.  You can change the text of the annotation, reposition it, as 
well as change its size, thickness or rotation.  Click on ``DONE'' when you
have finished.  

To entirely remove a text notation, click on ``REMOVE ANNOTATION.''  This will
give you a menu of annotation texts.  Click on the one you wish to remove (or 
on ``CANCEL'' to remove none).

\vskip 3mm \centerline{\rmtwelve Error Bars}
	
The ``Error Bars'' widget allows you to select various options for error bars.
If you supply uncertainty values through the XERROR or YERROR keywords, those
values will be used if you select ``USER VALUE''.  If you select ``CONSTANT,''
``PROPORTIONAL'' or ``SQUARE RT,'' the error bars will be that type, multiplied
by the number in the accompanying text windows.  For example, if you select
``PROPORTIONAL'' and ``0.25'' for Y error bars, the error bars will be 
$0.25 *$ Y.  If you do not supply a value for the error bars using the 
XERROR and YERROR keywords, the ``USER VALUE'' option will not work, but the
others will.

To suppress the error bars, select ``OFF'' for each data set in both 
dimensions.  Error bars may set for each data set independently of the others.

If you wish to add hats to your error bars, select the ``HATS'' button for that
data set.  You can change their length and/or thickness with the buttons.

\vskip 3mm \centerline{\rmtwelve Legend}

When you click on ``SET POSITION,'' a widget appears asking you to click on
the plot where you want your legend to go.  The position you select will be the
lower--left corner of the legend box.

You can change the names of the data sets by editing the text windows.  If you
wish to remove one set from the window, you can deactivate it by unselecting 
the button to the left of its name.  You can turn the item back on by 
reselecting the same button.  If you turn off all of the items, the legend
is turned off.  

The ``Delimeter'' text widget can be used to specify what character (if any) 
you wish to place between the symbols and the text.  Only the first character
of the delimeter text is used.  You may also make the text and symbols larger 
or smaller, as well as turn off the box surrounding the legend.  

The symbols are taken automatically from the symbol/line combinations that you
have already selected for your plot.  If you change the symbol and/or line of 
a data set, the change will be updated automatically in the legend.

\vskip 3mm \centerline{\rmtwelve Handling Widgets}

IDL's widgets essentially come in several ``flavors''.  A rough sketch of the
flavors appearing in the PubPlot widgets, and how they work from the user's 
end, follows (for more details, consult the IDL documentation).

\item{---} TEXT -- Text widgets are small windows which display text and may 
	receive text input (Not all text widgets are editable).  To edit a
	text widget, the user must click on the widget (to select it), then
	type in their changes, and then enter a Carriage Return.  IDL will not
	recognize changes in a text widget until the $<$CR$>$ is pressed.  
\item{\ } Please be warned that, on the VAX, the delete key does not work the
	same in a text widget as it does elsewhere.  In a text widget
	the delete key deletes the character to the right of the cursor, not
	the one to the left.
\item{---} SLIDERS -- Slider widgets may be set by pointing the mouse at the
	widget, clicking and holding the left mouse button and dragging the 
	slider to its new position.  The new position will not register with
	IDL until you release the mouse button.
\item{---} BUTTONS -- There are several flavors of buttons.  To push or release
	any button widget, click on the button with the left mouse button.
\itemitem{$\triangleright $} ONE POSITION -- These buttons pop up once you 
	release the mouse button.  Clicking on it a second time will only 
	cause IDL to perform the same task again.
\itemitem{$\triangleright $} TWO POSITION -- These buttons stay pushed down
	until you click on it a second time.  Clicking a second time will 
	usually ``undo'' whatever the first click did.  Some sets of buttons 
	will only permit you to have one button pressed down at a time, and 
	selecting one of these will unselect the others.
\item{---} LISTS -- Simply click on the desired item with the left mouse 
	button.
\item{---} LABELS -- Labels are informational text.  You may not change a label
	widget directly.
\item{---} FADED WIDGETS -- Sometimes, certain widgets are faded.  This means
	that they are inactive and may not be used at that time.  Please note 
	that some
	widget types (namely text and list widgets) do not fade when they are
	inactive.


\bye
