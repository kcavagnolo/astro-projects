\documentclass[letterpaper,11pt]{article}
\usepackage{graphicx,common}
\usepackage[nonamebreak,numbers,sort&compress]{natbib}
\bibliographystyle{plainnat}
\setlength{\textwidth}{6.5in} 
\setlength{\textheight}{9in}
\setlength{\topmargin}{-0.0625in} 
\setlength{\oddsidemargin}{0in}
\setlength{\evensidemargin}{0in} 
\setlength{\headheight}{0in}
\setlength{\headsep}{0in} 
\setlength{\hoffset}{0in}
\setlength{\voffset}{0in}

\makeatletter
\renewcommand{\section}{\@startsection%
{section}{1}{0mm}{-\baselineskip}%
{0.5\baselineskip}{\normalfont\Large\bfseries}}%
\makeatother

\begin{document}
\pagestyle{plain}
\pagenumbering{arabic}

\begin{center}
\bfseries\uppercase{Radio Feedback in Clusters and Galaxies}
\end{center}

\section{Scientific Justification}

High-resolution X-ray images of the cores of galaxy clusters and X-ray
halos of giant elliptical galaxies have revealed cavities and shock
fronts in many systems, providing evidence that the energy injected
from the accreting supermassive black hole of the central galaxies
interacts with the X-ray gas. Measurements of the $pV$ work required
to inflate cavities and to drive shocks provide reliable estimates of
the mechanical energy associated with radio jets and lobes. These
energies range from $10^54$ ergs in giant ellipticals to $10^62$ ergs
in the cores of rich clusters. Cavity ages, estimated independently
using buoyant rise times and shock speeds, have yielded mean jet
powers large enough to offset cooling flows in normal elliptical
galaxies and in rich clusters (see McNamara \& Nulsen 2007 for a
review). This has significant consequences for understanding the
evolution of galaxies and clusters. However, the details of how radio
jet energy is transferred to the intergalactic medium (IGM) or
intracluster medium (ICM) and then converted to heat is still poorly
understood, and better understanding these mechanisms is crucial to
constructing complete models for understanding galaxy formation and
evolution.

For example, the suppression of star formation by AGN feedback at late
times in the so-called ``radio mode" (as opposed to the early ``quasar
mode" that imprinted the Magorrian relation on galaxy bulges) may be
responsible for the turnover at the bright end of the galaxy
luminosity function and the dearth of bright, blue galaxies that are
present in simulations employing standard cold dark matter models
(\eg\ Croton et al. 2006). Furthermore, the energy released in AGN
outbursts contributes to the excess entropy in clusters, and may lead
to the breaking of self similarity in the scaling relations between,
X-ray luminosity, gas temperature, and mass (Voit 2005; Voit \& Donahue
2006; Cavagnolo 2009).

Study of cavities and shock fronts in the X-ray and radio have yielded
direct measurements of the radiative efficiencies of extragalactic
radio sources. In addition, the energetics of these systems have
placed strong constraints on the contents of extragalactic radio
sources (Dunn et al. 2005, De Young 2006). The origin and composition
of extragalactic radio jets has remained enigmatic since their
discovery more than a half century ago. By virtue of their synchrotron
emission, we know that they are in part composed of relativistic
electrons and magnetic fields. Theoretical models of jets (\eg\
Scheuer 1974; Begelman, Blandford \& Rees 1984) have shown that their
energetics are dominated not by photons, but rather by mechanical
energy. Combining the total energy measurements from X-ray
observations with synchrotron power measurements over a broad
frequency range can be used to place interesting limits on the ratio
(k) of energy flux carried by protons or other massive particles to
that in electrons.  Using samples of central cluster galaxies
harboring prominent cavity systems filled with radio emission, Dunn,
Fabian, and Taylor (2005) and Birzan et al. (2008) have shown that on
average k > 1, and in some cases k exceeds several thousand. This
implies that the energy flux in jets is dominated by protons,
presumably either launched at the base of the jet or entrained from
surrounding material as the jet advanced through the IGM. A
dramatically different interpretation by Diehl et al (2008) suggest
that the distribution of cavities in clusters is consistent with
current-dominated MHD jets.

However, the ratio of mechanical energy to synchrotron energy
(radiative efficiency) which provides a strong clue to their
composition cannot be determined by radio observations
alone. Combining VLA observations at several frequencies with Chandra
imagery of 18 clusters, Birzan et al. (2004, 2008; hereafter B04 and
B08, respectively) found radiative efficiencies ranging from unity to
a few parts in $10^5$, with median values of approximately one part in
a few thousand. Their measurements show that even weak radio sources
are often associated with powerful AGN outbursts, whose energies and
momenta are carried primarily by heavy, non-radiating particles, such
as protons (Dunn et al. 2005, De Young 2006, B08). Furthermore, B04
and B08 found that jet (mechanical) power correlates with radio power
approximately as $L_{jet} \sim L_{radio}^{1/2}$ in clusters, but with
a large intrinsic scatter. This important correlation provides the
means to use radio observations to estimate the mechanical feedback
power in galaxies over large volumes of the Universe, where X-ray
measurements cannot be made. Applying this relationship to NVSS radio
sources in Sloan ellipticals, Best et al. (2006) showed that
radio-mode feedback is energetically important in distant giant
ellipticals. However, Best's study was based on B04's cluster
measurements, and not on lower luminosity ellipticals whose numbers
dominate in the Universe, and are the focus of this proposal.

B08 found that VLA imaging at 320 MHz provides the best tracer of jet
power by providing a lower scatter relationship between radio and jet
power than was found at 1.4 and 8 GHz. This is because 320 MHz is
sensitive to the aging electron population that dominates the emission
of the cavities and radio lobes. We propose to extend our VLA imaging
survey of cavity systems to groups and isolated giant ellipticals with
lower jet and radio powers. In particular, we wish to determine
whether the power law relationships found by B04 and B08 apply over 8
orders of magnitude in jet power from giant cD galaxies in clusters to
to isolated galaxies. The observations proposed here will yield
calibrated scaling relationships between jet power and radio power
that can be used to study feedback through much of cosmic history. It
will also provide the larger sample necessary to study and understand
the primary factors that cause the scatter in this relationship, that
may include aging, adiabatic expansion, variations in magnetic field
strength, and particle content.

However, in the aggregate, k >> 1, with values varying from close to
unity to several thousand. The proposed observations of clusters with
the best jet cavity power measurements from X-ray observations will
provide the best constraints available on k, and thus the content of
extragalactic radio sources.  Low frequency radio observations will be
absolutely crucial in resolving the issue of the content of these
cavities, because they are most sensitive to the history of AGN
activity over timescales > 108 yr, while high frequency observations
are sensitive to the instantaneous jet power.

\section{Targets \& Strategy}

Our sample of 13 ellipticals was drawn from the 109 nearby giant
elliptical galaxies found by C. Jones and collaborators to have X-ray
emitting hot atmospheres. All have been observed with \chandra. Our
targets consist of a subsample of the 27 with detected cavity systems
(Nulsen et al. 2007). Our long term goal is to obtain high signal to
noise radio images of the entire subsample at at least three
frequencies: 327 MHz, 1.4 GHz, 8 GHz. All of these targets were
previously observed 327 MHz with the VLA in A-configuration. 327 MHz
is most sensitive to cavity systems, primarily because cavity systems
are typically a few $10^6$ yrs old, so that their radio emission is
dominated by an aging electron population. Thus 320 MHz is sensitive
to the history of AGN activity not just the most recent outburst. Our
strategy, as we adopted for B08, is to reduce and analyze the existing
observations in the VLA archive and fill in the frequency gaps in
future observing cycles. Of the 27 systems with cavities, 14 are being
observed with the Indian GMRT as part of another program (Vrtilek,
PI), and we are proposing here for the remaining 13. The two samples
will be combined to form a complete sample. The estimated 320 MHz
fluxes are typically 500 mJy. NGC 4261, NGC 4782, and NGC 4374 have
total fluxes estimated to be roughly 25 Jy. Our targets are listed in
order of decreasing priority in the Source List. We will use the
standard P-band observing mode (4IF, 3.125MHz per IF, 32 channels,
Hanning smoothed). The total flux density of all sources is larger
than 100 mJy, assuming a standard spectral index of $~0.8$. We are
asking for 3 hours per source, which will lead to a sensitivity limit
of 0.3 mJy/beam, and provide adequate uv coverage for imaging with
reasonable dynamic range ($~1000$). Our experience with the previous
sample of radio galaxies (B08) has shown that the proposed
observations will be adequate to detect the extended emission in the
sources associated with the X-ray cavities. The VLA A-array resolution
of $4.5\arcs$ at 320 MHz is comparable to or smaller than the cavities
in our targets, so most will be well resolved.

\section{References}

Athreya, R. 2008, ApJ, submitted;\\
Begelman, Blandford, Rees 1984, Rev. Mod. Phys. 56, 255;\\
Birzan, L., Rafferty, D.A., McNamara, B.R., Wise, M.W.; \& Nulsen, P.E.J. 2004, ApJ, 607, 800;\\
Birzan, L., et al. 2008, ApJ, 686, 859;\\
Clarke, T.E., Sarazin, C.L. Blanton, E.L., Neumann, D.M., \& Kassim, N.E. 2005, ApJ, 625, 748;\\
Diehl, S., et al. 2008, ApJ, 687, 173;\\
Dunn, R.J.H., Fabian, A.C., \& and Taylor, G.B. 2005, MNRAS, 364, 1343;\\
Fabian, A.C., Sanders, J.S., Allen et al., 2003, MNRAS, 344, L43;\\
Giacintucci, S., Vrtilek, J.M., Murgia, M., Raychaudhury, S., O'Sullivan, E.J.; Venturi, T., David, L.P., Mazzotta, P., Clarke, T.E., \& Athreya, R.M. 2008, ApJ, 682, 186;\\
McNamara, B.R., \& Nulsen, P.E.J., 2007, ARAA, 45, 117;\\
Nulsen, P.E.J., McNamara, B.R., Wise, M.W., \& David, L.P. 2005,ApJ, 628, 629;\\
Scheuer, P 1974, MNRAS, 166, 513;\\
Wise et al. 2007, ApJ 659, 1153

\end{document}
