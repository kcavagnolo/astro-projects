\documentclass[12pt]{article}
\usepackage[colorlinks=true,linkcolor=blue,urlcolor=blue]{hyperref}
\usepackage{mathptmx,multicol,common}
\parindent 0pt
\parskip
\baselineskip
\setlength{\topmargin}{-0.30in}
\setlength{\oddsidemargin}{-0.30in}
\setlength{\evensidemargin}{-0.30in}
\setlength{\headheight}{0in}
\setlength{\headsep}{0.25in}
\setlength{\topskip}{0.25in}
\setlength{\textwidth}{6.9in}
\setlength{\textheight}{9.25in}
\pagestyle{myheadings}
\newcommand{\myhead}{Cavagnolo, Teaching}

\begin{document}

\begin{center}
{\Large\sc{Description of Teaching Experience}}
\end{center}

\end{document}

Self-reflection over the role of teacher:

The description of teaching expertise should make clear not only what
the applicant has done but also how it has been done, why it was done
in just this way, and the results. The applicant is to state his or
her fundamental educational principles and the way in which these are
expressed in practice.

The self-reflection is to have a maximum length of five pages when
applying for employment as professor, senior lecturer or associate
senior lecturer, and a maximum length of one page when applying for
employment as postdoctoral research fellow.

Teaching experience:

Within undergraduate education at basic and advanced levels, research
education, further and higher education: Specify the extent, the
width, the level, and level of responsibility for the courses
listed. The volume of education and its type should not only be
specified, specify also the degree of responsibility, and give details
of active development work that took place for the listed courses.

Experience as supervisor:

Within undergraduate education at basic and advanced levels: Specify
the number of undergraduate projects for which the applicant has been
supervisor.

Within research education: Specify the name of the research student,
year of admission and year of award of the degree, and, where
relevant, the names of other supervisors.

Current supervision of doctoral students: Specify whether the
applicant is the principal supervisor or assistant supervisor, and
specify the year of admission of the student.

Studies in educational theory:

Courses in the theory of higher education; teacher training;
conferences, seminars and projects in educational theory. Specify the
dates and extent. Copies of certificates detailing studies into the
theory of higher education should be enclosed.

Course development and the administration of education:

Appointments as, for example, director of studies or study adviser.

Work in the theory of education, and teaching materials:

Books, articles, etc. Compendiums/course material. Specify the form,
level, extent, and significance within the education. The list should
make it clear which of the publications the applicant will send to the
members of the expert panel for assessment (see above, Section 3).

Educational distinctions, prizes
Other teaching expertise

Assessment of teaching activities:

Written statements that are available from the head of department or
director of studies with a qualitative assessment of the applicant’s
teaching skills should be submitted. Where relevant, summaries only of
course assessments may be submitted, but not of individual
assessments!
