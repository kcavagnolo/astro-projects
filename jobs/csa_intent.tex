\documentclass[11pt]{article}
\usepackage{common}
\setlength{\topmargin}{-.3in}
\setlength{\oddsidemargin}{-0.1in}
\setlength{\evensidemargin}{-0.1in}
\setlength{\textwidth}{6.7in}
\setlength{\headheight}{0in}
\setlength{\headsep}{0in}
\setlength{\topskip}{0.5in}
\setlength{\textheight}{9.25in}
\setlength{\parindent}{0.0in}
\setlength{\parskip}{1em}
\pagestyle{empty}
\begin{document}

The National Aeronautics and Space Administration (NASA) sponsors the
Center for Astronomy Education (CAE) which was setup to develop the
teaching and instruction skills of introductory astronomy
educators. The CAE serves all levels of educators, from elementary
schools to Tier A collegiate research institutes. While a graduate
student at Michigan State University (MSU) I attended several of the
CAE Workshops which are held throughout the year and throughout the
country. My Ph.D. advisor was a strident supporter of the CAE's work
and research, and she encouraged me to use the CAE as a means for
becoming a better instructor as the CAE methods had, in her opinion
and based on student review, vastly improved my advisor's
instruction. Indeed, I found my advisor's method of instructing me to
be superior to the methods used by other advisors and professors
within our own department.

As I am not formally trained in education and do not hold a teaching
certification, the CAE was my channel into seeing what education
research was telling the community about the classic dogmatic
techniques of lecture, homework, and testing. I was surprised to find
out that the methods with which I had been taught for over 20 years
were not optimal. It was through my interaction with the CAE and their
materials that I adopted the education philosophy and pedagogy of
interactive instruction and peer instruction. In interactive teaching,
much of the instruction time is dedicated to working with the students
by not simply lecturing, but asking the students questions, having
them work together on small tasks which can be debated at-large, and
using methods such as think-pair-share or lecture
tutorials. Similarly, in peer instruction, students are asked to
question each other, or to critique another student's answer. The
purpose of this approach is to generate teachable moments for the
instructor. Before a student can be instructed in such a way that the
retention is permanent and the understanding is full, the instructor
must identify where preconceptions are interfering or where a fallacy
exists. Once a teachable moment has been created, then the classic
Aristotelian dissemination of knowledge from instructor to pupil can
occur.

I find this method of instruction particularly suited to my
personality because I have a high level of energy, am very optimistic,
am an excellent listener, and have a great deal of compassion and
empathy for others. As where a traditional lecturer does not
necessarily need these traits, interactive teaching requires being a
part of the class as well as its leader. To that end, I need to listen
closely to the students. Keeping pace with their youthful energy to
learn is another must. It's also important to remain focused through
moments of frustration, to remember that the students can learn and
that I can do better. But most importantly to me is that I understand
how the students feel within the learning environment, ensuring they
are not being overwhelmed and shelling-up out of fear or frustration.

The educational philosophy I have adopted also fits well with my
educational background and professional experience in math and
science. The physics program and Georgia Tech and the astrophysics
program at MSU are both highly rated. As such, I had access to some of
the nation's best instructors and facilities while a student. This has
resulted in me having deep and highly extensible knowledge in the
areas of math and science and in particular physics and astronomy. I
am proud to say that my two Ph.D. advisors at MSU, and my new
colleague at the University of Waterloo, are among the most highly
respected and cited researchers in the area of astrophysics. This has
had the ancillary benefit of having exposed me to a broader network of
peers and fully understanding how one navigates the sciences to
achieve long-term goals. This would be a very desirable skill set for
any secondary or preparatory level institution interested in hiring me
and propelling their students to successful careers.

In my life, I have been fortunate enough to benefit from the hard-work
of others, such as my mother and two grandfathers. The opportunities
garnered from those benefits have been many and unique. I have not
lived a sheltered life and am well-aware of the educational inequality
which exists in the American public school system. As an educator, I
also want to work with an organization which provides a robust
infrastructure which I could utilize to redistribute the fruits of the
many opportunities I've received in life. For example, I would like to
use my position as an educator to do outreach work in communities
where upward mobility is limited or the where the students (and
especially gifted students) do not have access to teachers with broad
world-views or advanced degrees.

I would like to work within the outreach setting to give students the
opportunity to have their own ``AH HA!''  moment in life. To have a
moment where they think, ``I can do anything,'' or ``I want to
dedicate my life to \_\_\_\_\_\_.'' The lasting effects of those
moments in my life have had a profound and resonating impact on my
hopeful, optimistic life outlook and goal setting. As a measure of my
success in this endeavor, I would hope to leave students that set
specific and attainable goals which lead them into community
involvement and academic excellence in the form of scholarships or
grants. Ultimately it would be great to have those same students
achieve their dreams, and then return to the communities, via their
physical or financial influence, and better those communities.

In total, I have the skill sets and educational background to make
meaningful contributions across diverse socioeconomic demographics in
almost any school setting. Personally, I believe that the pursuit of
knowledge and the fostering of innate human curiosity are key to
enriching one's own life. This has been one of the driving forces in
my life, and I hope to share that students some day.

\end{document}
