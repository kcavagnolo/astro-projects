\documentclass[11pt]{article}
\usepackage{common}
\setlength{\topmargin}{-.3in}
\setlength{\oddsidemargin}{-0.1in}
\setlength{\evensidemargin}{-0.1in}
\setlength{\textwidth}{6.7in}
\setlength{\headheight}{0in}
\setlength{\headsep}{0in}
\setlength{\topskip}{0.5in}
\setlength{\textheight}{9.25in}
\setlength{\parindent}{0.0in}
\setlength{\parskip}{1em}
\pagestyle{empty}
\begin{document}

The topic of global climate change (GCC), formerly global warming, has
grasped an international audience as a result of environmental
scientists clearly communicating how GCC will impact everyone's
day-to-day life. However, scientists have done an equally poor job of
explaining how they have used verifiable, reproducible observational
evidence to arrive at the, often foreboding, conclusions. It is in
this disconnect between enunciation and explanation that GCC
detractors, many of whom are laymen, have seized the opportunity to
distort or misinterpret the observational data and computer
simulations. The result is a deafening din of nuanced discussion
between experts which often leaves the public in confusion about what
is ``right.'' The unfortunate consequence of this battle is that the
underlying message which most everyone can agree on is being trampled:
humans are making the environment which sustains us uninhabitable
through careless and reckless industrial, personal, and governmental
actions. History has a clear statement on situations which are debated
in such a way: cease the debate, focus on the primary problem, and
plot a course of action which is independent of right and wrong but
instead addresses what is mutually beneficial. Such an approach would
focus on what is in the best interest of humans, flora, fauna, and
{\it{terra firma}}.

\end{document}
