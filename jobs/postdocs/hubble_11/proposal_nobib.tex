\documentclass[letterpaper,12pt]{article}
\usepackage[numbers,sort&compress]{natbib}
\bibliographystyle{unsrt}
\usepackage{common,graphicx}
\pagestyle{myheadings}
\setlength{\textwidth}{7.2in} 
\setlength{\textheight}{9.3in}
\setlength{\topmargin}{-0.3in} 
\setlength{\oddsidemargin}{-0.35in}
\setlength{\evensidemargin}{0in} 
\setlength{\headheight}{0in}
\setlength{\headsep}{0.3in} 
\setlength{\hoffset}{0in}
\setlength{\voffset}{0in}
\newcommand{\myhead}{Cavagnolo, Hubble Proposal}
\newcommand{\pcos}{posed by NASA's Physics of the Cosmos Program}
\begin{document}

\begin{center}
  {\bf\uppercase{mapping galaxy cluster magnetic fields:\\an
      observational study of icm physics}}
\end{center}

\noindent{\bf{I. Motivation}}\\
\indent Clusters of galaxies are the largest structures in the
Universe to have reached dynamic equilibrium, and most cluster
baryonic mass resides in the intracluster medium (ICM), a hot, dilute,
weakly magnetized plasma filling a cluster's volume [1]. As the
defining characteristic of the most massive objects in the Universe,
the thermal properties of the ICM are well-known, but a similarly
detailed description of ICM non-thermal properties -- specifically
diffuse cluster magnetic fields (CMFs) -- and how they relate to the
thermodynamic nature of the ICM remains elusive. Filling this gap in
knowledge is vital because clusters help us constrain cosmological
parameters [2], develop hierarchical structure formation models [3],
and study the synergy of many physical processes to answer the
question \pcos, ``How does the Universe work?'' [4].

At present, one of the biggest challenges in cluster studies is
explaining the relative thermal equilibrium of the ICM. Many clusters
have core ICM cooling times much less than a Hubble time, and it was
hypothesized that these systems should host prodigious ``cooling
flows'' [5]. But, only minimal mass deposition rates and cooling
by-products have ever been detected, requiring that the ICM be heated
[6]. Observational and theoretical studies have strongly implicated
feedback from active galactic nuclei (AGN) in supplying the
{\it{energy}} needed to regulate ICM cooling and late-time galaxy
growth [7]. However, precisely how AGN feedback energy is thermalized
and which processes comprise a complete AGN feedback loop remain to be
fully understood [8].

Theoretical studies are now focused on coupling AGN feedback and ICM
heating using combinations of anisotropic thermal conduction, cosmic
ray diffusion, and subsonic turbulence [\eg\ 9--17] after observations
suggested the ICM is turbulent and conducting on small scales
[\eg\ 18--20]. These microphysical processes are intrinsically linked
to macroscopic CMF topologies through gas viscosity and
magnetohydrodynamic (MHD) instabilities [21, 22]. Thus, to
observationally test and refine this theoretical framework, it is
ideal to have uniform measurements of CMF strengths, orientations, and
spatial distributions for large cluster samples spanning a broad range
of evolutionary \& dynamical states. Unfortunately, CMFs must be
indirectly observed through steep spectrum, non-thermal synchrotron
emission best detected at low radio frequencies [$\la 2$ GHz; 23]. A
robust census of CMFs has been lacking because of limitations in the
sensitivity and resolution of past radio measurements [\eg\ 24],
limiting our knowledge of CMF demographics to a few clusters
[\eg\ 25]. Additionally, this shortfall has inhibited the study of CMF
origins and dynamical importance [26].

The greatly improved capabilities of NRAO's Expanded Very Large Array
(\evla) will significantly impact this field by bringing powerful
radio CMF survey and polarization capabilities through unprecedented
sensitivity, resolution, and frequency coverage [27]. {\bf{As a Hubble
    fellow, I propose to use radio polarimetry in conjunction with
    X-ray \& optical imaging to map CMFs (magnitudes, orientations, 3D
    structure) and evaluate their relationship with ICM thermal
    properties (\eg\ temperature, entropy, pressure) to constrain
    which physical mechanisms are responsible for the qualitative
    differences between observed and theoretical CMFs.}}  This work
will 1) determine which microphysical processes significantly
contribute to heating of the ICM by directly comparing the predictions
of theoretical models with CMF observations, and 2) place constraints
on the origin of CMFs and the cosmological implications of non-thermal
pressure support on cluster mass estimates. The proposed project
includes plans for an \evla\ radio survey and NOAO optical
\halpha\ survey of two well-studied cluster samples, and incorporates
an on-going pipeline analysis of an archive-limited sample of clusters
having X-ray data.\\

\noindent{\bf{II. Observations and Analysis}}\markright{\myhead}\\
\indent The \evla\ Polarimetry Cluster Survey (EPiCS) will target the
flux-limited HIFLUGCS [28] and representative REXCESS [29] cluster
samples for which uniform \chandra\ and \xmm\ X-ray data is
available. EPiCS will utilize \evla's increased polarimetry bandwidth
(8 GHz per polarization) and frequency accessibility (74 MHz; 330 MHz;
full coverage 1--2 GHz) to obtain deep ($\sigma_{\rm{rms}} < 10
~\mu$Jy beam$^{-1}$) full Stokes observations of each cluster. The
improved \evla\ efficiency and dynamic range mean extended sources as
faint as $\sim$2-3 $\mu$Jy will be detected with integration times
$\la 5$ hrs, well into the regime where $\mu$G CMFs excite
emission. One of \evla's two low-frequency ($< 0.5$ GHz) systems is
now functioning, and EPiCS will be cross-calibrated with data from
\lofar\ to expand the utility of the dataset. EPiCS observations are
designed to enable measurements of: 1) CMF strengths using Faraday
rotation measures (RM) of previously undetected embedded \& background
cluster radio sources [see Fig. 1 and 30 for method], 2) large-scale
CMF orientations using coherent polarized emission of orbiting cluster
member galaxies [see Fig. 2 and 31 for method], and 3) CMF spatial
distributions \& ordering using low-surface brightness emission of
radio halos [see 32 for method]. Combined with the archival X-ray data
for each source, the following outstanding issues will be
investigated.

{\bf{A. Testing Models of ICM Heating:}} The EPiCS campaign will
produce data of sufficient quality to measure RM dispersions, estimate
CMF radial amplitude profiles, directly reconstruct CMF power spectra,
and model 3D CMF structure using RM synthesis [see methods in
  33--35]. Each of these CMF diagnostics will be directly compared
with results from MHD models in the literature (see Fig. 3 for
example) to determine which predictions are replicated
(\eg\ preferentially radial CMFs, CMF profile shapes, CMF
magnitude--ICM \nelec\ \& \tx\ correlations), which predictions
indicate the input physics may be incomplete, and to help constrain
which microphysical processes might participate in ICM heating. Since
AGN feedback is the likely progenitor of heating, an investigation of
possible correlations between CMF properties and feedback signatures
(\eg\ cluster core entropy, jet powers for systems with cavities, 2D
thermal distributions, extent of central AGN activity) will be
undertaken. Further, turbulence is considered vital for promoting ICM
heating, but is difficult to directly measure. However, secondary
diagnostics (\eg\ AGN outflows, mergers, cold fronts, shocks) may
indicate the presence of turbulence even when the data is insufficient
to do so. These indicators will be considered during the analysis to
check if trends exist with CMF properties.

{\bf{B. CMFs in Cluster Cores:}} It is hypothesized that the
\halpha\ filaments seen in almost all cool core clusters provide a
local measure of CMF strength and orientation since they may form
along field lines and be excited by some combination of turbulent
mixing and conduction [36, 37]. To probe CMF configurations and
conductive heating on kpc scales, below the reach of the radio
observations, a uniform optical survey for extended \halpha\ filaments
in the EPiCS cluster samples will be undertaken using new NOAO
instruments (\ie\ Magellan Maryland Tunable Filter, WIYN HiRes IR
Camera, SOAR Spartan IR Camera) [see 38, for method]. The observations
will allow, for the first time, a complete characterization of
filament morphologies and energetics to be compared with uniform ICM
and CMF properties for the same objects. These observations will
confront model predictions by answering the question, ``Are filament
energetics and morphologies consistent with them being magnetic
structures conductively heated by the ICM?''  Combined with the
radio-derived CMF properties, inferences will also be drawn about if,
and possibly how, large- and small-scale CMF properties are related
(\eg\ the coherence length). The model comparisons from Section A will
also answer the questions: do filaments thrive in low-turbulence,
high-magnetic field strength environs? Does this imply MHD
instabilities are suppressed or inactive in some cluster cores?

{\bf{C. Constraining CMF Origins and Non-thermal Pressure Support:}}
Simply put, where do CMFs come from (\eg\ amplified primordial cosmic
field?  the Biermann battery process?  AGN/galactic outflow seeding of
protoclusters?), and are they dynamically important [39]? The EPiCS
project will help address these questions. As the quantities most
closely related to dynamo-driven CMF formation, I will investigate how
redshift, halo concentration, and cluster mass relate to the derived
CMF power spectra and radial profiles [40]. At a minimum, these
comparisons will place limits on the strength and distribution of
allowable seed field models, and may possibly suggest whether early-
or late-time amplification processes dominate [41]. Deriving halo
concentrations and cluster masses follow directly from the X-ray
analysis already in-hand. However, cluster masses are traditionally
derived by assuming the ICM is in hydrostatic equilibrium. If CMFs
provide significant ICM pressure support, then cluster masses may be
systematically overestimated, having interesting repercussions on
cluster cosmological studies. Thus, cluster masses and the cluster
mass function will be recalculated [\eg\ 42] including terms for CMF
pressure support determined from the EPiCS measurements. How
cosmological parameter uncertainties depend on CMFs will then be
determined. This exercise will be particularly interesting for the
REXCESS sample which has high-quality hydrostatic mass estimates [43].

{\bf{D. Archival Project and Legacy:}} Work has started on archival
\chandra\ and \vla\ data to build the infrastructure needed to
maximize the ultimate scientific impact of this project and produce
initial results for an archive-limited sample of clusters. There are
$\approx 450$ clusters which have archival \chandra\ ($\approx 900$
observations) and \vla\ ($\approx 1000$ observations) data. Of these,
325 clusters have had the X-ray data reduced using an extensible and
mature pipeline, while 50 of those clusters have had the
multifrequency radio data reduced. The X-ray results are being kept in
a public database\footnote{http://www.pa.msu.edu/astro/MC2/accept/}
while the radio analysis continues. The on-going analysis entails
production of 2D ICM temperature, density, pressure, \& entropy maps,
more radial profiles (\eg\ effective conductivity, implied suppression
factors), and refinement of the radio reduction pipeline. Mitigation
of radio frequency interference (RFI) is a lengthy and tedious step in
radio analysis. To alleviate this tension, a portable \python\ version
of the `RfiX' rejection algorithm [44] has been written and is being
tested. To widen this proposal's scientific impact and relevance to
future radio observatories (\eg\ \lofar, \lwa, \ska), all code,
software, and results will be made freely available to the research
community. The results from this project will also serve as science
drivers for future \evla\ upgrades, \ie\ Low Frequency System (full
50--1000 MHz) \& ultracompact E-array [45], being proposed
specifically to study cosmic magnetism.\\

\noindent{\bf{III. Host Institution and Timeline}}\markright{\myhead}\\
\indent The University of Maryland (UMD) is an ideal host for this
ambitious project as it boasts a team of experts well-suited to assist
with this work. The UMD Astronomy Department, the Center for Research
and Exploration in Space Science and Technology, and the Center for
Theory and Computation are hosts to (to name but a few) Keith Arnaud,
Tamara Bogdanovi{\'c} (current Einstein fellow), Michael Loewenstein,
Craig Markwardt, Cole Miller, Richard Mushotzky, Eve Ostriker, Chris
Reynolds (the sponsor), Massimo Ricotti, and Sylvain Veilleux. All
those listed are experts in one, or several, of the areas of AGN
feedback, ICM physics, computational modeling, magnetic field
polarimetry, plasma physics, and X-ray/radio observing \& analysis. In
addition, UMD has close ties with the Naval Research Lab where Tracy
Clarke and Namir Kassim are appointed -- both experts in the radio
techniques and science topics discussed here. {\bf{Year one:}} Data
acquisition begins, the archival project and tool development
continue; I initiate a theoretical/simulation collaboration with the
UMD plasma physics group to study new questions like: How do
convective instabilities couple with ICM cooling and the actual
accretion which drives AGN activity?  What is the relation between
these processes, ICM temperature \& density, and thermal instability
formation? {\bf{Year two:}} Data acquisition continues, first round of
archival-based results is published, and observation-model comparisons
begin. {\bf{Year three:}} Data acquisition and analysis of REXCESS
conclude, second round of results published, and the investigation of
CMF origins and non-thermal pressure support is underway.


%%%%%%%
% Bib %
%%%%%%%

\markright{\myhead}
\input{short.bbl}

%%%%%%%%%%%
% Figures %
%%%%%%%%%%%

\begin{figure}
  \begin{center}
    \begin{minipage}{\linewidth}
      \includegraphics*[width=\linewidth, trim=0mm 0mm 0mm 0mm, clip]{bonafede.ps}
    \end{minipage}
    \caption{Radial profile and power spectrum of the Coma CMF derived
      from 3D simulations which reproduce observed RMs of embedded and
      background radio sources (taken from Bonafede \etal\ 2010
      [29]). If one assumes the CMF and ICM thermal radial
      distributions trace each, then comparison of observed RM
      dispersions and 3D simulations lead to constraints like this on
      the CMF profile without the need to make measurements at every
      radius. One goal of this proposal is to expand upon the Bonafede
      \etal\ result using a larger cluster sample and EVLA
      observations.}
  \end{center}
\end{figure}

\begin{figure}
  \begin{center}
    \begin{minipage}{\linewidth}
      \includegraphics*[width=\linewidth, trim=0mm 0mm 0mm 0mm, clip]{pfrommer.ps}
    \end{minipage}
    \caption{Virgo CMF orientations (yellow arrows) taken from
      Pfrommer \& Dursi 2010 [30] where they argue draping of CMF
      lines at the ICM-infalling galaxy interface explains the CPE
      (left panel at 5 GHz). The CPE results from galactic cosmic rays
      gyrating around regularly compressed field lines. The authors
      argue the Virgo CMF is preferentially radial, consistent with
      the effects of a large-scale MHD convective instability
      [\ie\ the MTI; 21]. Similar measurements are a key feature of
      the EPiCS project and will help us constrain CMF orientations as
      never before.}
  \end{center}
\end{figure}

\begin{figure}
  \begin{center}
    \begin{minipage}{\linewidth}
      \includegraphics*[width=\linewidth, trim=10mm 4mm 2mm 5mm, clip]{kunz.eps}
    \end{minipage}
    \caption{Predicted CMF strength of Abell 1835 from model of Kunz
      \etal\ 2010 [16] which heats the ICM {\it{only}} via viscous
      dissipation of turbulent ICM motions. The Kunz \etal\ model
      takes in X-ray derived ICM density and temperature measurements
      alone and returns estimates of field strengths. These profiles
      have already been derived for over 300 clusters using the
      \chandra\ archival project, and will be compared with the
      results of the EPiCS program (\eg\ radial profiles and power
      spectra, like Fig. 1) to test how important turbulent heating is
      for an array of cluster types. This is one example of how the
      proposed CMF measurements will be directly compared with model
      predictions to aide theorists in refining models for ICM
      heating.}
  \end{center}
\end{figure}

\end{document}

%%%%%%%%%%%%%%%%%
%%%%%%%%%%%%%%%%%

``This result implies that the orientation of atomic filaments can
provide a local measure of the magnetic field direction in
clusters. It also provides a physical explanation for the filamentary
structures seen in optical emission-line observations of cluster cores
(Conselice et al. 2001; Sparks et al. 2004).  The filamentary
structure in the cold gas is also imprinted on the diffuse
X-ray-emitting plasma in the hot ICM (e.g., Figure 4). Because of the
large conductivity of the hot plasma (Equation (11)), it is natural
for a given magnetic field line to become relatively isothermal. If
different magnetic field lines undergo slightly different
heating/cooling, as must surely be the case to some extent, this will
lead to different temperatures, densities, and X-ray emissivities
along different magnetic field lines. This could potentially explain
the long, soft X-ray- emitting isothermal structures observed in some
clusters (Sun et al. 2010).''

``However, the observed atomic (e.g., Halpha) filaments are much
longer than this. This can be explained if the filaments are supported
by cosmic-ray (or some other form of isotropic non-thermal pressure,
e.g., due to small-scale magnetic fields) pressure which prevents the
collapse of the cold gas (see Figures 8 and 12). The presence of a
significant population of cosmic rays is also inferred by modeling the
atomic and molecular lines from clusters (Ferland et al. 2009).''

CMFs provide non-thermal pressure support to the ICM, as well as alter
transport processes like conduction, turbulence, and cosmic ray
diffusion.

%% -- what specific questions am I the ideal person to answer?
%% -- define a clear set of goals (I want to ABC!)




- is high conductivity correlated with halpha lum?
  + if so, suggests that conduction may be responsible for heating not
  just filaments, but much of the cool, multiphase has which is likely
  to fuel AGN activity (tada, the fine-tuning issue in the AGN
  feedback loop has a potential answer).
- what are the turbulent properties of the icm?
  + cannot be directly measured, only constrained (see sanders)
  + astro-h may be helpful, but spatial res is lacking
  + no ixo
  + so must resort to models to place constraints
- does any turb prop correlate with mag prop or filament prop?
  + if so, does it really look like filaments thrive in low-turb, high-b environs?
  + does this imply mhd processes are suppressed/inactive in some clusters? 
  + can any constraints be placed on the mechanism which forms filaments?
    (i) by channeling the infiow of clumps of gas along field lines,
        giving them coherent structure (Fabian et al. 2003)
    (ii) preventing the hot, turbulent ICM from shredding the filaments
         (Hatch et al. 2007)
    (iii) to help the growth of thermal instabilities, leading to thin
          high-density filaments (Hattori et al. 1995)
    (iv) suppressing conduction of heat from the ICM to the cooling gas
         (Voigt \& Fabian 2004).
- are filament morphs consistent with mag strn/struc preds of mhd models?
  + if so, are these models consistent with b-fiel measurements?
- with mag props in-hand, can I say anything about...
  + origin of fields?
  + importance of non-therm press support in mass estimates?

%% -- introduce innovative ideas

cmf ``maps'':
  + get magnitude from rm
  + get power spec from modeling
  + get ori from cpe
  + get small scale from filaments
  + combine into a map via bayseian method
use filaments as tracers of local mag fields, make comparisons to
global fields, how do they differ, are they similar?
advances in low freq data redux/analysis:
  + automation of radio data redux
  + addition of rfix to radio data redux
  + use of compressed sensing sparse sampling agols for automated pt
  and ext src det/extrac


`` Observationally, one of the most useful instruments for exploring
the physics of turbulence and convection in clusters would be a
high-spectral-resolution imaging X-ray spectrometer (calorimeter),
which could directly measure turbulent velocities in the ICM.''

UM-Mad has been at the forefront of this endeavor w/ S. Heinz leading
the way (give examples of work and code)

``Another area of active research involving plasma physics that will
be important for ad- dressing the cooling-fiow problem during the
coming decade is the interaction between jets and the ICM. In recent
years, increasingly high-resolution numerical simulations have helped
to elucidate different aspects of this interaction, including the
extent to which jets can ``drill through'' the ICM and the escape of
cosmic rays from cosmic-ray bubbles [26, 28, 29]. Al- though
anisotropic diffusion of cosmic rays has been included in recent
simulations [26], the inclusion of anisotropic viscosity will be
important for determining the rate at which cosmic- ray bubbles break
up,2 while the inclusion of anisotropic conductivity will be essential
for determining the extent of turbulent mixing in the ICM.''

-- direct measurement of icmmag field through (rm, polarization)
   +``Information on the intracluster magnetic fields can be obtained,
   in conjunction with X-ray observations of the hot gas, through the
   analysis of the rotation measure (RM) of radio galaxies in the
   background or in the galaxy clusters themselves. I will present a
   work aimed to establish a possible connection between the magnetic
   field strength and the gas temperature of the intracluster
   medium. For this purpose we investigated the RM in hot galaxy
   clusters and we compared these new data with RM information present
   in the literature for cooler galaxy clusters.''
-- look at whole zoology of properties, make constraints that way
-- no coming X-ray mission (IXO is dead), chandra and xmm archives
brimming... time to compile everything!
-- conduction alters radial props of temp, dens, and entropy
-- re-analyze mass relations using terms for non-therm p support
-- close correlation between agn feedback, star formation rates,
presence of multiphase gas, and state of hot icm
-- presence of multiphase gas directly linked to process of
conduction, and hence relates to magnetic field properties
-- relationship between the morphology of cold filaments in cool core
clusters and the structure of the icm magnetic field
-- magnetic fields important in for non-therm emission like radio
halos and cosmic ray emission
-- information about agn duty cycles relevant to understanding the
impact of intermittent agn outbursts on the evolution of icm mag
fields... infrequent stirring

Interesting observational constraints:
  -- halpha emission and morph
  -- radio halos
  -- cold fronts
  -- rotation measures
  -- polarizaton when possible

polarization at low frequencies:
-- see larger strucs at low freq because Alfven propogation speed is
higher and sync lifetimes are longer
-- measure Stokes' parameters to measure poln
-- p \propto B^2
-- poln gives direct measure of regular mag field along LOS
-- faraday rotation: rm \propto \nelec B
-- need poln measure at multiple freqs, slope gives RM
-- need bgd srcs
-- delta(rm) \approx 0.1 rad m^2 at 210-240 MHz for SN=10
-- rm synthesis \citep[\eg][]{2005A&A...441.1217B, 2010arXiv1008.3530P}:
   multiple srcs undergo multiple faraday rotations
   measure phi as a func of lambda^2
   disentangle multiple syn and rota components via fourier trans
   see brentjens & de Bruyn 05
   measure complex poln surbri as a fnc of lambda
   get spec in faraday depth
   rm syn is very similar to interfer
-- ``in the presence of polarized cluster radio sources, RM-synthesis
   is the key technique to unveil the 3-D structure of galaxy
   clusters''
-- ``reveal the origin of the observed depolarization of the cluster
sources towards low frequencies. By reducing, or even eliminating, the
importance of beam de- polarization, high resolution low-frequency
observations could test whether the depolarization is occurring
internally to the sources or in a foreground screen''
-- ``On a general point of view, the improved knowledge on the
non-thermal phenomena of the ICM will have an impact on
cosmology. If halos and relics are related to cluster mergers, the
study of the statistical properties of these sources will allow us to
test the current cluster formation scenario, giving hints on detailed
(astro)physics of large-scale structure formation (e.g. Evrard & Gioia
2002)''

``The study of polarization properties of cluster and background radio
sources is crucial to analyze the intracluster medium magnetic field
properties. Its intensity, radial decline and power spectrum can be
inferred by studying the Faraday rotation of several sources located
inside or behind the cluster, since Faraday rotation is sensitive to
the local magnetic field strength and structure.''

``The preliminary results that we obtained with these data can be
summarized as follows: 1) the Faraday Rotation measure shows a large
decrease going from the center to the periphery of the cluster. We
have obtained mean values ranging from ~136 rad/m/m to 20rad/m/m. Non
zero values of Faraday Rotation mean values indicate that the magnetic
field fluctuates on scales larger than the source's extension. 2)
Faraday rotation images are patchy, indicating that the magnetic field
fluctuates on scales smaller than the source's size; 3) the power
spectrum (|B_k|^2 ~ Lambda^n) maximum fluctuation scale is in the
range 30-250 kpc; 4) the magnetic field central intensity is in the
range 5.5 - 8 microG, the best fit is now achieved with central
intensity of 8 microG (see Fig 1) 5) the average magnetic field
intensity over the cluster volume, ~ 1 Mpc^3, is ~1.2 microG in good
agreement with the estimate derived from the radio halo emission, that
is in the range 0.7-1.9 microG (Thierbach et al. 2003).} Further
  observations have been requested to refine these estimates. The high
  value of the central magnetic field could be a feature of high mass
  system, possibly as a consequence of the magnetic field
  amplification expected in merger events.''

When emission from a radio source traverses a magnetic field, a change
in polarization angle is induced, \ie\ the EM wave undergoes Faraday
rotation. Measurement of XXX, or RM) is a widely accepted method for
coursely sampling the {\it{strength}} of CMFs
\citep{2008SSRv..134...93F}. The novel technique of Pfrommer \& Dursi
2010 \cite{2010NatPh...6..520P} exploits the coherent polarized
emission (CPE) induced by magnetic draping of cluster member galaxies
to infer CMF {\it{orientations}}. The CPE technique was applied to
member galaxies of the Virgo cluster revealing that Virgo's ICM
magnetic fields lines are preferentially radial, consistent with the
effects of the MTI. Combined, RM and CPE reveal the strength and
orientation of CMFs. The completed \evla\ upgrade has significantly
increased the sensitivity and frequency coverage of the \vla, enabling
several times more background sources per cluster to be detected,
expanding the number of RM probes, and hence enabling more detailed
measurements of CMFs.

These measurements will then be used to model the magnetic field power
spectrum \citep[\eg][]{2010A&A...513A..30B} and. This will provide
constraints on...XXX? The requested observations will also be
sufficient to measure CPE of cluster members and infer the field
orientations. These measurements will be compared with the CMF
configurations predicted by MHD models. The aim of this leg of the
proposal is to establish how CMF properties differ for clusters in
different evolutionary stages and allow searches for correlations
between the magnetic field and cluster properties.

For example, the CPE method has already been used to demonstrate the
large-scale CMF of Virgo is radially oriented, consistent with field
structures expected to arise from the influence of the magnetothermal
instability \citep{2010NatPh...6..520P}.

     + AIPS and CASA scripts handling the reduction workload
     + RM and synthesis measures will require hand analysis
     + indeterminant how many sources will have these
  -- filaments survey will require halpha obs for a huge number of cds
  and bcgs using new instrs on wiyn

The \evla\ upgrade allows for the detection of several times more
background/embedded radio sources per cluster, each of which is useful
for constraining the line-of-sight ICM magnetic field strength via
rotation measure (RM) analysis.

Turbulence appears to have the critical role of mixing the ICM and
suppressing the formation of both convective and thermal instabilities
in cluster cores, allowing the gas to be heated. But, directly
measuring ICM turbulent velocities requires the high-spatial and
-energetic resolution of a microcalorimeter, which is planned for
Astro-H and IXO (which may never be realized due to funding
restrictions). Regardless, theoretical models do predict turbulence
will imprint on CMFs and subsequently influence the properties of the
ICM. For example, the model of Kunz...

1) RM used to estimate CMF radial structure and model power spectrum,
all of which can be compared with predictions from mhd sims and turb
models; $RM \propto \int \rho B_{\|}$

2) CPE gives estimate of cmf orientation which can also be compared with
predictions of models with buoy instab

3) Halos used to constrain global properties of CMF

And given the bleak observational limitations, there isn't much other
choice. So, I suggest calculating turbulent ICM values using Kunz
model, compiling secondary diagnostics, and comparing with the
measured magnetic field and ICM thermal/non-thermal properties. It may
emerge that AGN jet power correlates with small turbulent length
scales, or that the magnetic field strengths predicted by the model
are consistent with the observations. Regardless of the results,

The luminosities and morphologies of \halpha\ filaments give an
estimate of how much conductive heating is taking place and the
strengths of the fields supporting the filaments.

For a few clusters, the measurements of CMF magnitude, direction,
spatial distribution, and line-of-sight distribution should be robust
enough to allow at least 2D, and possible 3D, maps to be generated. If
so, this will be the first time such maps have been created from
observations alone.

ICM thermal properties are directly observable via X-ray emission, and
CMF properties are inferred from radio observations of synchrotron
emission. Hence, this work relies heavily on archival X-ray \& radio
data, and new datasets to perform innovative analysis of optical and
polarized radio emission.

The radio, optical, and archival projects have the added benefit of
complimenting the science goals of the \evla\ Deep Cluster
Survey\footnote{http://www.atnf.csiro.au/people/Shea.Brown/cosweb/},
and the \lofar\ Surveys and Cosmic Magnetism Key Science
Projects\footnote{http://www.lofar.org/astronomy/key-science/lofar-key-science-projects}.

For cluster cores where the turbulent length scale is longer than the
Field Length, 

How the filaments form, be it fragmentation along field lines, CMF
promotion of thermal instability growth, inhibition of turbulent
shredding, or heat conduction suppression, will also be investigated.

Consequently, it is now possible to further evaluate thermodynamic
theories of the ICM by obtaining better measurements of CMF properties
for large samples of clusters and establishing how CMF and ICM
properties correlate as a function of

Similarly, a radio reduction pipeline has been developed and
was run on 72 \vla\ datasets to complete the work published in
Cavagnolo \etal\ 2010 \citep{pjet}.

``The galaxy cluster models so obtained are the first world-wide to
contain realistic magnetic fields. They serve as the basis for further
investigations into the role of magnetic fields in the process of
cluster formation. Some of the processes going on in galaxy clusters
can only be understood if magnetic fields are taken into
account. Here, too, our models can be used to construct a more
complete picture of the physical processes in these largest bound
objects in the universe.''

by an amount dependent on ICM density, line-of-sight CMF strength, and
ICM pathlength traversed: $RM \propto \int \nelec B_{\|}
~{\rm{d}}l$. For the ICM, RMs are strongest at radio wavelengths,
thus, multifrequency radio polarimetry of background and embedded
cluster radio sources, together with ICM X-ray observations, allow
estimation of CMF strengths.

CMF strengths are constrained by measuring the magnitude of Faraday
rotation (RM) induced on the intrinsic polarization of radio waves
traversing the ICM, their orientations can be infered using coherent
polarized emission

But, again, the observational constraints are lacking to determine
which processes are important.

synchrotron emission = field strength
polarization = field orientation and degree of ordering
Faraday rotation = integrated line of sight strength
