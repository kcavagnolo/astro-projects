\documentclass[letterpaper,12pt]{article}
\usepackage[numbers,sort&compress]{natbib}
\bibliographystyle{unsrt}
\usepackage{common,graphicx}
\pagestyle{myheadings}
\setlength{\textwidth}{7.2in} 
\setlength{\textheight}{9.3in}
\setlength{\topmargin}{-0.3in} 
\setlength{\oddsidemargin}{-0.35in}
\setlength{\evensidemargin}{0in} 
\setlength{\headheight}{0in}
\setlength{\headsep}{0.3in} 
\setlength{\hoffset}{0in}
\setlength{\voffset}{0in}
\newcommand{\myhead}{Cavagnolo, Research Summary}
\begin{document}

\begin{center}
  {\bf\uppercase{Summary of Past and On-going Research}}
\end{center}

\subsubsection*{I. Intracluster Medium (ICM) Temperature Inhomogeneity}

If simple galaxy cluster observables such as temperature or luminosity
are to serve as accurate mass proxies, how mergers alter these
observables needs to be quantified. It is known that cluster
substructure correlates well with dynamical state, and that the
apparently most relaxed clusters have the smallest deviations from
mean mass-observable relations [\eg\ 1]. If a cluster's ICM is nearly
isothermal in the projected region of interest, the X-ray temperature
inferred from a broadband (0.7-7.0 keV) spectrum should be identical
to the X-ray temperature inferred from a hard-band (2.0-7.0 keV)
spectrum. However, if there are unresolved, cool lumps of gas, the
temperature of a single-component thermal model will be cooler in the
broadband versus the hard-band. This difference is then possibly a
diagnostic to indicate the presence of cooler gas, \eg\ associated
with merging sub-clusters, even when the X-ray spectrum itself may not
have sufficient signal-to-noise to resolve multiple temperature
components [2]. In Cavagnolo \etal\ 2008 [3] we studied this
temperature band dependence for 192 clusters taken from the
\chandra\ Data Archive. We found, on average, that the hard-band
temperature was significantly higher than the broadband temperature,
and that their ratio was preferentially larger for known mergers. We
interpret this result to mean that, indeed, ICM temperature
inhomogeneity is detectable via a simple bandpass comparison and, on
average, it correlates with cluster dynamical state. Our results
suggest such a temperature diagnostic may be useful when designing
metrics to minimize the scatter about mean mass-scaling relations.

\subsubsection*{II. ICM Entropy and Active Galactic Nucleus (AGN) Feedback}

ICM temperature and density mostly reflect the shape and depth of a
cluster's dark matter potential, while the specific entropy governs
the density at a given pressure [4]. The ICM is convectively stable
when the lowest gas entropy occupies the bottom of the cluster
potential and the highest entropy gas has buoyantly risen to larger
radii. Further, ICM entropy is primarily changed through heat
exchange. Thus, deviations of the ICM entropy structure from a pure
cooling azimuthally symmetric, radial power-law is useful in
evaluating a cluster's thermodynamic history [5]. One such use of ICM
entropy is studying energetic feedback on the cluster environment
[6]. In Cavagnolo \etal\ 2009 [7], the ICM entropy structure of 239
clusters taken from the \chandra\ Data Archive were studied. We found
that most clusters have entropy profiles which are well-fit by a model
which is a power-law at large radii and approaches a constant entropy
value at small radii: $K(r) = \kna + \khun (r/100 ~\kpc)^{\alpha}$,
where \kna\ quantifies the excess of core entropy above the best
fitting power-law at larger radii and \khun\ is the entropy
normalization at 100 kpc. Our results are consistent with models which
predict cooling of a cluster's X-ray halo is offset by energy injected
via feedback from AGN [\eg\ 6]. We also showed that the distribution
of \kna\ values in our archival sample is bimodal, with a distinct gap
around $\kna \approx 40 ~\ent$.

If cluster halo cooling triggers AGN feedback, then certain ICM
properties (\eg\ entropy) may correlate tightly with signatures of
feedback and/or indicators of cooling. In Cavagnolo \etal\ 2008 [8] we
explored the relationship between cluster core \kna\ values and
presence of \halpha\ \& radio emission. We found that \halpha\ and
radio emission are almost strictly associated with \kna\ values less
than $30 ~\ent$, which is near the gap of the bimodal
\kna\ distribution. The prevalence of \halpha\ emission below this
threshold indicates it marks a dichotomy between cluster cores that
can harbor thermal instabilities and those that cannot. The fact that
strong AGN activity appears below this boundary suggests that feedback
turns on when the ICM starts to condense, strengthening the case for
AGN feedback as the mechanism that limits star formation in the
Universe's most luminous galaxies. In Voit \etal\ 2008 [9], we go on
to suggest that core entropy bimodality and the sharp entropy
threshold arises from the influence of thermal conduction. This result
is one of the key motivating factors for the fellowship proposal that
follows, and is discussed in more detail therein.

\subsubsection*{III. Properties of AGN Jets}

A long-standing problem in observational and theoretical studies of
AGN energetics is estimating their total kinetic energy output. These
estimates have historically been made using jet models built around
first principles and observations of how it {\it{appears}} jets
interact with their surroundings [\eg\ 10]. However, X-ray
observations of clusters have revealed that AGN outflows inflate
cavities in the ICM, and these cavities yield a direct measure of the
work, and hence total mechanical energy, exerted by the AGN on its
environment [see 11, for details]. Hence, correlations between derived
cavity power and associated synchrotron radio power yields a useful
device for constraining total AGN energy output when X-ray data or
cavities are unavailable. Such relations between jet power (\pjet) and
radio power (\prad) for clusters were presented by
\birzan\ \etal\ 2004, 2008 [12, 13]. In Cavagnolo \etal\ 2010 [14] we
appended a sample of 13 giant ellipticals (gEs) and found that the
\pjet-\prad\ relation is continuous, and has similar scatter, from
clusters down to gEs. We also found that, independent of frequency,
\pjet\ scales as $\sim \prad^{0.7}$ with a normalization of $\sim
10^{43}$ erg s$^{-1}$. Numerous jet models predict a power-law index
of $\approx 12/17$, consistent with our results, and normalizations of
$\sim 10^{43} ~\lum$ when the ratio of non-radiating particles to
relativistic electrons is $\ga 20$ (\ie\ moderately heavy jets). Our
results imply that there does exist a universal scaling relation
between jet power and radio power, which would be a useful tool for
calculating AGN kinetic output over huge swathes of cosmic time using
only all-sky, monochromatic radio surveys.

\markright{\myhead}
\subsubsection*{IV. Radio-mode and Quasar-mode Feedback}

Galaxy formation models typically segregate AGN feedback into an
early-time, radiatively-dominated quasar mode [\eg\ 15] and a
late-time, mechanically-dominated radio mode [\eg\ 16]. In
quasar-mode, nucleus radiation couples to gas within the host galaxy
and drives strong winds depriving the supermassive black hole (SMBH)
of additional fuel, regulating black hole mass growth. At later times,
during sub-Eddington accretion, SMBH launched jets regulate galaxy
mass growth through prolonged and intermittent mechanical heating of a
galaxy's gaseous halo. Though AGN feedback models are broken into two
generic modes, they still form a unified schema [\eg\ 17]. However,
the association of the modes, and whether they interact, is still
poorly understood. In Cavagnolo \etal\ 2010 [18] we present a study of
the galaxy IRAS 09104+4109 (IRAS09) which simultaneous exhibits all
the characteristics of a system in radio- and quasar-mode of feedback,
perhaps implying it is a ``transition'' object cycling between the
modes. A joint X-ray/radio analysis of IRAS09 reveals cavities in the
galaxy's X-ray halo associated with an AGN outflow having $\sim
10^{44} ~\lum$ of mechanical energy and an obscured nuclear quasar
with a luminosity of $\sim 10^{47} ~\lum$. We directly measure, for
the first time, that the radiative to mechanical feedback energy ratio
for a ``transition'' object is $\sim$1000:1. Further, the cavities
contain enough energy to offset $\approx 25-35\%$ of the host
cluster's ICM radiative cooling losses. However, how this energy is
thermalized remains unknown -- which is one aspect of the fellowship
proposal which follows. Nonetheless, our results suggest 3--4 similar
strength AGN outbursts are sufficient to suppress ICM core cooling and
freeze-out rapid BCG star formation. We also unambiguously demonstrate
that the beaming directions of the jets and nuclear radiation are
indeed misaligned, as previous studies have suggested. Our
interpretations of these findings are that IRAS09 may be providing a
local example of how the AGN feedback cycle of massive galaxies at
higher redshifts evolves, and may also be offering clues as to how the
evolution of black hole spin is closely correlated with the feedback
cycle.

\subsubsection*{V. Black Hole Spin}

While there is direct evidence that halo cooling and late-time AGN
feedback are linked [\eg\ 8], the observational constraints on how AGN
are fueled and powered remain loose. For example, what fraction of the
energy released in an AGN outburst is from gravitational binding
energy of accreting matter [19] or the SMBH's rotational energy [20]
is still unclear. Mass accretion alone can, in principle, fuel most
AGN [\eg\ 21]. However, there are gas-poor systems which host very
powerful AGN (energies $> 10^{61}$ erg) where mass accretion alone
appears unlikely as a power source. In these systems, either the AGN
fueling was astoundingly efficient, or the power came from an
alternate source, such as the release of angular momentum stored in a
rapidly spinning SMBH [\eg\ 22]. If more systems are found which are
best explained as being spin powered, we may need to incorporate a
spin feedback pathway into galaxy formation models. In Cavagnolo
\etal\ 2010 [23], we present analysis of the AGN outburst in the
galaxy cluster RBS 797 and, because of the extreme energetic demands
of the outburst (total energy output and jet power of the order
$10^{61}$ erg and $10^{46} ~\lum$, respectively), we suggest it may
have been powered by a rapidly spinning SMBH. The model of Garofalo
\etal\ 2010 [24] suggests that the evolution of a black hole's spin
state is closely tied to the process of AGN feedback. In their model,
retrograde accretion induced spin-down forces a black hole through a
state where the spin is $\approx 0$. At this point, a rapid asymmetric
accretion flow can drastically and quickly reorient a spin axis. This
process is the focus of Cavagnolo \etal\ 2010 [25] as it can possibly
give rise to the type of beamed jet-radiation misalignment observed in
IRAS09, and can also result in the extraction of extreme jet powers
like in RBS 797.

\markright{\myhead}
\input{short_summary.bbl}

\end{document}
