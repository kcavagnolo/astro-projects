\documentclass[letterpaper,12pt]{article}
\usepackage[numbers,sort&compress]{natbib}
\bibliographystyle{unsrt}
\usepackage{common,graphicx}
\pagestyle{myheadings}
\setlength{\textwidth}{7.2in} 
\setlength{\textheight}{9.3in}
\setlength{\topmargin}{-0.3in} 
\setlength{\oddsidemargin}{-0.35in}
\setlength{\evensidemargin}{0in} 
\setlength{\headheight}{0in}
\setlength{\headsep}{0.3in} 
\setlength{\hoffset}{0in}
\setlength{\voffset}{0in}
\newcommand{\myhead}{Cavagnolo, Jansky Proposal}
\newcommand{\frm}{{\it{RM}}}
\begin{document}

\begin{center}
  {\bf\uppercase{mapping galaxy cluster magnetic fields:\\an
      observational study of icm physics}}
\end{center}

\citep{2009IEEEP..97.1448P}

\noindent{\bf{I. Motivation}}\\
\indent Clusters of galaxies are the largest structures in the
Universe to have reached dynamic equilibrium, and most cluster
baryonic mass resides in the intracluster medium (ICM), a hot, dilute,
weakly magnetized plasma filling the cluster volume
\citep{sarazinbook}. As the defining characteristic of the most
massive objects in the Universe, the thermal properties of the ICM are
well-known, but a similarly detailed description of ICM non-thermal
properties -- specifically diffuse cluster magnetic fields (CMFs) --
and how they relate to the thermodynamic nature of the ICM remains
elusive. Filling this gap in knowledge is vital because clusters help
us constrain cosmological parameters \citep{voitreview}, develop
hierarchical structure formation models \citep{1995MNRAS.275...56N},
and study the synergy of many physical processes to answer the
question, ``How does the Universe work?''
\citep{2004cgpc.conf.....M}.

At present, one of the biggest challenges in cluster studies is
explaining the relative thermal equilibrium of the ICM. Many clusters
have core ICM cooling times much less than a Hubble time, and it was
hypothesized that these systems should host prodigious ``cooling
flows'' \citep{fabiancfreview}. But, only minimal mass deposition
rates and cooling by-products have ever been detected, requiring that
the ICM be heated \citep{cfreview}. Observational and theoretical
studies have strongly implicated feedback from active galactic nuclei
(AGN) in supplying the {\it{energy}} needed to regulate ICM cooling
and late-time galaxy growth \citep{mcnamrev}. However, precisely how
AGN feedback energy is thermalized and which processes comprise a
complete AGN feedback loop remain to be fully understood
\citep{2010ApJ...710..743D}.

Theoretical studies are now focused on coupling AGN feedback and ICM
heating using combinations of anisotropic thermal conduction, cosmic
ray diffusion, and subsonic turbulence
\citep[\eg][]{2006MNRAS.373L..65H, conduction, 2008ApJ...688..859G,
  2009ApJ...699..348S, 2009ApJ...704..211B, 2010ApJ...712L.194P,
  2010ApJ...713.1332R, 2010arXiv1003.2719K, 2010arXiv1010.2277R} after
observations suggested the ICM is turbulent and conducting on small
scales \citep[\eg][]{2004MNRAS.347.1130V, haradent,
  2010MNRAS.407.2046M}. These microphysical processes are
intrinsically linked to macroscopic CMF topologies through gas
viscosity and magnetohydrodynamic (MHD) instabilities
\citep{2000ApJ...534..420B, 2008ApJ...673..758Q}. Thus, to
observationally test and refine this theoretical framework, the
strength \& structure of CMFs must be measured in detail for large
samples of clusters spanning a broad-range of evolutionary \&
dynamical states and then compared with model predictions (see
Figs. 1--3 for individual examples). Unfortunately, measuring CMF
properties is notoriously difficult
\citep[\eg][]{2001ApJ...547L.111C}, which has limited our knowledge of
CMF demographics to a handful of clusters
\citep[\eg][]{2010arXiv1007.5207G}. Additionally, there is no
consensus on CMF origins and how much ICM pressure support they
provide \citep{2001PhR...348..163G}.

The \evla\ radio observatory will change this situation by providing
the unprecedented sensitivity, resolution, and frequency coverage
required to detect the faint ICM synchrotron emission and numerous,
weak radio sources needed to probe CMFs
\citep{2008SSRv..134...93F}. {\bf{As a Jansky fellow, I propose to use
    radio polarimetry in conjunction with optical \& X-ray imaging to
    map CMFs (magnitudes, orientations, power spectra, 3D structure)
    and evaluate their relationship with ICM thermal properties
    (\eg\ temperature, entropy, pressure).}} This work will 1)
determine which microphysical processes significantly contribute to
heating of the ICM by directly comparing the predictions of
theoretical models with CMF observations, and 2) place constraints on
the origin of CMFs and the cosmological implications of non-thermal
pressure support on cluster mass estimates. The proposed project
includes plans for an \evla\ radio survey and NOAO optical
\halpha\ survey of two well-studied cluster samples, and incorporates
an on-going pipeline analysis of an archive-limited sample of clusters
having radio and X-ray data.\\

\noindent{\bf{II. Observations and Analysis}}\\
\indent The \evla\ Polarimetry Cluster Survey (EPiCS) will target the
flux-limited HIFLUGCS \citep{hiflugcs1} and representative REXCESS
\citep{rexcess} cluster samples for which uniform \chandra\ and
\xmm\ X-ray data is available. EPiCS will utilize the increased
polarimetry bandwidth and frequency accessibility of \evla\ to obtain
uniform, deep ($\sigma_{\rm{rms}} \la 10 ~\mu$Jy beam$^{-1}$) full
Stokes continuum observations of each cluster. The observations will
enable measurements of: 1) rotation measures (RM) of embedded \&
background radio sources \citep[see][for method]{2010A&A...513A..30B},
2) coherent polarized emission (CPE) from orbiting cluster member
galaxies \citep[see][for method]{2010NatPh...6..520P}, and 3)
low-surface brightness polarized emission associated with synchrotron
halos \citep[see][for method]{2010ApJ...722..737K}. Combined with the
archival X-ray data for each source, the following outstanding issues
regarding the relation between CMFs and the ICM will be investigated.

{\bf{A. Testing Models of ICM Heating:}} The EPiCS campaign will
produce data of sufficient quality to estimate CMF radial amplitude
profiles, directly reconstruct CMF power spectra, and model 3D CMF
structure using RM synthesis \citep[methods in][]{2003A&A...412..373V,
  2004A&A...424..429M, 2005A&A...441.1217B}. Numerous MHD models make
predictions regarding the CMF strengths and configurations (\eg\ being
preferentially radial, declining in strength with radius) and each of
the above properties will be directly compared with these models to
determine which input physics are replicated and which are
incomplete. This will help address which microphysical processes
participate in heating the ICM. Since AGN feedback is the likely
progenitor of heating, I will also investigate possible correlations
between CMF properties and signatures of feedback and heating such as
the cool/non-cool core dichotomy, jet powers for systems with
cavities, 2D thermal distributions, and central AGN activity. Further,
turbulence is considered vital for promoting ICM heating, but is
difficult to directly measure. However, secondary diagnostics
(\eg\ AGN outflows, mergers, cold fronts, shocks) may indicate the
presence of turbulence even when the data is insufficient to do
so. These indicators will be considered during the analysis to check
if trends exist with CMF \& ICM properties.

{\bf{B. CMFs in Cluster Cores:}} It is hypothesized that the
\halpha\ filaments seen in almost all cool core clusters provide a
local measure of CMF strength and orientation since they may form
along field lines and be excited by some combination of turbulent
mixing and conduction \citep{2010ApJ...720..652S,
  2010MNRAS.407.2063W}. To probe CMF configurations and conductive
heating on scales of tens of kpc, below the reach of the radio
observations, a uniform optical survey for extended \halpha\ filaments
in the EPiCS cluster samples will be undertaken using new NOAO
instruments (\ie\ Magellan Maryland Tunable Filter, WIYN HiRes IR
Camera, SOAR Spartan IR Camera) \citep[see][for
  method]{mcdonald10}. The observations will allow, for the first
time, a complete characterization of filament morphologies and
energetics to be compared with uniform ICM and CMF properties for the
same objects. These observations will confront model predictions by
answering the question, ``Are filament energetics and morphologies
consistent with magnetic structures being conductively heated by the
ICM?''  Combined with the radio-derived CMF properties, inferences
will also be drawn about if, and possibly how, large- and small-scale
CMF properties are related (\eg\ the coherence length). The model
comparisons from Section A will also answer the questions: do
filaments thrive in low-turbulence, high-magnetic field strength
environs? Does this imply MHD instabilities are suppressed or inactive
in some cluster cores?

{\bf{C. Constraining CMF Origins and Non-thermal Pressure Support:}}
Simply put, where do CMFs come from, and are they dynamically
important? Just like ICM heating, this question has been addressed
extensively in the theoretical domain, but poorly observationally
because we lack samples of CMF measurements. The most prevalent
hypotheses for CMF origins are a cosmic seed field amplified as
structure forms, the Biermann battery process, and protocluster
seeding by AGN or galactic outflows \citep{2002ARA&A..40..319C}. The
CMF strength \& structure measurements determined in this project are
relevant to tackling these issues. As the quantities most closely
related to dynamo-driven CMF formation, I will investigate how
redshift, halo concentration, and cluster mass relate to the derived
CMF power spectra and radial profiles \citep{2009MNRAS.392.1008D}. At
a minimum, these comparisons will place limits on the strength and
distribution of allowable seed field models, and may possibly suggest
the extent to which dynamo processes dominate over secondary processes
like late-time turbulent amplification
\citep{2002A&A...387..383D}. Deriving halo concentrations and cluster
masses follow directly from the X-ray analysis already
in-hand. However, cluster masses are traditionally derived by assuming
the ICM is in hydrostatic equilibrium. If CMFs provide a significant
amount of this support, then cluster mass estimates may be
overestimated -- though lensing and X-ray derived masses tend to agree
-- which would have interesting repercussions on cluster cosmological
studies. Thus, cluster masses and the cluster mass function will be
recalculated \citep[\eg][]{2009ApJ...692.1060V} including terms for
CMF pressure support determined from the EPiCS measurements. How
cosmological parameter uncertainties depend on CMFs can then be
determined. This exercise will be particularly interesting for the
REXCESS sample which has high-quality hydrostatic mass estimates
\citep{2010A&A...517A..92A}.

{\bf{D. Tool and Software Development:}} Work has started on archival
\chandra\ and \vla\ data to build the infrastructure needed to
maximize the ultimate scientific impact of this project and produce
initial results for an archive-limited sample of clusters. There are
$\approx 450$ clusters which have archival \chandra\ ($\approx 900$
observations) and \vla\ ($\approx 1000$ observations) data. Of these,
325 clusters have had the X-ray data reduced using an extensible and
mature pipeline, while 50 of those clusters have had the
multifrequency radio data reduced. The X-ray results are being kept in
a public database\footnote{http://www.pa.msu.edu/astro/MC2/accept/}
while the radio analysis continues. The on-going analysis entails
production of 2D ICM temperature, density, pressure, \& entropy maps,
more radial profiles (\eg\ effective conductivity, implied suppression
factors), and refinement of the radio reduction pipeline. Removal of
radio frequency interference (RFI) is among the lengthiest steps in
radio analysis, and to allevaite this tension, a python version of the
`RfiX' rejection algorithm \citep{rfix} has been written and is being
tested. To widen this proposal's impact, all code, software, and
results produced will be made freely available to the research
community.\\

\noindent{\bf{III. Host Institution and Timeline}}\markright{\myhead}\\
\indent The University of Wisconsin-Madison (UW) is an ideal host for
this project. The proposed work is ambitious, requiring a team of
observational \& theoretical experts to interpret the data, distribute
results, and propose new projects motivated by this work. The UW
Astronomy and Physics Departments, Center for Plasma Theory \&
Computation, and the Center for Magnetic Self-Organization are hosts
to (to name but a few) Sebastian Heinz (the sponsor), Alex Lazarian,
Leonid Malyshkin, Dan McCammon, Eric Wilcots, and Ellen Zweibel, all
of whom are experts in one, or several, of the areas of AGN feedback,
computational modeling, magnetic field polarimetry, plasma physics,
and X-ray instrumentation. At UW I will have access to this broad
community of experts whom can provide invaluable expertise in
evaluating and interpreting the observational results. UW is also part
of the Great Lakes network of institutions (\eg\ MSU, UM, OSU, UMinn,
UChicago) which have groups actively involved in the topics of this
proposal. And it cannot be ignored that UW and MSU have guaranteed
time on the WIYN and SOAR telescopes, which will be used over the
course of this project.

In year one of the fellowship: radio and optical data acquisition will
begin, the archival project will continue, and tool development will
proceed. I will initiate a collaboration with the plasma physics group
and help determine the best strategy for executing new simulations to
probe questions like: How do convective instabilities couple with ICM
cooling and the actual accretion which drives AGN activity? What is
the relation between these processes and the ICM temperature and
density profiles? In year two: data acquisition will continue, the
first round of results using archival data will be published, and
comparisons between observations and model predictions will begin. In
year three, data acquisition and analysis for the REXCESS sample will
conclude, a second round of results will be published, and the
investigation of CMF origins and non-thermal pressure support will be
underway.

%%%%%%%%%%%
% Figures %
%%%%%%%%%%%

\begin{figure}
  \begin{center}
    \begin{minipage}{\linewidth}
      \includegraphics*[width=\textwidth, trim=0mm 0mm 0mm 0mm, clip]{rbs797.ps}
    \end{minipage}
    \caption{Fluxed, unsmoothed 0.7--2.0 keV clean image of \rbs\ in
      units of ph \pcmsq\ \ps\ pix$^{-1}$. Image is $\approx 250$ kpc
      on a side and coordinates are J2000 epoch. Black contours in the
      nucleus are 2.5--9.0 keV X-ray emission of the nuclear point
      source; the outer contour approximately traces the 90\% enclosed
      energy fraction (EEF) of the \cxo\ point spread function. The
      dashed green ellipse is centered on the nuclear point source,
      encloses both cavities, and was drawn by-eye to pass through the
      X-ray ridge/rims.}
    \label{fig:img}
  \end{center}
\end{figure}

\begin{figure}
  \begin{center}
    \begin{minipage}{0.495\linewidth}
      \includegraphics*[width=\textwidth, trim=0mm 0mm 0mm 0mm, clip]{325.ps}
    \end{minipage}
   \begin{minipage}{0.495\linewidth}
      \includegraphics*[width=\textwidth, trim=0mm 0mm 0mm 0mm, clip]{8.4.ps}
   \end{minipage}
   \begin{minipage}{0.495\linewidth}
      \includegraphics*[width=\textwidth, trim=0mm 0mm 0mm 0mm, clip]{1.4.ps}
    \end{minipage}
    \begin{minipage}{0.495\linewidth}
      \includegraphics*[width=\textwidth, trim=0mm 0mm 0mm 0mm, clip]{4.8.ps}
    \end{minipage}
     \caption{Radio images of \rbs\ overlaid with black contours
       tracing ICM X-ray emission. Images are in mJy beam$^{-1}$ with
       intensity beginning at $3\sigma_{\rm{rms}}$ and ending at the
       peak flux, and are arranged by decreasing size of the
       significant, projected radio structure. X-ray contours are from
       $2.3 \times 10^{-6}$ to $1.3 \times 10^{-7}$ ph
       \pcmsq\ \ps\ pix$^{-1}$ in 12 square-root steps. {\it{Clockwise
           from top left}}: 325 MHz \vla\ A-array, 8.4 GHz
       \vla\ D-array, 4.8 GHz \vla\ A-array, and 1.4 GHz
       \vla\ A-array.}
    \label{fig:composite}
  \end{center}
\end{figure}

\begin{figure}
  \begin{center}
    \begin{minipage}{0.495\linewidth}
      \includegraphics*[width=\textwidth, trim=0mm 0mm 0mm 0mm, clip]{sub_inner.ps}
    \end{minipage}
    \begin{minipage}{0.495\linewidth}
      \includegraphics*[width=\textwidth, trim=0mm 0mm 0mm 0mm, clip]{sub_outer.ps}
    \end{minipage}
    \caption{Red text point-out regions of interest discussed in
      Section \ref{sec:cavities}. {\it{Left:}} Residual 0.3-10.0 keV
      X-ray image smoothed with $1\arcs$ Gaussian. Yellow contours are
      1.4 GHz emission (\vla\ A-array), orange contours are 4.8 GHz
      emission (\vla\ A-array), orange vector is 4.8 GHz jet axis, and
      red ellipses outline definite cavities. {\it{Bottom:}} Residual
      0.3-10.0 keV X-ray image smoothed with $3\arcs$ Gaussian. Green
      contours are 325 MHz emission (\vla\ A-array), blue contours are
      8.4 GHz emission (\vla\ D-array), and orange vector is 4.8 GHz
      jet axis.}
    \label{fig:subxray}
  \end{center}
\end{figure}

\begin{figure}
  \begin{center}
    \begin{minipage}{\linewidth}
      \includegraphics*[width=\textwidth]{r797_nhfro.eps}
      \caption{Gallery of radial ICM profiles. Vertical black dashed
        lines mark the approximate end-points of both
        cavities. Horizontal dashed line on cooling time profile marks
        age of the Universe at redshift of \rbs. For X-ray luminosity
        profile, dashed line marks \lcool, and dashed-dotted line
        marks \pcav.}
      \label{fig:gallery}
    \end{minipage}
  \end{center}
\end{figure}

\begin{figure}
  \begin{center}
    \begin{minipage}{\linewidth}
      \setlength\fboxsep{0pt}
      \setlength\fboxrule{0.5pt}
      \fbox{\includegraphics*[width=\textwidth]{cav_config.eps}}
    \end{minipage}
    \caption{Cartoon of possible cavity configurations. Arrows denote
      direction of AGN outflow, ellipses outline cavities, \rlos\ is
      line-of-sight cavity depth, and $z$ is the height of a cavity's
      center above the plane of the sky. {\it{Left:}} Cavities which
      are symmetric about the plane of the sky, have $z=0$, and are
      inflating perpendicular to the line-of-sight. {\it{Right:}}
      Cavities which are larger than left panel, have non-zero $z$,
      and are inflating along an axis close to our line-of-sight.}
    \label{fig:config}
  \end{center}
\end{figure}

\begin{figure}
  \begin{center}
    \begin{minipage}{0.495\linewidth}
      \includegraphics*[width=\textwidth, trim=25mm 0mm 40mm 10mm, clip]{edec.eps}
    \end{minipage}
    \begin{minipage}{0.495\linewidth}
      \includegraphics*[width=\textwidth, trim=25mm 0mm 40mm 10mm, clip]{wdec.eps}
    \end{minipage}
    \caption{Surface brightness decrement as a function of height
      above the plane of the sky for a variety of cavity radii. Each
      curve is labeled with the corresponding \rlos. The curves
      furthest to the left are for the minimum \rlos\ needed to
      reproduce $y_{\rm{min}}$, \ie\ the case of $z = 0$, and the
      horizontal dashed line denotes the minimum decrement for each
      cavity. {\it{Left}} Cavity E1; {\it{Right}} Cavity W1.}
    \label{fig:decs}
  \end{center}
\end{figure}


\begin{figure}
  \begin{center}
    \begin{minipage}{\linewidth}
      \includegraphics*[width=\textwidth, trim=15mm 5mm 5mm 10mm, clip]{pannorm.eps}
      \caption{Histograms of normalized surface brightness variation
        in wedges of a $2.5\arcs$ wide annulus centered on the X-ray
        peak and passing through the cavity midpoints. {\it{Left:}}
        $36\mydeg$ wedges; {\it{Middle:}} $14.4\mydeg$ wedges;
        {\it{Right:}} $7.2\mydeg$ wedges. The depth of the cavities
        and prominence of the rims can be clearly seen in this plot.}
      \label{fig:pannorm}
    \end{minipage}
  \end{center}
\end{figure}

\begin{figure}
  \begin{center}
    \begin{minipage}{0.5\linewidth}
      \includegraphics*[width=\textwidth, angle=-90]{nucspec.ps}
    \end{minipage}
    \caption{X-ray spectrum of nuclear point source. Black denotes
      year 2000 \cxo\ data (points) and best-fit model (line), and red
      denotes year 2007 \cxo\ data (points) and best-fit model (line).
      The residuals of the fit for both datasets are given below.}
    \label{fig:nucspec}
  \end{center}
\end{figure}

\begin{figure}
  \begin{center}
    \begin{minipage}{\linewidth}
      \includegraphics*[width=\textwidth, trim=10mm 5mm 10mm 10mm, clip]{radiofit.eps}
    \end{minipage}
    \caption{Best-fit continuous injection (CI) synchrotron model to
      the nuclear 1.4 GHz, 4.8 GHz, and 7.0 keV X-ray emission. The
      two triangles are \galex\ UV fluxes showing the emission is
      boosted above the power-law attributable to the nucleus.}
    \label{fig:sync}
    \end{center}
\end{figure}

\begin{figure}
  \begin{center}
    \begin{minipage}{\linewidth}
      \includegraphics*[width=\textwidth, trim=0mm 0mm 0mm 0mm, clip]{rbs797_opt.ps}
    \end{minipage}
    \caption{\hst\ \myi+\myv\ image of the \rbs\ BCG with units e$^-$
      s$^{-1}$. Green, dashed contour is the \cxo\ 90\% EEF. Emission
      features discussed in the text are labeled.}
    \label{fig:hst}
  \end{center}
\end{figure}

\begin{figure}
  \begin{center}
    \begin{minipage}{0.495\linewidth}
      \includegraphics*[width=\textwidth, trim=0mm 0mm 0mm 0mm, clip]{suboptcolor.ps}
    \end{minipage}
    \begin{minipage}{0.495\linewidth}
      \includegraphics*[width=\textwidth, trim=0mm 0mm 0mm 0mm, clip]{suboptrad.ps}
    \end{minipage}
    \caption{{\it{Left:}} Residual \hst\ \myv\ image. White regions
      (numbered 1--8) are areas with greatest color difference with
      \rbs\ halo. {\it{Right:}} Residual \hst\ \myi\ image. Green
      contours are 4.8 GHz radio emission down to
      $1\sigma_{\rm{rms}}$, white dashed circle has radius $2\arcs$,
      edge of ACS ghost is show in yellow, and southern whiskers are
      numbered 9--11 with corresponding white lines.}
    \label{fig:subopt}
  \end{center}
\end{figure}


%%%%%%%
% Bib %
%%%%%%%

\markright{\myhead}
%\noindent \input{short.bbl}
\bibliography{cavagnolo}
\end{document}

%%%%%%%%%%%%%%%%%
%%%%%%%%%%%%%%%%%

``This result implies that the orientation of atomic filaments can
provide a local measure of the magnetic field direction in
clusters. It also provides a physical explanation for the filamentary
structures seen in optical emission-line observations of cluster cores
(Conselice et al. 2001; Sparks et al. 2004).  The filamentary
structure in the cold gas is also imprinted on the diffuse
X-ray-emitting plasma in the hot ICM (e.g., Figure 4). Because of the
large conductivity of the hot plasma (Equation (11)), it is natural
for a given magnetic field line to become relatively isothermal. If
different magnetic field lines undergo slightly different
heating/cooling, as must surely be the case to some extent, this will
lead to different temperatures, densities, and X-ray emissivities
along different magnetic field lines. This could potentially explain
the long, soft X-ray- emitting isothermal structures observed in some
clusters (Sun et al. 2010).''

``However, the observed atomic (e.g., Halpha) filaments are much
longer than this. This can be explained if the filaments are supported
by cosmic-ray (or some other form of isotropic non-thermal pressure,
e.g., due to small-scale magnetic fields) pressure which prevents the
collapse of the cold gas (see Figures 8 and 12). The presence of a
significant population of cosmic rays is also inferred by modeling the
atomic and molecular lines from clusters (Ferland et al. 2009).''

CMFs provide non-thermal pressure support to the ICM, as well as alter
transport processes like conduction, turbulence, and cosmic ray
diffusion.

%% -- what specific questions am I the ideal person to answer?
%% -- define a clear set of goals (I want to ABC!)




- is high conductivity correlated with halpha lum?
  + if so, suggests that conduction may be responsible for heating not
  just filaments, but much of the cool, multiphase has which is likely
  to fuel AGN activity (tada, the fine-tuning issue in the AGN
  feedback loop has a potential answer).
- what are the turbulent properties of the icm?
  + cannot be directly measured, only constrained (see sanders)
  + astro-h may be helpful, but spatial res is lacking
  + no ixo
  + so must resort to models to place constraints
- does any turb prop correlate with mag prop or filament prop?
  + if so, does it really look like filaments thrive in low-turb, high-b environs?
  + does this imply mhd processes are suppressed/inactive in some clusters? 
  + can any constraints be placed on the mechanism which forms filaments?
    (i) by channeling the infiow of clumps of gas along field lines,
        giving them coherent structure (Fabian et al. 2003)
    (ii) preventing the hot, turbulent ICM from shredding the filaments
         (Hatch et al. 2007)
    (iii) to help the growth of thermal instabilities, leading to thin
          high-density filaments (Hattori et al. 1995)
    (iv) suppressing conduction of heat from the ICM to the cooling gas
         (Voigt \& Fabian 2004).
- are filament morphs consistent with mag strn/struc preds of mhd models?
  + if so, are these models consistent with b-fiel measurements?
- with mag props in-hand, can I say anything about...
  + origin of fields?
  + importance of non-therm press support in mass estimates?

%% -- introduce innovative ideas

cmf ``maps'':
  + get magnitude from rm
  + get power spec from modeling
  + get ori from cpe
  + get small scale from filaments
  + combine into a map via bayseian method
use filaments as tracers of local mag fields, make comparisons to
global fields, how do they differ, are they similar?
advances in low freq data redux/analysis:
  + automation of radio data redux
  + addition of rfix to radio data redux
  + use of compressed sensing sparse sampling agols for automated pt
  and ext src det/extrac


`` Observationally, one of the most useful instruments for exploring
the physics of turbulence and convection in clusters would be a
high-spectral-resolution imaging X-ray spectrometer (calorimeter),
which could directly measure turbulent velocities in the ICM.''

UW-Mad has been at the forefront of this endeavor w/ S. Heinz leading
the way (give examples of work and code)

``Another area of active research involving plasma physics that will
be important for ad- dressing the cooling-fiow problem during the
coming decade is the interaction between jets and the ICM. In recent
years, increasingly high-resolution numerical simulations have helped
to elucidate different aspects of this interaction, including the
extent to which jets can ``drill through'' the ICM and the escape of
cosmic rays from cosmic-ray bubbles [26, 28, 29]. Al- though
anisotropic diffusion of cosmic rays has been included in recent
simulations [26], the inclusion of anisotropic viscosity will be
important for determining the rate at which cosmic- ray bubbles break
up,2 while the inclusion of anisotropic conductivity will be essential
for determining the extent of turbulent mixing in the ICM.''

-- direct measurement of icmmag field through (rm, polarization)
   +``Information on the intracluster magnetic fields can be obtained,
   in conjunction with X-ray observations of the hot gas, through the
   analysis of the rotation measure (RM) of radio galaxies in the
   background or in the galaxy clusters themselves. I will present a
   work aimed to establish a possible connection between the magnetic
   field strength and the gas temperature of the intracluster
   medium. For this purpose we investigated the RM in hot galaxy
   clusters and we compared these new data with RM information present
   in the literature for cooler galaxy clusters.''
-- look at whole zoology of properties, make constraints that way
-- no coming X-ray mission (IXO is dead), chandra and xmm archives
brimming... time to compile everything!
-- conduction alters radial props of temp, dens, and entropy
-- re-analyze mass relations using terms for non-therm p support
-- close correlation between agn feedback, star formation rates,
presence of multiphase gas, and state of hot icm
-- presence of multiphase gas directly linked to process of
conduction, and hence relates to magnetic field properties
-- relationship between the morphology of cold filaments in cool core
clusters and the structure of the icm magnetic field
-- magnetic fields important in for non-therm emission like radio
halos and cosmic ray emission
-- information about agn duty cycles relevant to understanding the
impact of intermittent agn outbursts on the evolution of icm mag
fields... infrequent stirring

Interesting observational constraints:
  -- halpha emission and morph
  -- radio halos
  -- cold fronts
  -- rotation measures
  -- polarizaton when possible

polarization at low frequencies:
-- see larger strucs at low freq because Alfven propogation speed is
higher and sync lifetimes are longer
-- measure Stokes' parameters to measure poln
-- p \propto B^2
-- poln gives direct measure of regular mag field along LOS
-- faraday rotation: rm \propto \nelec B
-- need poln measure at multiple freqs, slope gives RM
-- need bgd srcs
-- delta(rm) \approx 0.1 rad m^2 at 210-240 MHz for SN=10
-- rm synthesis \citep[\eg][]{2005A&A...441.1217B, 2010arXiv1008.3530P}:
   multiple srcs undergo multiple faraday rotations
   measure phi as a func of lambda^2
   disentangle multiple syn and rota components via fourier trans
   see brentjens & de Bruyn 05
   measure complex poln surbri as a fnc of lambda
   get spec in faraday depth
   rm syn is very similar to interfer
-- ``in the presence of polarized cluster radio sources, RM-synthesis
   is the key technique to unveil the 3-D structure of galaxy
   clusters''
-- ``reveal the origin of the observed depolarization of the cluster
sources towards low frequencies. By reducing, or even eliminating, the
importance of beam de- polarization, high resolution low-frequency
observations could test whether the depolarization is occurring
internally to the sources or in a foreground screen''
-- ``On a general point of view, the improved knowledge on the
non-thermal phenomena of the ICM will have an impact on
cosmology. If halos and relics are related to cluster mergers, the
study of the statistical properties of these sources will allow us to
test the current cluster formation scenario, giving hints on detailed
(astro)physics of large-scale structure formation (e.g. Evrard & Gioia
2002)''

``The study of polarization properties of cluster and background radio
sources is crucial to analyze the intracluster medium magnetic field
properties. Its intensity, radial decline and power spectrum can be
inferred by studying the Faraday rotation of several sources located
inside or behind the cluster, since Faraday rotation is sensitive to
the local magnetic field strength and structure.''

``The preliminary results that we obtained with these data can be
summarized as follows: 1) the Faraday Rotation measure shows a large
decrease going from the center to the periphery of the cluster. We
have obtained mean values ranging from ~136 rad/m/m to 20rad/m/m. Non
zero values of Faraday Rotation mean values indicate that the magnetic
field fluctuates on scales larger than the source's extension. 2)
Faraday rotation images are patchy, indicating that the magnetic field
fluctuates on scales smaller than the source's size; 3) the power
spectrum (|B_k|^2 ~ Lambda^n) maximum fluctuation scale is in the
range 30-250 kpc; 4) the magnetic field central intensity is in the
range 5.5 - 8 microG, the best fit is now achieved with central
intensity of 8 microG (see Fig 1) 5) the average magnetic field
intensity over the cluster volume, ~ 1 Mpc^3, is ~1.2 microG in good
agreement with the estimate derived from the radio halo emission, that
is in the range 0.7-1.9 microG (Thierbach et al. 2003).} Further
  observations have been requested to refine these estimates. The high
  value of the central magnetic field could be a feature of high mass
  system, possibly as a consequence of the magnetic field
  amplification expected in merger events.''

When emission from a radio source traverses a magnetic field, a change
in polarization angle is induced, \ie\ the EM wave undergoes Faraday
rotation. Measurement of XXX, or \frm) is a widely accepted method for
coursely sampling the {\it{strength}} of CMFs
\citep{2008SSRv..134...93F}. The novel technique of Pfrommer \& Dursi
2010 \cite{2010NatPh...6..520P} exploits the coherent polarized
emission (CPE) induced by magnetic draping of cluster member galaxies
to infer CMF {\it{orientations}}. The CPE technique was applied to
member galaxies of the Virgo cluster revealing that Virgo's ICM
magnetic fields lines are preferentially radial, consistent with the
effects of the MTI. Combined, \frm\ and CPE reveal the strength and
orientation of CMFs. The completed \evla\ upgrade has significantly
increased the sensitivity and frequency coverage of the \vla, enabling
several times more background sources per cluster to be detected,
expanding the number of \frm\ probes, and hence enabling more detailed
measurements of CMFs.

These measurements will then be used to model the magnetic field power
spectrum \citep[\eg][]{2010A&A...513A..30B} and. This will provide
constraints on...XXX? The requested observations will also be
sufficient to measure CPE of cluster members and infer the field
orientations. These measurements will be compared with the CMF
configurations predicted by MHD models. The aim of this leg of the
proposal is to establish how CMF properties differ for clusters in
different evolutionary stages and allow searches for correlations
between the magnetic field and cluster properties.

For example, the CPE method has already been used to demonstrate the
large-scale CMF of Virgo is radially oriented, consistent with field
structures expected to arise from the influence of the magnetothermal
instability \citep{2010NatPh...6..520P}.

     + AIPS and CASA scripts handling the reduction workload
     + RM and synthesis measures will require hand analysis
     + indeterminant how many sources will have these
  -- filaments survey will require halpha obs for a huge number of cds
  and bcgs using new instrs on wiyn

The \evla\ upgrade allows for the detection of several times more
background/embedded radio sources per cluster, each of which is useful
for constraining the line-of-sight ICM magnetic field strength via
rotation measure (RM) analysis.

Turbulence appears to have the critical role of mixing the ICM and
suppressing the formation of both convective and thermal instabilities
in cluster cores, allowing the gas to be heated. But, directly
measuring ICM turbulent velocities requires the high-spatial and
-energetic resolution of a microcalorimeter, which is planned for
Astro-H and IXO (which may never be realized due to funding
restrictions). Regardless, theoretical models do predict turbulence
will imprint on CMFs and subsequently influence the properties of the
ICM. For example, the model of Kunz...

1) RM used to estimate CMF radial structure and model power spectrum,
all of which can be compared with predictions from mhd sims and turb
models; $RM \propto \int \rho B_{\|}$

2) CPE gives estimate of cmf orientation which can also be compared with
predictions of models with buoy instab

3) Halos used to constrain global properties of CMF

And given the bleak observational limitations, there isn't much other
choice. So, I suggest calculating turbulent ICM values using Kunz
model, compiling secondary diagnostics, and comparing with the
measured magnetic field and ICM thermal/non-thermal properties. It may
emerge that AGN jet power correlates with small turbulent length
scales, or that the magnetic field strengths predicted by the model
are consistent with the observations. Regardless of the results,

The luminosities and morphologies of \halpha\ filaments give an
estimate of how much conductive heating is taking place and the
strengths of the fields supporting the filaments.

For a few clusters, the measurements of CMF magnitude, direction,
spatial distribution, and line-of-sight distribution should be robust
enough to allow at least 2D, and possible 3D, maps to be generated. If
so, this will be the first time such maps have been created from
observations alone.

ICM thermal properties are directly observable via X-ray emission, and
CMF properties are inferred from radio observations of synchrotron
emission. Hence, this work relies heavily on archival X-ray \& radio
data, and new datasets to perform innovative analysis of optical and
polarized radio emission.

The radio, optical, and archival projects have the added benefit of
complimenting the science goals of the \evla\ Deep Cluster
Survey\footnote{http://www.atnf.csiro.au/people/Shea.Brown/cosweb/},
and the \lofar\ Surveys and Cosmic Magnetism Key Science
Projects\footnote{http://www.lofar.org/astronomy/key-science/lofar-key-science-projects}.

For cluster cores where the turbulent length scale is longer than the
Field Length, 

How the filaments form, be it fragmentation along field lines, CMF
promotion of thermal instability growth, inhibition of turbulent
shredding, or heat conduction suppression, will also be investigated.

Consequently, it is now possible to further evaluate thermodynamic
theories of the ICM by obtaining better measurements of CMF properties
for large samples of clusters and establishing how CMF and ICM
properties correlate as a function of

There is also an UW X-ray Astrophysics group which is headed by Dan
McCammon and is deeply-involved in the development of the next
generation of X-ray instruments. This project has a large X-ray
component which has relevance to the development of the long-sought
space-based, high-spatial resolution microcalorimeter.

Similarly, a radio reduction pipeline has been developed and
was run on 72 \vla\ datasets to complete the work published in
Cavagnolo \etal\ 2010 \citep{pjet}.
