\begin{figure}
  \begin{center}
    \begin{minipage}{0.495\linewidth}
      \includegraphics*[width=\linewidth, trim=0mm 0mm 0mm 0mm, clip]{bonafede.ps}
    \end{minipage}
    \begin{minipage}{0.495\linewidth}
      \caption{Radial profile and power spectrum of the Coma CMF
        derived from 3D simulations which reproduce observed RMs of
        embedded and background radio sources (taken from Bonafede
        \etal\ 2010 [29]). If one assumes the CMF and ICM thermal
        radial distributions trace each, then comparison of observed
        RM dispersions and 3D simulations lead to constraints like
        this on the CMF profile without the need to make measurements
        at every radius. One goal of this proposal is to expand upon
        the Bonafede \etal\ result using a larger cluster sample and
        EVLA observations.}
    \end{minipage}
    \rule{\linewidth}{1pt}
    \begin{minipage}{0.495\linewidth}
      \caption{Virgo CMF orientations (yellow arrows) taken from
        Pfrommer \& Dursi 2010 [30] where they argue draping of CMF
        lines at the ICM-infalling galaxy interface explains the CPE
        (left panel at 5 GHz). The CPE results from galactic cosmic
        rays gyrating around regularly compressed field lines. The
        authors argue the Virgo CMF is preferentially radial,
        consistent with the effects of a large-scale MHD convective
        instability [\ie\ the MTI; 21]. Similar measurements are a key
        feature of the EPiCS project and will help us constrain CMF
        orientations as never before.}
    \end{minipage}
    \vspace{0.2cm}
    \begin{minipage}{0.495\linewidth}
      \includegraphics*[width=\linewidth, trim=0mm 0mm 0mm 0mm, clip]{pfrommer.ps}
    \end{minipage}
    \rule{\linewidth}{1pt}
    \begin{minipage}{0.495\linewidth}
      \includegraphics*[width=\linewidth, trim=10mm 4mm 2mm 5mm, clip]{kunz.eps}
    \end{minipage}
    \begin{minipage}{0.495\linewidth}
      \caption{Predicted CMF strength of Abell 1835 from model of Kunz
        \etal\ 2010 [16] which heats the ICM {\it{only}} via viscous
        dissipation of turbulent ICM motions. The Kunz \etal\ model
        takes in X-ray derived ICM density and temperature
        measurements alone and returns estimates of field
        strengths. These profiles have already been derived for over
        300 clusters using the \chandra\ archival project, and will be
        compared with the results of the EPiCS program (\eg\ radial
        profiles and power spectra, like Fig. 1) to test how important
        turbulent heating is for an array of cluster types. This is
        one example of how the proposed CMF measurements will be
        directly compared with model predictions to aide theorists in
        refining models for ICM heating.}
    \end{minipage}
  \end{center}
\end{figure}
