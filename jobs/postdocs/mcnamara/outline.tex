\documentclass[12pt]{article}
\usepackage{macros_cavag}
\setlength{\topmargin}{-0.25in}
\setlength{\oddsidemargin}{-0.1in}
\setlength{\evensidemargin}{0in}
\setlength{\textwidth}{6.7in}
\setlength{\headheight}{0in}
\setlength{\headsep}{0in}
\setlength{\topskip}{0.55in}
\setlength{\textheight}{9.25in}
\pagestyle{myheadings}
\renewcommand{\labelenumi}{\arabic{enumi}.}
\renewcommand{\labelenumii}{\arabic{enumi}.\arabic{enumii}}
\renewcommand{\labelenumiii}{\arabic{enumi}.\arabic{enumii}.\arabic{enumiii}}
\renewcommand{\labelenumiv}{\arabic{enumi}.\arabic{enumii}.\arabic{enumiii}.\arabic{enumiv}}
\markright{\hspace*{\fill}{\it Summary of Research, Kenneth W. Cavagnolo}\hspace{10mm}}

\begin{document}
\begin{center}
%\vspace{1.0mm}
{\bf Summary of Research Outline}\\
%\vspace{1.0mm}
\end{center}
\small

\begin{tabbing}
0) What \= have I been doing ?\\
\>	A) \=Studying feedback mechanisms, galaxy formation, and cooling\\
\>\>	   via entropy dist.\\
\>	B) Finding metrics which can quantify the virialization state\\
\>\>	   of a cluster\\
\>	C) Studying s-process abundances in PNe\\
\\
1) Entropy distribution in clusters:\\
\>	A) Constructed enormous sample (largest yet?) of Chandra\\
\>\>	   archi\=val data\\
\>\>\>		I)   324 observation\\
\>\>\>		II)  276 clusters and groups\\
\>	B) Wrote a robust pipeline for reducing massive flows of data\\
\>\>\>		I)   \= Seamlessly incorporate most up-to-date versions of CIAO and\\
\>\>\>\>	     CALDB\\
\>\>\>		II)  Full automation frees time for devotion to\\
\>\>\>\>	     multiple, simultaneous projects\\
\>	C) Gained intimate understanding of reducing and interpreting\\
\>\>	   complex datasets and incorporating multiwavelength\\
\>\>	   observations into analysis\\
\>\>\>		I)   Pipeline is adaptable to any other datasets which\\
\>\>\>\>	     use base package of analysis tools; i.e. XMM$->$SAS,\\
\>\>\>\>	     HST$->$IRAF, Spitzer$->$Post-BCD, VLA$->$AIPS\\
\>\>\>		II)  Complexity and rigor of Chandra analysis has\\
\>\>\>\>	     prepared me for work with any kind of data\\
\>	D) What are we studying?\\
\>\>\>		I)   K0 connected to L$_{radio}$\\
\>\>\>		II)  K0 connected to L$_{H\alpha}$\\
\>\>\>		III) Distribution of K0 establishes continuum of AGN\\
\>\>\>\>	     activity\\
\>\>\>		IV)  Distribution of t$_{cool}$ sets timescale\\
\>\>\>\>	     of AGN feedback\\
\>\>\>		V)   Dependence of X-ray loud cluster AGN distribution\\
\>\>\>\>	     on redshift and amount of substructure\\
\>	E) What has this work produced?\\
\>\>\>		I) A body of work for two first author papers\\
\>\>\>		II)   Two referreed journal papers\\
\>\>\>\>		i)   \cite{2007AJ....134...14D}\\
\>\>\>\>		ii)  \cite{2006ApJ...643..730D}\\
\>\>\>		III)  New, unique presentations each year\\
\>\>\>\>		i)   2007 Chandra Symposium\\
\>\>\>\>		ii)  2008 AAS thesis talk and poster\\
\>\>\>\>		iii) \cite{2006AAS...209.7711D}\\
\>\>\>\>		iv)  \cite{2005AAS...20713903C}\\
\>\>\>\>		v)   \cite{2004AAS...20514715C}\\
\>\>\>\>		vi)  \cite{2004AAS...205.6020D}\\
\>\>\>		IV)  Contribution to XXX proposals (YYY of which were\\
\>\>\>\>	     successful)\\
\\
2) Quantifying cluster virialization\\
\>	A) Connecting THFR w/ CC/NCC\\
\>	B) Connection of THFR w/ mergers\\
\>	C) Produced paper and contributed to successful Chandra theory\\
\>\>	   proposal\\
\>	D) Used in the thesis work of David Ventimigilia\\
\\
3) s-process abundances in PNe\\
\>	A) Reduced and analyzed double echelle spectra from MIKE on\\
\>\>	   Magellan\\
\>	B) Contributed to \cite{2007ApJ...659.1265S}\\
\>	C) Formed foundation of thesis work by Eric Pelligrini\\
\\
4) Future work\\
\>	A) How does AGN heating balances ICM cooling?\\
\>\>		I)   What is the physics of the ICM heating process?\\
\>\>\>\>		i)   Thermalizing of mechanical work by bubbles?\\
\>\>\>\>		ii)  Cosmic ray heating from non-thermal particles in bubbles?\\
\>\>		II)  How is accretion onto BH regulated by large\\
\>\>\>		     scale properties of ICM?\\
\>\>		III) How dramatic are departures from average AGN\\
\>\>\>		     state, i.e. what does the AGN energetic injection\\
\>\>\>		     function look like and how does it correlate with\\
\>\>\>		     cluster environment?\\
\>	B) How exactly is the AGN fueled?\\
\>\>\>		I)   Accretion of the hot ICM/ISM (is it via\\
\>\>\>\>	     Bondi-eque flows?)\\
\>\>\>		II)  Cold gas flowing out of ICM? (the leftovers of\\
\>\>\>\>	     star formation?)\\
\>\>\>		III) What is the efficiency of the accretion?\\
\>\>\>		IV)  Why do we see metallicity gradients? (no mixing?)\\
\>	C) Correlate findings from entropy work with galaxy\\
\>\>	   formation models\\
\>\>\>		I)   Low-z feedback which heats ICM should be same\\
\>\>\>\>	     high-z mechanism which truncates massive galaxy\\
\>\>\>\>	     formation\\
\>\>\>		II)  Excluded pt srcs in my analysis are mostly AGN...\\
\>\>\>\>		i)   Num. of AGN as a function of distance from\\
\>\>\>\>		     cluster center\\
\>\>\>\>		ii)  Go one step further and examine Magorrian\\
\>\>\>\>		     relation for these sources\\
\>	D) Sources of entropy injection at high-z\\
\>\>		I)   Role of mergers in heating ICM and boosting entropy to\\
\>\>\>		     t$_{cool} >$ Hubble time?\\
\>\>		II)  Galactic winds, AGN, thermal conduction,\\
\>\>\>		     radiative cooling\\
\>\>		III) Is radiation pressure from supermassive first\\
\>\>\>		     stars more important than any of these\\
\>\>\>		     mechanisms? (Norm Murray's work)\\
\>	E) Untouched aspects of project\\
\>\>		I)   Creation of 2D maps using Voronoi Tesselation of\\
\>\>\>		     Diehl and Statler\\
\>\>		II)  Analysis of clusters residing in the Virtual\\
\>\>\>		     Cluster Exploratory\\
\>\>		III) Substructure analysis to reduce scatter in\\
\>\>\>		     scaling relations: centroid shift, power\\
\>\>\>		     ratio, and axial ratio\\
\\
5) Summary\\
\>	A) Thesis has raised more questions than questions which have\\
\>\>	   been addressed\\
\>	B) Expertise with Chandra and clusters ideally suits me to\\
\>\>	   work more on feedback models and galaxy formation\\
\>	C) Adapting X-ray astronomy skills to radio, optical, and\\
\>\>	   infrared will be quick and next natural step\\
\>	D) I have a great deal to offer a group studying clusters both\\
\>\>	   in technical skills and furthering their research goals\\
\\
Sidenote: things which might be useful in other proposals or letters:\\
\\
XX) Clusters 101\\
\>	A) What are clusters?\\
\>	B) Why does anyone care about clusters? What can they tell us\\
\>\>	   about cosmology, galaxy formation, black holes, star formation?\\
\>	C) How do they form?\\
\>	D) What is the present state of the field?\\
\\
XX) Why have I been studying these things?\\
\>	A) Can we find observables which quantify location of a\\
\>\>	   cluster on M-T, M-L relations?\\
\>	B) Is it possible to quantify degree of virialization?\\
\>	C) Excess entropy compared with self-similar collapse model.\\
\>	D) Low-z feedback which retards unabated cooling is likely the\\
\>\>	   same high-z feedback which truncates massive galaxy formation\\
\>	E) Feedback signatures are everywhere: ICM cavities, relics,\\
\>\>	   cold filaments, shocks/compression waves,  cold\\
\>\>	   fronts... all these things must be assembled into one functional\\
\>\>	   framework.\\
\>	F) ``Entropy has the unique property of decreasing with radiative\\
\>\>	   cooling, increasing due to heating processes, but staying constant\\
\>\>	   with compression or expansion under energy conservation.''\\
\\
XX) Oddities\\
\>	A) IRAS 09104+4109\\
\>	B) Abell 2384 merger\\
\>	C) Prevalance of cold fronts in clusters\\

\end{tabbing}
\bibliographystyle{plan}
\bibliography{cavagnolo}
 
\end{document}
