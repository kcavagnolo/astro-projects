\documentclass[11pt]{article}
\usepackage[colorlinks=true,linkcolor=blue,urlcolor=blue]{hyperref}
\usepackage{macros_cavag}
\pagestyle{myheadings}
\font\cap=cmcsc10
\setlength{\topmargin}{-0.25in}
\setlength{\oddsidemargin}{-0.15in}
\setlength{\evensidemargin}{-0.15in}
\setlength{\headheight}{0.1in}
\setlength{\headsep}{0.5in}
\setlength{\topskip}{0.1in}
\setlength{\textwidth}{6.5in}
\setlength{\textheight}{9.5in}

\newcommand{\sekshun}[1]
{
\markboth{\hfill \bf{K. W. Cavagnolo}{#1} \bf{Summary of Research} \hfill}
	 {\hfill \bf{K. W. Cavagnolo}{#1} \bf{Summary of Research} \hfill}
}

\begin{document}

\sekshun{}
The general process of galaxy cluster formation through hierarchical
merging is well understood, but many details, such as the impact of
feedback sources on the cluster environment and radiative cooling in
the cluster core are not. My thesis research has focused on studying
these details via X-ray properties of the ICM in clusters of
galaxies. I have paid particular attention to ICM entropy
distribution, the process of virialization, and the role of AGN
feedback in shaping large scale cluster properties.

My primary research makes use of a 350 observation sample (276
clusters) taken from the {\textit{Chandra}} archive. This massive
undertaking necessitated the creation of a robust reduction and
analysis pipeline which 1) interacts with mission specific software,
2) utilizes analysis software (i.e. {\tt{XSPEC}}, {\tt{IDL}}), 3)
incorporates calibration and software updates, and 4) is highly
automated. Because my pipeline is written in a very general manner,
adding pre-packaged analysis tools from missions such as
{\textit{XMM}}, {\textit{Spitzer}}, and {\textit{VLA}} will be
straightforward. Most importantly, my pipeline deemphasizes data
reduction and accords me the freedom to move quickly into an analysis
phase and generating publishable results.

The picture of the ICM entropy-feedback connection emerging from my
work suggests that cD radio luminosity and H$\alpha$ emission are
anti-correlated with cluster central entropy. I have explored these
relations with my thesis sample and am finding a trend of high central
entropy favoring low L$_{H\alpha}$ and low L$_{Radio}$. I am
following up these results by examining the distribution of central
cooling times as a window onto the timescale of AGN feedback. In
addition, I am exploring the dependence of the X-ray loud AGN
distribution on redshift and amount of cluster substructure.

This work has been very fruitful thus far: I am a co-author for two
refereed journal papers (\cite{2007AJ....134...14D},
\cite{2006ApJ...643..730D}), generated new and unique work each year
(\cite{2008AAS},  \cite{2007Chandrasym}, \cite{2006AAS...209.7711D},
\cite{2005AAS...20713903C}, \cite{2004AAS...20514715C},
\cite{2004AAS...205.6020D}), a first author paper which is in draft,
and another first author paper in preparation containing my thesis
results. I have also contributed to several successful
{\textit{Chandra}}, {\textit{XMM}, {\textit{Suzaku}}, and
{\textit{Subaru}} proposals in addition to writing my own high
scoring -- although unsuccessful -- {\textit{Chandra}} proposal for time
observing an amazing ULIRG. I am also planning H$\alpha$ imaging
observations for several previously unobserved clusters with MSU's
SOAR telescope.

In another part of my thesis research I have studied bandpass dependence in
determining X-ray temperatures and what this dependence tells us about
the virialization state of a cluster. The ultimate goal of this
project is to find an aspect-independent measure for a cluster's
dynamic state. Prompted by the work of \cite{2001ApJ...546..100M} I
have investigated the net temperature skew in my sample of the
hard-band (2.0$_{rest}$-7.0 keV) and full-band (0.7-7.0 keV)
temperature ratio for core-excised apertures. I have
found this temperature ratio is statistically connected to
mergers and the presence of cool cores. This work
has produced a first author paper which is near ApJ submission and was
used in a successful {\textit{Chandra}} theory proposal.

Looking ahead, the natural extension of my thesis is to further study
questions regarding details of feedback and galaxy formation. What are
the micro-physics of ICM heating, including the thermalization of
mechanical work done by bubbles and the effect of non-thermal sources
like cosmic rays. How prevalent are cold fronts and do they play a
role in galaxy and star formation? Also of interest are how accretion
onto the cD SMBH is regulated by large-scale ICM properties and what the
AGN energy injection function looks like and how it correlates with
cluster environment.

There are also exciting theoretical cluster feedback model
developments on the horizon which will need observational
investigation, and for which I am well positioned to
study. Developments such as: how exactly are AGN fueled? Does
accretion of the hot ICM/ISM proceed via Bondi-eque flows? What is the
efficiency of the accretion? Why do we see metallicity gradients in
the ICM/ISM when some amount of mixing should take place? How is feedback
energy distributed symmetrically throughout the ICM?

\clearpage
\bibliographystyle{unsrt}
\bibliography{cavagnolo}
 
\end{document}
