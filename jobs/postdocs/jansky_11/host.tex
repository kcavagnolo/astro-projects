\indent The University of Maryland (UMD) is an ideal host for this
ambitious project as it boasts a team of experts well-suited to assist
with this work. The UMD Astronomy Department, the Center for Research
and Exploration in Space Science and Technology, and the Center for
Theory and Computation are hosts to (to name but a few) Keith Arnaud,
Tamara Bogdanovi{\'c} (current Einstein fellow), Michael Loewenstein,
Craig Markwardt, Cole Miller, Richard Mushotzky, Eve Ostriker, Chris
Reynolds (the sponsor), Massimo Ricotti, and Sylvain Veilleux. All
those listed are experts in one, or several, of the areas of AGN
feedback, ICM physics, computational modeling, magnetic field
polarimetry, plasma physics, and X-ray/radio observing \& analysis. In
addition, UMD has close ties with the Naval Research Lab where Tracy
Clarke and Namir Kassim are appointed -- both experts in the radio
techniques and science topics discussed here. {\bf{Year one:}} Data
acquisition begins, the archival project and tool development
continue; I initiate a theoretical/simulation collaboration with the
UMD plasma physics group to study new questions like: How do
convective instabilities couple with ICM cooling and the actual
accretion which drives AGN activity?  What is the relation between
these processes, ICM temperature \& density, and thermal instability
formation? {\bf{Year two:}} Data acquisition continues, first round of
archival-based results is published, and observation-model comparisons
begin. {\bf{Year three:}} Data acquisition and analysis of REXCESS
conclude, second round of results published, and the investigation of
CMF origins and non-thermal pressure support is underway.
