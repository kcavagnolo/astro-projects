\documentclass[letterpaper,12pt]{article}
\usepackage[numbers,sort&compress]{natbib}
\bibliographystyle{unsrt}
\usepackage{common,graphicx}
\pagestyle{myheadings}
\setlength{\textwidth}{7.2in} 
\setlength{\textheight}{9.3in}
\setlength{\topmargin}{-0.3in} 
\setlength{\oddsidemargin}{-0.35in}
\setlength{\evensidemargin}{0in} 
\setlength{\headheight}{0in}
\setlength{\headsep}{0.3in} 
\setlength{\hoffset}{0in}
\setlength{\voffset}{0in}
\newcommand{\myhead}{Cavagnolo, Research Summary}
\begin{document}

\begin{center}
  {\bf\uppercase{Summary of Past and On-going Research}}
\end{center}

In a broad sense, my research program focuses on galaxy clusters, both
as astrophysical laboratories and interesting structures in their own
right. In the case of the latter, it is the properties of the
intracluster medium (ICM) which have captivated me, while for the
former, I am most interested in feedback from active galactic nuclei
(AGN) and the processes which couple AGN activity and the ICM. Some
highlights of my research in these areas are presented below.\\

\noindent{\bf{I. ICM Temperature Inhomogeneity}}\\
\indent If simple galaxy cluster observables such as temperature or
luminosity are to serve as accurate mass proxies, how mergers alter
these observables need to be quantified. It is known that cluster
substructure correlates well with dynamical state, and that the
apparently most relaxed clusters have the smallest deviations from
mean mass-observable relations \citep[\eg][]{VV08}. But, 2D analysis
is at the mercy of perspective. If a cluster's ICM is nearly
isothermal in the projected region of interest, the X-ray temperature
inferred from a broadband (0.7-7.0 keV) spectrum should be identical
to the X-ray temperature inferred from a hard-band (2.0-7.0 keV)
spectrum. However, if there are unresolved, cool lumps of gas, the
temperature of a single-component thermal model will be cooler in the
broadband versus the hard-band. This difference is then possibly a
diagnostic to indicate the presence of cooler gas, \eg\ associated
with merging sub-clusters, even when the X-ray spectrum itself may not
have sufficient signal-to-noise to resolve multiple temperature
components \citep{me01}.

In Cavagnolo \etal\ 2008 \citep{xrayband} we studied this temperature
band dependence for 192 clusters taken from the \chandra\ Data
Archive. We found, on average, that the hard-band temperature was
significantly higher than the broadband temperature, and that their
ratio was preferentially larger for known mergers (Fig.
\ref{fig:thbr}). The interpretation of this result is that, indeed,
ICM temperature inhomogeneity is detectable via a simple bandpass
comparison and, on average, it correlates with cluster dynamical
state. Our results suggest such a temperature diagnostic may be useful
when designing metrics to minimize the scatter about mean mass-scaling
relations in an attempt to obtain smaller uncertainties in cluster
mass estimates.\\

\noindent{\bf{II. ICM Entropy and AGN Feedback}}\\
\indent ICM temperature and density mostly reflect the shape and depth
of a cluster's dark matter potential -- it is the specific entropy ($K
\approx \tx \nelec^{-2/3}$) which governs the density at a given
pressure \citep{voitbryan}. Ignoring convective instabilities induced
by anisotropic conduction, the ICM is convectively stable when the
lowest gas entropy occupies the bottom of the cluster potential and
the highest entropy gas has buoyantly risen to larger radii. Further,
ICM entropy is primarily changed through heat exchange. Thus,
deviations of the ICM entropy structure from the azimuthally
symmetric, radial power-law distribution which should result from pure
cooling are useful in evaluating a cluster's thermodynamic history
\citep{vkb05}. One key use of ICM entropy is for studying the effects
of energetic feedback on the cluster environment and investigating the
breakdown of cluster self-similarity \citep{agnframework}.

In Cavagnolo \etal\ 2009 \citep{accept}, the ICM entropy structure of
239 clusters taken from the \chandra\ Data Archive were studied. We
found that most clusters have entropy profiles which are well-fit by a
model which is a power-law at large radii and approaches a constant
entropy value at small radii: $K(r) = \kna + \khun (r/100
~\kpc)^{\alpha}$, where \kna\ quantifies the typical excess of core
entropy above the best fitting power-law found at larger radii and
\khun\ is the entropy normalization at 100 kpc
(Fig. \ref{fig:splots}). Our results are consistent with models which
predict cooling of a cluster's X-ray halo is offset by energy injected
via feedback from AGN \citep[\eg][]{agnframework}. We also showed that
the distribution of \kna\ values in our archival sample is bimodal,
with a distinct gap around $\kna \approx 40 ~\ent$.

If cooling of a galaxy cluster's halo triggers eventual heating via an
AGN-centric feedback loop, then certain ICM properties (\eg\ entropy)
may correlate tightly with signatures of feedback and/or indicators of
cooling. In Cavagnolo \etal\ 2008 \citep{haradent} we explored the
relationship between \halpha\ emission from cluster cores, radio
emission from cluster central galaxies, and cluster
\kna\ values. Utilizing the results of the archival study of
intracluster entropy, we found that \halpha\ and radio emission are
almost strictly associated with \kna\ values less than $30 ~\ent$
(Fig. \ref{fig:ha}), which is near the gap of the
\kna\ distribution. The prevalence of \halpha\ emission below this
threshold indicates that it marks a dichotomy between clusters that
can harbor thermal instabilities (\eg\ multiphase gas, star formation)
in their cores and those that cannot. The fact that strong central
radio emission also appears below this boundary suggests that feedback
from an AGN turns on when the ICM starts to condense, strengthening
the case for AGN feedback as the mechanism that limits star formation
in the Universe's most luminous galaxies. In Voit \etal\ 2008
\citep{conduction}, we go on to suggest that core entropy bimodality
and the sharp entropy threshold arises from the influence of thermal
conduction. This result is one of the key motivating factors for the
fellowship proposal that follows, and is discussed in more detail
therein.\\

\noindent{\bf{III. Details of AGN Feedback}}\markright{\myhead}\\
\indent{\bf{A. Properties of Jets:}} A long-standing problem in
observational and theoretical studies of energetic feedback AGN is
estimating total kinetic energy output. These estimates have
historically been made using jet models built around first principles
and observations of how it {\it{appears}} jets interact with their
surroundings \citep[\eg][]{w99}. However, X-ray observations of
clusters have revealed that AGN outflows inflate cavities in the ICM,
and these cavities yield a direct measure of the work, and hence total
mechanical energy, exerted by the AGN on its environment
\citep[see][for details]{mcnamrev}. Hence, any correlation between
derived cavity power and associated synchrotron radio power yields a
useful device for constraining total AGN energy output when X-ray data
or cavities are unavailable. Such relations between jet power (\pjet)
and radio power (\prad) for clusters were presented by Birzan
\etal\ 2004, 2008 \citep{birzan04, birzan08}, and in Cavagnolo
\etal\ 2010 \citep{pjet} we appended a sample of 13 giant ellipticals
(gEs) to these studies to determine if a single \pjet-\prad\ relation
extends from clusters down to isolated gEs.

Utilizing the analysis of $> 70$ multifrequency, archival
\vla\ observations and an array of low-frequency radio surveys, we
found that the \pjet-\prad\ relation is continuous, and has similar
scatter, from clusters down to gEs (Fig. \ref{fig:pjet}). We also
found that, independent of frequency, \pjet\ scales as $\sim
\prad^{0.7}$ with a normalization of $\sim 10^{43}$ erg
s$^{-1}$. Numerous jet models predict a power-law index of $\approx
12/17$, consistent with our results, and normalizations of $\sim
10^{43} ~\lum$ when the ratio of non-radiating particles to
relativistic electrons is $\ga 20$ (\ie\ moderately heavy jets). Our
results imply that there does exist a universal scaling relation
between jet power and radio power, which would be a useful tool for
calculating AGN kinetic output over huge swathes of cosmic time using
only all-sky, monochromatic radio surveys. The impact of this result
extends into the areas of galaxy formation, black hole growth, and the
mechanical heating of the universe.

\indent{\bf{B. Radio-mode and Quasar-mode Feedback:}} Galaxy formation
models typically segregate AGN feedback into a distinct early-time,
radiatively-dominated quasar mode \citep[\eg][]{2005Natur.435..629S}
and a late-time, mechanically-dominated radio mode
\citep[\eg][]{croton06}. In quasar-mode, it is believed that intense
quasar radiation ($> 10^{46} ~\lum$) couples to gas within the host
galaxy and drives strong winds which deprive the SMBH of additional
fuel, regulating growth of black hole mass. At later times, when
quasar activity has faded and radio mode feedback takes over, SMBH
launched jets regulate the growth of galaxy mass through prolonged and
intermittent mechanical heating of a galaxy's gaseous halo (\ie\ the
process discussed in Section A). Though AGN feedback models are broken
into two generic modes, they still form a unified schema
\citep[\eg][]{sijacki07} which predicts a continuous distribution of
AGN luminosities. However, the association of the modes, and whether
they interact, is still poorly understood. In Cavagnolo \etal\ 2010
\citep{iras09} we present a study of the famous \& enigmatic massive
galaxy IRAS 09104+4109 (IRAS09) which simultaneous exhibits all the
characteristics of a system in radio- and quasar-mode of feedback,
perhaps implying it is a ``transition'' object cycling between the
modes.

A joint X-ray/radio analysis of IRAS09 reveals cavities in the
galaxy's X-ray halo associated with an AGN outflow having $\sim
10^{44} ~\lum$ of mechanical energy and an obscured nuclear quasar
with a luminosity of $\sim 10^{47} ~\lum$. We directly measure, for
the first time, that the radiative to mechanical feedback energy ratio
for a ``transition'' object is $\sim$1000:1. Further, the cavities
contain enough energy to offset $\approx 25-35\%$ of the host
cluster's ICM radiative cooling losses. However, how this energy is
thermalized remains unknown -- which is one aspect of the fellowship
proposal which follows. Nonetheless, our results suggest 3--4 similar
strength AGN outbursts are sufficient to suppress ICM core cooling and
freeze-out rapid BCG star formation. We also unambiguously demonstrate
that the beaming directions of the jets and nuclear radiation are
indeed misaligned, as previous studies have suggested. Our
interpretations of these findings are that IRAS09 may be providing a
local example of how the AGN feedback cycle of massive galaxies at
higher redshifts evolves, and may also be offering clues as to how the
evolution of black hole spin is closely correlated with the feedback
cycle.

\indent{\bf{C. Black Hole Spin:}} Current models of the late-time
feedback loop posit that cooling processes in a galaxy's halo result
in mass accretion onto a central supermassive black hole (SMBH),
thereby driving AGN activity. While there is direct evidence that halo
cooling and feedback are linked \citep[\eg][]{haradent}, the
observational constraints on how AGN are fueled and powered remain
loose. For example, what fraction of the energy released in an AGN
outburst is attributable to the gravitational binding energy of the
accreting matter \citep{1984RvMP...56..255B} and the SMBH's rotational
energy \citep{2002NewAR..46..247M} is still unclear. Mass accretion
alone can, in principle, fuel most AGN
\citep[\eg][]{rafferty06}. However, there are gas-poor systems which
host very powerful AGN (energies $> 10^{61}$ erg) where mass accretion
alone appears unlikely as a power source. The importance of these
systems in understanding AGN feedback is that either the AGN fueling
has been astoundingly efficient, or the power has come from an
alternate source, such as the release of angular momentum stored in a
rapidly spinning SMBH \citep[\eg][]{msspin}. If more such systems can
be found, then it may be necessary to incorporate a SMBH spin feedback
pathway into galaxy formation models.

In Cavagnolo \etal\ 2010 \citep{r797}, we present analysis of the AGN
outburst in the galaxy cluster RBS 797 and, because of the extreme
energetic demands of the outburst, suggest it may have been powered by
black hole spin. We estimate the total energy output and jet power to
be of the order $10^{61}$ erg and $10^{46} ~\lum$, respectively. These
enormous energies demand that mass accretion alone is an implausible
explanation for how the outburst was powered, and we thus suggest that
the outburst resulted from the extraction of rotational energy stored
in a rapidly-spinning SMBH. We conclude that RBS 797 may be further
observational evidence that some AGN are powered by the release of
SMBH spin energy.

The model of Garafalo \etal\ 2010 \citep{gesspin} suggests that the
evolution of a black hole's spin state is closely tied to the
process of AGN feedback. In their model, during the process of
retrograde accretion induced spin-down, a black hole should pass
through a state where its spin is $\approx 0$. At this point, an
asymmetric accretion flow exceeding a mass-dependent critical
accretion rate can drastically and quickly reorient a spin axis. This
process is the focus of Cavagnolo \etal\ 2010 \citep{spinaxis} as it
can possibly give rise to the type of beamed jet-radiation
misalignment observed in IRAS09, and can also result in the extraction
of extreme jet powers like in RBS 797. In this work, we discuss the
complications of re-orienting the spin axis of a SMBH via mergers,
which has become a bit fashionable as an explanation for how AGN
feedback energy is distributed beyond the small cross-section of AGN
jets. In the following fellowship proposal, how AGN feedback energy
may be distributed is instead discussed in terms of microphysical ICM
processes like conduction, turbulence, and magnetohydrodynamic
instabilities.

\markright{K.W. Cavagnolo, Summary}
\begin{figure}[t]
    \begin{minipage}[t]{0.5\linewidth}
        \centering
        \includegraphics*[width=\textwidth, trim=17mm 3mm 10mm 11mm, clip]{thbr.eps}
        \caption{\footnotesize $T_{HBR}$ vs. $T_{0.7-7.0}$. The dashed
          line is the line of equivalence. Symbols and color coding
          are based on two criteria: 1) presence of a cool core (CC)
          and 2) value of $T_{HBR}$. Black stars are clusters with a
          CC and $T_{HBR}$ significantly greater than 1.1. Green
          upright-triangles are NCC clusters with $T_{HBR}$
          significantly greater than 1.1. Blue down-facing triangles
          are CC clusters and red squares are NCC clusters. It is
          found that most, if not all, of the clusters with $T_{HBR}
          \ga 1.1$ are merger systems.}
        \label{fig:thbr}
    \end{minipage}
    \hspace{0.1in}
    \begin{minipage}[t]{0.5\linewidth}
        \centering
        \includegraphics*[width=\textwidth, trim=28mm 7mm 30mm 17mm, clip]{splots_allt.eps}
        \caption{\footnotesize Composite plot of entropy profiles for
          archival sample. Profiles are color-coded based on average
          cluster temperature; units of the color bar are keV. The
          solid line is the pure-cooling model of \citep{voitbryan},
          the dashed line is the mean profile for clusters with $\kna
          \le 50 ~\ent$, and the dashed-dotted line is the mean
          profile for clusters with $\kna > 50 ~\ent$.}
        \label{fig:splots}
    \end{minipage}
    \hspace{0.1in}
    \begin{minipage}[t]{0.5\linewidth}
        \centering
        \includegraphics*[width=\columnwidth, trim=28mm 7mm 40mm 17mm, clip]{ha.eps}
        \caption{\footnotesize Central entropy
          vs. \halpha\ luminosity. Orange circles represent
          \halpha\ detections, black circles are non-detection upper
          limits, and blue squares with inset red stars or orange
          circles are peculiar clusters which do not adhere to the
          observed trend. The vertical dashed line marks $\kna = 30
          ~\ent$. Note the presence of a sharp \halpha\ detection
          dichotomy beginning at $\kna \la 30 ~\ent$.}
        \label{fig:ha}
    \end{minipage}
    \hspace{0.1cm}
    \begin{minipage}[t]{0.5\linewidth}
        \centering
        \includegraphics*[width=\textwidth, trim=30mm 5mm 40mm 15mm, clip]{pjet.eps}
        \caption{\footnotesize Cavity power vs. 1.4 GHz radio
          power. Orange triangles represent cluster and group sample
          of \citep{birzan08}, filled circles are our gE sample,
          colors represent the quality of cavities: green = `A,' blue
          = `B,' and red = `C.' Dotted red lines represent
          \citep{birzan08} best-fit relations. Dashed black lines
          represent our \bces\ best-fit power-law relations.}
        \label{fig:pjet}
    \end{minipage}
\end{figure}

\bibliography{cavagnolo}

\end{document}
