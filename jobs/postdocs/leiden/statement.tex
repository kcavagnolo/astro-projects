% ubercompact
\documentclass[11pt]{article}
\setlength{\topmargin}{-0.25in}
\setlength{\oddsidemargin}{-0.25in}
\setlength{\headheight}{0.0in}
\setlength{\headsep}{0.0in}
\setlength{\topskip}{0.0in}
\setlength{\textwidth}{7.0in}
\setlength{\textheight}{9.5in}

% declare packages and options
\usepackage[T1]{fontenc}
\usepackage{subfig,epsfig,colortbl,graphics,graphicx,wrapfig,amssymb,common,mathptmx,multicol,natbib}

% start the document
\begin{document}

% header
\begin{center}
\textbf{Statement of Research Interests}
\end{center}

Several decades of observations have helped define the current galaxy
formation paradigm in which supermassive black holes (SMBHs) and
active galactic nuclei (AGN) play a vital role in regulating structure
formation \cite[\eg][]{1995ARA&A..33..581K, magorrian,
2000ApJ...539L...9F, 2000ApJ...539L..13G, marconihunt03,
2005MNRAS.362...25B, perseus1, mcnamrev, birzan08}. Following the lead
of observations, the current generation of large-scale structure
formation models now include some variation of a positive feedback
loop in which secondary processes like radiative cooling and star
formation are offset via heating by AGN activity
\cite[\eg][]{croton06, bower06, saro06, sijacki07}. While these models
are successful in reproducing the bulk properties of the Universe, the
details of AGN feedback are still poorly understood. One reason being
that additional observation-based constraints are needed on, for
example, (1) how AGN energy is transported beyond jets and dissipated
as heat, (2) the role/importance of magnetic fields within the hot,
diffuse gas of galaxy clusters and groups, (3) the connection between
radio galaxy properties and their host environments, and (4) the phase
of obscuration that possibly all AGN experience during SMBH
assembly. Further exploration of these topics comprises the research
proposed here.

{\bf{(1 \& 2) I propose, as a Leiden Fellow, to participate in, or
undertake, an observational study of the evolution of galaxy
clusters/groups under the influence of magnetic fields. Preferably,
such a study will be part of the LOFAR Cosmic Magnetism Key Project.}}
My research program has revealed that certain environmental conditions
must be met to promote feedback, namely that the mean entropy of the
environment hosting a SMBH must be $\la 30 ~\ent$
\cite[][see also Fig. \ref{fig:figs}]{d06, haradent, accept}. By a
coincidence of scaling, this is also the entropy level above which
thermal electron conduction is capable of stabilizing an environment
against the formation of thermal instabilities
\cite{conduction}. The connection of large-scale environmental
properties with the process of conduction hints at a mechanism for
heating an environment via AGN feedback energy and possibly toward the
establishment of a self-regulating feedback loop. Simulators
investigating magnetohydrodynamic (MHD) processes in groups and
clusters have seized upon these findings due to the connection between
MHD processes, conduction, and entropy structure.

It has been suggested that the MHD heat-flux-driven-buoyancy
instability (HBI) is an important process in clusters with core
cooling times $\ll \Hn^{-1}$ \cite{2008ApJ...677L...9P}. Full MHD
simulations have shown that the HBI, in conjunction with reasonable
magnetic field strengths, modest heating from an AGN, and subsonic
turbulence, can feasibly stabilize a core against catastrophic cooling
\cite{2009ApJ...703...96P, 2009arXiv0911.5198R}. In addition, recent
radio polarization measurements for the Virgo cluster of galaxies
suggest the large-scale magnetic field of Virgo's intracluster medium
(ICM) is radially oriented \cite{2009arXiv0911.2476P}, which may
result from the influence of the MHD magnetothermal instability
mechanism \cite[MTI,][]{2000ApJ...534..420B}. Because both HBI and MTI
can connect regions of differing temperatures via magnetic fields,
both mechanisms are capable of channeling heat throughout the ICM via
conduction. If HBI and MTI have a significant influence in clusters
and groups, then it furthers the case that conduction is a vital
component of understanding galaxy cluster evolution. However, the
observational evidence remains circumstantial, and these MHD processes
(and conduction) require additional indirect investigation via
magnetic field strengths and structures.

The Low Frequency Array (LOFAR) radio observatory recently began
collecting data, marking the beginning of a new era in the study of
ICM and intragroup medium (IGM) magnetic fields via polarimetry
\cite{2009ASPC..407...33A}. Polarization measurements made with LOFAR
will enable direct study of ICM \& IGM field strengths and structure
on scales as small as group cores and as large as cluster virial
radii. A systematic study of a representative cluster/group sample
(such as REXCESS \cite{rexcess} or HIFLUGCS \cite{hiflugcs1}) using
LOFAR will broaden our view of magnetic field demographics and how
they relate to cluster/group properties such as temperature gradients,
core entropy, AGN activity, and the presence of cold gas filaments. In
addition, it is possible to investigate the origin and evolution of
the fields: could the fields have been seeded by early AGN activity?
Are fields amplified by mergers or recent AGN outbursts? Is there
further evidence of galactic draping?  Understanding cluster magnetic
fields will also place constraints on ICM/IGM properties, such as
viscosity, which govern the microphysics by which AGN feedback energy
might be dissipated as heat, \eg\ via turbulence and/or MHD waves.

{\bf{(3) As a Leiden Fellow, I propose to pursue research into forming
a more comprehensive understanding of the connection between the
properties of radio galaxies, redshift, and host environments, with a
focus on galaxy evolution and structure formation.}} A study we have
recently completed \cite{pjet} investigates a more precise calibration
between AGN jet power (\pjet) and emergent radio emission (\lrad) for
a sample of giant ellipticals (gEs) and BCGs. We have found,
regardless of observing frequency, that $\pjet \propto 10^{16}
~\lrad^{0.7}~\lum$, which is in general agreement with models for
confined heavy jets (see Fig. \ref{fig:figs}). The utility of this
relation lies in being able to estimate total jet power from
monochromatic all-sky radio surveys for large samples of AGN at
various stages of their outburst cycles. When applied to the radio
luminosity function at various redshifts, the
\pjet-\lrad\ relation can be used to infer the kinetic heating of the
Universe over cosmic time, and as a consequence, can be used to infer
the total accretion history and growth of SMBHs over those same
epochs. Further, inferences can be drawn regarding the amount of
preheating AGN could have contributed as large-scale structure
evolved, a long-standing question in cosmological studies
\cite[\ie][]{2001ApJ...555..597B}.

What is the relationship between redshift, environment, and AGN
feedback energy? The answer thus far is unclear, partly as a result of
limited observational constraints. Undertaking a systematic study of
radio galaxy properties (\ie\ jet composition, morphologies, outflow
velocities, magnetic field configurations) as a function of
environment (\ie\ ambient pressure, host galaxy X-ray halo
compactness) can help address how AGN energetics couple to
environment. Such a study can also be used to suggest how accretion
onto SMBHs depends on small and large scale environment. To this end,
a study of the faint radio galaxy population using archival \chandra\
and VLA data would be useful, as would deep \chandra\ observations for
a sample of FR-I's -- a poorly studied population in the X-ray.

An interesting result which has emerged from our work shows that FR-I
radio galaxies (classified on morphology and not \lrad) appear to be
systematically more radiatively efficient than FR-II
sources. Ostensibly this may serve as an indicator of intrinsic
differences in radio sources (light and heavy jets), or that possibly
all jets are born light and become heavy on large scales due to
entrainment. One method of investigating this result more deeply is to
undertake a systematic study of the environments hosting radio
galaxies utilizing archival \chandra, \xmm, and VLA data.

As an extension of the observational work, and with a
model-independent method of estimating the kinetic properties of AGN
jets, of interest to me is re-visiting existing models for
relativistic jets in an ambient medium. Utilizing
observationally-based estimates of jet power, it is possible to
further investigate the growth of a radio source including
prescriptions for entrainment, scale-dependent changes in jet
composition, and shocks \cite[\'a la][]{1999MNRAS.309.1017W}. The
\pjet-\lrad\ relation also enables the investigation of relations
between observable mass accretion surrogates (\ie\ nuclear \halpha\
luminosity, molecular/dust mass, or nuclear X-ray luminosity) and AGN
energetics for the purpose of establishing clearer connections with
accretion mechanisms and efficiencies.

{\bf{(4) I propose a comprehensive multiwavelength study of obscured
AGN, their host galaxies, and the progenitors of the host galaxies to
better understand SMBH formation and subsequent AGN feedback.}} The
study of mechanical AGN feedback has advanced quickly in the last
decade primarily because the hot gas phase which this mode of feedback
most efficiently interacts is resolved with the current generation of
X-ray observatories. However, our understanding of radiative feedback,
and the associated early era of rapid SMBH growth, has not proceeded
as quickly. This is partly because cold, dusty gas is required for
high efficiency radiative feedback, but the presence of cold/dusty gas
is typically accompanied by significant optical obscuration which
prevents direct observational study
\cite{2009arXiv0911.3911A}. Luckily, the quality and availability of
multi-frequency data needed to probe the epoch of SMBH growth and
obscuration is poised to improve with new facilities and instruments
coming on-line (\ie\ LOFAR, Herschel, SCUBA-2, SOFIA, ALMA, NuStar,
Simbol-X), and a number of questions regarding the formation and
evolution of SMBHs can be pursued.

What is the evolutionary track from young, gas-rich, dusty galaxies to
present-day old, parched gEs? It has been argued that high-$z$ sub-mm
galaxies (SMGs) are the progenitors for low-$z$ Magorrian galaxies,
suggesting SMGs are useful for studying the co-evolution of SMBHs and
host galaxies. SMGs have also been shown to reside in very dense
environments and have high AGN fractions ($\ga 50\%$)
\cite{2005ApJ...632..736A}, so they are excellent for identifying the
rapidly cooling high-$z$ gas-rich regions where star formation and AGN
activity are occurring. Thus, SMGs identify a unique population to
follow-up with far-IR and X-ray spectroscopy to study epochs of early
AGN feedback and environmental cooling. It has also been posited that
SMGs are high-$z$ analogs of low-$z$ ultraluminous infrared galaxies
(ULIRGs). If this is the case, insight to ULIRG evolution can be
gained from studying SMGs. ULIRGs are an interesting population on
their own, one for which limited X-ray spectroscopic studies have been
undertaken. We know these systems to, on average, be dominated by star
formation, however, some systems also have significant contribution
from very dusty AGN, and these systems can be used to further
understand the nature of evolving gas-rich environments.

How does the transition of the nuclear region of a forming galaxy from
an obscured to unobscured state correlate with AGN feedback and SMBH
growth? As suggested by the low AGN fraction in the \chandra\ Deep
Fields, a significant population of obscured AGN must exist at higher
redshifts. One method of selecting unbiased samples of these objects
is to assemble catalogs of candidate AGN using hard X-ray (\ie\
NuStar), far-IR (\ie\ SOFIA), and sub-mm (\ie\ SCUBA-2)
observations. Because current models suggest the luminous quasar
population begins in an obscured state, and rapid acquisition of SMBH
mass may occur in this phase because of high accretion rates,
understanding the transition from obscured to unobscured states is
vital. How does accretion proceed and where does the accreting
material come from: gas cooling out of an atmosphere? Gas deposited by
merging companions? A related curiosity which has emerged in recent
years is the role of multiple AGN within the core of a host galaxy. At
a minimum, SMBH mergers occur on a timescale determined by dynamical
friction, which for a typical dense bulge is $\ga 1$ Gyr, which is
$\gg t_{\mathrm{cool}}$ of an obscuring atmosphere. If the SMBHs which
are merging have, or acquire, their own accretion disks, then it is
reasonable to question how the atmospheres surrounding a host galaxy
with multiple AGN is affected.

{\bf{If offered a position as a Leiden Fellow, I look forward to
forming collaborations with Leiden Observatory faculty, research
associates, and students on all levels to further the Observatory's,
and my own, science goals.}} The research proposal suggested here
covers a number of areas where Leiden Observatory has already invested
resources, not the least of which are the Herschel and LOFAR
missions. My interests in high-energy astrophysics, galaxies,
large-scale structure, and modeling directly relate to the work of
Prof. Franx, Prof. Jaffe, Prof. Katgert, Prof. Miley,
Prof. R\"ottgering, Prof. Schaye, and Prof. Snellen. Due to my
established history of working within highly collaborative
environments with teams composed of people from various personal and
professional backgrounds, it will be a natural extension of my
existing research program to begin working with other researchers at
Leiden Observatory. I am also excited at the prospect of working
within the LOFAR Consortium and with researchers at other LOFAR
affiliated institutions.

\begin{center}
  \begin{figure}[htp]
    \begin{minipage}[htp]{0.5\linewidth}
    \includegraphics*[width=\columnwidth, trim=28mm 7mm 40mm 17mm, clip]{k0rad.eps}
    \end{minipage}
    \begin{minipage}[htp]{0.5\linewidth}
      \includegraphics*[width=\textwidth, trim=30mm 5mm 40mm 15mm, clip]{pcav-lrad_1400.eps}
    \end{minipage}

    \caption{\footnotesize{\it{Left:}} BCG radio power vs. core
    entropy (\kna) for clusters with redshift $z < 0.2$. Orange
    symbols represent radio detections and black symbols are
    non-detection upper-limits. Circles are for NVSS observations and
    squares are for SUMSS observations. The blue squares with inset
    red stars or orange circles are peculiar clusters which do not
    adhere to the observed trend of being radio-loud below $\approx
    30~\ent$.  Green triangles denote clusters plotted using the
    2$\sigma$ upper-limit of the best-fit \kna. The vertical dashed
    line marks $\kna = 30 \ent$. {\it{Right:}} Cavity power
    vs. bolometric radio power estimated from 1.4 GHz monochromatic
    flux. Orange triangles represent the galaxy clusters and groups
    sample from \cite{birzan08}. Filled circles represent our sample
    of gEs with colors representing the cavity system grade of green =
    `definite,' blue = `moderate,' and red = `marginal.' The dotted
    red line represents the best-fit power-law relations presented in
    \cite{birzan08} using only the orange triangles. The dashed black
    lines represent our \bces\ best-fit power-law relations.}
    \label{fig:figs} \end{figure}
\end{center}

\bibliographystyle{unsrt}
\bibliography{cavagnolo}
 
\end{document}
