% more spacious
\documentclass[12pt]{article}
\pagestyle{empty}
\parindent 0pt
\parskip
\baselineskip
\setlength{\topmargin}{-0.30in}
\setlength{\oddsidemargin}{-0.30in}
\setlength{\evensidemargin}{-0.30in}
\setlength{\headheight}{0in}
\setlength{\headsep}{0.25in}
\setlength{\topskip}{0.25in}
\setlength{\textwidth}{6.9in}
\setlength{\textheight}{9.25in}
\pagestyle{myheadings}

% declare packages and options
\usepackage[T1]{fontenc}
\usepackage{subfig,epsfig,colortbl,graphics,graphicx,wrapfig,amssymb,common,mathptmx,multicol,natbib}

% start the document
\begin{document}

% header
\begin{center}
{\large \textbf{Dr. Kenneth W. Cavagnolo\\Research Summary}}
\rule{17cm}{2pt}
\end{center}
\normalsize

{\bfseries{Graduate Work}}

{\bfseries{Data Analysis Pipeline}}: My dissertation work made use of
350+ \chandra\ archival observations ($\approx 11.6$ Msec of
data). The massive undertaking necessitated the creation of a robust
reduction and analysis pipeline which 1) interacts with mission
specific software, 2) utilizes analysis software (\eg\ \xspec, \idl),
3) incorporates calibration and software updates, and 4) is highly
automated. Because my pipeline was written in a very general manner,
adapting the pipeline for use with pre-packaged analysis tools for
missions such as \xmm, \spitzer, and
\vla\ has been straightforward. Most importantly, my pipeline
deemphasizes data reduction and accords me the freedom to move quickly
into an analysis phase and generating publishable results.

{\bfseries{ICM Temperature Inhomogeneity}}: Galaxy clusters are a
useful cosmological tool only if we can infer cluster masses from
observable properties such as X-ray luminosity, X-ray temperature,
lensing shear, optical luminosity, or galaxy velocity
dispersion. Empirically, the correlation of mass to these observable
properties is well-established. If we could identify parameters
reflecting the degree of relaxation in the cluster, the utility of
clusters as cosmological probes would be improved by parameterizing
and reducing the scatter in mass-observable scaling relations. The
work of \cite{2001ApJ...546..100M} suggested a complementary measure
of substructure which does not depend on projected perspective and
could be combined with power ratio, axial ratio, and centroid
variation to yield a more robust metric for quantifying a cluster's
degree of relaxation. I studied this auxiliary measure: the bandpass
dependence in determining X-ray temperatures and what this dependence
tells us about the virialization state of a cluster. I found the
hard-band to full-band X-ray temperature ratio is statistically
connected to mergers and the presence of cool cores. This project
produced a first author paper \cite{xrayband}. The dissertation work
of David Ventimiglia (at MSU) is following-up on this result using SPH
simulations. In addition, Seth Bruch (at MSU) is investigating the
deviation from the mass-luminosity relation as a function of hard-band
to full-band temperature ratio as part of his dissertation work.

{\bfseries{ICM Entropy}}: The picture of the ICM entropy-feedback
connection which emerged from my dissertation research showed that
cluster cD radio luminosity and core H$\alpha$ emission are
anti-correlated with cluster central entropy (\kna), and that the
distribution of central entropy/central cooling times is bimodal. The
work revealed that clusters with central entropy $\lesssim 30 \ent$
show signs of star formation and AGN activity, while clusters above
this threshold unilaterally do not have star formation and exhibit
diminished AGN radio feedback. An entropy level of $\sim 30 \ent$ is
auspicious as it coincides with the Field length at which thermal
conduction can stabilize a cluster core against ICM condensation. The
bimodal entropy distribution is likely related to a combination of
effects: AGN feedback, thermal conduction, and to a lesser extent,
mergers. These results are highly suggestive that conduction in the
cluster core is very important to solving the long-standing problem of
how ICM gas properties are coupled to feedback mechanisms such that
the system becomes self-regulating. All of the results from this stage
of my dissertation were published in several papers: \cite{accept},
\cite{haradent}, and \cite{conduction}. There is currently a paper in
preparation (\cite{entscale}) which investigates entropy scaling
relations for the archival sample. There is also a project underway at
MSU utilizing \xmm\ Optical Monitor and \spitzer\ data to disentangle
which cDs residing in systems with $\kna \lesssim 30 \ent$ are star
formation dominated and which are AGN dominated.

{\bfseries{Postdoctoral Work}}

{\bfseries{AGN Jet Power and Radio Power}}: A long-standing problem in
observational and theoretical studies of energetic feedback from
supermassive black holes as it relates to large-scale structure
formation is estimating the total kinetic output from an active
galactic nuclei. These estimates have historically been made using
models of AGN jets and their impact on the surrounding
environment. However, the ICM has proven to be a robust bolometer for
measuring jet power courtesy of X-ray cavities. Using a sample of
clusters, groups, and isolated giant ellipticals with cavities I have
completed a project which measured and calibrated jet power versus
radio luminosity. This was an extension of the oft-cited
\cite{birzan04, birzan08} work. For the project, I observed 13 gEs (39
hrs. total) at P-band (327 MHz; 90 cm) using the new EVLA system, and
analyzed > 50 archival observations for 21 additional objects at a
variety of frequencies (1.4 GHz, 5 GHz, and 8 GHz). We found that jet
power scales with radio luminosity to the 0.7 power with a
normalization of $\sim 10^{43}$ erg s$^{-1}$, in accord with current
jet models. Our results have implications for galaxy formation, black
hole growth, and the mechanical heating of the universe. The results
are being published in a first author paper \cite{pjet}. We are using
the results of this study to explore the AGN kinetic luminosity
function over cosmic time, and to analyze a subset of peculiar FR-I
radio galaxies in more detail.

{\bfseries{AGN Outburst in RBS 797}}: The most powerful AGN outbursts
in the Universe are useful for placing constraints on possible fueling
mechanisms for the AGN. Systems such as MS 0735.6+7421, Hercules A,
and Hydra A stress the limits of cold gas accretion models (such as
Bondi accretion), and open the door to new mechanisms such as black
hole spin \cite{bhspin}. The galaxy cluster RBS 797 is another such
system. R797 has a pair of X-ray cavities which suggest the AGN
outburst in the system is of order $\sim 10^{45-46}$ erg s$^{-1}$,
making it one of the most powerful outbursts ever observed. I have
undertaken the detailed analysis of this peculiar system using X-ray,
radio, infrared, optical, and UV data. The results of this work are
being published in a first author paper \cite{r797}.

{\bfseries{The QSO IRAS 09104+4109}}: The transition from the era of
quasar-mode to radio-mode feedback in hierarchical structure formation
is still a poorly understood process. The transition likely coincides
with the formation of dense galactic environments like clusters and
groups, in addition to the formation of the most massive galaxies
which will become present-day BCGs. But we know this process does not
proceed unhindered, lest extremely blue cDs residing in
catastrophically cooling cluster cores will form. As a probe of how
the BCG assembly and ICM heating process proceeds in this era, we
obtained deep Chandra imaging of the famous and peculiar ULIRG/QSO
IRAS 09104+4109. As suspected, we directly imaged a pair of cavities
in the X-ray halo surrounding IRAS09. These cavities contain enough
energy to offset $\approx 25-35\%$ of the cooling occurring within the
cooling radius of the host galaxy cluster. This result suggest only
3-4 such outbursts are needed to halt cooling in the cluster and
freeze-out the formation of the BCG via gas condensing out of the
X-ray halo. The QSO in this system has also (as of our current
analysis) changed from optically thick to thin over the course of the
last 20 years, making this the first-ever observed changing-look
QSO. Even more exciting is the change in beaming direction of the AGN
within the last few kyrs which has dredged up cool gas from the core,
and is likely forming new stars as a result. This work has produced a
first author paper which is being finished right now and will be
submitted before the end of 2009
\cite{iras09}.

{\bfseries{Supermassive Cluster Survey}}: I am a member of the
Supermassive Cluster Survey and am responsible for the X-ray analysis
in the project. The study is headed-up by Rachel Mandelbaum and seeks
to better understand the scatter between X-ray and weak lensing masses
for a sample of 12 galaxy clusters. The project is in its final stages
with the mass determinations from the X-ray data currently being
performed. The project will produce at least one paper on which I am
one of the primary co-authors.

{\bfseries{MS 0735.6+7421}}

{\bfseries{Work with Other UWateroo Students}}: I am also actively
participating in several research projects with the graduate and
undergraduate students in our group. Clif Kirkpatrick is a senior
Ph.D. student under Dr. McNamara. I have co-authored two papers with
Clif \cite{a1664, hydrametal}, and we are working on a series of new
papers which present analysis of the 1D and 2D heavy metal
distributions for a large sample of galaxy clusters. We are
specifically interested in how metal transport is related to the
process of AGN feedback, and what we can discern about ICM metal
enrichment over cosmic time using these results.

Mina Rohanizadegan is a junior Ph.D. student under Dr. McNamara. We
are currently working on the analysis of X-ray data to learn about the
instantaneous accretion onto SMBHs at the center of galaxy
clusters. The aim is to place constraints on the fueling mechanism
which gives rise to the AGN jets which bore cavities into the
ICM. Mina is also finishing up a co-authored paper which presents
comparisons of models for AGN power generation via cold gas accretion
and black spin using the robust jet power measures from X-ray
cavities.

Brad Whuiska and Rob Myers are senior undergraduates working with
Dr. McNamara. Brad is measuring the core radius for a BCGs in the HST
archive. The aim of his study is to find the largest cores and analyze
them under the assumption that the large cores were created via
scouring (the process of stellar ejection via SMBH mergers). The work
is producing results which will be presented in a paper on which I
will be a co-author. Rob is undertaking the detection of BCG radio
sources using the NVSS and SUMSS all-sky radio surveys. These sources
will then be run through our $P_{jet}-P_{radio}$ relations, and an
estimate of the heating resulting from these AGN will be assessed. Rob
is also examining the connection with cluster properties such as X-ray
luminosity and temperature. Rob's work is also producing results which
will be presented in a co-author paper.

\scriptsize
\bibliographystyle{unsrt}
\bibliography{cavagnolo}
 
\end{document}
