% ubercompact
\documentclass[11pt]{article}
\setlength{\topmargin}{-0.15in}
\setlength{\oddsidemargin}{-0.12in}
\setlength{\evensidemargin}{0in}
\setlength{\headheight}{0in}
\setlength{\headsep}{0.0in}
\setlength{\topskip}{0.0in}
\setlength{\textwidth}{6.9in}
\setlength{\textheight}{9.5in}

% declare packages and options
\usepackage[T1]{fontenc}
\usepackage{subfig,epsfig,colortbl,graphics,graphicx,wrapfig,amssymb,common,mathptmx,multicol,natbib}

% start the document
\begin{document}

% header
\begin{center}
{\large \textbf{Dr. Kenneth W. Cavagnolo\\Summary of Research Program}}
\rule{17cm}{2pt}
\end{center}
\normalsize

The general process of galaxy cluster formation through hierarchical
merging is well understood, but many details, such as the impact of
feedback sources on the cluster environment and radiative cooling in
the cluster core, are not. My thesis research focused on studying
these details via X-ray properties of the intracluster medium (ICM) in
clusters of galaxies. I paid particular attention to ICM entropy
distribution \cite{d06, accept}, the process of cluster virialization
\cite{xrayband}, the role of AGN feedback in shaping large scale
cluster properties \cite{conduction}, and how feedback signatures
correlate with the properties of cluster cores \cite{haradent}. The
picture of the ICM entropy-feedback connection which emerged from
these studies was that feedback is part of a finely-tuned mechanism
with the requirement that the mean entropy ($K$) of the fueling
environment hosting a SMBH must be $K \la 30~\ent$. In the three years
since our first publication, the suite of papers from this work has
garnered 100+ citations.

My dissertation work made use of 400+ \chandra\ archival X-ray
observations ($\approx 13$ Msec of data). The massive undertaking
necessitated the creation of a robust reduction and analysis pipeline
which interacts with mission specific software (be it \chandra, \xmm,
\suzaku), utilizes analysis tool (\eg\ \xspec, \idl, \iraf), smoothly
incorporates calibration/software updates, is highly automated, and
continues to mature. My pipeline is written in a very general manner,
and adaptation of the pipeline for use with pre-packaged analysis
tools from other missions has been straightforward. Most importantly,
the pipeline deemphasizes data reduction and accords the user with the
freedom to move quickly into an analysis phase and generating
publishable results.

More recently, my research has focused on extreme, individual examples
of AGN feedback which are useful for confronting existing models of
AGN feedback and galaxy formation. I have recently completed studies
for two such systems: RBS 797 and IRAS 09104+4109. The rich dataset
available for RBS 797 indicates that the AGN outburst is inclined
along the line-of-sight, and that the outburst is one of the most
powerful ever observed, \eg\ the cluster-scale class of burst similar
to MS 0735.6+7421 \cite{ms0735}. Our detailed study has been useful
for further understanding how large outbursts affect hydrostatic
equilibrium in clusters, whether such outbursts can be driven by
classical gas accretion mechanisms, and if such outbursts can halt
cooling in a cluster. The results for RBS 797 are being presented in
an ApJ manuscript \cite{r797}. IRAS 09104+4109 is an enigmatic system
with a long literature indicating the galaxy is simultaneously
undergoing a variety of normally orthogonal phases of massive galaxy
formation. The completed study highlights results from a new \chandra\
observation which shows the AGN in IRAS09 excavating cavities and
uplifting cool gas from the core. This is unique for a radio-quiet,
radiatively dominated QSO, and demonstrates that massive galaxies like
BCGs and cDs may go through a brief phase of quasar-mode feedback
which is immediately followed by a radio-mode. IRAS09 is currently the
only system where both processes have been directly observed. These
results are being presented in a MNRAS manuscript \cite{iras09}.

Another of my studies which was recently completed \cite{pjet}
investigates a more precise calibration between AGN jet power (\pjet)
and emergent radio emission (\lrad) for a sample of giant ellipticals
(gEs) and BCGs. In this study we estimated \pjet\ using cavities
excavated in the ICM as bolometers, and measured \lrad\ at multiple
frequencies using new and archival VLA observations. We found,
regardless of observing frequency, that $\pjet \propto 10^{16}
\lrad^{0.7} \lum$, which is in general agreement with models for
confined heavy jets. The utility of this relation lies in being able
to estimate total jet power from monochromatic all-sky radio surveys
for large samples of AGN at various stages of their outburst
cycles. This should yield constraints on the kinetic heating of the
Universe over swathes of cosmic time, and as a consequence, can be
used to infer the total accretion history and growth of SMBHs over
those same epochs. An interesting result which has emerged from our
work is that FR-I radio galaxies (classified on morphology and not
\lrad) appear to be systematically more radiatively efficient than
FR-II sources. This may mean there are intrinsic differences in radio
sources (light and heavy jets), or possibly that all jets are born
light and become heavy on large scales due to entrainment.

I am also involved in a number of other projects with both
undergraduate and graduate students at the University of Waterloo, in
addition to projects with peers. I am a member of the Supermassive
Cluster Survey and am responsible for the X-ray analysis in the
project. The study is headed-up by Rachel Mandelbaum (Princeton) and
seeks to better understand the scatter between X-ray and weak lensing
masses for a sample of 12 galaxy clusters. I am also working on the
analysis and interpretation of the \chandra\ Large Project for MS
0735.6+7421. This project centers around 700 ksec of X-ray data which
is being used to study the properties of the most energetic AGN
outburst found to date. Brian McNamara has two Ph.D. students and two
undergraduates, all of which I am helping to guide in their research
on: finding large optical cores for BCGs in the Hubble archive, the
spin of SMBHs, instantaneous accretion mechanisms for SMBHs,
quantifying the 2D abundance distributions in galaxy clusters, and
finding/studying radio sources from all-sky surveys coincident with
clusters. I also maintain a close working relationship with my thesis
research group at Michigan State University (David Ventimiglia and
Seth Bruch), and am collaborating with two Ph.D. students on separate
papers relating to the deviation of galaxy clusters from mean
mass-scaling relations. There are also two on-going large radio
observation based projects (PI's Somak Raychaudhury and Alastair Edge)
which focus on low-frequency AGN emission in clusters and groups. I am
responsible for data acquistion, reduction, and analysis in one
project, and X-ray data analysis for another. It is expected that both
of these projects will merge with Herschel and LOFAR Key Projects.

\scriptsize
\bibliographystyle{unsrt}
\bibliography{cavagnolo}
 
\end{document}
