\documentclass[12pt]{cv}
\usepackage[colorlinks=true,linkcolor=blue,urlcolor=blue]{hyperref}
\usepackage[T1]{fontenc}
\usepackage{mathptmx,multicol,common}
\pagestyle{empty}
\parindent 0pt
\parskip
\baselineskip
\setlength{\topmargin}{-0.30in}
\setlength{\oddsidemargin}{-0.30in}
\setlength{\evensidemargin}{-0.30in}
\setlength{\headheight}{0in}
\setlength{\headsep}{0.25in}
\setlength{\topskip}{0.25in}
\setlength{\textwidth}{6.9in}
\setlength{\textheight}{9.25in}
\pagestyle{myheadings}

\begin{document}

\begin{center}
{\large \textbf{Dr. Kenneth W. Cavagnolo\\Curriculum Vitae}}
\rule{17.35cm}{2pt}
\scriptsize
{\it Last updated \today; \textcolor{blue}{Hyperlinks colored blue}}
\normalsize
\end{center}

\addresses
{
University of Waterloo\\
Department of Physics \& Astronomy\\
200 University Avenue West\\
Waterloo, Ontario, Canada N2L 3G1
}
{
517-285-9062\\
519-888-4567 ext. 35074\\
\href{mailto:kencavagnolo@gmail.com}{\tt{kencavagnolo@gmail.com}}\\
\href{http://www.pa.msu.edu/people/cavagnolo/}{\tt www.pa.msu.edu/people/cavagnolo/}\\
}

\begin{llist}

%---------------------------------------------------------------%
%---------------------------------------------------------------%

\sectiontitle{Education}
Michigan State University
\location{2005 - 2008}
Doctor of Philosophy, Astronomy \& Astrophysics

Michigan State University
\location{2002 - 2005}
Master of Science, Astronomy \& Astrophysics

Georgia Institute of Technology
\location{1998 - 2002}
Bachelor of Science, Physics

%---------------------------------------------------------------%
%---------------------------------------------------------------%

\sectiontitle{Research\\Experience}

Postdoctoral Fellow
\location{2008 - Present}
Supervisor: Brian McNamara, {\textit{Univ. of Waterloo}}
%Investigating AGN feedback in giant ellipticals, content of AGN jets, and operation of SMBHs.

Graduate Research Assistant
\location{2003 - 2008}
Supervisor: Megan Donahue, {\textit{Mich. St. Univ.}}
%Investigated feedback mechanisms, galaxy evolution, and the process of virialization in galaxy clusters.

Graduate Research Assistant
\location{2002 - 2003}
Supervisor: Jack Baldwin, {\textit{Mich. St. Univ.}}
%Analyzed echelle spectra for use in studies of {\textit{s}}-process abundances in planetary nebulae.

Undergraduate Research Assistant
\location{2000 - 2002}
Supervisor: James Sowell, {\textit{Geor. Inst. of Tech.}}
%Obtained orbital solution for the eclipsing Algol binary ET Tau via UBV light curves and spectroscopic radial velocity curves.

%---------------------------------------------------------------%
%---------------------------------------------------------------%

\sectiontitle{Research\\Program\\\& Interests}

My research program is focused on better understanding the formation
and evolution of cosmic structure via physical properties of the most
massive gravitationally-bound objects (galaxy groups and clusters) and
their sub-systems, \eg\ galaxies, supermassive black holes, active
galactic nuclei \& jets, and thermal instabilities (\ie\ gaseous
nebulae, star formation, gas accretion).

Additional areas of interest:\\
$\bullet$ Intracluster medium magnetic fields\\
$\bullet$ Diffuse radio halos\\
$\bullet$ Mechanical and radiative AGN feedback\\
$\bullet$ Cosmological studies via structure formation

%---------------------------------------------------------------%
%---------------------------------------------------------------%

\sectiontitle{Honors}
$\bullet$ Referee for ApJ, AJ, and CanTAC \hfill 2008 - Present\\
$\bullet$ Sherwood K. Haynes Award for Outstanding Graduate Student \hfill 2008\\
$\bullet$ MSU College of Natural Science Dissertation Fellow \hfill 2007 - 2008\\
$\bullet$ $\Sigma \Xi$ National Scientific Research Society Member\hfill 2009 - Present\\
$\bullet$ $\Sigma \Pi \Sigma$ National Physics Honor Society Member\hfill 2001 - Present\\
$\bullet$ American Astronomical Society Member\hfill 2002 - Present\\
$\bullet$ American Physical Society Member\hfill 2002 - Present\\
$\bullet$ Perimeter Institute Black Hole Reading Group Member\hfill 2009 - Present\\
$\bullet$ Dean's List, Georgia Tech \hfill 1998-2002

%---------------------------------------------------------------%
%---------------------------------------------------------------%

\sectiontitle{Scientific\\Skills}
$\bullet$ Extensive experience with X-ray and radio data analysis\\
$\bullet$ Familiarity with infrared, optical, and UV data analysis\\
$\bullet$ Understanding of \aips, \casa, \ciao, \iraf, \osa, and \sas\ analysis software\\
$\bullet$ Fluent in \html, \idl, \LaTeX, and \perl\ programming languages\\
$\bullet$ Worked with \clang, \flash, \fortran, \mysql, \python, \supmo, and \tcl\\
$\bullet$ Mastery of DOS, Linux, Macintosh, and Windows computing architectures\\
$\bullet$ Expert of computer maintenance, system construction, and troubleshooting
\markright{K.W.C., Curriculum Vitae}

%---------------------------------------------------------------%
%---------------------------------------------------------------%

\sectiontitle{Observing\\Experience}

Giant Metrewave Radio Telescope (GMRT)
\location{Jan. 2010}
168 hours observing 13 galaxy groups and 20 galaxy clusters

Chandra X-ray Observatory (CXO)
\location{Jan. 2009}
21 hours queued observation of IRAS 09104+4109

Very Large Array Radio Telescope (VLA)
\location{Dec. 2008}
39 hours observing 13 giant ellipticals

%---------------------------------------------------------------%
%---------------------------------------------------------------%

\sectiontitle{Accepted\\Proposals\\\& Grants}

GMRT Cycle 17, Co-I
\location{2009}
The Power and Particle Content of Extragalactic Radio Sources\\%; 70 hrs.
PI: Somak Raychaudhury, {\textit{Univ. Birmingham}}

GMRT Cycle 17, Co-I
\location{2009}
The Morphology of Steepest Spectrum Radio Sources in Galaxy Cluster Cores\\%; 109 hrs.
PI: Alastair Edge, {\textit{Durham Univ.}}

NOAO Cycle 2008A \& 2009A/B, Co-I
\location{2008-2009}
Normalization and scatter of the $M-T$ relation for supermassive galaxy clusters\\
PI: Rachel Mandelbaum, {\textit{Inst. for Adv. Study}}

GMRT Cycle 16, Co-I
\location{2008}
The Content of Giant Cavities in the IGM of Galaxy Clusters\\%; 40 hrs.
PI: Somak Raychaudhury, {\textit{Univ. Birmingham}}

CXO Cycle 10, PI
\location{2008}
IRAS 09104+4109: An Extreme Brightest Cluster Galaxy%; \$45k USD

CXO Cycle 10, Co-I
\location{2008}
Conduction and Multiphase Structure in the ICM\\%; \$100k
PI: Mark Voit, {\textit{Mich. St. Univ.}}

Spitzer Cycle 5, Co-I
\location{2008}
Star Formation and AGN Feedback in BCGs\\%; \$100k
PI: Megan Donahue, {\textit{Mich. St. Univ.}}

Spitzer Cycle 5, Co-I
\location{2008}
Infrared Properties of a Control Sample of Brightest Cluster Galaxies\\%; \$50k
PI: Megan Donahue, {\textit{Mich. St. Univ.}}

NSF Grant, Co-I
\location{2008}
Star Formation in the Universe's Largest Galaxies\\%; \$100k
PI: Mark Voit, {\textit{Mich. St. Univ.}}

CXO Cycle 9, Co-I
\location{2007}
Quantifying Cluster Temperature Substructure\\%; \$100k
PI: Mark Voit, {\textit{Mich. St. Univ.}}

VLA A-configuration Cycle, Co-I
\location{2007}
Radio Feedback in Clusters and Galaxies\\%; 39 hrs.
PI: Brian McNamara, {\textit{Univ. Waterloo}}

%---------------------------------------------------------------%
%---------------------------------------------------------------%

%\sectiontitle{Public\\Outreach}
%Astronomers Without Borders (AWB)
%\location{2009-present}
%Organized the affiliate chapter of AWB at the University of Waterloo.

%International Year of Astronomy (IYA)
%\location{2009}
%Helped with events in Waterloo, Ontario for IYA such as observing nights, public talks, and workshops.

%---------------------------------------------------------------%
%---------------------------------------------------------------%

\sectiontitle{Teaching\\Experience}
Substitute Instructor
\location{Fall 2006}
Course: ``Visions of the Universe''
%Gave lectures covering stellar evolution.

Honors Physics Tutor
\location{Summer 2003}
Course: ``Introductory Honors Physics I \& II''
%Tutored physics students taking classical mechanics, optics, and electromagnetism.

Graduate Teaching Assistant
\location{2002 - 2003}
Course: ``Visions of the Universe''
%Directed and supervised laboratories for introductory astronomy course.

%---------------------------------------------------------------%
%---------------------------------------------------------------%

\markright{K.W.C., Curriculum Vitae}

\sectiontitle{References}

Megan Donahue, \href{mailto:donahue@pa.msu.edu}{\tt donahue@pa.msu.edu} \hfill +00-1-517-884-5618\\
Tenured professor, Michigan State University

Brian McNamara, \href{mailto:mcnamara@uwaterloo.ca}{\tt mcnamara@uwaterloo.ca} \hfill +00-1-519-888-4567 ext. 38170\\
Tenured professor, University of Waterloo

G. Mark Voit, \href{mailto:voit@pa.msu.edu}{\tt voit@pa.msu.edu} \hfill +00-1-517-884-5619\\
Tenured professor, Michigan State University

Chris Carilli, \href{mailto:ccarilli@nrao.edu}{\tt ccarilli@nrao.edu} \hfill +00-1-505-835-7000\\
National Radio Astronomy Observatory Chief Scientist

Jack Baldwin, \href{mailto:baldwin@pa.msu.edu}{\tt baldwin@pa.msu.edu} \hfill +00-1-517-884-5611\\
Associate Chair for Astronomy, Michigan State University

Paul Nulsen, \href{mailto:pnulsen@cfa.harvard.edu}{\tt pnulsen@cfa.harvard.edu} \hfill +00-1-617-495-7043\\
Research Scientist, Center for Astrophysics at Harvard University

Mike Wise, \href{mailto:wise@science.uva.nl}{\tt wise@science.uva.nl} \hfill +31-0-521-595-564\\
LOFAR Radio Observatory Chief Scientist

%---------------------------------------------------------------%
%---------------------------------------------------------------%

\sectiontitle{Personal\\Interests}
$\bullet$ Academic: Environmental sciences, ``Cradle2Cradle'' design, and urban planning.\\
$\bullet$ Athletics: Triathlons, baseball, rock climbing, and Georgia Tech athletics.\\
$\bullet$ Hobbies: Backpacking, reading, building model airplanes, and raising bonsai trees.

\end{llist}

\end{document}
