% more spacious
\documentclass[12pt]{article}
\pagestyle{empty}
\parindent 0pt
\parskip
\baselineskip
\setlength{\topmargin}{-0.30in}
\setlength{\oddsidemargin}{-0.30in}
\setlength{\evensidemargin}{-0.30in}
\setlength{\headheight}{0in}
\setlength{\headsep}{0.25in}
\setlength{\topskip}{0.25in}
\setlength{\textwidth}{6.9in}
\setlength{\textheight}{9.25in}
\pagestyle{myheadings}
%% \markright{K.W.C., Statement of Interest}

% ubercompact
%\documentclass[11pt]{article}
%\setlength{\topmargin}{-0.15in}
%\setlength{\oddsidemargin}{-0.12in}
%\setlength{\evensidemargin}{0in}
%\setlength{\headheight}{0in}
%\setlength{\headsep}{0.0in}
%\setlength{\topskip}{0.0in}
%\setlength{\textwidth}{6.9in}
%\setlength{\textheight}{9.5in}

% declare packages and options
\usepackage[T1]{fontenc}
\usepackage{subfig,epsfig,colortbl,graphics,graphicx,wrapfig,amssymb,common,mathptmx,multicol,natbib}

% start the document
\begin{document}

% header
\begin{center}
{\large \textbf{Dr. Kenneth W. Cavagnolo\\Summary of Research and Interests}}
\rule{17cm}{2pt}
\end{center}
\normalsize

The energy liberated by active galactic nuclei (AGN) plays a vital
role in regulating the process of hierarchical structure formation
\cite[\eg][]{perseus1, croton06, bower06, saro06, sijacki07,
birzan08}. Observations robustly indicate most, if not all, galaxies
harbor a centralized SMBH which has co-evolved with the host galaxy
giving rise to the well-known bulge luminosity-stellar velocity
dispersion correlation \cite{1995ARA&A..33..581K, magorrian}. The
current galaxy formation paradigm couples the processes of
environmental cooling and heating via feedback loops
\cite{2002MNRAS.333..145N, mcnamrev}. In broad terms, feedback has
been segregated into two modes which occur at different cosmic epochs:
an early-time radiatively-dominated quasar mode, and a late-time
mechanically-dominated AGN mode. While this model is successful in
reproducing the bulk properties of the Universe, the details (\ie\
interplay and seeding of ICM magnetic field, particle reacceleration,
accretion processes, obscuration, power generation, energy
dissipation) are poorly understood. It is these details which interest
me most.

{\bf{Relevant Completed Research and Raised Questions}}

My past research has focused primarily on understanding the mechanical
feedback from AGN and the associated effects on galaxy clusters. I
have devoted particular attention to intracluster medium (ICM) entropy
distribution \cite{accept}, the process of cluster virialization
\cite{xrayband}, the mechanisms by which SMBHs might acquire fuel from
their environments to become AGN \cite{conduction}, and how those
mechanisms correlate with properties of clusters cores
\cite{haradent}. From these studies it has become apparent that
certain conditions must be established within a cluster core (and
presumably any environment which supplies fuel for a SMBH, \eg\ cool
coronae \cite{coronae}), namely that the mean entropy ($K$) of the
large-scale environment hosting a SMBH must be $K \la 30~\ent$.

Coincidentally, $\sim 30~\ent$ is the entropy scale above which
thermal electron conduction is capable of stabilizing gas against
thermal instability, hinting at a method for coupling AGN heating to
ICM cooling therby establishing a self-regulating feedback loop. This
result is made more interesting if the heat-flux-driven-buoyancy
instability \cite[HBI,][]{2008ApJ...677L...9P} is an important process
in clusters with central cooling times $\ll \Hn^{-1}$. Full
magnetohydrodynamic (MHD) simulations have shown that the HBI, in
conjunction with reasonable magnetic field strengths ($\sim 1~\mu$G),
modest heating from an AGN ($\sim 10^{43}~\lum$) and subsonic
turbulence, can feasibly stabilize a core against catastrophic cooling
\cite{2009ApJ...703...96P, 2009arXiv0911.5198R}. These theoretical
findings can be tested further with LOFAR.

Recent radio polarization measurements for Virgo cluster galaxies
suggest the large-scale magnetic field of Virgo's ICM is radial
oriented \cite{2009arXiv0911.2476P}. This result is tantalizing since
it additionally suggests ICM magentic fields may play an important
role in cluster evolution. A radially oriented magnetic field may
result from the magnetothermal instability
\cite[MTI,][]{2000ApJ...534..420B}. The MTI functions most efficiently
outside cluster cores and can direct heat over large radial ranges via
conduction. If indeed large-scale radial magentic fields are common in
clusters, then it furthers the case that MHD effects like MTI (via
conduction) are a vital component of understanding galaxy cluster
evolution. While Virgo is only one object from the cluster population,
a similar study for a sample of clusters using LOFAR can cast more
light on the \cite{2009arXiv0911.2476P} result.

LOFAR's order of magnitude improvement in angular resolution and
sensitivity at low radio frequencies opens a new era in studying ICM
magnetic fields via polarimetry
\cite{2009ASPC..407...33A}. Polarization measurements made with LOFAR
will enable direct detection of ICM field strengths and structure on
scales as small as cluster cores ($\la 50$ kpc) and as large as
cluster virial radii ($\sim$ few Mpc). A systematic study of a cluster
sample using LOFAR will expand our view of magnetic field demographics
and how they relate to cluster properties like temperature gradients,
core entropy, recent AGN activity, and the structure of cold gas
filaments in cluster cores. In addition, we will be able to
investigate the origin and evolution of the fields: were they seeded
by early AGN activity? Are they amplified by mergers? Is there
evidence of draping or entrainment? Understanding cluster magnetic
fields will also place constraints on ICM properties, like viscosity,
which govern the microphysics by which AGN feedback energy might be
dissipated as heat, \eg\ via turbulence and/or MHD waves.

{\bf{Relevant On-going Research and Raised Questions}}

\markright{K.W.C., Statement of Interest}
My on-going research has focused on the SMBH engines which underlie
AGN. Presently, I am completing papers on two intriguing objects, RBS
797 and IRAS 09104+4109, which are useful for exploring the extreme
ends of AGN outburst energetics and galaxy formation, respectively.
Another of my studies which was recently completed \cite{pjet}
investigates a more precise calibration between AGN jet power (\pjet)
and emergent radio emission (\lrad) for a sample of giant ellipticals
(gEs) and BCGs. In this study we estimated \pjet\ using cavities
excavated in the ICM as bolometers, and measured \lrad\ at multiple
frequencies using new and archival VLA observations. We found,
regardless of observing frequency, that $\pjet \propto 10^{16}
\lrad^{0.7} \lum$, which is in general agreement with models for
confined heavy jets. The utility of this relation lies in being able
to estimate total jet power from monochromatic all-sky radio surveys
for large samples of AGN at various stages of their outburst
cycles. This should yield constraints on the kinetic heating of the
Universe over swathes of cosmic time, and as a consequence, can be
used to infer the total accretion history and growth of SMBHs over
those same epochs.

An interesting result which has emerged from our work, and which is
investigated in \cite{2008MNRAS.386.1709C}, is that FR-I radio
galaxies (classified on morphology and not \lrad) appear to be
systematically more radiatively efficient than FR-II sources. This may
mean there are intrinsic differences in radio sources (light and heavy
jets), or possibly that all jets are born light and become heavy on
large scales due to entrainment. One way to investigate this result
more deeply is to undertake a systematic study of the environments
hosting radio galaxies utilizing archival \chandra\ and VLA data.

With tighter observational constraints on the kinetic properties of
AGN jets, of interest to me is re-visiting existing models for
relativistic jets in an ambient medium. Utilizing
observationally-based estimates of jet power, it is possible to better
understand the growth of a radio source including effects like
entrainment and evolution of jet composition \cite[\'a
la][]{1999MNRAS.309.1017W}. Another interesting use of a universal
\pjet-\lrad\ relation is using radio luminosities, lobe morphologies,
and age estimates to predict ambient gas pressures: $p_{\mathrm{amb}}
\propto (t_{\mathrm{age}}\lrad) / V_{\mathrm{radio}}$. This yields an
estimate of ambient densities when basic assumptions are made about
environment temperatures: $\rho_{\mathrm{amb}} \propto p/T$. With an
estimate of ambient densities, X-ray observing plans for very
interesting radio sources which reside in faint group environments
(\ie\ FR-I sources) can be robustly prepared. An observationally-based
estimate of \pjet\ also enables the investigation of relations between
observable mass accretion surrogates (\ie\ \halpha\ luminosity,
molecular/dust mass, or nuclear X-ray luminosity) and AGN energetics
for the purpose of establishing clearer connections with accretion
mechanisms and efficiencies.

{\bf{Future Research}}

\markright{K.W.C., Statement of Interest}
The study of mechanically-dominated AGN feedback has advanced quickly
in the last decade primarily because the process is readily observed
at low-redshifts, and the hot gas phase which this mode of feedback
most efficiently interacts is accessible with the current generation
of X-ray observatories. However, our understanding of radiative
feedback, and the associated early era of rapid SMBH growth, has not
progressed as quickly. This is mostly because cold/dusty gas is
required for high efficiency radiative feedback, but the presence of
cold/dusty gas is typically accompanied by significant optical
obscuration which prevents direct observational study
\cite{2009arXiv0911.3911A}. Luckily, the quality and availability
of multi-frequency data (radio, sub-mm, IR, optical, UV, and X-ray)
needed to probe the epoch of SMBH growth and obscuration is poised to
improve with new facilities and instruments coming on-line (\ie\
LOFAR, Herschel, SCUBA-2, SOFIA, ALMA, NuStar, Simbol-X, LWA). As such,
there are a number of questions regarding the formation and evolution
of SMBHs that I would like to pursue.


{\bf{(1) What is the evolutionary track from young, gas-rich, dusty
galaxies to present-day old, parched gEs?}} It has been argued that
high-$z$ sub-mm galaxies (SMGs) are the progenitors for low-$z$
Magorrian spirals and ellipticals, suggesting SMGs are useful for
studying the co-evolution of SMBHs and host galaxies. It has been
shown SMGs are found in very dense environments and have high AGN
fractions ($\ga 50\%$) \cite{2005ApJ...632..736A}, so they are
excellent for identifying the rapidly cooling high-$z$ gas-rich
regions where star formation and AGN activity can be fueled. Hence,
SMGs identify a population primed for follow-up with far-IR and X-ray
spectroscopy to study feedback and cooling in unique environments. In
total, SMGs may be the missing piece to understanding how SMBH
evolution and AGN activity regulate the transition from gas-rich
progenitors to ``red and dead'' ellipticals. It has also been posited
that SMGs are high-$z$ analogs of low-$z$ ULIRGs (objects typically
associated with the sites of merging gas-rich spirals). If this is the
case, insight to ULIRG evolution can be gained from studying
SMGs. ULIRGs are an interesting population on their own, one for which
limited X-ray spectroscopic studies have been undertaken. We know
these systems to, on average, be dominated by star formation, however,
some systems may have significant contribution from AGN, and these
systems can be used to further understand the nature of evolving
gas-rich systems.

{\bf{(2) How does SMBH activity depend on environment?}} Specifically
what is the relationship between redshift, environment, and feedback
energy? The answer thus far is unclear, most likely because the
influence of environment on AGN jets (through entrainment and
confinement) has been neglected or treated too simply in models. This
is where observations step in to place interesting constraints on the
problem. To this end, a study of the faint radio galaxy population
using archival \chandra\ and VLA data would be
interesting. Undertaking a systematic study of radio galaxy properties
(\ie\ jet composition, morphologies, outflow velocities, magnetic
field configurations) as a function of environment (\ie\ ambient
pressure, halo compactness) can help address how AGN energetics couple
to environment, which ultimately suggests how accretion onto the SMBH
couples to environment on small and large scales. Deep \chandra\
observations for a sample of FR-I's (a poorly studied population in
the X-ray) would also be useful for such a study, using the
\pjet-\lrad\ relation to define robust observation requests.

{\bf{(3) How does the transition from an obscured to unobscured state
correlate with AGN feedback and SMBH growth?}} As suggested by the low
AGN fraction in the \chandra\ Deep Fields, a significant population of
obscured AGN must exist at higher redshifts. One method of selecting
unbiased samples of these objects is to assemble catalogs of candidate
AGN using hard X-ray (\ie\ NuStar), far-IR (\ie\ SOFIA), and sub-mm
(\ie\ SCUBA-2) observations. Because current models suggest the
luminous quasar population begins in an obscured state, and rapid
acquisition of SMBH mass may occur in this phase because of high
accretion rates (possibly exceeding $10-100~L_{\mathrm{Edd.}}$),
understanding the transition from obscured to unobscured states is
vital. How does accretion proceed and where does the accreting
material come from: gas cooling out of the atmosphere? Gas stripped
from merging companions? Is accretion spherical and dictated by local
gas density (\eg\ Bondi)?  A key component which has been neglected in
AGN studies is the contribution of dust (which should be a significant
component in the atmospheres of obscured AGN) in increasing the
allowed Eddington luminosity for an accreting SMBH (\ie\
$L_{\mathrm{Edd.}} \propto \mu$). A curiosity which has emerged in
recent years which may be interesting, particularly during the
obscured stage when the merger rate is presumably high, is the role of
multiple SMBHs within the core of a host galaxy. At a minimum, SMBH
mergers occur on a timescale determined by dynamical friction, which
for a typical dense bulge is $\ga 1$ Gyr, which is $\gg
t_{\mathrm{cool}}$ of an obscuring atmosphere. If the SMBHs which are
merging have their own accretion disks, then it is reasonable to
question how the atmospheres surrounding a host galaxy with multiple
AGN is affected, particularly since the transition from obscured to
unobscured should proceed more quickly.

{\bf{Summary}}

\markright{K.W.C., Statement of Interest}
My research interests span the formation and evolution of SMBHs,
particularly during their accreting mode as AGN. The general picture
of structure formation is much clearer now than a decade ago, and the
influence of SMBHs is undeniably important. But, missing is a better
understanding of the details of accretion, interaction with ambient
atmospheres, and energy redistribution via AGN. To this end, more
observational constraints are needed, particularly using
multiwavelength datasets from upcoming missions. I am well-positioned
to make meaningful contributions in such a pursuit, and would like to
do so as a member of the LOFAR consortium.

\scriptsize
\bibliographystyle{unsrt}
\bibliography{cavagnolo}
 
\end{document}
