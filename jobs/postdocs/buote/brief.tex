\documentclass[11pt]{article}
\usepackage[colorlinks=true,linkcolor=blue,urlcolor=blue]{hyperref}
\usepackage[T1]{fontenc}
\usepackage{subfig,epsfig,colortbl,graphics,graphicx,wrapfig,amssymb,common,mathptmx}
\setlength{\topmargin}{-0.15in}
\setlength{\oddsidemargin}{-0.12in}
\setlength{\evensidemargin}{0in}
\setlength{\headheight}{0in}
\setlength{\headsep}{0.0in}
\setlength{\topskip}{0.0in}
\setlength{\textwidth}{6.6in}
\setlength{\textheight}{9.5in}
\pagestyle{empty}

\begin{document}
\begin{center}
\large
\textbf{Summary of Experience and Future Interests}
\normalsize
\end{center}

The general process of galaxy cluster formation through hierarchical
merging is well understood, but many details, such as the impact of
feedback sources on the cluster environment and radiative cooling in
the cluster core, are not. My thesis research has focused on studying
these details in clusters of galaxies via X-ray properties of the
ICM. Utilizing a 350 observation (276 clusters; 11.6 Msec) sample
taken from the CDA, I have paid particular attention to ICM entropy
distribution, the process of cluster virialization, and the role of
AGN feedback in shaping large scale cluster properties.

The picture of the ICM entropy-feedback connection emerging from my
research suggests cluster cD radio luminosity and core H$\alpha$
emission are anti-correlated with cluster central entropy. Following
analysis of 169 cluster radial entropy profiles
(Fig. \ref{fig:splots}), I have found an apparent bimodality in the
distribution of central entropy and central cooling times
(Fig. \ref{fig:tcool}) which is likely related to AGN feedback (and to
a lesser extent, mergers). I have also found that clusters with
central entropy $\lesssim 20$ keV cm$^2$ show signs of star formation
(Fig. \ref{fig:ha}) and AGN activity (Fig. \ref{fig:rad}), while
clusters above this threshold unilaterally do not have star formation
and exhibit diminished AGN radio feedback. This entropy level is
auspicious as it coincides with the Field length at which thermal
conduction can stabilize a cluster core against ICM
condensation. These results are highly suggestive that conduction in
the cluster core is very important to solving the long-standing
problem of how ICM gas properties are coupled to feedback mechanisms
such that the system becomes self-regulating.

The final phase of my thesis is focused on further understanding why
we observe bimodality, what role star formation is playing in the cluster
feedback loop, refining a model for how conduction couples feedback to
the ICM, and examining the peculiar class of objects which fall below
the Field length criterion but {\it do not} have star formation and/or
radio-loud AGN (blue boxes with red stars in two of the figures).

There are additional areas of my present research I'd like to expand
on in the future. {\bf(1)} To check if bimodality is archival bias, I am
submitting a \Chandra\ Cycle 10 observing proposal for a sample of
clusters which predictably fall into the $t_{\mathrm{cool}}$ and $K_0$
gaps. {\bf(2)} Two classes of peculiar objects warrant intensive
multiwavelength study: high-$K_0$ clusters with radio-loud AGN
(e.g. AWM4) and low-$K_0$ clusters without any feedback sources
(e.g. Abell 2107). The former likely have prominent X-ray corona,
while the latter may be showing evidence that extremely low entropy
cores inhibit the growth of gas density contrasts. {\bf(3)} Thus far I have
only focused on AGN which are radio-loud according to the 1.4 GHz eye
of NVSS, but recent work has shown AGN radio halos are very powerful
at low frequencies too. I'd like to know what the radio power is at
these wavelengths for (ideally) my entire thesis sample and see if the
$K_0$-radio correlation tightens. {\bf(4)} Using the near-UV sensitivity of
{\it XMM}'s Optical Monitor and the far-IR channels of {\it Spitzer},
I plan to propose a joint archival project to disentangle which $K_0
\lesssim 20$ cDs are star formation dominated and which are AGN
dominated.

In another part of my thesis research I studied an aspect-independent
measure of temperature inhomogeneity as a means for quantifying
cluster virialization state. I found the hard-band to full-band
temperature ratio was robustly correlated to mergers and the absence
of cool cores. This project touched on quantifying and reducing the
scatter in mass-observable relations to bolster the utility of
clusters as cosmology tools. I am eager to keep this area of my work
alive as we get closer to having access to enormous catalogs of SZ
detected clusters (\eg\ from {\it Planck}) which require X-ray follow-up. To
maximize the utility of these surveys, we must continue to investigate
scatter, evolution, and covariance in the X-ray observables which
serve as vital mass surrogates.

There are additional areas of study which I have not touched on in this
summary but still interest me. Such as the micro-physics of ICM
heating (e.g. turbulence and weak shocking), the thermalization of
mechanical work done by bubbles, and the importance of non-thermal
sources, like cosmic rays, in bubble heating. How prevalent are cold
fronts? Can they be used to robustly quantify ICM magnetic fields and
viscosity? Are they important in the feedback loop? How robust is the
``X-ray Butcher-Oemler Effect'' of Paul Martini if one studies a large
sample of clusters? Can we deduce a low-scatter relation (or at least
constrain one) between jet power and radio power? What is the
explanation for the thermal inefficiency of jets? Many questions
abound as a result of my thesis work, I hope to pursue the answers to
them as a post-doc with you at Irvine.

\begin{figure}[t]
    \begin{minipage}[t]{0.5\linewidth}
        \centering
	\includegraphics*[width=\textwidth, trim=28mm 8mm 30mm 10mm, clip]{splots}
        \caption{\small Radial entropy profiles of 169 clusters of
	galaxies in my thesis sample. The observed range of $K_0 \lesssim
	70$ keV cm$^2$ is consistent with models of episodic AGN
	heating. Color coding indicates global cluster temperature (in keV)
	derived from core excised apertures of size R$_{2500}$.}
	\label{fig:splots}
    \end{minipage}
    \hspace{0.1in}
    \begin{minipage}[t]{0.5\linewidth}
        \centering
        \includegraphics*[width=\textwidth, trim=28mm 8mm 30mm 10mm, clip]{tcool}
        \caption{\small Distribution of central cooling times for 169
	clusters in my thesis sample. The peak in the range of cooling
	times (several hundred Myrs) is consistent with inferred AGN
	duty cycles of both weak ($\sim 10^{40-50}$ ergs) and strong ($\sim
	10^{60}$ ergs) outbursts. However, note the distinct gap at $0.6-1$
	Gyr. An explanation for this bimodality does not currently exist.}
	\label{fig:tcool}
    \end{minipage}
    \hspace{0.1cm}
    \begin{minipage}[t]{0.5\linewidth}
        \centering
        \includegraphics*[width=\textwidth, trim=28mm 8mm 30mm 10mm, clip]{ha_k0}
        \caption{\small Central entropy plotted against H$\alpha$
	luminosity. Orange dots are detections and black boxes with left-facing
	arrows are non-detection upper-limits. Notice the characteristic entropy threshold for star
	formation of $K_0 \lesssim 20$ keV cm$^2$. This is also the entropy scale at
	which conduction no longer balances radiative cooling and condensation
	of low entropy gas onto a cD can proceed.}
        \label{fig:ha}
    \end{minipage}
    \hspace{0.1in}
    \begin{minipage}[t]{0.5\linewidth}
        \centering
        \includegraphics*[width=\textwidth, trim=28mm 8mm 30mm 10mm, clip]{k0rad}
        \caption{\small Central entropy plotted against NVSS radio
	luminosity. Orange dots are detections and black boxes with left-facing
	arrows are non-detection upper-limits. Radio-loud AGN clearly
	prefer low entropy environs but the dispersion at low luminosity is
	large. It would be interesting to radio date these sources as this
	figure may have an age dimension.}
        \label{fig:rad}
    \end{minipage}
\end{figure}
\end{document}
