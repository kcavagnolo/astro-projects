\documentclass[11pt]{article}
\usepackage[colorlinks=true,linkcolor=blue,urlcolor=blue]{hyperref}
\usepackage[T1]{fontenc}
\usepackage{subfig,epsfig,colortbl,graphics,graphicx,wrapfig,amssymb,common,mathptmx}
\setlength{\topmargin}{-0.15in}
\setlength{\oddsidemargin}{-0.12in}
\setlength{\evensidemargin}{0in}
\setlength{\headheight}{0in}
\setlength{\headsep}{0.0in}
\setlength{\topskip}{0.0in}
\setlength{\textwidth}{6.6in}
\setlength{\textheight}{9.5in}
\pagestyle{empty}

\begin{document}
\begin{center}
\LARGE
\textbf{Statement of Interest in LOFAR}
\normalsize
\end{center}

\cite{accept}

magnetic fields in clusters
inverse compton emission
radio halos

why interesting?

why lofar?

why me?

implications: conduction (hbu, mti), agn b-field seeding, magnetic
draping (gal form/evo), gas turbulence, are jets MHD dom?

------

On Jun 5, 2009, at 4:14 PM, Kenneth Cavagnolo wrote:

A recurring theme at this AGN meeting and at the Leiden meeting was
the importance of better understanding magnetic fields in clusters.  >
My naive assumption has been that LOFAR would be the weapon of choice
in studying the large-scale cluster magnetic field structure because
LOFAR covers very low frequencies and can have a big field of view --
hence lots of background sources to make rotation measures with. A
little digging this morning showed me this is true (plenty of talks on
this at LOFAR workshops), and that people are eager to get at this
project.  So here is my question: who has dibs on such a project, e.g.
is there a LOFAR key project P.I. for studying RM in clusters? If no
such decision has been made... can we do this???

-------

Hey Ken,

An excellent question with a somewhat involuted answer. So here goes.
There are currently 6 Key Science projects (KSPs) in LOFAR. In
particular there is the Surveys KSP that, as its name implies, wants
to do a big survey of the radio sky at various frequencies, make
catalogs, produce large-scale image mosaics, etc. The main science
goals involve, AGN population studies, clusters, high redshift radio
galaxies, lensing, etc. So originally, this project had first rights
to cluster observations.

There is also now a newer Magnetism KSP led by the German community.
This KSP wants to study the "magnetized universe" that means in
particular polarization studies of nearby remnants, galaxies, and the
milky way. They also want to do RM studies of clusters, but that
starts to creep into the original Surveys KSP territory.

So this particular area is the focus of a bit of negotiation. In the
end, I suspect it will be a collaborative effort among a few of us
from each of the two KSPs. I'm in the cluster working group who will
be working on this one from the Surveys side.

Its probably worth noting that the "dibs" won't be absolute. None of
the KSPs will have ultimate data rights in perpetuity nor will they be
allowed to lay claim to all targets of a given type or particular
kinds of analysis. So these two KSPs might get to do the RM studies of
clusters *first*, but there will be the opportunity for others to
propose for more objects too.

Again, I'll have access to this data as part of the Surveys KSP
therefore we all will. So I think the simple answer to your question
is "yes", we can do this one. We may have to negotiate the details of
what to observe and how with others and the final author lists might
include a bigger list of names initially, but so what.

Cheers,
Mike

\bibliographystyle{unsrt}
\bibliography{cavagnolo}
 
\end{document}
