\documentclass[12pt]{cv}
\usepackage[colorlinks=true,linkcolor=blue,urlcolor=blue]{hyperref}
\usepackage[T1]{fontenc}
\usepackage{mathptmx,multicol,common}
\pagestyle{empty}
\parindent 0pt
\parskip
\baselineskip
\setlength{\topmargin}{-0.30in}
\setlength{\oddsidemargin}{-0.30in}
\setlength{\evensidemargin}{-0.30in}
\setlength{\headheight}{0in}
\setlength{\headsep}{0.25in}
\setlength{\topskip}{0.25in}
\setlength{\textwidth}{6.9in}
\setlength{\textheight}{9.25in}
\pagestyle{myheadings}

\begin{document}

\begin{center}
{\large \textbf{Kenneth W. Cavagnolo\\Curriculum Vitae}}\\
\rule{17.35cm}{2pt}\\
\footnotesize
{\it Last updated \today; \textcolor{blue}{Hyperlinks colored blue}}
\normalsize
\end{center}

\addresses
{
University of Waterloo\\
Department of Physics \& Astronomy\\
200 University Avenue West\\
Waterloo, Ontario, Canada N2L 3G1
}
{
Office: 519-888-4567 x35074\\
Home: 517-285-9062\\
E-mail: \href{mailto:kencavagnolo@gmail.com}{\tt{kencavagnolo@gmail.com}}\\
Web: \href{http://www.pa.msu.edu/people/cavagnolo/}{\tt www.pa.msu.edu/people/cavagnolo/}\\
}

\begin{llist}

%---------------------------------------------------------------%
%---------------------------------------------------------------%

\sectiontitle{Education}
\employer{{\bf Michigan State University}}
\location{2005 - 2008}
Ph.D., Astronomy \& Astrophysics\\
Dissertation: ``Investigating Feedback and Relaxation in Clusters of Galaxies\\ with the Chandra X-ray Observatory''\\
Advisor: Dr. Megan Donahue\\
GPA: 4.0/4.0

\employer{{\bf Michigan State University}}
\location{2002 - 2005}
M.S., Astrophysics, {\it Magna Cum Laude}\\
Thesis: ``Entropy Profiles of Cooling Flow Clusters''\\
Advisor: Dr. Megan Donahue\\
GPA: 3.44/4.0

\employer{{\bf Georgia Institute of Technology}}
\location{1998 - 2002}
B.S., Physics, {\it Magna Cum Laude}\\
Senior Thesis: ``Analysis of the Eclipsing Binary ET Tau''\\
Advisor: Dr. James Sowell\\
GPA: 3.55/4.0

%% \employer{{\bf South Forsyth High School}}
%% \location{1998}
%% College Preparatory Degree, {\it Summa Cum Laude}\\
%% Cumming, Georgia, USA\\
%% GPA: 3.9/4.0

%% \employer{{\bf Standardized Test Scores}}
%% GRE General (2001):
%% Verbal Score 550
%% Verbal Percentile 73
%% Quantitative Score 730
%% Quantitative Percentile 80
%% Analytical Score 660
%% Analytical Percentile 74 

%% SAT (1997):
%% Verbal 670
%% Math 650

%% ACT (1997): 
%% English 27
%% Reading 30
%% Natural Science 28
%% Mathematics 29
%% Composite 29 

%---------------------------------------------------------------%
%---------------------------------------------------------------%

\sectiontitle{Honors}
$\bullet$ Referee for Astrophysical Journal, Astronomical Journal, \& CanTAC \hfill 2008 - Present\\
$\bullet$ Sherwood K. Haynes Award for Outstanding Graduate Student \hfill 2008\\
$\bullet$ MSU College of Natural Science Dissertation Fellow \hfill 2007 - 2008\\
$\bullet$ American Astronomical Society Member\hfill 2002 - Present\\
$\bullet$ American Physical Society Member\hfill 2002 - Present\\
$\bullet$ Sigma Pi Sigma, National Physics Honor Society \hfill 2001 - Present\\
$\bullet$ Perimeter Institute Black Hole Reading Group \hfill 2009 - Present\\
$\bullet$ Dean's List, Georgia Tech \hfill 1998-2002

%---------------------------------------------------------------%
%---------------------------------------------------------------%

\sectiontitle{Research\\Interests}
$\bullet$ Galaxy Clusters\\
$\bullet$ Galaxy Formation\\
$\bullet$ Star Formation in Massive Galaxies\\
$\bullet$ Black Hole Formation and Evolution\\
$\bullet$ Large Scale Structure and Cosmology

%---------------------------------------------------------------%
%---------------------------------------------------------------%

\sectiontitle{Research\\Experience}
{\sc \bf{Postdoctoral Fellow}}
\location{2008 - Present}
Supervisor: Dr. Brian McNamara, {\textit{Univ. of Waterloo}}\\
Investigating AGN feedback in giant ellipticals, content of AGN jets,\\
and energy supply of SMBHs.

{\sc \bf{Supermassive Cluster Survey, Member}}
\location{2007 - Present}
Lead: Dr. Rachel Mandelbaum, {\textit{IoA}}\\
Weak lensing collaboration to measure the scatter between\\
X-ray observables and true projected mass.

{\sc \bf{Graduate Research Assistant}}
\location{2003 - 2008}
Supervisor: Dr. Megan Donahue, {\textit{Mich. St. Univ.}}\\
Investigated feedback mechanisms, galaxy evolution, and the process of\\
virialization in galaxy clusters.

\markright{K.W. Cavagnolo, Curriculum Vitae}

{\sc \bf{Graduate Research Assistant}}
\location{2002 - 2003}
Supervisor: Dr. Jack Baldwin, {\textit{Mich. St. Univ.}}\\
Analyzed echelle spectra for use in studies of {\textit{s}}-process\\
abundances in planetary nebulae.

{\sc \bf{Undergraduate Research Assistant}}
\location{2000 - 2002}
Supervisor: Dr. James Sowell, {\textit{Georgia Tech}}\\
Obtained orbital solution for the eclipsing Algol binary ET Tau via\\
UBV light curves and spectroscopic radial velocity curves.

%---------------------------------------------------------------%
%---------------------------------------------------------------%


\sectiontitle{Scientific\\Skills}
$\bullet$ Profound skills in reducing and analyzing data taken with \chandra\ X-ray Observatory.\\
$\bullet$ Extensive experience customizing and debugging \ciao\ and \caldb.\\
$\bullet$ Familiarity with analysis packages: \aips, \casa, \iraf, MOPEX, and \pyraf.\\
$\bullet$ Experience preparing radio observations with JObserve.\\
$\bullet$ Fluent in \html, \idl, \LaTeX, and \perl.\\
$\bullet$ Worked with \clang, \flash, \fortran, \mysql, \python, \supmo, and \tcl.\\
$\bullet$ Mastery of multiple computing architectures: DOS, Linux, Macintosh, and Windows.\\
$\bullet$ Expert of computer troubleshooting, maintenance, and system construction.

%---------------------------------------------------------------%
%---------------------------------------------------------------%

\sectiontitle{Observing\\Experience}

{\sc \bf{Giant Metrewave Radio Telescope (GMRT)}}
\location{2010}
Pune, India

{\sc \bf{Chandra X-ray Observatory (CXO)}}
\location{2009}
Boston, MA, USA

{\sc \bf{Very Large Array Radio Telescope (VLA)}}
\location{2008}
Socorro, NM, USA

%---------------------------------------------------------------%
%---------------------------------------------------------------%

\sectiontitle{Proposals\\\& Grants}

{\sc \bf{GMRT Cycle 17, Co-I}}
\location{2009}
The Power and Particle Content of Extragalactic Radio Sources%; 70 hrs.

{\sc \bf{GMRT Cycle 17, Co-I}}
\location{2009}
The Morphology of the Steepest Spectrum Radio Sources in the Cores of Clusters of Galaxies -
Echoes Of AGN Feedback?%; 109 hrs.

{\sc \bf{GMRT Cycle 16, Co-I}}
\location{2008}
The Content of Giant Cavities in the IGM of Galaxy Clusters%; 40 hrs.

{\sc \bf{Chandra Cycle 10, PI}}
\location{2008}
IRAS 09104+4109: An Extreme Brightest Cluster Galaxy%; \$45k

{\sc \bf{Chandra Cycle 10, Co-I}}
\location{2008}
Conduction and Multiphase Structure in the ICM%; \$100k

{\sc \bf{Spitzer Cycle 5, Co-I}}
\location{2008}
Star Formation and AGN Feedback in BCGs%; \$100k

{\sc \bf{Spitzer Cycle 5, Co-I}}
\location{2008}
Infrared Properties of a Control Sample of Brightest Cluster Galaxies%; \$50k

{\sc \bf{NSF Grant, Co-I}}
\location{2008}
Star Formation in the Universe's Largest Galaxies%; \$100k

{\sc \bf{Chandra Cycle 9, Co-I}}
\location{2007}
Quantifying Cluster Temperature Substructure%; \$100k

%---------------------------------------------------------------%
%---------------------------------------------------------------%

\sectiontitle{Public\\Outreach}
{\sc \bf{Astronomers Without Borders (AWB)}}
\location{2009-present}
Organized the affiliate chapter of AWB at the University of\\
Waterloo.

{\sc \bf{International Year of Astronomy (IYA)}}
\location{2009}
Helped with events in Waterloo for IYA such as observing nights,\\
public talks, and workshops.

%---------------------------------------------------------------%
%---------------------------------------------------------------%

\sectiontitle{Teaching\\Experience}
{\sc \bf{Substitute Instructor}}
\location{Fall 2006}
Course: ``Visions of the Universe''\\
Gave lectures covering stellar evolution, supernovae, white dwarves,\\
neutron stars, and black holes.

{\sc \bf{Physics Tutor}}
\location{Summer 2003}
Course: ``Introductory Honors Physics I \& II''\\
Tutored physics students taking introductory physics courses such as\\
classical mechanics, optics, and electromagnetism.

{\sc \bf{Graduate Teaching Assistant}}
\location{2002 - 2003}
Course: ``Visions of the Universe''\\
Directed and supervised laboratories for non-calculus based\\
astronomy course.

%---------------------------------------------------------------%
%---------------------------------------------------------------%

\markright{K.W. Cavagnolo, Curriculum Vitae}

\sectiontitle{References}

{\sc Dr. Megan Donahue}\\
(517) 884-5618; \href{mailto:donahue@pa.msu.edu}{\tt donahue@pa.msu.edu}\\
Tenured professor; Michigan State University

{\sc Dr. Brian McNamara}\\
(519) 888-4567 ext. 38170; \href{mailto:mcnamara@uwaterloo.ca}{\tt mcnamara@uwaterloo.ca}\\
Tenured professor; University of Waterloo

{\sc Dr. G. Mark Voit}\\
 (517) 884-5619; \href{mailto:voit@pa.msu.edu}{\tt voit@pa.msu.edu}\\
Tenured professor; Michigan State University

{\sc Dr. Jack Baldwin}\\
 (517) 884-5611; \href{mailto:baldwin@pa.msu.edu}{\tt baldwin@pa.msu.edu}\\
Associate Chair for Astronomy; Michigan State University

{\sc Dr. Chris Carilli}\\
(505) 835-7000; \href{mailto:ccarilli@nrao.edu}{\tt ccarilli@nrao.edu}\\
National Radio Astronomy Observatory Chief Scientist

{\sc Dr. Mike Wise}\\
+31 0 521 595 564; \href{mailto:wise@science.uva.nl}{\tt wise@science.uva.nl}\\
LOFAR Radio Observatory Chief Scientist

{\sc Dr. Paul Nulsen}\\
(617) 495-7043; \href{mailto:pnulsen@cfa.harvard.edu}{\tt pnulsen@cfa.harvard.edu}\\
Research Scientist; Center for Astrophysics, Harvard University

%---------------------------------------------------------------%
%---------------------------------------------------------------%

\sectiontitle{Personal\\Interests}
$\bullet$ Academic: Environmental sciences, ``Cradle2Cradle'' design, and urban planning.\\
$\bullet$ Athletics: Triathlons, baseball, rock climbing, and Georgia Tech athletics.\\
$\bullet$ Hobbies: Backpacking, reading, building model airplanes, and raising bonsai trees.\\

\end{llist}

\end{document}
