\documentclass[11pt]{article}
\usepackage[colorlinks=true,linkcolor=blue,urlcolor=blue]{hyperref}
\usepackage{subfig,epsfig,colortbl,graphics,graphicx,wrapfig,mathrsfs,common}
\pagestyle{myheadings}
\font\cap=cmcsc10
\setlength{\topmargin}{-0.25in}
\setlength{\oddsidemargin}{-0.1in}
\setlength{\evensidemargin}{0in}
\setlength{\headheight}{0in}
\setlength{\headsep}{0.1in}
\setlength{\topskip}{0.5in}
\setlength{\textwidth}{6.5in}
\setlength{\textheight}{9.25in}

\begin{document}
\begin{center}
\LARGE
\textbf{A Multiwavelength Study of Pre-AGN Outburst BCGs in X-ray
Luminous Clusters and Groups of Galaxies}
\end{center}
\normalsize

\section{Motivation}

The general process of galaxy cluster formation through hierarchical
merging is well understood, but many details, such as the impact of
feedback sources on the cluster environment and radiative cooling in
the cluster core are not. The most massive baryonic component of a
galaxy cluster is the intracluster medium (ICM) which has densities of
a few $10^{-3} \mathrm{~cm}^{-3}$ and temperatures ranging
$2-10$ keV. The ICM is observed as a luminous X-ray ``cloud'' which
pervades the cluster over several cubic Mpcs, and has varying
amounts of substructure which depend on the cluster dynamical
state. Clusters which have not experienced a major merger event for a
few Gyrs, the ICM is observed to be stratified, spherically symmetric,
and in approximate hydrostatic equilibrium.\\

In the absence of processes other than gravity, all clusters will
be scaled versions of each other with the defining characteristic
being cluster mass. The mass of a cluster determines the depth of the
gravitational potential well and the well depth in turn defines global
cluster properties such as ICM luminosity and temperature. But the
Universe does not operate so simplistically as to have gravity be the
only process which defines a cluster's properties. Portions of the
inhomogeneous ICM have cooling times shorter than the age of the
Universe and are thus subject to radiative cooling. As the ICM
radiatively cools, massive ``cooling flows'' of cold gas streaming to
the bottom of the clusters potential well should result. But these
cooling flows have never been observed with the predicted rates of
hundreds to thousands of solar masses per year, instead the flows are
more like trickles depositing at best a few solar masses per year in
the core.\\

I must digress for a moment to introduce the concept of ICM entropy
on which I rely heavily in this proposal. Density and temperature are
not the truest representation of the ICM's physical state because they
are most influenced by the underlying dark matter potential. These two
quantities are therefore not ideal for understanding the thermal
history of the ICM. But when we define a new quantity, $K=Tn^{-2/3}$,
which we call entropy (but is in actuality the adiabatic constant), we
have captured the gas thermal history because only heating and cooling
can alter entropy. Compression and expansion of the ICM does not
change $K$ the way it affects temperature. Even better, the radial
entropy distribution of the ICM is very telling because a
potential well is like a giant entropy sorting device: the ICM is only
convectively stable when $dK/dr \geq 0$. If a cluster were a sealed box
of gravitation-only processes whom's core was dominated by a cooling
flow, then the radial entropy distribution would strictly follow a
power-law relation across all radii.\\

In my thesis I have conclusively proven that a central entropy
pedestal (Fig. \ref{fig:splots}) is a universal feature of
clusters. But what secondary process(es) removes low-entropy gas from
the core of a cluster? Recently, bubbles blown in the ICM by AGN have
been observed in numerous clusters (\cite{2007ARA&A..45..117M}). These
bubbles contain the energy necessary to retard core cooling and thus
to eradicate low-entropy gas in the core. In support of this AGN
feedback framework is the picture of the ICM entropy-feedback
connection emerging from my thesis. My work suggests BCG radio
luminosity (Fig. \ref{fig:radk0}) and BCG H$\alpha$ emission
(Fig. \ref{fig:hak0}) are anti-correlated with cluster central
entropy. I have found that clusters with central entropy $\leq 20$ keV
cm$^2$ exhibit star formation and radio AGN activity in the BCG while
clusters above this threshold unilaterally do not have star formation
and exhibit diminished AGN radio feedback. Corroborating observations
of strong blue gradients as a function of decreasing central entropy
in BCGs has also been seen by Rafferty et. al. (2007 in press).\\

A satisfactory solution to the cooling flow problem and flattening of
entropy profiles has been found with AGN, but a nasty problem of these
models has also emerged: {\bf how does AGN feedback couple to ICM
properties such that the system becomes self-regulating?} The central
entropy level mentioned above, $K_0 = 20$ keV cm$^{2}$, is auspicious
as it coincides with the Field length at which thermal conduction can
stabilize a cluster core (Fig. \ref{fig:conduction}). It is possible
conduction is the physical mechanism by which ICM gas properties are
coupled to feedback mechanisms such that the system becomes
self-regulating. If an AGN outburst does not boost the ICM entropy of
the core to greater than $\sim 20 $ keV cm$^{2}$ then the core will
not be subject to stabilization by conduction and cooling plus
condensation will proceed, ultimately leading to future AGN outbursts
and prodigious star formation in the BCG.\\

But while this framework is exciting, it needs refinement to produce a
robust model which explains the fueling of AGN, resulting AGN heating,
couples the ICM with AGN, and properly predicts the effect on
BCG star formation rates. Observations of an AGN interacting with the
hot cluster atmosphere (a very useful diagnostic) abound
(i.e. \cite{2004ApJ...607..800B}), but the best diagnostic -- direct
observation of on-going shock heating from a young AGN -- are rare, 
most likely short-lived, or impossible to see because of observation
resolution limits or gas and dust enshroudment. I suggest we attack
the problem of understanding AGN feedback in clusters by inverting the
problem. Instead of basing models only on observations of feedback
{\it after} it has occurred, we should also add constraints by
focusing on the cluster core environment {\it before} an AGN outburst
is very near. But how does one go about selecting a sample of objects
which we know to be near the beginning of a feedback cycle?\\

Looking closer at Figure \ref{fig:radk0} and Figure \ref{fig:hak0} you
will notice blue boxes with red stars. These points indicate clusters
with BCGs which {\bf do not} have star formation or radio AGN, but
have central entropies below the conduction stabilization limit of $K_0
= 20$ keV cm$^2$. Henceforth I refer to all objects in my proposed
sample as ``clusters'', but in strict terms the sample is a mix of
large groups, poor clusters, and rich clusters. There is the added
curiosity that none of these clusters exhibit signs of mergers or
dramatic AGN feedback (bubbles or cavities) which might explain their
current physical state (the exception being Abell 133 which has a cool
``tongue'' and radio relic/ghost whose origin is not clear
\cite{2004ApJ...616..157F}). General properties of my proposed sample
are listed in Table \ref{tab:sample}.\\

My proposition is that these clusters are in the final stages of low
entropy gas condensation in the core which ultimately feeds an AGN
outburst cycle. This is a very important stage in the
life-cycle of the most massive galaxies in the Universe, and
performing an observational ``book-keeping'' of all the low entropy gas
should yield insight into how massive galaxy formation is ultimately
truncated via AGN feedback. {\bf As a IPM fellow, I propose to
conduct a multiwavelength (X-ray, UV, IR, radio) observational
campaign on this peculiar class of galaxy clusters/groups, which have
$K_0 \leq 20$ keV cm$^2$ and either no radio emission or no star
formation, in an effort to understand how AGN are fueled, how stars
form in the most massive galaxies in the Universe, and to further
develop a robust model for coupling the ICM to AGN feedback.} This
work will characterize the gas environments of low entropy systems
which have thus far been under-studied, and will hopefully yield new
constraints on the properties of gas accreting onto the SMBH in the
BCG of these systems. Additionally this project will address the
hypothesis that conduction plays a crucial role in setting the entropy
scale below which star formation occurs, and it will also involve
better understanding how AGN feedback energy is dissipated into the
ICM.

\markright{K.W. Cavagnolo, IPM Fellowship Proposal}

\section{Observations}
Multi-phase gas can only be properly accounted for by observing across
multiple wavelength regimes. The coldest gas will be brightest in the
Far-IR while the hottest gas has already been observed in the X-ray
(the impetus for this proposal). As reference, Table
\ref{tab:observations} lists data which currently resides in publicly
accessible archives and will serve to streamline this project.

\subsection{X-ray}
While high resolution X-ray spectra of cool core clusters have
disproved the presence of massive quantities of gas expected in
classical cooling flow models (\cite{2006PhR...427....1P}), they have
also proven it is possible to detect the heavy element recombination lines
of species such as FeXVII and OVIII. These are an important
diagnostic for calculating the true mass of gas cooling out of the ICM
and into a BCG. For my proposed project I will utilize the RGS
instruments on-board {\it XMM-Newton} to collect high resolution
spectra of the cores of my sample. To a lesser degree, {\it Chandra}'s
HETG and LETG instruments will be utilized for cross-calibration
purposes. As part of my thesis I have already analyzed the {\it
Chandra} X-ray data for all these clusters, which resulted in
temperature, density, entropy, pressure, and mass analysis. The {\it
XMM-Newton} archive will also be utilized to cross-calibrate these
results.

Only three clusters require new {\it XMM} observations: Abell
1204, Abell 2107, and Abell 2556. All of these objects are very
luminous (L$_X > 10^{44}$), cool (T$_X < 4$ keV), and relatively
nearby (z $<$ 0.18) making them ideal candidates for an {\it XMM}
observing proposal. There is excellent scientific justification for
observing these clusters, and it is reasonable to believe I will be
awarded observing time.

\subsection{Ultraviolet and Optical}
As an additional constraint on the gas cooling rate, data from {\it
XMM-Newton}'s Optical Monitor will be utilized. This data will be used
to calculate star formation rates (SFRs) which will compliment and further
constrain cooling rates calculated from RGS/HETG/LETG X-ray spectra
and X-ray imaging spectroscopy of {\it Chandra}. The research group
I'm presently working with has already utilized this technique of
calculating SFRs from UV excess for 2A 0335+096
(\cite{2007AJ....134...14D}).

\subsection{Infrared}
Not all UV indicators are unbiased in the estimation of SFRs
(\cite{2007ApJ...666..870C}) because ionizing radiation can be
reprocessed by dust enshrouding star forming regions. This is where
utilization of {\it Spitzer} MIPS and IRAC data will provide an
additional, tighter constraint on calculating SFRs and fully
accounting for the fate of the cool gas condensing in these low
entropy systems. Far-IR photometry will also be useful in determining
if a heavily dust obscured AGN is present in systems which no radio
AGN was previously detected. Removal of AGN contamination will also be
important for accurate determination of SFRs using IR data. The data
presently available in the {\it Spitzer} archive does not provide
full-coverage of my sample, but the observations I am suggesting in
the Mid-IR do not require the use of cryogen. Thus it is reasonable to
believe submission of an observing proposal of $\sim 10-12$ hrs. in
next year's {\it Spitzer} AOR cycle has a good chance of being
accepted.

\subsection{Radio}
Analysis of radio data is not predicated upon acceptance of observing
time as all the data needed already resides in the VLA archives. For
the BCGs which do exhibit radio AGN activity it will be important to
reanalyze VLA FIRST or SUMMS data to calculate the energetics of the
outflows from the AGN. If conduction is an important mechanism in
distributing heat throughout a cool core, then one should also expect
AGN kinetic energy to preferentially interact with low density gas
and leaving high density gas (in which star formation is most likely
to occur) intact. This effect has already been observed in a few
cluster cores, but for very powerful AGN outbursts and in BCGs which
are not currently experiencing star formation. Uplifting of low
entropy gas by an AGN will also need to be investigated as such an
effect can potentially skew the total cooling mass too high. I have
only listed VLA FIRST high resolution observations which are
publicly available in Table \ref{tab:observations}, but radio data
does exist for all my sources, it will need to be acquired through
personal request.

\section{Analysis: Star Formation Rates and Accounting for Cool Gas}
This project is an attempt to directly combine model independent
measurements for the majority of cooling gas in the cores of low
entropy clusters and characterize the environments in which star
formation and AGN feedback {\it are initiated}. The project I am
proposing can be summarized as an attempt to construct the broadest
wavelength spectral energy distribution (SED) possible for a sample of
extremely interesting cluster BCGs; this is a book-keeping project
which will be useful for making a better framework of understanding
what happens to cool gas just before and just after an AGN feedback
cycle starts.

To constrain the star formation rates in the BCGs of my proposed
sample clusters I have chosen a multi-pronged attack. The first task is
to calculate X-ray cooling rates for the cluster core from the
temperature and density distributions of the ICM. This does not
require much effort as I have already done this for my thesis. The
X-ray cooling rate establishes the expected amount of gas which will be
condensing onto the BCG. Recall, these systems have been chosen
because they are dramatically different than their well-heated
brethren: low entropy with no radio AGN or no star formation. All
three of which are indicators that the cluster is nearing or long
removed ($t \geq 1$ Gyr) from the last major heating event.

An additional robust constraint on the properties of the cooling gas
will be calculated using the high resolution spectra from RGS. The
relative strengths of lines from heavy elements species such as Fe, O,
Mg, Ne, and Si will be used to calculate cooling rates and also will
serve as a temperature diagnostic for gas which is not
spectroscopically resolved by EPIC/MOS and ACIS. Of course these
emission lines will not be present with the strengths classic cooling
flow models predict, but this is precisely the physical characteristic
I am proposing to study.

Complimentary Mid/Far-IR data from {\it Spitzer} and UV data from {\it
XMM}-OM will also be used to calculate cooling rates. The emission
lines from polycyclic aromatic hydrocarbons (PAHs) longward of
$\sim2\mu$ are very useful for detecting dusty starbursts which
would otherwise be missed. For the starlight which is not being
reprocessed by dust, it can be directly observed in the near-UV and
will show up as a luminosity excess. Assuming an IMF and using the
L$_{UVW1}$-L$_J$ relation of \cite{2005ApJ...635L...9H} (which is
founded upon archival {\it XMM}-OM data) any net UV excess will be
used to calculate star formation rates.

Taking all these constraints together will provide a stringent
accounting of the state of the lowest entropy gas in these
clusters. Taking this sample as a distinctly unique set of clusters
and comparing them against other well studied BCGs and cluster cores
should reveal if they are different in some fundamental way:
\begin{enumerate}
\item Do these clusters truly have cores with fewer stars?
\item Is the ionization state of the ICM in these cluster cores
different than typical cool core clusters?
\item Are the effects of conduction glaringly lacking?
\item Is there any evidence for very young dust obscured AGN?
\item Do the gas kinematics suggest bulk motions which are atypical of
other cool core clusters?
\end{enumerate}
The answer to these questions are interesting for several reasons.
If these cluster cores aren't different from other cool core clusters,
meaning we find more stars than are indicated by H$\alpha$ emission or
AGN which aren't radio-loud, then the standard feedback framework is
intact. I will be able to slot these clusters into the expected
feedback life-cycle. But, what if these clusters have cores that are
different from other cool core clusters. One must ask, ``what the heck
{\it is} going on?''  Why aren't there stars and/or AGN? The standard
framework of cluster feedback has a glaring hole in explaining this
class of object and a good opportunity to plug that hole will have
presented itself.

I can only conjecture at the moment. Imagine an overdense gas parcel
buried in a very low entropy medium. As the gas parcel sinks it will
reach a region of higher density, stop, and then buoyantly
rise. Reproducing this process over an $\approx$ 10 kpc region should
result in all the overdense gas parcels being washed out while the
overall entropy of the region continues to lower. The result would be
a low entropy core with no overdense regions which could produce stars
or gas streams which could reach the SMBH and initiate AGN
activity. But is this process stable? Does it require large magnetic
suppression of conduction? I'm not sure at the moment, this
idea requires more thought.

A logical next step will be in the utilization of radio data to
understand the kinematics of the radio AGN (for sources which have
them). Within this sample of peculiar clusters should arise another
dichotomy- clusters with AGN and clusters without. The star formation
rates and cooling rates should be different between these two classes
of cluster. This will be related to the plasma outflows of the AGN and
may also be related to the weak shocking of the ICM as the jets
supersonically move through the dense cluster core environment. An
additional exciting use for the radio data will be to ``radio date''
the AGN in an effort to assign an age for these sources. Several
diagnostics will be useful: 1) dynamic age -- the age inferred from
kinematics of the source (i.e. distance of jets/lobes from nuclear
source); 2) synchrotron age -- the age inferred from the break frequency
of the radio spectrum (presuming no in situ re-acceleration of the jet
outflow). Ages of the radio sources will be a very interesting piece
of information as it relates directly to the timescales of ICM
condensation and feedback energy thermalization.

\section{Benefits to IPM Science and Conclusion}
This proposal seeks to address many outstanding issues of the feedback
regulated cooling in the cores of clusters. All of the questions I've
raised in this proposal relate directly to research currently underway
at OSU/IPM.

How truncation/downsizing of the massive end of the galaxy luminosity
function (GLF) proceeds is not well constrained from
observation. However, cosmological simulations which include cooling
and feedback are beginning to generate gas distributions which agree
with observation (\cite{2007ApJ...668....1N}). These simulations also
have GLFs which have the appropriate density of massive galaxies and
have cluster BCGs which are blue and not red
(\cite{2006MNRAS.365...11C}).

But the successes of models including AGN feedback have also served to
highlight the failings of these models observationally. Specifically,
we do not currently understand how feedback energy is thermalized
within the ICM and most importantly we do not understand how AGN are
fueled via cooling from the ICM. As I presented in the introduction,
there is good observational and theoretical reason to believe
conduction is the answer to both these problems. But in the context of
conduction being the solution we will need to account for the peculiar
class of clusters which I have presented in this proposal which 
have low central entropy, no radio AGN (radio being the favored
mode of energy transport), and/or no star formation (as inferred from
H${\alpha}$ measurements). IPM has an established stake in all these
areas of research and will benefit greatly from endorsing me as a
fellow to further study them.

The research groups at OSU/IPM have an established reputation for use of
{\it XMM-Newton} which will only serve to make my proposed project all
the more fruitful. Specifically, I have not touched on the
implications of star formation on metal enrichment in the cores of
these clusters, turbulent mixing of the ICM from AGN, and the
ICM magnetic fields and their role in shaping the interaction of the
AGN with the ICM. All of these topics have been, or are currently,
under study by someone at IPM. Many other archival projects could be
produced from this proposal by expanding the scope to include the
above mentioned topics.

Along with this proposal are also many other topics which could
potentially be studied. For example, do AGN blown bubbles contain a very low
density non-relativistic thermal plasma or are they truly voids in the
ICM (potentially an SZ experiment)? Maybe bubbles contain cosmic rays,
a possibility which will make for an interesting GLAST project. How do
bubbles rise to distances $\geq 100$ kpc without being shredded by
instabilities? The answer to this question will likely entail better
understanding of ICM $\vec{B}$ fields, with their origin being either
from preheating, AGN deposition, or a combination of both.

In conclusion, the class of peculiar galaxy clusters I have presented
warrant extensive study in their own right, but a uniform, systematic
study of these objects will have broad implications for better
understanding AGN feedback and star formation in the most massive
galaxies in the Universe.

\clearpage
\begin{figure}[t]
    \begin{minipage}[t]{0.5\linewidth}
        \centering
        \includegraphics*[width=\textwidth, trim=26mm 8mm 30mm 10mm, clip]{splots}
        \caption{\small Entropy profiles for 164 clusters of galaxies
        in my thesis sample. The range of central entropies is
        consistent with models of episodic AGN heating which regulate
        the presence of low entropy gas in cluster cores. The
        so-called ``cooling flow'' problem does not appear to be a problem any
        longer.}
        \label{fig:splots}
    \end{minipage}
    \hspace{0.1in}
    \begin{minipage}[t]{0.5\linewidth}
        \centering
        \includegraphics*[width=\textwidth, trim=28mm 8mm 30mm 10mm, clip]{k0rad}
        \caption{\small Central entropy derived in my thesis work
        plotted against radio luminosity calculated using
        NVSS. Clusters without radio source detections are represented
	by upper-limits (left pointing arrows). The eight clusters in my
	sample without radio detections and $K_0 < 20$ are plotted as blue
	boxes with red stars.}
        \label{fig:radk0}
    \end{minipage}
    \hspace{0.1in}
    \begin{minipage}[t]{0.5\linewidth}
        \centering
        \includegraphics*[width=\textwidth, trim=28mm 8mm 30mm 10mm, clip]{ha_k0}
        \caption{\small Central entropy derived in my thesis work
        plotted against H$\alpha$ luminosity calculated from data in
	\cite{1999MNRAS.306..857C}. Clusters without H$\alpha$ source
	detections are represented by upper-limits (left pointing arrows). The
	five clusters in my sample without H$\alpha$ detections and $K_0 < 20$
	are plotted as blue boxes with red stars.}
        \label{fig:hak0}
    \end{minipage}
    \hspace{0.1in}
    \begin{minipage}[t]{0.5\linewidth}
        \centering
        \includegraphics*[width=0.95\textwidth, trim=0mm 0mm 0mm 0mm, clip]{conduction}
        \caption{\small 
	Toy entropy profiles plotted as a function of radius and overlaid with
	dashed lines representing cooling and conduction equivalence for two
	suppression factors. Above the dashed lines conduction is effective
	and condensation cannot occur, the opposite is true below the
	lines. $K_0 < 20$ keV cm$^2$ is the break point at which the Field
	length criterion suggests gas condensation (i.e. star formation and
	condensation on the SMBH) can proceed. Reproduced courtesy of Dr. G.
	Mark Voit.}
        \label{fig:conduction}
    \end{minipage}
\end{figure}

\begin{center}
General Properties of Cluster/Group Sample
\small
\begin{tabular}{lcccccccc}
\hline
Name & RA         & Dec           & $z$   & $T_X$ & $K_0$      & $L_{bol.}$    & $L_{H\alpha}$ & $L_{Radio}$\\
---  & hr:min:sec & $^\circ:':''$ & ---   & keV   & keV cm$^2$ & 10$^{44}$ cgs & 10$^{39}$ cgs & 10$^{39}$ cgs\\
\hline
\hline
Abell 133            & 01:02:41.756 & -21:52:49.79 & 0.0558 & 3.71 & 17.26 & 6.46 & 6.00    & $<$2.03\\
Abell 1204           & 11:13:20.419 & +17:35:38.45 & 0.1706 & 3.63 & 15.31 & 3.92 & 58.6    & $<$22.2\\
EXO 0422-086$^{(g)}$ & 04:25:51.271 & -08:33:36.42 & 0.0397 & 3.41 & 13.77 & 0.65 & $<$0.10 & 45.2\\
Abell 2556           & 23:13:01.413 & -21:38:04.47 & 0.0862 & 3.57 & 12.38 & 1.43 & planned & $<$5.07\\
Abell 2107           & 15:39:39.113 & +21:46:57.66 & 0.0411 & 3.82 & 11.11 & 3.02 & $<$1.65 & $<$0.79\\
Abell 2029           & 15:10:56.163 & +05:44:40.89 & 0.0765 & 8.20 & 10.50 & 13.9 & $<$5.91 & 822\\
AWM7$^{(pc)}$        & 02:54:27.631 & +41:34:47.07 & 0.0172 & 3.71 & 10.21 & 4.07 & planned & $<$0.18\\
MKW4$^{(g)}$         & 12:04:27.218 & +01:53:42.79 & 0.0198 & 2.16 & 6.86  & 0.46 & planned & $<$0.24\\
ESO 5520200$^{(g)}$  & 04:54:52.318 &  18:06:56.52 & 0.0314 & 2.34 & 5.89  & 1.40 & planned & $<$0.62\\
MS J1157.3+5531      & 11:59:52.295 & +55:32:05.61 & 0.0810 & 3.28 & 5.54  & 0.12 & planned & $<$4.44\\
Abell 2151$^{(pc)}$  & 16:04:35.887 & +17:43:17.36 & 0.0366 & 2.90 & 4.27  & 1.41 & $<$1.30 & 0.59\\
RBS 533$^{(pc)}$     & 04:19:38.111 & +02:24:35.62 & 0.0123 & 1.29 & 2.56  & 0.17 & $<$0.14 & 0.61\\
%Centaurus            & 12:48:48.926 & -41:18:44.75 & 0.0109 & 3.96 & 1.34  & 1.74 & 15.0    & 6.79\\
%Abell 1991           & 14:54:31.620 & +18:38:41.48 & 0.0565 & 5.40 & 1.33  & 1.35 & 5.48    & 32.5\\
\hline
\end{tabular}
\label{tab:sample}
\end{center}
\footnotesize
Notes: Clusters are ordered by decreasing $K_0$; (g) denotes a group;
(pc) denotes a poor cluster. Clusters without H${\alpha}$ are scheduled
to be observed using the SOAR Optical Imager (SOI) which is on MSU's
SOAR Telescope in Cerro Pach\'{o}n, Chile.\\
\normalsize

\begin{center}
Exisiting Archival Data
\small
\begin{tabular}{lcccccccccc}
\hline
Name & X-ray & Inst. & Grating & Inst. & UV & Inst. & IR & Inst. & Radio & Inst.\\
\hline
\hline
Abell 133       & Y & {\it XMM} & Y & RGS & Y & {\it XMM}-OM & N & ------- & Y & priv.\\
Abell 1204      & N & --- & N & --- & N & ------ & Y & {\it Spitzer} & Y & {\it VLA}\\
EXO 0422-086    & Y & {\it XMM} & Y & RGS & Y & {\it XMM}-OM & N & ------- & Y & priv.\\
Abell 2556      & N & --- & N & --- & N & ------ & N & ------- & Y & priv.\\
Abell 2107      & N & --- & N & --- & N & ------ & N & ------- & Y & priv.\\
Abell 2029      & Y & {\it XMM} & Y & RGS & Y & {\it XMM}-OM & Y & {\it Spitzer} & Y & {\it VLA}\\
AWM7            & Y & {\it XMM} & Y & RGS & Y & {\it XMM}-OM & N & ------- & Y & priv.\\
MKW4            & Y & {\it XMM} & Y & RGS & Y & {\it XMM}-OM & Y & {\it Spitzer} & Y & priv.\\
ESO 5520200     & Y & {\it XMM} & Y & RGS & Y & {\it XMM}-OM & N & ------- & Y & priv.\\
MS J1157.3+5531 & Y & {\it XMM} & Y & RGS & Y & {\it XMM}-OM & N & ------- & Y & priv.\\
Abell 2151      & Y & {\it XMM} & Y & RGS & Y & {\it XMM}-OM & Y & {\it Spitzer} & Y & {\it VLA}\\
RBS 533         & Y & {\it XMM} & Y & RGS & Y & {\it XMM}-OM & N & ------- & Y & priv.\\
%Centaurus       & Y & {\it XMM} & Y & RGS & Y & {\it XMM}-OM & N & ------- & Y & priv.\\
%Abell 1991      & Y & {\it XMM} & Y & RGS & Y & {\it XMM}-OM & N & ------- & Y & priv.\\
\hline
\end{tabular}
\label{tab:observations}
\end{center}
\footnotesize
Notes: All clusters have publically available {\it Chandra} data.
\normalsize


\clearpage
\bibliographystyle{unsrt}
\bibliography{cavagnolo}
 
\end{document}
