% more spacious
\documentclass[12pt]{article}
\pagestyle{empty}
\parindent 0pt
\parskip
\baselineskip
\setlength{\topmargin}{-0.30in}
\setlength{\oddsidemargin}{-0.30in}
\setlength{\evensidemargin}{-0.30in}
\setlength{\headheight}{0in}
\setlength{\headsep}{0.25in}
\setlength{\topskip}{0.25in}
\setlength{\textwidth}{6.9in}
\setlength{\textheight}{9.25in}
\pagestyle{myheadings}

% declare packages and options
\usepackage[T1]{fontenc}
\usepackage{subfig,epsfig,colortbl,graphics,graphicx,wrapfig,amssymb,common,mathptmx,multicol,natbib}

% start the document
\begin{document}

% header
\begin{center}
{\large \textbf{Dr. Kenneth W. Cavagnolo\\Statement of Research Interests}}
\rule{17cm}{2pt}
\end{center}
\normalsize

{\bfseries{Introduction}}

The energy liberated by active galactic nuclei (AGN) plays a vital
role in regulating the process of hierarchical structure formation
\cite[\eg][]{perseus1, croton06, bower06, saro06, sijacki07,
birzan08}. Observations robustly indicate most, if not all, galaxies
harbor a centralized SMBH which has co-evolved with the host galaxy
giving rise to the well-known bulge luminosity-stellar velocity
dispersion correlation \cite{1995ARA&A..33..581K, magorrian}. The
current galaxy formation paradigm couples the processes of
environmental cooling and heating via feedback loops
\cite{2002MNRAS.333..145N, mcnamrev}. In broad terms, feedback has
been segregated into two modes which occur at different cosmic epochs:
an early-time radiatively-dominated quasar mode, and a late-time
mechanically-dominated AGN mode. While this model is successful in
reproducing the bulk properties of the Universe, the details (\ie\
accretion processes, obscuration, power generation, energy
dissipation) are poorly understood. It is these details which interest
me most.

{\bfseries{Relevant Completed and On-going Research}}

My research has focused primarily on understanding the mechanical
feedback from AGN and the associated effects on galaxy clusters. I
have devoted particular attention to intracluster medium (ICM) entropy
distribution \cite{accept}, the process of cluster virialization
\cite{xrayband}, the mechanisms by which SMBHs might accrete fuel from
an environment \cite{conduction}, and how those mechanisms correlate
with properties of clusters cores \cite{haradent}.

From these studies it has become apparent that certain conditions must
be established within a cluster core (and presumably any environment
which supplies fuel for a SMBH, \eg\ cool coronae \cite{coronae}),
namely that the mean entropy ($K$) of the large-scale environment
hosting a SMBH must be $K \la 30~\ent$. Coincidentally, this is the
entropy scale above which thermal electron conduction is capable of
stabilizing a cluster core against the formation of thermal
instabilities, hinting at a method for coupling AGN feedback energy to
the ICM and establishing a self-regulating feedback loop. This result
is made more interesting if the heat-flux-driven-buoyancy instability
\cite[HBI,][]{2008ApJ...677L...9P} is an important process in clusters
with central cooling times $\ll \Hn^{-1}$. Full MHD simulations have
shown that the HBI, in conjunction with reasonable magnetic field
strengths ($\sim 1~\mu$G), modest heating from an AGN ($\sim
10^{43}~\lum$) and subsonic turbulence, can feasibly stabilize a core
against catastrophic cooling \cite{2009ApJ...703...96P,
2009arXiv0911.5198R}. In addition, recent radio polarization
measurements for Virgo cluster galaxies suggest the large-scale
magnetic field of Virgo's ICM is radial oriented
\cite{2009arXiv0911.2476P}. This result is tantalizing since it
suggests the magnetothermal instability \cite{2000ApJ...534..420B} may
be operating within Virgo, furthering the case that conduction is a
vital component of understanding galaxy cluster evolution. In total,
these studies touch on a larger subject which is of great interest to
me: magnetic fields in clusters.

LOFAR came online fall 2009, and the order of magnitude improvement in
angular resolution and sensitivity at low radio frequencies opens a
new era in studying ICM magnetic fields via polarimetry
\cite{2009ASPC..407...33A}. Polarization measurements made with LOFAR
will enable direct detection of ICM field strengths and structure on
scales as small as cluster cores ($\la 50$ kpc) and as large as
cluster virial radii ($\sim$ few Mpc). A systematic study of a cluster
sample using LOFAR will expand our view of magnetic field demographics
and how they relate to cluster properties like temperature gradients,
core entropy, recent AGN activity, and the structure of cold gas
filaments in cluster cores. In addition, we will be able to
investigate the origin and evolution of the fields: were they seeded
by early AGN activity? Are they amplified by mergers? Is there
evidence of draping or entrainment? Understanding cluster magnetic
fields will also place constraints on ICM properties, like viscosity,
which govern the microphysics by which AGN feedback energy might be
dissipated as heat, \eg\ via turbulence and/or MHD waves.

My most recent research has focused on the SMBH engines which underlie
AGN. One study recently completed \cite{pjet} investigates a more
precise calibration between AGN jet power (\pjet) and emergent radio
emission (\lrad) for a sample of giant ellipticals (gEs) and BCGs. In
this study we estimated \pjet\ using cavities excavated in the ICM as
bolometers, and measured \lrad\ at multiple frequencies using new and
archival VLA observations. We found, regardless of observing
frequency, that $\pjet \propto 10^{16} \lrad^{0.7} \lum$, which is in
general agreement with models for confined heavy jets. The utility of
this relation lies in being able to estimate total jet power from
monochromatic all-sky radio surveys for large samples of AGN at
various stages of their outburst cycles. This should yield constraints
on the kinetic heating of the Universe over swathes of cosmic time,
and as a consequence, can be used to infer the total accretion history
and growth of SMBHs over those same epochs.

An interesting result which has emerged from our work, and which is
investigated in \cite{2008MNRAS.386.1709C}, is that FR-I radio
galaxies (classified on morphology and not \lrad) appear to be
systematically more radiatively efficient than FR-II sources. This may
mean there are intrinsic differences in radio sources (light and heavy
jets), or possibly that all jets are born light and become heavy on
large scales due to entrainment. One way to investigate this result
more deeply is to undertake a systematic study of the environments
hosting radio galaxies utilizing archival \chandra\ and VLA data.

With tighter observational constraints on the kinetic properties of
AGN jets, of interest to me is re-visiting existing models for
relativistic jets in an ambient medium. Utilizing
observationally-based estimates of jet power, it is possible to better
understand the growth of a radio source including effects like
entrainment and evolution of jet composition \cite[\'a
la][]{1999MNRAS.309.1017W}. Another interesting use of a universal
\pjet-\lrad\ relation is using radio luminosities, lobe morphologies,
and age estimates to predict ambient gas pressures: $p_{\mathrm{amb}}
\propto (t_{\mathrm{age}}\lrad) / V_{\mathrm{radio}}$. This yields an
estimate of ambient densities when basic assumptions are made about
environment temperatures: $\rho_{\mathrm{amb}} \propto p/T$. With an
estimate of ambient densities, X-ray observing plans for very
interesting radio sources which reside in faint group environments
(\ie\ FR-I sources) can be robustly prepared. An observationally-based
estimate of \pjet\ also enables the investigation of relations between
observable mass accretion surrogates (\ie\ \halpha\ luminosity,
molecular/dust mass, or nuclear X-ray luminosity) and AGN energetics
for the purpose of establishing clearer connections with accretion
mechanisms and efficiencies.

{\bfseries{Future Research}}

\markright{K.W.C., Statement of Interest}
The study of mechanically-dominated AGN feedback has advanced quickly
in the last decade primarily because the process is readily observed
at low-redshifts, and the hot gas phase which this mode of feedback
most efficiently interacts is accessible with the current generation
of X-ray observatories. However, the frequency of AGN feedback as a
function of environment and our understanding of radiative feedback
has not progressed as quickly. The former results from the limitations
of existing X-ray samples, while the latter is a consequence of
obscuration which prevents direct observational study
\cite{2009arXiv0911.3911A}. Luckily, the quality and availability of
multi-frequency data (radio, sub-mm, IR, optical, UV, and X-ray)
needed to probe AGN duty cycles and obscuration is poised to improve
with new facilities and instruments coming on-line (\ie\ LOFAR,
Herschel, SCUBA-2, SOFIA, ALMA, NuStar, Simbol-X). As such, there are
a number of questions regarding the formation and evolution of SMBHs
that I would like to pursue.

{\bf{How does AGN activity depend on environment?}} Specifically what
is the relationship between redshift, environment, duty cycle, and
feedback energy? The answer thus far is unclear, most likely because
the influence of environment on AGN jets (through entrainment and
confinement) has been neglected or treated too simply in models. The
lack of comprehensive X-ray samples, particularly at low-masses, also
has prevented the study of duty cycles. This is where observations
step in to place interesting constraints on the problem. To this end,
a study of the faint radio galaxy population using archival \chandra\
and VLA data would be interesting. Undertaking a systematic study of
radio galaxy properties (\ie\ jet composition, morphologies, outflow
velocities, magnetic field configurations) as a function of
environment (\ie\ ambient pressure, halo compactness) can help address
how AGN energetics couple to environment, which ultimately suggests
how accretion onto the SMBH couples to environment on small and large
scales. Deep \chandra\ observations for a sample of FR-I's (a poorly
studied population in the X-ray) would also be useful for such a
study, using the \pjet-\lrad\ relation to define robust observation
requests.

{\bf{How does the transition from an obscured to unobscured state
correlate with AGN feedback and SMBH growth?}} As suggested by the low
AGN fraction in the \chandra\ Deep Fields, a significant population of
obscured AGN must exist at higher redshifts. One method of selecting
unbiased samples of these objects is to assemble catalogs of candidate
AGN using hard X-ray (\ie\ NuStar), far-IR (\ie\ SOFIA), and sub-mm
(\ie\ SCUBA-2) observations. Because current models suggest the
luminous quasar population begins in an obscured state, and rapid
acquisition of SMBH mass may occur in this phase because of high
accretion rates (possibly exceeding $10-100~L_{\mathrm{Edd.}}$),
understanding the transition from obscured to unobscured states is
vital. How does accretion proceed and where does the accreting
material come from: gas cooling out of the atmosphere? Gas stripped
from merging companions? Is accretion spherical and dictated by local
gas density (\eg\ Bondi)?  A key component which has been neglected in
AGN studies is the contribution of dust (which should be a significant
component in the atmospheres of obscured AGN) in increasing the
allowed Eddington luminosity for an accreting SMBH (\ie\
$L_{\mathrm{Edd.}} \propto \mu$). A curiosity which has emerged in
recent years which may be interesting, particularly during the
obscured stage when the merger rate is presumably high, is the role of
multiple SMBHs within the core of a host galaxy. At a minimum, SMBH
mergers occur on a timescale determined by dynamical friction, which
for a typical dense bulge is $\ga 1$ Gyr, which is $\gg
t_{\mathrm{cool}}$ of an obscuring atmosphere. If the SMBHs which are
merging have their own accretion disks, then it is reasonable to
question how the atmospheres surrounding a host galaxy with multiple
AGN is affected, particularly since the transition from obscured to
unobscured should proceed more quickly.

\scriptsize
\bibliographystyle{unsrt}
\bibliography{cavagnolo}
 
\end{document}
