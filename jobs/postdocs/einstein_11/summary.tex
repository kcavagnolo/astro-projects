\documentclass[letterpaper,12pt]{article}
\usepackage[numbers,sort&compress]{natbib}
\bibliographystyle{unsrt}
\usepackage{common,graphicx}
\pagestyle{myheadings}
\setlength{\textwidth}{7.2in} 
\setlength{\textheight}{9.3in}
\setlength{\topmargin}{-0.3in} 
\setlength{\oddsidemargin}{-0.35in}
\setlength{\evensidemargin}{0in} 
\setlength{\headheight}{0in}
\setlength{\headsep}{0.3in} 
\setlength{\hoffset}{0in}
\setlength{\voffset}{0in}
\newcommand{\myhead}{Cavagnolo, Research Summary}
\begin{document}

\begin{center}
  {\bf\uppercase{Research Summary}}
\end{center}

{\bf Intracluster Medium (ICM) Temperature Inhomogeneity}: If simple
galaxy cluster X-ray observables, \eg\ temperature ($T$) or luminosity
($L$), are to serve as accurate mass proxies, how processes like
mergers alter these observables need to be quantified. It is known
that scaled ratios of $L$ and $T$, along with measures of ICM
substructure, correlate well with cluster dynamical state, and that
the apparently most relaxed clusters have the smallest deviations from
mean mass-observable relations (\eg\ \cite{kravtsov06, VV08}). If a
cluster's intracluster medium (ICM) is nearly isothermal in the
projected region of interest, the X-ray temperature inferred from a
broadband (0.7-7.0 keV) spectrum should be identical to the X-ray
temperature inferred from a hard-band (2.0-7.0 keV) spectrum. However,
if there are unresolved, cool lumps of gas, the estimated cluster
temperature may be cooler in the broadband versus the hard-band. This
difference is then another diagnostic to indicate the presence of
cooler gas, \eg\ associated with merging sub-clusters, even when the
X-ray spectrum itself may not have sufficient signal-to-noise to
resolve multiple temperature components \cite{me01}. Cavagnolo
\etal\ 2008 is a study of this band dependence for 192 clusters taken
from the \chandra\ Data Archive. We found, on average, that the
hard-band temperature was significantly higher than the broadband
temperature, and that their ratio was preferentially larger for known
mergers. Our results suggest a temperature diagnostic may be a useful
tool for further minimizing the scatter about mean mass-scaling
relations and obtaining more precise cluster mass estimates.

{\bf ICM Entropy}: ICM temperature and density mostly reflect the
shape and depth of a cluster's dark matter potential -- it is the
specific entropy which governs the density at a given pressure
\cite{voitbryan}. Without disturbance, the ICM becomes convectively
stable when the lowest entropy gas occupies the bottom of the cluster
potential and the highest entropy gas has buoyantly risen to large
radii. Further, ICM entropy is primarily changed through heat
exchange. Thus, deviations of the ICM entropy structure from the
azimuthally symmetric, radial power-law distribution which should
result from pure cooling are useful in evaluating a cluster's
thermodynamic history \cite{voitbryan}. Hence, one reason to study ICM
entropy distributions is to better understand the effects of energetic
feedback processes, \eg\ from active galactic nuclei (AGN), on the
cluster environment and investigating the breakdown of cluster
self-similarity.

In Cavagnolo \etal\ 2009 \cite{accept}, the ICM entropy structure of
239 clusters taken from the \chandra\ Data Archive were studied. We
found that most clusters have entropy profiles which are well-fit by a
model which is a power-law at large radii and approaches a constant
entropy value at small radii: $K(r) = \kna + \khun (r/100
~\kpc)^{\alpha}$, where \kna\ quantifies the typical excess of core
entropy above the best fitting power-law found at larger radii and
\khun\ is the entropy normalization at 100 kpc. Our results are
consistent with models which predict cooling of a cluster's X-ray halo
is offset by energy injected via feedback from active galactic nuclei
\cite[\eg][]{agnframework}. We also showed that the distribution of
\kna\ values in our archival sample is bimodal, with a distinct gap
around a $\kna \approx 40 ~\ent$.

If cooling of a galaxy cluster's halo triggers eventual heating via an
AGN-centric feedback loop, then certain properties of the ICM may
correlate tightly with signatures of feedback and/or indicators of
cooling. In Cavagnolo \etal\ 2008 \cite{haradent} we explored the
relationship between \halpha\ emission from cluster cores, radio
emission from cluster central galaxies, and cluster
\kna\ values. Utilizing the results of the archival study of
intracluster entropy, we found that \halpha\ and radio emission are
almost strictly associated with \kna\ values less than $30 ~\ent$. The
prevalence of \halpha\ emission below this threshold indicates that it
marks a dichotomy between clusters that can harbor multiphase gas and
star formation in their cores and those that cannot. The fact that
strong central radio emission also appears below this boundary
suggests that feedback from an AGN turns on when the ICM starts to
condense, strengthening the case for AGN feedback as the mechanism
that limits star formation in the Universe's most luminous
galaxies. In Voit \etal\ 2008 \cite{conduction} we go on to suggest
that \kna\ bimodality and the entropy threshold may occur as a result
of thermal conduction in the ICM. We are currently investigating the
self-similiar scaling properties of the entropy distributions for our
sample \cite{entscale}.

{\bf Details of AGN Feedback}: There are still many unknowns regarding
the process of AGN feedback, one of which is how much total kinetic
energy is released via relativistic jets. Pre-{\it{Chandra}}, jet
power (\pjet) estimates were made primarily with jet models
\cite[\eg][]{w99}. However, the discovery of ICM X-ray cavities
\cite[\eg][]{hydraa0} removed the model dependance by enabling direct
measurement of the work an AGN performs on its surroundings
\cite{mcnamrev}. It was then possible to determine how a simple
observable like monochromatic AGN radio power (\prad) scales with
\pjet\ \cite[\eg][]{birzan04, birzan08}. One aim of such studies was
to hopefully reduce the need for X-ray data to study the heating of
the Universe as a function of redshift using large samples of radio
galaxies taken from monochromatic all-sky radio surveys. Using a
sample of clusters, groups, and giant ellipticals (gEs) with detected
cavities, Cavagnolo \etal\ 2010 presents an updated version of the
B\^irzan \etal\ 2008 \pjet-\prad\ relation. We found, independent of
radio frequency, that \pjet\ scales as $\prad^{0.7}$ with a
normalization of $\sim 10^{43}$ erg s$^{-1}$, in accordance with
several jet models. We also identifies several gEs with unusually
large \prad\ for their \pjet, all of which have radio sources which
extend beyond the densest regions of their hot halos. We suggested
that these systems may result from their jets being unable to entrain
appreciable amounts of gas.

It is well-known that the process of galaxy growth and cluster/group
evolution is likely regulated by AGN feedback
\cite[\eg][]{2005Natur.435..629S, croton06}, lest ultramassive
galaxies prodigiously forming stars will emerge in the cores of galaxy
clusters/groups which are catastrophically cooling. For simplicity,
galaxy formation models divide AGN feedback into an early-time
quasar-mode (radiatively dominated feedback) to a late-time radio-mode
(mechanically dominated feedback). However, the observational
constaints on how systems transition from one to the other are very
poor. The importance of this transition is increased because it may
coincide with 1) the formation of dense galactic environments which
host the poorly studied population of obscured AGN, and 2) the merger
of massive galaxies which become present-day brightest cluster
galaxies (BCGs).
\markright{\myhead}

As a test of these models, we performed a X-ray study of the famous
and peculiar ULIRG/BCG IRAS 09104+4109 (I09) which hosts an obscured
quasar. We were able image X-ray halo cavities (mechanical feedback)
and irradiation of the halo by the central quasar (radiative
feedback), making this the first known system where both channels of
feedback have been simultaneously studied. We found that the
mechanical to radiative feedback ratio is $\approx 100:1$, and that
the cavities contain enough energy to offset $\approx 25\%$ of the
halo cooling. We argue that I09 may be a local example of how massive
galaxies at higher redshift form and evolve, and suggest that other
odd properties of the system (\eg\ the misaligned beaming and jet
axes) may be related to evolution of the central supermassive black
hole's spin axis.

The most powerful AGN outbursts in the Universe are useful for placing
constraints on possible fueling mechanisms for the AGN. Systems such
as MS 0735.6+7421, Hercules A, and Hydra A stress the limits of cold
gas accretion models (such as Bondi accretion), and open the door to
new mechanisms such as black hole spin \cite{bhspin}. The galaxy
cluster RBS 797 is another system which has undergone a cluster-scale
AGN outburst. R797 has a pair of X-ray cavities which suggest the AGN
outburst in the system is of order $\sim 10^{45-46}$ erg s$^{-1}$,
making it one of the most powerful outbursts ever observed. I have
undertaken the detailed analysis of this peculiar system using X-ray,
radio, infrared, optical, and UV data. The results of this work are
being published in a first author paper \cite{r797}.

\cite{steepspec}

Mina Rohanizadegan is a junior Ph.D. student under Dr. McNamara. We
are currently working on the analysis of X-ray data to learn about the
instantaneous accretion onto SMBHs at the center of galaxy
clusters. The aim is to place constraints on the fueling mechanism
which gives rise to the AGN jets which bore cavities into the
ICM. Mina is also finishing up a co-authored paper which presents
comparisons of models for AGN power generation via cold gas accretion
and black spin using the robust jet power measures from X-ray
cavities. As a supplement to this work, I am writing a paper which
discusses the complications of reorienting the spin axis of a SMBH via
mergers. Spin axis reorientation has become somewhat of a ``fad'' in
the last decade to explain the morphology of some radio sources and as
an explanation for the distribution of AGN feedback energy beyond the
small cross-section of AGN jets. However, spin axis reorientation is
exceedingly difficult and requires a very specific set of impact
parameters which, as discerned from cosmological simulations, are
found to be very rare \cite{spinaxis}.

\bibliography{cavagnolo}

\markright{K.W. Cavagnolo, Summary}
\begin{figure}[t]
    \begin{minipage}[t]{0.5\linewidth}
        \centering
        \includegraphics*[width=\columnwidth, trim=28mm 7mm 40mm 17mm, clip]{ha.eps}
        \caption{\footnotesize Central entropy
          vs. \halpha\ luminosity. Orange circles represent
          \halpha\ detections, black circles are non-detection upper
          limits, and blue squares with inset red stars or orange
          circles are peculiar clusters which do not adhere to the
          observed trend. The vertical dashed line marks $\kna = 30
          ~\ent$. Note the presence of a sharp \halpha\ detection
          dichotomy beginning at $\kna \la 30 ~\ent$.}
        \label{fig:ha}
    \end{minipage}
    \hspace{0.1cm}
    \begin{minipage}[t]{0.5\linewidth}
        \centering
        \includegraphics*[width=\textwidth, trim=30mm 5mm 40mm 15mm, clip]{pjet.eps}
        \caption{\footnotesize Cavity power vs. 1.4 GHz radio
          power. Orange triangles represent cluster and group sample
          of \cite{birzan08}, filled circles are our gE sample, colors
          represent the quality of cavities: green = `A,' blue = `B,'
          and red = `C.' Dotted red lines represent \cite{birzan08}
          best-fit relations. Dashed black lines represent our
          \bces\ best-fit power-law relations.}
        \label{fig:k0hist}
    \end{minipage}
\end{figure}
 
\end{document}
