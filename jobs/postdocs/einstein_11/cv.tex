\documentclass[12pt]{cv}
\usepackage[colorlinks=true,linkcolor=blue,urlcolor=blue]{hyperref}
\usepackage{mathptmx,multicol,common}
\parindent 0pt
\parskip
\baselineskip
\setlength{\topmargin}{-0.30in}
\setlength{\oddsidemargin}{-0.30in}
\setlength{\evensidemargin}{-0.30in}
\setlength{\headheight}{0in}
\setlength{\headsep}{0.25in}
\setlength{\topskip}{0.25in}
\setlength{\textwidth}{6.9in}
\setlength{\textheight}{9.25in}
\pagestyle{myheadings}
\newcommand{\myhead}{Cavagnolo, Curriculum Vitae}

\begin{document}

\begin{center}
{\Large Kenneth W. Cavagnolo\\Curriculum Vitae}
\rule{\linewidth}{1pt}
\normalsize
\end{center}
\vspace{-0.5cm}
\addresses
{
Observatoire de la C\^ote d'Azur\\
Boulevard de l'Observatoire\\
B.P. 4229\\
F-06304, Nice CEDEX 4, France\\
+33 (0)6 87 09 83 67
}
{
Citizenship: U.S.A.\\
Marital Status: Married\\
Birthdate: Jan. 27\ths, 1980\\
\href{mailto:kencavagnolo@gmail.com}{\tt{kencavagnolo@gmail.com}}\\
\href{http://www.pa.msu.edu/people/cavagnolo/}{\tt www.pa.msu.edu/people/cavagnolo/}\\
}
\vspace{-0.5cm}
\rule{\linewidth}{1pt}
\begin{llist}

%---------------------------------------------------------------%
%---------------------------------------------------------------%

\sectiontitle{Education}
Michigan State University
\location{2005 -- 2008}
Ph.D., Astronomy \& Astrophysics

Michigan State University
\location{2002 -- 2005}
M.S., Astronomy \& Astrophysics, \textit{magna cum laude}

Georgia Institute of Technology
\location{1998 -- 2002}
B.S., Physics, \textit{magna cum laude}

%---------------------------------------------------------------%
%---------------------------------------------------------------%

\sectiontitle{Research\\Experience}
Opales Postdoctoral Fellow
\location{2010 -- Present}
Supervisor: Chiara Ferrari, {\textit{Obs. C\^ote d'Azur}}
%\\Focus: Study of galaxy cluster/group non-thermal emission; ICM magnetic fields

UW Postdoctoral Fellow
\location{2008 -- 2010}
Supervisor: Brian McNamara, {\textit{Univ. of Waterloo}}
%\\Focus: AGN-environment interaction; SMBH accretion mechanisms; properties of relativistic jets

Graduate Research Assistant
\location{2003 -- 2008}
Supervisor: Megan Donahue, {\textit{Mich. St. Univ.}}
%\\Focus: Function \& regulation of galaxy cluster feedback; galaxy cluster virialization

Graduate Research Assistant
\location{2002 -- 2003}
Supervisor: Jack Baldwin, {\textit{Mich. St. Univ.}}
%\\Focus: Planetary nebulae {\textit{s}}-process abundances

Undergraduate Research Assistant
\location{2000 -- 2002}
Supervisor: James Sowell, {\textit{Geor. Inst. of Tech.}}
%\\Focus: Obtaining orbital solutions for eclipsing stellar binaries

%---------------------------------------------------------------%
%---------------------------------------------------------------%

\sectiontitle{Research\\Program\\\& Interests}

My research program is focused on better understanding the physics of
the intracluster and intragroup medium, and the role of feedback from
active galactic nuclei \& quasars on the formation and evolution of
galaxies, galaxy groups, and galaxy clusters.

Specific areas of interest:\\
$\bullet$ Cosmic magnetic fields\\
$\bullet$ Non-thermal galaxy cluster emission\\
$\bullet$ Black hole accretion physics\\
$\bullet$ Relativistic jets\\
$\bullet$ Cosmological studies of structure formation

%---------------------------------------------------------------%
%---------------------------------------------------------------%

\sectiontitle{Honors}
$\bullet$ Referee for ApJ, ApJL, AJ, CanTAC, \& MNRAS \hfill 2008 -- Present\\
$\bullet$ Sherwood K. Haynes Award for Outstanding Graduate Student \hfill 2008\\
$\bullet$ MSU College of Natural Science Dissertation Fellow \hfill 2007 -- 2008\\
$\bullet$ $\Sigma \Xi$ National Scientific Research Society Member\hfill 2009 -- Present\\
$\bullet$ $\Sigma \Pi \Sigma$ National Physics Honor Society Member\hfill 2001 -- Present\\
$\bullet$ American Astronomical Society Member\hfill 2002 -- Present\\
$\bullet$ American Physical Society Member\hfill 2002 -- Present\\
$\bullet$ LOFAR Consortium Member\hfill 2010 -- Present\\
$\bullet$ Perimeter Institute Black Hole Reading Group Member\hfill 2009 -- 2010\\
$\bullet$ Dean's List, Georgia Inst. of Tech. \hfill 1998 -- 2002

%---------------------------------------------------------------%
%---------------------------------------------------------------%

\sectiontitle{Scientific\\Skills}
$\bullet$ Expert of radio and X-ray data analysis \& interpretation\\
$\bullet$ Extensive experience analyzing infrared, optical, UV, and gamma-ray data\\
$\bullet$ Mastery of \aips, \casa, \ciao, \iraf, \osa, and \sas\ analysis software\\
$\bullet$ Fluent in \html, \idl, \LaTeX, and \perl\ programming languages\\
$\bullet$ Familiar with \clang, \fortran, \mysql, \python, \supmo, and \tickle\\
$\bullet$ Command of DOS, Linux, Macintosh, and Windows computing architectures\\
$\bullet$ Skillful in computer maintenance, construction, and troubleshooting
\markright{\myhead}

%---------------------------------------------------------------%
%---------------------------------------------------------------%

\sectiontitle{Observing\\Experience}

Very Long Baseline Array (VLBA)
\location{TBD}
12 hours observing IRAS 09104+4109

Giant Metrewave Radio Telescope (GMRT)
\location{Jan. 2010}
60 hours observing 15 galaxy clusters

Chandra X-ray Observatory (CXO)
\location{Jan. 2009}
21 hour queued observation of IRAS 09104+4109

Very Large Array Radio Telescope (VLA)
\location{Dec. 2008}
39 hours observing 13 giant ellipticals

%---------------------------------------------------------------%
%---------------------------------------------------------------%

\sectiontitle{Accepted\\Proposals\\\& Grants}

VLBA Cycle 10, PI
\location{2010}
Imaging the Misdirected QSO of IRAS 09104+4109%; 12 hrs.

GMRT Cycle 17--19, Co-I
\location{2009 -- 2010}
Power and Particle Content of Extragalactic Radio Sources I--III\\%; 144 hrs.
PI: Somak Raychaudhury, {\textit{Univ. Birmingham}}

GMRT Cycle 17, Co-I
\location{2009}
Morphology of Steepest Spectrum Radio Sources in Galaxy Cluster Cores\\%; 109 hrs.
PI: Alastair Edge, {\textit{Durham Univ.}}

NOAO Cycle 2008A, 2009A/B, \& 2010A, Co-I
\location{2008 -- 2010}
Normalization and scatter of the $M-T$ relation for supermassive galaxy clusters\\
PI: Rachel Mandelbaum, {\textit{Princeton Univ.}}

GMRT Cycle 16, Co-I
\location{2008}
Content of Giant Cavities in the IGM of Galaxy Clusters\\%; 40 hrs.
PI: Somak Raychaudhury, {\textit{Univ. Birmingham}}

CXO Cycle 10, PI
\location{2008}
IRAS 09104+4109: An Extreme Brightest Cluster Galaxy%; \$45k USD

CXO Cycle 10, Co-I
\location{2008}
Conduction and Multiphase Structure in the ICM\\%; \$100k
PI: Mark Voit, {\textit{Mich. St. Univ.}}

Spitzer Cycle 5, Co-I
\location{2008}
Star Formation and AGN Feedback in BCGs\\%; \$100k
PI: Megan Donahue, {\textit{Mich. St. Univ.}}

Spitzer Cycle 5, Co-I
\location{2008}
Infrared Properties of a Control Sample of Brightest Cluster Galaxies\\%; \$50k
PI: Megan Donahue, {\textit{Mich. St. Univ.}}

NSF Grant, Co-I
\location{2008}
Star Formation in the Universe's Largest Galaxies\\%; \$100k
PI: Mark Voit, {\textit{Mich. St. Univ.}}

CXO Cycle 9, Co-I
\location{2007}
Quantifying Cluster Temperature Substructure\\%; \$100k
PI: Mark Voit, {\textit{Mich. St. Univ.}}

VLA A-configuration Cycle, Co-I
\location{2007}
Radio Feedback in Clusters and Galaxies\\%; 39 hrs.
PI: Brian McNamara, {\textit{Univ. Waterloo}}

%---------------------------------------------------------------%
%---------------------------------------------------------------%

\sectiontitle{Students\\Advised}
Clif Kirkpatrick, Ph.D. candidate, {\textit{Univ. Waterloo}}
\location{2008 -- 2010}
The 2-Dimensional metal abundance distributions in galaxy clusters

Mina Rohanizadegan, Ph.D. candidate, {\textit{Univ. Waterloo}}
\location{2008 -- 2010}
Understanding SMBH accretion and spin

Jason King, Undergraduate research, {\textit{Univ. Waterloo}}
\location{2010}
Quantifying scatter in the \pjet-\prad\ relation

Brad Whuiska, Undergraduate research, {\textit{Univ. Waterloo}}
\location{2009}
Finding the largest galactic cores in the \hst\ archive

Rob Myers, Undergraduate research, {\textit{Univ. Waterloo}}
\location{2009}
In search of galaxy cluster radio galaxies in the 400 deg$^2$ Survey

%---------------------------------------------------------------%
%---------------------------------------------------------------%

\sectiontitle{Outreach}
Non-thermal Phenomena in Colliding Galaxy Clusters
\location{2010}
Conference Local Organizing Committee

International Year of Astronomy
\location{2009}
Organized observing nights, talks, and workshops in Waterloo, ON


%---------------------------------------------------------------%
%---------------------------------------------------------------%

\sectiontitle{Teaching\\Experience}
Substitute Instructor
\location{Fall 2006}
Course: ``Visions of the Universe''
%Gave lectures covering stellar evolution.

Honors Physics Tutor
\location{Summer 2003}
Course: ``Introductory Honors Physics I \& II''
%Tutored physics students taking classical mechanics, optics, and electromagnetism.

Graduate Teaching Assistant
\location{2002 - 2003}
Course: ``Visions of the Universe''
%Directed and supervised laboratories for introductory astronomy course.

%---------------------------------------------------------------%
%---------------------------------------------------------------%

\markright{\myhead}

\sectiontitle{References}

Megan Donahue, \href{mailto:donahue@pa.msu.edu}{\tt donahue@pa.msu.edu} \hfill 517-884-5618\\
Tenured faculty, Michigan State University

G. Mark Voit, \href{mailto:voit@pa.msu.edu}{\tt voit@pa.msu.edu} \hfill 517-884-5619\\
Tenured faculty, Michigan State University

Brian McNamara, \href{mailto:mcnamara@uwaterloo.ca}{\tt mcnamara@uwaterloo.ca} \hfill 519-888-4567 ext. 38170\\
Tenured faculty, University of Waterloo

Chris Carilli, \href{mailto:ccarilli@nrao.edu}{\tt ccarilli@nrao.edu} \hfill 575-835-7306\\
Chief Scientist, National Radio Astronomy Observatory

Jack Baldwin, \href{mailto:baldwin@pa.msu.edu}{\tt baldwin@pa.msu.edu} \hfill 517-884-5611\\
Associate Chair of Astronomy, Michigan State University

Mike Wise, \href{mailto:wise@science.uva.nl}{\tt wise@science.uva.nl} \hfill 05-2159-5564\\
Chief Scientist, LOFAR Radio Observatory

Paul Nulsen, \href{mailto:pnulsen@cfa.harvard.edu}{\tt pnulsen@cfa.harvard.edu} \hfill 617-495-7043\\
Research Scientist, Center for Astrophysics at Harvard University

Chiara Ferrari, \href{mailto:ferrari@oca.eu}{\tt ferrari@oca.eu} \hfill 04-9200-3028\\
Adjunct Astronomer, Observatoire de la C\^ote d'Azur

%---------------------------------------------------------------%
%---------------------------------------------------------------%

\sectiontitle{Personal\\Interests}
$\bullet$ Academic: Environmental sciences, ``Cradle2Cradle'' design, and urban planning.\\
$\bullet$ Athletics: Triathlons, running, baseball, and Georgia Tech athletics.\\
$\bullet$ Hobbies: Backpacking, reading, building model airplanes, and raising bonsai trees.

\end{llist}

\end{document}
