\documentclass[11pt]{article}
\usepackage[colorlinks=true,linkcolor=blue,urlcolor=blue]{hyperref}
\usepackage{subfig,epsfig,colortbl,graphics,graphicx,wrapfig,amssymb}
\usepackage{macros_cavag}
\pagestyle{myheadings}
\font\cap=cmcsc10
\setlength{\topmargin}{-0.2in}
\setlength{\oddsidemargin}{-0.1in}
\setlength{\evensidemargin}{0.in}
\setlength{\headheight}{0.1in}
\setlength{\headsep}{0.25in}
\setlength{\topskip}{0.1in}
\setlength{\textwidth}{6.5in}
\setlength{\textheight}{9.25in}

\markright{K.W. Cavagnolo Summary}

\begin{document}
\begin{center}
\textbf{Summary of Past Research and Future Interests}\\
\end{center}

The general process of galaxy cluster formation through hierarchical
merging is well understood, but many details, such as the impact of
feedback sources on the cluster environment and radiative cooling in
the cluster core are not. Mergers and feedback activity are interesting for
two reasons: they potentially compromise the use of clusters for
cosmological studies, and there is a tremendous amount of interesting
astrophysics going on. My thesis research has focused on studying
the details of feedback and mergers via X-ray properties of the ICM in
clusters of galaxies. I have paid particular attention to ICM entropy
distribution and the role of AGN feedback in shaping large scale
cluster properties.

\subsection*{Mining the CDA}

My thesis makes use of a 350 observation sample (276 clusters; 11.6
Msec) taken from the {\it Chandra} archive. This massive
undertaking necessitated the creation of a robust reduction and
analysis pipeline which 1) interacts with mission specific software,
2) utilizes analysis tools (i.e. {\tt{XSPEC}}, {\tt{IDL}}), 3)
incorporates calibration and software updates, and 4) is highly
automated. Because my pipeline is written in a very general manner,
adding pre-packaged analysis tools from missions such as
{\textit{XMM}}, {\textit{Spitzer}}, and {\textit{VLA}} will be
straightforward. Most importantly, my pipeline deemphasizes data
reduction and accords me the freedom to move quickly into an analysis
phase and generating publishable results.

\subsection*{Cluster Feedback and ICM Entropy}

The picture of the ICM entropy-feedback connection emerging from my
work suggests cluster radio luminosity and H$\alpha$ emission are
anti-correlated with cluster central entropy. Following my analysis of 169
cluster radial entropy profiles (Fig. \ref{fig:splots}) I have found
an apparent bimodality in the distribution of central
entropy and central cooling times (Fig. \ref{fig:tcool}) which is
likely related to AGN feedback (and to a lesser extent, mergers). I
have also found that clusters with central entropy $\leq 20$ keV
cm$^2$ show signs of star formation (Fig. \ref{fig:ha}) and AGN
activity (Fig. \ref{fig:rad}) while clusters above this threshold
unilaterally do not have star formation and exhibit diminished AGN
radio feedback. This entropy level is auspicious as it coincides with
the Field length, $\lambda_F$, (assuming reasonable suppression from
magnetic fields) at which thermal conduction can stabilize a cluster
core against further cooling and gas condensation. It is possible my
work has opened a window to solving a long-standing problem in massive
galaxy formation (and truncation): how are ICM gas properties coupled
to feedback mechanisms such that the system becomes self-regulating?
However, this result serves to highlight unresolved issues requiring
further intensive study.\\

\noindent {\bf 1) What is the origin of the bimodality in $K_0$ and $t_{cool}$?}\\
Is it archival bias? Are clusters with $K_0 \sim 70$ keV
cm$^2$ ``boring'' (and faint) and thus have not been
proposed for observation? To explore this possibility I have selected a
representative sample of clusters which predictably fill the $K_0$ and
$t_{cool}$ gaps and will be submitting a Cycle 10 proposal to observe
these clusters with {\it Chandra}. There is also the possibility that
the gap is a physical manifestation of underlying timescales. For
example, is the gap indicating there is a very short period in a
clusters life when AGN activity has boosted the core entropy to the
point of being conductively stable ($K_0 > 20$ keV cm$^2$) and
subsequent mergers quickly eliminate $K_0 \sim 70$ keV cm$^2$ clusters? A
possible answer to this question might be found from analysis of
simulations by asking the additional question: what is the timescale
for depletion of $\sim 10^{12-13} M_{\odot}$ subclusters in a full
dark matter halo? If this timescale is of the order a few Gyrs then
this likely points to a collusion of AGN feedback and mergers to give
rise to bimodality. But ultimately the questions I posed are related
with two primary underlying questions: what does the distribution of
$K_0$ for a complete sample of clusters look like? And what does the
AGN energy injection distribution look like?\\

\noindent {\bf 2) What role is star formation playing in the feedback cycle of clusters?}\\
Indications from the literature thus far are that most (possibly all?)
cDs in X-ray luminous clusters with $K_0 \leq 20$ keV cm$^2$ are
dominated by star formation. But we can see from Figure \ref{fig:rad}
that most of these systems contain radio AGN. So one can ask the
question: are there any AGN dominated nebular cDs? An interesting
project to pursue with the {\it Spitzer} archive would be to examine
the shape of spectral energy distributions (SEDs) for all clusters
with a cD galaxy and attempt to reveal if the cD is star formation or
AGN dominated. A cross-reference of my thesis sample (which is
essentially the entire CDA) with the {\it Spitzer} data archive
reveals 150+ clusters have already been observed by {\it Spitzer}
(combinations of 75+ MIPS, 50+ IRAC, 30+ IRS) covering a broad
entropy, luminosity (X-ray, H$\alpha$, radio), and mass range. This
large data pool to draw from makes selection of a representative subsample
immediately possible to answer the question, does star formation
precede/inhibit/enhance/stunt AGN feedback? Currently we do not
know. All we know is these two processes are triggered in cluster cDs
which reside in low entropy environments. It is important to
disentangle these two processes if a cohesive model of feedback is to
be built.\\

\noindent {\bf 3) How is energy generated on the parsec scale from a SMBH
deposited uniformly over volumes which are orders of magnitude larger?}\\
The role of AGN feedback in shaping global cluster, group, and galaxy
properties is quite complex (Perseus being a perfect example) and to some extent poorly
understood. Models for the process of thermalizing energy in AGN blown
bubbles have been proposed, but details of these models still need to
be explored. Equally important are models which account for the range
of environments we know AGN to be interacting with: spirals, gEs, and cDs.
While bubbles are well studied and abundant, a fundamental question
still remains unanswered: what's *inside* these bubbles? Are they
pressure supported by a very low density non-relativistic thermal
plasma or are they truly voids in the ICM and ISM? Observational
studies of bubbles in clusters have been fruitful, but a corresponding
study of bubbles in gEs and galaxy groups has been sorely lacking. An
obvious project to pursue with {\it Chandra} is to replicate the
seminal work of Bir\^{z}an et al. 2004 where they studied bubbles in
clusters, but instead of focusing on clusters, focusing on groups and
gEs. An additional missing piece of the AGN feedback puzzle is what
role X-ray coronae may be playing in promoting feedback. Coronae
have been seen in groups and some clusters, but their progenitors
should also be seen in smaller scale objects. A search for coronae in
a sample of radio-loud groups and clusters with moderate to high
central entropy would also be very interesting.

\clearpage
\begin{figure}[t]
    \begin{minipage}[t]{0.5\linewidth}
        \centering
	\includegraphics*[width=\textwidth, trim=28mm 8mm 30mm 10mm, clip]{splots}
        \caption{\small Radial entropy profiles of 169 clusters of
	galaxies in my thesis sample. The observed range of $K_0 \lesssim
	40$ keV cm$^2$ is consistent with models of episodic AGN
	heating. Color coding indicates global cluster temperature (in keV)
	derived from core excised apertures of size R$_{2500}$.}
	\label{fig:splots}
    \end{minipage}
    \hspace{0.1in}
    \begin{minipage}[t]{0.5\linewidth}
        \centering
        \includegraphics*[width=\textwidth, trim=28mm 8mm 30mm 10mm, clip]{tcool}
        \caption{\small Distribution of central cooling times for 169
	clusters in my thesis sample. The peak in the range of cooling
	times (several hundred Myrs) is consistent with inferred AGN
	duty cycles of both weak ($\sim 10^{40-50}$ ergs) and strong ($\sim
	10^{60}$ ergs) outbursts. However, note the distinct gap at $0.6-1$
	Gyr. An explanation for this bimodality does not currently exist.}
	\label{fig:tcool}
    \end{minipage}
    \hspace{0.1cm}
    \begin{minipage}[t]{0.5\linewidth}
        \centering
        \includegraphics*[width=\textwidth, trim=28mm 8mm 30mm 10mm, clip]{ha}
        \caption{\small Central entropy plotted against H$\alpha$
	luminosity. Orange dots are detections and black boxes with
	arrows are non-detection upper-limits. Notice the characteristic entropy threshold for star
	formation of $K_0 \lesssim 20$ keV cm$^2$. This is also the entropy scale at
	which conduction no longer balances radiative cooling and condensation
	of low entropy gas onto a cD can proceed.}
        \label{fig:ha}
    \end{minipage}
    \hspace{0.1in}
    \begin{minipage}[t]{0.5\linewidth}
        \centering
        \includegraphics*[width=\textwidth, trim=28mm 8mm 30mm 10mm, clip]{rad}
        \caption{\small Central entropy plotted against NVSS or PKS radio
	luminosity. Orange dots are detections and black boxes with
	arrows are non-detection upper-limits. There appears to be a dichotomy which might be related to AGN
	fueling mechanisms: AGN which are feed via low entropy gas, and the
	smattering of points at $K_0 > 50$ keV cm$^2$ which are likely
	fueled by mergers or have X-ray coronae which promote ICM cooling.}
        \label{fig:rad}
    \end{minipage}
\end{figure}
\end{document}
