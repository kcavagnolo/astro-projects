\documentclass[11pt]{article}
\usepackage[colorlinks=true,linkcolor=blue,urlcolor=blue]{hyperref}
\usepackage{subfig,epsfig,colortbl,graphics,graphicx,wrapfig,amssymb}
\usepackage{macros_cavag}
\pagestyle{myheadings}
\font\cap=cmcsc10
\setlength{\topmargin}{-0.2in}
\setlength{\oddsidemargin}{-0.1in}
\setlength{\evensidemargin}{0.in}
\setlength{\headheight}{0.1in}
\setlength{\headsep}{0.25in}
\setlength{\topskip}{0.1in}
\setlength{\textwidth}{6.5in}
\setlength{\textheight}{9.25in}

\begin{document}
\begin{center}
\textbf{Summary of Past Research and Future Interests}
\end{center}

The general process of galaxy cluster formation through hierarchical
merging is well understood, but many details, such as the impact of
feedback sources on the cluster environment and radiative cooling in
the cluster core, are not. My thesis research has focused on studying
these details via X-ray properties of the ICM in clusters of galaxies
using a 350 observation sample (276 clusters; 11.6 Msec) taken from
the {\it Chandra} archive. I have paid particular attention to ICM
entropy distribution, the process of virialization, and the role of
AGN feedback in shaping large scale cluster properties.\\

The picture of the ICM entropy-feedback connection emerging from my
research suggests cluster cD radio luminosity and core H$\alpha$
emission are anti-correlated with cluster central entropy. Following
analysis of 169 cluster radial entropy profiles
(Fig. \ref{fig:splots}), I have found an apparent bimodality in the
distribution of central entropy and central cooling times
(Fig. \ref{fig:tcool}) which is likely related to AGN feedback (and to
a lesser extent, mergers). I have also found that clusters with
central entropy $\leq 20$ keV cm$^2$ show signs of star formation
(Fig. \ref{fig:ha}) and AGN activity (Fig. \ref{fig:rad}), while
clusters above this threshold unilaterally do not have star formation
and exhibit diminished AGN radio feedback. This entropy level is
auspicious as it coincides with the Field length (assuming reasonable
magnetic suppression) at which thermal conduction can stabilize a
cluster core against further cooling and gas condensation. It is
possible my research has opened a window to solving a long-standing
problem in massive galaxy truncation and cluster feedback: how are ICM
gas properties coupled to feedback mechanisms such that the system
becomes self-regulating? However, these findings serve to highlight
unresolved issues requiring further intensive study.\\

\noindent {\bf 1) What is the origin of the bimodality in $K_0$ and $t_\mathrm{cool}$?}\\
Is it archival bias? Are clusters with $K_0 \sim 70$ keV
cm$^2$ ``boring'' (and faint) and thus have not been
proposed for observation? To explore this possibility I have selected a
representative sample of clusters which predictably fill the $K_0$ and
$t_\mathrm{cool}$ gaps, and will be submitting a Cycle 10 proposal to observe
these clusters with {\it Chandra}. There is also the possibility 
that bimodality is the physical manifestation of two underlying
timescales. The gap may be indicating there is a very short period in a
clusters life when AGN activity has boosted the core entropy to the
point of being conductively stable ($K_0 > 20$ keV cm$^2$) and
subsequent mergers quickly eliminate $K_0 \sim 70$ keV cm$^2$
clusters. A possible answer to this problem might be found from
analysis of simulations by asking the additional question: what is the
timescale for depletion of $\sim 10^{12-13} M_{\odot}$ subclusters in
a full dark matter halo? If this timescale is of the order a few Gyrs,
then this likely points to a collusion of AGN feedback and mergers to
give rise to bimodality. But ultimately the questions I posed are
related with two primary underlying questions: what does the
distribution of $K_0$ for a complete sample of clusters look like? And
what does the AGN energy injection distribution look like?\\

\noindent {\bf 2) What role is star formation playing in the feedback cycle of clusters?}\\
Indications from the literature thus far are that most (possibly all?)
cDs in X-ray luminous clusters with $K_0 \leq 20$ keV cm$^2$ are
dominated by star formation. But we can see from Figure \ref{fig:rad}
that most of these systems contain radio AGN. So one should ask the
question: are there any AGN dominated nebular cDs? An interesting
project to pursue with the {\it Spitzer} archive would be to study if
the cD galaxy in a carefully chosen sample of clusters is star
formation or AGN dominated. This study is quite simple using the
Far-IR bands to look for polycyclic aromatic hydrocarbons as
definitive signatures of star formation and mid-IR excesses as an
indicator of dust enshrouded AGN. An additional constraint on star
formation can be made by analyzing archival {\it XMM} Optical Monitor
data for near-UV excess from starlight not reprocessed by dust.

A cross-reference of my thesis sample with the {\it Spitzer} data
archive reveals 150+ clusters have already been observed, with 130+
also having OM data in the {\it XMM} archive. This large data pool to
draw from makes selection of a representative sub-sample immediately
possible to answer the question, how is star formation affected by AGN
feedback? Currently we do not know. All we know is these two processes
are triggered in cluster cDs which reside in low entropy
environments. It is important to disentangle these two processes if a
cohesive model of feedback is to be built.\\

\markright{K.W. Cavagnolo Summary}

\noindent {\bf 3) How is energy generated on the parsec scale from a SMBH
deposited uniformly over volumes which are orders of magnitude larger?}\\
The role of AGN feedback in shaping global cluster, group, and galaxy
properties is quite complex, and to some extent poorly
understood. Models for thermalizing energy from AGN blown bubbles have
been proposed, but details of these models (e.g. explaining the
thermal inefficiency of jets or finding a low scatter relation between
jet power-radio power) still need to be explored. Equally important
are models which account for the range of environments we know AGN to
be interacting with: spirals, giant ellipticals (gEs), and cDs.

While bubbles are well studied and abundant, a fundamental question
still remains unanswered: what's *inside* these bubbles? Are they
pressure supported by a very low density non-relativistic thermal
plasma, or maybe by relativistic particles like cosmic rays, or even
stranger, could they truly be voids in the ICM and ISM? Observational
studies of bubbles in clusters have been fruitful, but a corresponding
study of bubbles in gEs and galaxy groups has been sorely lacking. An
obvious project to pursue with {\it Chandra} is to replicate the
seminal work of Bir\^{z}an et al. 2004 where they studied bubbles in
clusters, but instead of focusing on clusters, focusing on groups and
gEs.

An additional missing piece of the AGN feedback puzzle is what
role X-ray coronae may be playing in promoting feedback. Coronae
have been seen in groups and some clusters, but their progenitors
should also be seen in smaller scale objects. A search for coronae in
a sample of radio-loud groups and clusters with moderate to high
central entropy would also be very interesting as these systems
shouldn't have feedback activity, but sometimes do (i.e. AWM4) because
a corona is most likely insulating core gas from the hot atmosphere
and allows for low entropy gas to cool.\\

\noindent {\bf 4) How important is conduction in the feedback loop and
how does it couple to the ICM?}\\
Assuming pure free-free cooling, the Field length (the size at which a
gas cloud is stabilized against condensation by conduction) is a
function of entropy alone, $\lambda_F \equiv K^{3/2}$. Assuming
realistic magnetic suppression, $K_0 = 20$ keV cm$^2$ is the entropy
level above which conductive heating wins against cooling
(Fig. \ref{fig:conduc}). This would nicely explain the dichotomy in
Figure \ref{fig:ha}, but additional theoretical and observational
evidence is needed to support this idea. Mark Voit is undertaking a
theoretical study of this topic, and collaborating as observational
counterparts would be an excellent project. For example, selecting a
sample of clusters which span the $K_0 = 20$ boundary and
investigating ICM conductive properties should yield robust
differences between clusters above and below this entropy
threshold. In addition, a systematic study of cold fronts as
surrogates for internal ICM physics (such as origin, strength, and
structure of magnetic fields) would be enlightening for better
understanding conduction.\\

\noindent {\bf 5) What about the oddities?}\\
In Figures \ref{fig:ha} and \ref{fig:rad} there are a total of 12
unique clusters (blue boxes with red stars) which lie below the
auspicious $K_0 = 20$ cut-off and {\it do not} have star formation
and/or radio-loud AGN. One must ask, ``why the heck not?''  Why aren't
there stars and/or AGN? I can only conjecture at the moment as a
thorough study of these objects does not exist. Imagine an overdense
gas parcel buried in a very low entropy medium. As the gas parcel
sinks it will reach a region of higher density, stop, buoyantly rise,
and dissipate. Reproducing this process over an $\approx$ 10-20 kpc
region should result in all gas overdensities being washed out
while the overall entropy of the region continues to lower (kind of
like a beating heart). The result would be a low entropy core with no
overdense regions which could produce stars or SMBH feeding gas
streams. But is this process stable? Does it require large magnetic
suppression of conduction? How would one even observe this process
with existing instruments (this is possibly a good feasibility study
for Con-X to help push for higher-res optics)? What are the other
possible explanations for this odd class of cluster: are we not seeing
the star formation in H$\alpha$? Are there AGN and they're just not
radio-loud? I'm not sure at the moment, these ideas require much more
thought and study.\\

Looking ahead, the natural extension of my thesis is to further study
questions regarding details of feedback and galaxy formation. There
are also exciting theoretical cluster feedback model developments on
the horizon which will need observational investigation, and for which
I am well positioned to study.

\clearpage
\begin{figure}[t]
\centering
    \begin{minipage}[t]{0.6\linewidth}
	\includegraphics*[width=\textwidth, trim=28mm 8mm 30mm 10mm, clip]{splots}
        \caption{\small Radial entropy profiles of 169 clusters of
	galaxies in my thesis sample. The observed range of $K_0 \lesssim
	40$ keV cm$^2$ is consistent with models of episodic AGN
	heating. Color coding indicates global cluster temperature (in keV)
	derived from core excised apertures of size R$_{2500}$.}
	\label{fig:splots}
    \end{minipage}
    \hspace{0.1in}
    \begin{minipage}[t]{0.6\linewidth}
        \includegraphics*[width=\textwidth, trim=28mm 8mm 30mm 10mm, clip]{tcool}
        \caption{\small Distribution of central cooling times for 169
	clusters in my thesis sample. The peak in the range of cooling
	times (several hundred Myrs) is consistent with inferred AGN
	duty cycles of both weak ($\sim 10^{40-50}$ ergs) and strong ($\sim
	10^{60}$ ergs) outbursts. However, note the distinct gap at $0.6-1$
	Gyr. An explanation for this bimodality does not currently exist.}
	\label{fig:tcool}
    \end{minipage}
\end{figure}

\clearpage
\begin{figure}[t]
    \hspace{0.1cm}
    \begin{minipage}[t]{0.5\linewidth}
        \centering
        \includegraphics*[width=\textwidth, trim=28mm 8mm 30mm 10mm, clip]{ha_k0}
        \caption{\small Central entropy plotted against H$\alpha$
	luminosity. Orange dots are detections and black boxes with
	arrows are non-detection upper-limits. Notice the characteristic entropy threshold for star
	formation of $K_0 \lesssim 20$ keV cm$^2$. This is also the entropy scale at
	which conduction no longer balances radiative cooling and condensation
	of low entropy gas onto a cD can proceed.}
        \label{fig:ha}
    \end{minipage}
    \hspace{0.1in}
    \begin{minipage}[t]{0.5\linewidth}
        \centering
        \includegraphics*[width=\textwidth, trim=28mm 8mm 30mm 10mm, clip]{k0rad}
        \caption{\small Central entropy plotted against NVSS or PKS radio
	luminosity. Orange dots are detections and black boxes with
	arrows are non-detection upper-limits. There appears to be a dichotomy which might be related to AGN
	fueling mechanisms: AGN which are feed via low entropy gas, and the
	smattering of points at $K_0 > 50$ keV cm$^2$ which are likely
	fueled by mergers or have X-ray coronae which promote ICM cooling.}
        \label{fig:rad}
    \end{minipage}
    \hspace{0.1in}
    \begin{minipage}[t]{0.5\linewidth}
        \centering
        \includegraphics*[width=0.95\textwidth, trim=0mm 0mm 0mm 0mm, clip]{conduction}
        \caption{\small 
        Toy entropy profiles plotted as a function of radius and overlaid with
        dashed lines representing cooling and conduction equivalence for two
        suppression factors. Above the dashed lines conduction is effective
        and condensation cannot occur, the opposite is true below the
        lines. $K_0 < 20$ keV cm$^2$ is the break point at which the Field
        length criterion suggests gas condensation (i.e. star formation and
        condensation onto a SMBH) can proceed. Reproduced courtesy of Dr. G.
        Mark Voit.}
        \label{fig:conduc}
    \end{minipage}
\end{figure}
\end{document}
