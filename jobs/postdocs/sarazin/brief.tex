\documentclass[11pt]{article}
\usepackage[colorlinks=true,linkcolor=blue,urlcolor=blue]{hyperref}
\usepackage{subfig,epsfig,colortbl,graphics,graphicx,wrapfig,amssymb,macros_cavag}
\font\cap=cmcsc10
\setlength{\topmargin}{-0.25in}
\setlength{\oddsidemargin}{-0.15in}
\setlength{\evensidemargin}{-0.15in}
\setlength{\headheight}{0.15in}
\setlength{\headsep}{0.15in}
\setlength{\topskip}{0.15in}
\setlength{\textwidth}{6.5in}
\setlength{\textheight}{9.25in}
\pagestyle{myheadings}

\begin{document}
\begin{center}
\large
\textbf{Summary of Experience and Interests}
\normalsize
\end{center}

The general process of galaxy cluster formation through hierarchical
merging is well understood, but many details, such as the impact of
feedback sources on the cluster environment and radiative cooling in
the cluster core are not. My thesis research has focused on studying
these details in clusters of galaxies via X-ray properties of the
ICM. I have paid particular attention to ICM entropy distribution, the
process of cluster virialization, and the role of AGN feedback in
shaping large scale cluster properties.

My research makes use of a 350 observation sample (276 clusters, 11.6
Msec) taken from the {\textit{Chandra}} archive. This massive
undertaking necessitated the creation of a robust reduction and
analysis pipeline which 1) interacts with mission specific software,
2) utilizes analysis software (e.g. {\tt{XSPEC}}, {\tt{IDL}}), 3)
incorporates calibration and software updates, and 4) is highly
automated. Because my pipeline is written in a very general manner,
adding pre-packaged analysis tools from missions such as
{\textit{XMM}}, {\textit{Spitzer}}, and {\textit{VLA}} will be
straightforward. Most importantly, my pipeline deemphasizes data
reduction and accords me the freedom to move quickly into an analysis
phase and generating publishable results.

The picture of the ICM entropy-feedback connection emerging from my
research suggests cluster cD radio luminosity and core H$\alpha$
emission are anti-correlated with cluster central entropy. Following
analysis of 169 cluster radial entropy profiles
(Fig. \ref{fig:splots}), I have found an apparent bimodality in the
distribution of central entropy and central cooling times
(Fig. \ref{fig:tcool}) which is likely related to AGN feedback (and to
a lesser extent, mergers). I have also found that clusters with
central entropy $\leq 20$ keV cm$^2$ show signs of star formation
(Fig. \ref{fig:ha}) and AGN activity (Fig. \ref{fig:rad}), while
clusters above this threshold unilaterally do not have star formation
and exhibit diminished AGN radio feedback. This entropy level is
auspicious as it coincides with the Field length (assuming reasonable
magnetic suppression) at which thermal conduction can stabilize a
cluster core against run-away cooling and ICM condensation. These
results are highly suggestive that conduction is very important to
solving the long-standing problem of how ICM gas properties are
coupled to feedback mechanisms such that the system becomes
self-regulating.

The final phase of my thesis is focused on further understanding why
we observe bimodality, what role star formation is playing in the cluster
feedback loop, refining a model for how conduction couples feedback to
the ICM, and examining the peculiar class of objects which fall below
the Field length criterion but {\it do not} have star formation and/or
radio-loud AGN (blue boxes with red stars in two of the figures).

There are additional areas of my present research I'd like to expound
on in the future:
\begin{enumerate}
\item I am proposing {\it Chandra} Cycle 10 observations for a sample
of clusters which predictably fall into the $t_{\mathrm{cool}}$ and
$K_0$ gaps to see if bimodality is archival bias or physical.
\item Two classes of peculiar objects warrant intensive
multiwavelength study: high-$K_0$ clusters with radio-loud AGN
(e.g. AWM4) and low-$K_0$ clusters without any feedback sources
(e.g. Abell 2107). The former likely have prominent X-ray corona,
while the latter may be showing evidence that extremely low entropy
cores inhibit the growth of gas density contrasts.
\item At present, I am putting all my reduced data products and thesis
results into a static website so they are available to any interested
researcher. Long-term however, I plan on submitting an archive grant
proposal to convert this site into an interactive database which can
be easily used by novices (i.e. undergraduate labs or course
instructors), expert X-ray astronomers, and curious theoreticians.
\item Thus far I have only focused on AGN which are radio-loud
according to the 1.4 GHz eye of NVSS. But recent work has shown AGN
radiate profusely at low radio frequencies (e.g. 300 MHz). I'd like to
know what the radio power is at these wavelengths for (ideally) my
entire thesis sample and see if the $K_0$-radio correlation tightens.
\item Using the near-UV sensitivity of {\it XMM}'s Optical Monitor and
the far-IR channels of {\it Spitzer} I'd like to pursue a joint
archival project to disentangle which $K_0 \leq 20$ cDs are star
formation dominated and which are AGN dominated. A quick check of these
archives shows 130+ clusters have the necessary band data available.
\item I'd also like to pursue a systematic study of AGN bubbles in
groups and ellipticals -- akin to the seminal work of Bir\^{z}an et
al. 2004 -- but with the focus of this project being adaptation of
existing cluster feedback models to smaller scale objects.
\end{enumerate}

In another part of my thesis research I studied the bandpass
dependence in determining X-ray temperatures and what this dependence
tells us about the virialization state of a cluster. The ultimate goal
of this project was to find an aspect-independent measure of a
cluster's dynamic state. Prompted by the work of Mathiesen \& Evrard
2001, I investigated the net temperature skew of the hard-band
(2.0$_{\mathrm{rest}}$-7.0 keV) and full-band (0.7-7.0 keV) temperature ratio
for core-excised apertures of my entire CDA sample. I found this
temperature ratio was robustly and significantly connected to mergers
and the absence of cool cores. This project touched on quantifying and
reducing the scatter in mass-observable relations to bolster the
utility of clusters as cosmology tools. I am eager to keep this area
of my work alive as we get closer to having access to enormous
catalogs of SZ detected clusters which require X-ray follow-up. To
maximize the utility of these surveys, we must continue to investigate
scatter, evolution, and covariance in X-ray observables which serve as
vital mass surrogates.

Looking ahead, the natural extension of my thesis is to further study
questions regarding the details of cluster feedback and galaxy
formation. Is conduction the long-sought answer for 1) how energy is
uniformly distributed in the ICM; 2) how stars form in the most
massive galaxies; 3) why feedback is so tightly correlated with the
state of the ICM. There are additional avenues which I have not
touched on in this summary but still interest me, such as the
micro-physics of ICM heating (e.g. turbulence and weak shocking), the
thermalization of mechanical work done by bubbles, and the importance
of non-thermal sources like cosmic rays. How prevalent are cold
fronts? Can they be used to robustly quantify ICM magnetic fields and
viscosity? Are they important in the feedback loop? Building on the
work of Paul Martini and Greg Sivakoff, how robust is their ``X-ray
Butcher-Oemler Effect'' if one expands their work to a very large
sample of clusters? This is a project which I essentially have in hand
because identifying full-field point sources is an integral first step
in my reduction. Can we deduce a low-scatter relation (or at least
constrain one) between jet power and radio power? What is the
explanation for the thermal inefficiency of jets? Many questions
abound as a result of my thesis work, I hope to pursue the answers to
them a post-doc with you at UVA.

\clearpage
\begin{figure}[t]
    \begin{minipage}[t]{0.5\linewidth}
        \centering
	\includegraphics*[width=\textwidth, trim=28mm 8mm 30mm 10mm, clip]{splots}
        \caption{\small Radial entropy profiles of 169 clusters of
	galaxies in my thesis sample. The observed range of $K_0 \lesssim
	70$ keV cm$^2$ is consistent with models of episodic AGN
	heating. Color coding indicates global cluster temperature (in keV)
	derived from core excised apertures of size R$_{2500}$.}
	\label{fig:splots}
    \end{minipage}
    \hspace{0.1in}
    \begin{minipage}[t]{0.5\linewidth}
        \centering
        \includegraphics*[width=\textwidth, trim=28mm 8mm 30mm 10mm, clip]{tcool}
        \caption{\small Distribution of central cooling times for 169
	clusters in my thesis sample. The peak in the range of cooling
	times (several hundred Myrs) is consistent with inferred AGN
	duty cycles of both weak ($\sim 10^{40-50}$ ergs) and strong ($\sim
	10^{60}$ ergs) outbursts. However, note the distinct gap at $0.6-1$
	Gyr. An explanation for this bimodality does not currently exist.}
	\label{fig:tcool}
    \end{minipage}
    \hspace{0.1cm}
    \begin{minipage}[t]{0.5\linewidth}
        \centering
        \includegraphics*[width=\textwidth, trim=28mm 8mm 30mm 10mm, clip]{ha_k0}
        \caption{\small Central entropy plotted against H$\alpha$
	luminosity. Orange dots are detections and black boxes with left-facing
	arrows are non-detection upper-limits. Notice the characteristic entropy threshold for star
	formation of $K_0 \lesssim 20$ keV cm$^2$. This is also the entropy scale at
	which conduction no longer balances radiative cooling and condensation
	of low entropy gas onto a cD can proceed.}
        \label{fig:ha}
    \end{minipage}
    \hspace{0.1in}
    \begin{minipage}[t]{0.5\linewidth}
        \centering
        \includegraphics*[width=\textwidth, trim=28mm 8mm 30mm 10mm, clip]{k0rad}
        \caption{\small Central entropy plotted against NVSS radio
	luminosity. Orange dots are detections and black boxes with left-facing
	arrows are non-detection upper-limits. Radio-loud AGN clearly
	prefer low entropy environs but the dispersion at low luminosity is
	large. It would be interesting to radio date these sources as this
	figure may have an age dimension.}
        \label{fig:rad}
    \end{minipage}
\end{figure}
\end{document}
