\documentclass[11pt]{article}
\usepackage[colorlinks=true,linkcolor=blue,urlcolor=blue]{hyperref}
\usepackage{subfig,epsfig,colortbl,graphics,graphicx,wrapfig,amssymb}
\usepackage{macros_cavag}
\pagestyle{myheadings}
\font\cap=cmcsc10
\setlength{\topmargin}{-0.2in}
\setlength{\oddsidemargin}{-0.1in}
\setlength{\evensidemargin}{0.in}
\setlength{\headheight}{0.1in}
\setlength{\headsep}{0.25in}
\setlength{\topskip}{0.1in}
\setlength{\textwidth}{6.5in}
\setlength{\textheight}{9.45in}

\markright{K.W. Cavagnolo Summary}

\begin{document}
\begin{center}
\textbf{Summary of Past Research and Future Interests}\\
\end{center}

The general process of galaxy cluster formation through hierarchical
merging is well understood, but many details, such as the impact of
feedback sources on the cluster environment and radiative cooling in
the cluster core are not. My thesis research has focused on studying
the details of feedback and mergers via X-ray properties of the ICM in
clusters of galaxies. I have paid particular attention to ICM entropy
distribution and the role of AGN feedback in shaping large scale
cluster properties.

\subsection*{Mining the CDA}

My thesis makes use of a 350 observation sample (276 clusters; 11.6
Msec) taken from the {\it Chandra} archive. This massive
undertaking necessitated the creation of a robust reduction and
analysis pipeline which 1) interacts with mission specific software,
2) utilizes analysis tools (i.e. {\tt{XSPEC}}, {\tt{IDL}}), 3)
incorporates calibration and software updates, and 4) is highly
automated. Because my pipeline is written in a very general manner,
adding pre-packaged analysis tools from missions such as
{\textit{XMM}}, {\textit{Spitzer}}, and {\textit{VLA}} will be
straightforward. Most importantly, my pipeline deemphasizes data
reduction and accords me the freedom to move quickly into an analysis
phase and generating publishable results.

\subsection*{Cluster Feedback and ICM Entropy}

The picture of the ICM entropy-feedback connection emerging from my
work suggests cluster radio luminosity and H$\alpha$ emission are
anti-correlated with cluster central entropy. Following my analysis of 169
cluster radial entropy profiles (Fig. \ref{fig:splots}) I have found
an apparent bimodality in the distribution of central
entropy and central cooling times (Fig. \ref{fig:tcool}) which is
likely related to AGN feedback (and to a lesser extent, mergers). I
have also found that clusters with central entropy $\leq 20$ keV
cm$^2$ show signs of star formation (Fig. \ref{fig:ha}) and AGN
activity (Fig. \ref{fig:rad}) while clusters above this threshold
unilaterally have no signs of star formation and exhibit diminished AGN
radio feedback. This entropy level is auspicious as it coincides with
the Field length, $\lambda_F$, (assuming reasonable suppression from
magnetic fields) at which thermal conduction can stabilize a cluster
core against further cooling and gas condensation. It is possible my
work has opened a window to solving a long-standing problem in massive
galaxy formation (and truncation): how are ICM gas properties coupled
to feedback mechanisms such that the system becomes self-regulating?
However, this result serves to highlight unresolved issues requiring
further intensive study.

Looking ahead, the natural extension of my thesis is to further study
questions regarding details of feedback and galaxy formation. What are
the micro-physics of ICM heating, including the thermalization of
mechanical work done by bubbles and the effect of non-thermal sources
like cosmic rays? How prevalent are cold fronts and do they play a
role in galaxy and star formation? Also of interest are how accretion
onto the cD SMBH is regulated by large-scale ICM properties and what
the AGN energy injection function looks like and how it correlates
with cluster environment.

There are also exciting theoretical cluster feedback model
developments on the horizon which will need observational
investigation, and for which I am well positioned to
study. Developments such as: how exactly are AGN fueled? Does
accretion of the hot ICM/ISM proceed via Bondi-eque flows? What is the
efficiency of the accretion? Why do we see metallicity gradients in
the ICM/ISM when some amount of mixing should take place? How is
feedback energy distributed symmetrically throughout the ICM?

\subsection*{Cold Fronts in Clusters}

As part of my thesis work I have extracted radial surface brightness,
temperature, and pressure profiles for well over 220 clusters taken from
the {\it Chandra} archive. Visual inspection of these profiles and
the corresponding images of the cluster are an integral step in my
analysis, and as other authors have pointed out (e.g. Markevitch and
Vikhlinin) cold fronts are a common feature of even the most
``relaxed'' clusters.

The crux of my thesis is understanding the
entropy distribution in the core of clusters as it relates to active
feedback (AGN, star formation, etc.), but I often wonder what effect
gas sloshing (either from mergers or from AGN bubbles) and ``soft''
mergers (which result in prominent cold fronts) have had on altering ICM
entropy. Currently mergers are viewed as a hammer instead of a scalpel:
mergers shock heat the ICM, end of story. But we now know this is most
certainly not the case and more sophisticated models of mergers are
needed to explain the zoology of cluster substructure. As demonstrated
by Ascasibar et al. 2006, accurately generating cold fronts in SPH
simulations is possible, and a useful next step would be to analyze
these simulations as if they were real data taken with {\it
Chandra}. The research group I am part of at MSU has recently started
using software written by Elena Rasia which recasts simulated data as
real data. The purpose of the present project (which I do not discuss
here) is to better understand the process of cluster virialization. I
am gaining invaluable insight and experience in the task of analyzing
simulations as real observations and think the undertaking of a
similar project to analyze cold fronts in simulations would be
fruitful.

Cold fronts in and of themselves are interesting ICM features because
they correlate with a number of physical processes (sloshing, mergers,
AGN, etc.). But cold fronts are also interesting because one can
utilize them as a laboratory for studying the internal physics of the
ICM. There are long-standing debates regarding the structure,
strength, and origins of ICM magnetic fields, and the properties of
cold fronts are sensitive to all three of these features. But cold
fronts can also yield information regarding conduction and
viscosity in the ICM. These are extremely interesting to me because as
I discussed in an earlier section both of these processes are
probably very important in transferring feedback energy to the
ICM. Conduction is likely the coupling mechanism between AGN feedback,
ICM heating, and star formation, while viscosity is likely important
in thermalizing bubble and jet energy. But there is a big piece of the
puzzle missing: how strong is magnetic suppression and how viscous is
the ICM?

I can envision a project where we start with an ensemble of SPH
cluster simulations covering a variety of input ICM magnetic fields and
viscosities. We then take the ensemble and generate mock observations,
meaning proper {\it Chandra} events files with instrument convolved
spectra for each position in the data cube. We then analyze these mock
observations using standard observational tools (i.e. {\tt CIAO} and
{\tt XSPEC}) to look for and analyze cold fronts. The objective of
this approach being quantification of cold front properties as a
function of input magnetic and viscosity parameters. The advantage
of course is that we know ``the truth'' about the ICM, and analyzing
the simulations as real data will allow us to put constraints on what
can be observationally learned about conduction and
viscosity. Ultimately the goal would be to use cold fronts as a
surrogate for getting at the internal ICM physics so we can build
robust models which include conduction and viscosity in the
explanation of how feedback mechanisms alter cluster properties like
entropy.

There is also a glaring lack of uniformly analyzed cold fronts in
the literature. An ancillary project to the one envisioned 
above is a large observational study of cold fronts. The {\it
XMM-Newton} and {\it Chandra} archives are replete with enough data to
make selection of a representative cluster sample and immediate
initiation of such a project possible. There are few people more well
prepared to spearhead this ambitious effort than me as I already have the
pipeline and analysis techniques necessary to quickly complete the
study. The expected results of this project might also be more
illuminating when set in the context of my thesis work. The prospect
of studying cold fronts in detail as a post-doc is exciting because it
expands upon my current work and also might give the results of my
thesis greater depth and meaning, and vice versa.

\clearpage
\begin{figure}[t]
    \begin{minipage}[t]{0.5\linewidth}
        \centering
	\includegraphics*[width=\textwidth, trim=28mm 8mm 30mm 10mm, clip]{splots}
        \caption{\small Radial entropy profiles of 169 clusters of
	galaxies in my thesis sample. The observed range of $K_0 \lesssim
	40$ keV cm$^2$ is consistent with models of episodic AGN
	heating. Color coding indicates global cluster temperature (in keV)
	derived from core excised apertures of size R$_{2500}$.}
	\label{fig:splots}
    \end{minipage}
    \hspace{0.1in}
    \begin{minipage}[t]{0.5\linewidth}
        \centering
        \includegraphics*[width=\textwidth, trim=28mm 8mm 30mm 10mm, clip]{tcool}
        \caption{\small Distribution of central cooling times for 169
	clusters in my thesis sample. The peak in the range of cooling
	times (several hundred Myrs) is consistent with inferred AGN
	duty cycles of both weak ($\sim 10^{40-50}$ ergs) and strong ($\sim
	10^{60}$ ergs) outbursts. However, note the distinct gap at $0.6-1$
	Gyr. An explanation for this bimodality does not currently exist.}
	\label{fig:tcool}
    \end{minipage}
    \hspace{0.1cm}
    \begin{minipage}[t]{0.5\linewidth}
        \centering
        \includegraphics*[width=\textwidth, trim=28mm 8mm 30mm 10mm, clip]{ha}
        \caption{\small Central entropy plotted against H$\alpha$
	luminosity. Orange dots are detections and black boxes with
	arrows are non-detection upper-limits. Notice the characteristic entropy threshold for star
	formation of $K_0 \lesssim 20$ keV cm$^2$. This is also the entropy scale at
	which conduction no longer balances radiative cooling and condensation
	of low entropy gas onto a cD can proceed.}
        \label{fig:ha}
    \end{minipage}
    \hspace{0.1in}
    \begin{minipage}[t]{0.5\linewidth}
        \centering
        \includegraphics*[width=\textwidth, trim=28mm 8mm 30mm 10mm, clip]{rad}
        \caption{\small Central entropy plotted against NVSS or PKS radio
	luminosity. Orange dots are detections and black boxes with
	arrows are non-detection upper-limits. There appears to be a dichotomy which might be related to AGN
	fueling mechanisms: AGN which are feed via low entropy gas, and the
	smattering of points at $K_0 > 50$ keV cm$^2$ which are likely
	fueled by mergers or have X-ray coronae which promote ICM cooling.}
        \label{fig:rad}
    \end{minipage}
\end{figure}
\end{document}
