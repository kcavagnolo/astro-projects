\documentclass[11pt]{article}
\usepackage[colorlinks=true,linkcolor=blue,urlcolor=blue]{hyperref}
\usepackage{subfig,epsfig,colortbl,graphics,graphicx,wrapfig,amssymb}
\usepackage{macros_cavag}
\pagestyle{myheadings}
\font\cap=cmcsc10
\setlength{\topmargin}{-0.2in}
\setlength{\oddsidemargin}{-0.1in}
\setlength{\evensidemargin}{0.in}
\setlength{\headheight}{0.1in}
\setlength{\headsep}{0.25in}
\setlength{\topskip}{0.1in}
\setlength{\textwidth}{6.5in}
\setlength{\textheight}{9.5in}

\markright{K.W. Cavagnolo Summary}

\begin{document}
\begin{center}
\textbf{Summary of Past Research and Future Interests}\\
\end{center}

The general process of galaxy cluster formation through hierarchical
merging is well understood, but many details, such as the impact of
feedback sources on the cluster environment and radiative cooling in
the cluster core, are not. My thesis research has focused on studying
these details via X-ray properties of the ICM in clusters of
galaxies. I have paid particular attention to ICM entropy
distribution, the process of virialization, and the role of AGN
feedback in shaping large scale cluster properties.

\subsection*{Mining the CDA}

My thesis makes use of a 350 observation sample (276 clusters; 11.6
Msec) taken from the {\it Chandra} archive. This massive
undertaking necessitated the creation of a robust reduction and
analysis pipeline which 1) interacts with mission specific software,
2) utilizes analysis tools (i.e. {\tt{XSPEC}}, {\tt{IDL}}), 3)
incorporates calibration and software updates, and 4) is highly
automated. Because my pipeline is written in a very general manner,
adding pre-packaged analysis tools from missions such as
{\textit{XMM}}, {\textit{Spitzer}}, and {\textit{VLA}} will be
straightforward. Most importantly, my pipeline deemphasizes data
reduction and accords me the freedom to move quickly into an analysis
phase and generating publishable results.

\subsection*{Quantifying Cluster Virialization}

The normalization, shape, and evolution of the cluster mass function
are useful for measuring cosmological parameters. Cluster evolution
tests the effect of dark matter and dark energy on the evolution of
dark matter halos, and therefore provides a complementary and
distinct constraint on cosmological parameters to those tests which
constrain them geometrically (e.g. supernovae and baryon acoustic
oscillations). If we could identify a parameter possibly reflecting
the degree of relaxation in the cluster we could improve the utility
of clusters as cosmological probes by parameterizing and reducing the
scatter in mass-observable scaling relations.

One study that examined how relaxation affects the observable
properties of clusters was conducted by Mathiesen and Evrard
2001. They found that most clusters which had experienced a recent
merger were cooler than the cluster mass-observable scaling relations
predicted. They attributed this to the presence of cool,
spectroscopically unresolved accreting subclusters.

I have followed up their work by studying the bandpass dependence in
determining X-ray temperatures and what this dependence tells us about
the virialization state of a cluster. The ultimate goal of this
project was to find an aspect-independent measure for a cluster's
dynamic state. I thus investigated the net temperature skew of the
hard-band (2.0-7.0 keV) and full-band (0.7-7.0 keV) temperature
ratio. I have found this temperature ratio is statistically connected
to mergers and the presence of cool cores. Having confirmed the
predicted effect, the next step is to make a comparison to the
predicted distribution of temperature ratios and their relationship to
putative cool lumps and/or non-thermal soft X-ray emission in cluster
simulations.

\subsection*{Cluster Feedback and ICM Entropy}

The picture of the ICM entropy-feedback connection emerging from my
work suggests cluster radio luminosity and H$\alpha$ emission are
anti-correlated with cluster central entropy. Following my analysis of 169
cluster radial entropy profiles (Fig. \ref{fig:splots}) I have found
an apparent bimodality in the distribution of central
entropy and central cooling times (Fig. \ref{fig:tcool}) which is
likely related to AGN feedback (and to a lesser extent, mergers). I
have also found that clusters with central entropy $\leq 20$ keV
cm$^2$ show signs of star formation (Fig. \ref{fig:ha}) and AGN
activity (Fig. \ref{fig:rad}) while clusters above this threshold
unilaterally do not have star formation and exhibit diminished AGN
radio feedback. This entropy level is auspicious as it coincides with
the Field length, $\lambda_F$, (assuming reasonable suppression from
magnetic fields) at which thermal conduction can stabilize a cluster
core against further cooling and gas condensation. It is possible my
work has opened a window to solving a long-standing problem in massive
galaxy formation (and truncation): how are ICM gas properties coupled
to feedback mechanisms such that the system becomes self-regulating?
But my thesis has also highlighted some unresolved and new
issues.

What is the origin of the bimodality in $K_0$ and $t_{cool}$?
What role is star formation playing in the feedback cycle of clusters?
How is energy generated on the parsec scale from a SMBH deposited
uniformly over volumes which are orders of magnitude larger? There are
also exciting theoretical cluster feedback model developments on the
horizon which will need observational investigation. Developments such
as: how exactly are AGN fueled -- through a combination of hot/cold
accretion, mergers, and consumption of low entropy gas via cooling; or
is there a universal mode underlying all these processes? Does
accretion of the hot ICM/ISM proceed via Bondi-eque flows or is it
more like Eddington accretion? What is the efficiency of accretion and
is energy return from a SMBH really the presumed $\sim 10\%$? Why do
we see steep metallicity gradients in the ICM/ISM when some amount of
turbulent mixing should take place? How is feedback energy distributed
symmetrically throughout the ICM?

\subsection*{Future Work}

Looking ahead, the natural extension of my thesis is to further study
questions regarding cluster environments and their impact on galaxy
formation. I'd also like to participate in analyzing X-ray follow-up
observations for clusters found using large SZE surveys. More
specifically, I'd like to use these samples to measure the evolution
of the cluster mass function as a direct means of breaking the
degeneracy between $\Omega_M$ and $\sigma_8$. Combined with
complementary surveys (specifically those using the SZE which will
yield tens of thousands of cluster candidates) X-ray surveys will help
further constrain the fundamental parameters defining the current
cosmological model.

But, the detailed analysis of the cluster population at redshifts
greater than z $\sim 1$ will be very difficult, and establishing the
self similar model as a reliable tool for calibrating the cluster mass
function will lead to better studies of hierarchical structure
formation and dark energy. In addition, if we are to use SZE as
effectively as desired SZE flux must be calibrated to accurately predict
cluster mass. But even calibration is not enough, we must also
understand the scatter in scaling relations. And to this end one needs
two components: verification of cluster candidates and methods for
quantifying deviation from mean mass-scaling relations. But the simple
application of existing metrics which have been calibrated to low-z
samples or high resolution simulations may begin to breakdown as
spatial and spectroscopic information is reduced at high
redshifts. There is the even worse possibility that scaling relations
evolve with redshift which will present a number of technical
difficulties all their own (e.g. covariance and time evolving
scatter). I look forward to being a part of generating new, novel
solutions to these problems.

With potentially enumerable, unbiased samples of clusters emerging
from SZE surveys and low flux, all-sky X-ray surveys, the entropy
distribution and signatures of feedback culled from these samples
could tell us a great deal about the evolution of clusters and galaxy
formation. Many questions remain unanswered in this area, such as:
What are the micro-physics of ICM heating, including the
thermalization of mechanical work done by bubbles and the effect of
non-thermal sources like cosmic rays. How prevalent are cold fronts
and can they be used as an indicator of merger activity and onset of
feedback? Also of interest are how accretion onto the cD SMBH is
regulated by large-scale ICM properties, what the AGN energy injection
function looks like, and how it correlates with cluster
environment. It will also be useful to have a low-scatter, universal
relation between jet power and radio power -- a tool which can then be
directly applied to understanding both cluster feedback and could
possibly be useful in SZE studies.

\clearpage
\begin{figure}[t]
    \begin{minipage}[t]{0.5\linewidth}
        \centering
        \includegraphics*[width=\textwidth, trim=28mm 8mm 30mm 10mm, clip]{splots}
        \caption{\small Radial entropy profiles of 169 clusters of
        galaxies in my thesis sample. The observed range of $K_0 \lesssim
        40$ keV cm$^2$ is consistent with models of episodic AGN
        heating. Color coding indicates global cluster temperature (in keV)
        derived from core excised apertures of size R$_{2500}$.}
        \label{fig:splots}
    \end{minipage}
    \hspace{0.1in}
    \begin{minipage}[t]{0.5\linewidth}
        \centering
        \includegraphics*[width=\textwidth, trim=28mm 8mm 30mm 10mm, clip]{tcool}
        \caption{\small Distribution of central cooling times for 169
        clusters in my thesis sample. The peak in the range of cooling
        times (several hundred Myrs) is consistent with inferred AGN
        duty cycles of both weak ($\sim 10^{40-50}$ ergs) and strong ($\sim
        10^{60}$ ergs) outbursts. However, note the distinct gap at $0.6-1$
        Gyr. An explanation for this bimodality does not currently exist.}
        \label{fig:tcool}
    \end{minipage}
    \hspace{0.1cm}
    \begin{minipage}[t]{0.5\linewidth}
        \centering
        \includegraphics*[width=\textwidth, trim=28mm 8mm 30mm 10mm, clip]{ha}
        \caption{\small Central entropy plotted against H$\alpha$
        luminosity. Orange dots are detections and black boxes with
        arrows are non-detection upper-limits. Notice the characteristic entropy threshold for star
        formation of $K_0 \lesssim 20$ keV cm$^2$. This is also the entropy scale at
        which conduction no longer balances radiative cooling and condensation
        of low entropy gas onto a cD can proceed.}
        \label{fig:ha}
    \end{minipage}
    \hspace{0.1in}
    \begin{minipage}[t]{0.5\linewidth}
        \centering
        \includegraphics*[width=\textwidth, trim=28mm 8mm 30mm 10mm, clip]{rad}
        \caption{\small Central entropy plotted against NVSS or PKS radio
        luminosity. Orange dots are detections and black boxes with
        arrows are non-detection upper-limits. There appears to be a dichotomy which might be related to AGN
        fueling mechanisms: AGN which are feed via low entropy gas, and the
        smattering of points at $K_0 > 50$ keV cm$^2$ which are likely
        fueled by mergers or have X-ray coronae which promote ICM cooling.}
        \label{fig:rad}
    \end{minipage}
\end{figure}
 
\end{document}
