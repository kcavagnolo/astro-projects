\documentclass[12pt]{article}
\usepackage[colorlinks=true,linkcolor=blue,urlcolor=blue]{hyperref}
\usepackage{macros_cavag,setspace}
\pagestyle{myheadings}
\font\cap=cmcsc10
\setlength{\topmargin}{-0.15in}
\setlength{\oddsidemargin}{-0.15in}
\setlength{\evensidemargin}{-0.15in}
\setlength{\headheight}{0.1in}
\setlength{\headsep}{0.25in}
\setlength{\topskip}{0.1in}
\setlength{\textwidth}{6.5in}
\setlength{\textheight}{9.25in}

\doublespacing
\begin{document}
\begin{singlespace}
\noindent{\bf{Michigan State University}}\\
{\bf{2003-present}}\\
{\bf{Dr. Megan Donahue}}\\
{\bf{The Entropy-Feedback Connection and Quantifying Cluster Virialization}}
\end{singlespace}
Some of the intracluster medium (ICM) has a cooling time shorter than the age of
the Universe. And as the ICM cools it radiates away much of the energy
acquired during formation. As the ICM cools, portions of it condense
and flow to the bottom of the cluster potential well. But, the
hypothesized products of these ``cooling flows'', such as stars or
molecular clouds, are not observed in the cores of clusters. There is
clearly some feedback mechanism operating within clusters which retards
unabated cooling. The most likely candidate for the feedback is
currently active galactic nuclei (AGN), and this area of study is
under heavy focus by both observationalists and theoreticians.

The picture of the ICM entropy-feedback connection emerging from my
work suggests that cD radio luminosity and H$\alpha$ emission are
anti-correlated with cluster central entropy. I have explored these
relations with my sample of clusters observed with {\textit{Chandra}}
and am finding a trend of high central entropy favoring low
L$_{H\alpha}$ and low L$_{Radio}$. These results fit well with the
current framework for AGN heating and cooling flow retardation through
the inflation of bubbles in the ICM and star formation in the cores of
cooling flows. I am following up these results by examining the
distribution of central cooling times as a window onto the timescale
of AGN feedback. In addition, I am exploring  the dependence of the
X-ray loud AGN distribution on redshift and amount of cluster
substructure.

One method of quantifying cluster substructure -- a property of
clusters which results in the underestimate of cluster temperatures
and therefore cluster mass -- employs the ratios of X-ray surface
brightness moments to quantify the degree of relaxation. Although an
excellent tool, power ratio suffers from being aspect dependent, much
like other substructure measures such as axial ratio or centroid
variation. The work of Mathiesen \& Evrard (2001) found an auxiliary
measure of substructure which does not depend on perspective and could
be combined with power ratio, axial ratio, and centroid variation to
yield a more robust metric for quantifying a cluster's degree of
relaxation.

I have studied this auxiliary measure: the bandpass dependence in
determining X-ray temperatures and what this dependence tells us about
the virialization state of a cluster. The ultimate goal of this
project is to find an aspect-independent measure for a cluster's
dynamic state. I have investigated the net temperature skew in my
sample of the hard-band (2.0$_{rest}$-7.0 keV) and full-band (0.7-7.0
keV) temperature ratio for core-excised apertures. I have found this
temperature ratio is statistically connected to mergers and the
presence of cool cores. Having confirmed the prediction of
Mathiesen \& Evrard (2001), the next step is to make a comparison to
the predicted distribution of temperature ratios and their
relationship to putative cool lumps and/or non-thermal soft X-ray
emission in cluster simulations.

\begin{singlespace}
\noindent{\bf{Michigan State University}}\\
{\bf{2002-2003}}\\
{\bf{Dr. Jack Baldwin}}\\
{\bf{{\textit{s}}-Process Abundances in Planetary Nebulae}}
\end{singlespace}
In my earliest work as a graduate student I identified and analyzed
spectral lines for the planetary nebulae IC 2501, IC 4191, and NGC
2440 using data taken with MIKE, a double echelle spectrograph at the
Las Campanas Observatory. In this work we found enhancements above
Solar of s-process element abundances. This is indicative of a
progenitor star which has experienced slow neutron capture and
dredge-up in its giant phase. These results go directly to the
efficiency of the s-process in stellar models, and ultimately to a
better understanding of ISM enrichment.

\begin{singlespace}
\noindent{\bf{Georgia Institute of Technology}}\\
{\bf{2000-2002}}\\
{\bf{Dr. James Sowell}}\\
{\bf{Orbital Solutions for the Eclipsing Binary ET Tau}}
\end{singlespace}
I participated in a project to find an orbital solution of the
Algol eclipsing binary ET Tau using UBV data taken with the 0.9 meter
Fernbank Telescope. By combining the light curves and radial velocity
curves over a baseline of four years we were able to generate a stable
solution using the Wilson-Devinney code which yielded a period of 6.0
days for a semi-detached configuration. We were also able to infer
mass, radius, temperature, luminosity, and specific gravity for both
of the binary members.

\end{document}
