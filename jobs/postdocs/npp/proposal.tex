\documentclass[11pt]{article}
\usepackage[colorlinks=true,linkcolor=blue,urlcolor=blue]{hyperref}
\usepackage{macros_cavag}
\pagestyle{myheadings}
\font\cap=cmcsc10
\setlength{\topmargin}{-0.15in}
\setlength{\oddsidemargin}{-0.15in}
\setlength{\evensidemargin}{-0.15in}
\setlength{\headheight}{0.1in}
\setlength{\headsep}{0.25in}
\setlength{\topskip}{0.1in}
\setlength{\textwidth}{6.5in}
\setlength{\textheight}{9.25in}

\begin{document}
\markright{K.W. Cavagnolo Research Proposal}
\begin{center}
\textbf{Investigating a Unified Model for Feedback in Cool Core Clusters}\\
\end{center}
\normalsize

\subsection*{Abstract}
Cluster mass functions and the evolution of the cluster mass function
are useful for measuring cosmological parameters
(\cite{1989ApJ...341L..71E}, \cite{1998ApJ...508..483W}, \cite{2001ApJ...553..545H},
\cite{2003PhRvD..67h1304H}, \cite{2004PhRvD..70l3008W}). Cluster evolution tests the
effect of dark matter and dark energy on the evolution of dark matter
halos, and therefore provide a complementary and distinct constraint
on cosmological parameters to those tests which constrain
them geometrically (e.g. supernovae (\cite{1998AJ....116.1009R},
\cite{2007ApJ...659...98R}) and baryon acoustic oscillations
(\cite{2005ApJ...633..560E})).

However, clusters are a useful cosmological tool only if we can infer
cluster masses -- the fundamental cluster property inferred from
cosmological simulations (\cite{1990ApJ...363..349E}) -- from
observable properties such as X-ray luminosity, X-ray temperature,
lensing shear, optical luminosity, and galaxy velocity
dispersion. Empirically, the relationship of mass and these observable
properties is well-established (\cite{2005RvMP...77..207V}). However,
if we could identify a ``3rd parameter" -- possibly reflecting the
degree of relaxation in the cluster -- we could improve the utility of
clusters as cosmological probes.

The general process of galaxy cluster formation through hierarchical
merging is well understood, but many details, such as the impact of
feedback sources on the cluster environment and radiative cooling in
the cluster core are not. My thesis research has focused on studying
these details via X-ray properties of the ICM in clusters of
galaxies. I have paid particular attention to ICM entropy
distribution, the process of virialization, and the role of AGN
feedback in shaping large scale cluster properties.

\subsection*{Statement of Problem}
\subsection*{Background and Relevance to Previous Work}
\subsection*{General Methodology and Procedure}
\subsection*{Explanation of Techniques}
\subsection*{Expected Results, Significance, and Application}

\clearpage
\bibliographystyle{unsrt}
\bibliography{cavagnolo}
 
\end{document}
