\documentclass[12pt]{proposal}
\usepackage[colorlinks=true,linkcolor=blue,urlcolor=blue]{hyperref}
\usepackage{macros_cavag}
\usepackage{setspace}
\pagestyle{myheadings}
\DeclareFontFamily{OT1}{psyr}{}
\DeclareFontShape{OT1}{psyr}{m}{n}{<-> psyr}{}
\def\times{{\fontfamily{psyr}\selectfont\char180}}

\begin{document}

\begin{center}
{\Large{\bf NASA Postdoctoral Program Research Proposal}}\\*[3mm]
{\bf Weighing Clusters: Reducing Feedback Related Scatter in
Mass-Observables and Constraining the Dark Energy Equation of State} \\*[3mm]

Kenneth W. Cavagnolo \\
Dr. Richard F. Mushotzky, Dr. Megan Donahue, Dr. G. Mark Voit

\end{center}

\doublespacing

\section*{\large Abstract}

\subsection*{\normalsize Intellectual Merit}

\subsection*{\normalsize Broader Impacts}

\newpage

\section*{\large Statement of Problem}

Cluster mass functions and the evolution of the cluster mass function
are useful for measuring cosmological parameters
\citep{1989ApJ...341L..71E, 1998ApJ...508..483W, 2001ApJ...553..545H,
2003PhRvD..67h1304H, 2004PhRvD..70l3008W}. Cluster evolution tests the
effect of dark matter and dark energy on the evolution of dark matter
halos, and therefore provide a complementary and distinct constraint
on cosmological parameters to those tests which constrain
them geometrically (e.g. supernovae \citep{1998AJ....116.1009R,
2007ApJ...659...98R} and baryon acoustic oscillations
\citep{2005ApJ...633..560E}).

However, clusters are a useful cosmological tool only if we can infer
cluster masses -- the fundamental cluster property inferred from
cosmological simulations \citep{1990ApJ...363..349E} -- from
observable properties such as X-ray luminosity, X-ray temperature,
lensing shear, optical luminosity, and galaxy velocity
dispersion. Empirically, the relationship of mass and these observable
properties is well-established \citep{2005RvMP...77..207V}.

\section*{\large Background and Relevance to Previous Work}

\section*{\large General Methodology and Procedure}

\section*{\large Explanation of New or Unusual Techniques}

\section*{\large Expected Results and Their Significance and Application}

\subsection*{\normalsize Intellectual Merit}

\subsection*{\normalsize Broader Impacts}

\newpage
\singlespacing
\pagenumbering{roman}

\bibliographystyle{jponew}
\bibliography{cavagnolo}

\end{document}
