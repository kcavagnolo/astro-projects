\documentclass[11pt]{article}
\usepackage[colorlinks=true,linkcolor=blue,urlcolor=blue]{hyperref}
\usepackage{subfig,epsfig,colortbl,graphics,graphicx,wrapfig,amssymb}
\usepackage{macros_cavag}
\pagestyle{myheadings}
\font\cap=cmcsc10
\setlength{\topmargin}{-0.2in}
\setlength{\oddsidemargin}{-0.1in}
\setlength{\evensidemargin}{0.in}
\setlength{\headheight}{0.1in}
\setlength{\headsep}{0.25in}
\setlength{\topskip}{0.1in}
\setlength{\textwidth}{6.5in}
\setlength{\textheight}{9.25in}

\markright{K.W. Cavagnolo Summary}

\begin{document}
\begin{center}
\textbf{Summary of Past Research and Future Interests}\\
\end{center}

The general process of galaxy cluster formation through hierarchical
merging is well understood, but many details, such as the impact of
feedback sources on the cluster environment and radiative cooling in
the cluster core are not. Mergers and feedback activity are interesting for
two reasons: they potentially compromise the use of clusters for
cosmological studies, and there is a tremendous amount of interesting
astrophysics going on. My thesis research has focused on studying
the details of feedback and mergers via X-ray properties of the ICM in
clusters of galaxies. I have paid particular attention to ICM entropy
distribution and the role of AGN feedback in shaping large scale
cluster properties. Additionally I have examined the quantification
of cluster virialization via aspect-independent metrics, with emphasis
on understanding temperature inhomogeneity as a surrogate for cluster
dynamic state.

\subsection*{Mining the CDA}

My thesis makes use of a 350 observation sample (276 clusters; 11.6
Msec) taken from the {\it Chandra} archive. This massive
undertaking necessitated the creation of a robust reduction and
analysis pipeline which 1) interacts with mission specific software,
2) utilizes analysis software (i.e. {\tt{XSPEC}}, {\tt{IDL}}), 3)
incorporates calibration and software updates, and 4) is highly
automated. Because my pipeline is written in a very general manner,
adding pre-packaged analysis tools from missions such as
{\textit{XMM}}, {\textit{Spitzer}}, and {\textit{VLA}} will be
straightforward. Most importantly, my pipeline deemphasizes data
reduction and accords me the freedom to move quickly into an analysis
phase and generating publishable results.

\subsection*{Quantifying Cluster Virialization}

Cluster mass functions and the evolution of the cluster mass function
are useful for measuring cosmological parameters. Cluster evolution
tests the effect of dark matter and dark energy on the evolution of
dark matter halos, and therefore provides a complementary and distinct
constraint on cosmological parameters to those tests which constrain
them geometrically (e.g. supernovae and baryon acoustic
oscillations).

Empirically, the relationship of mass and some observable properties
is well-established. However, if we could identify a set of parameters
-- possibly reflecting the degree of relaxation in the cluster -- we
could improve the utility of clusters as cosmological probes. The work
of Mathiesen and Evrard 2001 found an auxiliary measure of substructure
which does not depend on perspective and could be combined with power
ratio, axial ratio, and centroid variation to yield a more robust
metric for quantifying a cluster's degree of relaxation.

I have studied this auxiliary measure: the bandpass dependence in
determining X-ray temperatures and what this dependence tells us about
the virialization state of a cluster. The ultimate goal of this
project is to find an aspect-independent measure for a cluster's
dynamic state. To this end, I have investigated the net temperature skew in my
sample of the hard-band (2.0$_{rest}$-7.0 keV) and full-band (0.7-7.0
keV) temperature ratio for core-excised apertures. I have found this
temperature ratio is statistically connected to mergers and the
presence of cool cores. The next step is to make a comparison to
the predicted distribution of temperature ratios and their
relationship to putative cool lumps and/or non-thermal soft X-ray
emission in cluster simulations. This will be carried out by a fellow
graduate student as part of his thesis and funded by a successful
{\textit{Chandra}} theory proposal by Dr. Mark Voit which was
motivated by my work. In addition, this project has produced a first
author paper which is near ApJ submission.

\subsection*{Cluster Feedback and ICM Entropy}

The picture of the ICM entropy-feedback connection
(Fig. \ref{fig:splots}) emerging from my work suggests cluster radio
luminosity and H$\alpha$ emission are anti-correlated with cluster
central entropy ($K=T_Xn_e^{2/3}$). There also appears to be a
bimodality in the distribution of central cooling times
(Fig. \ref{fig:tcool}) which is likely related to AGN feedback (and to
a lesser extent, mergers). I have found that clusters with central
entropy $\leq 20$ keV cm$^2$ exhibit star formation
(Fig. \ref{fig:ha}) and AGN activity (Fig. \ref{fig:rad}) in the BCG
while clusters above this threshold unilaterally do not have star formation
and exhibit diminished AGN radio feedback. This entropy level is
auspicious as it coincides with the Field length, $\lambda_F$,
(assuming reasonable suppression) at which thermal conduction can
stabilize a cluster core. It is possible we have opened a window to
solving a long-standing problem in massive galaxy formation (and
truncation): how are ICM gas properties coupled to feedback mechanisms
such that the system becomes self-regulating? However, this result
serves to highlight unresolved issues requiring further intensive
study.\\

{\bf 1) What is the origin of the bimodality in $K_0$?}\\
Is it archival bias? Meaning, are clusters with $K_0 \sim 70$ keV
cm$^2$ ``boring'' (and faint) and thus have not been
proposed for observation? In which case I will select a representative
sample of clusters from a flux-limited survey, such as {\it ROSAT}
400$\square^\circ$, which predictably fill this gap and observe them with
{\it Chandra}. Or, is the gap physically driven? Is the gap
representative of a very short period in a clusters life when AGN
activity has boosted the core entropy to the point of being
conductively stable ($K_0 > 20$ keV cm$^2$) and subsequent mergers
have further elevated the ICM entropy to $K_0 > 100$ keV cm$^2$? A
possible answer to this question may be found in analysis of
simulations by asking the additional question: what is the timescale
for depletion of $\sim 10^{12-13} M_{\odot}$ subclusters in a full dark matter
halo? If this timescale is of the order a few Gyrs then this likely
points to a collusion of AGN feedback and mergers to give rise to
bimodality. But ultimately the questions I posed are related with two
primary underlying questions: what does the distribution of $K_0$ for
a complete sample of clusters look like? And what does the AGN energy
injection distribution look like?\\

{\bf 2) What role is star formation playing in the feedback cycle of clusters?}\\
Thus far, indications from the literature are that most (possibly all?)
BCGs in X-ray luminous clusters with $K_0 \leq 20$ keV cm$^2$ are
dominated by star formation. But we can see from Figure \ref{fig:rad} that most of
these systems contain radio AGN. So one can ask the question: are
there any AGN dominated nebular BCGs? An interesting project to pursue
with the {\it Spitzer} archive would be to examine the shape of
spectral energy distributions (SEDs) for all clusters with a BCG and
attempt to reveal if the BCG is star formation or AGN dominated.
A cross-reference of my thesis sample (which is essentially the
entire CDA) with the {\it Spitzer} data archive reveals 150+
clusters have already been observed by {\it Spitzer} (combinations of
75+ MIPS, 50+ IRAC, 30+ IRS) covering a broad entropy, luminosity (X-ray,
H$\alpha$, radio), and mass range. The large pool to draw from makes
selection of a representative subsample immediately possible. Does
star formation precede/inhibit/enhance/stunt AGN feedback? Currently
we do not know. All we know is these two processes are triggered in
cluster BCGs which reside in low entropy environments. Surely they are
coupled somehow, which is why I highlighted several poor clusters/rich
groups in Figures \ref{fig:ha} and \ref{fig:rad} with blue boxes and
red stars. These systems are in the proper regime for feedback, yet
they exhibit only one or neither of star formation or AGN. Follow-up of
these objects with {\it Spitzer} and {\it XMM}'s Optical Monitor to
search for polycyclic aromatic hydrocarbon features, UV excess, or
dusty AGN would be interesting.\\

{\bf 3) How is energy generated on the parsec scale from a SMBH
deposited uniformly in the ICM over a few cubic megaparsecs?}\\
As you are well aware and have shown quite elegantly through your work
on Perseus, the role of AGN feedback in shaping global cluster
properties is quite complex and to some extent poorly
understood. Models for the process of thermalizing energy in AGN blown
bubbles have been proposed, but details of these models still need to be explored.
For example, do bubbles contain a very low density non-relativistic
thermal plasma or are they truly voids in the ICM? We'd like to know
if bubbles are pressure supported, and this could be studied via SZ
effects. Radio sources are also being revealed as much more powerful
than ever expected now that they have been observed at low radio
frequencies (i.e. 330 MHz). Use of surveys such as LOFAR, LWA and
EVLA will make study of clusters across a broad radio range a rich
field for years to come. Also, what is the contribution of cosmic rays in
bubbles? The presence of cosmic rays should be detectable with GLAST
using observation of $\gamma$-rays from the decay of $\pi^0$ in bubble
lobes. How do bubbles rise to distances $\geq 100$ kpc without being
shredded by instabilities? What is the role of $\vec{B}$ fields in
stabilizing bubbles? And what is the origin of these fields? This area
of cluster feedback studies is littered with more questions than
current answers, which makes for an attractive research avenue for a
post-doc to write many observing and grant proposals.\\

I have attempted to highlight without too much depth the areas I have
already worked and the directions I would like to go. Most of my
experience is with X-ray data, but multiwavelength analysis is the
next necessary step in my career, and I hope it will be under your
direction at IoA.

\clearpage
\begin{figure}[t]
    \begin{minipage}[t]{0.5\linewidth}
        \centering
	\includegraphics*[width=\textwidth, trim=28mm 8mm 30mm 10mm, clip]{splots}
        \caption{\small Radial entropy profiles of 143 clusters of
	galaxies in my thesis sample. The observed range of $K_0 \lesssim
	40$ keV cm$^2$ is consistent with models of episodic AGN
	heating. Color coding indicates global cluster temperature (in keV)
	derived from core excised apertures of size R$_{2500}$.}
	\label{fig:splots}
    \end{minipage}
    \hspace{0.1in}
    \begin{minipage}[t]{0.5\linewidth}
        \centering
        \includegraphics*[width=\textwidth, trim=28mm 8mm 30mm 10mm, clip]{tcool}
        \caption{\small Distribution of central cooling times for an
	unbiased sub-sample of the clusters analyzed for my
	thesis. The peak in the range of cooling times (several hundred Myrs)
	is consistent with inferred AGN duty cycles of both weak ($\sim
	10^{40-50}$ ergs) and strong ($\sim 10^{60}$ ergs)
	outbursts. However, note the distinct gap at $0.6-1$ Gyr. An
	explanation for this bimodality does not currently exist.}
	\label{fig:tcool}
    \end{minipage}
    \hspace{0.1cm}
    \begin{minipage}[t]{0.5\linewidth}
        \centering
        \includegraphics*[width=\textwidth, trim=28mm 8mm 30mm 10mm, clip]{ha}
        \caption{\small Central entropy plotted against H$\alpha$
	luminosity. Orange dots are detections, black boxes with
	arrows are non-detection upper-limits, and blue boxes with red stars
	are poor clusters/rich groups which do not match the
	trend. Notice the characteristic entropy threshold for star
	formation of $K_0 \lesssim 20$ keV cm$^2$. This is also the entropy scale at
	which conduction no longer balances radiative cooling and condensation
	of low entropy gas onto a BCG can proceed.}
        \label{fig:ha}
    \end{minipage}
    \hspace{0.1in}
    \begin{minipage}[t]{0.5\linewidth}
        \centering
        \includegraphics*[width=\textwidth, trim=28mm 8mm 30mm 10mm, clip]{rad}
        \caption{\small Central entropy plotted against NVSS or PKS radio
	luminosity. Orange dots are detections, black boxes with
	arrows are non-detection upper-limits, and blue boxes with red stars
	are poor clusters/rich groups which do not match the
	trend. There appears to be a dichotomy which might be related to AGN
	fueling mechanisms: AGN which are feed via low entropy gas, and the
	smattering of points at $K_0 > 50$ keV cm$^2$ which are likely
	fueled by mergers.}
        \label{fig:rad}
    \end{minipage}
\end{figure}
\end{document}
