\documentclass[11pt]{article}
\usepackage[colorlinks=true,linkcolor=blue,urlcolor=blue]{hyperref}
\usepackage[T1]{fontenc}
\usepackage{subfig,epsfig,colortbl,graphics,graphicx,wrapfig,amssymb,common,mathptmx}
\setlength{\topmargin}{-0.15in}
\setlength{\oddsidemargin}{-0.12in}
\setlength{\evensidemargin}{0in}
\setlength{\headheight}{0in}
\setlength{\headsep}{0.0in}
\setlength{\topskip}{0.0in}
\setlength{\textwidth}{6.6in}
\setlength{\textheight}{9.5in}
\pagestyle{empty}

\begin{document}
\begin{center}
\large
\textbf{Summary of Experience and Future Interests}
\normalsize
\end{center}

The general process of galaxy cluster formation through hierarchical
merging is well understood, but many details, such as the impact of
feedback sources on the cluster environment and radiative cooling in
the cluster core, are not. My thesis research has focused on studying
these details via X-ray properties of the ICM in clusters of
galaxies. I have paid particular attention to ICM entropy
distribution, the process of virialization, and the role of AGN
feedback in shaping large scale cluster properties.

\subsection*{Mining the CDA}

My primary research makes use of a 350 observation sample (276
clusters; 11.6 Msec) taken from the \chandra\ archive. Of these 276
clusters, 16 lie in the redshift range $0.6 < z < 1.2$. Ongoing and
future X-ray surveys will be heavily focused on the cluster population
at $z > 1.0$. By gaining experience with low count, low surface
brightness clusters now, I am amply prepared to work with much larger
datasets of these objects in the future. In addition, this massive
undertaking necessitated the creation of a robust reduction and
analysis pipeline which 1) interacts with mission specific software,
2) utilizes analysis software (\eg\ \xspec, \idl), 3)
incorporates calibration and software updates, and 4) is highly
automated. Because my pipeline is written in a very general manner,
adapting the pipeline for use with pre-packaged analysis tools from
missions such as \xmm, \spitzer, and \vla\ will be
straightforward. Most importantly, my pipeline deemphasizes data
reduction and accords me the freedom to move quickly into an analysis
phase and generating publishable results.

\subsection*{Quantifying Cluster Virialization}

The normalization, shape, and evolution of the cluster mass function
are useful for measuring cosmological parameters. The evolution of large
scale structure formation is a test of how dark matter and dark energy
effect the cluster-scale evolution of dark matter halos, and therefore
provides a complementary and distinct constraint on cosmological
parameters to those tests which constrain them geometrically,
such as supernovae and baryon acoustic oscillations.

However, clusters are a useful cosmological tool only if we can infer
cluster masses from observable properties such as X-ray luminosity,
X-ray temperature, lensing shear, optical luminosity, or galaxy
velocity dispersion. Empirically, the correlation of mass to these
observable properties is well-established. However, if we could
identify a ``2nd parameter" -- possibly reflecting the degree of
relaxation in the cluster -- we could improve the utility of clusters
as cosmological probes by parameterizing and reducing the scatter in
mass-observable scaling relations.

One empirical method of quantifying the degree of relaxation involves
using ICM substructure and employs the power in ratios of X-ray surface
brightness moments. Although an excellent tool, power ratios suffer
from being aspect-dependent. The work of Mathiesen \& Evrard 2001
suggested a complementary measure of substructure which does not
depend on projected perspective and could be combined with power
ratio, axial ratio, and centroid variation to yield a more robust
metric for quantifying a cluster's degree of relaxation.

I have studied this auxiliary measure: the bandpass dependence in
determining X-ray temperatures and what this dependence tells us about
the virialization state of a cluster. The ultimate goal of this
project was to find an aspect-independent measure for a cluster's
dynamic state. To this end, I have investigated the net temperature
skew in my archive sample of the hard-band (2.0$_{rest}$-7.0 \keV) and
full-band (0.7-7.0 \keV) temperature ratio for core-excised
apertures. I have found this temperature ratio is statistically
connected to mergers and the presence of cool cores. The next step is
to make a comparison to the predicted distribution of temperature
ratios and their relationship to putative cool lumps and/or
non-thermal soft X-ray emission in cluster simulations. This will be
carried out by a fellow graduate student as part of his thesis and
funded by a successful \chandra\ theory proposal by Dr. Mark
Voit which was motivated by my work. In addition, this project has
produced a first author paper and has further stimulated the
discussion for the continuing need of accurate cross-calibration
between \xmm\ and \chandra.

\subsection*{Cluster Feedback and ICM Entropy}

The picture of the ICM entropy-feedback connection emerging from my
research suggests cluster cD radio luminosity and core H$\alpha$
emission are anti-correlated with cluster central entropy. Following
analysis of 169 cluster radial entropy profiles
(Fig. \ref{fig:splots}), I have found bimodality in the distribution
of central entropy and central cooling times (Fig. \ref{fig:tcool})
which is likely related to AGN feedback (and to a lesser extent,
mergers). I have also found that clusters with central entropy
$\lesssim 20$ keV cm$^2$ show signs of star formation
(Fig. \ref{fig:ha}) and AGN activity (Fig. \ref{fig:rad}), while
clusters above this threshold unilaterally do not have star formation
and exhibit diminished AGN radio feedback. This entropy level is
auspicious as it coincides with the Field length at which thermal
conduction can stabilize a cluster core against ICM
condensation. These results are highly suggestive that conduction in
the cluster core is very important to solving the long-standing
problem of how ICM gas properties are coupled to feedback mechanisms
such that the system becomes self-regulating.

The final phase of my thesis is focused on further understanding why
we observe bimodality, what role star formation is playing in the cluster
feedback loop, refining a model for how conduction couples feedback to
the ICM, and examining the peculiar class of objects which fall below
the Field length criterion but {\it do not} have star formation and/or
radio-loud AGN (blue boxes with red stars in two of the figures).

There are additional areas of my present research I'd like to expand
on in the future. {\bf(1)} To check if bimodality is archival bias, I am
submitting a \chandra\ Cycle 10 observing proposal for a sample of
clusters which predictably fall into the $t_{\mathrm{cool}}$ and $K_0$
gaps. {\bf(2)} Two classes of peculiar objects warrant intensive
multiwavelength study: high-$K_0$ clusters with radio-loud AGN
(\eg\ AWM4) and low-$K_0$ clusters without any feedback sources
(\eg\ Abell 2107). The former likely have prominent X-ray corona,
while the latter may be showing evidence that extremely low entropy
cores inhibit the growth of gas density contrasts. {\bf(3)} Thus far I have
only focused on AGN which are radio-loud according to the 1.4 GHz eye
of NVSS, but recent work has shown AGN radio halos are very powerful
at low frequencies too. I'd like to know what the radio power is at
these wavelengths for (ideally) my entire thesis sample and see if the
$K_0$-radio correlation tightens. {\bf(4)} Using the near-UV sensitivity of
{\it XMM}'s Optical Monitor and the far-IR channels of {\it spitzer},
I plan to propose a joint archival project to disentangle which $K_0
\lesssim 20$ cDs are star formation dominated and which are AGN
dominated. 

\subsection*{Future Work with \planck}

As I mentioned earlier, there are several extensions of my thesis work
which I can pursue independently. But in the context of the post-doc
position at Saclay and specifically working toward exploitation of
the \planck\ cluster catalogue, I see a multitude of projects. Unless
it is possible to commandeer \xmm\ and \chandra\ for an entire year,
X-ray follow-up of every SZ detected cluster is not possible. It is
therefore of the utmost importance to calibrate SZ flux to mass
scale so that a robust scaling relation can be used to directly infer
masses from SZ observations. But while this sounds simple, there are
complications which must be sorted out prior to the analysis of a
large SZ cluster catalogue like \planck's.

Existing studies so far suggest there is no redshift evolution in
X-ray mass-scaling relations. But these studies suffer from a major
flaw: they are hardly complete or unbiased. It would be wise to
carefully select a representative sample of clusters, calculate their
masses using a ``robust, low-scatter'' proxy (\eg\ the $Y_X$
parameter of Kravtsov \etal\ 2006), and check for redshift dependence
in mass-scaling relations. There is the added complication that
non-gravitational effects in clusters (\ie\ AGN feedback, radiative
cooling, and especially mergers) become more important at higher
redshifts and at the lower end of the mass spectrum where the SZ
effect will be a valuable probe. Understanding how these processes
conspire to scatter a cluster away from tight scaling relations will
also be integral to utilizing SZ flux for mass determination.

Beyond the technical issues and preparatory work, the \planck\ cluster
catalogue will be a powerful observational tool. One can readily
select interesting sub-samples (such as the 100 brightest SZ clusters)
to be used in other studies. For example a study of their entropy
distributions using \chandra\ data. From my own work, and the work of people
like Ian McCarthy and Michael Balogh, we know some clusters must have
experienced some amount of ``pre-heating'' to reach the entropy levels
seen in the ICM at present ($K > 150 \ent$). Because the normalization
of the Y-M relation is sensitive to pre-heating, the SZ effect can
be used to place constraints on the level of pre-heating which
occurred at high-redshift, and thus can tell us about the feedback
mechanisms which were active in clusters at early epochs -- mechanisms
which most likely played a role in shaping properties of the earliest
galaxies. There are many other uses for the catalogue: combining SZ
and X-ray data to constrain $H_0$, analyzing the SZ power spectrum of
clusters to constrain the dark energy equation of state, measuring
$f_{gas}$ for a sample of clusters and truly testing it's utility as a
cosmology tool, the presence of high energy electrons (even at the low
spatial resolution of \planck) would be confirmation that non-thermal
processes are important in cluster formation (\eg\ from AGN bubbles),
and there is even the possibility that cooling flows could be
identified by culling outliers from scaling relations.

Models of cluster formation, evolution, feedback, and dynamics are
converging such that use of clusters in high precision cosmology is
possible. I have the skill sets necessary to make meaningful and
unique contributions both now and in the future of this field. Given
the opportunity to branch out from my X-ray roots, I will be an
excellent collaborator for maximizing the scientific returns of
\planck's galaxy cluster studies.

%- find clusters with SZ
%- get redshifts from optical and infrared photometry
%- get gas density and temperature from X-ray
%- get Mgas by direct measure
%- get Mtot from hydrostat assump
%- sniffle... cosmo dimming (1+z)^-4
%- ahhh! SZ flux is independent of redshift
%- use joint SZ/X-ray analysis to get masses
%- either get X-ray data for everything (not possible) or...
%- calibrate SZ flux to mass scale and then infer masses
%- planck has advantage of being complete over a very large redshift range
%	+ test for evolution in scaling relations
%	+ test for sensitivity to mergers
%	+ look for outliers because of non-grav phys
%- SZ effect is fairly sensitive probe of to non-grav heating
%       + disting between SS and pre-heating models
%	+ place constraints of level of pre-heating
%- prep work for catalogue
%       + make mock SZ effect obs (or use those from VCE)
%	+ fold in inst resp
%	+ place constraints on non-grav phys out to z of ???
%- cluster finding with SZ is done by studying the ``contaminant'' emission of the clusters on the CMB signal
%- thermal AND kinetic SZ signal will be measured
%- clusters not spatially resolved
%- difference of 353 MHz and 217 MHz channels reveals clusters at maxima
%note: what's ang res of planck? 10'
%note: what are effects of cooling on SZ scaling relations?
%       + K0 as a fnc of z for a complete sample of clusters
%	+ SZ is gas pressure based, core is over-pressured in entropy floor, thus SZ effect
%	+ heating/cooling modifies entropy, and thus pressure, thus SZ is modified
%	+ thus, SZ scaling rel says something of ent history
%	+ no cosmo dimming, go to arbitrarily high redshift
%	+ constraining ent floor out to high-z yields info on source of excess entropy:
%	        ++ pre-heating
%		++ cooling
%		++ or something else... reionization?
%		++ early SN
%	+ ``central compton parameter'': use beta-model to estimate y0
%	        ++ use JACO to do...?

\begin{figure}[t]
    \begin{minipage}[t]{0.5\linewidth}
        \centering
        \includegraphics*[width=\textwidth, trim=28mm 8mm 30mm 10mm, clip]{splots}
        \caption{\small Radial entropy profiles of 169 clusters of
        galaxies in my thesis sample. The observed range of $K_0 \lesssim
        70$ keV cm$^2$ is consistent with models of episodic AGN
        heating. Color coding indicates global cluster temperature (in keV)
        derived from core excised apertures of size R$_{2500}$.}
        \label{fig:splots}
    \end{minipage}
    \hspace{0.1in}
    \begin{minipage}[t]{0.5\linewidth}
        \centering
        \includegraphics*[width=\textwidth, trim=28mm 8mm 30mm 10mm, clip]{tcool}
        \caption{\small Distribution of central cooling times for 169
        clusters in my thesis sample. The peak in the range of cooling
        times (several hundred Myrs) is consistent with inferred AGN
        duty cycles of both weak ($\sim 10^{40-50}$ ergs) and strong ($\sim
        10^{60}$ ergs) outbursts. However, note the distinct gap at $0.6-1$
        Gyr. An explanation for this bimodality does not currently exist.}
        \label{fig:tcool}
    \end{minipage}
    \hspace{0.1cm}
    \begin{minipage}[t]{0.5\linewidth}
        \centering
        \includegraphics*[width=\textwidth, trim=28mm 8mm 30mm 10mm, clip]{ha_k0}
        \caption{\small Central entropy plotted against H$\alpha$
        luminosity. Orange dots are detections and black boxes with left-facing
        arrows are non-detection upper-limits. Notice the characteristic entropy threshold for star
        formation of $K_0 \lesssim 20$ keV cm$^2$. This is also the entropy scale at
        which conduction no longer balances radiative cooling and condensation
        of low entropy gas onto a cD can proceed.}
        \label{fig:ha}
    \end{minipage}
    \hspace{0.1in}
    \begin{minipage}[t]{0.5\linewidth}
        \centering
        \includegraphics*[width=\textwidth, trim=28mm 8mm 30mm 10mm, clip]{k0rad}
        \caption{\small Central entropy plotted against NVSS radio
        luminosity. Orange dots are detections and black boxes with left-facing
        arrows are non-detection upper-limits. Radio-loud AGN clearly
        prefer low entropy environs but the dispersion at low luminosity is
        large. It would be interesting to radio date these sources as this
        figure may have an age dimension.}
        \label{fig:rad}
    \end{minipage}
\end{figure}

\bibliographystyle{unsrt}
\bibliography{cavagnolo}
 
\end{document}
