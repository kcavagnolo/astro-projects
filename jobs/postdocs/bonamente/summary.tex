\documentclass[11pt]{article}
\usepackage[colorlinks=true,linkcolor=blue,urlcolor=blue]{hyperref}
\usepackage{macros_cavag}
\pagestyle{myheadings}
\font\cap=cmcsc10
\setlength{\topmargin}{-0.15in}
\setlength{\oddsidemargin}{-0.15in}
\setlength{\evensidemargin}{-0.15in}
\setlength{\headheight}{0.1in}
\setlength{\headsep}{0.25in}
\setlength{\topskip}{0.1in}
\setlength{\textwidth}{6.5in}
\setlength{\textheight}{9.25in}

\begin{document}
\markright{K.W. Cavagnolo Summary of Research}
\begin{center}
\textbf{Summary of Research}\\
\end{center}
\normalsize

The general process of galaxy cluster formation through hierarchical
merging is well understood, but many details, such as the impact of
feedback sources on the cluster environment and radiative cooling in
the cluster core, are not. My thesis research has focused on studying
these details via X-ray properties of the ICM in clusters of
galaxies. I have paid particular attention to ICM entropy
distribution, the process of virialization, and the role of AGN
feedback in shaping large scale cluster properties.

\subsection*{Mining the CDA}

My primary research makes use of a 350 observation sample (276
clusters) taken from the {\textit{Chandra}} archive. Of these 276
clusters, 16 lie in the redshift range $0.6 <$ z $< 1.2$. Ongoing and
future X-ray surveys will be heavily focused on the cluster population
at z $> 1.0$. By gaining experience with low count, low surface
brightness clusters now I am amply prepared to work with much larger
datasets of these objects in the future. In addition, this massive
undertaking necessitated the creation of a robust reduction and
analysis pipeline which 1) interacts with mission specific software,
2) utilizes analysis software (i.e. {\tt{XSPEC}}, {\tt{IDL}}), 3)
incorporates calibration and software updates, and 4) is highly
automated. Because my pipeline is written in a very general manner,
adding pre-packaged analysis tools from missions such as
{\textit{XMM}}, {\textit{Spitzer}}, and {\textit{VLA}} will be
straightforward. Most importantly, my pipeline deemphasizes data
reduction and accords me the freedom to move quickly into an analysis
phase and generating publishable results.

\subsection*{Quantifying Cluster Virialization}

Cluster mass functions and the evolution of the cluster mass function
are useful for measuring cosmological parameters. Cluster evolution
tests the effect of dark matter and dark energy on the evolution of
dark matter halos, and therefore provides a complementary and distinct
constraint on cosmological parameters to those tests which constrain
them geometrically (e.g. supernovae and baryon acoustic
oscillations).

However, clusters are a useful cosmological tool only if we can infer
cluster masses from observable properties such as X-ray luminosity,
X-ray temperature, lensing shear, optical luminosity, and
galaxy velocity dispersion. Empirically, the relationship of mass and
these observable properties is well-established. However, if we could
identify a ``3rd parameter" -- possibly reflecting the degree of
relaxation in the cluster -- we could improve the utility of clusters
as cosmological probes.

One method of quantifying cluster substructure -- a property of
clusters which results in the underestimate of cluster temperatures
and therefore cluster mass -- employs the ratios of X-ray surface
brightness moments to quantify the degree of relaxation. Although an
excellent tool, power ratio suffers from being aspect dependent, much
like other substructure measures such as axial ratio or centroid
variation. The work of \cite{2001ApJ...546..100M} found an auxiliary
measure of substructure which does not depend on perspective and could
be combined with power ratio, axial ratio, and centroid variation to
yield a more robust metric for quantifying a cluster's degree of
relaxation.

I have studied this auxiliary measure: the bandpass dependence in
determining X-ray temperatures and what this dependence tells us about
the virialization state of a cluster. The ultimate goal of this
project is to find an aspect-independent measure for a cluster's
dynamic state. I have investigated the net temperature skew in my
sample of the hard-band (2.0$_{rest}$-7.0 keV) and full-band (0.7-7.0
keV) temperature ratio for core-excised apertures. I have found this
temperature ratio is statistically connected to mergers and the
presence of cool cores. Having confirmed the prediction of
\cite{2001ApJ...546..100M}, the next step is to make a comparison to
the predicted distribution of temperature ratios and their
relationship to putative cool lumps and/or non-thermal soft X-ray
emission in cluster simulations. This will be carried out by a fellow
graduate student as part of his thesis and funded by a successful
{\textit{Chandra}} theory proposal by Dr. Mark Voit which cites my
work. In addition, this project has produced a first author paper
which is near ApJ submission.

\subsection*{Cluster Feedback and ICM Entropy}

The picture of the ICM entropy-feedback connection emerging from my
work suggests that cD radio luminosity and H$\alpha$ emission are
anti-correlated with cluster central entropy. I have explored these
relations with my thesis sample and am finding a trend of high central
entropy favoring low L$_{H\alpha}$ and low L$_{Radio}$. I have also
found the distribution central entropy and central cooling times are
bimodal - a result which has implications for the timescales of
feedback mechanisms operating in the core of clusters. The newest
result from my work is a correlation between cD black hole mass
and central entropy, which is more evidence feedback is regulated by
AGN activity. These results fit well with the current framework for
AGN heating and cooling flow retardation through the inflation of
bubbles in the ICM and star formation in the cores of cooling
flows. In addition, I am exploring the dependence of the X-ray loud
AGN distribution on redshift and amount of cluster substructure.

This work has been very fruitful thus far: I am a co-author for two
refereed journal papers (\cite{2007AJ....134...14D},
\cite{2006ApJ...643..730D}), generated new and unique work each year
(\cite{2008AAS},  \cite{2007Chandrasym}, \cite{2006AAS...209.7711D},
\cite{2005AAS...20713903C}, \cite{2004AAS...20514715C},
\cite{2004AAS...205.6020D}), a first author paper which is in draft,
and another first author paper in preparation containing my thesis
results. I have also contributed to several successful
{\textit{Chandra}}, {\textit{XMM}, {\textit{Suzaku}}, and
{\textit{Subaru}} proposals in addition to writing my own high
scoring -- although unsuccessful -- {\textit{Chandra}} proposal for time
observing an amazing ULIRG. I am also planning H$\alpha$ imaging
observations for several previously unobserved clusters with MSU's
SOAR telescope.

\subsection*{Future Work}

Looking ahead, the natural extension of my thesis is to further study
questions regarding cluster environments and their impact on galaxy
formation and participating in the analysis of large samples of
clusters found in SZE surveys and followed up with X-ray
observations. More specifically, I'd like to use these samples to 
measure the evolution of the cluster mass function as a direct means
of breaking the degeneracy between $\Omega_M$ and $\sigma_8$. Combined
with complimentary surveys (specifically those using the SZE which
will yield tens of thousands of cluster candidates) X-ray surveys will
help further constrain the fundamental parameters defining the current
cosmological model.

But, the detailed analysis of the cluster population at redshifts
greater than z $\sim 1$ will be very difficult, and establishing the
self similar model as a reliable tool for calibrating the cluster mass
function will lead to better studies of hierarchical structure
formation and dark energy. In addition, if we are to use SZE as
effectively as desired SZE flux must be calibrated to accurately predict
cluster mass. But even calibration is not enough, we must also
understand the scatter in scaling relations. And to this end one needs
two components: verification of cluster candidates and methods for
quantifying deviation from mean mass-scaling relations (such as those
discussed earlier or the $Y_X$ parameter of
\cite{2006ApJ...650..128K}). But the simple application of existing
metrics which have been calibrated to low-z samples or high resolution
simulations may begin to breakdown as spatial and spectroscopic
information is reduced at high redshifts, or if there is evolution in
scaling relations with redshift. I look forward to being a part of
generating new, novel solutions to these problems.

With potentially enumerable, unbiased samples of clusters emerging from SZE
surveys and low flux, all-sky X-ray surveys, the entropy distribution
and signatures of feedback culled from these samples could tell us
a great deal about the evolution of clusters and galaxy formation. 
Many questions remain unanswered in this area, such as:
What are the micro-physics of ICM heating, including the thermalization of
mechanical work done by bubbles and the effect of non-thermal sources
like cosmic rays. How prevalent are cold fronts and can they be used
as an indicator of merger activity and onset of feedback? Also of
interest are how accretion onto the cD SMBH is regulated by
large-scale ICM properties, what the AGN energy injection function
looks like, and how it correlates with cluster environment. It will
also be useful to have a low-scatter, universal relation between jet
power and radio power -- a tool which can then be directly applied to
understanding both cluster feedback and could possibly be useful in
SZE studies.

There are also exciting theoretical cluster feedback model
developments on the horizon which will need observational
investigation. Developments such as: how exactly are AGN fueled --
through a combination of hot/cold accretion (\cite{hotcoldaccretion}),
mergers, and consumption of low entropy gas via cooling; or is there a
universal mode underlying all these processes? Does accretion of the
hot ICM/ISM proceed via Bondi-eque flows or is it more like Eddington
accretion? What is the efficiency of accretion and is energy return
from a SMBH really the presumed $\sim 10\%$? Why do we see steep
metallicity gradients in the ICM/ISM when some amount of turbulent
mixing should take place? How is feedback energy distributed
symmetrically throughout the ICM?

Models of cluster formation, evolution, feedback, and dynamics are
converging such that use of clusters in high precision cosmology is
possible. I have the skill sets necessary to make meaningful and
unique contributions both now and in the future of this field.

\bibliographystyle{unsrt}
\bibliography{cavagnolo}
 
\end{document}
