% more spacious
\documentclass[12pt]{article}
\parindent 0pt
\parskip
\baselineskip
\setlength{\topmargin}{-0.30in}
\setlength{\oddsidemargin}{-0.30in}
\setlength{\evensidemargin}{-0.30in}
\setlength{\headheight}{0in}
\setlength{\headsep}{0.25in}
\setlength{\topskip}{0.25in}
\setlength{\textwidth}{6.9in}
\setlength{\textheight}{9.25in}
\pagestyle{empty}
\usepackage[T1]{fontenc}
\usepackage{subfig,epsfig,colortbl,graphics,graphicx,wrapfig,amssymb,common,mathptmx,multicol,natbib,setspace}
\begin{document}

\doublespacing
\begin{center}Kenneth W. Cavagnolo\end{center}
The gravitational binding energy liberated by active galactic nuclei
(AGN) plays a vital role in the process of hierarchical structure
formation \cite[\eg][]{perseus1, croton06, bower06, saro06, sijacki07,
  birzan08}. Observations robustly indicate most galaxies harbor a
centralized supermassive black hole (SMBH) which likely co-evolved
with the host galaxy giving rise to the well-known bulge
luminosity-stellar velocity dispersion correlation
\cite{1995ARA&A..33..581K, magorrian}. A key component in the galaxy
formation paradigm which explains these observed correlations is that
host environment thermodynamics are regulated via feedback from AGN
\cite{2002MNRAS.333..145N, mcnamrev}. In broad terms, this model is
successful in reproducing the bulk properties of the Universe,
specifically the thermal properties of the intracluster medium (ICM)
in galaxy clusters and the intragroup medium (IGM) in galaxy
groups. However, the details of ICM/IGM evolution under the influence
of AGN activity is still poorly understood, as is the ICM/IGM
non-thermal component. There are many open questions regarding ICM/IGM
magnetic fields, the origins of diffuse cluster-scale radio halos, and
how AGN feedback is coupled to environment. It is these open questions
which interest me as I develop a more diverse research program.

{\bf{Relevant Completed Research}}

Part of my research program has focused primarily on understanding the
mechanical feedback from AGN and the associated effects on galaxy
clusters. I have devoted particular attention to ICM entropy
distribution \cite{accept}, the process of cluster virialization
\cite{xrayband}, the mechanisms by which SMBHs might acquire fuel from
their environments \cite{conduction}, and how those mechanisms
correlate with properties of clusters cores \cite{haradent}. From
these studies it has become apparent that certain conditions must be
established within a cluster core (and presumably any environment
which supplies fuel for a SMBH, \eg\ cool coronae \cite{coronae}),
namely that the mean entropy, $K$, of a large-scale environment
hosting a SMBH must be $K \la 30~\ent$.

By a coincidence of scaling, $K \sim 30~\ent$ is the entropy scale
above which thermal electron conduction is capable of stabilizing gas
against thermal instability. This link between large-scale environment
and small-scale structure formation hints at a mechanism for
channeling AGN feedback energy to cooling regions. If conduction
operates in this fashion, then it may be a solution to the
long-standing problem of tuning AGN heating to establish a
self-regulating feedback loop. However, it is well-known that
conduction on its own does not operate efficiently within the ICM, and
that for most clusters, conduction has a minor role in defining ICM
properties \cite{2001ApJ...562L.129N, 2002ApJ...581..223R,
  2004MNRAS.347.1130V, dunn08}.

But, if magnetohydrodynamic (MHD) processes like the
heat-flux-driven-buoyancy instability
\cite[HBI,][]{2008ApJ...677L...9P} are functioning in large-scale
environments with cooling times $\ll \Hn^{-1}$, then conduction may be
important after all. In the presence of reasonable magnetic fields
($\sim 1~\mu$G), modest AGN heating ($\sim 10^{43}~\lum$) and subsonic
turbulence, full MHD simulations have shown that the HBI aides
conduction in stabilizing the cores of galaxy clusters against
catastrophic cooling \cite{2009ApJ...703...96P,
  2009arXiv0911.5198R}. What is most promising though it that these
theoretical studies make specific observational predictions regarding
the magnetic field configurations in clusters as a function of AGN
activity and cluster dynamic state -- predictions which can be tested
using LOFAR and Simbol-X.

Furthermore, recent radio polarization measurements for galaxies in
the Virgo cluster suggest Virgo's ICM magnetic fields are radially
oriented \cite{2009arXiv0911.2476P}. This result is tantalizing since
radially oriented magnetic fields can result from the effects of the
MHD magnetothermal instability mechanism
\cite[MTI,][]{2000ApJ...534..420B}. The results for Virgo further
suggest that through the assistance of particular ICM magnetic field
configurations, conduction may play an important role in cluster
evolution. If large-scale radial magnetic fields are common in
clusters, then one can safely infer that MHD processes like MTI are
indeed a vital component of understanding galaxy cluster
evolution. While the results for Virgo provide only a single data
point, it is sufficiently interesting that follow-up using a larger
cluster sample should be undertaken. Such a study is possible using
the capabilities of LOFAR and Simbol-X.

LOFAR's order of magnitude improvement in angular resolution and
sensitivity at low radio frequencies opens a new era in studying ICM/IGM
magnetic fields via polarimetry
\cite{2009ASPC..407...33A}. Polarization measurements made with LOFAR
will enable direct detection of ICM/IGM field strengths and structure on
scales as small as cluster cores ($\la 50$ kpc, the scale where HBI
operates) and as large as cluster virial radii ($\sim$ few Mpc, the
scale where MTI functions). A systematic study of a cluster sample
using LOFAR will expand our view of magnetic field demographics and
how they relate to cluster properties like temperature gradients, core
entropy, merger activity which induce bulk motions, recent AGN
activity, and the structure of cold gas filaments in cluster cores. In
addition, we will be able to infer the possible origins and evolution
of ICM/IGM fields: were they seeded by early AGN activity? Are they
amplified and modified by mergers? Understanding cluster magnetic
fields will also place constraints on ICM/IGM properties, like viscosity,
which may govern the microphysics by which AGN feedback energy can be
dissipated as heat, \eg\ via turbulence and/or MHD waves.

{\bf{Relevant On-going Research}}

My on-going research has focused on the SMBH engines which underlie
AGN. A study which was recently completed \cite{pjet} investigates a
more precise calibration between AGN jet power (\pjet) and emergent
radio emission (\prad) for a sample of giant ellipticals (gEs) and
BCGs. In this study we estimated \pjet\ using cavities excavated in
the ICM as bolometers, and measured \prad\ at multiple frequencies
using new and archival VLA observations. We found, regardless of
observing frequency, that $\pjet \propto 10^{16} \prad^{0.7} \lum$,
which is in general agreement with models for confined heavy jets. The
utility of this relation lies in being able to estimate total jet
power from monochromatic all-sky radio surveys for large samples of
radio galaxies. Such a study should yield interesting constraints on
the kinetic heating of the Universe over vast swathes of cosmic
time. As a consequence, inferences can be drawn about AGN duty cycles,
the total accretion history of SMBHs, and the growth of SMBHs as a
function of redshift. A low-frequency all-sky survey from LOFAR should
provide an ideal catalog for conducting such a study.

An interesting result which has emerged from the \pjet-\prad\ work is
that FR-I radio galaxies (classified on morphology and not \prad)
appear to be systematically more radiatively efficient than FR-II
sources. This may mean there are intrinsic differences in radio
galaxies (\ie\ light vs. heavy relativisic jet compositions), or
possibly that all AGN jets are born light and become heavy on large
scales due to entrainment. One way to investigate this result more
deeply is to undertake a systematic study of the environments hosting
radio galaxies utilizing archival \chandra\ and VLA
data. Supplementary low-frequency data from LOFAR would be invaluable
for such a study as the low-frequency data provides important
constraints on the full extent of the energy in the radio lobes.

The \pjet-\prad\ work has also provided a means to establish tighter
observational constraints on the kinetic properties of AGN jets. With
this new leverage, of interest to me is re-visiting existing models
for relativistic jets in an ambient medium. Utilizing
observationally-based estimates of jet power, it is possible to better
understand the growth of a radio source including effects like
entrainment and evolution of jet composition \cite[\'a
  la][]{1999MNRAS.309.1017W}. Another interesting use of a universal
\pjet-\prad\ relation is using radio luminosities, lobe morphologies,
and age estimates to predict ambient gas pressures: $p_{\mathrm{amb}}
\propto (t_{\mathrm{age}}\prad) / V_{\mathrm{radio}}$. This yields an
estimate of ambient densities when basic assumptions are made about
environment temperatures: $\rho_{\mathrm{amb}} \propto p/T$. With an
estimate of ambient densities, X-ray observing plans for very
interesting radio sources which reside in faint group environments
(\ie\ FR-I sources) can be robustly prepared. An observationally-based
estimate of \pjet\ also enables the investigation of relations between
observable mass accretion surrogates (\ie\ \halpha\ luminosity,
molecular/dust mass, or nuclear X-ray luminosity) and AGN energetics
for the purpose of establishing clearer connections with accretion
mechanisms and efficiencies.

{\bf{Future Research}}

The study of AGN feedback and ICM/IGM thermal properties has advanced
quickly in the last decade primarily because the the current
generation of X-ray and radio observatories have provided access to
the datasets needed for detailed studies. However, our understanding
of non-thermal cluster emission and the origin of the emitting
particles has not progressed as quickly. Serendipitously, the quality
and availability of multi-frequency data (low-frequency radio, sub-mm,
IR, optical, UV, and hard X-ray) needed to probe non-thermal emission
is poised to improve with new facilities and instruments coming
on-line (\ie\ LOFAR, Herschel, SCUBA-2, SOFIA, ALMA, NuStar, Simbol-X,
LWA). As such, there are a number of research topics I am interested
in pursuing at NRL using LOFAR and Simbol-X.

{\bf{What is the origin of cluster-scale radio halos?}} Detection of
large-scale, diffuse radio halos in clusters emphasized the need to
further understand the non-thermal component of the ICM/IGM
\cite[\eg][]{2009ApJ...704L..54G, 2009A&A...507.1257G}. Though the
case connecting radio halos to mergers is increasingly convincing
\cite{2009A&A...507..661B}, the prevalence of radio halos in clusters
is not as high as expected given that all clusters are in some stage
of merger. Moreover, galaxy groups provide an additional constraint on
the properties of radio halos and their possible origins, yet no study
of these lower-mass analogs of clusters has been undertaken. Adding to
the mystery of radio halos is that the details regarding the processes
which generate the synchrotron emission are unknown. A number of
models have been proposed to explain the emission (\eg\ {\textit{in
    situ}} acceleration), but discerning between them observationally
has not been possible prior to LOFAR coming online. The systematic
study of a large sample of X-ray selected clusters with LOFAR
(\eg\ replicating the work of \cite{2007A&A...463..937V,
  2008A&A...484..327V}) will aide in addressing how radio halos form
and evolve.

{\bf{How does AGN activity depend on environment?}} Specifically what
is the relationship between redshift, environment, and feedback
energy? The answer thus far is unclear, most likely because the
influence of environment on AGN jets (through entrainment and
confinement) has been neglected or treated too simply in models. This
is where observations step in to place interesting constraints on the
problem. To this end, a study of the faint radio galaxy population
using archival \chandra\ and VLA data would be
interesting. Undertaking a systematic study of radio galaxy properties
(\ie\ jet composition, morphologies, outflow velocities, magnetic
field configurations) as a function of environment (\ie\ ambient
pressure, halo compactness) can help address how AGN energetics couple
to environment, which ultimately suggests how accretion onto the SMBH
couples to environment on small and large scales. Deep
\chandra\ observations for a sample of FR-I's (a poorly studied
population in the X-ray) would also be useful for such a study, using
the \pjet-\prad\ relation to define robust observation requests.

{\bf{How does the obscuration state of a SMBH correlate with radiative
    and mechanical AGN feedback and SMBH growth?}} As suggested by the
low AGN fraction in the \chandra\ Deep Fields, a significant
population of obscured AGN must exist at higher redshifts. One method
of selecting unbiased samples of these objects is to assemble catalogs
of candidate AGN using hard X-ray (\ie\ Simbol-X), far-IR
(\ie\ SOFIA), sub-mm (\ie\ SCUBA-2), and low-frequency radio
(\ie\ LOFAR) observations. Because current models suggest the luminous
quasar population begins in an obscured state, and rapid acquisition
of SMBH mass may occur in this phase because of high accretion rates
(possibly exceeding $10-100~L_{\mathrm{Edd.}}$), understanding the
transition from obscured to unobscured states is vital. How does
accretion proceed and where does the accreting material come from: gas
cooling out of the atmosphere? Gas stripped from merging companions?
Is accretion spherical and dictated by local gas density (\eg\ Bondi)?
A key component which has been neglected in AGN studies is the
contribution of dust (which should be a significant component in the
atmospheres of obscured AGN) in increasing the allowed Eddington
luminosity for an accreting SMBH (\ie\ $L_{\mathrm{Edd.}} \propto
\mu$). A curiosity which has emerged in recent years which may be
interesting, particularly during the obscured stage when the merger
rate is presumably high, is the role of multiple SMBHs within the core
of a host galaxy. At a minimum, SMBH mergers occur on a timescale
determined by dynamical friction, which for a typical dense bulge is
$\ga 1$ Gyr, which is $\gg t_{\mathrm{cool}}$ of an obscuring
atmosphere. If the SMBHs which are merging have their own accretion
disks, then it is reasonable to question how the atmospheres
surrounding a host galaxy with multiple AGN is affected, particularly
since the transition from obscured to unobscured should proceed more
quickly.

{\bf{Summary}}

The general picture of structure formation is much clearer now than a
decade ago, and the role of SMBHs and mergers in defining the thermal
and non-thermal emission from clusters and groups is undeniably
important. But, missing is a better understanding of cosmic magnetic
fields, AGN feedback properties, the feedback-environment connection,
diffuse cluster radio emission, modes of SMBH accretion, and how AGN
interact with/heat host atmospheres. To this end, more observational
constraints are needed, particularly using multiwavelength datasets
from upcoming missions. I am well-positioned to make meaningful
contributions in such pursuits, and would like to do so as a member of
the LOFAR consortium at NRL.

\scriptsize
\bibliographystyle{unsrt}
\bibliography{cavagnolo}
 
\end{document}
