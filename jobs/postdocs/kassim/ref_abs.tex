\documentclass[12pt]{article}
\pagestyle{empty}
\parindent 0pt
\parskip
\baselineskip
\setlength{\topmargin}{-0.30in}
\setlength{\oddsidemargin}{-0.30in}
\setlength{\evensidemargin}{-0.30in}
\setlength{\headheight}{0in}
\setlength{\headsep}{0.25in}
\setlength{\topskip}{0.25in}
\setlength{\textwidth}{6.9in}
\setlength{\textheight}{9.25in}
\pagestyle{myheadings}
\usepackage[T1]{fontenc}
\usepackage{subfig,epsfig,colortbl,graphics,graphicx,wrapfig,amssymb,common,mathptmx,multicol,natbib}
\begin{document}

In the last two decades, it has become apparent that active galactic
nuclei (AGN) play a vital role in the formation and evolution of
galaxies, galaxy groups, and galaxy clusters. In broad terms, the
galaxy formation paradigm which includes AGN feedback is successful in
explaining the bulk properties of large-scale structure, specifically
the thermal properties of the intracluster medium (ICM) in galaxy
clusters and the intragroup medium (IGM) in galaxy groups. However,
our understanding of the origins and influence of non-thermal ICM/IGM
components is incomplete, primarily because the observational
facilities needed for detailed studies of non-thermal emission have
been unavailable. But, with LOFAR being commissioned, and hard X-ray
observatories like Simbol-X and NuStar in-development, detailed
investigation of ICM/IGM magnetic fields \& MHD processes, diffuse
large-scale radio halos, and the cosmic ray content of the ICM/IGM can
be undertaken. Furthering our understanding of non-thermal components
will also reveal additional connections between host environments and
the AGN feedback loop. I am proposing to undertake studies of these
non-thermal components using the rich X-ray and radio datasets from
the current and future generation of facilities accessible through the
NRL.

\end{document}
