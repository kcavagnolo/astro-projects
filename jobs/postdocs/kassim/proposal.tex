\documentclass[12pt]{article}
\parindent 0pt
\parskip
\baselineskip
\setlength{\topmargin}{-0.30in}
\setlength{\oddsidemargin}{-0.30in}
\setlength{\evensidemargin}{-0.30in}
\setlength{\headheight}{0in}
\setlength{\headsep}{0.25in}
\setlength{\topskip}{0.25in}
\setlength{\textwidth}{6.9in}
\setlength{\textheight}{9.25in}
\pagestyle{empty}
\usepackage[T1]{fontenc}
\usepackage{subfig,epsfig,colortbl,graphics,graphicx,wrapfig,amssymb,common,mathptmx,multicol,natbib,setspace}
\begin{document}

\doublespacing
\begin{center}
  {\bf\uppercase{mapping galaxy cluster magnetic fields:\\an
      observational study of icm physics}}
\end{center}

\noindent{\bf{I. Motivation}}\\
\indent Clusters of galaxies are the largest structures in the
Universe to have reached dynamic equilibrium, and most cluster
baryonic mass resides in the intracluster medium (ICM), a hot, dilute,
weakly magnetized plasma filling a cluster's volume [1]. As the
defining characteristic of the most massive objects in the Universe,
the thermal properties of the ICM are well-known, but a similarly
detailed description of ICM non-thermal properties -- specifically
diffuse cluster magnetic fields (CMFs) -- and how they relate to the
thermodynamic nature of the ICM remains elusive. Filling this gap in
knowledge is vital because clusters help us constrain cosmological
parameters [2], develop hierarchical structure formation models [3],
and study the synergy of many physical processes to answer the
question, ``How does the Universe work?'' [4].

At present, one of the biggest challenges in cluster studies is
explaining the relative thermal equilibrium of the ICM. Many clusters
have core ICM cooling times much less than a Hubble time, and it was
hypothesized that these systems should host prodigious ``cooling
flows'' [5]. But, only minimal mass deposition rates and cooling
by-products have ever been detected, requiring that the ICM be heated
[6]. Observational and theoretical studies have strongly implicated
feedback from active galactic nuclei (AGN) in supplying the
{\it{energy}} needed to regulate ICM cooling and late-time galaxy
growth [7]. However, precisely how AGN feedback energy is thermalized
and which processes comprise a complete AGN feedback loop remain to be
fully understood [8].

Theoretical studies are now focused on coupling AGN feedback and ICM
heating using combinations of anisotropic thermal conduction, cosmic
ray diffusion, and subsonic turbulence [\eg\ 9--17] after observations
suggested the ICM is turbulent and conducting on small scales
[\eg\ 18--20]. These microphysical processes are intrinsically linked
to macroscopic CMF topologies through gas viscosity and
magnetohydrodynamic (MHD) instabilities [21, 22]. Thus, to
observationally test and refine this theoretical framework [\eg\ 23],
it is ideal to have uniform measurements of CMF strengths,
orientations, and spatial distributions for large cluster samples
spanning a broad range of evolutionary \& dynamical
states. Unfortunately, CMFs must be indirectly observed through steep
spectrum, non-thermal synchrotron emission best detected at low radio
frequencies [$\la 2$ GHz; 24]. A robust census of CMFs has been
lacking because of limitations in the sensitivity and resolution of
past radio measurements [\eg\ 25], limiting our knowledge of CMF
demographics to a few clusters [\eg\ 26]. Additionally, this shortfall
has inhibited the study of CMF origins and dynamical importance [27].

The greatly improved capabilities of the National Radio Astronomy
Observatory's Expanded Very Large Array (\evla) will significantly
impact this field by bringing powerful radio CMF survey and
polarization capabilities through unprecedented sensitivity,
resolution, and frequency coverage [28]. {\bf{As an NRL-NRC fellow, I
    propose to use radio polarimetry in conjunction with X-ray \&
    optical imaging to map CMFs (magnitudes, orientations, 3D
    structure) and evaluate their relationship with ICM thermal
    properties (\eg\ temperature, entropy, pressure) to constrain
    which physical mechanisms are responsible for the qualitative
    differences between observed and theoretical CMFs.}}  This work
will 1) determine which microphysical processes significantly
contribute to heating of the ICM by directly comparing the predictions
of theoretical models with CMF observations, and 2) place constraints
on the origin of CMFs and the cosmological implications of non-thermal
pressure support on cluster mass estimates. The proposed project
includes plans for an \evla\ radio survey and NOAO optical
\halpha\ survey of two well-studied cluster samples, and incorporates
an on-going pipeline analysis of an archive-limited sample of clusters
having X-ray data.\\

\noindent{\bf{II. Observations and Analysis}}\\
\indent The \evla\ Polarimetry Cluster Survey (EPiCS) will target the
flux-limited HIFLUGCS [29] and representative REXCESS [30] cluster
samples for which uniform \chandra\ and \xmm\ X-ray data is
available. EPiCS will utilize \evla's increased polarimetry bandwidth
(8 GHz per polarization) and frequency accessibility (74 MHz; 330 MHz;
full coverage 1--2 GHz) to obtain deep ($\sigma_{\rm{rms}} < 10
~\mu$Jy beam$^{-1}$) full Stokes observations of each cluster. The
improved \evla\ efficiency and dynamic range mean extended sources as
faint as $\sim$2-3 $\mu$Jy will be detected with integration times
$\la 5$ hrs, well into the regime where $\mu$G CMFs excite
emission. One of \evla's two low-frequency ($< 0.5$ GHz) systems is
now functioning, and EPiCS will be cross-calibrated with data from
\lofar\ to expand the utility of the dataset. EPiCS observations are
designed to enable measurements of: 1) CMF strengths using Faraday
rotation measures (RM) of previously undetected embedded \& background
cluster radio sources [see Fig. 1 and 31 for method], 2) large-scale
CMF orientations using coherent polarized emission of orbiting cluster
member galaxies [see Fig. 2 and 32 for method], and 3) CMF spatial
distributions \& ordering using low-surface brightness emission of
radio halos [see 33 for method]. Combined with the archival X-ray data
for each source, the following outstanding issues will be
investigated.

{\bf{A. Testing Models of ICM Heating:}} The EPiCS campaign will
produce data of sufficient quality to measure RM dispersions, estimate
CMF radial amplitude profiles, directly reconstruct CMF power spectra,
and model 3D CMF structure using RM synthesis [see methods in
  34--36]. Each of these CMF diagnostics will be directly compared
with results from MHD models in the literature (see Fig. 3 for
example) to determine which predictions are replicated
(\eg\ preferentially radial CMFs, CMF profile shapes, CMF
magnitude--ICM \nelec\ \& \tx\ correlations), which predictions
indicate the input physics may be incomplete, and to help constrain
which microphysical processes might participate in ICM heating. Since
AGN feedback is the likely progenitor of heating, an investigation of
possible correlations between CMF properties and feedback signatures
(\eg\ cluster core entropy, jet powers for systems with cavities, 2D
thermal distributions, extent of central AGN activity) will be
undertaken. Further, turbulence is considered vital for promoting ICM
heating, but is difficult to directly measure. However, secondary
diagnostics (\eg\ AGN outflows, mergers, cold fronts, shocks) may
indicate the presence of turbulence even when the data is insufficient
to do so. These indicators will be considered during the analysis to
check if trends exist with CMF properties.

{\bf{B. CMFs in Cluster Cores:}} It is hypothesized that the
\halpha\ filaments seen in almost all cool core clusters provide a
local measure of CMF strength and orientation since they may form
along field lines and be excited by some combination of turbulent
mixing and conduction [37, 38]. To probe CMF configurations and
conductive heating on kpc scales, below the reach of the radio
observations, a uniform optical survey for extended \halpha\ filaments
in the EPiCS cluster samples will be undertaken using new NOAO
instruments (\ie\ Magellan Maryland Tunable Filter, WIYN HiRes IR
Camera, SOAR Spartan IR Camera) [see 39, for method]. The observations
will allow, for the first time, a complete characterization of
filament morphologies and energetics to be compared with uniform ICM
and CMF properties for the same objects. These observations will
confront model predictions by answering the question, ``Are filament
energetics and morphologies consistent with them being magnetic
structures conductively heated by the ICM?''  Combined with the
radio-derived CMF properties, inferences will also be drawn about if,
and possibly how, large- and small-scale CMF properties are related
(\eg\ the coherence length). The model comparisons from Section A will
also answer the questions: do filaments thrive in low-turbulence,
high-magnetic field strength environs? Does this imply MHD
instabilities are suppressed or inactive in some cluster cores?

{\bf{C. Constraining CMF Origins and Non-thermal Pressure Support:}}
Simply put, where do CMFs come from (\eg\ amplified primordial cosmic
field?  the Biermann battery process?  AGN/galactic outflow seeding of
protoclusters?), and are they dynamically important [40]? The EPiCS
project will help address these questions. As the quantities most
closely related to dynamo-driven CMF formation, I will investigate how
redshift, halo concentration, and cluster mass relate to the derived
CMF power spectra and radial profiles [41]. At a minimum, these
comparisons will place limits on the strength and distribution of
allowable seed field models, and may possibly suggest whether early-
or late-time amplification processes dominate [42]. Deriving halo
concentrations and cluster masses follow directly from the X-ray
analysis already in-hand. However, cluster masses are traditionally
derived by assuming the ICM is in hydrostatic equilibrium. If CMFs
provide significant ICM pressure support, then cluster masses may be
systematically overestimated, having interesting repercussions on
cluster cosmological studies. Thus, cluster masses and the cluster
mass function will be recalculated [\eg\ 43] including terms for CMF
pressure support determined from the EPiCS measurements. How
cosmological parameter uncertainties depend on CMFs will then be
determined. This exercise will be particularly interesting for the
REXCESS sample which has high-quality hydrostatic mass estimates [44].

{\bf{D. Archival Project and Legacy:}} Work has started on archival
\chandra\ and \vla\ data to build the infrastructure needed to
maximize the ultimate scientific impact of this project and produce
initial results for an archive-limited sample of clusters. There are
$\approx 450$ clusters which have archival \chandra\ ($\approx 900$
observations) and \vla\ ($\approx 1000$ observations) data. Of these,
325 clusters have had the X-ray data reduced using an extensible and
mature pipeline, while 50 of those clusters have had the
multifrequency radio data reduced. The X-ray results are being kept in
a public database\footnote{http://www.pa.msu.edu/astro/MC2/accept/}
while the radio analysis continues. The on-going analysis entails
production of 2D ICM temperature, density, pressure, \& entropy maps,
more radial profiles (\eg\ effective conductivity, implied suppression
factors), and refinement of the radio reduction pipeline. Mitigation
of radio frequency interference (RFI) is a lengthy and tedious step in
radio analysis. To alleviate this tension, a portable \python\ version
of the `RfiX' rejection algorithm [45] has been written and is being
tested. To widen this proposal's scientific impact and relevance to
future radio observatories (\eg\ \lofar, \lwa, \ska), all code,
software, and results will be made freely available to the research
community. The results from this project will also serve as science
drivers for future \evla\ upgrades, \ie\ Low Frequency System (full
50--1000 MHz) \& ultracompact E-array [46], being proposed
specifically to study cosmic magnetism.\\

\noindent{\bf{III. Host Institution and Timeline}}\\
\indent The University of Maryland (UMD) is an ideal host for this
ambitious project as it boasts a team of experts well-suited to assist
with this work. The UMD Astronomy Department, the Center for Research
and Exploration in Space Science and Technology, and the Center for
Theory and Computation are hosts to (to name but a few) Keith Arnaud,
Tamara Bogdanovi{\'c} (current Einstein fellow), Michael Loewenstein,
Craig Markwardt, Cole Miller, Richard Mushotzky, Eve Ostriker, Chris
Reynolds (the sponsor), Massimo Ricotti, and Sylvain Veilleux. All
those listed are experts in one, or several, of the areas of AGN
feedback, ICM physics, computational modeling, magnetic field
polarimetry, plasma physics, and X-ray/radio observing \& analysis. In
addition, UMD has close ties with the Naval Research Lab where Tracy
Clarke and Namir Kassim are appointed -- both experts in the radio
techniques and science topics discussed here. {\bf{Year one:}} Data
acquisition begins, the archival project and tool development
continue; I initiate a theoretical/simulation collaboration with the
UMD plasma physics group to study new questions like: How do
convective instabilities couple with ICM cooling and the actual
accretion which drives AGN activity?  What is the relation between
these processes, ICM temperature \& density, and thermal instability
formation? {\bf{Year two:}} Data acquisition continues, first round of
archival-based results is published, and observation-model comparisons
begin. {\bf{Year three:}} Data acquisition and analysis of REXCESS
conclude, second round of results published, and the investigation of
CMF origins and non-thermal pressure support is underway.


%%%%%%%
% Bib %
%%%%%%%

\noindent{\bf{IV. References}}\\
\input{short.bbl}

%%%%%%%%%%%
% Figures %
%%%%%%%%%%%

\begin{figure}
  \begin{center}
    \begin{minipage}{\linewidth}
      \includegraphics*[width=\linewidth, trim=0mm 0mm 0mm 0mm, clip]{bonafede.ps}
    \end{minipage}
    \caption{Radial profile and power spectrum of the Coma CMF derived
      from 3D simulations which reproduce observed RMs of embedded and
      background radio sources (taken from Bonafede \etal\ 2010
      [29]). If one assumes the CMF and ICM thermal radial
      distributions trace each, then comparison of observed RM
      dispersions and 3D simulations lead to constraints like this on
      the CMF profile without the need to make measurements at every
      radius. One goal of this proposal is to expand upon the Bonafede
      \etal\ result using a larger cluster sample and EVLA
      observations.}
  \end{center}
\end{figure}

\begin{figure}
  \begin{center}
    \begin{minipage}{\linewidth}
      \includegraphics*[width=\linewidth, trim=0mm 0mm 0mm 0mm, clip]{pfrommer.ps}
    \end{minipage}
    \caption{Virgo CMF orientations (yellow arrows) taken from
      Pfrommer \& Dursi 2010 [30] where they argue draping of CMF
      lines at the ICM-infalling galaxy interface explains the CPE
      (left panel at 5 GHz). The CPE results from galactic cosmic rays
      gyrating around regularly compressed field lines. The authors
      argue the Virgo CMF is preferentially radial, consistent with
      the effects of a large-scale MHD convective instability
      [\ie\ the MTI; 21]. Similar measurements are a key feature of
      the EPiCS project and will help us constrain CMF orientations as
      never before.}
  \end{center}
\end{figure}

\begin{figure}
  \begin{center}
    \begin{minipage}{\linewidth}
      \includegraphics*[width=\linewidth, trim=10mm 4mm 2mm 5mm, clip]{kunz.eps}
    \end{minipage}
    \caption{Predicted CMF strength of Abell 1835 from model of Kunz
      \etal\ 2010 [16] which heats the ICM {\it{only}} via viscous
      dissipation of turbulent ICM motions. The Kunz \etal\ model
      takes in X-ray derived ICM density and temperature measurements
      alone and returns estimates of field strengths. These profiles
      have already been derived for over 300 clusters using the
      \chandra\ archival project, and will be compared with the
      results of the EPiCS program (\eg\ radial profiles and power
      spectra, like Fig. 1) to test how important turbulent heating is
      for an array of cluster types. This is one example of how the
      proposed CMF measurements will be directly compared with model
      predictions to aide theorists in refining models for ICM
      heating.}
  \end{center}
\end{figure}

\end{document}
