\documentclass[12pt]{cv}
\usepackage[colorlinks=true,linkcolor=blue,urlcolor=blue]{hyperref}
\usepackage[T1]{fontenc}
\usepackage{common,subfig,epsfig,colortbl,graphics,graphicx,wrapfig,amssymb,common,mathptmx,multicol,natbib}
\pagestyle{empty}
\parindent 0pt
\parskip
\baselineskip
\setlength{\topmargin}{-0.30in}
\setlength{\oddsidemargin}{-0.30in}
\setlength{\evensidemargin}{-0.30in}
\setlength{\headheight}{0in}
\setlength{\headsep}{0.25in}
\setlength{\topskip}{0.25in}
\setlength{\textwidth}{6.9in}
\setlength{\textheight}{9.25in}
\pagestyle{myheadings}
\begin{document}

{\bf{NB:}} Michigan State University is denoted as MSU below, while
the University of Waterloo is denoted as UW.

{\bfseries{ICM Temperature Inhomogeneity, MSU, 2002-2008, M. Donahue}}
\markright{Kenneth W. Cavagnolo, Summary}

To more accurately weigh galaxy clusters, how secondary dynamical
processes (\eg\ mergers and AGN feedback) alter cluster observables
must first be quantified if cluster temperature or luminosity are to
serve as accurate mass proxies. It has been demonstrated that spatial
cluster substructure correlates well with dynamical state, and that
the most relaxed clusters have the smallest deviations from mean
mass-observable relations (\eg\ \cite{VV08}). But spatial analysis
is at the mercy of perspective. If equally robust aspect-independent
measures of dynamical state could be found, then quantifying deviation
from mean mass-scaling relations would be improved and the uncertainty
of inferred cluster masses could be further reduced. The Cavagnolo
dissertation confronts this difficulty via temperature inhomogeneity.

If the hot ICM is nearly isothermal in the projected region of
interest, the X-ray temperature inferred from a broadband (0.7-7.0
keV) spectrum should be identical to the X-ray temperature inferred
from a hard-band (2.0-7.0 keV) spectrum. However, if unresolved cool
lumps of gas are contributing soft X-ray emission, the temperature of
a best-fit single-component thermal model will be cooler for the
broadband spectrum than for the hard-band spectrum. Using this
difference as a diagnostic, the ratio of best-fitting hard-band and
broadband temperatures may indicate the presence of cooler gas even
when the X-ray spectrum itself may not have sufficient signal-to-noise
ratio to resolve multiple temperature components \cite{me01}.

Building on the \cite{me01} simulation results, the dissertation
investigates the band dependence of the inferred X-ray temperature of
the ICM for 192 well-observed galaxy clusters selected from the
\chandra\ Data Archive. X-ray spectra from core-excised annular
regions of fixed fractions of the virial radius, $R_{2500}$ and
$R_{5000}$, are extracted for each cluster in the archival sample. A
comparison is made of the X-ray temperatures inferred from
single-temperature fits when the energy range of the fit is 0.7-7.0
keV (broad) and when the energy range is 2.0/(1+$z$)-7.0 keV
(hard). On average, the hard-band temperature is found to be
significantly higher than the broadband temperature, and the ratio of
the temperatures is quantified as $T_{HBR} = T_{2.0-7.0}/T_{0.7-7.0}$,
shown in Figure \ref{fig:thbr}. On further exploration, it is found
that the temperature ratio $T_{HBR}$ is enhanced preferentially for
clusters which are known merging systems. In addition, cool-core
clusters tend to have best-fit hard-band temperatures that are in
closer agreement with their best-fit broadband temperatures, shown
using symbols in Figure \ref{fig:thbr}. Presuming cool cores and
mergers are good indicators of dynamical state, the dissertation
concludes that $T_{HBR}$ is a useful metric for further assessing the
process of cluster relaxation. The work associated with this part of
the dissertation is published in \cite{xrayband}.

{\bfseries{ICM Entropy Profiles, MSU, 2002-2008, M. Donahue \& M. Voit}}
\markright{Kenneth W. Cavagnolo, Summary}

ICM temperature and density alone primarily reflect the shape and
depth of the cluster dark matter potential, but it is the specific
entropy of a gas parcel which governs the density at a given pressure
\cite{voitbryan}. In addition, the ICM is convectively stable when,
without dramatic perturbation, the lowest entropy gas is near the core
and high entropy gas has buoyantly risen to large radii. ICM entropy
can also only be changed by addition or subtraction of heat, thus the
entropy of the ICM reflects most of the cluster thermal
history. Therefore, properties of the ICM can be viewed as a
manifestation of the dark matter potential and cluster thermal history
- which is encoded in the entropy structure
(\eg\ \cite{voitbryan}). ICM Entropy is therefore a useful quantity
for studying the effects of feedback on the cluster environment and
investigating the breakdown of cluster self-similarity.

The dissertation studies feedback using radial entropy profiles of the
ICM for a collection of 239 clusters taken from the \chandra\ Data
Archive, presented in Figure \ref{fig:splots}. It is found that most
ICM entropy profiles are well-fit by a model which is a power-law at
large radii and approaches a constant entropy value at small radii:
$K(r) = \kna + \khun (r/100 ~\kpc)^{\alpha}$, where \kna\ quantifies
the typical excess of core entropy above the best fitting power-law
found at larger radii and \khun\ is the entropy normalization at 100
kpc. Discussion is presented in relation to theoretical models
(\eg\ \cite{agnframework}) explaining why non-zero \kna\ values are
consistent with the process of energy injection from AGN
feedback. Further, it is shown that the \kna\ distributions of both
the full archival sample and the flux-limited, unbiased primary
{\it{HIFLUGCS}} sample of \cite{hiflugcs1,hiflugcs2} are bimodal with
a distinct gap centered at $\kna \approx 40 ~\ent$ and population
peaks at $\kna \sim 15 ~\ent$ and $\kna \sim 150 ~\ent$ (Figure
\ref{fig:k0hist}). It is suggested that the bimodal distribution may
result from the effects of ICM thermal conduction and cluster-cluster
mergers. The results from this work are presented in \cite{accept}. We
are also making comparisons between entropy scaled as a function of
cluster mass, temperature, and luminosity with expectations from
large-scale structure models. The results of this additional work is
being presented in \cite{entscale}.

{\bfseries{ICM Entropy-Feedback Correlations, MSU, 2002-2008, M. Donahue \& M. Voit}}
\markright{Kenneth W. Cavagnolo, Summary}

Also of interest is how cluster core entropy state is associated with
AGN feedback and star formation in the galaxy which resides at the
center of a cluster. As an extension of the radial entropy analysis,
the dissertation delves into exploring the relationship between some
expected by-products of ICM cooling -- \eg\ gaseous instabilities,
star formation, and AGN activity -- and the \kna\ values of
clusters. To determine the activity level of feedback in cluster
cores, the readily available observables \halpha\ and radio emission
are selected as tracers.

Utilizing the results of the archival study of intracluster entropy,
the dissertation goes on to show that \halpha\ and radio emission from
central cluster galaxies are much more pronounced when the cluster's
core gas entropy is $\la 30 ~\ent$. The prevalence of \halpha\ emission
below this threshold indicates that it marks a dichotomy between
clusters that can harbor multiphase gas and star formation in their
cores and those that cannot. The fact that strong central radio
emission also appears below this boundary suggests that feedback from
an AGN turns on when the ICM starts to condense, strengthening the
case for AGN feedback as the mechanism that limits star formation in
the Universe's most luminous galaxies. The results of this work are
presented in \cite{haradent}. The dissertation results also suggest
that the sharp entropy threshold for the formation of thermal
instabilities in the ICM and initiation of processes such as star
formation and AGN activity arises from thermal conduction. A
discussion of this topic is presented in \cite{conduction}.

{\bfseries{RBS 797, UW, 2008-present, B. McNamara \& M. Gitti}}
\markright{Kenneth W. Cavagnolo, Summary}

The most powerful AGN outbursts in the Universe are useful for placing
constraints on possible fueling mechanisms for the AGN. Systems such
as MS 0735.6+7421, Hercules A, and Hydra A stress the limits of cold
gas accretion models (such as Bondi accretion), and open the door to
new mechanisms such as black hole spin \cite{bhspin}. The galaxy
cluster RBS 797 is another system which has undergone a cluster-scale
AGN outburst. R797 has a pair of X-ray cavities which suggest the AGN
outburst in the system is of order $\sim 10^{45-46}$ erg s$^{-1}$,
making it one of the most powerful outbursts ever observed. I have
undertaken the detailed analysis of this peculiar system using X-ray,
radio, infrared, optical, and UV data. The results of this work are
being published in a first author paper \cite{r797}.

{\bfseries{AGN Jet Power-Radio Luminosity Relation, UW, 2008-2010, B. McNamara \& C. Carilli}}
\markright{Kenneth W. Cavagnolo, Summary}

A long-standing problem in observational and theoretical studies of
energetic feedback from supermassive black holes as it relates to
large-scale structure formation is estimating the total kinetic output
from an active galactic nucleus. These estimates have historically
been made using models of AGN jets and their impact on the surrounding
environment. However, the ICM has proven to be a robust bolometer for
measuring jet power courtesy of X-ray cavities. Using a sample of
clusters, groups, and isolated giant ellipticals with cavities I have
completed a project which measured and calibrated jet power versus
radio luminosity. This was an extension of the oft-cited
\cite{birzan04, birzan08} work. For the project, I observed 13 gEs (39
hrs. total) at P-band (327 MHz; 90 cm) using the new EVLA system, and
analyzed > 50 archival observations for 21 additional objects at a
variety of frequencies (1.4 GHz, 5 GHz, and 8 GHz). We found that jet
power scales with radio luminosity to the 0.7 power with a
normalization of $\sim 10^{43}$ erg s$^{-1}$, in accord with current
jet models. Our results have implications for galaxy formation, black
hole growth, and the mechanical heating of the universe. The results
are being published in a first author paper \cite{pjet}. We are using
the results of this study to explore the AGN kinetic luminosity
function over cosmic time, and to analyze a subset of peculiar FR-I
radio galaxies in more detail.

{\bfseries{IRAS 09104+4109, UW, 2009-present, M. Donahue \& B. McNamara}}
\markright{Kenneth W. Cavagnolo, Summary}

The transition from quasar-mode to radio-mode feedback in hierarchical
structure formation is a poorly understood process. The transition
likely coincides with the formation of dense galactic environments
like clusters and groups, in addition to the formation of the most
massive galaxies which will become present-day BCGs. But we know this
process does not proceed unregulated, lest extremely blue cDs residing
in catastrophically cooling cluster cores will form. As a probe of how
the BCG assembly and ICM heating process proceeds, we obtained deep
\chandra\ imaging of the famous and peculiar ULIRG/QSO IRAS
09104+4109. As suspected, we directly imaged a pair of cavities in the
X-ray halo surrounding IRAS09. These cavities contain enough energy to
offset $\approx 25-35\%$ of the cooling occurring within the cooling
radius of the host galaxy cluster. This result suggests only 3-4 such
outbursts are needed to halt cooling in the cluster and freeze-out
rapid star formation in the BCG. The line-of-sight nuclear absorber in
this system has also (as of our current analysis) changed from
optically thick to thin over the course of the last 20 years, making
this the first-ever observed changing-look QSO. Even more exciting is
the change in beaming direction of the AGN within the last few kyrs
which has dredged up cool gas from the core, and is likely forming new
stars as a result. This work has produced a first author paper
\cite{iras09}.

{\bfseries{Steep Spectrum Radio Sources, UW, 2010-present, A. Edge}}
\markright{Kenneth W. Cavagnolo, Summary}

The cores of galaxy clusters are active environments where
thermodynamic balance is struck via AGN feedback energy. Many ongoing
and previous studies have focused on systems hosting AGN feedback
where radio emission and X-ray cavities directly indicate ``active''
feedback. Along with Alastair Edge at Durham University, we have
acquired 50+ hours of 325 MHz radio data for 14 galaxy clusters where
unresolved steep radio spectrum sources have been identified. The
expectation is that these sources host radio relics from past AGN
activity, or the sources result from a subcluster merger has created a
radio halo. This study is a key forerunner to the analysis of large
samples of similar sources found with LOFAR.

{\bfseries{X-ray Mass Estimates, UW, 2009-present, R. Mandelbaum}}
\markright{Kenneth W. Cavagnolo, Summary}

I am a member of the Supermassive Cluster Survey and am responsible
for the X-ray mass analysis in the project. The study is headed-up by
Rachel Mandelbaum and seeks to better understand the scatter between
X-ray and weak lensing masses for a sample of 12 galaxy clusters. The
project is in its final stages with the mass determinations from the
X-ray data currently being performed. The project will produce at
least one paper on which I am one of the primary co-authors.

{\bfseries{2D Abundance Distributions, UW, 2008-present, C. Kirkpatrick \& B. McNamara}}
\markright{Kenneth W. Cavagnolo, Summary}

Clif Kirkpatrick is a senior Ph.D. student under Dr. McNamara. I have
co-authored two papers with Clif \cite{a1664, hydrametal}, and we are
working on a series of new papers which present analysis of the 1D and
2D heavy metal distributions for a large sample of galaxy clusters. We
are specifically interested in how metal transport is related to the
process of AGN feedback, and what we can discern about ICM metal
enrichment over cosmic time using these results.

{\bfseries{Black Hole Spin, UW, 2009-present, B. McNamara \& M. Rohanizadegan}}
\markright{Kenneth W. Cavagnolo, Summary}

Mina Rohanizadegan is a junior Ph.D. student under Dr. McNamara. We
are currently working on the analysis of X-ray data to learn about the
instantaneous accretion onto SMBHs at the center of galaxy
clusters. The aim is to place constraints on the fueling mechanism
which gives rise to the AGN jets which bore cavities into the
ICM. Mina is also finishing up a co-authored paper which presents
comparisons of models for AGN power generation via cold gas accretion
and black spin using the robust jet power measures from X-ray
cavities. As a supplement to this work, I am writing a paper which
discusses the complications of reorienting the spin axis of a SMBH via
mergers. Spin axis reorientation has become somewhat of a ``fad'' in
the last decade to explain the morphology of some radio sources and as
an explanation for the distribution of AGN feedback energy beyond the
small cross-section of AGN jets. However, spin axis reorientation is
exceedingly difficult and requires a very specific set of impact
parameters which, as discerned from cosmological simulations, are
found to be very rare.

{\bfseries{Search for Giant Galactic Cores and Radio-Xray Cross-correlations, UW, 2009-present, B. Whuiska, R. Myers, \& B. McNamara}}
\markright{Kenneth W. Cavagnolo, Summary}

Brad Whuiska and Rob Myers are senior undergraduates working with
Dr. McNamara. Brad is measuring the core radius for BCGs in the HST
archive. The aim of his study is to find the largest cores and analyze
them under the assumption that the large cores were created via
scouring (the process of stellar ejection via SMBH mergers). The work
is producing results which will be presented in a paper on which I
will be a co-author. Rob is undertaking the detection of BCG radio
sources in X-ray selected clusters using the NVSS and SUMSS all-sky
radio surveys. These sources will then be run through our
$P_{jet}-P_{radio}$ relations, and an estimate of the heating
resulting from these AGN will be assessed. Rob is also examining the
connection with cluster properties such as X-ray luminosity and
temperature. Rob's work is also producing results which will be
presented in a co-author paper.

\bibliographystyle{unsrt}
\bibliography{cavagnolo}

\clearpage
\markright{K.W. Cavagnolo, Summary}
\begin{figure}[t]
    \begin{minipage}[t]{0.5\linewidth}
        \centering
        \includegraphics*[width=\textwidth, trim=17mm 3mm 10mm 11mm, clip]{thbr.eps}
        \caption{\footnotesize $T_{HBR}$ vs. $T_{0.7-7.0}$. The dashed
          line is the line of equivalence. Symbols and color coding
          are based on two criteria: 1) presence of a cool core (CC)
          and 2) value of $T_{HBR}$. Black stars are clusters with a
          CC and $T_{HBR}$ significantly greater than 1.1. Green
          upright-triangles are NCC clusters with $T_{HBR}$
          significantly greater than 1.1. Blue down-facing triangles
          are CC clusters and red squares are NCC clusters. It is
          found that most, if not all, of the clusters with $T_{HBR}
          \gtrsim 1.1$ are merger systems.}
        \label{fig:thbr}
    \end{minipage}
    \hspace{0.1in}
    \begin{minipage}[t]{0.5\linewidth}
        \centering
        \includegraphics*[width=\textwidth, trim=28mm 7mm 30mm 17mm, clip]{splots_allt.eps}
        \caption{\footnotesize Composite plot of entropy profiles for
          archival sample. Profiles are color-coded based on average
          cluster temperature; units of the color bar are keV. The
          solid line is the pure-cooling model of \cite{voitbryan},
          the dashed line is the mean profile for clusters with $\kna
          \le 50 ~\ent$, and the dashed-dotted line is the mean profile
          for clusters with $\kna > 50 ~\ent$.}
        \label{fig:splots}
    \end{minipage}
    \hspace{0.1in}
    \begin{minipage}[t]{0.5\linewidth}
        \centering
        \includegraphics*[width=\textwidth, trim=32mm 8mm 32mm 18mm, clip]{k0hist.eps}
        \caption{\footnotesize Histogram of best-fit \kna\ for all the
          clusters in the archival study. Bin widths are 0.15 in log
          space. The distinct bimodality in \kna is bracketed by the
          vertical dashed lines.}
        \label{fig:k0hist}
    \end{minipage}
    \hspace{0.1cm}
    \begin{minipage}[t]{0.5\linewidth}
        \centering
        \includegraphics*[width=\columnwidth, trim=28mm 7mm 40mm 17mm, clip]{ha.eps}
        \caption{\footnotesize Central entropy
          vs. \halpha\ luminosity. Orange circles represent
          \halpha\ detections, black circles are non-detection upper
          limits, and blue squares with inset red stars or orange
          circles are peculiar clusters which do not adhere to the
          observed trend. The vertical dashed line marks $\kna = 30
          ~\ent$. Note the presence of a sharp \halpha\ detection
          dichotomy beginning at $\kna \la 30 ~\ent$.}
        \label{fig:ha}
    \end{minipage}
\end{figure}
 
\end{document}
