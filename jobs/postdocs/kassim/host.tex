\indent The Naval Research Laboratory (NRL) is an ideal host for this
ambitious project as it boasts a team of experts well-suited to assist
with this work. The advisor, Dr. Namir Kassim, and his close
collaborator, Dr. Tracy Clarke, are specialists in the areas of
radio/polarization studies of clusters and AGN feedback, and their
experience with the latest techniques in numerical simulations, radio
interferometry, and high-energy physics will be invaluable. In
addition, NRL has close ties with the nearby University of Maryland
(UMD) where additional experts in the proposed fields of study are
found. The UMD Astronomy Department, the Center for Research and
Exploration in Space Science and Technology, and the Center for Theory
and Computation are hosts to (to name but a few) Keith Arnaud, Tamara
Bogdanovi{\'c}, Michael Loewenstein, Craig Markwardt, Cole Miller,
Richard Mushotzky, Eve Ostriker, Chris Reynolds, Massimo Ricotti, and
Sylvain Veilleux. All those listed are experts in one, or several, of
the areas of AGN feedback, ICM physics, computational modeling,
magnetic field polarimetry, plasma physics, and X-ray/radio
observing \& analysis. {\bf{Year one:}} Data acquisition begins, the
archival project and tool development continue; I initiate a
theoretical/simulation collaboration with the UMD plasma physics group
to study new questions like: How do convective instabilities couple
with ICM cooling and the actual accretion which drives AGN activity?
What is the relation between these processes, ICM temperature \&
density, and thermal instability formation? {\bf{Year two:}} Data
acquisition continues, first round of archival-based results is
published, and observation-model comparisons begin. {\bf{Year three:}}
Data acquisition and analysis of REXCESS conclude, second round of
results published, and the investigation of CMF origins and
non-thermal pressure support is underway.
