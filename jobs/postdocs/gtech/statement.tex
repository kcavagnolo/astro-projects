%% more spacious
%\documentclass[12pt]{article}
%\pagestyle{empty}
%\parindent 0pt
%\parskip
%\baselineskip
%\setlength{\topmargin}{-0.30in}
%\setlength{\oddsidemargin}{-0.30in}
%\setlength{\evensidemargin}{-0.30in}
%\setlength{\headheight}{0in}
%\setlength{\headsep}{0.25in}
%\setlength{\topskip}{0.25in}
%\setlength{\textwidth}{6.9in}
%\setlength{\textheight}{9.25in}
%\pagestyle{myheadings}
%\markright{K.W.C., Statement of Interest}

% ubercompact
\documentclass[11pt]{article}
\setlength{\topmargin}{-0.15in}
\setlength{\oddsidemargin}{-0.12in}
\setlength{\evensidemargin}{0in}
\setlength{\headheight}{0in}
\setlength{\headsep}{0.0in}
\setlength{\topskip}{0.0in}
\setlength{\textwidth}{6.9in}
\setlength{\textheight}{9.5in}

% declare packages and options
\usepackage[T1]{fontenc}
\usepackage{subfig,epsfig,colortbl,graphics,graphicx,wrapfig,amssymb,common,mathptmx,multicol,natbib}

% start the document
\begin{document}

% header
\begin{center}
%{\large
\textbf{Statement of Research Interests}
%\rule{17cm}{2pt}
\end{center}
%\normalsize

The gravitational energy liberated by active galactic nuclei (AGN),
\ie\ accreting supermassive black holes (SMBHs), plays a vital role in
regulating the process of hierarchical structure formation
\cite[\eg][]{perseus1, croton06, bower06, saro06, sijacki07,
birzan08}. Current cosmological models invoke a feedback loop where
the processes of environmental cooling and heating are coupled via AGN
\cite{2002MNRAS.333..145N, mcnamrev}. In broad terms, AGN feedback has
been segregated into two modes which occur at different cosmic epochs:
an early-time radiatively-dominated mode, and a late-time
mechanically-dominated mode. While this model is successful in
reproducing the bulk properties of the Universe, the details of AGN
feedback are poorly understood. It is these details which interest me
most.

My past research has focused on understanding the mechanical feedback
from AGN and the associated effects on galaxy clusters. I have devoted
particular attention to intracluster medium (ICM) entropy distribution
\cite{accept}, the process of cluster virialization
\cite{xrayband}, the mechanisms by which SMBHs might acquire fuel from
their environments \cite{conduction}, and how those mechanisms
correlate with properties of clusters cores \cite{haradent}.

These studies have revealed that certain conditions must be
established within a cluster core, namely that the mean entropy of the
large-scale environment hosting a SMBH must be $\la 30~
\ent$. Coincidentally, this is the entropy scale above which thermal
electron conduction is capable of stabilizing a cluster core against
the formation of thermal instabilities, hinting at a mechanism for
coupling AGN feedback energy to the ICM and establishing a
self-regulating feedback loop. This result is made more interesting if
the heat-flux-driven-buoyancy instability
\cite[HBI,][]{2008ApJ...677L...9P} is an important process in clusters
with central cooling times $\ll \Hn^{-1}$. Full MHD simulations have
shown that the HBI, in conjunction with reasonable magnetic field
strengths, modest heating from an AGN, and subsonic turbulence can
feasibly stabilize a core against catastrophic cooling
\cite{2009ApJ...703...96P, 2009arXiv0911.5198R}. In addition, recent
radio polarization measurements for Virgo cluster galaxies suggest the
large-scale magnetic field of Virgo's ICM is radial oriented
\cite{2009arXiv0911.2476P}. This result is tantalizing since it
suggests the magnetothermal instability \cite{2000ApJ...534..420B} may
be operating within Virgo, furthering the case that conduction is a
vital component of understanding galaxy cluster evolution under the
influence of AGN. In total, these studies touch on the subject of
magnetic fields in clusters, which is of great interest to me.

The Low Frequency Array (LOFAR) radio observatory began collecting
data in fall 2009, and has opened a new era in studying ICM magnetic
fields via polarimetry \cite{2009ASPC..407...33A}. Polarization
measurements made with LOFAR will enable direct detection of ICM field
strengths and structure on scales as small as cluster cores and as
large as cluster virial radii. A systematic study of a representative
cluster sample (such as REXCESS \cite{rexcess}) using LOFAR will
expand our view of magnetic field demographics and how they relate to
cluster properties like temperature gradients, core entropy, recent
AGN activity, and the structure of cold gas filaments in cluster
cores. In addition, we will be able to investigate the origin and
evolution of the fields: were they seeded by early AGN activity? Are
they amplified by mergers? Is there evidence of draping or
entrainment? Understanding cluster magnetic fields will also place
constraints on ICM properties, like viscosity, which govern the
microphysics by which AGN feedback energy might be dissipated as heat,
\eg\ via turbulence and/or MHD waves.

A study I have recently completed \cite{pjet} investigates a more
precise calibration between AGN jet power (\pjet) and emergent radio
emission (\lrad) for a sample of giant ellipticals (gEs) and BCGs. We
found, regardless of observing frequency, that $\pjet \propto 10^{16}
\lrad^{0.7} \lum$, which is in general agreement with models for
confined heavy jets. The utility of this relation lies in being able
to estimate total jet power from monochromatic all-sky radio surveys
for large samples of AGN at various stages of their outburst
cycles. When applied to the radio luminosity function at various
redshifts, the \pjet-\lrad\ relation can be used to infer the kinetic
heating of the Universe over cosmic time, and as a consequence, can be
used to infer the total accretion history and growth of SMBHs over
those same epochs. Further, inferences can be drawn regarding the
amount of preheating AGN could have contributed as large-scale
structure evolved, a long-standing question in cosmology
\cite{2001ApJ...555..597B}.

An interesting result which has emerged from our work is that FR-I
radio galaxies (classified on morphology and not \lrad) appear to be
systematically more radiatively efficient than FR-II sources. This may
mean there are intrinsic differences in radio sources (light and heavy
jets), or possibly that all jets are born light and become heavy on
large scales due to entrainment. One way to investigate this result
more deeply is to undertake a systematic study of the environments
hosting radio galaxies utilizing archival \chandra\ and VLA data.

With better observational constraints on the kinetic properties of AGN
jets, of interest to me is re-visiting existing models for
relativistic jets in an ambient medium. Utilizing
observationally-based estimates of jet power, it is possible to better
investigate the growth of a radio source including processes like
entrainment, scale-dependent changes in jet composition, and shocks
\cite[\'a la][]{1999MNRAS.309.1017W}. The \pjet-\lrad\ relation also
enables the investigation of relations between observable mass
accretion surrogates (\ie\ nuclear \halpha\ luminosity, molecular/dust
mass, or nuclear X-ray luminosity) and AGN energetics for the purpose
of establishing clearer connections with accretion mechanisms and
efficiencies.

The study of mechanical AGN feedback has advanced quickly in the last
decade primarily because the hot gas phase which this mode of feedback
most efficiently interacts is resolved with the current generation of
X-ray observatories. However, our understanding of radiative feedback,
and the associated early era of rapid SMBH growth, has not progressed
as quickly. This is mostly because cold/dusty gas is required for high
efficiency radiative feedback, but the presence of cold/dusty gas is
typically accompanied by significant optical obscuration which
prevents direct observational study
\cite{2009arXiv0911.3911A}. Luckily, the quality and availability
of multi-frequency data needed to probe the epoch of SMBH growth and
obscuration is poised to improve with new facilities and instruments
coming on-line (\ie\ LOFAR, Herschel, SCUBA-2, SOFIA, ALMA, NuStar,
Simbol-X). As such, there are a number of questions regarding the
formation and evolution of SMBHs that I would like to pursue.

{\bf{(1) What is the evolutionary track from young, gas-rich, dusty
galaxies to present-day old, parched gEs?}} It has been argued that
high-$z$ sub-mm galaxies (SMGs) are the progenitors for low-$z$
Magorrian galaxies, suggesting SMGs are useful for studying the
co-evolution of SMBHs and host galaxies. SMGs have also been shown to
reside in very dense environments and have high AGN fractions ($\ga
50\%$) \cite{2005ApJ...632..736A}, so they are excellent for
identifying the rapidly cooling high-$z$ gas-rich regions where star
formation and AGN activity are occurring. Thus, SMGs identify a unique
population to follow-up with far-IR and X-ray spectroscopy to study
epochs of early AGN feedback and environmental cooling. It has also
been posited that SMGs are high-$z$ analogs of low-$z$ ultraluminous
infrared galaxies (ULIRGs). If this is the case, insight to ULIRG
evolution can be gained from studying SMGs. ULIRGs are an interesting
population on their own, one for which limited X-ray spectroscopic
studies have been undertaken. We know these systems to, on average, be
dominated by star formation, however, some systems also have
significant contribution from very dusty AGN, and these systems can be
used to further understand the nature of evolving gas-rich
environments.

{\bf{(2) What is the relationship between redshift, environment, and
AGN feedback energy?}} The answer thus far is unclear, most because of
limited observational constraints. To this end, a study of the faint
radio galaxy population using archival \chandra\ and VLA data would be
interesting. Undertaking a systematic study of radio galaxy properties
(\ie\ jet composition, morphologies, outflow velocities, magnetic
field configurations) as a function of environment (\ie\ ambient
pressure, halo compactness) can help address how AGN energetics couple
to environment, which ultimately suggests how accretion onto SMBHs
depends on small and large scale environment. Deep \chandra\
observations for a sample of FR-I's (a poorly studied population in
the X-ray) would be useful for such a study. Using the
\pjet-\lrad\ relation, radio luminosities, lobe morphologies, and age
estimates can be used to predict ambient gas densities for the purpose
of robustly preparing the X-ray observations.

{\bf{(3) How does the transition of the nuclear region of a forming
galaxy from an obscured to unobscured state correlate with AGN
feedback and SMBH growth?}} As suggested by the low AGN fraction in
the \chandra\ Deep Fields, a significant population of obscured AGN
must exist at higher redshifts. One method of selecting unbiased
samples of these objects is to assemble catalogs of candidate AGN
using hard X-ray (\ie\ NuStar), far-IR (\ie\ SOFIA), and sub-mm (\ie\
SCUBA-2) observations. Because current models suggest the luminous
quasar population begins in an obscured state, and rapid acquisition
of SMBH mass may occur in this phase because of high accretion rates,
understanding the transition from obscured to unobscured states is
vital. How does accretion proceed and where does the accreting
material come from: gas cooling out of an atmosphere? Gas deposited by
merging companions? A related curiosity which has emerged in recent
years is the role of multiple AGN within the core of a host galaxy. At
a minimum, SMBH mergers occur on a timescale determined by dynamical
friction, which for a typical dense bulge is $\ga 1$ Gyr, which is
$\gg t_{\mathrm{cool}}$ of an obscuring atmosphere. If the SMBHs which
are merging have, or acquire, their own accretion disks, then it is
reasonable to question how the atmospheres surrounding a host galaxy
with multiple AGN is affected.

\scriptsize
\bibliographystyle{unsrt}
\bibliography{cavagnolo}
 
\end{document}
