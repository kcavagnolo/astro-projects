\documentclass[11pt]{article}
\usepackage[colorlinks=true,linkcolor=blue,urlcolor=blue]{hyperref}
\usepackage{subfig,epsfig,colortbl,graphics,graphicx,wrapfig,mathrsfs}
\usepackage{macros_cavag}
\pagestyle{myheadings}
\font\cap=cmcsc10
\setlength{\topmargin}{-0.35in}
\setlength{\oddsidemargin}{-0.25in}
\setlength{\evensidemargin}{-0.25in}
\setlength{\headheight}{0.1in}
\setlength{\headsep}{0.1in}
\setlength{\topskip}{0.1in}
\setlength{\textwidth}{6.8in}
\setlength{\textheight}{9.85in}

\begin{document}
The general process of galaxy cluster formation through hierarchical
merging is well understood, but many details, such as the impact of
feedback sources on the cluster environment and radiative cooling in
the cluster core are not. My thesis research has focused on studying
these details via X-ray properties of the ICM in clusters of
galaxies. I have paid particular attention to ICM entropy
distribution and the role of AGN feedback in shaping large scale
cluster properties.

My thesis makes use of a 350 observation sample (276 clusters; 11.6
Msec) taken from the {\it Chandra} archive. The picture of the ICM
entropy-feedback connection (Fig. \ref{fig:splots}) emerging from my work
suggests cluster radio luminosity and H$\alpha$ emission are
anti-correlated with cluster central entropy ($K=T_Xn_e^{2/3}$). There
also appears to be a bimodality in the distribution of central
entropies (Fig. \ref{fig:k0hist}) which is likely related to AGN
feedback (and to a lesser extent, mergers). I have found that clusters
with central entropy $\leq 20$ keV cm$^2$ exhibit star formation and
AGN activity in the BCG while clusters above this threshold
unilaterally do not have star formation and exhibit diminished AGN
radio feedback. This entropy level is auspicious as it coincides with
the Field length (assuming reasonable suppression) at which thermal
conduction can stabilize a cluster core. It is possible we have opened
a window to solving a long-standing problem in massive galaxy
formation (and truncation): how are ICM gas properties coupled to
feedback mechanisms such that the system becomes self-regulating?
However, this result serves to highlight unresolved issues requiring
further intensive study.

Most pressing of these issues is to better understand the fueling and
feedback from AGN. We know low entropy ($K_0 \leq 20$) systems contain
multi-phase gas (stars, cold molecular gas, warm/hot dust, et cetera), but as
evidenced by copious radio emission, some of this gas is likely
condensing onto the SMBH and resulting in episodic AGN feedback which retards
further cooling in the cluster core. My work in the X-ray can
only tell us about the hot atmospheres with which AGN are interacting,
but to attain a more complete picture of this multi-phase gas, and its
connection to fueling the AGN, it behooves us to look in other bands,
specifically the infrared.

In Figure \ref{fig:radk0} I have plotted central entropy derived in
my thesis work versus NVSS radio luminosity and overlaid symbols
indicating availability of data in the {\it Spitzer} archive. Thus
far, indications from the literature are that most, if not all, of the
BCGs in X-ray luminous clusters with $K_0 \leq 20$ keV cm$^2$ are
dominated by star formation. But we can see from the figure that most of
these systems contain radio AGN. So one can ask the question: are
there any AGN dominated nebular BCGs? An interesting project to pursue
with the {\it Spitzer} archive would be to examine the shape of
spectral energy distributions (SEDs) for all clusters with a BCG and
attempt to reveal if the BCG is star formation or AGN dominated.

For those BCGs which do exhibit AGN dominance it will be
interesting to exclude extended galaxy emission and analyze spectra
from only the (unresolved) nuclear region in an effort to characterize
AGN spectral features. The ultimate goal being an accounting of the
very lowest entropy gas which is likely feeding the SMBH, and at the
very least enshrouding the AGN. Clusters without star formation and
no AGN will also be an important constraint in such a project.

There are a multitude of other directions I would also like to pursue
which are essentially extensions of my thesis. The role of AGN
feedback in shaping global cluster properties is still poorly
understood. Models for the process of thermalizing energy in AGN blown bubbles
have been proposed, but details of these models still need to be explored.
For example, do bubbles contain a very low density non-relativistic
thermal plasma or are they truly voids in the ICM (potentially an SZ
experiment)? Maybe bubbles contain cosmic rays, a possibility which
will make for an interesting GLAST project. How do bubbles rise to
distances $\geq 100$ kpc without being shredded by instabilities? The
answer to this question will likely entail better understanding ICM
$\vec{B}$ fields, with their origin being either from preheating, AGN
deposition, or a combination of both.

I have also contributed to several successful {\it Chandra},
{\it XMM}, {\it Suzaku}, {\it NSF}, and {\it Subaru}
proposals in addition to writing my own high scoring -- although
unsuccessful -- {\it Chandra} proposal for time observing the amazing
ULIRG IRAS 09104+4109. I am also planning H$\alpha$ imaging
observations for several previously unobserved BCGs using SOI on MSU's
SOAR telescope, and will be active in submitting {\it Chandra} and
{\it XMM} proposals (both spectroscopy and grating) for unobserved and
interesting clusters, groups, and galaxies which have turned up in my
thesis work.

\clearpage
\begin{figure}[t]
    \begin{minipage}[t]{0.5\linewidth}
        \centering
	\includegraphics*[width=\textwidth, trim=26mm 8mm 30mm 10mm, clip]{splots}
        \caption{\small Entropy profiles for 143 clusters of galaxies
	in my thesis sample. The range of central entropies is
	consistent with models of episodic AGN heating which regulate
	the presence of low entropy gas in cluster cores. The
	so-called ``cooling flow'' problem does not appear to be a problem any
	longer.}
	\label{fig:splots}
    \end{minipage}
    \hspace{0.1in}
    \begin{minipage}[t]{0.5\linewidth}
        \centering
        \includegraphics*[width=\textwidth, trim=28mm 8mm 30mm 10mm, clip]{k0hist}
        \caption{\small Distribution of central entropy for an
	unbiased sub-sample of the clusters analyzed for my thesis. Note the
	fall-off of clusters with $K_0 \sim 30-50$ keV
	cm$^2$. An explanation for this bimodality utilizing AGN
	feedback (the most likely candidate) does not currently
	exist.}
	\label{fig:k0hist}
    \end{minipage}
    \hspace{0.1cm}
    \begin{minipage}[t]{0.5\linewidth}
        \centering
        \includegraphics*[width=\textwidth, trim=28mm 8mm 30mm 10mm, clip]{spitzer_radk0}
        \caption{\small Central entropy derived in my thesis work
	plotted against radio luminosity calculated using
	NVSS. Cluster centers with MIPS observations are plotted with
	red squares; IRAC observations have blue circles; IRS observations
	have green triangles. The Spitzer archive provides excellent
	coverage for a possible study of low entropy, radio-loud and
	radio-quiet systems.}
        \label{fig:radk0}
    \end{minipage}
\end{figure}
\end{document}
