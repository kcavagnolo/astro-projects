\documentclass[11pt]{article}
\usepackage[colorlinks=true,linkcolor=blue,urlcolor=blue]{hyperref}
\usepackage{subfig,epsfig,colortbl,graphics,graphicx,wrapfig,mathrsfs}
\usepackage{macros_cavag}
\pagestyle{myheadings}
\font\cap=cmcsc10
\setlength{\topmargin}{-0.15in}
\setlength{\oddsidemargin}{-0.15in}
\setlength{\evensidemargin}{-0.15in}
\setlength{\headheight}{0.1in}
\setlength{\headsep}{0.25in}
\setlength{\topskip}{0.1in}
\setlength{\textwidth}{6.5in}
\setlength{\textheight}{9.25in}

\begin{document}
\markright{OATS Fellowship Proposal, Kenneth W. Cavagnolo}
\begin{center}
\large
\textbf{A Multiwavelength Study of Pre-AGN Outburst BCGs in X-ray
Luminous Clusters and Groups of Galaxies}
\end{center}
\normalsize

\section{Motivation}

The general process of galaxy cluster formation through hierarchical
merging is well understood, but many details, such as the impact of
feedback sources on the cluster environment and radiative cooling in
the cluster core are not. My thesis research has focused on studying
these details via X-ray properties of the ICM in clusters of
galaxies. I have paid particular attention to ICM entropy
distribution and the role of AGN feedback in shaping large scale
cluster properties.\\

I make heavy use of the physical quantity ``entropy'' in this
proposal, thus I will review it briefly here. Gas temperature and
density taken alone do not reveal the entire thermal history of a gas
parcel. But placed together in the context of the adiabatic constant,
$K=Tn^{-2/3}$, we have a quantity which is a function of heat input
and radiative losses. Gravitational potential wells are like giant
entropy sorting devices, low entropy gas sinks to the bottom while
high entropy gas buoyantly rises, all in obedience of establishing
convective stability, $dK/dr \geq 0$. Therefore the radial entropy
distribution of a cluster is a tracer of the underlying dark matter
potential, and deviations from a self-similar power-law are indicative
of heating or cooling (for a thorough review see
\cite{2005RvMP...77..207V}).\\

Plotted in Figure \ref{fig:splots} is entropy as a function of radius
for 140+ clusters which I analyzed from a 350 observation sample (276
clusters; 11.6 Msec) taken from the {\it Chandra} Data Archive and
used in my thesis. The characteristic entropy ``pedestal'' seen in the
cores of all these clusters is consistent with models of episodic AGN
feedback (i.e. \cite{2005ApJ...634..955V}), with the largest departures
($K_0 \geq 100$ keV cm$^2$) likely due to a combination of AGN
feedback and subsequent mergers. The systems with the lowest central
entropy have cooling times which are only a few tens of Myrs and
should be returning to a state where AGN feedback will again be initiated.\\

There has been great success recently in explaining the lack of
massive cooling flows in the cores of galaxy clusters with AGN providing
the necessary heat to retard catastrophic cooling
(\cite{2007ARA&A..45..117M}). But a similarly successful model for the
fueling of AGN and heating from AGN has been slow
coming. Observations of interaction between the hot atmospheres of
clusters and the AGN at their centers abound
(i.e. \cite{2004ApJ...607..800B}), but direct observations of ongoing shock
heating from a young AGN are rare. Clusters such as Centaurus and
Abell 1991 (lowest $K_0$ points in both entropy figures) are good
examples of BCGs housing AGN which have recently started their duty
cycles, but one should also ask the question, what did the environment
around these sources look like just prior to their most recent
outburst?\\

The picture of the ICM entropy-feedback connection emerging from my
thesis work suggests BCG radio luminosity (Fig. \ref{fig:radk0}) and
BCG H$\alpha$ emission (Fig. \ref{fig:hak0}) are anti-correlated with
cluster central entropy. I have found that clusters with central
entropy $\leq 20$ keV cm$^2$ exhibit star formation and radio AGN activity
in the BCG while clusters above this threshold unilaterally do not
have star formation and exhibit diminished AGN radio
feedback. Corroborating observations of strong blue gradients as a
function of decreasing central entropy in BCGs has also been seen by
Rafferty et. al. (2007 in press). This entropy level is auspicious as
it coincides with the Field length (assuming reasonable suppression)
at which thermal conduction can stabilize a cluster core
(Fig. \ref{fig:conduction}). It is possible conduction is the physical
process by which ICM gas properties are coupled to feedback mechanisms
such that the system becomes self-regulating. If an AGN outburst does
not boost the ICM entropy of the core to greater than $\sim 20 $ keV
cm$^{2}$ then the core will not be subject to stabilization by
conduction and cooling plus condensation will proceed, ultimately
leading to future AGN outburst(s).\\

However, looking closer at Figures \ref{fig:radk0} and Figure
\ref{fig:hak0} you will notice blue boxes with red stars. These points
indicate clusters (henceforth I refer to all objects in my proposed sample as
``clusters'', but the sample is a mix of large groups/poor
clusters/rich clusters) with BCGs which {\bf do not} have star
formation or radio AGN, or in the case of Abell 2107, neither. There
is the added curiosity that none of the clusters exhibit signs of
mergers or dramatic AGN feedback (bubbles or cavities) which might
explain their current physical state (the exception being Abell 133
which has a cool ``tongue'' and radio relic/ghost whose origin is not
clear \cite{2004ApJ...616..157F}). General properties of my proposed
sample are listed in Table \ref{tab:sample}.\\ 

My proposition is that these clusters are in the final stages of low
entropy gas condensation onto the BCG resulting in star formation,
production of cold molecular gas clouds, and with a modest amount of
gas flowing to the very core where the SMBH is waiting to be fed and
initiate a feedback cycle. This is a very important stage in the
life-cycle of the most massive galaxies in the Universe, and
performing an observational ``bookkeeping'' of all the low entropy gas
should yield insight to how massive galaxy formation is ultimately
truncated via AGN feedback. {\bf As a fellow at OATS, I propose to
conduct a multiwavelength (X-ray, UV, IR, radio) observational
campaign for this peculiar class of galaxy clusters/groups, which have $K_0 \leq
20$ keV cm$^2$ and either no radio emission or no star formation, in an
effort to understand how AGN are fueled and stars form in the most
massive galaxies in the Universe.} This work will characterize the gas
environments of low entropy systems which have thus far been
under-studied and hopefully yield new constraints on the properties of
gas accreting onto the SMBH in the BCG of these systems. Additionally
this project will address the hypothesis that conduction plays a crucial
role in setting the entropy scale below which star formation occurs
and AGN heat input is dissipated into the ICM.

\section{Observations}
Multi-phase gas can only be properly accounted for by observing across
multiple wavelength regimes. The coldest gas will be brightest in the
Far-IR while the hottest gas has already been observed in the X-ray
(the impetus for this proposal). As reference, Table
\ref{tab:observations} lists data which currently resides in publicly
accessible archives and will serve to streamline this project.

\subsection{X-ray}
While high resolution X-ray spectra of cool core clusters have
disproved the prescience of massive quantities of gas expected in
classical cooling flow models (\cite{2006PhR...427....1P}), they have
also proven it is possible to detect the heavy element recombination lines
of species such as FeXVII and OVIII. These are an important
diagnostics for calculating the true mass of gas cooling out of the ICM
and into a BCG. For my proposed project I will utilize the RGS
instruments on-board {\it XMM-Newton} to collect high resolution
spectra of the cores of my sample. To a lesser degree, {\it Chandra}'s
HETG and LETG instruments will be utilized for cross-calibration
purposes. The {\it XMM-Newton} archive will also be utilized to 
cross-calibrate the existing {\it Chandra} spectroscopic analysis in
my thesis.

Only three clusters require proposed {\it XMM} observing time: Abell
1204, Abell 2107, and Abell 2556. All of these objects are very
luminous (L$_X > 10^{44}$), cool (T$_X < 4$ keV), and relatively
nearby (z $<$ 0.18) making them ideal candidates for an {\it XMM}
observing proposal. As a fellow, and with the excellent science
justification for observing these clusters, it is reasonable to
believe I will be awarded observing time.

\subsection{Ultraviolet and Optical}
As an additional constraint on the gas cooling rate, data from {\it
XMM-Newton}'s Optical Monitor will be utilized. This data will be used
to calculate star formation rates which will compliment and further
constrain rates calculated from RGS/HETG/LETG X-ray spectra and
X-ray imaging spectroscopy of {\it Chandra}. The research group I'm
presently working with has already utilized this technique of
calculating SFR from UV excess for 2A 0335+096,
\cite{2007AJ....134...14D}).

\subsection{Infrared}
Not all UV indicators are unbiased in the estimation of SFRs
(\cite{2007ApJ...666..870C}). Because ionizing radiation can be
reprocessed by dust enshrouding star forming regions, the SFRs
calculated from UV alone can be misleading. This is where utilization
of {\it Spitzer} MIPS and IRAC data will provide an additional, tighter
constraint on calculating SFRs and fully accounting for the fate of
the cool gas condensing in these low entropy systems. Far-IR
photometry will also be useful in determining if a heavily dust
obscured AGN is present in systems which no radio AGN was previously
detected. Removal of AGN contamination will also be important for
accurate determination of SFRs using IR data. The data presently
available in the {\it Spitzer} archive does not provide full-coverage
of my sample, but the observations I am suggesting in the Far-IR do
not require the use of cryogen. Thus it is reasonable to believe 
submission of an observing proposal of $\sim 10-12$ hr. in next year's
{\it Spitzer} AOR cycle which targets my sample's unobserved clusters
has a good chance of being accepted.

\subsection{Radio}
Analysis of radio data is not predicated upon acceptance of observing
time as all the data needed already resides in the VLA archives. For
the BCGs which do exhibit radio AGN activity it will be important to
reanalyze VLA FIRST or SUMMS data to calculate the energetics of the
outflows from the AGN. If conduction is an important mechanism in
distributing heat throughout a cool core, then one should also expect
AGN kinetic energy to preferentially interact with low density gas
and leaving high density gas (in which star formation is most likely
to occur) intact. This effect has already been observed in a few
cluster cores, but for very powerful AGN outbursts and in BCGs which
are not currently experiencing star formation. Uplifting of low
entropy gas by an AGN will also need to be investigated as such an
effect can potentially skew the total cooling mass too high. I have
only listed VLA FIRST high resolution observations which are
publicly available in Table \ref{tab:observations}, but radio data
does exist for all my sources, it will need to be acquired through
personal request.

\section{Analysis: Star Formation Rates and Accounting for Cool Gas}
This project is an attempt to directly combine model independent
measurements for the majority of cooling gas in the cores of low
entropy clusters and characterize the environments in which star
formation and AGN feedback {\it are initiated}. The project I am
proposing can be summarized as an attempt to construct the broadest
wavelength spectral energy distribution (SED) possible for a sample of
extremely interesting cluster BCGs; this is a bookkeeping project
which will be a useful for a better  framework of understanding what
happens to cool gas just before and just after an AGN feedback cycle
starts.

To constrain the star formation rates in the BCGs of my proposed
sample clusters I have chosen a multi-pronged attack. The first task is
to calculate X-ray cooling rates for the cluster core from the
temperature and density distributions of the ICM. This does not
require much effort as I have already done this for my thesis. The
X-ray cooling rate establishes the expected amount of gas which will be
condensing onto the BCG. Recall, these systems have been chosen
because they are dramatically different than their well-heated
brethren: low entropy with no radio AGN or no star formation, all
three of which are indicators that the cluster is nearing or long
removed (t $>$ 1 Gyr) from the last major heating event.

An additional robust constraint on the properties of the cooling gas
will be calculated using the high resolution spectra from RGS. The
relative strengths of lines from heavy elements species such as Fe, O,
Mg, Ne, and Si will be used to calculate cooling rates and also will
serve as a temperature diagnostic for gas which is not
spectroscopically resolved by EPIC/MOS and ACIS. Of course these
emission lines will not be present with the strengths classic cooling
flow models predict, but this precisely the physical characterisitc I
am proposing to study.

Complimentary Far-IR data from {\it Spitzer} and UV data from {\it
XMM}-OM will also be used to calculate cooling rates. The emission
lines from polycyclic aromatic hydrocarbons (PAHs) longward of
$\sim2\mu$ are very useful for detecting dusty starbursts which
would otherwise be missed. For the starlight which is not being
reprocessed by dust, it can be directly observed in the near-UV and
will show up as a luminosity excess. Assuming an IMF and using the
L$_{UVW1}$-L$_J$ relation of \cite{2005ApJ...635L...9H} (which is
founded upon archival {\it XMM}-OM data) any net UV excess can be
used to calculate to star formation rates.

Taking all these constraints together will provide a stringent
accounting of the state of the lowest entropy gas in these
clusters. Taking this sample as a distinctly unique set of clusters
and comparing them other well studied BCGs and cluster cores should
reveal they are different in some fundamental way. Certainly these
clusters will be like all others and have ``mini''-cooling flows, but
will these flows be more massive with more by-products (stars, GMC,
H$_2$, CO, etc.)? The answer to this question is interesting for
several reasons: 1) if the flows aren't ``different'', then the standard framework
of episodic AGN heating is intact and I will have proven these objects
are only nascent/quiescent as is expected; 2) if these clusters are different
however then the process of feedback is acting differently on these
objects, a result which most likely points to the effects of
conduction.

A logical next step will be in the utilization of radio data to
understand the kinematics of the radio AGN (for sources which have
them). Within this sample of peculiar clusters should arise another
dichotomy- clusters with AGN and clusters without. The star formation
rates and cooling rates should be different between these two classes
of cluster. This will be related to the plasma outflows of the AGN and
may also be related to the weak shocking of the ICM as the jets
supersonically move through the dense cluster core environment. An
additional exciting use for the radio data will be to ``radio date''
the AGN in an effort to assign an age for these sources. Several
diagnostics will be useful: 1) dynamic age -- the age inferred from
kinematics of the source (i.e. distance of jets/lobes from nuclear
source); 2) synchrotron age -- the age inferred from the break frequency
of the radio spectrum (presuming no in situ reacceleration of the jet
outflow). Ages of the radio sources will be a very interesting piece
of information as it relates directly to the timescales of ICM
condensation and feedback energy thermalization.

\section{Benefits to OATS Science and Conclusion}
This proposal seeks to address many outstanding issues in the feedback
regulated cooling in the cores of clusters, all of which relate
directly to research currently underway at OATS. How truncation/downsizing
of the massive end of the galaxy luminosity function (GLF) proceeds is not well
constrained from observation. However, cosmological simulations which
include cooling and feedback are beginning to generate gas
distributions which agree with observation
(\cite{2007ApJ...668....1N}). These simulations also have GLFs which
have the appropriate density of massive galaxies and have cluster BCGs
which are blue and not red (\cite{2006MNRAS.365...11C}).

But the successes of models including AGN feedback have also served to
highlight the failings of these models observationally, specifically,
we do not currently understand how feedback energy is thermalized
within the ICM and most importantly we do not understand how AGN are
fueled via cooling from the ICM. As I presented in the introduction,
there is good observational and theoretical reason to believe
conduction is the answer to both these problems. But in the context of
conduction being the solution we will need to account for the peculiar
class of clusters which I have presented in this proposal which 
have low central entropy, no radio AGN (radio being the favored
mode of energy transport), and/or no star formation (as inferred from
H${\alpha}$ measurements). OATS has an established stake in all these
areas of research and will benefit greatly from endorsing me as a
fellow to further study them.

The research groups at OATS have an established reputation for use of
{\it XMM-Newton} which will only serve to make my proposed project all
the more fruitful. Specifically, I have not touched on the
implications of star formation on metal enrichment in the cores of
these clusters, turbulent mixing of the ICM from AGN, and the
ICM magnetic fields and their role in shaping the interaction of the
AGN with the ICM. All of these topics have been or are under study by
someone at OATS. Many other archival projects could be produced from
this proposal by expanding the scope to include the above mentioned
topics.

Along with this proposal are also many other topics which could
potentially be studied. For example, do AGN blown bubbles contain a very low
density non-relativistic thermal plasma or are they truly voids in the
ICM (potentially an SZ experiment)? Maybe bubbles contain cosmic rays,
a possibility which will make for an interesting GLAST project. How do
bubbles rise to distances $\geq 100$ kpc without being shredded by
instabilities? The answer to this question will likely entail better
understanding of ICM $\vec{B}$ fields, with their origin being either
from preheating, AGN deposition, or a combination of both.

In conclusion, the class of peculiar galaxy clusters I have presented
warrant extensive study in their own right, but a uniform, systematic
study of these objects will have broad implications for better
understanding AGN feedback and star formation in the most massive
galaxies in the Universe.

\clearpage
\begin{figure}[t]
    \begin{minipage}[t]{0.5\linewidth}
        \centering
        \includegraphics*[width=\textwidth, trim=26mm 8mm 30mm 10mm, clip]{splots}
        \caption{\small Entropy profiles for 143 clusters of galaxies
        in my thesis sample. The range of central entropies is
        consistent with models of episodic AGN heating which regulate
        the presence of low entropy gas in cluster cores. The
        so-called ``cooling flow'' problem does not appear to be a problem any
        longer.}
        \label{fig:splots}
    \end{minipage}
    \hspace{0.1in}
    \begin{minipage}[t]{0.5\linewidth}
        \centering
        \includegraphics*[width=\textwidth, trim=28mm 8mm 30mm 10mm, clip]{k0rad}
        \caption{\small Central entropy derived in my thesis work
        plotted against radio luminosity calculated using
        NVSS. Clusters without radio source detections are represented
	by upper-limits (left pointing arrows). The eight clusters in my
	sample without radio detections and $K_0 < 20$ are plotted as blue
	boxes with red stars.}
        \label{fig:radk0}
    \end{minipage}
    \hspace{0.1in}
    \begin{minipage}[t]{0.5\linewidth}
        \centering
        \includegraphics*[width=\textwidth, trim=28mm 8mm 30mm 10mm, clip]{ha_k0}
        \caption{\small Central entropy derived in my thesis work
        plotted against H$\alpha$ luminosity calculated from data in
	\cite{1999MNRAS.306..857C}. Clusters without H$\alpha$ source
	detections are represented by upper-limits (left pointing arrows). The
	five clusters in my sample without H$\alpha$ detections and $K_0 < 20$
	are plotted as blue boxes with red stars.}
        \label{fig:hak0}
    \end{minipage}
    \hspace{0.1in}
    \begin{minipage}[t]{0.5\linewidth}
        \centering
        \includegraphics*[width=0.95\textwidth, trim=0mm 0mm 0mm 0mm, clip]{conduction}
        \caption{\small 
	Toy entropy profiles plotted as a function of radius and overlaid with
	dashed lines representing cooling and conduction equivalence for two
	suppression factors. Above the dashed lines conduction is effective
	and condensation cannot occur, the opposite is true below the
	lines. $K_0 < 20$ keV cm$^2$ is the break point at which the Field
	length criterion suggests gas condensation (i.e. star formation and
	condensation on the SMBH) can proceed. Reproduced courtesy of Dr. G.
	Mark Voit.}
        \label{fig:conduction}
    \end{minipage}
\end{figure}

\begin{center}
General Properties of Cluster/Group Sample
\small
\begin{tabular}{lcccccccc}
\hline
Name & RA         & Dec           & z   & T$_X$ & K$_0$      & L$_{bol.}$    & L$_{H\alpha}$ & L$_{Radio}$\\
---  & hr:min:sec & $^\circ:':''$ & --- & keV   & keV cm$^2$ & 10$^{44}$ cgs & 10$^{39}$ cgs & 10$^{39}$ cgs\\
\hline
\hline
Abell 133            & 01:02:41.756 & -21:52:49.79 & 0.0558 & 3.71 & 17.26 & 6.46 & 6.00    & $<$2.03\\
Abell 1204           & 11:13:20.419 & +17:35:38.45 & 0.1706 & 3.63 & 15.31 & 3.92 & 58.6    & $<$22.2\\
EXO 0422-086$^{(g)}$ & 04:25:51.271 & -08:33:36.42 & 0.0397 & 3.41 & 13.77 & 0.65 & $<$0.10 & 45.2\\
Abell 2556           & 23:13:01.413 & -21:38:04.47 & 0.0862 & 3.57 & 12.38 & 1.43 & planned & $<$5.07\\
Abell 2107           & 15:39:39.113 & +21:46:57.66 & 0.0411 & 3.82 & 11.11 & 3.02 & $<$1.65 & $<$0.79\\
Abell 2029           & 15:10:56.163 & +05:44:40.89 & 0.0765 & 8.20 & 10.50 & 13.9 & $<$5.91 & 822\\
AWM7$^{(pc)}$        & 02:54:27.631 & +41:34:47.07 & 0.0172 & 3.71 & 10.21 & 4.07 & planned & $<$0.18\\
MKW4$^{(g)}$         & 12:04:27.218 & +01:53:42.79 & 0.0198 & 2.16 & 6.86  & 0.46 & planned & $<$0.24\\
ESO 5520200$^{(g)}$  & 04:54:52.318 &  18:06:56.52 & 0.0314 & 2.34 & 5.89  & 1.40 & planned & $<$0.62\\
MS J1157.3+5531      & 11:59:52.295 & +55:32:05.61 & 0.0810 & 3.28 & 5.54  & 0.12 & planned & $<$4.44\\
Abell 2151$^{(pc)}$  & 16:04:35.887 & +17:43:17.36 & 0.0366 & 2.90 & 4.27  & 1.41 & $<$1.30 & 0.59\\
RBS 533$^{(pc)}$     & 04:19:38.111 & +02:24:35.62 & 0.0123 & 1.29 & 2.56  & 0.17 & $<$0.14 & 0.61\\
%Centaurus            & 12:48:48.926 & -41:18:44.75 & 0.0109 & 3.96 & 1.34  & 1.74 & 15.0    & 6.79\\
%Abell 1991           & 14:54:31.620 & +18:38:41.48 & 0.0565 & 5.40 & 1.33  & 1.35 & 5.48    & 32.5\\
\hline
\end{tabular}
\label{tab:sample}
\end{center}
\footnotesize
Notes: Clusters are ordered by decreasing $K_0$; (g) denotes a group;
(pc) denotes a poor cluster. Clusters without H${\alpha}$ are scheduled
to be observed using the SOAR Optical Imager (SOI) which is on MSU's
SOAR Telescope in Cerro Pach\'{o}n, Chile.\\
\normalsize

\begin{center}
Exisiting Archival Data
\small
\begin{tabular}{lcccccccccc}
\hline
Name & X-ray & Inst. & Grating & Inst. & UV & Inst. & IR & Inst. & Radio & Inst.\\
\hline
\hline
Abell 133       & Y & {\it XMM} & Y & RGS & Y & {\it XMM}-OM & N & ------- & Y & priv.\\
Abell 1204      & N & --- & N & --- & N & ------ & Y & {\it Spitzer} & Y & {\it VLA}\\
EXO 0422-086    & Y & {\it XMM} & Y & RGS & Y & {\it XMM}-OM & N & ------- & Y & priv.\\
Abell 2556      & N & --- & N & --- & N & ------ & N & ------- & Y & priv.\\
Abell 2107      & N & --- & N & --- & N & ------ & N & ------- & Y & priv.\\
Abell 2029      & Y & {\it XMM} & Y & RGS & Y & {\it XMM}-OM & Y & {\it Spitzer} & Y & {\it VLA}\\
AWM7            & Y & {\it XMM} & Y & RGS & Y & {\it XMM}-OM & N & ------- & Y & priv.\\
MKW4            & Y & {\it XMM} & Y & RGS & Y & {\it XMM}-OM & Y & {\it Spitzer} & Y & priv.\\
ESO 5520200     & Y & {\it XMM} & Y & RGS & Y & {\it XMM}-OM & N & ------- & Y & priv.\\
MS J1157.3+5531 & Y & {\it XMM} & Y & RGS & Y & {\it XMM}-OM & N & ------- & Y & priv.\\
Abell 2151      & Y & {\it XMM} & Y & RGS & Y & {\it XMM}-OM & Y & {\it Spitzer} & Y & {\it VLA}\\
RBS 533         & Y & {\it XMM} & Y & RGS & Y & {\it XMM}-OM & N & ------- & Y & priv.\\
%Centaurus       & Y & {\it XMM} & Y & RGS & Y & {\it XMM}-OM & N & ------- & Y & priv.\\
%Abell 1991      & Y & {\it XMM} & Y & RGS & Y & {\it XMM}-OM & N & ------- & Y & priv.\\
\hline
\end{tabular}
\label{tab:observations}
\end{center}
\footnotesize
Notes: All clusters have publically available {\it Chandra} data.
\normalsize


\clearpage
\bibliographystyle{unsrt}
\bibliography{cavagnolo}
 
\end{document}
