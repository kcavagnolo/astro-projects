\documentclass{article}
\usepackage{psfig}
\usepackage{amssymb}


\usepackage{graphics,graphicx}
\textheight 9.0in
\textwidth 6.0in
\topmargin -.65in
\oddsidemargin 0.1in


\begin{document} 
%\input head.sty
\special{papersize 8.7in 11.2in}


\begin{center}
\large
{\bf The content of giant cavities in the IGM of galaxy clusters}
\normalsize
\end{center}

\bigskip



ABSTRACT {\sf


% BEGIN TYPING BELOW THIS LINE 
X-ray images of the cores of massive clusters of galaxies have
revealed giant cavities and shock fronts in the hot gas, revealing
evidence of energy injection from active nuclei of the central
galaxies. How the energy is injection in the intergalactic medium
(IGM) via radio jets and how this energy is converted to heat is
poorly understood, and is crucial to the understanding of galaxy
evolution.  It is interesting to note that though excellent Chandra
X-ray observations exist for a number of these systems, comparable 
radio observations, which are vital to this study,
are absent.
We propose to observe five clusters of galaxies that have
excellent X-ray observations (3 in Cycle~16, 2 in Cycle~17) at 3 GMRT
frequencies (150, 240 \& 610 MHz), which, together with extant higher
frequency VLA data, will reveal the content of these cavities, the
extent of radio plasma and its relation to the cavities and shock
features in the X-ray gas.


% END TYPING ABOVE THIS LINE 
	} % Please don't delete this line.




% INCLUDE SCIENTIFIC JUSTIFICATION HERE

\bigskip


{\bf 1. Scientific Justification}






The origin and composition of extragalactic radio
jets has remained
enigmatic since their discovery more than a half century ago.  By
virtue of their synchrotron emission, we know that they are in part
composed of relativistic electrons and magnetic fields.  Theoretical
models of jets (Scheuer 1974; Begelman, Blandford, Rees 1984) have
shown that their energetics are dominated not by photons, but rather
by mechanical energy.  However, the ratio of mechanical energy to
synchrotron energy (radiative efficiency) which provides a strong clue
to their composition cannot be determined by radio observations alone.
Further clues emerge form X-ray observations of the intergalactic
medium surrounding these galaxies.  X-ay images of the  
hot inter-galactic medium (IGM) of 
galaxies and clusters have revealed giant cavities, measuring a few to
more than 200 kpc across embedded in their halos.  The commonness and
variety of bubbles, cavities, and edges observed both in the radio and
in X-rays in groups and clusters provides direct evidence of the
widespread presence of AGN-driven phenomena (see, e.g., McNamara and
Nulsen 2007 for a general discussion of the field; Fabian et al. 2003,
Perseus; Nulsen et al. 2005 for shock heating in Hydra; Birzan et
al. 2004 for a survey of cavities in clusters and groups).

The pV work required to
inflate these cavities, which can be
measured in a straightforward way in Chandra observations,
gives a measure of the total energy released during a radio outburst.
Coupling this with the ages of the cavities based on the well
justified assumption of buoyancy and in many cases the ages of shock
fronts associated with the cavities, provides a direct measurement
(lower limit) of their total mechanical energy and mean jet power.
Combining the total energy measurements from X-ray observations with
synchrotron power measurements over a broad frequency range is can be
used to place interesting limits on the ratio ($k$) of energy flux
carried by protons or other massive particles to that in electrons.

Using samples of central cluster galaxies harbouring prominent cavity
systems filled with radio emission, Dunn, Fabian, and Taylor (2005)
and Birzan et al. (2008) have shown that on average $k \gg 1$, and in
some cases $k$ exceeds several thousand.  This implies that the energy
flux in jets is dominated by protons, presumably
either launched at the
base of the jet or  entrained from surrounding material as
the jet advanced through the IGM.  A dramatically
different interpretation by Diehl et al (2008) suggest
that the distribution of cavities in clusters is consistent with
current-dominated, MHD jets.  However, none of these studies were
based on radio observations below 1.4~GHz (some have shallow 
VLA 320~MHz observations), which are crucial for
understanding the content of radio sources.

Birzan et al. (2008) and Wise et al. (2007) have shown from low
frequency radio observations that radio-emitting plasma completely
fills X-ray cavities in some systems.  When coupled with X-ray
observations, low frequency radio observations provide the best tracer
to the total energy output of radio jets.  Observations below 300~MHz
are crucial because they probe the possible existence of a faint, yet
energetically important population of electrons that can only be
probed below 300~MHz.  Earlier estimates of $k$ required extrapolating
the synchrotron spectrum below 320~MHz to 10~MHz based on spectral
fits anchored at 320~MHz, 1.4~GHz, and 8~GHz, yielding large
uncertainties in $k$ (factors of tens).  

However, in the aggregate, $k
\gg 1$, with values varying from close to unity to several thousand. The
proposed observations of clusters with the best jet cavity power
measurements from X-ray observations will provide the best constraints
available on $k$, and thus the content of extragalactic radio sources.


Low frequency radio observations will be absolutely crucial in
resolving the issue of the content of these cavities, 
 because they are most
sensitive to the history of AGN activity over timescales $> 10^8$ yr,
while high frequency observations are sensitive to the instantaneous
jet power.  The GMRT is uniquely placed to address this question, as
can be seen in a composite X-ray and radio image of
Hydra A, which is shown in Fig.~1 (Wise et al. 2007, ApJ, 659, 1153).
Low frequency 320 MHz emission fills the enormous (100-200 kpc
across), older ($\sim 10^8$ yr), and more energetic $\sim 10^{61}$ 
erg cavities,
while the 1.4~GHz emission fills the smaller ($\sim 20$ kpc across), 
younger
($10^7$ yr), less energetic ($\sim 10^{59}$ erg) cavities.  
Thus low frequency
radio studies have the potential to detect and evaluate faint, large
scale cavity systems of enormous power that might otherwise go
undetected in X-ray and high frequency radio observations.  Very low
frequency radio observations are at the frontier of AGN feedback
studies and they will provide crucial clues to the nature of the
particles or fields that carry the bulk of the momentum and energy of
extragalactic radio sources on large scales.








\medskip
{\bf 2. Observing plan}

We have chosen 5~targets from the sample of Birzan et al. (2008), which are clusters that exhibit the best evidence of cavities in Chandra observations,
and evidence of higher frequency VLA observations of radio-emitting plasma 
filling these cavities.
Both MS0735 and A2052 have 
been awarded $>$500~ks of Chandra observations to us and
our collaborators. the others too have excellent Chandra observations which we have analysed. {\it These, together with the Perseus cluster, form the best sample from which the issue of the content of cavities can be addressed}.

We propose to observe these clusters at 610/240 MHz (dual frequency)
and at 150 MHz to obtain detailed spectral index maps of the radio
emission interacting with the hot IGM. For each observation, we ask
for a full synthesis of 8 or 9 hours (according to availability of
source) to allow adequate u-v coverage for detailed mapping. All short
spacings are important since we wish to map extended emission
occupying a significant fraction of the field. Our previous experience
with observations of groups (e.g. 13SGa01) informs us that the 
integration times will be adequate to reach flux densities of 0.3 mJy/b
at 240 and 610 MHz. 


We propose to extend our observations to the lowest available
frequency, 150~MHz, motivated by the following considerations:\\ (1)
Broad frequency coverage is essential for the preparation of a useful
set of spectral index maps and the reliable interpretation of the
radio spectrum; and \\ (2) The lowest frequencies are needed to show
the fullest history of AGN activity, as they show the oldest electron
populations, and accordingly often have the greatest spatial extent.
In the past, observations at 150~MHz have been limited by
nearly-prohibitive RFI.  We have recently been encouraged by early
tests of an innovative RFI mitigation approach (Athreya 2008), which
on an observation of NGC~7626 at 150 MHz has shown remarkable
improvements, with sensitivity of 0.5~mJy/beam and a dynamic range of
$10^4$ (Fig.~4), a factor of 10 better than normally obtained even at
higher frequencies and important to many of our observations which
contain strong point sources in the field. (The new technique should
provide substantial assistance also at 235 MHz.)  In addition to
furthering the goals of this proposal, the results of our trials with
low-frequency RFI mitigation are likely to benefit numerous other
observers.  The proposed observations represent the first exploration
of which we are aware of a broad sample of radio galaxies at this low
a frequency, with high angular resolution and sensitivity.

A total of 52 hours is requested.

\vfill\eject

{\centerline {\bf Table 1. Project source list (Cycles 16 plus 17). }}

%\begin{table}
%[h!]
\begin{center}
\begin{tabular}{lccccrccc}
\hline
Source & RA$_{J2000}$ & DEC$_{J2000}$ & z \\
       &              &               &   \\
\hline

Abell 2052  & 15 16 45.5& +07 00 01 & 0.0355 \\
Abell 1835  & 14 01 02.0& +02 51 32 & 0.2532\\
Abell 2597  &23 25 18.0& -12 06 30 & 0.0852 \\
\hline
MS0735.6+742 & 07 41 44.47 & 74 14 38.10  & 0.216 \\
Hydra~A  & 09 18 05.7 & -12 05 44 & 0.0549 \\

\hline
\end{tabular}
%&&&&&&\\
\end{center}
%\end{table}
We request night time observations- so we propose to observe
the first three targets in Cycle~16, and the other two in the
next Cycle.
\medskip



\medskip
{\bf 3. Status report on earlier proposals: 10SGa01, 12SGa01, 13SGa01, and 14SGa01}

Two of us (Raychaudhury, Athreya) have been involved in a survey of
feedback in groups of galaxies.
The first paper, on AWM~4, from these observations has been published,
and a copy is included with this proposal.  
Since the advent of X-ray data of high quality from XMM,
this apparently relaxed group has been puzzling in showing no cooling core,
but strong temperature and abundance substructure not reflected in the
X-ray intensity maps; arguably we are observing a system reheated by
AGN activity (O'Sullivan et al 2005).  GMRT data now grant us
a new and detailed look into the process: we determine that the radio
structure -- which we map with high fidelity at three frequencies --
is nearly in the plane of the sky, measure the galaxy motion with respect
to the surrounding medium, estimate significant physical parameters of
the radio source, produce spectral index maps that show a smooth progression
of electron aging moving out from the AGN, and determine the source age.
The level of detail at which this remarkable source can be understood is
indicative of both the strength of the radio and X-ray instrumentation
and the scientific synergy obtained from employing them jointly.  
We have early indications that quite a number of the other sources on
our target list will prove equally valuable. 

Results from our earlier GMRT proposals have been presented at the
following conferences:\\
(1) ``SESTO 2007-Tracing Cosmic Evolution with Clusters of Galaxies: Six Years
Later'' June 25-29, 2007.\\
(2) Eight Years of Science with Chandra, October 2007, Huntsville, AL.\\
(3) American Astronomical Society, HEAD meeting, March 2008, Los
Angeles, CA.\\
(4) ``Radio Galaxies in the Chandra Era'' July 8-11, 2008, Cambridge, MA.\\ 
(5)  A summary of
our recent results and approaches has been presented in talks
at the recent Low Frequency Radio Universe conference (December 2008, Pune, India) by Raychaudhury.


\medskip
{\bf References}

\small

Athreya, R. 2008, ApJ, submitted;

Begelman, Blandford, Rees 1984, Rev. Mod. Phys. 56, 255;

Birzan, L., Rafferty, D.A., McNamara, B.R., Wise, M.W.; \& Nulsen, P.E.J.
2004, ApJ, 607, 800; 

Birzan, L., et al. 2008, ApJ, 686, 859;

Clarke, T.E., Sarazin, C.L. Blanton, E.L., Neumann, D.M., \& Kassim, N.E. 2005, ApJ, 625, 748;



Diehl, S., et al. 2008, ApJ, 687, 173; 

Dunn, R.J.H., Fabian, A.C., \& and Taylor, G.B. 2005, MNRAS, 364, 1343;

Fabian, A.C., Sanders, J.S., Allen et al., 2003, MNRAS, 344, L43;
 
Giacintucci, S., Vrtilek, J.M., Murgia, M., Raychaudhury, S., O'Sullivan, E.J.; Venturi, T., David, L.P., Mazzotta, P., Clarke, T.E.,
\& Athreya, R.M. 2008, ApJ, 682, 186;

McNamara, B.R., \& Nulsen, P.E.J., 2007, ARAA, 45, 117; 

Nulsen, P.E.J., McNamara, B.R., Wise, M.W., \& David, L.P. 2005,ApJ,
 628, 629; 

Scheuer, P 1974, MNRAS, 166, 513;

Wise et al. 2007,  ApJ 659, 1153





\begin{figure}

%\includegraphics[width=0.50\linewidth,bb=37 125 572 664,clip]{/data/jmv/jmv1/gmrt/2009jan_proposal/images/awm4_610_on_dss.ps}
%\includegraphics[width=0.50\linewidth,bb=37 125 572 664,clip]{/data/jmv/jmv1/gmrt/2009jan_proposal/images/ngc6269_610_on_dss.ps}
%\includegraphics[width=0.50\linewidth,lb=37 125 572 664,clip]{/data/jmv/jmv1/gmrt/2009jan_proposal/images/ngc3411_610_on_dss.ps}
%\includegraphics[width=0.50\linewidth,bb=37 141 572 651,clip]{/data/jmv/jmv1/gmrt/2009jan_proposal/images/n507_610_on_dss.ps}


\includegraphics[width=0.50\linewidth,bb=20 20 520 490,clip]{a2052-blanton.eps}
\includegraphics[width=0.50\linewidth,bb=75 55 572 490,clip]{a2597-clarke.ps}
\includegraphics[width=0.50\linewidth]{ms07-fig1.ps}
\includegraphics[width=0.50\linewidth]{hydra_rgb.ps}



{\footnotesize
Figure 1:  Four clusters from our sample. We propose to observe the
top 2 (plus another) in Cycle~16 
and the bottom two in Cycle~17 (for night time observations).
\vspace{0.05in}\\
{\it (upper left)}. Adaptive smoothed Chandra 
X-ray image of the core of the cluster Abell 2052, with
1.4 GHz VLA FIRST survey contours superposed  (Blanton et al. 2003).
The radio emission seems to fill the inner cavities and extensions go into the apparent cavities to the NW and SE of the cluster centre (latter is better seen
in the unpublished Cycle~6 observations). This cluster has been awarded a
500~Ms observation, which will yield a 3 times deeper X-ray image, in the current Chandra cycle.
\vspace{0.05in}\\
{\it (upper right)}. Chandra X-ray image 
of cluster Abell 2597 (residual with smoothed cluster removed),
shown with VLA 1.3 GHz contours in green and 5 GHz in cyan,
reveals two large ghost cavities (dark) coinciding with 
extension seen in the radio images. At GMRT low fruwncies, the true
extent of the radio plasma in the cavities will be revealed.
(Clarke et al. 2005)
\vspace{0.05in}\\
{\it (lower left)}. HST optical image of the MS0735.6+7421 cluster
superposed with the Chandra X-ray image (blue) and VLA 330 MHz radio 
image (red) shows an enormous pair of cavities, each roughly 200 kpc in diameter filled with radio-emitting plasma. The radio jets inflating the cavities
for $10^8$ yr have an average power of $\sim$10$^{46}$ erg/s. The cavities
are well outside the central galaxy. the supermassive BH grew by at leaast $3\times 10^8\,M_\odot$ during this outburst. (
McNamara et al. 2005 Nature, 433,
45, McNamara et al. 2008, arXiv:0811.3020).
\vspace{0.05in}\\
{\it (lower right)}. Composite colour image of 
the Hydra~A cluster that illustrates the close connection between
the observed X-ray cavity system (Chandra, blue) and the VLA
330~MHz (green) and 1.4 Ghz (yellow) radio emission.
(Wise et al. 2007)

}

\end{figure}


%\begin{figure}





\end{document}
