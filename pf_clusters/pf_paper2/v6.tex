\newcommand{\clnum}{\ensuremath{222}}
\newcommand{\accept}{\textit{ACCEPT}}
\newcommand{\kthr}{\ensuremath{K_{\mathrm{thresh}}}}
\newcommand{\fha}{\ensuremath{\kna = 13.9 \pm 4.9 \ent}}
\newcommand{\nfha}{\ensuremath{\kna = 128 \pm 48 \ent}}
\newcommand{\frad}{\ensuremath{\kna = 18.8 \pm 7.6 \ent}}
\newcommand{\nfrad}{\ensuremath{\kna = 112 \pm 45 \ent}}

%%%%%%%%%%
% Header %
%%%%%%%%%%

%\documentclass[12pt, preprint]{aastex}
\documentclass{emulateapj}
\usepackage{apjfonts,graphicx,here,lscape}
\usepackage{common} 
\bibliographystyle{apj}
\begin{document}
\title{An Entropy Threshold for Strong \halpha\ and Radio Emission\\in the Cores of Galaxy Clusters}
\author{
  Kenneth. W. Cavagnolo,
  Megan Donahue,
  G. Mark Voit,
  and Ming Sun}
\affil{Michigan State University, Department of Physics and Astronomy, East Lansing, MI 48824; cavagnolo@pa.msu.edu}
\shorttitle{The Entropy-Feedback Connection}
\shortauthors{K. W. Cavagnolo et al.}
%\journalinfo{Accepted to ApJ Letters}
%\accepted{Accepted May XX, 2008 }
%\submitted{Submitted May XX, 2008}

%%%%%%%%%%%%
% Abstract %
%%%%%%%%%%%%

\begin{abstract}
  Our \Chandra\ archival study of intracluster entropy in a sample of
  \clnum\ galaxy clusters shows that \halpha\ and radio emission from
  the brightest cluster galaxy are much more pronounced when the
  cluster's core gas entropy is $\la 30 \ent$. The prevalence of
  \halpha\ emission below this threshold indicates that it marks a
  dichotomy between clusters that can harbor multiphase gas and star
  formation in their cores and those that cannot. The fact that strong
  central radio emission also appears below this boundary suggests
  that AGN feedback turns on when the intracluster medium starts to
  condense, strengthening the case for AGN feedback as the mechanism
  that limits star formation in the Universe's most luminous galaxies.
\end{abstract}

\keywords{galaxies: clusters: general -- X-rays: active -- galaxies:
  conduction -- galaxies: cooling flows -- galaxies}

%%%%%%%%%%%%%%%%%%%%%%
\section{Introduction}
\label{sec:intro}
%%%%%%%%%%%%%%%%%%%%%%

In recent years the ``cooling flow problem'' has been the focus of
intense scrutiny as the solutions have broad impact on our theories of
galaxy formation (see \cite{cfreview} for a review). Current models
predict that the most massive galaxies in the Universe -- brightest
cluster galaxies (BCGs) -- should be bluer and more massive than
observations find, unless AGN feedback intervenes to stop late-time
star formation \citep{bower06, croton06, saro06}. X-ray observations
of galaxy clusters have given this hypothesis considerable
traction. From the properties of X-ray cavities in the intracluster
medium (ICM), \cite{birzan04} concluded that AGN feedback provides the
necessary energy to retard cooling in the cores of clusters (see
\citealt{mcnamrev} for a review). This result suggests that, under the
right conditions, AGN are capable of quenching star formation by
heating the surrounding ICM.

If AGN feedback is indeed responsible for regulating star formation in
cluster cores, then the radio and star-forming properties of galaxy
clusters should be related to the distribution of ICM specific
entropy\footnote{In this paper we quantify entropy in terms of the
  adiabatic constant $K = kTn_e^{-/3}$.}. In previous observational
work (see \citealt{radioquiet, d06, accept}), we have focused on ICM
entropy as a means for understanding the cooling and heating processes
in clusters because it is a more fundamental property of the ICM than
temperature or density alone \citep{voitbryan,voitreview}. ICM
temperature mainly reflects the depth and shape of the dark matter
potential well, while entropy depends more directly on the history of
heating and cooling within the cluster and determines the density
distribution of gas within that potential.

We have therefore undertaken a large \Chandra\ archival project to
study how the entropy structure of clusters correlates with other
cluster properties. \cite{accept} presents the radial entropy profiles
we have measured for a sample of \clnum\ clusters taken from the
\Chandra\ Data Archive. We have named this project the Archive of
Chandra Cluster Entropy Profile Tables, or \accept\ for short. To
characterize the ICM entropy distributions of the clusters, we fit the
equation $K(r) = K_0 +K_{100}(r/100 \kpc)^{\alpha}$ to each entropy
profile. In this equation, \khun\ is the normalization of the
power-law component at 100 kpc and we refer to \kna\ as the central
entropy. Bear in mind, however, that \kna\ is not necessarily the
minimum core entropy or the entropy at $r=0$, nor is it the gas
entropy which would be measured immediately around the AGN or in a BCG
X-ray coronae. Instead, \kna\ represents the typical excess of core
entropy above the best fitting power-law found at larger
radii. \cite{accept} shows that \kna\ is non-zero for almost all
clusters in our sample.

In this letter we present the results of exploring the relationship
between the expected by-products of cooling, \eg\ \halpha\ emission,
star formation, and AGN activity, and the \kna\ values of clusters in
our survey. To determine the activity level of feedback in cluster
cores, we selected two readily available observables: \halpha\ and
radio emission. We have found that there is a critical entropy level
below which \halpha\ and radio emission are often present, while above
this threshold these emission sources are much fainter and in most
cases undetected. Our results suggest that the formation of thermal
instabilities in the ICM and initiation of processes such as star
formation and AGN activity are closely connected to core entropy, and
we suspect that the sharp entropy threshold we have found arises from
thermal conduction (Voit et al. 2008, ApJL, in press).

This letter proceeds in the following manner: In \S\ref{sec:data} we
cover the basics of our data analysis. The
entropy-\halpha\ relationship is discussed in \S\ref{sec:sf}, while
the entropy-radio relationship is discussed in \S\ref{sec:agn}. A
brief summary is provided in \S\ref{sec:diss}. For this letter we
have assumed a flat \LCDM\ Universe with cosmogony $\OM=0.3$,
$\OL=0.7$, and $\Hn=70\km\ps\pMpc$. All uncertainties are 90\%
confidence.

%%%%%%%%%%%%%%%%%%%%%%%
\section{Data Analysis}
\label{sec:data}
%%%%%%%%%%%%%%%%%%%%%%%

This section briefly describes our data reduction and methods for
producing entropy profiles. More thorough explanations are given in
\cite{d06}, \cite{accept}, and \cite{xrayband}.

%%%%%%%%%%%%%%%%%%%%
\subsection{X-ray}
\label{sec:xray}
%%%%%%%%%%%%%%%%%%%%

X-ray data was taken from publicly available observations in the
\Chandra\ Data Archive. Following standard \Ciao\ reduction
techniques\footnote{http://cxc.harvard.edu/ciao/guides/}, data was
reprocessed using \Ciao\ 3.4.1 and \Caldb\ 3.4.0, resulting in point
source and flare clean events files at level-2. Entropy profiles were
derived from the radial ICM temperature and electron density profiles.

Radial temperature profiles were created by dividing each cluster into
concentric annuli with the requirement of at least three annuli
containing a minimum of 2500 counts each. Source spectra were
extracted from these annuli, while corresponding background spectra
were extracted from blank-sky backgrounds tailored to match each
observation. Each blank-sky background was corrected to account for
variation of the hard-particle background, while spatial variation of
the soft-galactic background was accounted for through addition of a
fixed background component during spectral fitting. Weighted responses
which account for spatial variations of the CCD calibration were also
created for each observation. Spectra were then fit over the energy
range 0.7-7.0 keV in \xspec\ 11.3.2ag \citep{xspec} using a
single-component absorbed thermal model.

Radial electron density profiles were created using surface brightness
profiles and spectroscopic information. Exposure-corrected,
background-subtracted, point-source-clean surface brightness profiles
were extracted from $5\arcsec$ concentric annular bins over the energy
range 0.7-2.0 keV. In conjunction with the spectroscopic normalization
and 0.7-2.0 keV count rate, surface brightness was converted to
electron density using the deprojection technique of
\cite{kriss83}. Errors were estimated using 5000 Monte Carlo
realizations of the surface brightness profile.

A radial entropy profile for each cluster was then produced from the
temperature and electron density profiles. The entropy profiles were
fit with a simple model which is a power-law at large radii and
approaches a constant value, \kna, at small radii (see
\S\ref{sec:intro} for the equation). We define central entropy as
\kna\ from the best-fit model.

%%%%%%%%%%%%%%%%%%%%
\subsection{\halpha}
\label{sec:ha}
%%%%%%%%%%%%%%%%%%%%

One goal of our project was to determine if ICM entropy is connected
to processes like star formation. Here we do not directly measure star
formation but instead use \halpha, which is usually a strong indicator
of ongoing star formation in galaxies \citep{kennicuttrelation}. It is
possible that some of the \halpha\ emission from BCGs is not produced
by star formation \citep{begelman90, sparks04, rusz08,
 ferland08}. Nevertheless, \halpha\ emission unambiguously indicates
the presence of $\sim 10^4$ K gas in the cluster core and therefore
the presence of a multiphase intracluster medium that could
potentially form stars.

Our \halpha\ values have been gathered from several sources, most
notably \cite{crawford99}. Additional sources of data are M. Donahue's
observations taken at Las Campanas and Palomar (reported in
\citealt{accept}), \cite{heckman89}, \cite{dsg92}, \cite{lawrence96},
\cite{1996AJ....112.1390V}, \cite{white97},
\cite{2005MNRAS.363..216C}, and \cite{ir_quillen}. We have
recalculated the \halpha\ measurements from these sources assuming a
\LCDM\ cosmological model to place them on a more equal footing.
However, they were made with a variety of apertures and in many cases
may not reflect the full \halpha\ luminosity of the BCG. Thus, the
exact levels of \lha\ are not important for the purposes of this
letter and we suggest the reader interpret the \lha\ values as a
binary indicator of multiphase gas: either \halpha\ emission and cool
gas are present or they are not.

%%%%%%%%%%%%%%%%%%
\subsection{Radio}
\label{sec:radio}
%%%%%%%%%%%%%%%%%%

Another goal of this work was to explore the relationship between ICM
entropy and AGN activity. It has long been known that BCGs are more
likely to host radio-loud AGN than other cluster galaxies
\citep{burns81, valentijn83, burns90}. Thus, we chose to look for
radio emission from the BCG of each \accept\ cluster as a sign of AGN
activity.

To make the radio measurements, we have taken advantage of the nearly
all-sky flux-limited coverage of the NRAO VLA Sky Survey (NVSS,
\citealt{nvss}) and Sydney University Molonglo Sky Survey (SUMSS,
\citealt{sumss1, sumss2}). NVSS is a continuum survey at 1.4 GHz of
the entire sky north of $\delta = -40\degr$, while SUMSS is a
continuum survey at 843 MHz of the entire sky south of $\delta =
-30\degr$. The completeness limit of NVSS is $\approx 2.5$ mJy and for
SUMSS it is $\approx 10$ mJy when $\delta > -50\degr$ or $\approx 6$
mJy when $\delta \leq -50\degr$. The NVSS positional uncertainty for
both right ascension and declination is $\la 1''$ for sources brighter
than 15 mJy, and $\approx 7''$ at the survey detection limit
\citep{nvss}. At $z=0.2$, these uncertainties equal distances on the
sky of $\sim3-20$ kpc. For SUMSS, the positional uncertainty is $\la
2''$ for sources brighter than 20 mJy, and is always less than $10''$
\citep{sumss1,sumss2}. The distance at $z=0.2$ associated with these
uncertainties is $\sim6-30$ kpc. We calculate the radio power for each
radio source using the standard relation $\nu L_{\nu} = 4 \pi D_L^2
S_{\nu} f_0$ where $S_{\nu}$ is the 1.4 GHz or 843 MHz flux from NVSS
or SUMSS, $D_L$ is the luminosity distance, and $f_0$ is the central
beam frequency of the observations.

Radio sources were found using two methods. The first method was to
search for sources within a fixed angular distance of $20\arcsec$
around the cluster X-ray peak. The probability of randomly finding a
radio source within an aperture of $20\arcsec$ is exceedingly low ($<
0.004$ for NVSS). Thus, in \clnum\ total field searches, we expect to
find no more than one spurious source. The second method involved
searching for sources within 20 kpc of the cluster X-ray peak. At $z
\approx 0.051$, $1\arcsec$ equals 1 kpc, thus for clusters at $z \ga
0.05$ the 20 kpc aperture is smaller than the $20\arcsec$ aperture and
the likelihood of finding a spurious source gets smaller. Both methods
produce nearly the same list of radio sources with the differences
being the very large, extended lobes of low-redshift radio sources
such as Hydra A.

To make a spatial and morphological assessment of the radio emission's
origins, \ie\ determining if the radio emission is associated with the
BCG, high angular resolution is necessary. However, NVSS and SUMSS are
low-resolution surveys with FWHM at $\approx 45\arcsec$. We therefore
cannot distinguish between ghost cavities/relics, extended lobes,
point sources, re-accelerated regions, or if the emission is coming
from a galaxy very near the BCG or a background/foreground source. We
have handled this complication by visually inspecting each radio
source in relation to the optical (using DSS I/II) and infrared (using
2MASS) emission of the BCG. We have used this method to establish that
the radio emission is most likely coming from the BCG. When available,
high resolution data from VLA FIRST\footnote{Faint Images of the Radio
 Sky at Twenty cm; http://sundog.stsci.edu} was added to the visual
inspection.

%%%%%%%%%%%%%%%%%%%%%%%%%%%%%%%%%%%%%%%%%%%%%%%
\section{\halpha\ Emission and Central Entropy}
\label{sec:sf}
%%%%%%%%%%%%%%%%%%%%%%%%%%%%%%%%%%%%%%%%%%%%%%%

\begin{figure}
 \begin{center}
    \includegraphics*[width=\columnwidth, trim=28mm 7mm 40mm 17mm, clip]{ha}
    \caption{Central entropy versus \halpha\ luminosity. Orange
      circles represent detections, black circles are upper limits,
      and blue boxes with inset red stars or orange circles are
      peculiar clusters which do not adhere to the observed trend (see
      text). The vertical dashed line marks $\kna = 30 \ent$. Note
      the presence of a sharp \halpha\ detection dichotomy beginning
      at $\kna \la 30 \ent$.}
    \label{fig:ha}
  \end{center}
\end{figure}

Of the \clnum\ clusters in \accept, we located \halpha\ observations
from the literature for 110 clusters. Of those 110, \halpha\ was
detected in 46, while the remaining 64 have upper limits. The mean
central entropy for clusters with detections is \fha, and for clusters
with only upper-limits \nfha.

In Figure \ref{fig:ha} central entropy is plotted versus
\halpha\ luminosity. One can immediately see the dichotomy between
clusters with and without \halpha\ emission. If a cluster has a
central entropy $\la 30 \ent$ then \halpha\ emission is usually
``on'', while above this threshold the emission is predominantly
``off''. For brevity we refer to this threshold as
\kthr\ hereafter. The cluster above \kthr\ which has \halpha\ emission
(blue box with orange dot) is Zwicky 2701 ($\kna = 39.7 \pm 3.9
\ent$). There are also clusters below \kthr\ without \halpha\ emission
(blue boxes with red stars): A2029, A2107, A2151\footnote{Not plotted
  because \kna\ is statistically consistent with zero}, EXO 0422-086,
and RBS 533. These five clusters are clearly exceptions to the much
larger trend. The mean and dispersion of the redshifts for clusters
with and without \halpha\ are nearly identical, $z = 0.124 \pm 0.106$
and $z = 0.134 \pm 0.084$ respectively, and applying a redshift cut
(\ie\ $z = 0-0.15$ or $z = 0.15-0.3$) does not change the
\kna-\halpha\ dichotomy. Most important to note is that changes in the
\halpha\ luminosities because of aperture effects will move points up
or down in Figure \ref{fig:ha}, while mobility along the \kna\ axis is
minimal. Qualitatively, the correlation between low central entropy
and presence of \halpha\ emission is very robust.

The clusters with \halpha\ detections are typically between $10-30
\ent$, have short central cooling times ($<$ 1 Gyr), and under older
nomenclature would be classified as ``cooling flow'' clusters.  It has
long been known that star formation and associated \halpha\ nebulosity
appear only in cluster cores with cooling times less than a Hubble
time \citep{hu85, johnstone87, mcnamara89, voit97,cardiel98}. However,
our results suggest that the central cooling time must be at least a
factor of 10 smaller than a Hubble time for these manifestations of
cooling and star formation to appear.  It is also very interesting
that the characteristic entropy threshold for strong \halpha\ emission
is so sharp. \cite{conduction} have recently proposed electron thermal
conduction may be responsible for setting this threshold. This
hypothesis has received further support from the theoretical work of
\cite{2008arXiv0804.3823G} showing that thermal conduction can
stabilize non-cool core clusters against the formation of thermal
instabilities, and that AGN feedback may be required to limit star
formation when conduction is insufficient.

%%%%%%%%%%%%%%%%%%%%%%%%%%%%%%%%%%%%%%%%%%%
\section{Radio Sources and Central Entropy}
\label{sec:agn}
%%%%%%%%%%%%%%%%%%%%%%%%%%%%%%%%%%%%%%%%%%%

\begin{figure}
  \begin{center}
    \includegraphics*[width=\columnwidth, trim=28mm 7mm 40mm 17mm,
      clip]{radio_zcut}
    \caption{BCG radio power vs. \kna\ for clusters at $z <
      0.2$. Orange points represent detections, black points are
      non-detection upper-limits, and blue boxes with inset red stars
      or orange points are peculiar clusters which do not adhere to
      the observed trend (see text). Circles are for NVSS observations
      and squares are for SUMSS observations. The vertical dashed line
      marks $\kna = 30 \ent$. The radio sources show the same trend as
      \halpha: bright radio emission is preferentially ``on'' for
      $\kna \la 30 \ent$.}
    \label{fig:radzcut}
  \end{center}
\end{figure}

Of the \clnum\ clusters in \accept, 100 have radio-source detections
with a mean $\kna$ of $23.7 \pm 9.3 \ent$, while the other 122
clusters with only upper limits have a mean $\kna$ of $134 \pm 52
\ent$. NVSS and SUMSS are low resolution surveys with FWHM at $\approx
45\arcsec$ which at $\red = 0.2$ is $\approx 150\kpc$. This scale is
larger than the size of a typical cluster cooling region and makes it
difficult to determine absolutely that the radio emission is
associated with the BCG. We therefore focus only on clusters at $z <
0.2$. After the redshift cut, 135 clusters remain -- 64 with radio
detections (mean \frad) and 71 without (mean \nfrad).

In Figure \ref{fig:radzcut} we have plotted radio power versus \kna.
The obvious dichotomy seen in the \halpha\ measures and characterized
by \kthr, is also present in the radio. Clusters with $\nu L_{\nu}
\gtrsim 10^{40} \ergps$ generally have $\kna \la \kthr$. This trend
was first evident in \citet{radioquiet} and suggests that AGN activity
in BCGs, while not exclusively limited to clusters with low core
entropy, is much more likely to be found in clusters which have a core
entropy less than \kthr. That star formation and AGN activity are
subject to the same entropy threshold suggests the mechanism which
promotes or initiates one is also involved in the activation of the
other. If the entropy of the hot gas in the vicinity of the AGN is
correlated with \kna, then the lack of correlation between radio power
and \kna\ below the $30 \ent$ threshold suggests that cold-mode
accretion \citep{pizzolato05, hardcastle07} may be the dominant method
of fueling AGN in BCGs.

We have again highlighted a few exceptions to the general trend in
Figure \ref{fig:radzcut}: clusters below \kthr\ without radio sources
(blue boxes with red stars) and clusters above \kthr\ with radio
sources (blue boxes with orange dots). The peculiar clusters below
\kthr\ are A133, A539, A1204, A2107, A2556, AWM7, ESO 5520200, MKW4,
MS J0440.5+0204, and MS J1157.3+5531. The peculiar clusters above
\kthr\ are 2PIGG J0011.5-2850, A193, A586, A2063, A2147, A2244, A3558,
A4038, and RBS 461. In addition, three clusters, A2151, AS405, MS
0116.3-0115, are not plotted as their best-fit \kna\ are statistically
consistent with zero. Finding a few clusters in our sample without
radio sources where we expect to find them is not surprising given
that AGN feedback could be episodic. However, the clusters above
\kthr\ with a radio-loud source are interesting, and may be special
cases of BCGs with embedded coronae. \cite{coronae} extensively
studied coronae and found they are like ``mini-cooling cores'' with
low temperatures and high densities. Coronae are a low-entropy
environment isolated from the high-entropy ICM and may provide the
conditions necessary for gas cooling to proceed. Indeed, some of these
peculiar clusters show indications that a very compact ($r \la 5\kpc$)
X-ray source is associated with the BCG \citep{accept}.

%%%%%%%%%%%%%%%%%
\section{Summary}
\label{sec:diss}
%%%%%%%%%%%%%%%%%

We have presented a comparison of ICM central entropy values and
measures of BCG \halpha\ and radio emission for a \Chandra\ archival
sample of galaxy clusters. We find that below a characteristic central
entropy threshold of $\kna \approx 30 \ent$, \halpha\ and bright radio
emission are more likely to be detected, while above this threshold
\halpha\ is not detected and radio emission, if detected at all, is
significantly fainter. The mean \kna\ for clusters with and without
\halpha\ detections are \fha\ and \nfha, respectively. For clusters at
$z < 0.2$ with BCG radio emission the mean \frad, while for BCGs with
only upper limits, the mean \nfrad. While other mechanisms can produce
\halpha\ or radio emission besides star formation and AGN, if one
assumes the \halpha\ and radio emission are coming from these two
feedback sources, then our results suggest the development of
multiphase gas in cluster cores (which can fuel both star formation
and AGN) is strongly coupled to ICM entropy.

\acknowledgements
We were supported in this work through \Chandra\ grants AR-6016X,
AR-4017A, and NASA LTSA program NNG-05GD82G. The CXC is operated by
the SAO for and on behalf of NASA under contract NAS8-03060.

%%%%%%%%%%%%%%%%
% Bibliography %
%%%%%%%%%%%%%%%%

\bibliography{cavagnolo}

\end{document}
