\documentclass[11pt]{article}
\setlength{\topmargin}{-.3in}
\setlength{\oddsidemargin}{-0.1in}
\setlength{\evensidemargin}{-0.1in}
\setlength{\textwidth}{6.7in}
\setlength{\headheight}{0in}
\setlength{\headsep}{0in}
\setlength{\topskip}{0.5in}
\setlength{\textheight}{9.25in}
\setlength{\parindent}{0.0in}
\setlength{\parskip}{1em}
\usepackage{common,graphicx,hyperref,epsfig}

\pagestyle{empty}

\begin{document}

To the editor:

Below is our reply to the Referee's Report of ApJL \#22961. The
referee's comments are in quotes and {\it{italicized}}, while our
replies are in regular font. We have found the referee's comments to
be very helpful in making the focus and discussion of our paper more
concise and thorough.

TITLE: An Entropy Threshold for Strong \halpha\ and Radio Emission in
the Cores of Galaxy Clusters

AUTHORS: Kenneth. W. Cavagnolo, Megan Donahue, G. Mark Voit, and Ming
Sun

---------------------------------------------------------------------

{\it{``My only concern about the methods used is that the
    parametrization of entropy used in the paper is at least partly a
    measure of the *shape* of the entropy profile, rather than its
    normalisation. Do the results on Halpha and radio associations
    hold up if you consider the absolute entropy level at a fixed
    physical radius and/or a fixed scaled radius?''}}

In the upcoming paper which presents all of our entropy profiles, we
show that the average curvature of our entropy profiles and the number
of radial bins in a profile are uncorrelated with the best-fit
\kna\ values which, suggests there is no systematic degeneracy between
profile shape and \kna.

As requested, we have calculated the gas entropy at a fixed radius of
$r=12$ kpc (denoted $K_{12}$) and find that the \halpha\ dichotomy is
still very distinct (left panel of figure below). While the scatter in
the \radpow-$K_{12}$ relation is larger than for \radpow-\kna, we
still find the behavior to be qualitatively the same in that powerful
radio sources ($\radpow \ga 10^{40} \ergps$) are preferentially found
in clusters with low $K_{12}$. We have added a sentence to \S5 which
notes we have looked at entropy of a fixed radius.

Because of page limitations we cannot include this new figure in the
letter.

\begin{figure}[h]
  \begin{minipage}[t]{0.5\linewidth}
    \centering
    \includegraphics*[width=\textwidth, trim=28mm 8mm 35mm 18mm, clip]{resp_f1}
  \end{minipage}
  \hspace{0.25cm}
  \begin{minipage}[t]{0.5\linewidth}
    \centering
    \includegraphics*[width=\textwidth, trim=28mm 8mm 35mm 18mm, clip]{resp_f2}
  \end{minipage}
\end{figure}

---------------------------------------------------------------------

{\it{``I disagree with the statement in the second paragraph of
    Section 4 that the lack of a correlation between radio power and
    $K_0$ supports a cold-mode accretion model. Radio power is not a
    good proxy for total mechanical output by the AGN, and as you
    later argue in the context of some outlying objects, the radio
    outbursts may be episodic. If the current radio power can be very
    low, despite the cluster being in the low-entropy state, then
    clearly in the authors' model the current radio power is not
    directly tracing the AGN energy input. It's certainly worth
    commenting on the lack of correlation between $K_0$ and radio
    power, but some caveats should be added.''}}

This is an excellent point, and we have clarified in \S4, paragraph
three that our speculation regarding why there is no correlation
between \kna\ and radio power is subject to assumptions which may not
be correct. We have left this paragraph of text in however because we
want to highlight that there is a lack of correlation between $K_0$
and \radpow, even though it may a coincidence unrelated to cold-mode
accretion.

---------------------------------------------------------------------

{\it{``Are the high $K_0$ clusters with radio sources discussed at the
    end of Section 4 indistinguishable from those without, e.g. in
    terms of cluster dynamical state and galaxy populations? It is
    likely that some cluster radio-galaxy populations are trigged
    during cluster mergers and/or galaxy infall (e.g. narrow-angle
    tails), and these sources are probably independent of the central
    entropy. Although NATs aren't usually found in cluster centres,
    it's plausible that a small fraction of sources close to cluster
    centres may be triggered by other means (since we know such means
    must exist as a significant fraction of the radio-galaxy
    populations are not found in the centres of rich clusters). The
    presence of a dense corona is one possibility, but there may be
    others."}}

At this time, we do not have an answer to this question. This is a
very interesting question however, one we have previously been
investigating for the very reason that X-ray coronae are but one of a
few possible answers to what these compact sources are.

---------------------------------------------------------------------

\end{document}
