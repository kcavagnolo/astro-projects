\newcommand{\clnum}{\ensuremath{215}}
\newcommand{\accept}{\textit{ACCEPT}}
\newcommand{\kthr}{\ensuremath{K_{\mathrm{thresh}}}}

%%%%%%%%%%
% Header %
%%%%%%%%%%

% \documentclass[12pt, preprint]{aastex}
\documentclass{emulateapj}
\usepackage{apjfonts,graphicx,here,lscape}
\usepackage{common} 
\bibliographystyle{apj}

\begin{document}
\title{Star Formation, AGN, and the Entropy-Feedback\\
  Connection in Galaxy Clusters}
\author{
  Kenneth W. Cavagnolo\altaffilmark{1,2},
  Megan Donahue\altaffilmark{1},
  G. Mark Voit\altaffilmark{1}, 
  and Ming Sun\altaffilmark{1}}
\altaffiltext{1}{Michigan State University, Department of Physics and
  Astronomy, BPS Building, East Lansing, MI 48824}
\altaffiltext{2}{cavagnolo@pa.msu.edu}
\shorttitle{Entropy, Star Formation, and AGN}
\shortauthors{K. W. Cavagnolo et al.}
\journalinfo{Submitted to ApJ Letters}
%\accepted{ }
\submitted{April 10, 2008}

%%%%%%%%%%%%
% Abstract %
%%%%%%%%%%%%

\begin{abstract}
  We present the first results from our study of the entropy
  distribution for a \Chandra\ archival sample of \clnum\ galaxy
  clusters. After parameterizing the central entropy, \kna\, of each
  cluster as the value at which the entropy distribution departs from
  a power-law, we make comparisons with indicators of feedback such as
  \halpha\ emission and radio-loud sources. Interpreting these
  indicators to mean that star formation and AGN are active in the BCG
  of the cluster, we find that for almost all clusters below an
  entropy threshold of $\approx 30 \ent$ feedback is active, while
  above this threshold the feedback is not detectable. This entropy
  threshold is coincident with the level at which heating from thermal
  electron conduction can offset radiative cooling and thereby prevent
  multi-phase gas from forming.
\end{abstract}

\keywords{galaxies: clusters: general -- X-rays: active -- galaxies:
  conduction -- galaxies: cooling flows -- galaxies}

%%%%%%%%%%%%%%%%%%%%%%%%%
\section{Introduction}
\label{sec:intro}
%%%%%%%%%%%%%%%%%%%%%%%%%

In recent years the ``cooling flow problem'' has been the focus of
intense scrutiny as the solutions have broad impact in the area of
galaxy formation. The adiabatic model of hierarchical structure
formation predicts an over-production of dwarf and massive galaxy
haloes. Brightest cluster galaxies (BCGs) -- the most massive galaxies
in the Universe -- are also predicted to be too blue and too massive.
Compounding the issue is that the star formation rates in massive
galaxies were larger at higher redshifts \citep{1996AJ....112..839C,
  2005ApJ...619L.135J}, contradicting model predictions and resulting
in the observation of present-day ellipticals being ``red and dead''.
There is also the so-called cosmic down-sizing problem
\citep{1996AJ....112..839C}, where massive galaxies are observed to
form earlier than hierarchical galaxy formation models predict. There
is intense on-going study of the hypothesis that these problems with
the existing galaxy formation models are related to galactic feedback,
\ie\ in the form of active galactic nuclei (AGN) and supernovae. This
hypothesis has gained considerable traction in recent years both
theoretically \citep{bower06, croton06} and observationally
\citep{peterson2003, mcnamrev}.

A fruitful area of work which has sought to explain the differences
between galaxy formation models and observations involves studying the
interaction of feedback sources with the hot atmosphere of their host
galaxy cluster.  Using the properties of X-ray cavities in the
intracluster medium (ICM) in combination with the energetics of the
BCG radio source, \cite{birzan04} demonstrated that AGN do provide the
necessary energy to retard cooling in the cores of clusters. This in
turn indicates AGN are capable of quenching star formation in the BCG
by heating the surrouding ICM and preventing gas from condensing onto
the galaxy and forming stars.

In previous observational work (\citealt{radioquiet, dhc06, accept2})
we have focused on ICM entropy as a means for understanding the
cooling and heating processes in clusters. Entropy is a more
fundamental property of the ICM than temperature or density alone
\citep{voitbryan,voitreview}. ICM temperature mainly reflects the
depth and shape of the dark matter potential well, while density
represents how much the well can compress the ICM.  Entropy on the
other hand determines the density of an isobaric gas parcel, and only
gains or losses of heat energy can change the entropy. Thus, study of
the ICM entropy distribution is an investigation of the cluster
thermal history.

In the second paper of this series, \cite{accept2} -- hereafter Paper
II, we present the radial entropy profiles for a sample of \clnum\
clusters taken from the \Chandra\ Data Archive. We have named this
project the Archive of Chandra Cluster Entropy Profile Tables, or
\accept\ for short. There are two results in Paper II which are
relevant to this letter: 1) a nearly universal departure of the radial
entropy distribution in cluster cores from a self-similar power-law,
and 2) a bimodal central entropy distribution.  We define central
entropy as the parametric scale, \kna\ in the equation $K(r) = K_0
+K_{100}(r/100 \kpc)^{\alpha}$, at which the entropy distribution
departs from a power-law. This is not always the minimum core entropy,
nor is it the gas entropy which would be measured immediately around
the AGN, it is a measure of the average entropy within the cluster
core -- a region which can be anywhere from 10 to 100 kpc. In this
letter we present the results of exploring the relationship between
the expected by-products of cooling, \eg\ star formation and
radio-loud AGN, to the central entropy value of the clusters in our
sample.

To determine the activity level of feedback in cluster cores we have
selected two readily available observables: \halpha\ emission
resulting from star formation and radio-loud sources associated with
AGN. Using these observables as a window on the processes operating in
the core of clusters, we find there exists a critical entropy level
below which star formation and AGN are almost always present, while
above this threshold feedback is diminished and in most cases has
ceased completely. Through a coincidence of scaling, this entropy
threshold is associated with the limit at which conductive stability
is established in a cluster core.  In the light of these results, we
suggest that low entropy ICM environments, along with electron thermal
conduction, play a vital role in determining the state of feedback
activity within cluster cores.

This letter proceeds in the following fashion: In \S\ref{sec:data} we
cover the basics of our data analysis. The entropy-star formation
relationship is discussed in \S\ref{sec:sf}, while the entropy-AGN
relationship is discussed in \S\ref{sec:agn}. Discussion of results
and conclusions are presented in \S\ref{sec:diss}.  For this letter we
have assumed a flat \LCDM\ Universe with cosmogony $\OM=0.3$,
$\OL=0.7$, and $\Hn=70\km\ps\pMpc$. All uncertainties are 90\%
confidence.

%%%%%%%%%%%%%%%%%%%%%%%%%%
\section{Data Analysis}
\label{sec:data}
%%%%%%%%%%%%%%%%%%%%%%%%%%

In this letter we only briefly describe our data reduction and methods
for producing entropy profiles and refer interested readers to the
more thorough explanations in \cite{dhc06}, \cite{xrayband}, and Paper
II.

%%%%%%%%%%%%%%%%%%%%
\subsection{X-ray}
\label{sec:xray}
%%%%%%%%%%%%%%%%%%%%

X-ray data was taken from publicly available observations in the
\Chandra\ Data Archive. Following standard \Ciao\ reduction
techniques\footnote{http://cxc.harvard.edu/ciao/guides/}, data was
reprocessed using \Ciao\ 3.4.1 and \Caldb\ 3.4.0, resulting in point
source and flare clean events files at level-2. Entropy profiles were
derived using the relation $K(r) = T(r) \nelec(r)^{-2/3}$ where $T(r)$
is the radial ICM temperature and $\nelec(r)$ is the radial ICM
electron density.

Radial temperature profiles were created by spatially dividing each
cluster into concentric annuli with the minimum requirement of three
annuli containing 2500 counts. Source spectra were extracted from
these annuli, while corresponding background spectra were extracted
from blank-sky backgrounds tailored to match each observation.  Each
background is corrected to account for variation of the hard-particle
and soft-galactic backgrounds. Weighted responses which account for
spatial variations of the CCD calibration were also created for each
observation. Spectra were then fit over the energy range 0.7-7.0 keV
in \xspec\ 11.3.2ag \citep{xspec} using a single-component absorbed
thermal model.

Exposure-corrected, background-subtracted, point-source-clean surface
brightness profiles were extracted from $5\arcsec$ concentric annular
bins over the energy range 0.7-2.0 keV. In conjunction with the
spectroscopic normalization and 0.7-2.0 keV count rate, surface
brightness was converted to electron density using the deprojection
technique of \cite{kriss83}. Errors were estimated using 5000 Monte
Carlo realizations of the surface brightness profiles.

A radial entropy profile for each cluster was then produced from the
temperature and electron density profiles. The entropy profiles were
fit with a hybrid model which is a power-law at large radii and
flattens at small radii:
\begin{equation}
  K(r) = \kna + \khun\ \left(\frac{r}{100 \kpc}\right)^{\alpha} \nonumber \\
\end{equation}
where \khun\ is a normalization of the power-law component at 100 kpc
and \kna\ describes the flattening of the profile in the core region.
In this letter we define \kna\ from the best-fit model as the central
entropy. In Paper II we demonstrate that the best-fit \kna\ and
observational \kna\ do not significantly differ for most clusters. As
noted \S\ref{sec:intro}, \kna\ is not the minimum gas entropy found in
the core, nor does it represent an absolute flattening of the profile.
Rather \kna\ is a parametric representation of where the entropy
distribution departs from a power-law, and is a characterization of
the average entropy of the core.

%%%%%%%%%%%%%%%%%%%%%%
\subsection{\halpha}
\label{sec:ha}
%%%%%%%%%%%%%%%%%%%%%%

\halpha\ emission has proven to be a useful indicator of ongoing star
formation \citep{johnstone87, mcnamara89, voit97}.  Therefore, to
check for star formation in the BCG of our sample clusters, we have
taken \halpha\ values from several sources, most notably
\cite{crawford99}.  Alternative sources are M.  Donahue's work at
Carnegie Observatory, \cite{ir_quillen}, \cite{heckman89},
\cite{lawrence96}, \cite{white97}, and \cite{2005MNRAS.363..216C}. To
place all the \halpha\ measurements on equal footing, we have rolled
back from the cosmogony used in each study cited and re-calculated
\halpha\ luminosities using our assumed \LCDM\ values.

The \lha\ values listed in this paper are not meant for strict
interpretation of star formation rates, \eg\ using the Kennicutt
relation \citep{kennicuttrelation}. The individual values of \lha\ are
not important for the purposes of this paper and we suggest the reader
interpret the \lha\ values as an analog indicator of star formation:
either it is off or on.

%%%%%%%%%%%%%%%%%%
\subsection{Radio}
\label{sec:radio}
%%%%%%%%%%%%%%%%%%

We take advantage of the nearly all-sky flux-limited coverage of the
NRAO VLA Sky Survey (NVSS, \citealt{1998AJ....115.1693C}) and Sydney
University Molonglo Sky Survey (SUMSS, \citealt{1999AJ....117.1578B,
  2003MNRAS.342.1117M}) to look for radio emission from the central
region of each cluster in our sample. NVSS is a continuum survey at
1.4 GHz of the entire sky north of $\delta = -40\degr$, while SUMSS is
a continuum survey at 843 MHz of the entire sky south of $\delta =
-30\degr$. The completeness limit of NVSS is $\approx 2.5$ mJy and for
SUMSS it is $\approx 10$ mJy when $\delta > -50\degr$ or $\approx 6$
mJy when $\delta \leq -50\degr$. Both surveys probe very low radio
luminosities and are excellent for our purposes. We calculate the
radio power for each radio source using the standard relation $\nu
L_{\nu} = 4 \pi D_L^2 S_{\nu} f_0$ where $S_{\nu}$ is the 1.4 GHz or
843 MHZ flux from NVSS or SUMSS, $D_L$ is the luminosity distance in
cm, and $f_0$ is the central beam frequency of the survey.

Radio sources were found using two methods. The first method was to
search for sources within a fixed angular distance, $20''$, of the
cluster X-ray peak. The probability of randomly finding a radio source
within an aperture of $20''$ is exceedingly low ($< 0.004$ for NVSS),
thus in a sample of $\approx 200$ clusters we expect no more than one
false match. The second method involved searching within $20\kpc$ of
the X-ray peak for sources. At $z \approx 0.051$, $1''$ equals 1 \kpc,
thus for clusters at $z \ga 0.05$ the 20 kpc aperture is smaller than
the $20''$ aperture and the likelihood of a false match is even
smaller. Both methods produce nearly the same list of radio sources
with the differences being the very extended lobes of low-redshift
radio sources such as Hydra A,

To make a spatial and morphological assessment of the radio emission's
origins, \ie\ determining if the radio emission is associated with the
BCG, high angular resolution is necessary. However, NVSS and SUMSS are
low-resolution surveys with FWHM at $\approx 45\arcsec$. We therefore
cannot distinguish between ghost cavities/relics, extended lobes,
point sources, re-accelerated regions, or if the emission is coming
from a galaxy very near the BCG or a background/foreground source. We
have handled this complication by visually inspecting each radio
source in relation to the optical (using DSS I/II) and infrared (using
2MASS) emission of the BCG. We have used this method to establish that
the radio emission is most likely coming from the BCG. When available,
high resolution data from VLA FIRST\footnote{Faint Images of the Radio
 Sky at Twenty cm; http://sundog.stsci.edu} was added to the
visual inspection.

%A comparison of the NVSS search results for each method shows that the
%$20\kpc$ search radius rarely (frequency $< 2\%$) finds fewer or more
%sources than the $20''$ aperture. Using higher-resolution images from
%VLA First, we examined For the cases where a source was not found in
%one aperture and not the other, the source is typically the extended
%lobes of very a large radio source (\eg\ Hydra A). The cases where an
%additional source was found are typically low-redshift clusters with a
%radio galaxy near the BCG. For each cluster, a final by-eye check of
%the radio emission in relation to the optical (using DSS) and infrared
%(using 2MASS) emission of the BCG is made.  The
%NED\footnote{http://nedwww.ipac.caltech.edu/} database was also
%queried with the central coordinates of each radio source to ensure
%that for those sources which have been cataloged, their redshift was
%within the limits of the cluster redshift

%%%%%%%%%%%%%%%%%%%%%%%%%%%%%%%%%%%%%%%%%%%%%%%%%%%%
\section{Star Formation and Central Entropy}
\label{sec:sf}
%%%%%%%%%%%%%%%%%%%%%%%%%%%%%%%%%%%%%%%%%%%%%%%%%%%%

Of the \clnum\ clusters in \accept, we located \halpha\ luminosities
from the literature for 97 clusters. Of those 97, 41 have
\halpha\ detections and 56 do not. We treat the non-detections as
upper-limits using the observation's flux-limit. The mean central
entropy for clusters with detections is $\kna = 13.6 \pm 4.72 \ent$,
and for clusters with only upper-limits $\kna = 118 \pm 46.6 \ent$. In
Figure \ref{fig:ha} is plotted \lha\ versus \kna. One can immediately
see there is a dichotomy between clusters with and without
\halpha\ emission. If a cluster has a central entropy $\la 30 \ent$
then star formation is ``on'', while above this threshold the star
formation is almost universally ``off''. For brevity we refer to this
threshold as \kthr\ hereafter. The exceptions above \kthr\ which have
\halpha\ emission (blue boxes with orange dots) are RX J1000.4+4409
($\kna = 35.8 \pm 3.8 \ent$) and Zwicky 2701 ($\kna = 45.6 \pm 4.8
\ent$).  There are also exceptions below \kthr\ without
\halpha\ emission (blue boxes with red stars): Abell 2029, A2107,
A2151, EXO 0422-086, and RBS 533.

The entropy threshold we observe is also insensitive to redshift
effects, \ie\ star formation unaccounted for in high or low redshifts
galaxies will not change our results. The mean and dispersion of the
redshifts for clusters with and without \halpha\ are nearly identical,
$z = 0.116 \pm 0.103$ and $z = 0.131 \pm 0.083$ respectively, and
applying a redshift cut (\ie\ $z = 0-0.15$ or $z = 0.15-0.3$) does not
change the \kna-\halpha\ dichotomy. Most important is that changes in
the \halpha\ luminosities will only move clusters left and right in
Figure \ref{fig:ha}, mobility along the \kna\ axis is
minimal. Qualitatively, the correlation between low central entropy
and presence of \halpha\ emission is very robust.

\begin{figure}
  \begin{center}
    \includegraphics*[width=\columnwidth, trim=30mm 7mm 40mm 17mm, clip]{ha}
    \caption{Plotted here is central entropy versus \halpha\
      luminosity. Orange circles represent detections, black squares
      are non-detection upper-limits, and blue boxes with inset red
      stars or orange dots are peculiar clusters which do not adhere
      to the observed trend (see text). The horizontal dashed line
      marks $\kna = 30 \ent$. Note the presence of a star formation
      dichotomy beginning at $\approx \kna \la 30 \ent$.}
    \label{fig:ha}
  \end{center}
\end{figure}

\begin{figure}
  \begin{center}
    \centering
    \includegraphics*[width=\columnwidth, trim=30mm 7mm 40mm 17mm, clip]{k0hist}
    \caption{Shown is the log-space distribution of \kna\ for all
      clusters in \accept. The vertical dashed line marks $\kna = 30
      \ent$. The obvious bimodal distribution is predicted to be a
      consequence of AGN feedback and thermal conduction in the
      cluster core.}
    \label{fig:k0hist}
  \end{center}
\end{figure}

In Paper II we present the result that the \kna\ distribution for
\accept\ is bimodal (shown in Figure \ref{fig:k0hist}). The clusters
which populate the peak around $\kna \approx 15 \ent$ are clusters
with short central cooling times ($\tcool \ll \Hn^{-1}$) and would be
classified as ``cooling flow'' clusters. It does not come as a
surprise that modest star formation is occurring in these BCGs
\citep{hu85}, but it is very interesting that there appears to be a
characteristic entropy threshold only below which multi-phase gas, and
presumably stars, form. We will discuss this point further in
\S\ref{sec:diss}.

%%%%%%%%%%%%%%%%%%%%%%%%%%%%%%%%%%%%%%%%%%%%%%%%%%%%%%
\section{Radio-loud AGN and Central Entropy}
\label{sec:agn}
%%%%%%%%%%%%%%%%%%%%%%%%%%%%%%%%%%%%%%%%%%%%%%%%%%%%%%

Of the \clnum\ clusters in \accept, 100 have radio source detections
with a mean $\kna = 20.8 \pm 7.85 \ent$, while the other 115 clusters
with only upper-limits have a mean $\kna = 125 \pm 48.1 \ent$. Recall
NVSS and SUMSS are low resolution surveys with FWHM at $\approx
45\arcsec$ which at $\red = 0.2$ is $\approx 150\kpc$. This scale is
larger then the size of a typical cluster cooling region and makes it
difficult to determine if the radio emission is associated with the
BCG. We therefore focus only on clusters at $z < 0.2$. After the
redshift cut, 128 clusters remain -- 59 with radio detections (mean
$\kna = 17.1 \pm 6.59 \ent$) and 69 without (mean $\kna = 108 \pm 43.7
\ent$).

In Figure \ref{fig:radzcut} we have plotted radio power versus \kna.
The obvious dichotomy seen in the \halpha\ measures and characterized
by \kthr\, is also present in the radio as an overabundance of
clusters at $\lradio \gtrsim 10^{40} \ergps$ and $\kna \la \kthr$.
This suggests that AGN activity in BCGs, while not exclusively limited
to low core entropy clusters, still favors cluster cores which have an
average entropy less than the threshold at which conductive stability
dominates.

\begin{figure}
  \begin{center}
    \includegraphics*[width=\columnwidth, trim=30mm 7mm 40mm 17mm, clip]{radio_zcut}
    \caption{Plotted is the radio power versus \kna only for clusters
      with $z \leq 0.2$. Orange circles represent detections, black
      squares are non-detection upper-limits, and blue boxes with
      inset red stars or orange dots are peculiar clusters which do
      not adhere to the observed trend (see text). The trend of AGN
      feedback preferentially being ``on'' below $\kna \approx 30
      \ent$ is once again very clear.}
    \label{fig:radzcut}
  \end{center}
\end{figure}

We have again highlighted two subsets of clusters in Figure
\ref{fig:radzcut}: clusters below \kthr\ without radio sources (blue
boxes with red stars) and clusters above \kthr\ with radio sources
(blue boxes with orange dots). The peculiar clusters below \kthr\ are
A133, A744, A1204, A2107, A2556, AWM7, ESO 5520200, MACS J1931.8-2634,
MKW 04, and MS J1157.3+5531. The peculiar clusters above \kthr\ are
A193, A368, A586, A1942, A2063, A2244, A3528S, A3558, A4038, CL
J1226.9+3332, MACS J0520.7-1328, MACS J1206.2-0847, MACS J2245.0+2637,
RBS 0461, RX J0528.9-3927, and Zwicky 2701. Radio-quiet AGN are not
uncommon, and having a few clusters in our sample without radio-loud
sources where we expect to find them is not surprising.  However, the
clusters above \kthr\ with a radio-loud source are surprising, and may
be special cases of cluster BCGs with corona.  \cite{coronae} have
studied coronae extensively and find they are like ``mini-cooling
cores'' and may provide the environment to promote the gas cooling
necessary to fuel an AGN. Indeed, some of these peculiar clusters show
indications that a very compact X-ray source is associated with the
BCG. These peculiar clusters are most likely exceptions to the low
entropy-AGN activity relation and do not weaken the overall trend.

%\begin{figure}
%  \begin{center}
%    \includegraphics*[width=\columnwidth, trim=30mm 7mm 40mm 17mm, clip]{radio_allz}
%    \caption{Plotted is the power at 1.4 GHz (calculated from NVSS
%      flux) versus \kna. Orange circles represent detections, black
%      squares are non-detection upper-limits, and blue boxes with
%      inset red stars or orange dots are peculiar clusters which do
%      not adhere to the observed trend. Once again we see that only
%      below $\kna \la 30 \ent$ are the most powerful radio sources --
%      most likely associated with AGN -- ``on'', while above the
%      threshold the trend is much weaker.}
%    \label{fig:radall}
%  \end{center}
%\end{figure}
%The weaker adherence to the \kthr\ for the radio sources can partially
%be explained by resolution effects and fueling sources. NVSS and SUMSS
%have low angular resolution of $\approx 45\arcsec$ at full width at
%half maximum. At $z = 0.3$ this resolution is $\approx 200\kpc$ and at
%$\red = 0.2$ it is $\approx 150\kpc$, both of which are much larger
%then the scale size of a typical cluster cooling region. At these
%physical scales we have nearly no information about the radio source
%morphology and thus are not able to distinguish between AGN point
%sources, extended radio lobes, detached lobes, ghosts, relics, and
%even diffuse, amorphous re-accelerated regions. Being blind to
%morphological distinction, we expect larger scatter in the radio-\kna\
%relation than if we were to only focus on, for example, FRII radio
%sources.
%When we make a cut in redshift space and only
%consider clusters at $\red \leq 0.2$, we find the entropy-AGN relation
%to be more pronounced (see Figure \ref{fig:radzcut}). After the
%redshift cut, 128 clusters remain -- 59 with radio detections (mean
%$\kna = 17.1 \pm 6.59 \ent$) and 69 without (mean $\kna = 108 \pm 43.7
%\ent$). Ignoring clusters above \kthr, the mean of clusters with radio
%detections becomes $\kna = 12.1 \pm 4.31 \ent$. That the entropy
%threshold at $\kna \approx 30 \ent$ is robust against the cut in
%redshift space is important.

%%%%%%%%%%%%%%%%%%%%%%%
\section{Discussion}
\label{sec:diss}
%%%%%%%%%%%%%%%%%%%%%%%

In \cite{radioquiet} and \cite{conduction}, the prediction is made
that if thermal electron conduction is an important mechanism in some
cluster cores, then the entropy distribution should ``bifurcate''. The
model presented in \cite{conduction} proposes that above a certain
entropy threshold, which we find observationally to be at least
$\approx 30 \ent$, electron thermal conduction is capable of balancing
the energy losses from radiative cooling is of great interest. If we
assume heat conduction between a gas parcel and its surroundings is
proceeding at the Spitzer rate, and we further assume the dominant
cooling mechanism is free-free emission, then one can derive the size
of the radiative cooling region specified by the Field length,
\begin{equation}
\lambda_F = \left(\frac{\kappa_S T}{n_e^2 \Lambda(T,Z)}\right)^{1/2} \approx 4 \kpc \left(\frac{K}{10 \ent}\right)^{3/2}f_c^{1/2}, \nonumber
\end{equation}
where $\kappa_S$ is the Spitzer conduction coefficient ($\kappa_S =
6\times10^{-7} f_c T^{5/2} \ergps \pcm \mathrm{~K}^{-7/2}$), \nelec\
is electron gas density, $T$ is gas temperature, $\Lambda(T,Z)$ is the
cooling function at a given temperature and metallicity, and $f_c$ is
a suppression factor. By a coincidence of scaling, the Field length is
a function of entropy only, $\lambda_F \propto K^{3/2}$. The critical
entropy profile which separates conductively stable systems from
unstable systems is then $K(r) \approx 10 \ent f_c^{-1/3}
(r/4\kpc)^{2/3}$ \citep{radioquiet}.

Clusters below this critical entropy profile will be capable of
supporting a multi-phase medium because thermal conduction cannot wipe
out gas inhomogeneity. Hence, stars can form and small streams of gas
can work their way onto the SMBH in the BCG and initiate an AGN
feedback cycle. However, above the critical entropy profile,
conduction proceeds faster than cooling and any cool gas clouds which
might condense or be deposited in the core via mergers will evaporate.
We suggest that our finding of a robust relation between indicators of
feedback and central entropy strongly supports the hypothesis that
electron thermal conduction is an important mechanism in the cluster
feedback cycle. Recently \cite{2008arXiv0802.1864R} found that BCGs in
low entropy environments have strong blue gradients (another indicator
of active/recent star formation), a result which further supports our
hypothesis.

We propose that within the framework of an episodic AGN feedback model
which includes thermal conduction, that our observation of the bimodal
central entropy distribution is well explained as part of a larger
cluster entropy life-cycle. Using the model of AGN feedback model of
\cite{agnframework}, let us consider a cluster left in isolation for a
few Gyrs. AGN outbursts originating from the BCG with power outputs of
$10^{45} \ergps$ lasting a few tens of Myrs and occurring every few
hundred Myrs are capable of supplying the energy to form a
pseudo-stable core entropy of $0-30 \ent$. However, on the rare
occasion that a large AGN outburst occurs ($E > 10^{61}$ ergs), the
core entropy can be pushed above $30 \ent$. At these entropy levels
the cluster becomes conductively stable. Unable to cool, mergers can
quickly remove $40-60 \ent$ clusters, leaving a distinct gap in the
\kna\ distribution and creating a second population with $\kna > 100
\ent$. A detailed theoretical investigation of the role conduction
plays in the entropy life-cycle of a cluster is a warranted next step.
\\
\\
We were supported in this work through \Chandra\ grants AR-6016X,
AR-4017A, an MSU start-up grant for Megan Donahue, and NASA LTSA
program NNG-05GD82G. We thank Brian McNamara, David Rafferty, and Paul
Nulsen for comparing data and stimulating discussion on this topic.
The CXC is operated by the SAO for and on behalf of NASA under
contract NAS8-03060.

%%%%%%%%%%%%%%%%
% Bibliography %
%%%%%%%%%%%%%%%%

\bibliography{cavagnolo}

\end{document}
