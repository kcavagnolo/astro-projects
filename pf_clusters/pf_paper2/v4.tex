\newcommand{\clnum}{\ensuremath{217}}
\newcommand{\accept}{\textit{ACCEPT}}
\newcommand{\kthr}{\ensuremath{K_{\mathrm{thresh}}}}

%%%%%%%%%%
% Header %
%%%%%%%%%%

%\documentclass[12pt, preprint]{aastex}
\documentclass{emulateapj}
\usepackage{apjfonts,graphicx,here,lscape}
\usepackage{common} 
\bibliographystyle{apj}

\begin{document}
\title{The Relation of \halpha\ Emission and Radio Sources to \\
  Intracluster Gas Entropy in Galaxy Clusters}
\author{
  Kenneth W. Cavagnolo\altaffilmark{1,2},
  Megan Donahue\altaffilmark{1},
  G. Mark Voit\altaffilmark{1}, 
  and Ming Sun\altaffilmark{1}}
\altaffiltext{1}{Michigan State University, Department of Physics and
  Astronomy, BPS Building, East Lansing, MI 48824}
\altaffiltext{2}{cavagnolo@pa.msu.edu}
\shorttitle{The Entropy, \halpha, and Radio Connection}
\shortauthors{K. W. Cavagnolo et al.}
\journalinfo{Submitted to ApJ Letters}
%\accepted{ }
\submitted{April 10, 2008}

%%%%%%%%%%%%
% Abstract %
%%%%%%%%%%%%

\begin{abstract}
  We present results from a \Chandra\ archival study of the
  intracluster entropy distribution for a sample of \clnum\ galaxy
  clusters showing that \halpha\ and radio emission from the brightest
  cluster galaxy are sensitive to gas entropy of the cluster
  core. After parameterizing the central entropy, \kna, of each
  cluster as the value at which the entropy distribution departs from
  a power-law, we compare the \kna\ values with the
  \halpha\ luminosities and radio powers of the brightest cluster
  galaxies. Interpreting these \halpha\ and radio emission as
  indicators that star formation and/or AGN are active in the BCG, we
  find that for most clusters with $\kna \la 30 \ent$ galactic
  feedback is active, while above this threshold the feedback is
  significantly fainter and in most clusters not detected.
\end{abstract}

\keywords{galaxies: clusters: general -- X-rays: active -- galaxies:
  conduction -- galaxies: cooling flows -- galaxies}

%%%%%%%%%%%%%%%%%%%%%%%%%
\section{Introduction}
\label{sec:intro}
%%%%%%%%%%%%%%%%%%%%%%%%%

In recent years the ``cooling flow problem'' has been the focus of
intense scrutiny as the solutions have broad impact in the area of
galaxy formation. The adiabatic model of hierarchical structure
formation predicts an over-production of dwarf and massive galaxy
haloes, and that the most massive galaxies in the Universe --
brightest cluster galaxies (BCGs) -- should be bluer and more massive
than observations find. Compounding this issue is that star formation
in massive galaxies was more prodigious at higher redshifts
\citep{1996AJ....112..839C, 2005ApJ...619L.135J}, contradicting model
predictions and resulting in the observation of present-day
ellipticals being ``dead and red''. There is also the so-called cosmic
down-sizing problem \citep{1996AJ....112..839C}, where massive
galaxies are observed to form earlier than hierarchical galaxy
formation models predict. It has been hypothesized that these problems
with the existing galaxy formation models are related to galactic
feedback, specifically in the form of active galactic nuclei (AGN) and
star formation/supernovae. Both theoretical \citep{bower06, croton06}
and observational (see \citealt{mcnamrev} for a review) studies have
given this hypothesis considerable traction in recent years.

Studies which seek to explain the differences between galaxy formation
models and observations by investigating the interaction of feedback
sources with the hot atmosphere of their host galaxy cluster have thus
far been fruitful. Using properties of X-ray cavities in the
intracluster medium (ICM) in combination with the energetics of the
BCG radio source, \cite{birzan04} observed that AGN feedback provides
the necessary energy to retard cooling in the cores of clusters. This
result suggests that, under the right conditions, AGN are capable of
quenching star formation by heating the surrounding ICM. One quantity
which has proven useful in studying AGN heating is ICM entropy.

In previous observational work we have focused on ICM entropy as a
means for understanding the cooling and heating processes in clusters
(see \citealt{radioquiet, d06, accept}). Entropy, $K=P\rho^{-2/3}$, is
useful as it is a more fundamental property of the ICM than
temperature or density alone \citep{voitbryan,voitreview}. ICM
temperature mainly reflects the depth and shape of the dark matter
potential well, while density represents how much the gravitational
well can compress the gas. Entropy on the other hand determines the
density of an isobaric gas parcel, and only gains or losses of heat
energy can change the entropy. Thus, study of the ICM entropy
distribution is an investigation of the cluster thermal history.

In \cite{accept} we present the radial entropy profiles for a sample
of \clnum\ clusters taken from the \Chandra\ Data Archive. We have
named this project the Archive of Chandra Cluster Entropy Profile
Tables, or \accept\ for short. To characterize the ICM entropy
distributions of \accept\ we fit the equation $K(r) = K_0
+K_{100}(r/100 \kpc)^{\alpha}$ to each entropy profile. In this
equation, \kna\ is the central entropy and \khun\ is a normalization
of the power-law component at 100 kpc. \kna\ is not the minimum core
entropy or the entropy at $r=0$, nor is it the gas entropy which would
be measured immediately around the AGN or in a BCG X-ray
coronae. \kna\ simply parameterizes the departure of the radial
entropy distribution from a power-law. There are two results in
\cite{accept} which are relevant to this letter: 1) \kna\ is non-zero
for most clusters in \accept\, and 2) the \kna\ distribution is
bimodal.

In this letter we present the results of exploring the relationship
between the expected by-products of cooling, \eg\ star formation and
AGN activity, to the \kna\ values for the clusters in \accept. To
determine the activity level of feedback in cluster cores, we selected
two readily available observables: \halpha\ and radio emission. Using
these observables as a window on the processes operating in the core
of clusters, we find there exists a critical entropy level below which
\halpha\ and radio emission are predominantly present, while above
this threshold these emission sources are much fainter and in most
cases are not detected. Our results suggest the formation of thermal
instabilities in the ICM, and initiation of processes such as star
formation and AGN activity, are connected to core entropy.

This letter proceeds in the following manner: In \S\ref{sec:data} we
cover the basics of our data analysis. The
entropy-\halpha\ relationship is discussed in \S\ref{sec:sf}, while
the entropy-radio relationship is discussed in \S\ref{sec:agn}. A
brief summary is provided in \S\ref{sec:diss}.  For this letter we
have assumed a flat \LCDM\ Universe with cosmogony $\OM=0.3$,
$\OL=0.7$, and $\Hn=70\km\ps\pMpc$. All uncertainties are 90\%
confidence.

%%%%%%%%%%%%%%%%%%%%%%%%%%
\section{Data Analysis}
\label{sec:data}
%%%%%%%%%%%%%%%%%%%%%%%%%%

In this letter we briefly describe our data reduction and methods for
producing entropy profiles and refer interested readers to the more
thorough explanations in \cite{d06}, \cite{accept}, and
\cite{xrayband}.

%%%%%%%%%%%%%%%%%%%%
\subsection{X-ray}
\label{sec:xray}
%%%%%%%%%%%%%%%%%%%%

X-ray data was taken from publicly available observations in the
\Chandra\ Data Archive. Following standard \Ciao\ reduction
techniques\footnote{http://cxc.harvard.edu/ciao/guides/}, data was
reprocessed using \Ciao\ 3.4.1 and \Caldb\ 3.4.0, resulting in point
source and flare clean events files at level-2. Entropy profiles were
derived using the relation $K(r) = T(r) \nelec(r)^{-2/3}$ where $T(r)$
is the radial ICM temperature and $\nelec(r)$ is the radial ICM
electron density.

Radial temperature profiles were created by spatially dividing each
cluster into concentric annuli with the minimum requirement of three
annuli containing 2500 counts. Source spectra were extracted from
these annuli, while corresponding background spectra were extracted
from blank-sky backgrounds tailored to match each observation. Each
blank-sky background was corrected to account for variation of the
hard-particle background, while spatial variation of the soft-galactic
background was accounted for through addition of a fixed background
component during spectral fitting. Weighted responses which account
for spatial variations of the CCD calibration were also created for
each observation. Spectra were then fit over the energy range 0.7-7.0
keV in \xspec\ 11.3.2ag \citep{xspec} using a single-component
absorbed thermal model.

Radial electron density profiles were created using surface brightness
profiles and spectroscopic information. Exposure-corrected,
background-subtracted, point-source-clean surface brightness profiles
were extracted from $5\arcsec$ concentric annular bins over the energy
range 0.7-2.0 keV. In conjunction with the spectroscopic normalization
and 0.7-2.0 keV count rate, surface brightness was converted to
electron density using the deprojection technique of
\cite{kriss83}. Errors were estimated using 5000 Monte Carlo
realizations of the surface brightness profile.

A radial entropy profile for each cluster was then produced from the
temperature and electron density profiles. The entropy profiles were
fit with a hybrid model which is a power-law at large radii and can
flatten at small radii (see \S\ref{sec:intro} for equation). We define
central entropy as \kna\ from the best-fit model. In \cite{accept} we
demonstrate that the best-fit \kna\ and observational \kna\ do not
significantly differ for most clusters. As noted in Section
\ref{sec:intro}, \kna\ is not the minimum gas entropy found in the
core, nor does it represent an absolute flattening of the
profile. Rather \kna\ is a characterization of the entropy
distribution's departure from a power-law, and is a representation of
the azimuthally averaged entropy for the innermost radial bins.

%%%%%%%%%%%%%%%%%%%%%%
\subsection{\halpha}
\label{sec:ha}
%%%%%%%%%%%%%%%%%%%%%%

One goal of our project was to determine if ICM entropy is connected
to processes like star formation. Thus, we selected \halpha\ emission,
which has proven to be a strong indicator of ongoing star formation
\citep{kennicuttrelation}, to facilitate investigation of possible
connections between entropy and star formation.

We have gathered \halpha\ values from several sources, most notably
\cite{crawford99}. Additional sources of data are M. Donahue's
observations while at Carnegie, \cite{heckman89}, \cite{dsg92},
\cite{lawrence96}, \cite{1996AJ....112.1390V}, \cite{white97},
\cite{2005MNRAS.363..216C}, and \cite{ir_quillen}. To place all the
\halpha\ measurements on equal footing, we have rolled back from the
cosmogony used in each study and re-calculated \halpha\ luminosities
using our assumed \LCDM\ values.

The \lha\ values shown in this letter are not meant to be interpreted
as a surrogate for star formation rates, \eg\ using the Kennicutt
relation. The individual values of \lha\ are not important for the
purposes of this letter and we suggest the reader interpret the
\lha\ values as a binary indicator: either \halpha\ emission is off or
on.

%%%%%%%%%%%%%%%%%%%%
\subsection{Radio}
\label{sec:radio}
%%%%%%%%%%%%%%%%%%%%

We were also interested to know if ICM entropy is connected to AGN
activity. It has long been known that BCGs are more likely to house
radio-loud AGN than other cluster galaxies \citep{burns81,
  valentijn83, burns90}. Thus, we chose to look for radio emission
from the BCG of each \accept\ cluster as a sign that an AGN was active
in the cluster core.

We take advantage of the nearly all-sky flux-limited coverage of the
NRAO VLA Sky Survey (NVSS, \citealt{nvss}) and Sydney University
Molonglo Sky Survey (SUMSS, \citealt{sumss1, sumss2}). Both surveys
probe very low radio fluxes and are excellent for our purposes. NVSS
is a continuum survey at 1.4 GHz of the entire sky north of $\delta =
-40\degr$, while SUMSS is a continuum survey at 843 MHz of the entire
sky south of $\delta = -30\degr$. The completeness limit of NVSS is
$\approx 2.5$ mJy and for SUMSS it is $\approx 10$ mJy when $\delta >
-50\degr$ or $\approx 6$ mJy when $\delta \leq -50\degr$. The NVSS
positional uncertainty for both right ascension and declination is
$\la 1''$ for sources brighter than 15 mJy, and $\approx 7''$ at the
survey detection limit \citep{nvss}. At $z=0.2$, these uncertainties
equal distances on the sky of $\sim3-20$ kpc. For SUMSS, the
positional uncertainty is $\la 2''$ for sources brighter than 20 mJy,
and is always less than $10''$ \citep{sumss1,sumss2}. The distances at
$z=0.2$ associated with these uncertainties is $\sim6-30$ kpc. We
calculate the radio power for each radio source using the standard
relation $\nu L_{\nu} = 4 \pi D_L^2 S_{\nu} f_0$ where $S_{\nu}$ is
the 1.4 GHz or 843 MHz flux from NVSS or SUMSS, $D_L$ is the
luminosity distance, and $f_0$ is the central beam frequency of the
observations.

Radio sources were found using two methods. The first method was to
search for sources within a fixed angular distance of $20''$ around
the cluster X-ray peak. The probability of randomly finding a radio
source within an aperture of $20''$ is exceedingly low ($< 0.004$ for
NVSS). Thus, in \clnum\ total field searches, we expect to find no
more than one spurious source. The second method involved searching
for sources within $20\kpc$ of the cluster X-ray peak. At $z \approx
0.051$, $1''$ equals 1 kpc, thus for clusters at $z \ga 0.05$ the 20
kpc aperture is smaller than the $20''$ aperture and the likelihood of
finding a spurious source gets smaller. Both methods produce nearly
the same list of radio sources with the differences being the very
large, extended lobes of low-redshift radio sources such as Hydra A.

To make a spatial and morphological assessment of the radio emission's
origins, \ie\ determining if the radio emission is associated with the
BCG, high angular resolution is necessary. However, NVSS and SUMSS are
low-resolution surveys with FWHM at $\approx 45\arcsec$. We therefore
cannot distinguish between ghost cavities/relics, extended lobes,
point sources, re-accelerated regions, or if the emission is coming
from a galaxy very near the BCG or a background/foreground source. We
have handled this complication by visually inspecting each radio
source in relation to the optical (using DSS I/II) and infrared (using
2MASS) emission of the BCG. We have used this method to establish that
the radio emission is most likely coming from the BCG. When available,
high resolution data from VLA FIRST\footnote{Faint Images of the Radio
 Sky at Twenty cm; http://sundog.stsci.edu} was added to the
visual inspection.

%%%%%%%%%%%%%%%%%%%%%%%%%%%%%%%%%%%%%%%%%%%%%%%%%%%%%%%%
\section{\halpha\ Emission and Central Entropy}
\label{sec:sf}
%%%%%%%%%%%%%%%%%%%%%%%%%%%%%%%%%%%%%%%%%%%%%%%%%%%%%%%%

\begin{figure}
  \begin{center}
    \includegraphics*[width=\columnwidth, trim=28mm 7mm 40mm 17mm, clip]{ha}
    \caption{Plotted here is central entropy versus
      \halpha\ luminosity. Orange circles represent detections, black
      circles are non-detection upper-limits, and blue boxes with
      inset red stars or orange circles are peculiar clusters which do
      not adhere to the observed trend (see text). The vertical dashed
      line marks $\kna = 30 \ent$.  Note the presence of an
      \halpha\ detection dichotomy starting at $\kna \la 30 \ent$.}
    \label{fig:ha}
  \end{center}
\end{figure}

Of the \clnum\ clusters in \accept, we located \halpha\ observations
from the literature for 105 clusters. Of those 105, \halpha\ was
detected in 45, while the remaining 60 have no detection. We treat the
non-detections as upper limits using the observation's flux limit.  The
mean central entropy for clusters with detections is $\kna = 14.1 \pm
4.9 \ent$, and for clusters with only upper-limits $\kna = 124 \pm 47
\ent$.

In Figure \ref{fig:ha} central entropy is plotted versus
\halpha\ luminosity. One can immediately see there is a dichotomy
between clusters with and without \halpha\ emission. If a cluster has
a central entropy $\la 30 \ent$ then \halpha\ emission is ``on'',
while above this threshold the emission is predominantly ``off''. For
brevity we refer to this threshold as \kthr\ hereafter. The cluster
above \kthr\ which has \halpha\ emission (blue box with orange dot) is
Zwicky 2701 ($\kna = 45.6 \pm 4.8 \ent$). There are also clusters
below \kthr\ without \halpha\ emission (blue boxes with red stars):
A2029, A2107, EXO 0422-086, and RBS 533. These five clusters are
clearly exceptions to the much larger trend.

The entropy threshold we observe is also insensitive to redshift
effects, \ie\ \halpha\ emission unaccounted for in high or low
redshifts galaxies will not change our results. The mean and
dispersion of the redshifts for clusters with and without \halpha\ are
nearly identical, $z = 0.126 \pm 0.106$ and $z = 0.136 \pm 0.084$
respectively, and applying a redshift cut (\ie\ $z = 0-0.15$ or $z =
0.15-0.3$) does not change the \kna-\halpha\ dichotomy. Most important
is that changes in the \halpha\ luminosities will only move points up
or down in Figure \ref{fig:ha}, mobility along the \kna\ axis is
minimal. Qualitatively, the correlation between low central entropy
and presence of \halpha\ emission is very robust.

In \cite{accept} we present the result that the \kna\ distribution for
\accept\ is bimodal. The clusters which populate the peak around $\kna
\approx 15 \ent$ in that distribution are clusters with short central
cooling times ($\tcool \ll \Hn^{-1}$) and would be classified as
``cooling flow'' clusters. Assuming star formation is associated with
the \halpha\ nebulosity \citep{voit97,cardiel98}, it does not come as
a surprise that modest star formation is occurring in these BCGs
\citep{johnstone87, mcnamara89}. But, it is very interesting that
there appears to be a characteristic entropy threshold only below
which multi-phase gas, and presumably stars, form. \cite{conduction}
have recently proposed electron thermal conduction may be responsible
for setting this threshold. This hypothesis has received further
support from the theoretical work of \cite{2008arXiv0804.3823G}
showing that thermal conduction can stabilize non-cool core clusters
against the formation of thermal instabilities.

%%%%%%%%%%%%%%%%%%%%%%%%%%%%%%%%%%%%%%%%%%%%%%%%%%%%
\section{Radio Sources and Central Entropy}
\label{sec:agn}
%%%%%%%%%%%%%%%%%%%%%%%%%%%%%%%%%%%%%%%%%%%%%%%%%%%%

\begin{figure}
  \begin{center}
    \includegraphics*[width=\columnwidth, trim=28mm 7mm 40mm 17mm, clip]{radio_zcut}
    \caption{Shown is BCG associated radio power versus \kna\ for
      clusters with $z < 0.2$. Orange points represent detections,
      black points are non-detection upper-limits, and blue boxes with
      inset red stars or orange points are peculiar clusters which do
      not adhere to the observed trend (see text). Circles are for
      NVSS observations and squares are for SUMSS observations. The
      vertical dashed line marks $\kna = 30 \ent$. The radio sources
      show the same trend as \halpha: for $\kna \la 30 \ent$ bright
      radio emission is preferentially ``on''.}
    \label{fig:radzcut}
  \end{center}
\end{figure}

Of the \clnum\ clusters in \accept, 97 have radio source detections
with a mean $\kna$ of $22.8 \pm 9.2 \ent$, while the other 120
clusters with only upper-limits have a mean $\kna$ of $130 \pm 51
\ent$. Recall NVSS and SUMSS are low resolution surveys with FWHM at
$\approx 45\arcsec$ which at $\red = 0.2$ is $\approx 150\kpc$. This
scale is larger then the size of a typical cluster cooling region and
makes it difficult to determine absolutely that the radio emission is
associated with the BCG. We therefore focus only on clusters at $z <
0.2$. After the redshift cut, 130 clusters remain -- 61 with radio
detections (mean $\kna = 17.5 \pm 7.3 \ent$) and 69 without (mean
$\kna = 107 \pm 44 \ent$).

In Figure \ref{fig:radzcut} we have plotted radio power versus \kna.
The obvious dichotomy seen in the \halpha\ measures and characterized
by \kthr\, is also present in the radio as an overabundance of
clusters with $\nu L_{\nu} \gtrsim 10^{40} \ergps$ and $\kna \la
\kthr$.  This suggests that AGN activity in BCGs, while not
exclusively limited to low core entropy clusters, is more likely to be
found in clusters which have an average core entropy less than
\kthr. That star formation and AGN activity are subject to the same
entropy threshold before turning ``on'' suggests the mechanism which
promotes or initiates one is also involved in the activation of the
other. Our results also suggest that cold-mode accretion
\citep{pizzolato05, hardcastle07} may be the dominant method of
fueling AGN in BCGs.

We have again highlighted two subsets of clusters in Figure
\ref{fig:radzcut}: clusters below \kthr\ without radio sources (blue
boxes with red stars) and clusters above \kthr\ with radio sources
(blue boxes with orange dots). The peculiar clusters below \kthr\ are
A133, A539, A1204, A2107, A2462, A2556, AWM7, ESO 5520200, MKW 04, and
MS J1157.3+5531. The peculiar clusters above \kthr\ are 2PIGG
J0011.5-2850, A586, A2063, A2147, A2244, A3528S, A3558, A4038, and RBS
0461. Radio-quiet AGN are not uncommon, and having a few clusters in
our sample without radio-loud sources where we expect to find them is
not surprising. However, the clusters above \kthr\ with a radio-loud
source are interesting, and may be special cases of BCGs with embedded
corona. \cite{coronae} extensively studied coronae and found they are
like ``mini-cooling cores'' and may provide an environment insulated
from the harsh ICM in which AGN fueling via cold gas blobs could
proceed. Indeed, some of these peculiar clusters show indications that
a very compact ($r \la 5\kpc$) X-ray source is associated with the
BCG. These peculiar clusters are most likely exceptions to the low
entropy-AGN activity relation and do not weaken the overall trend.

%%%%%%%%%%%%%%%%%%%%%
\section{Summary}
\label{sec:diss}
%%%%%%%%%%%%%%%%%%%%%

We have presented a comparison of ICM central entropy values and
measures of BCG \halpha\ and radio emission for a \Chandra\ archival
sample of galaxy clusters. The mean \kna\ for clusters with and
without \halpha\ detections are $14.1 \pm 4.9 \ent$ and $124 \pm 47
\ent$, respectively. For clusters at $z < 0.2$ with BCG radio emission
the mean $\kna = 17.5 \pm 7.3 \ent$, while for BCGs with only upper
limits, the mean $\kna = 107 \pm 44 \ent$. We find that below a
characteristic central entropy threshold of $\kna \approx 30 \ent$,
\halpha\ and bright radio emission are more likely to be detected,
while above this threshold \halpha\ is not detected and radio
emission, if detected at all, is significantly fainter. While other
mechanisms can produce \halpha\ or radio emission besides star
formation and AGN, if one assumes the \halpha\ and radio emission are
coming from these two feedback sources, then our results suggest the
development of multiphase gas in cluster cores (from which stars could
form and AGN could be feed) is strongly coupled to ICM entropy.

\acknowledgements
We were supported in this work through \Chandra\ grants AR-6016X,
AR-4017A, and NASA LTSA program NNG-05GD82G. The CXC is operated by
the SAO for and on behalf of NASA under contract NAS8-03060.

%%%%%%%%%%%%%%%%
% Bibliography %
%%%%%%%%%%%%%%%%

\bibliography{cavagnolo}

\end{document}
