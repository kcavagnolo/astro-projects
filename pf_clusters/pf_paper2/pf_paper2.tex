\newcommand{\kthr}{\ensuremath{K_{\mathrm{thresh}}}}

%%%%%%%%%%
% Header %
%%%%%%%%%%

%\documentclass[11pt, preprint]{aastex}
%\documentclass[onecolumn]{emulateapj}
\documentclass{emulateapj}
\usepackage{apjfonts,graphicx,here,lscape}
\usepackage{common}
\newcommand{\accept}{\textit{ACCEPT}}
\begin{document}
\title{Athenaeum of Chandra Cluster Entropy Profile Tables -- I.\\
Star Formation, AGN, and the Entropy-Feedback Connection}
\author{Kenneth W. Cavagnolo\altaffilmark{1,2},
	Megan Donahue\altaffilmark{1},
	G. Mark Voit\altaffilmark{1}, and
	Ming Sun\altaffilmark{1}}
\altaffiltext{1}{Michigan State University, Department of Physics and
Astronomy, BPS Building, East Lansing, MI 48824}
\altaffiltext{2}{cavagnolo@pa.msu.edu}
\shorttitle{Entropy, Star Formation, and AGN}
\shortauthors{K. W. Cavagnolo et al.}
\bibliographystyle{apj}

%%%%%%%%%%%%
% Abstract %
%%%%%%%%%%%%

\begin{abstract}
\end{abstract}

%%%%%%%%%%%%
% Keywords %
%%%%%%%%%%%%

\keywords{ }

%%%%%%%%%%%%%%%%%%%%%%
\section{Introduction}
\label{sec:intro}
%%%%%%%%%%%%%%%%%%%%%%

In recent years the ``cooling flow problem'' has been the focus of
much study as the solutions have broad impact in the areas of galaxy
formation, e.g. explaining apparent suppression of the high-mass end
of the galaxy luminosity function. The adiabatic model of hierarchical
structure formation improperly predicts an over-production of dwarf
and massive galaxy halos. The brightest cluster galaxies (BCGs) -- the
most massive galaxies in the Universe -- are also predicted to be much
too blue and massive. Compounding the issue is that the star formation
rate in massive galaxies was larger at higher redshifts ($\red
\approx 1-2$) \citep{1996AJ....112..839C, 2005ApJ...619L.135J}. Theory
also predicts these massive galaxies form late from smaller halos, and
this problem is commonly termed cosmic down-sizing
\citep{1996AJ....112..839C}. The idea that truncation of the high mass
end of the galaxy luminosity function is related to feedback, \ie\
active galactic nuclei (AGN), has gained considerable traction in
recent years both theoretically \citep{2006MNRAS.370..645B,
2006MNRAS.365...11C} and observationally \citep{2003ApJ...590..207P,
2007ARA&A..45..117M}. The cluster community therefore is very
interested in better undestanding the feedback mechanisms that act to
retard the formation of a continuous cooling gas phase in cluster
cores. But while several robust models for heating the ICM via AGN
feedback now exist, the details of the feedback loop remain
undetermined.

To garner this better understanding of how feedback regulates the
formation and growth of galaxies, it is necessary to further study the
means by which a feedback source quenches star formation. A fruitful
area of this work has been to study the interaction of feedback
sources with hot atmosphere's of galaxy clusters. Using the properties
of X-ray cavities in the intracluster medium (ICM) in combination with
the energetics of the radio source, \cite{2004ApJ...607..800B}
demonstrated that AGN do provide the necessary energy to retard
cooling in the cores of clusters. This in turn should indicate AGN are
capable of quenching star formation in the giant elliptical which
resides at the bottom of the potential well for many clusters. All
indications are that the process of making ``dead and red''
ellipticals involves AGN feedback.

In our group's previous observational work \citep{2005ApJ...630L..13D,
2006ApJ...643..730D, accept2} we focused on studying ICM entropy
as a means for understanding the cooling and heating processes in
clusters. We focus on entropy as it is a more fundamental property of
the ICM than temperature or density alone. The temperature of the ICM
simply reflects the depth and shape of the dark matter potential well,
while density is a representation of how much the well can
compress that gas. Entropy on the other hand determines the density of
gas in an isobaric system, and only heating or cooling can change the
entropy. Thus, when one studies the entropy distribution, they are
seeing the thermal history of the cluster.

In the second paper of this series (\cite{accept2}, hereafter Paper II)
we present the radial entropy profiles for a sample of $+200$ clusters
taken from the \Chandra\ Data Archive. There are two results in Paper
II which are relevant to this letter: a universal departure of the
radial entropy distribution from a self-similar power-law and a bimodal
central entropy distribution. In this letter we further
examine the relationship between the expected by-products of cooling
and resulting feedback, \eg\ star formation and radio-loud AGN, to the
central entropy value of the cluster. We define central entropy as the
parametric scale, \kna in the equation $K(r) = K_0 +K_{100}(r/100
\kpc)^{\alpha}$, at which the entropy distribution departs from
a power-law. This is not always the minimum core entropy, nor is it
the gas entropy which would be measured immediately around the AGN, it
is a measure of the average entropy within the cluster core -- which
can be anywhere from 10-100 kpc.

To determine the presence of feedback we selected two abundant and
robust surrogates: \halpha\ emission for star formation and radio-loud
sources for AGN. Using these surrogates as a window on the processes
operating in the core of the \accept\ clusters, we find that there
exists a critical entropy level below which star formation and AGN are
almost always active, while above this threshold, the surrogates
indicate feedback has slowed and in most cases ceased completely. This
entropy threshold also coincides with a theoretical limit for
establishing conductive stability in a cluster core. In the light of
these results, we suggest that low entropy ICM environments, along with
electron thermal conduction, play a vital role in dictating the
feedback active within a cluster. Any model which seeks to explain the
global properties of massive galaxies, and possibly intermediate and
low mass galaxies, may need to incorporate conduction as a means of
establishing a self-regulatory feedback loop and thus regulatiing
growth.

This letter proceeds in the following fashion:
In \S\ref{sec:data} we cover the basics of our data analysis. The
entropy-star formation relationship is discussed in \S\ref{sec:sf},
while the entropy-AGN correlation is threshed out in
\S\ref{sec:agn}. Discussion of results and conclusions are presented
in \S\ref{sec:diss}.  For this letter we have assumed a flat \LCDM\
Universe with cosmogony $\OM=0.3$, $\OL=0.7$, and
$\Hn=70\km\ps\pMpc$. All uncertainties are 90\% confidence.

%%%%%%%%%%%%%%%%
\section{Data}
\label{sec:data}
%%%%%%%%%%%%%%%%

We only briefly describe our data reduction and methods for producing
entropy profiles in this letter and refer interested readers to the
more thorough explanations of our methods in
\cite{2006ApJ...643..730D}, \cite{xrayband}, and the forthcoming
Paper II.

%%%%%%%%%%%%%%%%%%
\subsection{X-ray}
\label{sec:xray}
%%%%%%%%%%%%%%%%%%

Publicly available X-ray data was taken from the \Chandra\ Data
Archive. Following standard \Ciao\ reduction
techniques\footnote{http://cxc.harvard.edu/ciao/guides/}, data was
reprocessed using {\tt CIAO 3.4.1} and {\tt CALDB 3.4.0} resulting in
point source and flare clean events files at level-2. Entropy profiles
were derived using the relation $K(r) = T(r) \nelec(r)^{-2/3}$ where
$T(r)$ is the radial temperature and $\nelec(r)$ is the radial
electron gas density. A parametric form is then fit to each profile
and the central entropy for each cluster is taken from the best-fit.

Radial temperature profiles were created by spatially dividing each
cluster into concentric annuli with the minimum requirement of three 
annuli containing 2500 counts. Source spectra were then extracted from
these annuli, while corresponding background spectra were extracted from
deep blank-sky backgrounds tailored to match each
observation. Weighted responses which account for spatial variations
of the CCD calibration were also created for each spectrum. Spectra
were then fit over the energy range 0.7-7.0 keV in \xspec\ 11.3.2ag
using a single-component absorbed thermal model.

Exposure-corrected, background-subtracted, point source clean surface
brightness profiles were then extracted from $5\arcsec$ concentric
annular bins over the energy range 0.7-2.0 keV. Surface brightness was
converted to electron density using the deprojection technique of
\cite{1983ApJ...272..439K} in conjunction with the spectroscopic
normalization and 0.7-2.0 keV count rate. Errors were estimated using
5000 Monte Carlo realizations of the surface brightness profiles.

An entropy profile for each cluster was produced from the temperature
and electron density radial profiles. The entropy profiles were fit
with a hybrid model which is a power-law at large radii and has a
central entropy pedestal at small radii:
\begin{equation}
K(r) = \kna + \khun\ \left(\frac{r}{100 \kpc}\right)^{\alpha} \nonumber \\
\end{equation}
where \kna\ is the central entropy and \khun\ is a normalization of
the power-law component at 100 \kpc. In this letter we define central
entropy as the \kna\ term from the best-fit model. In Paper II we
demonstrate that the best-fit \kna\ and observational \kna\ do not
significantly differ for most clusters.

As noted previously, \kna\ is not the minimum gas entropy found in the
core. Instead \kna\ is a parametric reprsentation of where the entropy
distribution departs from a power-law, and is a characterization of
the average entropy of the core. It is well known that very small ($r
< 4 \kpc$), cool ($T_X < 1$ keV), high-density ($n_e > 0.1 \pcmsq$)
X-ray coronae are found embedded in the cores of some galaxies
\citep{2007ApJ...657..197S}. These coronae are difficult to classify
and sometimes may be indistinguishable from an AGN point
source. Without intent, we have excluded coronae from our analysis as
point sources, and as they are very low entropy systems we cannot
rightly refer to \kna\ as the lowest or absolute central entropy.

%%%%%%%%%%%%%%%%%%%%
\subsection{\halpha}
\label{sec:ha}
%%%%%%%%%%%%%%%%%%%%

\halpha\ values are taken from several sources, most notably
\cite{1999MNRAS.306..857C}. Alternative sources are donahue,
\cite{ir_quillen}, \cite{heckman89}, \cite{1996ApJS..107..541L},
\cite{1997MNRAS.292..419W}, and \cite{2005MNRAS.363..216C}. To place
all the \halpha\ measurements on equal footing, we have rolled back
from the cosmogony used in each study cited and re-calculated \halpha
luminosities using our assumed \LCDM\ values.

The \halpha\ values listed in this paper are not meant for strict
interpretation of star formation rates, \eg\ using the Kennicutt
relation \citep{kennicuttrelation}. In actuality, the quantitative
values are nearly meaningless for the purposes of this paper. For a
nearby cluster, \halpha\ imaging or slit-spectroscopy may only probe a
small portion of the entire central galaxy. Hence the global star
formation rate will be underestimated. In addition, other star
formation indicators in the near-UV, mid-IR, or PAH emission lines are
necessary to place tight constraints on the true star formation rate
\citep{2007ApJ...666..870C}.

We suggest readers interpret the \halpha\ measures as a ``toggle'',
either star formation is off or on. One may also ask if by focusing
only on \halpha\ we are missing some star formation, for example in
very dense regions which are highly obscured by dust, and hence we are
biased to finding star formation only in BCGs which meet a very
particular set of criteria. The answer is probably yes, but the extent
to which those missing star formation regions would change the results
presented in this letter is expected to be very small. 

%%%%%%%%%%%%%%%%%%
\subsection{Radio}
\label{sec:radio}
%%%%%%%%%%%%%%%%%%

We take advantage of the nearly all-sky flux-limited coverage of the
NRAO VLA Sky Survey (NVSS, \citealt{1998AJ....115.1693C}) to look for
radio emission from the central region of each cluster in
\accept. NVSS is a continuum survey at 1.4 GHz of the entire sky north
of $\delta = -40\degr$. The completeness limit of NVSS is $\approx
2.5$ mJy so the survey reaches to very low radio luminosities and is
excellent for our purposes.

Radio sources were determined to be associated with the cluster core in
two ways. The first method was to search for sources within a fixed
angular distance of $20\arcsec$ around the cluster X-ray peak. The
probability of finding an unrelated radio source within an aperture of
$20\arcsec$ is exceedingly low ($< 0.004$)
\citep{1998AJ....115.1693C}, thus in a sample of $\approx 200$
clusters we expect no more than one false match. The second method
involved searching within $20\kpc$ of the X-ray peak for sources. At
$z \approx 0.51$, $1 \kpc \approx 1''$ thus for clusters above this
redshift the 20 kpc aperture becomes smaller than the $20''$
aperture. A comparison of the search results shows that the $20\kpc$
search radius rarely (frequency $< 2\%$) finds fewer or more sources
than the $20\arcsec$ aperture. In cases of a missed source, it is
typically the extended lobes of very a large source (\eg\ Hydra A),
while the additional sources are radio galaxies near the BCG. For each
cluster, a final by-eye check of the radio emission in relation to the
optical (using DSS) and infrared (using 2MASS) emission of the BCG is
made. The NED\footnote{http://nedwww.ipac.caltech.edu/} database was also
queried with the central coordinates of each radio source to ensure
that for those sources which have been cataloged, their redshift was
within the limits of the cluster redshift thus further diminishing the
chance superposition of background/foreground radio sources or
mistakenly attributing emission to the BCG which is from a neighboring
galaxy.

We again stress that the purpose here is not to calculate accurate
radio luminosities, but is instead to make a check for the presence or
absence of radio emission. We are simply seeking to assess if radio
emission, and hence AGN, obey an entropy toggle similar to star
formation. We calculate the power at $1.4\GHz$ for each radio source
using the standard relation $\nu L_{\nu} = 4 \pi D_L^2 S_{\nu} f_0$
where $S_{\nu}$ is the 1.4 GHz flux from NVSS, $D_L$ is the luminosity
distance in cm, and $f_0$ is the frequency of the survey.

%Following the methodology of \cite{2004ApJ...607..800B}, we integrate
%the radio power to obtain the luminosity,
%\begin{equation}
%L_{\mathrm{radio}} = \int_{\nu_1}^{\nu_2} 4 \pi D_L^2 S_{\nu_0} \left(\frac{\nu}{\nu_0}\right)^{-\alpha} d\nu
%\end{equation}
%where we've assumed a power-law spectrum of $S_\nu \sim
%\nu^{-\alpha}$, $\nu_0 = 1.4 \GHz$, $\nu_1 = 10.0 \MHz$, $\nu_2 = 5.0
%\GHz$, $S_{\nu_0}$ is the 1.4 GHz flux from NVSS, and $D_L$ is the
%luminosity distance. For clusters where the spectral index $\alpha$
%was available, we used the value in calculating
%$L_{\mathrm{radio}}$. Otherwise, we have assumed $\alpha = 1$. This is
%again not the most robust method for determining the radio luminosity,
%but as with the \halpha\ measures, we are more interested in the
%qualitative behavior of cluster core radio sources -- either on or
%off.

The resolution limitations of NVSS present a unique constraint. If we
desired to make a strict comparison of radio sources in cluster cores,
then radio morphology would be very useful. However, NVSS has a
resolution of $45\arcsec$ at FWHM. We therefore cannot distinguish
between radio point sources, lobes, relics, ghosts, detached lobes, and
re-accelerated regions. So a comparison of only sources which are
point-like or lobed is not possible. But our focus is not this
specific, we are only interested if the radio emission is most likely
coming from the BCG or not.

%%%%%%%%%%%%%%%%%%%%%%%%%%%%%%%%%%%%%%%%%%%%
\section{Star Formation and Central Entropy}
\label{sec:sf}
%%%%%%%%%%%%%%%%%%%%%%%%%%%%%%%%%%%%%%%%%%%%

Of the XXX clusters in \accept, we have located \halpha\ luminosities
from the literature for 97 clusters. Of those 97: 41 have \halpha\
detections and 56 do not. We treat the non-detections as upper-limits
using the observational flux-limit. The mean \kna\ for clusters
with detections is $\kna = 13.6 \pm 4.72 \ent$, for clusters
with only upper-limits $\kna = 118 \pm 46.6 \ent$. In Figure
\ref{fig:ha} is plotted \lha\ versus \kna. One can immediately see
there is a stark dichotomy between clusters with and without \halpha\
emission. If a cluster has a central entropy $\lesssim 30 \ent$ then
star formation is ``on'', while above this threshold the star
formation is almost universally ``off''. For brevity we refer to this
threshold as \kthr\ hereafter. The only two exceptions being
RX J1000.4+4409 and Zwicky 2701 which squeak by at $\kna = 35.8 \pm
3.8 \ent$ and $\kna = 45.6 \pm 4.8 \ent$, respectively.

In Paper II we present the result that the \kna\ distribution for
\accept\ is bimodal (shown in Figure \ref{fig:k0hist}). The clusters
which populate the peak around $\kna \approx 15 \ent$ are clusters
with short central cooling times ($\tcool \ll \Hn^{-1}$) and would be
classified as ``classic cooling flows''. It does not come as a
surprise that modest star formation is occurring in these BCGs, but it
is very interesting that there appears to be a characteristic entropy
threshold for this group of clusters only below which multi-phase gas,
and presumably stars, form. Even more fortuitous is that in
\cite{2005ApJ...630L..13D} and in the upcoming paper
\cite{conduction}, the prediction was made that if thermal electron
conduction is an important mechanism in some cluster cores, then the
entropy distribution should ``bifurcate''. The model presented in
\cite{conduction} proposes that above a certain entropy threshold,
which we find observationally to be $\approx 30 \ent$, conductive
stability sets-in and replaces the heat removed by radiative
cooling. This process may result in suppression of multiphase gas
and promoting homogeneity. We will return to this point in
\S\ref{sec:diss}.

The exceptions to this entropy threshold are: Abell 2029 ($\kna = 10.5
\ent$), Abell 2151 ($\kna = 10.1 \ent$), EXO 0422-086 ($\kna = 13.8
\ent$), and RBS 533 ($\kna = 2.6 \ent$). These clusters are shown in Fig.
\ref{fig:ha} as blue boxes with red stars. The clusters A2029, A2151,
E422, and R533 all reside at the lowest end of \lha. We do not have a
convincing explanation for why these clusters do not have \halpha\
emission although the star forming regions could be very small or
diffuse and lie below the flux-limit of available surveys.

We also note the lack of high \halpha\ luminosity at $\kna \la 7
\ent$. This might be the result of resolution limititations. Or there
is the possibility that in a very low entropy environment, gas
overdensities are washed out from being continually subjected to
internal pressure changes via sinking and buoyantly rising. The gas
parcel never reaches an equilibrium state where rapid cooling is
promoted and multiphase gas can form.

This overview of star formation is by no means a complete and thorough
treatment. The \halpha\ surveys we draw from are most likely not
detecting star formation in the highest redshift BCGs and are also
missing star formation in the most nearby BCGs. But the mean and
dispersion of the redshifts for clusters with and without \halpha\ are
nearly identical, $\red = 0.116 \pm 0.103$ and $\red = 0.131 \pm
0.083$ respectively, and applying a redshift cut (\ie\ $red 0-0.1$ or
$\red 0.15-0.3$) does not make this dichotomy go away. Most
importantly however is that changes in the overall \halpha\ luminosity
are only going to move clusters left and right in Figure \ref{fig:ha},
mobility in the \kna\ plane is negligible. Qualitatively, the
correlation between low central entropy and \halpha\ emission is very
robust.

%%%%%%%%%%%%%%%%%%%%%%%%%%%%%%%%%%%%%%%%%%%%
\section{Radio-loud AGN and Central Entropy}
\label{sec:agn}
%%%%%%%%%%%%%%%%%%%%%%%%%%%%%%%%%%%%%%%%%%%%

Of the XXX clusters in \accept\, only YY fall outside the NVSS survey
area ($\delta \geq -40\degr$) giving us excellent coverage of the
entire sample. In this section we focus on radio sources identified
within $20\kpc$ of the X-ray peak for each cluster. Of the 191
clusters, 90 have radio source detections with a mean $\kna =
16.5^{+23.7}_{-9.74} \ent$, while the other 101 clusters with
only upper-limits have a mean $\kna = 96.1^{+165}_{-60.8} \ent$. In
Figure \ref{fig:radall} we have plotted radio power versus \kna. The
obvious dichotomy present in the \halpha\ figure is not as clear-cut
for the radio sources, although with the overabundance of clusters at
$\lradio \gtrsim 10^{42} \ergps$ and $\kna \la \kthr$ it is still
present.

The weaker adherence to the \kthr\ for the radio
sources can partially be explained by systematics. NVSS has a low
angular resolution of $45\arcsec$ at full width at half maximum. At
$z = 0.3$ this resolution is $\approx 200\kpc$ and at $\red = 0.2$ it
is $\approx 150\kpc$, both of which are much larger then the scale
size of a typical cluster cooling region. At these physical scales we
have nearly no information about the radio source morphology and thus
are not able to distinguish between AGN point sources, extended radio
lobes, detached lobes, ghosts, relics, and even diffuse, amorphous
re-accelerated regions. Being blind to morphological distinction, we
expect larger scatter in the radio-\kna\ relation than if we were to
only focus on, for example, FRII radio sources.

It is still very interesting however that the most powerful radio
sources all reside below the same \kthr\ seen in the \kna-\halpha\
relation. This suggests that AGN activity in BCGs, while not
exclusively limited to low core entropy clusters, still favors
clusters which reside below the theoretical conduction stability
limit. When we make a cut in redshift space and only consider clusters
at $\red \leq 0.2$, we find the entropy-AGN relation to be more
pronounced (see Figure \ref{fig:radzcut}). After the redshift cut,
there are 119 total clusters remaining -- 57 with radio detections
(mean $\kna = 16.5 \pm 6.40 \ent$) and 62 without (mean $\kna =
104 \pm 42.8 \ent$). Ignoring clusters above \kthr, the mean of
clusters with radio detections becomes $\kna = 11.7 \pm 4.16
\ent$. That the entropy threshold at $\kna \approx 30 \ent$ is robust
against the cut in redshift space is important.

We have again highlighted two subsets of clusters in
Figs. \ref{fig:radall} and \ref{fig:radzcut}: clusters below \kthr\
without radio sources (blue boxes with red stars) and clusters above
\kthr\ with radio sources (blue boxes with orange dots). The peculiar
clusters below \kthr\ are A133, A744, A1060, A1204, A2107, A2556,
AWM7, ESO 5520200, MKW4, MS J1157.3+5531, and MACS J1931.8-2634. The
peculiar clusters above \kthr\ are A193, A586, A1942, A2063, A2244,
A3558, A4038, MACS J0520.7-1328, MACS J1206.2-0847, MACS J2245.0+2637,
RBS 461, RX J0528.9-3927, and Zwicky 2701. Radio-quiet AGN are not
uncommon, and having a few clusters in our sample without radio-loud
sources where we expect to find AGN activity is not
surprising. However, the clusters above \kthr\ with a radio-loud
source are surprising, and may be special cases of clusters with a
corona in the BCG. These coronae are like ``mini-cooling cores'' and
could provide the environment necessary to provide cool gas as fuel
for an AGN. This topic however is beyond the scope of this
letter. These peculiar clusters do not weaken the correlation of core
entropy with the presence of radio-loud sources.

%AWM 4 \citep{2008ApJ...673L..17G}

%%%%%%%%%%%%%%%%%%%%
\section{Discussion}
\label{sec:diss}
%%%%%%%%%%%%%%%%%%%%

It has been previously pointed out by \cite{2005ApJ...630L..13D} and
detailed recently by \cite{conduction} that there should exist a
critical entropy profile for which electron thermal conduction is
capable of balancing the energy losses from radiative cooling. If we
assume heat conduction between a gas parcel and its surroundings is
proceeding at the Spitzer rate, and we further assume the dominant
cooling mechanism is free-free emission, then one can derive the
size of the radiative cooling region specified by the Field length,
\begin{equation}
\lambda_F = \left(\frac{\kappa_S T}{n_e^2 \Lambda(T,Z)}\right)^{1/2} \approx 4 \kpc \left(\frac{K}{10 \ent}\right)^{3/2}f_c^{1/2}, \nonumber
\end{equation}
where $\kappa_S$ is the Spitzer conduction coefficient ($\kappa_S =
6\times10^{-7} f_c T^{5/2} \ergps \pcm \mathrm{~K}^{-7/2}$), \nelec
is electron gas density, $T$ is temperature, $\Lambda(T,Z)$ is the
cooling function at a given temperature and metal abundance, and $f_c$
is a suppression factor. By a coincidence of scaling, the Field length
is a function of entropy only, $\lambda_F \propto K^{3/2}$. The
critical entropy profile which separates conductively stable systems
from unstable systems is then $K(r) \approx 10 \ent f_c^{-1/3}
(r/4\kpc)^{2/3}$ \citep{2005ApJ...630L..13D}.

Clusters below this critical entropy profile will be capable of
supporting a multi-phase medium because thermal conduction cannot wipe
out multiphase gas. Hence, stars can form and small streams of
gas can work their way onto the SMBH in the BCG and initiate
an AGN feedback cycle. However, above the critical entropy profile,
conduction proceeds faster than cooling and any cool gas clouds which
may form or be deposited in the ICM will evaporate. We suggest that
our finding of a robust correlation between indicators of feedback and
\kna\ values below this critical threshold strongly support the
hypothesis that electron thermal conduction is an important mechanism
in the cluster feedback cycle. Recently \cite{2008arXiv0802.1864R}
found that galaxies in low entropy environments have strong blue
gradients (another indicator of active/recent star formation), a
result which further supports our hypothesis.

We propose that within the framework of an episodic AGN feedback
model which includes thermal conduction as a form of feedback, that
our observation of the bimodal central entropy distribution is well
explained as part of a larger entropy life-cycle. Let us imagine a
cluster which has been left in isolation for a Gyr or so. AGN
outbursts with power outputs of $10^{45} \ergps$ lasting a few tens of
Myrs and occurring every few hundred Myrs are capable of supplying the
energy to nicely reproduce the range and size of the entropy pedestals
($< 100 \ent$) in \accept. However, on the rare occasion that a large
AGN outburst occurs ($E > 10^{61} \ergps$), the core entropy can be
pushed above $30 \ent$. At these entropy levels the cluster becomes
conductively stable. Unable to cool, mergers can quickly remove
$40-60 \ent$ clusters, leaving a distinct gap in the \kna\
distribution and creating a second population with $\kna > 100
\ent$. A detailed theoretical investigation of the role conduction
plays in the entropy lifecycle of a cluster is a warranted next step.

%%%%%%%%%%%%%%%%%%%%%%
\acknowledgements
%%%%%%%%%%%%%%%%%%%%%%

\acknowledgements
KWC was supported in this work by NASA through \Chandra\ grants
AR-6016X and AR-4017A, with additional support from an MSU start-up
grant for Megan Donahue. Megan Donahue and Michigan State University
acknowledge support from the NASA LTSA program NNG-05GD82G.  We thank
Brian McNamara, David Rafferty, and Paul Nulsen for comparing data and
stimulating discussion on this topic. The CXC is operated by the SAO
for and on behalf of NASA under contract NAS8-03060. This research has
made use of software provided by the CXC in the application packages
\Ciao, \chips, and \sherpa. This research has made use of the
NASA/IPAC Extragalactic Database which is operated by JPL, California
Institute of Technology, under contract with NASA. This research has
also made use of NASA's Astrophysics Data System.

%%%%%%%%%%%%%%%%
% Bibliography %
%%%%%%%%%%%%%%%%

\bibliography{cavagnolo}

%%%%%%%%%%%%%%%%%%%%%%
% Figures  and Tables%
%%%%%%%%%%%%%%%%%%%%%%

\clearpage
\begin{figure}
  \begin{center}
    \begin{minipage}{\linewidth}
      \includegraphics*[width=\textwidth, trim=0mm 0mm 0mm 0mm, clip]{rbs797.ps}
    \end{minipage}
    \caption{Fluxed, unsmoothed 0.7--2.0 keV clean image of \rbs\ in
      units of ph \pcmsq\ \ps\ pix$^{-1}$. Image is $\approx 250$ kpc
      on a side and coordinates are J2000 epoch. Black contours in the
      nucleus are 2.5--9.0 keV X-ray emission of the nuclear point
      source; the outer contour approximately traces the 90\% enclosed
      energy fraction (EEF) of the \cxo\ point spread function. The
      dashed green ellipse is centered on the nuclear point source,
      encloses both cavities, and was drawn by-eye to pass through the
      X-ray ridge/rims.}
    \label{fig:img}
  \end{center}
\end{figure}

\begin{figure}
  \begin{center}
    \begin{minipage}{0.495\linewidth}
      \includegraphics*[width=\textwidth, trim=0mm 0mm 0mm 0mm, clip]{325.ps}
    \end{minipage}
   \begin{minipage}{0.495\linewidth}
      \includegraphics*[width=\textwidth, trim=0mm 0mm 0mm 0mm, clip]{8.4.ps}
   \end{minipage}
   \begin{minipage}{0.495\linewidth}
      \includegraphics*[width=\textwidth, trim=0mm 0mm 0mm 0mm, clip]{1.4.ps}
    \end{minipage}
    \begin{minipage}{0.495\linewidth}
      \includegraphics*[width=\textwidth, trim=0mm 0mm 0mm 0mm, clip]{4.8.ps}
    \end{minipage}
     \caption{Radio images of \rbs\ overlaid with black contours
       tracing ICM X-ray emission. Images are in mJy beam$^{-1}$ with
       intensity beginning at $3\sigma_{\rm{rms}}$ and ending at the
       peak flux, and are arranged by decreasing size of the
       significant, projected radio structure. X-ray contours are from
       $2.3 \times 10^{-6}$ to $1.3 \times 10^{-7}$ ph
       \pcmsq\ \ps\ pix$^{-1}$ in 12 square-root steps. {\it{Clockwise
           from top left}}: 325 MHz \vla\ A-array, 8.4 GHz
       \vla\ D-array, 4.8 GHz \vla\ A-array, and 1.4 GHz
       \vla\ A-array.}
    \label{fig:composite}
  \end{center}
\end{figure}

\begin{figure}
  \begin{center}
    \begin{minipage}{0.495\linewidth}
      \includegraphics*[width=\textwidth, trim=0mm 0mm 0mm 0mm, clip]{sub_inner.ps}
    \end{minipage}
    \begin{minipage}{0.495\linewidth}
      \includegraphics*[width=\textwidth, trim=0mm 0mm 0mm 0mm, clip]{sub_outer.ps}
    \end{minipage}
    \caption{Red text point-out regions of interest discussed in
      Section \ref{sec:cavities}. {\it{Left:}} Residual 0.3-10.0 keV
      X-ray image smoothed with $1\arcs$ Gaussian. Yellow contours are
      1.4 GHz emission (\vla\ A-array), orange contours are 4.8 GHz
      emission (\vla\ A-array), orange vector is 4.8 GHz jet axis, and
      red ellipses outline definite cavities. {\it{Bottom:}} Residual
      0.3-10.0 keV X-ray image smoothed with $3\arcs$ Gaussian. Green
      contours are 325 MHz emission (\vla\ A-array), blue contours are
      8.4 GHz emission (\vla\ D-array), and orange vector is 4.8 GHz
      jet axis.}
    \label{fig:subxray}
  \end{center}
\end{figure}

\begin{figure}
  \begin{center}
    \begin{minipage}{\linewidth}
      \includegraphics*[width=\textwidth]{r797_nhfro.eps}
      \caption{Gallery of radial ICM profiles. Vertical black dashed
        lines mark the approximate end-points of both
        cavities. Horizontal dashed line on cooling time profile marks
        age of the Universe at redshift of \rbs. For X-ray luminosity
        profile, dashed line marks \lcool, and dashed-dotted line
        marks \pcav.}
      \label{fig:gallery}
    \end{minipage}
  \end{center}
\end{figure}

\begin{figure}
  \begin{center}
    \begin{minipage}{\linewidth}
      \setlength\fboxsep{0pt}
      \setlength\fboxrule{0.5pt}
      \fbox{\includegraphics*[width=\textwidth]{cav_config.eps}}
    \end{minipage}
    \caption{Cartoon of possible cavity configurations. Arrows denote
      direction of AGN outflow, ellipses outline cavities, \rlos\ is
      line-of-sight cavity depth, and $z$ is the height of a cavity's
      center above the plane of the sky. {\it{Left:}} Cavities which
      are symmetric about the plane of the sky, have $z=0$, and are
      inflating perpendicular to the line-of-sight. {\it{Right:}}
      Cavities which are larger than left panel, have non-zero $z$,
      and are inflating along an axis close to our line-of-sight.}
    \label{fig:config}
  \end{center}
\end{figure}

\begin{figure}
  \begin{center}
    \begin{minipage}{0.495\linewidth}
      \includegraphics*[width=\textwidth, trim=25mm 0mm 40mm 10mm, clip]{edec.eps}
    \end{minipage}
    \begin{minipage}{0.495\linewidth}
      \includegraphics*[width=\textwidth, trim=25mm 0mm 40mm 10mm, clip]{wdec.eps}
    \end{minipage}
    \caption{Surface brightness decrement as a function of height
      above the plane of the sky for a variety of cavity radii. Each
      curve is labeled with the corresponding \rlos. The curves
      furthest to the left are for the minimum \rlos\ needed to
      reproduce $y_{\rm{min}}$, \ie\ the case of $z = 0$, and the
      horizontal dashed line denotes the minimum decrement for each
      cavity. {\it{Left}} Cavity E1; {\it{Right}} Cavity W1.}
    \label{fig:decs}
  \end{center}
\end{figure}


\begin{figure}
  \begin{center}
    \begin{minipage}{\linewidth}
      \includegraphics*[width=\textwidth, trim=15mm 5mm 5mm 10mm, clip]{pannorm.eps}
      \caption{Histograms of normalized surface brightness variation
        in wedges of a $2.5\arcs$ wide annulus centered on the X-ray
        peak and passing through the cavity midpoints. {\it{Left:}}
        $36\mydeg$ wedges; {\it{Middle:}} $14.4\mydeg$ wedges;
        {\it{Right:}} $7.2\mydeg$ wedges. The depth of the cavities
        and prominence of the rims can be clearly seen in this plot.}
      \label{fig:pannorm}
    \end{minipage}
  \end{center}
\end{figure}

\begin{figure}
  \begin{center}
    \begin{minipage}{0.5\linewidth}
      \includegraphics*[width=\textwidth, angle=-90]{nucspec.ps}
    \end{minipage}
    \caption{X-ray spectrum of nuclear point source. Black denotes
      year 2000 \cxo\ data (points) and best-fit model (line), and red
      denotes year 2007 \cxo\ data (points) and best-fit model (line).
      The residuals of the fit for both datasets are given below.}
    \label{fig:nucspec}
  \end{center}
\end{figure}

\begin{figure}
  \begin{center}
    \begin{minipage}{\linewidth}
      \includegraphics*[width=\textwidth, trim=10mm 5mm 10mm 10mm, clip]{radiofit.eps}
    \end{minipage}
    \caption{Best-fit continuous injection (CI) synchrotron model to
      the nuclear 1.4 GHz, 4.8 GHz, and 7.0 keV X-ray emission. The
      two triangles are \galex\ UV fluxes showing the emission is
      boosted above the power-law attributable to the nucleus.}
    \label{fig:sync}
    \end{center}
\end{figure}

\begin{figure}
  \begin{center}
    \begin{minipage}{\linewidth}
      \includegraphics*[width=\textwidth, trim=0mm 0mm 0mm 0mm, clip]{rbs797_opt.ps}
    \end{minipage}
    \caption{\hst\ \myi+\myv\ image of the \rbs\ BCG with units e$^-$
      s$^{-1}$. Green, dashed contour is the \cxo\ 90\% EEF. Emission
      features discussed in the text are labeled.}
    \label{fig:hst}
  \end{center}
\end{figure}

\begin{figure}
  \begin{center}
    \begin{minipage}{0.495\linewidth}
      \includegraphics*[width=\textwidth, trim=0mm 0mm 0mm 0mm, clip]{suboptcolor.ps}
    \end{minipage}
    \begin{minipage}{0.495\linewidth}
      \includegraphics*[width=\textwidth, trim=0mm 0mm 0mm 0mm, clip]{suboptrad.ps}
    \end{minipage}
    \caption{{\it{Left:}} Residual \hst\ \myv\ image. White regions
      (numbered 1--8) are areas with greatest color difference with
      \rbs\ halo. {\it{Right:}} Residual \hst\ \myi\ image. Green
      contours are 4.8 GHz radio emission down to
      $1\sigma_{\rm{rms}}$, white dashed circle has radius $2\arcs$,
      edge of ACS ghost is show in yellow, and southern whiskers are
      numbered 9--11 with corresponding white lines.}
    \label{fig:subopt}
  \end{center}
\end{figure}


%%%%%%%%%%%%%%%%%%%%
% End the document %
%%%%%%%%%%%%%%%%%%%%
\end{document}
