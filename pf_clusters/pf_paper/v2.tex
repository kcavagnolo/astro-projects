% Basic commands
\newcommand{\accept}{\textit{ACCEPT}}
\newcommand{\hifl}{\textit{HIFLUGCS}}
\newcommand{\numobs}{305}
\newcommand{\numcluster}{222}
\newcommand{\expt}{9.58 Msec}

% K0 stats for full sample
\newcommand{\alphafs}{\ensuremath{\alpha = 1.21 \pm 0.39}}
\newcommand{\knafs}{\ensuremath{\kna = 71.1 \pm 32.9 \ent}}
\newcommand{\khunfs}{\ensuremath{\khun = 127 \pm 45 \ent}}
% K0 < 50
\newcommand{\alphaga}{\ensuremath{\alpha = 1.19 \pm 0.38}}
\newcommand{\knaga}{\ensuremath{\kna = 16.0 \pm  5.7 \ent}}
\newcommand{\khunga}{\ensuremath{\khun = 151 \pm 50 \ent}}
% K0 > 50
\newcommand{\alphagb}{\ensuremath{\alpha = 1.23 \pm 0.40}}
\newcommand{\knagb}{\ensuremath{\kna = 154 \pm 53 \ent}}
\newcommand{\khungb}{\ensuremath{\khun = 108 \pm 39 \ent}}

% K0 stats for cent src clusters
\newcommand{\centsrcnum}{\ensuremath{28}}
\newcommand{\alphacs}{\ensuremath{\alpha = 1.22 \pm 0.39}}
\newcommand{\knacs}{\ensuremath{\kna = 39.5 \pm 16.6 \ent}}
\newcommand{\khuncs}{\ensuremath{\khun = 140 \pm 47 \ent}}
% K0 < 50
\newcommand{\alphacsa}{\ensuremath{\alpha = 1.16 \pm 0.38}}
\newcommand{\knacsa}{\ensuremath{\kna = 15.6 \pm 5.2 \ent}}
\newcommand{\khuncsa}{\ensuremath{\khun = 146 \pm 48 \ent}}
% K0 > 50
\newcommand{\alphacsb}{\ensuremath{\alpha = 1.33 \pm 0.44}}
\newcommand{\knacsb}{\ensuremath{\kna =  136 \pm 44 \ent}}
\newcommand{\khuncsb}{\ensuremath{\khun = 122 \pm 45 \ent}}

% KMM test for all clusters
\newcommand{\kmma}{\ensuremath{K_1 = 17.8 \pm 6.6 \ent}}
\newcommand{\kmmb}{\ensuremath{K_2 = 154 \pm 52 \ent}}
\newcommand{\kmmc}{\ensuremath{118}}
\newcommand{\kmmd}{\ensuremath{104}}
\newcommand{\kmme}{\ensuremath{p = 1.16\times10^{-7}}}
% KMM test w/o kna < 4
\newcommand{\kmmf}{\ensuremath{K_1 = 15.0\pm 5.0 \ent}}
\newcommand{\kmmg}{\ensuremath{K_2 = 129 \pm 45 \ent}}
\newcommand{\kmmh}{\ensuremath{87}}
\newcommand{\kmmi}{\ensuremath{126}}
\newcommand{\kmmj}{\ensuremath{p = 1.90\times10^{-13}}}

% Hifl stats
\newcommand{\hifla}{\ensuremath{\alpha = 1.19 \pm 0.38}}
\newcommand{\hiflb}{\ensuremath{\kna = 11.8 \pm 4.3 \ent}}
\newcommand{\hiflc}{\ensuremath{\khun = 241 \pm 92 \ent}}
\newcommand{\hifld}{\ensuremath{\alpha = 1.15 \pm 0.38 \ent}}
\newcommand{\hifle}{\ensuremath{\kna = 149 \pm 52 \ent}}
\newcommand{\hiflf}{\ensuremath{\khun = 116 \pm 42 \ent}}
% HIFL KMM for all
\newcommand{\hiflkmma}{\ensuremath{K_1 = 10.4 \pm 3.8 \ent}}
\newcommand{\hiflkmmb}{\ensuremath{K_2 = 133 \pm 46 \ent}}
\newcommand{\hiflkmmc}{\ensuremath{28}}
\newcommand{\hiflkmmd}{\ensuremath{28}}
\newcommand{\hiflkmme}{\ensuremath{p = 5.13\times10^{-3}}}
% HIFL KMM w/o kna < 4
\newcommand{\hiflkmmf}{\ensuremath{K_1 = 10.8 \pm 3.5 \ent}}
\newcommand{\hiflkmmg}{\ensuremath{K_2 = 112 \pm 40 \ent}}
\newcommand{\hiflkmmh}{\ensuremath{21}}
\newcommand{\hiflkmmi}{\ensuremath{31}}
\newcommand{\hiflkmmj}{\ensuremath{p = 3.21\times10^{-6}}}

% cooling time stats
\newcommand{\tckmma}{\ensuremath{t_{c1} = 0.58 \pm 0.23 \Gyr}}
\newcommand{\tckmmb}{\ensuremath{t_{c2} = 6.05 \pm 2.12 \Gyr}}
\newcommand{\tckmmc}{\ensuremath{128}}
\newcommand{\tckmmd}{\ensuremath{94}}
\newcommand{\tckmme}{\ensuremath{p = 1.59\times10^{-6}}}

%%%%%%%%%%
% Header %
%%%%%%%%%%

%\documentclass[12pt, preprint]{aastex}
\documentclass{emulateapj}
\usepackage{apjfonts,graphicx,here,lscape,common}
\bibliographystyle{apj}
\begin{document}
\title{Chandra Archival Sample of Intracluster Entropy Profiles}
\author{Kenneth. W. Cavagnolo\altaffilmark{1}, Megan Donahue, G. Mark
  Voit, and Ming Sun}
\affil{Department of Physics and Astronomy, BPS Building, Michigan
  State University, East Lansing, MI 48824}
\altaffiltext{1}{cavagnolo@pa.msu.edu}
\shorttitle{Archive of Entropy Profiles}
\shortauthors{K. W. Cavagnolo et al.}
%\journalinfo{Submitted to ApJ Supplement}
%\submitted{Submitted July XX, 2008}
%\accepted{Accepted July XX, 2008 }

%%%%%%%%%%%%
% Abstract %
%%%%%%%%%%%%

\begin{abstract}
%%   We present radial entropy profiles for a sample of
%%   \numcluster\ clusters taken from the \chandra\ X-ray Observatory's
%%   Data Archive. We have named this project the Archive of Chandra
%%   Cluster Entropy Profile Tables, or \accept\ for short. The entropy
%%   profiles are calculated from projected temperature profiles and
%%   electron density profiles derived from direct deprojection of the
%%   cluster surface brightness. We find that for most clusters the
%%   entropy profile is best-fit by a constant core entropy, \kna, plus a
%%   power-law normalized at 100 kpc: $K(r) = \kna + \khun (r/100
%%   \kpc)^{\alpha}$. In addition, we show that the distribution of
%%   \kna\ is bimodal with a distinct gap at $\kna \approx 30-60
%%   \ent$. We have also studied the \hifl\ sample of \cite{hiflugcs1}
%%   and find the trends of the full sample are more pronounced in this
%%   carefully selected, unbiased subsample. All of the data products
%%   associated with this project have been made available online, and we
%%   encourage their use by both theorists and observers.
And he says, ``Oh, uh, there won't be any money, but when you die, on
your deathbed, you will receive total consciousness.'' So I got that
goin' for me, which is nice.
\end{abstract}

%%%%%%%%%%%%
% Keywords %
%%%%%%%%%%%%

\keywords{conduction -- cooling flows -- galaxies: clusters: general
  -- X-rays: galaxies: clusters -- galaxies: evolution}

%%%%%%%%%%%%%%%%%%%%%%
\section{Introduction}
\label{sec:intro}
%%%%%%%%%%%%%%%%%%%%%%

The general process of galaxy cluster formation through hierarchical
merging is well understood, but many details, such as the impact of
feedback sources on the cluster environment and radiative cooling in
the cluster core, are not. The nature of feedback operating within
clusters is of great interest because of the implications for better
understanding massive galaxy formation and using mass-observable
scaling relations in cluster cosmological studies. Early theories of
hierarchical structure formation predicted clusters of galaxies should
be scaled versions of each other \citep{kaiser86, 1996ApJ...469..494E,
1997MNRAS.292..289E}. These models also predicted that the most
massive galaxies in the Universe -- brightest cluster galaxies
(BCGs)-- should be rife with young stellar populations and increase in
mass from their formation up to the present. However, observations
have long shown that clusters do not adhere to simple, low-scatter
scaling relations \citep{edge91, 1999ApJ...520...78H,
2000ApJ...536...73N, 2001A&A...368..749F}, and that BCGs are much
lower mass systems with old stellar populations than were predicted
with the theoretical models. Moreover, massive galaxies were in place
by redshifts of $\sim 1-2$ \citep{1996MNRAS.283.1388M,
1996Natur.384..439S} and have shown little sign of evolution since --
the so called cosmic down-sizing problem
\citep{1996AJ....112..839C}. But the problems with the theories
are starting to be patched as more physics are added to hierarchical
models, physics such as radiative cooling and feedback sources.

The core cooling time in many clusters is much shorter than both the
Hubble time and cluster age (\eg\ \citealt{1998MNRAS.298..416P}). Thus
it is reasonable to assume the cores of these clusters have gone
through bouts of prodigious radiative cooling and its associated
effects. One expected consequence of this radiative cooling is the
formation of massive cooling flows (see \citealt{fabiancfreview} for a
thorough review). But, measured cooling flow mass deposition rates
were found to be much lower than predicted, with the ICM never
reaching temperatures lower than $T_{virial}/3$ \citep{tamura01,
peterson01, peterson03}. The torrents of cool gas once thought to be
flowing toward the bottom of some cluster potentials turned out to
more like cooling trickles. In addition to the lack of soft X-ray line
emission from cooling flows, earlier methodical searches for the end
products of cooling flows (\ie\ molecular clouds and stars) revealed
far less mass was locked-up in these sources than expected
\citep{heckman89, mcnamara90, odea94, voit95}. These disconnects between
observation and theory have been termed ``the cooling flow problem''
and raise the question, ``Where has all the cool gas gone?''
Obviously, some source of energetic feedback has heated the ICM to
selectively remove gas with a short cooling time and establish
quasi-stable thermal balance.

Recent revisions to models of how clusters form and evolve by
including feedback sources, such as active galactic nuceli (AGN) and
supernovae, has led to better agreement between observation and theory
\citep{bower06, croton06, saro06}. The current paradigm holds that
energetic feedback from AGN outbursts in the form of ``bubbles''
deposit the requisite heat into the ICM to retard, and in some cases
quench, cooling of the ICM (see \citealt{mcnamrev} for a
review). Thanks to the high-resolution optics aboard \chandra,
previously unresolved ICM bubbles and cavities have indeed been
observed in numerous clusters. In one of the first high-resolution
X-ray studies of a sample of bubbles, \cite{birzan} found that AGN
feedback plausible supplies the energy necessary to quench cooling in
cluster cores. Many other bubble studies, including in groups and
ellipticals, have come to similar conclusions.

There is yet another consequence of core cooling and secondary heating
like AGN and supernovae: these processes temporarily decouple baryons
in the core from the dark matter potential and give rise to the
breaking of self-similarity. As a consequence of radiative cooling,
best-fit total cluster temperature decreases while total cluster
luminosity increases. Thus, at a given mass scale, radiative cooling
conspires to create dispersion in otherwise tight correlations between
mass-luminosity and mass-temperature. In addition, feedback sources
can drive cluster cores (where most the cluster flux originates) away
from hydrostatic equilibrium. Both of these effects result in larger
uncertainties when deriving cluster masses, particularly when using
mass proxies such as temperature or luminosity. This diminishes the
precision desired when using clusters for detailed cosmological
studies. While considerable progress has been made in this area both
observationally and theoretically \citep{1996ApJ...458...27B,
2005ApJ...624..606J, kravtsov06, nagai07, VV08}, it is still important
to understand how, taken as a whole, non-gravitational processes
effect cluster formation and evolution.

In this paper we present the data and results from a \chandra\
archival project in which we studied the cores of \numcluster\ galaxy
clusters via their ICM entropy distributions. We have named this
project the Archive of \chandra\ Cluster Entropy Profile Tables, or
\accept\ for short. In contrast to \cite{d06} (hereafter D06),
\accept\ covers a broader range of luminosities, temperatures, and
morphologies, focusing on more than just cool core clusters. One of
our primary objectives for this project was to provide the research
community with an additional resource to study galaxy formation and
cluster evolution. By studying ICM entropy we sought to further
illuminate how cluster and BCG properties correlate with feedback
activity, what can be learned of how feedback mechanisms operate, and
how feedback energy is deposited in the cluster atmosphere.

Taken individually, ICM temperature and density do not fully reveal
the cluster thermal history because they are most influenced by the
underlying dark matter potential. Gas temperature reflects the depth
of the potential well, while density reflects the capacity of the well
to compress the gas. However, at constant pressure the density of a
gas is determined by its specific entropy. Rewriting the expression
for the adiabat, $K=P\rho^{-5/3}$, in terms of temperature and
electron density, one can define a new quantity, $K=T n_e^{-2/3}$,
where $T$ is temperature and $n_e$ is electron gas density. This new
quantity, $K$, captures the complete thermal history of the gas
because only heating and cooling can change $K$. This is the quantity
commonly referred to as entropy, but in actuality $K$ is only a
pseudo-entropy while the classic thermodynamic entropy is $s = \ln
K^{3/2} + \mathrm{constant}$.

One important property of gas entropy is that a gas cloud is
convectively stable when $dK/dr \geq 0$. Thus, gravitational potential
wells are like giant entropy sorting devices: low entropy gas sinks to
the bottom of the potential well, while high entropy gas buoyantly
rises to a radius of equal entropy. If a cluster were a sealed box of
gravitation-only processes, then the radial entropy distribution the
cluster would exhibit power-law behavior across all radii with the
lowest entropy gas at the core \citep{vkb05}. Thus, any departures of
the radial entropy distribution from a power-law is indicative of
additional entropy altering processes such as heating and cooling, for
example from AGN.

\accept\ gives us a powerful baseline to study any entropy-feedback
connections which may exist. We have found that the departure of
entropy profiles from a self-similar power-law, which has been
discussed previously by, for example \cite{piffaretti05, radioquiet,
d06, morandi07}), is not limited to cooling flow clusters, but is a
feature of most clusters, and given high enough angular resolution,
possibly all clusters. We have also found that indicators of feedback
-- namely radio sources assumed to be associated with AGN and
\halpha\ emission assumed to be the result of star formation in the
brightest cluster galaxy (BCG) -- are strongly anti-correlated with
core entropy (see \citealt{haradent}). We also find that the core
entropy distribution of both the full \accept\ collection and \hifl\
subsample are bimodal.

As part of this work, we are making all our work publicly available
through two access points: 1) the NASA High Energy Space Archive
(HEASARC) under the Chandra section of
W3Browse\footnote{http://heasarc.gsfc.nasa.gov/W3Browse/chandra}, and
2) our own searchable, interactive
database\footnote{http://www.pa.msu.edu/astro/MC2/accept}. At our
site, the research community will find all data tables, plots,
spectra, reduced \chandra\ data products (forthcoming), reduction
scripts, and much more.

The structure of this paper is as follows: In \S\ref{sec:sample} we
outline initial sample-selection criteria and information about the
\chandra\ observations selected under these criteria. Data reduction
is discussed in \S\ref{sec:data}. Spectral extraction and analysis are
discussed in \S\ref{sec:temppr}, while our method for deriving
deprojected electron density profiles is outlined in
\S\ref{sec:dene}. A few possible sources of systematics are discussed
in \S\ref{sec:sys}. Results and discussion are presented in
\S\ref{sec:r&d}. A brief summary is given in \S\ref{sec:summary}. For
this work we have assumed a flat \LCDM\ Universe with cosmogony
$\OM=0.3$, $\OL=0.7$, and $\Hn=70\km\ps\pMpc$. All quoted
uncertainties are 90\% confidence ($\Delta\chisq = 2.71$,
$1.6\sigma$). Given the large number of clusters in our sample,
figures, fits, and tables showing/listing results for individual
clusters have been omitted and are available at the \accept\ website.

%%%%%%%%%%%%%%%%%%%%%%%%%
\section{Data Collection}
\label{sec:sample}
%%%%%%%%%%%%%%%%%%%%%%%%%

Our sample was initially collected from observations publicly
available in the \chandra\ Data Archive (CDA) as of June 2006. We
first assembled a list of targets from multiple flux-limited surveys:
the \rosat\ Brightest Cluster Sample \citep{1998MNRAS.301..881E}, RBCS
Extended Sample \citep{2000MNRAS.318..333E}, \rosat\ Brightest 55
Sample \citep{1990MNRAS.245..559E, 1998MNRAS.298..416P},
\einstein\ Extended Medium Sensitivity Survey
\citep{1990ApJS...72..567G}, North Ecliptic Pole Survey
\citep{2006ApJS..162..304H}, \rosat\ Deep Cluster Survey
\citep{1995ApJ...445L..11R}, \rosat\ Serendipitous Survey
\citep{1998ApJ...502..558V}, Massive Cluster Survey
\citep{2001ApJ...553..668E}, and {\it{REFLEX}} Survey
\citep{reflex}. After the first round of data analysis concluded, we
continued to expand our collection by adding new archival data listed
under the CDA Science Categories ``clusters of galaxies'' or ``active
galaxies''. As of submission, we have analyzed 491 archival
observations with a pre-analysis exposure time of 16.16 Msec. The Coma
and Fornax clusters have been intentionally left-out of our sample
because they are very well studied nearby clusters which require a
more intensive analysis then we undertook in this project.

While observations abounded, for some clusters the available data
limited our ability to derive an entropy profile. Calculation of
entropy requires measurement of the radial gas temperature and density
structures (discussed further in \S\ref{sec:data}). To infer a
temperature which is reasonably well constrained ($\pm 1.0 \keV$) we
imposed a minimum requirement of three temperature bins containing
2500 counts each. After applying this constraint, the final sample
presented in this paper has \numobs\ observations of \numcluster\
clusters with a total exposure time of \expt. The sample covers
cluster temperatures of $kT \sim 1-20$ keV, a bolometric luminosity
range of $L_{bol} \sim 10^{42-46} \ergps$, and redshifts of $z \sim
0.05-0.89$. Table \ref{tab:sample} lists the general properties for
each cluster in \accept.

We were unable to analyze some clusters for this study because of
complications other than not meeting our minimum requirements for
analysis. These clusters were: 2PIGG J0311.8-2655, 3C 129, A168, A514,
A753, A1060, A1367, A2256, A2634, A2670, A2877, A3074, A3128, A3627,
AS0463, APMCC 0421, MACS J2243.3-0935, MS J1621.5+2640, RX
J1109.7+2145, RX J1206.6+2811, RX J1423.8+2404, SDSS
J198.070267-00.984433, Triangulum Australis, and Zw5247.

We also report \halpha\ observations taken by M. Donahue while a
Carnegie Fellow. These observations were not utilized in this paper
but are important for \cite{haradent}. The new $N_{II}/\halpha$ ratios
and \halpha\ fluxes are listed in Table \ref{tab:newha}. The
observations were taken with either the 5 m Hale Telescope at the
Palomar Observatory, USA, or the DuPont 2.5 m telescope at the Las
Campanas Observatory, Chile. All observations were made with a
$2\arcsec$ slit centered on the BCG using two position angles: one
along the semi-major axis and one along the semi-minor axis of the
galaxy. The overlap area was $10$ pixels$^2$. The red light (555-798
nm) setup on the Hale Double Spectrograph used a 316 lines/mm grating
with a dispersion of 0.31 nm/pixel and an effective resolution of
0.7-0.8 nm. The DuPont Modular Spectrograph setup included a 1200
lines/mm grating with a dispersion of 0.12 nm/pixel and an effective
resolution of 0.3 nm. The statistical and calibration uncertainties
for the observations are both $\sim 10\%$. The statistical uncertainty
results primarily from variability of the spectral continuum and hence
imperfect background subtraction. The upper-limits listed in Table
\ref{tab:newha} are $3\sigma$ significance.

%%%%%%%%%%%%%%%%%%%%%%%
\section{Data Analysis}
\label{sec:data}
%%%%%%%%%%%%%%%%%%%%%%%

Measuring radial ICM entropy first requires measurement of radial ICM
temperature and density. The radial temperature structure of each
cluster was measured by fitting a single-temperature thermal model to
spectra extracted from concentric annuli centered on the cluster X-ray
``center''. As discussed in \cite{xrayband}, the ICM X-ray peak of the
point-source cleaned, exposure-corrected cluster image was used as the
cluster center, unless the iteratively determined X-ray centroid was
more than 70 kpc away from the X-ray peak in which case the centroid
is used as the radial analysis zero-point. To derive the gas density
profile, we first deprojected an exposure-corrected,
background-subtracted, point source clean surface brightness profile
extracted in the 0.7-2.0\keV\ energy range to attain a volume emission
density. This emission density, along with spectroscopic information,
was then used to convert observed surface brightness into gas density.
The resulting entropy profiles were then fit with two models: a simple
model which has only a radial power-law component, and a model which
is the sum of a constant central entropy term, \kna\, and the radial
power-law component.

In this paper we cover the basics of deriving gas entropy from X-ray
observables, and direct interested readers to D06 for in-depth
discussion of our data reprocessing and reduction, and \cite{xrayband}
for details regarding determination of each cluster's ``center'' and
how the X-ray background was handled. The only difference between the
analysis presented in this paper and that of D06 and \cite{xrayband},
is that we have used newer versions of \ciao\ and the \caldb\ (\ciao\
3.4.1 and \caldb\ 3.4.0) when reducing data for this project.

%%%%%%%%%%%%%%%%%%%%%%%%%%%%%%%%%
\subsection{Temperature Profiles}
\label{sec:temppr}
%%%%%%%%%%%%%%%%%%%%%%%%%%%%%%%%%

One of the two components needed to derive gas entropy is the
temperature as a function of radius. We therefore constructed radial
temperature profiles for each cluster in our collection. To reliably
constrain a temperature and allow for the detection of temperature
structure beyond simple monotonicity, we required each temperature
profile to have a minimum of three annuli containing 2500 counts
each. The annuli for each cluster were generated by first extracting a
background-subtracted cumulative counts profile using 1 pixel width
annular bins originating from the cluster center and extending to a
radius bounded by the detector edge, or $0.5 R_{180}$, whichever was
smaller. Profiles were truncated at $0.5 R_{180}$ as we were most
interested in the radial entropy behavior of cluster core regions ($r
\la 100$ kpc) and $0.5 R_{180}$ is the approximate radius where
temperature profiles turnover
\citep{2005ApJ...628..655V}. Additionally, analysis of diffuse gas
temperature structure at large radii, which spectroscopically is
dominated by background, requires a time consuming,
observation-specific analysis of the X-ray background (see
\citealt{minggroups} for an excellent detailed discussion on this
point).

Cumulative counts profiles were divided into annuli containing at
least 2500 counts. For well resolved clusters, the number of counts
per annulus was increased to reduce the resulting uncertainty of
$kT_X$ and, for simplicity, to keep the number of annuli less than
50. The method we use to derive entropy profiles is most sensitive to
the surface brightness radial bin size and not the resolution, or
uncertainties, of the temperature profile. Thus, the loss of
resolution in the temperature profile from increasing the number of
counts per bin, and thereby reducing the number of annuli, has an
insignificant effect on the final entropy profiles and best-fit
entropy models.

Background analysis was performed using the blank-sky datasets
provided in the \caldb. Backgrounds were reprocessed and reprojected
to match each observation. Off-axis chips were used to normalize for
variations of the hard-particle background by comparing blank-sky and
observation 9.5-12\keV\ count rates. Soft residuals were also created
and fitted for each observation to account for the spatially-varying
soft Galactic background. This component was added as an additional,
fixed background component during spectral fitting. Errors associated
with the soft background are estimated and added in quadrature to the
final error.

For each radial annular region, source and background spectra were
extracted from the target cluster and corresponding normalized
blank-sky dataset. Using standard
\ciao\ techniques\footnote{http://cxc.harvard.edu/ciao/guides/esa.html}
we created weighted response files (WARF) and redistribution matrices
(WRMF) for each cluster using a flux-weighted map (WMAP) across the
entire extraction region. These files quantify the effective area,
quantum efficiency, and imperfect resolution of the
\chandra\ instrumentation as a function of chip position. Each
spectrum was binned to contain a minimum of 25 counts per energy bin.

Spectra were fitted with \xspec\ 11.3.2ag \citep{xspec} using an
absorbed, single-temperature \mekal\ model \citep{mekal1, mekal2} over
the energy range 0.7-7.0 \keV. Galactic absorption values, \nhi, are
taken from \cite{dickeylockman}. A comparison between the \nhi\ values
of \cite{dickeylockman} and the higher-resolution LAB Survey
\citep{lab} reveals that the two surveys agree to within $\pm 20\%$
for 80\% of the clusters in our sample. For the other 20\% of the
sample, using the LAB value, or allowing \nhi\ to be free, does not
result in best-fit temperatures or metallicities which differ
significantly from fits using the \cite{dickeylockman} values.

The potentially free parameters of the absorbed thermal model are
\nhi, X-ray temperature, metal abundance normalized to Solar
(elemental ratios taken from \citealt{ag89}), and a normalization
proportional to the integrated emission measure within the extraction
region,
\begin{equation}
\label{eqn:norm}
\eta = \frac{10^{-14}}{4\pi D_A^2(1+z)^2}\int \nelec \np dV,
\end{equation}
where $D_A$ is the angular diameter distance, $z$ is cluster redshift,
\nelec\ and \np\ are the electron and proton densities respectively,
and $V$ is the volume of the emission region. In all fits the metal
abundance in each annulus was a free parameter and \nhi\ was fixed to
the Galactic value. No systematic error is added during fitting and
thus all quoted errors are statistical only. The statistic used during
fitting was $\chi^2$ (\xspec\ statistics package \textsc{chi}). All
uncertainties were calculated 90\% confidence.

For some clusters, more than one observation was available in the
archive. We utilized the combined exposure time by first extracting
independent spectra, WARFs, WRMFs, normalized background spectra, and
soft residuals for each observation. These independent spectra were
then read into \xspec\ simultaneously and fit with the same spectral
model which had all parameters, except normalization, tied among the
spectra.

As in D06, we find spectral deprojection does not result in
significant differences between best-fit temperatures inferred for
projected or deprojected quantities. Thus, for this work, we quote
projected temperatures only. Deprojection of temperature should result
in slightly lower temperatures in the central bins of only the
clusters with the steepest temperature gradients. For these clusters,
the end result would be a negligible lowering of the entropy for the
central-most bins. We stress that spectral deprojection does not
significantly change the shape of the entropy profiles nor the
best-fit \kna\ values.

%%%%%%%%%%%%%%%%%%%%%%%%%%%%%%%%%%%%%%%%%%%%%%%%%%
\subsection{Deprojected Electron Density Profiles}
\label{sec:dene}
%%%%%%%%%%%%%%%%%%%%%%%%%%%%%%%%%%%%%%%%%%%%%%%%%%

For predominantly free-free emission, as is the case for the cluster
ICM, gas emissivity strongly depends on gas density and only weakly on
temperature, $\epsilon \propto \rho^2 T^{1/2}$. Therefore the measured
flux in a narrow temperature range is an excellent measure of ICM
density. To reconstruct the relevant gas density as a function of
physical radius we deprojected the cluster emission from
high-resolution surface brightness profiles and converted to electron
density using normalizations and count rates taken from the spectral
analysis.

We extracted surface brightness profiles from the 0.7-2.0 keV energy
range (where the bulk of cluster emission occurs) using concentric
annular bins of size $5\arcsec$ (10 ACIS pixels) originating from the
cluster center. Each surface brightness profile was corrected with an
observation specific, normalized radial exposure profile to remove the
effects of vignetting and exposure time fluctuations. Following the
recommendation in the \ciao\ guide for analyzing extended sources,
exposure maps were created using the monoenergetic value associated
with the observed count rate peak. The more sophisticated method of
creating exposure maps using spectral weights calculated for an
incident spectrum with the temperature and metallicity of the observed
cluster was also tested. For the narrow energy band we consider, the
chip response is relatively flat and we find no significant
differences between the two methods. For all clusters the
monoenergetic value was between $0.8-1.7\keV$.

The 0.7-2.0 keV spectroscopic count rate and spectral normalization
were interpolated from the radial temperature profile grid to match
the surface brightness radial grid. Utilizing the deprojection
technique of \cite{kriss83}, the interpolated spectral parameters were
used to convert observed surface brightness to deprojected electron
density. Radial electron density written in terms of relevant
quantities is,
\begin{equation}
\nelec(r) = \sqrt{\frac{1.2~4 \pi [D_A(1+z)]^2~C(r)~\eta(r)}{10^{-14}~f(r)}}
\end{equation}
where 1.2 is an ionization ratio (\nelec=1.2\np), $C(r)$ is the radial
emission density derived from Eqn. A1 in \cite{kriss83}, $\eta$ is the
interpolated spectral normalization from Eqn. \ref{eqn:norm}, $D_A$ is
the angular diameter distance, $z$ is cluster redshift, and $f(r)$ is
the interpolated spectroscopic count rate. Cosmic reddening of source
surface brightness is accounted for by the $D_A^2 (1+z)^2$ term. This
method of deprojection takes into account temperature and metallicity
fluctuations which affect observed gas emissivity. Errors for the gas
density profile were estimated using 5000 Monte Carlo simulations of
the original surface brightness profile. This deprojection technique
requires an assumption of spatial symmetry, and while we only consider
spherical symmetry, it was shown in \cite{d06} that this assumption
has little effect on our results.

%%%%%%%%%%%%%%%%%%%%%%%%%%%%%%%
\subsection{$\beta$-model Fits}
\label{sec:beta}
%%%%%%%%%%%%%%%%%%%%%%%%%%%%%%%

Noisy surface brightness profiles, or profiles with irregularities
such as inversions or extended flat cores, result in unstable,
unphysical quantities when using the ``onion'' deprojection
technique. For cases where deprojection of the raw data was
problematic, we resorted to fitting the surface brightness profile
with a $\beta$-model \citep{1978A&A....70..677C}. It is well known
that the $\beta$-model is only an approximation for an isothermal gas
distribution and does not precisely represent all the features of the
ICM \citep{2000MNRAS.311..313E, 2002ApJ...579..571L,
2007ApJ...665..911H}. However, for the profiles we found required a
fit, the $\beta$-model is a suitable approximation, and the models use
is only a means for creating a smooth function which is easily
deprojected. The single ($N=1$) and double ($N=2$) $\beta$-models were
used in fitting,
\begin{eqnarray}
S_X &=& \displaystyle\sum_{i=1}^N S_i
\left[1+\left(\frac{r}{r_{c,i}}\right)^2\right]^{-3\beta_i+\onehalf}.
\end{eqnarray}
The models were fitted using Craig Markwardt's robust non-linear least
squares minimization IDL routines\footnote{available at
  http://cow.physics.wisc.edu/\~craigm/idl/}. The data input to the
fitting routines were weighted using the inverse square of the
observational errors. Using this weighting scheme resulted in
residuals which were near unity for, on average, the inner 80\% of the
radial range considered. Accuracy of errors output from the fitting
routine were checked against a bootstrap Monte Carlo analysis of 1000
surface brightness realizations. Both the single- and double-$\beta$
models were fit to each profile and we established which model best
represented the surface brightness profile by making a model
comparison via an F-test.

A best-fit $\beta$-model was used in place of the data when deriving
electron density for the clusters listed in Table
\ref{tab:betafits}. These clusters are also flagged in Table
\ref{tab:sample} with the note letter 'a'. The best-fit $\beta$-models
and background-subtracted, exposure-corrected surface brightness
profiles are shown in Figure \ref{fig:betamods} and the parameters are
given in Table \ref{tab:betamods}. See Appendix \ref{app:beta} for
notes discussing individual clusters. The disagreement between the
best-fit $\beta$-model and the surface brightness in the central
regions for some clusters is also discussed in Appendix
\ref{app:beta}. In short, the discrepancy arises from the presence of
compact X-ray sources, a topic which is addressed in
\S\ref{sec:centsrc}. All clusters requiring a $\beta$-model fit have
$\kna > 95 \ent$ with a mean $\kna = 232 \pm 77 \ent$.

%%%%%%%%%%%%%%%%%%%%%%%%%%%%%
\subsection{Entropy Profiles}
\label{sec:kpr}
%%%%%%%%%%%%%%%%%%%%%%%%%%%%%

Radial entropy profiles were calculated using the widely adopted
formulation $K(r) = T(r)\nelec(r)^{-2/3}$ \cite{davies00}. To create
the radial entropy profiles, the temperature and density profiles must
be on the same radial grid. This was accomplished by interpolating the
temperature profile across the higher-resolution radial grid of the
deprojected electron density profile. In general, the higher radial
resolution of the density profiles means that the central bin of the
temperature profile spans several bins of the density profile. Since
we are most interested in the behavior of the entropy profiles in the
central regions, how the interpolation was performed in this part of
the profiles was important. Thus, temperature interpolation over the
region of the density profile where a single temperature bin
encompasses several central density bins was applied in two ways: 1)
as a linear gradient consistent with the slope of the temperature
profile at radii larger than the central $T_X$ bin ($\Delta T_{center}
\ne 0$), and 2) as a constant ($\Delta T_{center}=0$). Shown in Figure
\ref{fig:kcomp} is the ratio of best-fit core entropy, \kna, derived
using the above two methods. The five points lying below the line of
equality are clusters which are best-fit by a power-law or have
\kna\ statistically consistent with zero. It is worth noting that both
schemes yield statistically consistent values for \kna\ except for the
clusters marked by red points which have a ratio significantly
different from unity.

These special cases all have steep temperature gradients with the
maximum and minimum radial temperatures differing by a factor of
1.3-5.0. Extrapolation of a steep temperature gradient as $r
\rightarrow 0$ results in very low central temperatures (typically
$T_X \leq T_{virial}/3$) which are inconsistent with observations,
most notably \cite{peterson03}. Most important however, is that the
flattening of entropy we observe in the cores of our sample (discussed
in \S\ref{sec:nonzerok0}) is {\bfseries\em{not}} a result of the
method chosen for interpolating the temperature profile. For this
paper we therefore focus on the results derived assuming a constant
temperature across the central-most bins.

Uncertainty in $K(r)$ arising from assuming gas inhomogeneity
contributes negligibly to our final fits, \eg\ assuming gas within an
annular region is best described by a single temperature model, is
discussed in detail in the Appendix of D06. Briefly summarizing D06:
we have primarily measured the entropy of the lowest entropy gas
because it is the most luminous gas. For the best-fit entropy values
to be significantly changed, the volume filling fraction of a
higher-entropy component must be non-trivial ($> 50\%$). As discussed
in D06, our results are robust to the presence of multiple, low
luminosity gas phases and mostly insensitive to X-ray surface
brightness decrements, such as X-ray cavities and bubbles, although in
extreme cases their influence on an entropy profile can be detected
(for an example see the cluster A2052).

Each entropy profile was fit with two models: (1) a simple model which
is a power-law at large radii and approaches a constant value at small
radii, and (2) a model which is a power-law only:
\begin{eqnarray}
(1)~K(r) &=& \kna + \khun\ \left(\frac{r}{100 \kpc}\right)^{\alpha}\nonumber \\
(2)~K(r) &=& \khun\ \left(\frac{r}{100 \kpc}\right)^{\alpha}.\nonumber
\end{eqnarray}
In these models, \kna\ is what we call core entropy, \khun\ is a
normalization for entropy at 100 kpc, and $\alpha$ is the power-law
index. Note, however, that \kna\ does not necessarily represent the
minimum core entropy or the entropy at $r=0$. Nor does \kna\ capture
the gas entropy which would be measured immediately around an AGN or
in a BCG X-ray coronae. Instead, \kna\ represents the typical excess
of core entropy above the best fitting power-law found at larger
radii. Fits were truncated at a maximum radius (determined by-eye) to
avoid the influence of noisy bins at large radii which result from
instability of our deprojection method. A listing of all the best-fit
parameters for each cluster are published at the \accept\ website.

Some clusters have a surface brightness profile which is comparable to
a double $\beta$-model. Our models for the behavior of $K(r)$ are
intentionally simplistic and are not intended to fully describe all
the features of $K(r)$. Thus, for the small number of clusters with
discernible double-$\beta$ behavior, fitting of the entropy profiles
was restricted to the innermost of the two $\beta$-like
features. These clusters have been flagged in Table \ref{tab:sample}
with the note letter 'b'. The best-fit power-law index is typically
much steeper for these clusters, but the outer regions, which we do
not discuss here, have power-law indices which are typical of the rest
of the sample, \ie\ $\alpha \sim 1.2$.

%%%%%%%%%%%%%%%%%%%%%%%%%%%%%%%%%%%%%%%%%
\subsection{Exclusion of Central Sources}
\label{sec:centsrc}
%%%%%%%%%%%%%%%%%%%%%%%%%%%%%%%%%%%%%%%%%

For many clusters in our sample the ICM X-ray peak, ICM X-ray
centroid, optical BCG emission, and infrared BCG emission are
coincident or well within 70 kpc of one another. This made
identification of the cluster center robust and trivial. However, in
some clusters, there is an X-ray point source or compact X-ray source
($r \la 5$ kpc) found very near ($r < 10 kpc$) the cluster center and
always associated with a BCG. For some clusters -- such as 3C 295,
A2052, A426, Cygnus A, Hydra A, or M87 -- the source is an AGN and
there was no question the source must be removed.

However, determining how to handle the compact X-ray sources was not
so straight-forward. These compact sources are larger than the PSF,
fainter than an AGN, but are typically more luminous than the
surrounding ICM such that the compact source's extent was
distinguishable from the ICM. These sources are most prominent, and
thus the most troublesome, in non-cool core clusters (\ie\ clusters
which are mostly isothermal). They are troublesome because the compact
source is typically much cooler and denser than the surrounding ICM
and hence has an entropy much lower than the ambient ICM. We suspect
most of these compact sources are X-ray coronae \citep{coronae},
although they may be something else as we have not studied them in
detail yet.

After deriving radial entropy profiles for clusters with a compact
source at the center, without removing the source, we found $K(r)$
abruptly declines when the outer edge of the compact source is reached
(typically only one or two central surface brightness bins). This
discontinuity in $K(r)$ means the simple models we used for fitting
$K(r)$ no longer were a good description of the profile. Obviously,
two solutions were available: exclude or keep the compact sources
during analysis. Including the compact sources results in the central
cluster region(s) appearing overdense, and at a given temperature the
region will be lower entropy than if the source were excluded. Aside
from producing poor fits, this significantly lower entropy influences
the value of best-fit parameters because the shape of $K(r)$ is
drastically changed. Deciding what to do with these sources depends
upon what cluster properties we specifically are interested in
quantifying.

Simply put, the compact X-ray sources discussed in this section are
not representative of the cluster's core entropy; these sources are
representative of the entropy within and immediately surrounding the
BCG -- which is not the quantity we are interested in measuring at
this time. Our focus in this work was to quantify the entropy
structure of the cluster core region, not to determine the minimum
entropy of cluster cores or to quantify the entropy of peculiar core
objects such as BCG coronae. Thus, we decided to exclude these compact
sources during our analysis.

{\bf{[Do I want to add an appendix for... Notes regarding individual
      clusters/sources are provided in Appendix \ref{app:centsrc}.]}}

For most compact sources, the exclusion region used had an area which
was $<10\%$ of the central surface brightness bin area. For a few
extraordinary sources the central one to two bins of the surface
brightness profile were ignored during analysis {\bf{[Which clusters?
      Flag them.]}} The \centsrcnum\ clusters which had a central
source removed have been flagged in Table \ref{tab:sample} with the
note letter 'd' for AGN and 'e' for compact sources. The mean best-fit
parameters for these clusters are \knacs, \alphacs, and \khuncs. These
clusters cover the redshift range $z = 0.0044-0.4641$ with mean $z =
0.1196 \pm 0.1234$, and temperature range $kT_X = 1-12$ keV with mean
$kT_X = 4.43 \pm 2.53$ keV. Dividing these clusters into two
subgroups, those above and below $\kna = 50 \ent$, the means for
clusters with $\kna < 50 \ent$ are \knacsa, \alphacsa, and
\khuncsa. For clusters with $\kna > 50 \ent$ the means are \knacsb,
\alphacsb, and \khuncsb. The range and mean of the redshifts and
temperatures for the two subgroups are not significantly different.

It is worth noting that when any source is excluded from the data, the
empty pixels where the source once was are not included in the
calculation of the surface brightness (counts and pixels are both
excluded). Thus, the decrease in surface brightness of a bin where a
source has been removed is not a result of the count to area ratio
being artificially reduced.

%%%%%%%%%%%%%%%%%%%%%
\section{Systematics}
\label{sec:sys}
%%%%%%%%%%%%%%%%%%%%%

Our models for $K(r)$ were created so the best-fit \kna\ values take
on the definition of being a good measure of the entropy profile
flattening at small radii. There is the possibility that this
flattening may be exaggerated or reduced through the effects of
systematics such as PSF smearing and surface brightness profile
angular resolution. To quantify the extent to which our \kna\ values
are being affected by these systematics, we have analyzed mock
\chandra\ observations created using the ray-tracing program
MARX\footnote{http://space.mit.edu/CXC/MARX/}, and also by analyzing
degraded entropy profiles generated from artificially redshifting
well-resolved clusters.

%%%%%%%%%%%%%%%%%%%%%%%%
\subsection{PSF Effects}
\label{sec:psf}
%%%%%%%%%%%%%%%%%%%%%%%%

To assess the effect of PSF smearing on our entropy profiles we have
updated the analysis presented in \S4.1 of D06 to include MARX
simulations. In the D06 analysis we assumed the density and
temperature structure of the cluster core obeyed power-laws with $n_e
\propto r^{-1}$ and $T_X \propto r^{1/3}$. This results in a power-law
entropy profile with $K \propto r$. Further assuming the main emission
mechanism is thermal Bremsstrahlung, \ie\ $\epsilon_X \propto
T_X^{1/2}$, yields a surface brightness profile which has the form $S_X
\propto r^{-5/6}$. A source image consistent with these parameters was
created in \idl\ and then input to MARX to create the mock \chandra\
observations.

The MARX simulations were performed using the spectrum of a 4.0 keV,
$0.3 \Zsol$ abundance \mekal\ model. We have tested using input
spectra with $kT = 2-10$ keV with varying abundances and find the
effect of temperature and metallicity on the distribution of photons
in MARX to be insignificant for our discussion here. We have neglected
the X-ray background in this analysis as it is overwhelmed by cluster
emission in the core and is only important at large
radii. Observations for both ACIS-S and ACIS-I instruments were
simulated using an exposure time of 40 ksec. A surface brightness
profile was then extracted from the mock observations using the same
$5\arcsec$ bins used on the real data.

For $5\arcsec$ bins, we find the difference between the central bins
of the input surface brightness and the output MARX observation to be
less than the statistical uncertainty. The on-axis \chandra\ PSF is
$\la 1\arcsec$ and the surface brightness bins we have used on the
data are five times this size. Our analysis using MARX bears out that
PSF effects are negligible in our study because the data bins are so
much larger than the PSF. Most interesting and important though, is
that our analysis using MARX suggests any deviation of the surface
brightness -- and consequently the entropy profile -- from a
power-law, even if only in the central bin, is real and cannot be
attributed to PSF effects. Even for the most poorly resolved clusters,
the deviation away from a power-law we observe in so many of our
entropy profiles is not an artifact of the data analysis or
deprojection technique.

%%%%%%%%%%%%%%%%%%%%%%%%%%%%%%%%%%%%%%%
\subsection{Angular Resolution Effects}
\label{sec:angres}
%%%%%%%%%%%%%%%%%%%%%%%%%%%%%%%%%%%%%%%

Another possible limitation on evaluating \kna\ is the effect of using
fixed angular size bins for extracting surface brightness
profiles. This choice may introduce a redshift-dependence into the
best-fit \kna\ values because as redshift increases, a fixed angular
size encompasses a larger physical volume and the value of \kna\ may
increase if the bin includes a broad range of gas entropy. Shown in
Figure \ref{fig:k0res} is a plot of the best-fit \kna\ values for our
entire sample versus redshift. Above $\kna \approx 10 \ent$, our
sample is fairly complete at all redshifts {\bf{[How to quantify
      this?]}}. However, at the lowest redshifts ($z < 0.02$), there
are a few objects with $\kna < 10 \ent$ and only one at higher
redshifts (A1991 -- $\kna = 1.53 \pm 0.32$, $z = 0.0587$ -- which is a
very peculiar cluster \citep{2004ApJ...613..180S}). This raises the
question: can the lack of clusters with $\kna \la 10 \ent$ at $z >
0.02$ be completely explained by resolution effects?

To answer this question we tested the effect redshift has on measuring
\kna\ by selecting all clusters with $\kna \leq 10 \ent$ and $z \leq
0.1$ and degrading their surface brightness profiles to mimic the
effect of increasing the cluster redshift. Our test is best
illustrated using an example: first, consider a cluster at $z =
0.1$. For this cluster, $5\arcsec$ equals 9 kpc. Were the cluster at
$z = 0.2$, $5\arcsec$ would equal 16 kpc. To mimic moving this example
cluster from $z = 0.1 \rightarrow 0.2$, we can extract a new surface
brightness profile using a bin size of 16 kpc instead of
$5\arcsec$. This will result in a new surface brightness profile which
has the angular resolution for a cluster at a higher redshift. This is
the procedure we used for degrading the profiles of our
subsample. This process of calculating new bin sizes to artificially
redshift clusters was repeated over an evenly distributed grid of
redshifts in the range $z = 0.1-0.4$ using step sizes of 0.02. The
temperature profiles for each cluster were also degraded by starting
at the innermost temperature profile annulus and pairing-up
neighboring annuli moving outward. New spectra were extracted for
these enlarged regions and analyzed following the same procedure
detailed in \S\ref{sec:temppr}.

The ensemble of artificially redshifted clusters was then analyzed
using the procedure detailed in \S\ref{sec:kpr}. There are a few
notable effects on the entropy profiles arising from lower angular
resolution. Obviously, as redshift increases, the number of radial
bins decreases. The resulting effect is that detail of the entropy
profile's curvature is diminished, \eg\ the profiles become
straighter. On its own this effect should lead to lower best-fit
\kna\ values, but, while profile curvature is reduced, the entropy of
the central-most bins is increasing because they encompass a broader
range of entropy. From $z = 0.1-0.2$ this last effect dominates,
resulting in an increase of $(\kna^{\prime}-\kna)/\kna \approx
[XX]-[YY]$ {\bf{[Is this really how I want to quantify
      this?]}}. However, at $z > 0.2$, too much radial resolution is
lost and the degraded profiles begin to resemble power-laws except in
the innermost bin which still lies above the power-law (this however
is accompanied by a large increase of the uncertainty in \kna). The
result of a power-law profile with a discrepant central bin is that
the degraded \kna\ values are only slightly larger than the original
\kna, $(\kna^{\prime}-\kna)/\kna \approx [XX]-[YY]$ {\bf{[Missing
      data]}}, but have much larger uncertainties.

Our analysis of the degraded entropy profiles suggests that \kna\ is
more sensitive to the value of $K(r)$ in the central bins than it is
to the shape of the profile or the number of radial bins (results we
discuss further in \S\ref{sec:curve}). Most importantly however is
that low-redshift clusters with \kna\ values in the range $2-10 \ent$
look like $\kna \approx 10-25 \ent$ clusters at $z > 0.1$. Thus we
conclude that the lack of $\kna < 10 \ent$ clusters at $z \ga 0.1$ can
be attributed to resolution effects.

%%%%%%%%%%%%%%%%%%%%%%%%%%%%%%%%%%%%%%%%%%%%%%%%%
\subsection{Profile Curvature and Number of Bins}
\label{sec:curve}
%%%%%%%%%%%%%%%%%%%%%%%%%%%%%%%%%%%%%%%%%%%%%%%%%

In the previous section, our analysis of degraded entropy profiles
illuminated two concerning points: 1) that \kna\ is sensitive to the
curvature of the entropy profile, and 2) that the number of radial
bins also effects the best-fit \kna. This raises the possibility of
two troubling systematics in our analysis. To evaluate the dependence
of \kna\ on profile curvature we first calculated average profile
curvatures, $\kappa_A$. For each profile $\kappa_A$ was calculated
using the standard formulation $\kappa =
\|y^{''}\|/(1+y^{'2})^{3/2}$, where for our application $y = K(r) =
\kna+\khun(r/100\kpc)^{\alpha}$, yielding:
\begin{equation}
\kappa_A = \frac{1}{\int dr}\int\frac{\| 100^{-\alpha} (\alpha-1) \alpha \khun
  r^{\alpha-2}\|}{[1+(100^{-\alpha} \alpha \khun
  r^{\alpha-1})^2]^{3/2}}
\end{equation}
where $\alpha$ and \khun\ are the best-fit parameters unique to each
entropy profile. We find that at any value of \kna\ a large range of
curvatures are covered and that there is no underlying systematic
trend in \kna\ associated with curvature.

%Shown in Figure \ref{fig:curve} is a plot of $\kappa_A$ versus
%\kna. From Fig. \ref{fig:curve} it is apparent that at any given value
%It appears from this figure that for $\kna
%\ga 300 \ent$ there is a tendency for increasing \kna\ to be
%associated with profiles having less curvature. But this effect is
%expected because for very large values of \kna\ the entropy profiles
%are flat across nearly the entire core region and then deviate sharply
%when transitioning into the power-law behavior at $r \ga 100$ kpc.
%In \S\ref{sec:bimod} we discuss the bimodality of the \kna\
%distribution, a feature which is evident in
%Fig. \ref{fig:curve}. Dividing the sample into clusters above and
%below $\kna = 40 \ent$ we find for clusters with $\kna < 40 \ent$ the
%mean $\kappa_A = 9.17\times10^{-9} \pm 7.27\time10^{-9}$, while for
%$\kna \ge 50 \ent$ clusters the mean $\kappa_A = 1.23\times10^{-8} \pm
%1.02\time10^{-8}$. The grouping and smaller dispersion of clusters
%between $\kna < 40 \ent$ is interesting: it suggests that clusters
%with low core entropy (which also happen to have short central cooling
%times), along with having many other comparable physical properties,
%have very similar entropy profile shapes, a point which has been
%discussed previously by \cite{2008MNRAS.386.1309M}.

In \S\ref{sec:angres} we found that profiles with fewer radial bins
tend toward lower best-fit \kna\ values. After examining plots of the
number of bins fit in each entropy profile versus \kna\ we find only
scatter and no trends. The radial resolution effect on \kna\ we found
in the degraded profiles is not present in the real data.

We do not find any systematic trends with profile shape or number of
fit bins which would significantly effect our best-fit \kna\
values. Thus we conclude that the \kna\ values presented in the
following sections are an excellent measure of the core entropy and
any undetected dependence on profile shape or radial resolution effect
our results at significance levels much smaller than the measured
uncertainties.

%Shown in Figure
%\ref{fig:nbins} is a plot of \kna\ versus the number of bins fit in
%each entropy profile. The effect we observed in the previous section
%was for profiles having less than 10 fit bins. From
%Fig. \ref{fig:nbins} it is evident that that there is no trend at work
%for $N_{bins} > 10$. However, for profiles with $N_{bins} < 10$ there
%is a trend of decreasing \kna\ the fewer bins there are in the
%profile. But, for these eight clusters, the uncertainties on \kna\ are
%very small ($\approx[XX\%]$) {\bf{[missing]}}, which is counter to the
%result we found with the degraded profiles where uncertainty was large
%($\approx [XX\%]$) {\bf{[missing]}} the fewer bins were fit.

%%%%%%%%%%%%%%%%%%%%%%%%%%%%%%%%
\section{Results and Discussion}
\label{sec:r&d}
%%%%%%%%%%%%%%%%%%%%%%%%%%%%%%%%

Presented in Figure \ref{fig:splots} is a montage of \accept\ entropy
profiles for different temperature ranges. These figures highlight the
cornerstone result of \accept: a uniformly analyzed collection of
entropy profiles covering a broad range of core entropy. Each profile
is color-coded in representation of the global cluster
temperature. Also plotted in the figure is the ``baseline'' entropy
profile calculated by \cite{vkb05}. This theoretical curve represents
the entropy profile of self-similar clusters without feedback and for
which the radial entropy profile scales linearly with cluster
temperature. This profile ignores the effects of non-gravitational
processes and thus gives a lower-limit with which we can compare our
profiles.

In the following sections we discuss the results gleaned from analysis
of our library of entropy profiles. Results such as the departure of
most entropy profiles from a simple radial power-law profile, the
bimodal distribution of core entropy, and the asymptotic convergence
of the entropy profiles to the self-similar $K(r) \propto r^{1.1}$
power-law at $r \geq 100\kpc$.

%%%%%%%%%%%%%%%%%%%%%%%%%%%%%%%%%%
\subsection{Non-Zero Core Entropy}
\label{sec:nonzerok0}
%%%%%%%%%%%%%%%%%%%%%%%%%%%%%%%%%%

Arguably the most striking feature of Figure \ref{fig:splots} is the
departure of most profiles from a simple power-law. Flattening of
surface brightness profiles (and conversely ICM gas density) is a well
known feature of clusters (\eg\ \citealt{1999ApJ...517..627M},
\citealt{2000MNRAS.318..715X}, and references therein). What is
notable in our work however is that, based on comparison of reduced
$\chi^2$, very few of the clusters in our sample have an entropy
distribution which is best-fit by the power-law only model.

Of the \numcluster\ in \accept, only four clusters have a \kna\ value
which is statistically consistent with zero, or are better fit by the
power-law only model: A2151, AS0405, MS 0116.3-0115, and NGC 507 (used
during analysis of \hifl\ sample only). Two clusters, A1991 and A4059,
are better fit by the power-law model when the temperature profile in
the core is not constant (see \S\ref{sec:temppr}). For these six
clusters it may be the case that the ICM entropy departs from a
power-law at a radial scale smaller than the $5\arcsec$ bins we used
for extracting surface brightness profiles. After extracting new
surface brightness profiles for these six clusters using $2.5\arcsec$
bins and repeating the analysis, we find that the profiles for A4059
and AS0405 do flatten. This leaves A1919, A2151, MS 0116.3-0115, and
NGC 507 as the only clusters in \accept\ which are statistically
best-fit by power-laws.

For the entire sample, the mean \knafs. Subdividing the collection
into two categories -- clusters with \kna\ below and above $50 \ent$
-- we find means of \knaga\ and \knagb, respectively. We show in the
following section, \S\ref{sec:bimod}, that the cut at $\kna=50
\ent$ is not completely arbitrary.

For clusters with central cooling times shorter than the age of the
cluster, non-zero core entropy is an expected consequence of episodic
heating of the ICM \citep{2001ApJ...557..546D, kaiser03}, with AGN as
one possible heating source (see \citealt{agnframework} as an example
feedback model and \citealt{mcnamrev} for a review). Clusters with
cooling times of order the age of the Universe, however, require other
mechanisms to generate their core entropy, for example via mergers or
extremely energetic AGN outbursts. For the very highest \kna\ values,
$\kna > 100 \ent$, the mechanism by which the core entropy of these
clusters came to be so large is not well understood as it is difficult
to boost the entropy of a gas parcel to $> 100 \ent$ via merger shocks
\citep{2008MNRAS.386.1309M} or requires unsubstantiated AGN outburst
energies. We have provided the data within \accept\ to the public with
the hope that it provides the research community a new resource to
further understand the processes which result in non-zero cluster core
entropy.

%%%%%%%%%%%%%%%%%%%%%%%%%%%%%%%%%%%%%%%%%%%%%%%%%%%%
\subsection{Bimodality of Core Entropy Distribution}
\label{sec:bimod}
%%%%%%%%%%%%%%%%%%%%%%%%%%%%%%%%%%%%%%%%%%%%%%%%%%%%

The time required for a gas parcel to radiate away its thermal energy
is a function of the gas entropy. Low entropy gas radiates profusely
and is thus subject to rapid cooling and vice versa for high entropy
gas. Thus, the distribution of \kna\ is of particular interest because
it is an approximate indicator of the cooling timescale in the cluster
core. The \kna\ distribution is also interesting because it may be
useful in better understanding the physical processes operating in
cluster cores. For example, if processes such as thermal conduction
and AGN feedback are important in establishing the entropy state of
cluster cores, then models which incorporate these processes should
approximately reproduce the observed \kna\ distribution.

In the top panel of Figure \ref{fig:k0hist} is plotted the
logarithmically binned distribution of \kna. In the bottom panel of
Figure \ref{fig:k0hist} is plotted the cumulative distribution of
\kna. One can immediately see from these distributions that there are
at least two distinct populations separated by a small number of
clusters with $\kna \approx 40-60 \ent$. If the distinct bimodality of
the \kna\ distribution seen in the binned histogram were an artifact
of binning, then the cumulative distribution should be relatively
smooth. But there are clearly plateaus in the cumulative distribution,
with one of these plateaus coincident with the division between the
two populations at $\kna \approx 40-60 \ent$.

To further test for the presence of a bimodal population we utilized
the KMM test of \cite{1994AJ....108.2348A}. The KMM test estimates the
probability that a set of data points is better described by the sum
of multiple Gaussians than by a single Gaussian. We tested the
unimodal case versus the bimodal case with the assumption that the
dispersion of the two Gaussian components are not the same (\eg\ the
heteroscedastic case). We have used the updated KMM code of Waters et
al. 2008 (ApJ, in press) which incorporates bootstrap resampling to
determine uncertainties for all parameters. A post-analysis comparison
of the hetero- and homoscedastic cases confirms that our initial guess
that the heteroscedastic case is a better model.

The KMM test, as with any statistical test, is very specific. At
zeroth order, the KMM test simply determines if a population is
bimodal or not. The KMM test also reliably finds the means of these
populations. However, the dispersions of these populations are subject
to the quality of sampling and the presence of outliers (\eg\ KMM must
assign all data points to a population). The output of the KMM test
are the best-fit populations to the data, not necessarily the best-fit
populations of the underlying distribution (hence no goodness of fit
is output).

%The best-fit populations from the KMM test are overlaid on the
%log-space distribution of \kna\ in Figure \ref{fig:kmm}. The small
%number of clusters with $\kna \le 4 \ent$ significantly alter the
%best-fit populations, thus we provide two fits to the
%\kna\ distribution: a fit with $\kna \le 4 \ent$ clusters (dashed red
%curve), and a fit without these clusters (solid blue curve).

There are a small number of clusters with $\kna \le 4 \ent$ that when
included in the KMM test significantly change the results. Thus we
provide two sets of results: results with and without the $\kna \le 4
\ent$ clusters. The results of the KMM test neglecting $\kna \le 4
\ent$ clusters were two statistically distinct peaks at \kmma\ and
\kmmb. \kmmc\ clusters were assigned to the first distribution, while
\kmmd\ were assigned to the second. Including $\kna \le 4 \ent$
clusters, the KMM test found populations at \kmmf\ (\kmmh\ clusters)
and \kmmg\ (\kmmi\ clusters). The KMM test also outputs a P-value,
$p$, and with the assumption that \chisq\ describes the distribution
of the likelihood ratio statistic, $1-p$ is the confidence interval
for the model considered. The KMM test neglecting $\kna \le 4 \ent$
clusters returned \kmme, while the test including all clusters
returned \kmmj. These tiny $p$-values indicate the probability the
\kna\ distribution we observe in our data arises from a single
population is astoundingly remote.

One explanation for the bimodal entropy distribution is that it arises
from the effects of episodic AGN feedback and electron thermal
conduction in the cluster core. \cite{agnframework} put forth a model
of AGN induced heating of the ICM whereby outbursts of $\sim 10^{45}
\ergps$ occurring every $\sim 10^8 \yrs$ can maintain a quasi-steady
core entropy of $\approx 10-30 \ent$. In addition, very energetic and
infrequent AGN outbursts of $10^{61} \erg$ can increase the core
entropy into the $\approx 30-50 \ent$ range. This model predicts quite
well the distribution at $\kna \lesssim 50 \ent$, but depletion of the
$\kna = 40-60 \ent$ region and populating $\kna > 60 \ent$ requires
more physics. \cite{conduction} have recently suggested that the
dramatic fall-off of clusters beginning at $\kna \approx 30 \ent$ may
be the result of heat input via electron thermal conduction overtaking
energy losses via radiative cooling thus preventing, or at least
severely slowing, a cluster's core from appreciably cooling and
returning to a core entropy state below $\kna > 30 \ent$.

One consequence of these proposed models is that thermal instabilities
resulting from inefficient conduction should preferentially, but not
exclusively, form only below an entropy threshold where conduction and
cooling are balanced. These thermal instabilities may then result in
various by-products, such as star formation and AGN. If so, then
indicators of these feedback sources being active (such as \halpha\
emission or radio emission from the BCG) would be sensitive to and
entropy threshold. In \cite{haradent} we show that indeed \halpha\ and
powerful radio emission ($\radpow > 10^{40} \ergps$) are
predominantly detected for BCGs residing in clusters with $\kna \la 30
\ent$, while above $\kna = 30 \ent$ \halpha\ is not detected and when
radio emission is detected, it is extremely faint ($\nu L_{\nu} <
10^{40} \ergps$). Our speculation regarding the origins of bimodality
are further bolstered by the recent theoretical work of
\cite{2008arXiv0804.3823G} which shows that a combination of AGN feedback
and conduction are capable of preventing the formation of thermal
instabilities in cluster cores. The recent observational work of
\cite{2008arXiv0802.1864R} which finds blue gradients only for BCGs in
clusters with $\kna \la 30 \ent$ is an additional piece of evidence
that conduction may be important.

We acknowledge that \accept\ is not a complete, uniformly selected
sample of clusters. This raises the possibility that our sample is
biased, possibly towards clusters that have historically drawn the
attention of observers, such as cooling flows or mergers. If that were
the case, then one reasonable explanation of the \kna\ bimodality is
that $\kna = 30-60 \ent$ clusters are ``boring'' and thus go
unobserved. However, as we show in \S\ref{sec:hifl}, the unbiased
flux-limited \hifl\ sample is also bimodal. A sociological explanation
of bimodality for both \accept\ and \hifl\ is highly unlikely.

%%%%%%%%%%%%%%%%%%%%%%%%%%%%%%%%%%
\subsection{The \hifl\ Sub-Sample}
\label{sec:hifl}
%%%%%%%%%%%%%%%%%%%%%%%%%%%%%%%%%%

The clusters comprising \accept\ were not selected using any
well-defined criteria. To ensure our results are not affected by an
unknown selection bias, we culled the Highest X-Ray Flux Galaxy
Cluster Sample (\hifl, \citealt{hiflugcs1, hiflugcs2}) from \accept\
for analysis. \hifl\ is a flux-limited sample ($f_X \ge 2 \times
10^{-11} \flux$) of objects selected by flux only from the
{\it{REFLEX}} sample \citep{reflex} with no consideration of
morphology. Thus, at any given luminosity in \hifl\ there is a good
sampling of different morphologies, \ie\ the bias toward cool-cores or
mergers has been removed. The sample also covers most of the sky with
holes near Virgo and the Large and Small Magellanic Clouds. The sample
also has no known incompleteness \citep{2007A&A...466..805C}. There
are a total of 106 objects in \hifl: 63 in the primary sample and 43
in the extended sample. Of these 106 objects, no public \chandra\
observations were available for 16 objects (A548e, A548w, A1775,
A1800, A3528n, A3530, A3532, A3560, A3695, A3827, A3888, AS0636, HCG
94, IC 1365, NGC 499, RXCJ 2344.2-0422), ten objects did not meet our
minimum analysis requirements and were thus insufficient for study (3C
129, A400, A1060, A1367, A2256, A2634, A2877, A3627, MKW 08,
Triangulum Australis), and as discussed in \S\ref{sec:sample}, Coma
and Fornax were intentionally ignored. This left a total of 78 \hifl\
objects which we analyzed, 57 from the primary sample and 23 from the
extended sample. The primary sample is the most complete of the two,
thus we focus our following discussion on the primary sample only.

As shown in Figure \ref{fig:hiflk0}, the bimodality seen in the full
\accept\ collection is also present in the \hifl\ sub-sample. Dividing
the clusters into those below and above $\kna = 50 \ent$, the mean
parameters are \hifla, \hiflb\, \hiflc, and \hifld, \hifle, \hiflf,
respectively.
% Plotted in Figure \ref{fig:hiflkmm} is the log-space distribution of
%\kna\ overplotted with KMM determined populations.
We again performed two KMM tests, one with %(dashed red curve) and
another without %(solid blue curve)
clusters having $\kna \le 4 \ent$. For the test including $\kna \le 4
\ent$ clusters we find populations at \hiflkmma\ (\hiflkmmc\ clusters)
and \hiflkmmb\ (\hiflkmmd\ clusters with \hiflkmme. Excluding clusters
with $\kna \le 4 \ent$ we find peaks at \hiflkmmf\ and \hiflkmmg\ each
having \hiflkmmh\ and \hiflkmmi\ clusters, respectively. The
probability these populations originate from a unimodal distribution
is \hiflkmmj. It is interesting that the gap in the
\kna\ distributions for \accept\ and \hifl\ occur in the same location,
$\kna \approx 40-60 \ent$. That bimodality is present in both \accept\
and the unbiased \hifl\ subsample suggests bimodality cannot be the
result of simple archival bias.

% and is shown in the right panel of Figure \ref{fig:hifltx}.
In a conference proceeding, \cite{2007hvcg.conf...42H} note a similar
core entropy bimodality for the \hifl\ sample as the one we find here.
\cite{2007hvcg.conf...42H} discuss two distinct groupings of objects in
a plot of average cluster temperature versus core entropy, with the
dividing point being $K \approx 40 \ent$. Our results agree with the
findings of \cite{2007hvcg.conf...42H} with the gap in \kna\ occurring
around $\kna = 40 \ent$.

%Shown in the left panel of
%Figure \ref{fig:hifltx} is the same plot using results from our
%analysis. We have color-coded for the points based on the \kna\ value:
%blue points are $\kna \le 30 \ent$, green points have $30 \ent < \kna
%\le 60 \ent$, and red points have $\kna > 60
%\ent$. The dashed line marks $\kna = 40 \ent$.

%%%%%%%%%%%%%%%%%%%%%%%%%%%%%%%%%%%%%%%%%%%%%%%
\subsection{Distribution of Core Cooling Times}
\label{sec:hifl}
%%%%%%%%%%%%%%%%%%%%%%%%%%%%%%%%%%%%%%%%%%%%%%%

In the X-ray regime, cooling time and entropy are related in that
decreasing gas entropy also means shorter cooling time. Thus, if the
\kna\ distribution is bimodal, the distribution of cooling times might
also be bimodal. We have calculated cooling time profiles from the
spectral analysis using the relation
\begin{equation}
\tcool = \frac{3nkT}{2\nelec \nH \Lambda(T,Z)}
\label{eqn:tcool}
\end{equation}
where \tcool\ is in seconds, $n$ is the total ion number ($\approx
2.3\nH$ for a fully ionized plasma), \nelec\ and \nH\ are the electron
and proton densities respectively, $\Lambda(T,Z)$ is the cooling
function for a given temperature and metal abundance, and $3/2$ is a
constant associated with isochoric cooling. The values of the cooling
function, $\Lambda(T,Z)$, were calculated for each bin of a
temperature profile with \xspec\ using the flux of the best-fit
spectral model. Following the procedure discussed in
\S\ref{sec:kpr}, $\Lambda$ and $kT$ were interpolated across the
radial grid of the electron density profile. The cooling time profiles
were then fit with a model analogous to that used for fitting $K(r)$:
\begin{equation}
\tcool(r) = t_0 + t_{100} \left(\frac{r}{100 \kpc}\right)^{\alpha}.\nonumber \\
\end{equation}
Shown in Figure \ref{fig:t0} is the logarithmically binned and
cumulative distributions of core cooling times.

As can be seen in Fig. \ref{fig:t0}, the distinct bimodality found in
\kna\ is also present in cooling time. The KMM test for bimodality
finds peaks at \tckmma\ and \tckmmb\ with \tckmmc\ and \tckmmd\ in
each population. The probability that the unimodal distribution is a
better fit is once again exceedingly small, \tckmme. The gap in $t_0$
occurs at $t_0 \approx 1-2$ Gyrs. The pile-up of core cooling times
below 1 Gyr has been noted many times before, for example by
\cite{hu85} and more recently by \cite{dunn08}. This demonstrates that
our cooling times are consistent with the results of others.

However, the \kna\ distribution can also be used to explore the
distribution of core cooling times. Assuming free-free interactions
are the dominant gas cooling mechanism (\ie\ $\epsilon \propto
T^{1/2}$), \cite{radioquiet} show that entropy is related to cooling
time via the formulation:
\begin{equation}
\tcool \approx 10^8 \yrs\ \left(\frac{K}{10 \keV \cmsq}\right)^{3/2} \left(\frac{kT}{5 \keV}\right)^{-1}.
\end{equation}
Using this relationship to calculate core cooling time from entropy,
we find the gap in cooling time occurs at $\tcool \approx 0.7-1.0$
Gyrs. That the gaps in cooling time calculated from spectra and \kna\
are offset from each other suggests that while bimodality occurs only
below a particular cooling time scale ($\sim 1$ Gyr), this may not be
enough. The fundamental explanation of what is happening in cluster
cores may be more closely coupled to gas entropy than gas cooling
time. It is also interesting that the bimodality in \kna\ is sharper
and deeper than it is in $t_0$.

Since cooling time profiles are more sensitive to the resolution of
the temperature profiles than are the entropy profiles, it may be that
resolution effects (\eg\ our temperature interpolation scheme is to
coarse, or averaging over many small-scale temperature fluctuations
significantly increases $t_0$) are limiting the quantification of the
true cooling time of the core and hence altering $t_0$. But again, if
entropy is more closely related to the physical processes in the
cluster core than cooling time, then that the cooling time
distribution does not present with the sharp, deep bimodality seen in
\kna\ suggests entropy is the fundamental quantity related to
bimodality.

%%%%%%%%%%%%%%%%%%%%%%%%%%%%%%%%%%%%%%%%%%%%%%%%%%%%%%%%%%%%
\subsection{Slope and Normalization of Power-law Components}
\label{sec:slopes}
%%%%%%%%%%%%%%%%%%%%%%%%%%%%%%%%%%%%%%%%%%%%%%%%%%%%%%%%%%%%

Beyond $r \approx 100 \kpc$ the entropy profiles show a striking
similarity in the slope of the power-law component which is
independent of the core entropy value. For the full sample, the mean
value of \alphafs\ while for clusters with $\kna < 50 \ent$, mean
\alphaga, and for clusters with $\kna \geq 50 \ent$, mean
\alphagb. The scaling factor of the power-law component,
$\khun$... {\bf{[Discuss minimization of khun?]}}  For the full
sample, the mean value of \khunfs. Again distinguishing between
clusters below and above $\kna\ = 50 \ent$, we find \khunga\ and
\khungb, respectively. Scaling the entropy profiles by the cluster
virial temperature and virial radius does reduce the dispersion of
$\alpha$ and scatter of \khun, but we reserve detailed discussion of
scaling relations for a future paper. This mean slope of $\alpha
\approx 1.2$, is not statistically different from the theoretical
value of 1.1 found by \cite{tozzi01} in hierarchical cluster
formation.

%%%%%%%%%%%%%%%%%%%%%%%%%%%%%%%%%
\section{Summary and Conclusions}
\label{sec:summary}
%%%%%%%%%%%%%%%%%%%%%%%%%%%%%%%%%

We have presented a sample of galaxy cluster entropy profiles created
using archival \chandra\ data for \numcluster\ clusters (\expt), a
project which we have named the Archive of Chandra Cluster Entropy
Profile Tables, or \accept. Our data reduction/analysis code (written
in \perl\ and \idl), reduced data products, data tables,
figures/plots, cluster images, and results of our analysis for all the
clusters and observations are freely available at the \accept\ web
page\footnote{http://www.pa.msu.edu/astro/MC2/accept}. We encourage
observers and theorists to utilize this library of entropy profiles in
their own work.

We created temperature profiles using spectra extracted from a minimum
of three concentric annuli containing 2500 counts each and extending
to either the chip edge or $0.5 R_{180}$, whichever was smaller. We
deprojected surface brightness profiles extracted from $5\arcsec$ bins
to obtain the electron gas density as a function of radius. Entropy
profiles were then calculated from gas density and temperature using
$K(r) = T(r)n(r)^{-2/3}$. Two models for the entropy distribution were
then fit to each profile: a power-law only model and a power-law which
approaches a constant value at small radii.

We have demonstrated that the entropy profiles for the vast majority
of \accept\ clusters are best fit by the model which approaches a
constant entropy, \kna, in the core. The distribution of \kna\ for the
full sample is also bimodal with the two populations separated by a
poorly populated gap at $\kna \approx 40-60 \ent$. For clusters
approximately below and above this gap we find mean \knaga\ and
\knagb, respectively. We also find bimodality in the primary
\hifl\ subsample with the \kna\ gap and two populations having values
consistent with the full \accept\ sample. This indicates bimodality is
not a result of archival bias. All of the entropy profiles in
\accept\ are remarkably similar at radii greater than 100 kpc, and
asymptotically approach the self-similar pure-cooling curve ($r
\propto 1.1$) with a slope of \alphafs.

Two sets of cooling times were calculated for each cluster: 1) by
using the results of the spectroscopic analysis, and 2) from
converting \kna\ into a cooling time. The distribution of cooling
times exhibits the same bimodality found in the \kna\ distribution.
After comparing the cooling times from method (1) and (2), we find
that the gap in the bimodal cooling time distributions occur at
significantly different timescales: $t_{gap} = 2-3$ Gyrs for the
spectroscopic cooling times and $t_{gap} = 0.7-1$ Gyrs for the cooling
times determined using \kna.

After analyzing an ensemble of artificially redshifted entropy
profiles, we find the lack of $\kna \la 10 \ent$ clusters at $z > 0.1$
is most likely a result of resolution effects. Investigation of
possible systematics affecting bes-fit \kna\ values, such as profile
curvature and number of profile bins, revealed no trends which would
significantly effect our results. We come to the conclusion that,
indeed, \kna\ is a measure of average core entropy and not profile
shape or resolution.

\cite{agnframework} put forth a model of AGN feedback which explains
observation of non-zero core entropy for the range $\kna \la 70
\ent$. In addition, the results of this project have already been used
to reveal the existence of a core entropy threshold only below which
\halpha\ and powerful radio emission ($\radpow > 10^{40} \ergps$) are
predominantly found \citep{haradent}. \cite{conduction} have suggested
this entropy threshold is set by the process of electron thermal
conduction in the cluster core. If merger events are capable of
producing cluster cores with $\kna > 70 \ent$, then taking these
processes together -- AGN feedback, conduction, and mergers -- a
closed-loop picture of the ICM's entropy life-cycle is starting to
emerge.

There are still many open questions regarding the evolution of the ICM
and formation of thermal instabilities in cluster cores: How are
clusters with $\kna > 100 \ent$ produced? What are the role of MHD
instabilities, \eg\ MTI \citep{2000ApJ...534..420B,
  2008ApJ...673..758Q} and HBI \citep{2008ApJ...677L...9P}, in shaping
the ICM?  Are the compact X-ray sources we find at the cores of some
BCGs truly coronae?  If so, how did they form and survive in the harsh
ICM? We hope \accept\ will be a useful resource in answering these
questions.

%%%%%%%%%%%%%%%%%
\acknowledgements
%%%%%%%%%%%%%%%%%

K.W.C. thanks Chris Waters for supplying and supporting his new KMM
bimodality code. K.W.C. was supported in this work through
\chandra\ X-ray Observatory Archive grants AR-6016X and
AR-4017A. M.D. acknowledges support from the NASA LTSA program
NNG-05GD82G. The \chandra\ X-ray Observatory Center is operated by the
Smithsonian Astrophysical Observatory for and on behalf of NASA under
contract NAS8-03060. This research has made use of software provided
by the Chandra X-ray Center in the application packages \ciao, \chips,
and \sherpa. This research has made use of the NASA/IPAC Extragalactic
Database which is operated by the Jet Propulsion Laboratory,
California Institute of Technology, under contract with NASA. This
research has also made use of NASA's Astrophysics Data System. Some
software was obtained from the High Energy Astrophysics Science
Archive Research Center, provided by NASA's Goddard Space Flight
Center.

%%%%%%%%%%%%%%%%
% Bibliography %
%%%%%%%%%%%%%%%%

\bibliography{cavagnolo}

%%%%%%%%%%%%%%
% Appendices %
%%%%%%%%%%%%%%

\input{appendices.tex}

%%%%%%%%%%%%%%%%%%%%%%
% Figures  and Tables%
%%%%%%%%%%%%%%%%%%%%%%

\clearpage
\LongTables
\begin{deluxetable}{lcccccccc}
\tablewidth{0pt}
\tabletypesize{\scriptsize}
\tablecaption{Summary of Sample\label{tab:sample}}
\tablehead{\colhead{Cluster} & \colhead{Obs.ID} & \colhead{R.A.} & \colhead{Dec.} & \colhead{ExpT} & \colhead{Mode} & \colhead{ACIS} & \colhead{$z$} & \colhead{$L_{bol.}$}\\
\colhead{ } & \colhead{ } & \colhead{hr:min:sec} & \colhead{$\degr:\arcmin:\arcsec$} & \colhead{ksec} & \colhead{ } & \colhead{ } & \colhead{ } & \colhead{$10^{44}$ ergs s$^{-1}$}\\
\colhead{{(1)}} & \colhead{{(2)}} & \colhead{{(3)}} & \colhead{{(4)}} & \colhead{{(5)}} & \colhead{{(6)}} & \colhead{{(7)}} & \colhead{{(8)}} & \colhead{{(9)}}
}
\startdata
1E0657 56 & \dataset [ADS/Sa.CXO\#obs/03184] {3184} & 06:58:29.627 & -55:56:39.79 & 87.5 & VF & I3 & 0.296 & 52.48\\
1E0657 56 & \dataset [ADS/Sa.CXO\#obs/05356] {5356} & 06:58:29.619 & -55:56:39.35 & 97.2 & VF & I2 & 0.296 & 52.48\\
1E0657 56 & \dataset [ADS/Sa.CXO\#obs/05361] {5361} & 06:58:29.670 & -55:56:39.80 & 82.6 & VF & I3 & 0.296 & 52.48\\
1RXS J2129.4-0741 & \dataset [ADS/Sa.CXO\#obs/03199] {3199} & 21:29:26.274 & -07:41:29.18 & 19.9 & VF & I3 & 0.570 & 20.58\\
1RXS J2129.4-0741 & \dataset [ADS/Sa.CXO\#obs/03595] {3595} & 21:29:26.281 & -07:41:29.36 & 19.9 & VF & I3 & 0.570 & 20.58\\
2PIGG J0011.5-2850 & \dataset [ADS/Sa.CXO\#obs/05797] {5797} & 00:11:21.623 & -28:51:14.44 & 19.9 & VF & I3 & 0.075 &  2.15\\
2PIGG J0311.8-2655 $\dagger$ & \dataset [ADS/Sa.CXO\#obs/05799] {5799} & 03:11:33.904 & -26:54:16.48 & 39.6 & VF & I3 & 0.062 &  0.25\\
2PIGG J2227.0-3041 & \dataset [ADS/Sa.CXO\#obs/05798] {5798} & 22:27:54.560 & -30:34:34.84 & 22.3 & VF & I2 & 0.073 &  0.81\\
3C 220.1 & \dataset [ADS/Sa.CXO\#obs/00839] {839} & 09:32:40.218 & +79:06:29.46 & 18.9 &  F & S3 & 0.610 &  3.25\\
3C 28.0 & \dataset [ADS/Sa.CXO\#obs/03233] {3233} & 00:55:50.401 & +26:24:36.47 & 49.7 & VF & I3 & 0.195 &  4.78\\
3C 295 & \dataset [ADS/Sa.CXO\#obs/02254] {2254} & 14:11:20.280 & +52:12:10.55 & 90.9 & VF & I3 & 0.464 &  6.92\\
3C 388 & \dataset [ADS/Sa.CXO\#obs/05295] {5295} & 18:44:02.365 & +45:33:29.31 & 30.7 & VF & I3 & 0.092 &  0.52\\
4C 55.16 & \dataset [ADS/Sa.CXO\#obs/04940] {4940} & 08:34:54.923 & +55:34:21.15 & 96.0 & VF & S3 & 0.242 &  5.90\\
ABELL 0013 $\dagger$ & \dataset [ADS/Sa.CXO\#obs/04945] {4945} & 00:13:37.883 & -19:30:09.10 & 55.3 & VF & S3 & 0.094 &  1.41\\
ABELL 0068 & \dataset [ADS/Sa.CXO\#obs/03250] {3250} & 00:37:06.309 & +09:09:32.28 & 10.0 & VF & I3 & 0.255 & 12.70\\
ABELL 0119 $\dagger$ & \dataset [ADS/Sa.CXO\#obs/04180] {4180} & 00:56:15.150 & -01:14:59.70 & 11.9 & VF & I3 & 0.044 &  1.39\\
ABELL 0168 & \dataset [ADS/Sa.CXO\#obs/03203] {3203} & 01:14:57.909 & +00:24:42.55 & 40.6 & VF & I3 & 0.045 &  0.23\\
ABELL 0168 & \dataset [ADS/Sa.CXO\#obs/03204] {3204} & 01:14:57.925 & +00:24:42.73 & 37.6 & VF & I3 & 0.045 &  0.23\\
ABELL 0209 & \dataset [ADS/Sa.CXO\#obs/03579] {3579} & 01:31:52.565 & -13:36:39.29 & 10.0 & VF & I3 & 0.206 & 10.96\\
ABELL 0209 & \dataset [ADS/Sa.CXO\#obs/00522] {522} & 01:31:52.595 & -13:36:39.25 & 10.0 & VF & I3 & 0.206 & 10.96\\
ABELL 0267 & \dataset [ADS/Sa.CXO\#obs/01448] {1448} & 01:52:29.181 & +00:57:34.43 & 7.9 &  F & I3 & 0.230 &  8.62\\
ABELL 0267 & \dataset [ADS/Sa.CXO\#obs/03580] {3580} & 01:52:29.180 & +00:57:34.23 & 19.9 & VF & I3 & 0.230 &  8.62\\
ABELL 0370 & \dataset [ADS/Sa.CXO\#obs/00515] {515} & 02:39:53.169 & -01:34:36.96 & 88.0 &  F & S3 & 0.375 & 11.95\\
ABELL 0383 & \dataset [ADS/Sa.CXO\#obs/02321] {2321} & 02:48:03.364 & -03:31:44.69 & 19.5 &  F & S3 & 0.187 &  5.32\\
ABELL 0399 & \dataset [ADS/Sa.CXO\#obs/03230] {3230} & 02:57:54.931 & +13:01:58.41 & 48.6 & VF & I0 & 0.072 &  4.37\\
ABELL 0401 & \dataset [ADS/Sa.CXO\#obs/00518] {518} & 02:58:56.896 & +13:34:14.48 & 18.0 &  F & I3 & 0.074 &  8.39\\
ABELL 0478 & \dataset [ADS/Sa.CXO\#obs/06102] {6102} & 04:13:25.347 & +10:27:55.62 & 10.0 & VF & I3 & 0.088 & 16.39\\
ABELL 0514 & \dataset [ADS/Sa.CXO\#obs/03578] {3578} & 04:48:19.229 & -20:30:28.79 & 44.5 & VF & I3 & 0.072 &  0.66\\
ABELL 0520 & \dataset [ADS/Sa.CXO\#obs/04215] {4215} & 04:54:09.711 & +02:55:23.69 & 66.3 & VF & I3 & 0.202 & 12.97\\
ABELL 0521 & \dataset [ADS/Sa.CXO\#obs/00430] {430} & 04:54:07.004 & -10:13:26.72 & 39.1 & VF & S3 & 0.253 &  9.77\\
ABELL 0586 & \dataset [ADS/Sa.CXO\#obs/00530] {530} & 07:32:20.339 & +31:37:58.59 & 10.0 & VF & I3 & 0.171 &  8.54\\
ABELL 0611 & \dataset [ADS/Sa.CXO\#obs/03194] {3194} & 08:00:56.832 & +36:03:24.09 & 36.1 & VF & S3 & 0.288 & 10.78\\
ABELL 0644 $\dagger$ & \dataset [ADS/Sa.CXO\#obs/02211] {2211} & 08:17:25.225 & -07:30:40.03 & 29.7 & VF & I3 & 0.070 &  6.95\\
ABELL 0665 & \dataset [ADS/Sa.CXO\#obs/03586] {3586} & 08:30:59.231 & +65:50:37.78 & 29.7 & VF & I3 & 0.181 & 13.37\\
ABELL 0697 & \dataset [ADS/Sa.CXO\#obs/04217] {4217} & 08:42:57.549 & +36:21:57.65 & 19.5 & VF & I3 & 0.282 & 26.10\\
ABELL 0773 & \dataset [ADS/Sa.CXO\#obs/05006] {5006} & 09:17:52.566 & +51:43:38.18 & 19.8 & VF & I3 & 0.217 & 12.87\\
ABELL 0781 & \dataset [ADS/Sa.CXO\#obs/00534] {534} & 09:20:25.431 & +30:30:07.56 & 9.9 & VF & I3 & 0.298 &  0.00\\
ABELL 0907 & \dataset [ADS/Sa.CXO\#obs/03185] {3185} & 09:58:21.880 & -11:03:52.20 & 48.0 & VF & I3 & 0.153 &  6.19\\
ABELL 0963 & \dataset [ADS/Sa.CXO\#obs/00903] {903} & 10:17:03.744 & +39:02:49.17 & 36.3 &  F & S3 & 0.206 & 10.65\\
ABELL 1063S & \dataset [ADS/Sa.CXO\#obs/04966] {4966} & 22:48:44.294 & -44:31:48.37 & 26.7 & VF & I3 & 0.354 & 71.09\\
ABELL 1068 $\dagger$ & \dataset [ADS/Sa.CXO\#obs/01652] {1652} & 10:40:44.520 & +39:57:10.28 & 26.8 &  F & S3 & 0.138 &  4.19\\
ABELL 1201 $\dagger$ & \dataset [ADS/Sa.CXO\#obs/04216] {4216} & 11:12:54.489 & +13:26:08.76 & 39.7 & VF & S3 & 0.169 &  3.52\\
ABELL 1204 & \dataset [ADS/Sa.CXO\#obs/02205] {2205} & 11:13:20.419 & +17:35:38.45 & 23.6 & VF & I3 & 0.171 &  3.92\\
ABELL 1361 $\dagger$ & \dataset [ADS/Sa.CXO\#obs/02200] {2200} & 11:43:39.827 & +46:21:21.40 & 16.7 &  F & S3 & 0.117 &  2.16\\
ABELL 1423 & \dataset [ADS/Sa.CXO\#obs/00538] {538} & 11:57:17.026 & +33:36:37.44 & 9.8 & VF & I3 & 0.213 &  7.01\\
ABELL 1651 & \dataset [ADS/Sa.CXO\#obs/04185] {4185} & 12:59:22.830 & -04:11:45.86 & 9.6 & VF & I3 & 0.084 &  6.66\\
ABELL 1664 $\dagger$ & \dataset [ADS/Sa.CXO\#obs/01648] {1648} & 13:03:42.478 & -24:14:44.55 & 9.8 & VF & S3 & 0.128 &  2.59\\
ABELL 1682 & \dataset [ADS/Sa.CXO\#obs/03244] {3244} & 13:06:50.764 & +46:33:19.86 & 9.8 & VF & I3 & 0.226 &  0.00\\
ABELL 1689 & \dataset [ADS/Sa.CXO\#obs/01663] {1663} & 13:11:29.612 & -01:20:28.69 & 10.7 &  F & I3 & 0.184 & 24.71\\
ABELL 1689 & \dataset [ADS/Sa.CXO\#obs/05004] {5004} & 13:11:29.606 & -01:20:28.61 & 19.9 & VF & I3 & 0.184 & 24.71\\
ABELL 1689 & \dataset [ADS/Sa.CXO\#obs/00540] {540} & 13:11:29.595 & -01:20:28.47 & 10.3 &  F & I3 & 0.184 & 24.71\\
ABELL 1758 & \dataset [ADS/Sa.CXO\#obs/02213] {2213} & 13:32:42.978 & +50:32:44.83 & 58.3 & VF & S3 & 0.279 & 21.01\\
ABELL 1763 & \dataset [ADS/Sa.CXO\#obs/03591] {3591} & 13:35:17.957 & +40:59:55.80 & 19.6 & VF & I3 & 0.187 &  9.26\\
ABELL 1795 $\dagger$ & \dataset [ADS/Sa.CXO\#obs/05289] {5289} & 13:48:52.829 & +26:35:24.01 & 15.0 & VF & I3 & 0.062 &  7.59\\
ABELL 1835 & \dataset [ADS/Sa.CXO\#obs/00495] {495} & 14:01:01.951 & +02:52:43.18 & 19.5 &  F & S3 & 0.253 & 39.38\\
ABELL 1914 & \dataset [ADS/Sa.CXO\#obs/03593] {3593} & 14:26:01.399 & +37:49:27.83 & 18.9 & VF & I3 & 0.171 & 26.25\\
ABELL 1942 & \dataset [ADS/Sa.CXO\#obs/03290] {3290} & 14:38:21.878 & +03:40:12.97 & 57.6 & VF & I2 & 0.224 &  2.27\\
ABELL 1995 & \dataset [ADS/Sa.CXO\#obs/00906] {906} & 14:52:57.758 & +58:02:51.34 & 0.0 &  F & S3 & 0.319 & 10.19\\
ABELL 2029 $\dagger$ & \dataset [ADS/Sa.CXO\#obs/06101] {6101} & 15:10:56.163 & +05:44:40.89 & 9.9 & VF & I3 & 0.076 & 13.90\\
ABELL 2034 & \dataset [ADS/Sa.CXO\#obs/02204] {2204} & 15:10:11.003 & +33:30:46.46 & 53.9 & VF & I3 & 0.113 &  6.45\\
ABELL 2065 $\dagger$ & \dataset [ADS/Sa.CXO\#obs/031821] {31821} & 15:22:29.220 & +27:42:46.54 & 0.0 & VF & I3 & 0.073 &  2.92\\
ABELL 2069 & \dataset [ADS/Sa.CXO\#obs/04965] {4965} & 15:24:09.181 & +29:53:18.05 & 55.4 & VF & I2 & 0.116 &  3.82\\
ABELL 2111 & \dataset [ADS/Sa.CXO\#obs/00544] {544} & 15:39:41.432 & +34:25:12.26 & 10.3 &  F & I3 & 0.230 &  7.45\\
ABELL 2125 & \dataset [ADS/Sa.CXO\#obs/02207] {2207} & 15:41:14.154 & +66:15:57.20 & 81.5 & VF & I3 & 0.246 &  0.77\\
ABELL 2163 & \dataset [ADS/Sa.CXO\#obs/01653] {1653} & 16:15:45.705 & -06:09:00.62 & 71.1 & VF & I1 & 0.170 & 49.11\\
ABELL 2204 $\dagger$ & \dataset [ADS/Sa.CXO\#obs/0499] {499} & 16:32:45.437 & +05:34:21.05 & 10.1 &  F & S3 & 0.152 & 20.77\\
ABELL 2204 & \dataset [ADS/Sa.CXO\#obs/06104] {6104} & 16:32:45.428 & +05:34:20.89 & 9.6 & VF & I3 & 0.152 & 22.03\\
ABELL 2218 & \dataset [ADS/Sa.CXO\#obs/01666] {1666} & 16:35:50.831 & +66:12:42.31 & 48.6 & VF & I0 & 0.171 &  8.39\\
ABELL 2219 $\dagger$ & \dataset [ADS/Sa.CXO\#obs/0896] {896} & 16:40:21.069 & +46:42:29.07 & 42.3 &  F & S3 & 0.226 & 33.15\\
ABELL 2255 & \dataset [ADS/Sa.CXO\#obs/00894] {894} & 17:12:40.385 & +64:03:50.63 & 39.4 &  F & I3 & 0.081 &  3.67\\
ABELL 2256 $\dagger$ & \dataset [ADS/Sa.CXO\#obs/01386] {1386} & 17:03:44.567 & +78:38:11.51 & 12.4 &  F & I3 & 0.058 &  4.65\\
ABELL 2259 & \dataset [ADS/Sa.CXO\#obs/03245] {3245} & 17:20:08.299 & +27:40:11.53 & 10.0 & VF & I3 & 0.164 &  5.37\\
ABELL 2261 & \dataset [ADS/Sa.CXO\#obs/05007] {5007} & 17:22:27.254 & +32:07:58.60 & 24.3 & VF & I3 & 0.224 & 17.49\\
ABELL 2294 & \dataset [ADS/Sa.CXO\#obs/03246] {3246} & 17:24:10.149 & +85:53:09.77 & 10.0 & VF & I3 & 0.178 & 10.35\\
ABELL 2384 & \dataset [ADS/Sa.CXO\#obs/04202] {4202} & 21:52:21.178 & -19:32:51.90 & 31.5 & VF & I3 & 0.095 &  1.95\\
ABELL 2390 $\dagger$ & \dataset [ADS/Sa.CXO\#obs/04193] {4193} & 21:53:36.825 & +17:41:44.38 & 95.1 & VF & S3 & 0.230 & 31.02\\
ABELL 2409 & \dataset [ADS/Sa.CXO\#obs/03247] {3247} & 22:00:52.567 & +20:58:34.11 & 10.2 & VF & I3 & 0.148 &  7.01\\
ABELL 2537 & \dataset [ADS/Sa.CXO\#obs/04962] {4962} & 23:08:22.313 & -02:11:29.88 & 36.2 & VF & S3 & 0.295 & 10.16\\
ABELL 2550 & \dataset [ADS/Sa.CXO\#obs/02225] {2225} & 23:11:35.806 & -21:44:46.70 & 59.0 & VF & S3 & 0.154 &  0.58\\
ABELL 2554 $\dagger$ & \dataset [ADS/Sa.CXO\#obs/01696] {1696} & 23:12:19.939 & -21:30:09.84 & 19.9 & VF & S3 & 0.110 &  1.57\\
ABELL 2556 $\dagger$ & \dataset [ADS/Sa.CXO\#obs/02226] {2226} & 23:13:01.413 & -21:38:04.47 & 19.9 & VF & S3 & 0.086 &  1.43\\
ABELL 2631 & \dataset [ADS/Sa.CXO\#obs/03248] {3248} & 23:37:38.560 & +00:16:28.64 & 9.2 & VF & I3 & 0.278 & 12.59\\
ABELL 2667 & \dataset [ADS/Sa.CXO\#obs/02214] {2214} & 23:51:39.395 & -26:05:02.75 & 9.6 & VF & S3 & 0.230 & 19.91\\
ABELL 2670 & \dataset [ADS/Sa.CXO\#obs/04959] {4959} & 23:54:13.687 & -10:25:08.85 & 39.6 & VF & I3 & 0.076 &  1.39\\
ABELL 2717 & \dataset [ADS/Sa.CXO\#obs/06974] {6974} & 00:03:11.996 & -35:56:08.01 & 19.8 & VF & I3 & 0.048 &  0.26\\
ABELL 2744 & \dataset [ADS/Sa.CXO\#obs/02212] {2212} & 00:14:14.396 & -30:22:40.04 & 24.8 & VF & S3 & 0.308 & 29.00\\
ABELL 3128 $\dagger$ & \dataset [ADS/Sa.CXO\#obs/00893] {893} & 03:29:50.918 & -52:34:51.04 & 19.6 &  F & I3 & 0.062 &  0.35\\
ABELL 3158 $\dagger$ & \dataset [ADS/Sa.CXO\#obs/03201] {3201} & 03:42:54.675 & -53:37:24.36 & 24.8 & VF & I3 & 0.059 &  3.01\\
ABELL 3158 $\dagger$ & \dataset [ADS/Sa.CXO\#obs/03712] {3712} & 03:42:54.683 & -53:37:24.37 & 30.9 & VF & I3 & 0.059 &  3.01\\
ABELL 3164 & \dataset [ADS/Sa.CXO\#obs/06955] {6955} & 03:46:16.839 & -57:02:11.38 & 13.5 & VF & I3 & 0.057 &  0.19\\
ABELL 3376 & \dataset [ADS/Sa.CXO\#obs/03202] {3202} & 06:02:05.122 & -39:57:42.82 & 44.3 & VF & I3 & 0.046 &  0.75\\
ABELL 3376 & \dataset [ADS/Sa.CXO\#obs/03450] {3450} & 06:02:05.162 & -39:57:42.87 & 19.8 & VF & I3 & 0.046 &  0.75\\
ABELL 3391 $\dagger$ & \dataset [ADS/Sa.CXO\#obs/04943] {4943} & 06:26:21.511 & -53:41:44.81 & 18.4 & VF & I3 & 0.056 &  1.44\\
ABELL 3921 & \dataset [ADS/Sa.CXO\#obs/04973] {4973} & 22:49:57.829 & -64:25:42.17 & 29.4 & VF & I3 & 0.093 &  3.37\\
AC 114 & \dataset [ADS/Sa.CXO\#obs/01562] {1562} & 22:58:48.196 & -34:47:56.89 & 72.5 &  F & S3 & 0.312 & 10.90\\
CL 0024+17 & \dataset [ADS/Sa.CXO\#obs/00929] {929} & 00:26:35.996 & +17:09:45.37 & 39.8 &  F & S3 & 0.394 &  2.88\\
CL 1221+4918 & \dataset [ADS/Sa.CXO\#obs/01662] {1662} & 12:21:26.709 & +49:18:21.60 & 79.1 & VF & I3 & 0.700 &  8.65\\
CL J0030+2618 & \dataset [ADS/Sa.CXO\#obs/05762] {5762} & 00:30:34.339 & +26:18:01.58 & 17.9 & VF & I3 & 0.500 &  3.41\\
CL J0152-1357 & \dataset [ADS/Sa.CXO\#obs/00913] {913} & 01:52:42.141 & -13:57:59.71 & 36.5 &  F & I3 & 0.831 & 13.30\\
CL J0542.8-4100 & \dataset [ADS/Sa.CXO\#obs/00914] {914} & 05:42:49.994 & -40:59:58.50 & 50.4 &  F & I3 & 0.630 &  6.18\\
CL J0848+4456 & \dataset [ADS/Sa.CXO\#obs/01708] {1708} & 08:48:48.235 & +44:56:17.11 & 61.4 & VF & I1 & 0.574 &  0.62\\
CL J0848+4456 & \dataset [ADS/Sa.CXO\#obs/00927] {927} & 08:48:48.252 & +44:56:17.13 & 125.1 & VF & I1 & 0.574 &  0.62\\
CL J1113.1-2615 & \dataset [ADS/Sa.CXO\#obs/00915] {915} & 11:13:05.167 & -26:15:40.43 & 104.6 &  F & I3 & 0.730 &  2.22\\
CL J1213+0253 & \dataset [ADS/Sa.CXO\#obs/04934] {4934} & 12:13:34.948 & +02:53:45.45 & 18.9 & VF & I3 & 0.409 &  0.00\\
CL J1226.9+3332 & \dataset [ADS/Sa.CXO\#obs/03180] {3180} & 12:26:58.373 & +33:32:47.36 & 31.7 & VF & I3 & 0.890 & 30.76\\
CL J1226.9+3332 & \dataset [ADS/Sa.CXO\#obs/05014] {5014} & 12:26:58.372 & +33:32:47.18 & 32.7 & VF & I3 & 0.890 & 30.76\\
CL J1641+4001 & \dataset [ADS/Sa.CXO\#obs/03575] {3575} & 16:41:53.704 & +40:01:44.40 & 46.5 & VF & I3 & 0.464 &  0.00\\
CL J2302.8+0844 & \dataset [ADS/Sa.CXO\#obs/00918] {918} & 23:02:48.156 & +08:43:52.74 & 108.6 &  F & I3 & 0.730 &  2.93\\
DLS J0514-4904 & \dataset [ADS/Sa.CXO\#obs/04980] {4980} & 05:14:40.037 & -49:03:15.07 & 19.9 & VF & I3 & 0.091 &  0.68\\
EXO 0422-086 $\dagger$ & \dataset [ADS/Sa.CXO\#obs/04183] {4183} & 04:25:51.271 & -08:33:36.42 & 10.0 & VF & I3 & 0.040 &  0.65\\
HERCULES A $\dagger$ & \dataset [ADS/Sa.CXO\#obs/01625] {1625} & 16:51:08.161 & +04:59:32.44 & 14.8 & VF & S3 & 0.154 &  3.27\\
IRAS 09104+4109 & \dataset [ADS/Sa.CXO\#obs/00509] {509} & 09:13:45.481 & +40:56:27.49 & 9.1 &  F & S3 & 0.442 &  0.00\\
LYNX E & \dataset [ADS/Sa.CXO\#obs/017081] {17081} & 08:48:58.851 & +44:51:51.44 & 61.4 & VF & I2 & 1.260 &  0.00\\
LYNX E & \dataset [ADS/Sa.CXO\#obs/09271] {9271} & 08:48:58.858 & +44:51:51.46 & 125.1 & VF & I2 & 1.260 &  0.00\\
MACS J0011.7-1523 & \dataset [ADS/Sa.CXO\#obs/03261] {3261} & 00:11:42.965 & -15:23:20.79 & 21.6 & VF & I3 & 0.360 & 10.75\\
MACS J0011.7-1523 & \dataset [ADS/Sa.CXO\#obs/06105] {6105} & 00:11:42.957 & -15:23:20.76 & 37.3 & VF & I3 & 0.360 & 10.75\\
MACS J0025.4-1222 & \dataset [ADS/Sa.CXO\#obs/03251] {3251} & 00:25:29.368 & -12:22:38.05 & 19.3 & VF & I3 & 0.584 & 13.00\\
MACS J0025.4-1222 & \dataset [ADS/Sa.CXO\#obs/05010] {5010} & 00:25:29.399 & -12:22:38.10 & 24.8 & VF & I3 & 0.584 & 13.00\\
MACS J0035.4-2015 & \dataset [ADS/Sa.CXO\#obs/03262] {3262} & 00:35:26.573 & -20:15:46.06 & 21.4 & VF & I3 & 0.364 & 19.79\\
MACS J0111.5+0855 & \dataset [ADS/Sa.CXO\#obs/03256] {3256} & 01:11:31.515 & +08:55:39.21 & 19.4 & VF & I3 & 0.263 &  0.64\\
MACS J0152.5-2852 & \dataset [ADS/Sa.CXO\#obs/03264] {3264} & 01:52:34.479 & -28:53:38.01 & 17.5 & VF & I3 & 0.341 &  6.33\\
MACS J0159.0-3412 & \dataset [ADS/Sa.CXO\#obs/05818] {5818} & 01:59:00.366 & -34:13:00.23 & 9.4 & VF & I3 & 0.458 & 18.92\\
MACS J0159.8-0849 & \dataset [ADS/Sa.CXO\#obs/03265] {3265} & 01:59:49.453 & -08:50:00.90 & 17.9 & VF & I3 & 0.405 & 26.31\\
MACS J0159.8-0849 & \dataset [ADS/Sa.CXO\#obs/06106] {6106} & 01:59:49.422 & -08:50:00.42 & 35.3 & VF & I3 & 0.405 & 26.31\\
MACS J0242.5-2132 & \dataset [ADS/Sa.CXO\#obs/03266] {3266} & 02:42:35.906 & -21:32:26.30 & 11.9 & VF & I3 & 0.314 & 12.74\\
MACS J0257.1-2325 & \dataset [ADS/Sa.CXO\#obs/01654] {1654} & 02:57:09.130 & -23:26:06.25 & 19.8 &  F & I3 & 0.505 & 21.72\\
MACS J0257.1-2325 & \dataset [ADS/Sa.CXO\#obs/03581] {3581} & 02:57:09.152 & -23:26:06.21 & 18.5 & VF & I3 & 0.505 & 21.72\\
MACS J0257.6-2209 & \dataset [ADS/Sa.CXO\#obs/03267] {3267} & 02:57:41.024 & -22:09:11.12 & 20.5 & VF & I3 & 0.322 & 10.77\\
MACS J0308.9+2645 & \dataset [ADS/Sa.CXO\#obs/03268] {3268} & 03:08:55.927 & +26:45:38.34 & 24.4 & VF & I3 & 0.324 & 20.42\\
MACS J0329.6-0211 & \dataset [ADS/Sa.CXO\#obs/03257] {3257} & 03:29:41.681 & -02:11:47.67 & 9.9 & VF & I3 & 0.450 & 12.82\\
MACS J0329.6-0211 & \dataset [ADS/Sa.CXO\#obs/03582] {3582} & 03:29:41.688 & -02:11:47.81 & 19.9 & VF & I3 & 0.450 & 12.82\\
MACS J0329.6-0211 & \dataset [ADS/Sa.CXO\#obs/06108] {6108} & 03:29:41.681 & -02:11:47.57 & 39.6 & VF & I3 & 0.450 & 12.82\\
MACS J0404.6+1109 & \dataset [ADS/Sa.CXO\#obs/03269] {3269} & 04:04:32.491 & +11:08:02.10 & 21.8 & VF & I3 & 0.355 &  3.90\\
MACS J0417.5-1154 & \dataset [ADS/Sa.CXO\#obs/03270] {3270} & 04:17:34.686 & -11:54:32.71 & 12.0 & VF & I3 & 0.440 & 37.99\\
MACS J0429.6-0253 & \dataset [ADS/Sa.CXO\#obs/03271] {3271} & 04:29:36.088 & -02:53:09.02 & 23.2 & VF & I3 & 0.399 & 11.58\\
MACS J0451.9+0006 & \dataset [ADS/Sa.CXO\#obs/05815] {5815} & 04:51:54.291 & +00:06:20.20 & 10.2 & VF & I3 & 0.430 &  8.20\\
MACS J0455.2+0657 & \dataset [ADS/Sa.CXO\#obs/05812] {5812} & 04:55:17.426 & +06:57:47.15 & 9.9 & VF & I3 & 0.425 &  9.77\\
MACS J0520.7-1328 & \dataset [ADS/Sa.CXO\#obs/03272] {3272} & 05:20:42.052 & -13:28:49.38 & 19.2 & VF & I3 & 0.340 &  9.63\\
MACS J0547.0-3904 & \dataset [ADS/Sa.CXO\#obs/03273] {3273} & 05:47:01.582 & -39:04:28.24 & 21.7 & VF & I3 & 0.210 &  1.59\\
MACS J0553.4-3342 & \dataset [ADS/Sa.CXO\#obs/05813] {5813} & 05:53:27.200 & -33:42:53.02 & 9.9 & VF & I3 & 0.407 & 32.68\\
MACS J0717.5+3745 & \dataset [ADS/Sa.CXO\#obs/01655] {1655} & 07:17:31.654 & +37:45:18.52 & 19.9 &  F & I3 & 0.548 & 46.58\\
MACS J0717.5+3745 & \dataset [ADS/Sa.CXO\#obs/04200] {4200} & 07:17:31.651 & +37:45:18.46 & 59.2 & VF & I3 & 0.548 & 46.58\\
MACS J0744.8+3927 & \dataset [ADS/Sa.CXO\#obs/03197] {3197} & 07:44:52.802 & +39:27:24.41 & 20.2 & VF & I3 & 0.686 & 24.67\\
MACS J0744.8+3927 & \dataset [ADS/Sa.CXO\#obs/03585] {3585} & 07:44:52.779 & +39:27:24.41 & 19.9 & VF & I3 & 0.686 & 24.67\\
MACS J0744.8+3927 & \dataset [ADS/Sa.CXO\#obs/06111] {6111} & 07:44:52.800 & +39:27:24.41 & 49.5 & VF & I3 & 0.686 & 24.67\\
MACS J0911.2+1746 & \dataset [ADS/Sa.CXO\#obs/03587] {3587} & 09:11:11.325 & +17:46:31.06 & 17.9 & VF & I3 & 0.541 & 10.52\\
MACS J0911.2+1746 & \dataset [ADS/Sa.CXO\#obs/05012] {5012} & 09:11:11.309 & +17:46:30.92 & 23.8 & VF & I3 & 0.541 & 10.52\\
MACS J0949+1708   & \dataset [ADS/Sa.CXO\#obs/03274] {3274} & 09:49:51.824 & +17:07:05.62 & 14.3 & VF & I3 & 0.382 & 19.19\\
MACS J1006.9+3200 & \dataset [ADS/Sa.CXO\#obs/05819] {5819} & 10:06:54.668 & +32:01:34.61 & 10.9 & VF & I3 & 0.359 &  6.06\\
MACS J1105.7-1014 & \dataset [ADS/Sa.CXO\#obs/05817] {5817} & 11:05:46.462 & -10:14:37.20 & 10.3 & VF & I3 & 0.466 & 11.29\\
MACS J1108.8+0906 & \dataset [ADS/Sa.CXO\#obs/03252] {3252} & 11:08:55.393 & +09:05:51.16 & 9.9 & VF & I3 & 0.449 &  8.96\\
MACS J1108.8+0906 & \dataset [ADS/Sa.CXO\#obs/05009] {5009} & 11:08:55.402 & +09:05:51.14 & 24.5 & VF & I3 & 0.449 &  8.96\\
MACS J1115.2+5320 & \dataset [ADS/Sa.CXO\#obs/03253] {3253} & 11:15:15.632 & +53:20:03.71 & 8.8 & VF & I3 & 0.439 & 14.29\\
MACS J1115.2+5320 & \dataset [ADS/Sa.CXO\#obs/05008] {5008} & 11:15:15.646 & +53:20:03.74 & 18.0 & VF & I3 & 0.439 & 14.29\\
MACS J1115.2+5320 & \dataset [ADS/Sa.CXO\#obs/05350] {5350} & 11:15:15.632 & +53:20:03.37 & 6.9 & VF & I3 & 0.439 & 14.29\\
MACS J1115.8+0129 & \dataset [ADS/Sa.CXO\#obs/03275] {3275} & 11:15:52.048 & +01:29:56.56 & 15.9 & VF & I3 & 0.120 &  1.47\\
MACS J1131.8-1955 & \dataset [ADS/Sa.CXO\#obs/03276] {3276} & 11:31:56.011 & -19:55:55.85 & 13.9 & VF & I3 & 0.307 & 17.45\\
MACS J1149.5+2223 & \dataset [ADS/Sa.CXO\#obs/01656] {1656} & 11:49:35.856 & +22:23:55.02 & 18.5 & VF & I3 & 0.544 & 21.60\\
MACS J1149.5+2223 & \dataset [ADS/Sa.CXO\#obs/03589] {3589} & 11:49:35.848 & +22:23:55.05 & 20.0 & VF & I3 & 0.544 & 21.60\\
MACS J1206.2-0847 & \dataset [ADS/Sa.CXO\#obs/03277] {3277} & 12:06:12.276 & -08:48:02.40 & 23.5 & VF & I3 & 0.440 & 37.02\\
MACS J1226.8+2153 & \dataset [ADS/Sa.CXO\#obs/03590] {3590} & 12:26:51.207 & +21:49:55.22 & 19.0 & VF & I3 & 0.370 &  2.63\\
MACS J1311.0-0310 & \dataset [ADS/Sa.CXO\#obs/03258] {3258} & 13:11:01.665 & -03:10:39.50 & 14.9 & VF & I3 & 0.494 & 10.03\\
MACS J1311.0-0310 & \dataset [ADS/Sa.CXO\#obs/06110] {6110} & 13:11:01.680 & -03:10:39.75 & 63.2 & VF & I3 & 0.494 & 10.03\\
MACS J1319+7003   & \dataset [ADS/Sa.CXO\#obs/03278] {3278} & 13:20:08.370 & +70:04:33.81 & 21.6 & VF & I3 & 0.328 &  7.03\\
MACS J1427.2+4407 & \dataset [ADS/Sa.CXO\#obs/06112] {6112} & 14:27:16.175 & +44:07:30.33 & 9.4 & VF & I3 & 0.477 & 14.18\\
MACS J1427.6-2521 & \dataset [ADS/Sa.CXO\#obs/03279] {3279} & 14:27:39.389 & -25:21:04.66 & 16.9 & VF & I3 & 0.220 &  1.55\\
MACS J1621.3+3810 & \dataset [ADS/Sa.CXO\#obs/03254] {3254} & 16:21:25.552 & +38:09:43.56 & 9.8 & VF & I3 & 0.461 & 11.49\\
MACS J1621.3+3810 & \dataset [ADS/Sa.CXO\#obs/03594] {3594} & 16:21:25.558 & +38:09:43.44 & 19.7 & VF & I3 & 0.461 & 11.49\\
MACS J1621.3+3810 & \dataset [ADS/Sa.CXO\#obs/06109] {6109} & 16:21:25.535 & +38:09:43.34 & 37.5 & VF & I3 & 0.461 & 11.49\\
MACS J1621.3+3810 & \dataset [ADS/Sa.CXO\#obs/06172] {6172} & 16:21:25.559 & +38:09:43.63 & 29.8 & VF & I3 & 0.461 & 11.49\\
MACS J1731.6+2252 & \dataset [ADS/Sa.CXO\#obs/03281] {3281} & 17:31:39.902 & +22:52:00.55 & 20.5 & VF & I3 & 0.366 &  9.32\\
MACS J1824.3+4309 & \dataset [ADS/Sa.CXO\#obs/03255] {3255} & 18:24:18.444 & +43:09:43.39 & 14.9 & VF & I3 & 0.487 &  0.00\\
MACS J1931.8-2634 & \dataset [ADS/Sa.CXO\#obs/03282] {3282} & 19:31:49.656 & -26:34:33.99 & 13.6 & VF & I3 & 0.352 & 23.14\\
MACS J2046.0-3430 & \dataset [ADS/Sa.CXO\#obs/05816] {5816} & 20:46:00.522 & -34:30:15.50 & 10.0 & VF & I3 & 0.413 &  5.79\\
MACS J2049.9-3217 & \dataset [ADS/Sa.CXO\#obs/03283] {3283} & 20:49:56.245 & -32:16:52.30 & 23.8 & VF & I3 & 0.325 &  8.71\\
MACS J2211.7-0349 & \dataset [ADS/Sa.CXO\#obs/03284] {3284} & 22:11:45.856 & -03:49:37.24 & 17.7 & VF & I3 & 0.270 & 22.11\\
MACS J2214.9-1359 & \dataset [ADS/Sa.CXO\#obs/03259] {3259} & 22:14:57.467 & -14:00:09.35 & 19.5 & VF & I3 & 0.503 & 24.05\\
MACS J2214.9-1359 & \dataset [ADS/Sa.CXO\#obs/05011] {5011} & 22:14:57.481 & -14:00:09.39 & 18.5 & VF & I3 & 0.503 & 24.05\\
MACS J2228+2036   & \dataset [ADS/Sa.CXO\#obs/03285] {3285} & 22:28:33.241 & +20:37:11.42 & 19.9 & VF & I3 & 0.412 & 17.92\\
MACS J2229.7-2755 & \dataset [ADS/Sa.CXO\#obs/03286] {3286} & 22:29:45.358 & -27:55:38.41 & 16.4 & VF & I3 & 0.324 &  9.49\\
MACS J2243.3-0935 & \dataset [ADS/Sa.CXO\#obs/03260] {3260} & 22:43:21.537 & -09:35:44.30 & 20.5 & VF & I3 & 0.101 &  0.78\\
MACS J2245.0+2637 & \dataset [ADS/Sa.CXO\#obs/03287] {3287} & 22:45:04.547 & +26:38:07.88 & 16.9 & VF & I3 & 0.304 &  9.36\\
MACS J2311+0338   & \dataset [ADS/Sa.CXO\#obs/03288] {3288} & 23:11:33.213 & +03:38:06.51 & 13.6 & VF & I3 & 0.300 & 10.98\\
MKW3S & \dataset [ADS/Sa.CXO\#obs/0900] {900} & 15:21:51.930 & +07:42:31.97 & 57.3 & VF & I3 & 0.045 &  1.14\\
MS 0016.9+1609 & \dataset [ADS/Sa.CXO\#obs/00520] {520} & 00:18:33.503 & +16:26:12.99 & 67.4 & VF & I3 & 0.541 & 32.94\\
MS 0302.7+1658 & \dataset [ADS/Sa.CXO\#obs/00525] {525} & 03:05:31.614 & +17:10:02.06 & 10.0 & VF & I3 & 0.424 &  0.00\\
MS 0440.5+0204 $\dagger$ & \dataset [ADS/Sa.CXO\#obs/04196] {4196} & 04:43:09.952 & +02:10:18.70 & 59.4 & VF & S3 & 0.190 &  2.17\\
MS 0451.6-0305 & \dataset [ADS/Sa.CXO\#obs/00902] {902} & 04:54:11.004 & -03:00:52.19 & 44.2 &  F & S3 & 0.539 & 33.32\\
MS 0735.6+7421 & \dataset [ADS/Sa.CXO\#obs/04197] {4197} & 07:41:44.245 & +74:14:38.23 & 45.5 & VF & S3 & 0.216 &  7.57\\
MS 0839.8+2938 & \dataset [ADS/Sa.CXO\#obs/02224] {2224} & 08:42:55.969 & +29:27:26.97 & 29.8 &  F & S3 & 0.194 &  3.10\\
MS 0906.5+1110 & \dataset [ADS/Sa.CXO\#obs/00924] {924} & 09:09:12.753 & +10:58:32.00 & 29.7 & VF & I3 & 0.163 &  4.64\\
MS 1006.0+1202 & \dataset [ADS/Sa.CXO\#obs/00925] {925} & 10:08:47.194 & +11:47:55.99 & 29.4 & VF & I3 & 0.221 &  4.75\\
MS 1008.1-1224 & \dataset [ADS/Sa.CXO\#obs/00926] {926} & 10:10:32.312 & -12:39:56.80 & 44.2 & VF & I3 & 0.301 &  6.44\\
MS 1054.5-0321 & \dataset [ADS/Sa.CXO\#obs/00512] {512} & 10:56:58.499 & -03:37:32.76 & 89.1 &  F & S3 & 0.830 & 27.22\\
MS 1455.0+2232 & \dataset [ADS/Sa.CXO\#obs/04192] {4192} & 14:57:15.088 & +22:20:32.49 & 91.9 & VF & I3 & 0.259 & 10.25\\
MS 1621.5+2640 & \dataset [ADS/Sa.CXO\#obs/00546] {546} & 16:23:35.522 & +26:34:25.67 & 30.1 &  F & I3 & 0.426 &  6.49\\
MS 2053.7-0449 & \dataset [ADS/Sa.CXO\#obs/01667] {1667} & 20:56:21.295 & -04:37:46.81 & 44.5 & VF & I3 & 0.583 &  2.96\\
MS 2053.7-0449 & \dataset [ADS/Sa.CXO\#obs/00551] {551} & 20:56:21.297 & -04:37:46.80 & 44.3 &  F & I3 & 0.583 &  2.96\\
MS 2137.3-2353 & \dataset [ADS/Sa.CXO\#obs/04974] {4974} & 21:40:15.178 & -23:39:40.71 & 57.4 & VF & S3 & 0.313 & 11.28\\
MS J1157.3+5531 $\dagger$ & \dataset [ADS/Sa.CXO\#obs/04964] {4964} & 11:59:52.295 & +55:32:05.61 & 75.1 & VF & S3 & 0.081 &  0.12\\
NGC 6338 $\dagger$ & \dataset [ADS/Sa.CXO\#obs/04194] {4194} & 17:15:23.036 & +57:24:40.29 & 47.3 & VF & I3 & 0.028 &  0.13\\
PKS 0745-191 & \dataset [ADS/Sa.CXO\#obs/06103] {6103} & 07:47:31.469 & -19:17:40.01 & 10.3 & VF & I3 & 0.103 & 18.41\\
RBS 0797 & \dataset [ADS/Sa.CXO\#obs/02202] {2202} & 09:47:12.971 & +76:23:13.90 & 11.7 & VF & I3 & 0.354 & 26.07\\
RDCS 1252-29    & \dataset [ADS/Sa.CXO\#obs/04198] {4198} & 12:52:54.221 & -29:27:21.01 & 163.4 & VF & I3 & 1.237 &  2.28\\
RX J0232.2-4420 & \dataset [ADS/Sa.CXO\#obs/04993] {4993} & 02:32:18.771 & -44:20:46.68 & 23.4 & VF & I3 & 0.284 & 18.17\\
RX J0340-4542   & \dataset [ADS/Sa.CXO\#obs/06954] {6954} & 03:40:44.765 & -45:41:18.41 & 17.9 & VF & I3 & 0.082 &  0.33\\
RX J0439+0520   & \dataset [ADS/Sa.CXO\#obs/00527] {527} & 04:39:02.218 & +05:20:43.11 & 9.6 & VF & I3 & 0.208 &  3.57\\
RX J0439.0+0715 & \dataset [ADS/Sa.CXO\#obs/01449] {1449} & 04:39:00.710 & +07:16:07.65 & 6.3 &  F & I3 & 0.230 &  9.44\\
RX J0439.0+0715 & \dataset [ADS/Sa.CXO\#obs/03583] {3583} & 04:39:00.710 & +07:16:07.63 & 19.2 & VF & I3 & 0.230 &  9.44\\
RX J0528.9-3927 & \dataset [ADS/Sa.CXO\#obs/04994] {4994} & 05:28:53.039 & -39:28:15.53 & 22.5 & VF & I3 & 0.263 & 12.99\\
RX J0647.7+7015 & \dataset [ADS/Sa.CXO\#obs/03196] {3196} & 06:47:50.029 & +70:14:49.66 & 19.3 & VF & I3 & 0.584 & 26.48\\
RX J0647.7+7015 & \dataset [ADS/Sa.CXO\#obs/03584] {3584} & 06:47:50.014 & +70:14:49.69 & 20.0 & VF & I3 & 0.584 & 26.48\\
RX J0819.6+6336 $\dagger$ & \dataset [ADS/Sa.CXO\#obs/02199] {2199} & 08:19:26.007 & +63:37:26.53 & 14.9 &  F & S3 & 0.119 &  0.98\\
RX J0910+5422   & \dataset [ADS/Sa.CXO\#obs/02452] {2452} & 09:10:44.478 & +54:22:03.77 & 65.3 & VF & I3 & 1.100 &  1.33\\
RX J1053+5735   & \dataset [ADS/Sa.CXO\#obs/04936] {4936} & 10:53:39.844 & +57:35:18.42 & 92.2 &  F & S3 & 1.140 &  0.00\\
RX J1347.5-1145 & \dataset [ADS/Sa.CXO\#obs/03592] {3592} & 13:47:30.593 & -11:45:10.05 & 57.7 & VF & I3 & 0.451 & 100.36\\
RX J1347.5-1145 & \dataset [ADS/Sa.CXO\#obs/00507] {507} & 13:47:30.598 & -11:45:10.27 & 10.0 &  F & S3 & 0.451 & 100.36\\
RX J1350+6007   & \dataset [ADS/Sa.CXO\#obs/02229] {2229} & 13:50:48.038 & +60:07:08.39 & 58.3 & VF & I3 & 0.804 &  2.19\\
RX J1423.8+2404 & \dataset [ADS/Sa.CXO\#obs/01657] {1657} & 14:23:47.759 & +24:04:40.45 & 18.5 & VF & I3 & 0.545 & 15.84\\
RX J1423.8+2404 & \dataset [ADS/Sa.CXO\#obs/04195] {4195} & 14:23:47.763 & +24:04:40.63 & 115.6 & VF & S3 & 0.545 & 15.84\\
RX J1504.1-0248 & \dataset [ADS/Sa.CXO\#obs/05793] {5793} & 15:04:07.415 & -02:48:15.70 & 39.2 & VF & I3 & 0.215 & 34.64\\
RX J1525+0958   & \dataset [ADS/Sa.CXO\#obs/01664] {1664} & 15:24:39.729 & +09:57:44.42 & 50.9 & VF & I3 & 0.516 &  3.29\\
RX J1532.9+3021 & \dataset [ADS/Sa.CXO\#obs/01649] {1649} & 15:32:55.642 & +30:18:57.69 & 9.4 & VF & S3 & 0.345 & 20.77\\
RX J1532.9+3021 & \dataset [ADS/Sa.CXO\#obs/01665] {1665} & 15:32:55.641 & +30:18:57.31 & 10.0 & VF & I3 & 0.345 & 20.77\\
RX J1716.9+6708 & \dataset [ADS/Sa.CXO\#obs/00548] {548} & 17:16:49.015 & +67:08:25.80 & 51.7 &  F & I3 & 0.810 &  8.04\\
RX J1720.1+2638 & \dataset [ADS/Sa.CXO\#obs/04361] {4361} & 17:20:09.941 & +26:37:29.11 & 25.7 & VF & I3 & 0.164 & 11.39\\
RX J1720.2+3536 & \dataset [ADS/Sa.CXO\#obs/03280] {3280} & 17:20:16.953 & +35:36:23.63 & 20.8 & VF & I3 & 0.391 & 13.02\\
RX J1720.2+3536 & \dataset [ADS/Sa.CXO\#obs/06107] {6107} & 17:20:16.949 & +35:36:23.68 & 33.9 & VF & I3 & 0.391 & 13.02\\
RX J1720.2+3536 & \dataset [ADS/Sa.CXO\#obs/07225] {7225} & 17:20:16.947 & +35:36:23.69 & 2.0 & VF & I3 & 0.391 & 13.02\\
RX J2011.3-5725 & \dataset [ADS/Sa.CXO\#obs/04995] {4995} & 20:11:26.889 & -57:25:09.08 & 24.0 & VF & I3 & 0.279 &  2.77\\
RX J2129.6+0005 & \dataset [ADS/Sa.CXO\#obs/00552] {552} & 21:29:39.944 & +00:05:18.83 & 10.0 & VF & I3 & 0.235 & 12.56\\
S0463 & \dataset [ADS/Sa.CXO\#obs/06956] {6956} & 04:29:07.040 & -53:49:38.02 & 29.3 & VF & I3 & 0.099 & 22.19\\
S0463 & \dataset [ADS/Sa.CXO\#obs/07250] {7250} & 04:29:07.063 & -53:49:38.11 & 29.1 & VF & I3 & 0.099 & 22.19\\
TRIANG AUSTR $\dagger$ & \dataset [ADS/Sa.CXO\#obs/01281] {1281} & 16:38:22.712 & -64:21:19.70 & 11.4 &  F & I3 & 0.051 &  9.41\\
V 1121.0+2327 & \dataset [ADS/Sa.CXO\#obs/01660] {1660} & 11:20:57.195 & +23:26:27.60 & 71.3 & VF & I3 & 0.560 &  3.28\\
ZWCL 1215 & \dataset [ADS/Sa.CXO\#obs/04184] {4184} & 12:17:40.787 & +03:39:39.42 & 12.1 & VF & I3 & 0.075 &  3.49\\
ZWCL 1358+6245 & \dataset [ADS/Sa.CXO\#obs/00516] {516} & 13:59:50.526 & +62:31:04.57 & 54.1 &  F & S3 & 0.328 & 12.42\\
ZWCL 1953 & \dataset [ADS/Sa.CXO\#obs/01659] {1659} & 08:50:06.677 & +36:04:16.16 & 24.9 &  F & I3 & 0.380 & 17.11\\
ZWCL 3146 & \dataset [ADS/Sa.CXO\#obs/00909] {909} & 10:23:39.735 & +04:11:08.05 & 46.0 &  F & I3 & 0.290 & 29.59\\
ZWCL 5247 & \dataset [ADS/Sa.CXO\#obs/00539] {539} & 12:34:21.928 & +09:47:02.83 & 9.3 & VF & I3 & 0.229 &  4.87\\
ZWCL 7160 & \dataset [ADS/Sa.CXO\#obs/00543] {543} & 14:57:15.158 & +22:20:33.85 & 9.9 &  F & I3 & 0.258 & 10.14\\
ZWICKY 2701 & \dataset [ADS/Sa.CXO\#obs/03195] {3195} & 09:52:49.183 & +51:53:05.27 & 26.9 & VF & S3 & 0.210 &  5.19\\
ZwCL 1332.8+5043 & \dataset [ADS/Sa.CXO\#obs/05772] {5772} & 13:34:20.698 & +50:31:04.64 & 19.5 & VF & I3 & 0.620 &  4.46\\
ZwCl 0848.5+3341 & \dataset [ADS/Sa.CXO\#obs/04205] {4205} & 08:51:38.873 & +33:31:08.00 & 11.4 & VF & S3 & 0.371 &  4.58
\enddata
\tablecomments{(1) Cluster name, (2) CDA observation identification number, (3) R.A. of cluster center, (4) Dec. of cluster center, (5) nominal exposure time, (6) observing mode, (7) CCD location of centroid, (8) redshift, (9) NRAO absorbing Galactic neutral hydrogen column density, (10) bolometric luminosity. $\dagger$ indicates clusters analyzed within R$_{5000}$ only.}
\end{deluxetable}

\clearpage
\begin{deluxetable}{lcccccccc}
\tablewidth{0pt}
\tabletypesize{\scriptsize}
\tablecaption{Summary of $\beta$-Model Fits\label{tab:betafits}}
\tablehead{\colhead{Cluster} & \colhead{$S_{01}$} & \colhead{$r_{c1}$} & \colhead{$\beta_{1}$} & \colhead{$S_{02}$} & \colhead{$r_{c2}$} & \colhead{$\beta_{2}$} & \colhead{D.O.F.} & \colhead{$\chi_{\mathrm{red}}^2$}\\
\colhead{} & \colhead{10$^{-6}$ cts s$^{-1}$ arcsec$^{2}$} & \colhead{\arcsec} & \colhead{} & \colhead{10$^{-6}$ cts s$^{-1}$ arcsec$^{2}$} & \colhead{\arcsec} & \colhead{} & \colhead{} & \colhead{}\\
\colhead{{(1)}} & \colhead{{(2)}} & \colhead{{(3)}} & \colhead{{(4)}} & \colhead{{(5)}} & \colhead{{(6)}} & \colhead{{(7)}} & \colhead{{(8)}} & \colhead{{(9)}}
}
\startdata
Abell 119 &  4.93 $\pm$  0.73 &  39.1 $\pm$  15.3 &  0.34 $\pm$  0.07 &  3.52 $\pm$  0.96 & 735.2 $\pm$ 479.4 &  1.27 $\pm$  1.27 &    52 &  1.76\\
Abell 160 &  2.32 $\pm$  0.27 &  53.4 $\pm$  11.1 &  0.57 $\pm$  0.12 &  1.29 $\pm$  0.22 & 284.0 $\pm$  52.2 &  0.74 $\pm$  0.10 &    90 &  1.18\\
Abell 193 & 24.72 $\pm$  1.62 &  80.8 $\pm$   2.2 &  0.43 $\pm$  0.01 & \nodata & \nodata & \nodata &    38 &  0.43\\
Abell 400 &  4.66 $\pm$  0.09 & 151.3 $\pm$   6.4 &  0.42 $\pm$  0.01 & \nodata & \nodata & \nodata &    96 &  0.57\\
Abell 1060 & 21.95 $\pm$  0.44 &  93.5 $\pm$   8.1 &  0.35 $\pm$  0.01 & \nodata & \nodata & \nodata &    42 &  1.44\\
Abell 1240 &  1.58 $\pm$  0.07 & 247.9 $\pm$  46.9 &  1.01 $\pm$  0.22 & \nodata & \nodata & \nodata &    58 &  1.58\\
Abell 1736 &  3.81 $\pm$  0.56 &  55.6 $\pm$  16.1 &  0.42 $\pm$  0.12 &  2.49 $\pm$  0.47 & 1470.0 $\pm$  87.2 &  5.00 $\pm$  0.73 &    35 &  1.58\\
Abell 2125 &  3.50 $\pm$  0.20 &  26.0 $\pm$   4.9 &  0.49 $\pm$  0.05 &  1.02 $\pm$  0.13 & 159.9 $\pm$   9.2 &  1.32 $\pm$  0.16 &    35 &  0.33\\
Abell 2255 &  8.38 $\pm$  0.15 & 222.7 $\pm$   9.8 &  0.62 $\pm$  0.02 & \nodata & \nodata & \nodata &    94 &  1.45\\
Abell 2256 & 21.69 $\pm$  0.19 & 407.8 $\pm$  17.9 &  0.99 $\pm$  0.05 & \nodata & \nodata & \nodata &    88 &  0.83\\
Abell 2319 & 47.39 $\pm$  0.61 & 128.8 $\pm$   3.1 &  0.49 $\pm$  0.01 & \nodata & \nodata & \nodata &    92 &  1.67\\
Abell 2462 &  8.19 $\pm$  1.43 &  60.8 $\pm$   9.6 &  0.64 $\pm$  0.11 &  1.87 $\pm$  0.25 & 762.7 $\pm$  39.1 &  5.00 $\pm$  0.87 &    67 &  1.54\\
Abell 2631 & 20.55 $\pm$  1.01 &  66.0 $\pm$   4.0 &  0.73 $\pm$  0.03 & \nodata & \nodata & \nodata &    58 &  1.15\\
Abell 3376 &  4.21 $\pm$  0.09 & 125.5 $\pm$   5.6 &  0.40 $\pm$  0.01 & \nodata & \nodata & \nodata &    98 &  1.42\\
Abell 3391 & 10.65 $\pm$  0.31 & 132.3 $\pm$   7.9 &  0.48 $\pm$  0.01 & \nodata & \nodata & \nodata &    84 &  1.86\\
Abell 3395 &  6.85 $\pm$  0.67 &  90.9 $\pm$   6.7 &  0.49 $\pm$  0.03 & \nodata & \nodata & \nodata &    38 &  0.96\\
MKW 8 &  7.71 $\pm$  0.62 &  25.2 $\pm$   2.5 &  0.32 $\pm$  0.01 &  1.51 $\pm$  0.08 & 1124.0 $\pm$  64.1 &  5.00 $\pm$  0.40 &    88 &  0.65\\
RBS 461 & 12.84 $\pm$  0.34 & 102.2 $\pm$   4.1 &  0.52 $\pm$  0.01 & \nodata & \nodata & \nodata &    84 &  1.56\\
\enddata
\tablecomments{Col. (1) Cluster name; col. (2) central surface brightness of first component; col. (3) core radius of first component; col. (4) $\beta$ parameter of first component; col. (5) central surface brightness of second component; col. (6) core radius of second component; col. (7) $\beta$ parameter of second component; col. (8) model degrees of freedom; and col. (9) reduced chi-squared statistic for best-fit model.}
\end{deluxetable}

\clearpage
\begin{deluxetable}{lcccc}
\tabletypesize{\scriptsize}
\tablecaption{M. Donahue's \halpha\ Observations.\label{tab:newha}}
\tablewidth{0pt}
\tablehead{
  \colhead{Cluster} & \colhead{Telescope} & \colhead{$z$} & \colhead{$[NII]$/\halpha} & \colhead{\halpha\ Flux}\\
  \colhead{}      & \colhead{}        & \colhead{}  & \colhead{}               & \colhead{$10^{-15}$ \flux}
}
\startdata
Abell 85     & PO & 0.0558 & 2.67    &    0.581\\
Abell 119    & LC & 0.0442 & \nodata & $<$0.036\\
Abell 133    & LC & 0.0558 & \nodata &    0.88\\
Abell 496    & LC & 0.0328 & 2.50    &    2.90\\
Abell 1644   & LC & 0.0471 & \nodata &    1.00\\
Abell 1650   & LC & 0.0843 & \nodata & $<$0.029\\
Abell 1689   & LC & 0.1843 & \nodata & $<$0.029\\
Abell 1736   & LC & 0.0338 & \nodata & $<$0.026\\
Abell 2597   & PO & 0.0854 & 0.85    &    29.7\\
Abell 3112   & LC & 0.0720 & 2.22    &    2.66\\
Abell 3158   & LC & 0.0586 & \nodata & $<$0.036\\
Abell 3266   & LC & 0.0590 & 1.62    & $<$0.027\\
Abell 4059   & LC & 0.0475 & 3.60    &    2.22\\
Cygnus A     & PO & 0.0561 & 1.85    &    28.4\\
EXO 0422-086 & LC & 0.0397 & \nodata & $<$0.031\\
Hydra A      & LC & 0.0522 & 0.85    &    13.4\\
PKS 0745-191 & LC & 0.1028 & 1.02    &    10.4
\enddata
\tablecomments{The abbreviation ``PO'' denotes observations taken on
  the 5 m Hale Telescope at the Palomar Observatory, USA, while ``LC''
  are observations taken on the DuPont 2.5 m telescope at the Las
  Campanas Observatory, Chile. Upperlimits for \halpha\ fluxes are
  $3\sigma$.}
\end{deluxetable}

\clearpage
\clearpage
\begin{figure}[htp]
  \begin{center}
    \begin{minipage}[htp]{0.9\linewidth}
      \includegraphics*[width=\textwidth, trim=15mm 10mm 10mm 10mm, clip]{beta.eps}
      \caption{Surface brightness profiles for clusters requiring a
        $\beta$-model fit for deprojection (discussed in
        \S\ref{sec:beta}). The best-fit $\beta$-model for each cluster
        is overplotted as a dashed line. The discrepancy between the
        data and best-fit model for some clusters results from the
        presence of a compact X-ray source at the center of the
        cluster. These cases are discussed in Appendix
        \ref{app:beta}.}
      \label{fig:betamods}
    \end{minipage}
  \end{center}
\end{figure}
\clearpage
\begin{figure}[htp]
  \begin{center}
    \begin{minipage}[htp]{0.9\linewidth}
      \includegraphics*[width=\textwidth, trim=5mm 0mm 5mm 5mm, clip]{itplflat_rat.eps}
      \caption{Ratio of best-fit \kna\ for the two treatments of
        central temperature interpolation (see \S\ref{sec:temppr}):
        (1) temperature is free to decline across the central density
        bins ($\Delta T_{center} \ne 0$), and (2) the temperature
        across the central density bins is isothermal ($\Delta
        T_{center} = 0$). Filled black squares are clusters for which
        the \kna\ ratio is inconsistent with unity.}
      \label{fig:kcomp}
    \end{minipage}
  \end{center}
\end{figure}
\clearpage
\begin{figure}[htp]
  \begin{center}
    \begin{minipage}[htp]{0.9\linewidth}
      \includegraphics*[width=\textwidth, trim=5mm 0mm 5mm 5mm, clip]{k0res.eps}
      \caption{Best-fit \kna\ vs. redshift. Some clusters have
        \kna\ error bars smaller than the point. The clusters with
        upper-limits ({\it{black points with downward arrows}}) are:
        A2151, AS0405, MS 0116.3-0115, and RX J1347.5-1145. The
        numerically labeled clusters are: (1) M87, (2) Centaurus
        Cluster, (3) RBS 533, (4) HCG 42, (5) HCG 62, (6) SS2B153, (7)
        A1991, (8) MACS0744.8+3927, and (9) CL J1226.9+3332. For
        CLJ1226, \cite{2007ApJ...659.1125M} found best-fit $\kna = 132
        \pm 24 \ent$ which is not significantly different from our
        value of $\kna = 166 \pm 45 \ent$. The lack of $\kna < 10
        \ent$ clusters at $z > 0.1$ is most likely the result of
        insufficient angular resolution (see \S\ref{sec:angres}).}
      \label{fig:k0res}
    \end{minipage}
  \end{center}
\end{figure}
\clearpage
\begin{center}
  \begin{figure}[htp]
    \begin{minipage}[htp]{0.5\linewidth}
      \includegraphics*[width=\textwidth, trim=28mm 7mm 30mm 17mm, clip]{curvk0.eps}
    \end{minipage}
    \begin{minipage}[htp]{0.5\linewidth}
      \includegraphics*[width=\textwidth, trim=28mm 7mm 30mm 17mm, clip]{nbins_k0.eps}
    \end{minipage}
    \begin{minipage}[htp]{0.5\linewidth}
      \includegraphics*[width=\textwidth, trim=28mm 7mm 30mm 17mm, clip]{texpk0.eps}
    \end{minipage}
    \begin{minipage}[htp]{0.5\linewidth}
      \includegraphics*[width=\textwidth, trim=28mm 7mm 30mm 17mm, clip]{ntxbins_k0.eps}
    \end{minipage}
    \caption{Plots of possible systematics versus best-fit \kna.
      {\it{Top left:}} Best-fit \kna\ plotted versus average curvature
      of the corresponding entropy profile (see eq. \ref{eqn:avgcurv})
      There is no trend between these two quantities suggesting that
      \kna\ is not heavily influenced by the total shape of the
      entropy profile. {\it{Top right:}} Best-fit \kna\ plotted versus
      number of bins in the entropy profile which were used during
      fitting. Again, no trend is found. {\it{Bottom left:}} Best-fit
      \kna\ plotted versus the total used exposure time for each
      cluster. No trend is found. {\it{Bottom right:}} Best-fit
      \kna\ plotted versus the number of bins in the temperature
      profile for each cluster. As expected, fewer $\Tx(r)$ does not
      correlate with \kna.}
    \label{fig:sys}
  \end{figure}
\end{center}
\clearpage
\begin{center}
  \begin{figure}[htp]
    \begin{minipage}[htp]{0.5\linewidth}
      \includegraphics*[width=\textwidth, trim=28mm 7mm 30mm 17mm, clip]{splots_allt.eps}
    \end{minipage}
    \begin{minipage}[htp]{0.5\linewidth}
      \includegraphics*[width=\textwidth, trim=28mm 7mm 30mm 17mm, clip]{splots_tle4.eps}
    \end{minipage}
    \begin{minipage}[htp]{0.5\linewidth}
      \includegraphics*[width=\textwidth, trim=28mm 7mm 30mm 17mm, clip]{splots_gt4tle8.eps}
    \end{minipage}
    \begin{minipage}[htp]{0.5\linewidth}
      \includegraphics*[width=\textwidth, trim=28mm 7mm 30mm 17mm, clip]{splots_tgt8.eps}
    \end{minipage}
    \caption{Composite plots of entropy profiles for varying cluster
      temperature ranges. Profiles are color-coded based on average
      cluster temperature. Units of the color bars are keV. The solid
      line is the pure-cooling model of \cite{voitbryan}, the dashed
      line is the mean profile for clusters with $\kna \le 50 \ent$,
      and the dashed-dotted line is the mean profile for clusters with
      $\kna > 50 \ent$. {\it{Top left:}} This panel contains all the
      entropy profiles in our study. {\it{Top right:}} Clusters with
      $kT_X < 4$ keV. {\it{Bottom left:}} Clusters with $4\keV < kT_X
      < 8\keV$. {\it{Bottom right:}} Clusters with $kT_X > 8$
      keV. Note that while the dispersion of core entropy for each
      temperature range is large, as the $kT_X$ range increases so to
      does the mean core entropy.}
    \label{fig:splots}
  \end{figure}
\end{center}
\clearpage
\begin{figure}[htp]
  \begin{center}
    \begin{minipage}[htp]{0.9\linewidth}
      \includegraphics*[width=\textwidth, trim=20mm 10mm 10mm 10mm, clip]{k0hist.eps}
      \caption{{\it{Top panel:}} Histogram of best-fit \kna\ for all
        the clusters in \accept. Bin widths are 0.15 in log space.
        {\it{Bottom panel:}} Cumulative distribution of \kna\ values
        for the full sample. The distinct bimodality in \kna\ is
        present in both distributions, which would not be seen if it
        were an artifact of the histogram binning. A KMM test finds
        the \kna\ distribution cannot arise from a simple unimodal
        Gaussian.}
      \label{fig:k0hist}
    \end{minipage}
  \end{center}
\end{figure}
\clearpage
\begin{figure}[htp]
  \begin{center}
    \begin{minipage}[htp]{0.9\linewidth}
      \includegraphics*[width=\textwidth, trim=20mm 10mm 10mm 10mm, clip]{hifl_k0hist.eps}
      \caption{{\it{Top panel:}} Histogram of best-fit \kna\ values
        for the primary \hifl\ sample. Bin widths are 0.15 in log
        space.  {\it{Bottom panel:}} Cumulative distribution of
        best-fit \kna\ values. The distinct bimodality seen in the
        full \accept\ sample (Fig. \ref{fig:k0hist}) is also present
        in the \hifl\ subsample and shares the same gap between the
        low-entropy peak at 10-20 \ent\ and the high-entropy peak at
        100-200 \ent. That bimodality is present in both samples is
        strong evidence it is not a result of an unknown archival
        bias.}
      \label{fig:hiflk0}
    \end{minipage}
  \end{center}
\end{figure}
\clearpage
\begin{figure}[htp]
  \begin{center}
    \begin{minipage}[htp]{0.8\linewidth}
      \includegraphics*[width=\textwidth, trim=20mm 10mm 10mm 10mm, clip]{t0.eps}
    \end{minipage}
    \begin{minipage}[htp]{0.8\linewidth}
      \includegraphics*[width=\textwidth, trim=20mm 10mm 10mm 10mm, clip]{k0cool.eps}
    \end{minipage}
    \caption{{\it{Top panel:}} Log-binned histogram and cumulative
      distribution of best-fit core cooling times, $t_{c0}$
      (eqn. \ref{eqn:tc0}), for all the clusters in \accept. Histogram
      bin widths are 0.2 in log space. {\it{Bottom panel:}} Log-binned
      histogram and cumulative distribution of core cooling times
      calculated from best-fit \kna\ values, $t_{c0}(\kna)$
      (eqn. \ref{eqn:tck0}), for all the clusters in
      \accept. Histogram bin widths are 0.2 in log space. The
      bimodality we observe in the \kna\ distribution is also present
      in best-fit $t_{c0}$. However, the gaps between the two
      populations of $t_{c0}$ and $t_{c0}(\kna)$ differ by $\sim 0.3$
      Gyrs which may be an artifact of the binning.}
    \label{fig:t0}
  \end{center}
\end{figure}



%%%%%%%%%%%%%%%%%%%%
% End the document %
%%%%%%%%%%%%%%%%%%%%
\end{document}
