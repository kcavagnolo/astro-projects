\clearpage
\begin{figure}[htp]
  \begin{center}
    \begin{minipage}[htp]{0.9\linewidth}
      \includegraphics*[width=\textwidth, trim=15mm 10mm 10mm 10mm, clip]{beta.eps}
      \caption{Surface brightness profiles for clusters requiring a
        $\beta$-model fit for deprojection (discussed in \S\ref{sec:beta}). The best-fit
        $\beta$-model for each cluster is overplotted as a dashed
        line. The discrepancy between the data and best-fit model for
        some clusters results from the presence of a compact X-ray
        source at the center of the cluster. These cases are discussed
        in Appendix \ref{app:beta}.}
      \label{fig:betamods}
    \end{minipage}
  \end{center}
\end{figure}

\clearpage
\begin{figure}[htp]
  \begin{center}
    \begin{minipage}[htp]{0.9\linewidth}
      \includegraphics*[width=\textwidth, trim=5mm 0mm 5mm 5mm, clip]{itplflat_rat.eps}
      \caption{Ratio of best-fit \kna\ for the two treatments of
        central temperature interpolation (see \S\ref{sec:temppr}): (1)
        temperature is free to decline across the central density bins
        ($\Delta T_{center} \ne 0$), and (2) the temperature across the
        central density bins is isothermal ($\Delta T_{center} =
        0$). Filled black squares are clusters for which the \kna\ ratio is
        inconsistent with unity.}
      \label{fig:kcomp}
    \end{minipage}
  \end{center}
\end{figure}

%% \clearpage
%% \begin{figure}[htp]
%%   \begin{center}
%%     \begin{minipage}[htp]{0.9\linewidth}
%%       \includegraphics*[width=\textwidth, trim=0mm 0mm 0mm 0mm, clip]{example.eps}
%%       \caption{}
%%       \label{fig:example}
%%     \end{minipage}
%%   \end{center}
%% \end{figure}

\clearpage
\begin{figure}[htp]
  \begin{center}
    \begin{minipage}[htp]{0.9\linewidth}
      \includegraphics*[width=\textwidth, trim=5mm 0mm 5mm 5mm, clip]{k0res.eps}
      \caption{Best-fit \kna\ vs. redshift. Some clusters have
        \kna\ error bars smaller than the point. The clusters with
        upper-limits ({\it{black points with downward arrows}}) are: A2151,
        AS0405, MS 0116.3-0115, and RX J1347.5-1145. The numerically
        labeled clusters are: (1) M87, (2) Centaurus Cluster, (3) RBS
        533, (4) HCG 42, (5) HCG 62, (6) SS2B153, (7) A1991, (8)
        MACS0744.8+3927, and (9) CL J1226.9+3332. For CLJ1226,
        \cite{2007ApJ...659.1125M} found best-fit $\kna = 132 \pm 24
        \ent$ which is not significantly different from our value of
        $\kna = 166 \pm 45 \ent$. The lack of $\kna < 10 \ent$ clusters
        at $z > 0.1$ is most likely the result of insufficient angular
        resolution (see \S\ref{sec:angres}).}
      \label{fig:k0res}
    \end{minipage}
  \end{center}
\end{figure}


%% \clearpage
%% \begin{figure}[htp]
%%   \begin{center}
%%     \begin{minipage}[htp]{0.9\linewidth}
%%       \includegraphics*[width=\textwidth, trim=28mm 7mm 40mm 17mm, clip]{curvk0.eps}
%%       \caption{Plot of \kna\ vs. total curvature. Clusters
%%         with \kna\ values consistent with zero are plotted with using
%%         the $2-\sigma$ upper-limit and downward arrows. The lack of a
%%         trend in average curvature with \kna\ suggests the \kna\ values
%%         we find are more sensitive to core entropy than to shape of a
%%         profile. The cluster with the smallest curvature is A370, while
%%         the largest curvature is Centaurus.}
%%       \label{fig:curve}
%%     \end{minipage}
%%   \end{center}
%% \end{figure}

%% \clearpage
%% \begin{figure}[htp]
%%   \begin{center}
%%     \begin{minipage}[htp]{0.9\linewidth}
%%       \includegraphics*[width=\textwidth, trim=28mm 7mm 40mm 17mm, clip]{nbins_k0.eps}
%%       \caption{Plot of \kna\ vs. number of bins fit in the
%%         entropy profile. The lack of a trend between $N_{bins}$ and
%%         \kna\ again suggests that our best-fit \kna\ values properly
%%         represent the core entropy and not the radial resolution of the
%%         profile. The cluster with the fewest and most bins are [X] and
%%         [Y], respectively.}
%%       \label{fig:nbins}
%%     \end{minipage}
%%   \end{center}
%% \end{figure}

\clearpage
\begin{center}
  \begin{figure}[htp]
    \begin{minipage}[htp]{0.5\linewidth}
      \includegraphics*[width=\textwidth, trim=28mm 7mm 30mm 17mm, clip]{splots_allt.eps}
    \end{minipage}
    \begin{minipage}[htp]{0.5\linewidth}
      \includegraphics*[width=\textwidth, trim=28mm 7mm 30mm 17mm, clip]{splots_tle4.eps}
    \end{minipage}
    \begin{minipage}[htp]{0.5\linewidth}
      \includegraphics*[width=\textwidth, trim=28mm 7mm 30mm 17mm, clip]{splots_gt4tle8.eps}
    \end{minipage}
    \begin{minipage}[htp]{0.5\linewidth}
      \includegraphics*[width=\textwidth, trim=28mm 7mm 30mm 17mm, clip]{splots_tgt8.eps}
    \end{minipage}
    \caption{Composite plots of entropy profiles for varying cluster
    temperature ranges. Profiles are color-coded based on average
    cluster temperature. Units of the color bars are keV. The
    solid line is the pure-cooling model of \cite{voitbryan}, the
    dashed line is the mean profile for clusters with $\kna \le 50
    \ent$, and the dashed-dotted line is the mean profile for clusters
    with $\kna > 50 \ent$. {\it{Top left:}} This panel contains all
    the entropy profiles in our study. {\it{Top right:}} Clusters with
    $kT_X < 4$ keV. {\it{Bottom left:}} Clusters with $4\keV < kT_X <
    8\keV$. {\it{Bottom right:}} Clusters with $kT_X > 8$ keV. Note that
    while the dispersion of core entropy for each temperature range is
    large, as the $kT_X$ range increases so to does the mean core
    entropy.}
    \label{fig:splots}
  \end{figure}
\end{center}

\clearpage
\begin{figure}[htp]
  \begin{center}
    \begin{minipage}[htp]{0.9\linewidth}
      \includegraphics*[width=\textwidth, trim=20mm 10mm 10mm 10mm, clip]{k0hist.eps}
      \caption{{\it{Top panel:}} Histogram of best-fit \kna\ for all
      the clusters in \accept. Bin widths are 0.15 in log space.
      {\it{Bottom panel:}} Cumulative distribution of \kna\ values for
      the full sample. The distinct bimodality in \kna\ is present in
      both distributions, which would not be seen if it were an
      artifact of the histogram binning. A KMM test finds the \kna\
      distribution cannot arise from a simple unimodal Gaussian.}
      \label{fig:k0hist}
    \end{minipage}
  \end{center}
\end{figure}

%% \clearpage
%% \begin{figure}[htp]
%%   \begin{center}
%%     \begin{minipage}[htp]{0.9\linewidth}
%%       \includegraphics*[width=\textwidth, trim=28mm 7mm 30mm 17mm, clip]{kmm.eps}
%%       \caption{Histogram of the log-space \kna\ distribution
%%         overplotted with the best-fit Gaussian populations determined
%%         from the KMM test. The solid blue curve is the composite
%%         population found when neglecting clusters with $\kna \le 4
%%         \ent$, while the dashed red curve is the best-fit composite
%%         population including those clusters. The probability that this
%%         population arises from a unimodal distribution is exceedingly
%%         small.}
%%       \label{fig:kmm}
%%     \end{minipage}
%%   \end{center}
%% \end{figure}

\clearpage
\begin{figure}[htp]
  \begin{center}
    \begin{minipage}[htp]{0.9\linewidth}
      \includegraphics*[width=\textwidth, trim=20mm 10mm 10mm 10mm, clip]{hifl_k0hist.eps}
      \caption{{\it{Top panel:}} Histogram of best-fit \kna\ values
      for the primary \hifl\ sample. Bin widths are 0.15 in log space.
      {\it{Bottom panel:}} Cumulative distribution of best-fit \kna\
      values. The distinct bimodality seen in the full \accept\ sample
      (Fig. \ref{fig:k0hist}) is also present in the \hifl\ subsample
      and shares the same gap between the low-entropy peak at 10-20 \ent\ and the high-entropy peak at 100-200 \ent. That
      bimodality is present in both samples is strong evidence it is
      not a result of an unknown archival bias.}
      \label{fig:hiflk0}
    \end{minipage}
  \end{center}
\end{figure}

%% \clearpage
%% \begin{figure}[htp]
%%   \begin{center}
%%     \begin{minipage}[htp]{0.9\linewidth}
%%       \includegraphics*[width=\textwidth, trim=28mm 7mm 30mm 17mm, clip]{hifl_kmm.eps}
%%       \caption{Histogram of the log-space \kna\ distribution
%%         overplotted with the best-fit Gaussian populations determined
%%         from the KMM test for the \hifl\ subsample. The solid blue curve
%%         is the composite population found when neglecting clusters with
%%         $\kna \le 4 \ent$, while the dashed red curve is the best-fit
%%         composite population including those clusters. As in the full
%%         \accept\ sample, the KMM test once again finds two distinct
%%         \kna\ populations. The probability that this population arises
%%         from a unimodal distribution is exceedingly small.}
%%       \label{fig:hiflkmm}
%%     \end{minipage}
%%   \end{center}
%% \end{figure}

%% \clearpage
%% \begin{center}
%%   \begin{figure}[htp]
%%     \begin{minipage}[htp]{0.5\linewidth}
%%       \includegraphics*[width=\textwidth, trim=0mm 0mm 0mm 0mm, clip]{hudson.eps}
%%     \end{minipage}
%%     \begin{minipage}[htp]{0.5\linewidth}
%%       \includegraphics*[width=\textwidth, trim=0mm 0mm 0mm 0mm, clip]{hifl_txk0.eps}
%%     \end{minipage}
%%     \caption{{\it{Left panel:}} Shown here is a figure presented in
%%       \cite{2007hvcg.conf...42H} in which is plotted entropy, $K$,
%%       vs. average cluster temperature for the \hifl\
%%       sample. {\it{Right panel:}} This is a reproduction of the
%%       \cite{2007hvcg.conf...42H} figure using our best-fit \kna\
%%       values for the \hifl\ clusters which we analyzed.  In both
%%       panels the dashed lines mark $K = 40 \ent$. We do not know if
%%       the same clusters are shown in both plots, but the bimodality
%%       noted in \cite{2007hvcg.conf...42H} is also evident in our data
%%       and occurs at approximately the same location along the entropy
%%       axis.}
%%     \label{fig:hifltx}
%%   \end{figure}
%% \end{center}

\clearpage
% \begin{figure}[htp]
\begin{figure}[htp]
  \begin{center}
    \begin{minipage}[htp]{0.8\linewidth}
      \includegraphics*[width=\textwidth, trim=20mm 10mm 10mm 10mm, clip]{t0.eps}
    \end{minipage}
    \begin{minipage}[htp]{0.8\linewidth}
      \includegraphics*[width=\textwidth, trim=20mm 10mm 10mm 10mm, clip]{k0cool.eps}
    \end{minipage}
    % \begin{minipage}[htp]{0.9\linewidth}
    %   \includegraphics*[width=\textwidth, trim=25mm 10mm 10mm 10mm, clip]{t0.eps}
    \caption{{\it{Top panel:}} Log-binned histogram and cumulative
    distribution of best-fit core cooling times, $t_{c0}$
    (eqn. \ref{eqn:tc0}), for all the clusters in \accept. Histogram
    bin widths are 0.2 in log space. {\it{Bottom panel:}} Log-binned
    histogram and cumulative distribution of core cooling times
    calculated from best-fit \kna\ values, $t_{c0}(\kna)$
    (eqn. \ref{eqn:tck0}), for all the clusters in \accept. Histogram
    bin widths are 0.2 in log space. The bimodality we observe in the
    \kna\ distribution is also present in best-fit $t_{c0}$. However,
    the gaps between the two populations of $t_{c0}$ and
    $t_{c0}(\kna)$ differ by $\sim 0.3$ Gyrs which may be an artifact of the binning.}
    \label{fig:t0}
    % \end{minipage}
    % \end{center}
  \end{center}
\end{figure}

