% Basic commands
\newcommand{\accept}{\textit{ACCEPT}}
\newcommand{\hifl}{\textit{HIFLUGCS}}
\newcommand{\numobs}{310}
\newcommand{\numcluster}{233}
\newcommand{\expt}{9.66 Msec}
\newcommand{\dkna}{\ensuremath{(\kna^{\prime}-\kna)/\kna}}
% K0 stats for full sample
\newcommand{\alphafs}{\ensuremath{\alpha = 1.21 \pm 0.39}}
\newcommand{\knafs}{\ensuremath{\kna = 72.9 \pm 33.7 \ent}}
\newcommand{\khunfs}{\ensuremath{\khun = 126 \pm 45 \ent}}
% K0 < 50
\newcommand{\alphaga}{\ensuremath{\alpha = 1.20 \pm 0.38}}
\newcommand{\knaga}{\ensuremath{\kna = 16.1 \pm  5.7 \ent}}
\newcommand{\khunga}{\ensuremath{\khun = 150 \pm 50 \ent}}
% K0 > 50
\newcommand{\alphagb}{\ensuremath{\alpha = 1.23 \pm 0.40}}
\newcommand{\knagb}{\ensuremath{\kna = 156 \pm 54 \ent}}
\newcommand{\khungb}{\ensuremath{\khun = 107 \pm 39 \ent}}
% K0 stats for cent src clusters
\newcommand{\centsrcnum}{\ensuremath{37}}
\newcommand{\alphacs}{\ensuremath{\alpha = 1.19 \pm 0.39}}
\newcommand{\knacs}{\ensuremath{\kna = 61.9 \pm 27.4 \ent}}
\newcommand{\khuncs}{\ensuremath{\khun = 132 \pm 45 \ent}}
% K0 < 50
\newcommand{\alphacsa}{\ensuremath{\alpha = 1.16 \pm 0.38}}
\newcommand{\knacsa}{\ensuremath{\kna = 15.6 \pm 5.2 \ent}}
\newcommand{\khuncsa}{\ensuremath{\khun = 146 \pm 48 \ent}}
% K0 > 50
\newcommand{\alphacsb}{\ensuremath{\alpha = 1.23 \pm 0.40}}
\newcommand{\knacsb}{\ensuremath{\kna = 148 \pm 49 \ent}}
\newcommand{\khuncsb}{\ensuremath{\khun = 118 \pm 42 \ent}}
% KMM test for all clusters
\newcommand{\kmma}{\ensuremath{K_1 = 17.8 \pm 6.6 \ent}}
\newcommand{\kmmb}{\ensuremath{K_2 = 154 \pm 52 \ent}}
\newcommand{\kmmc}{\ensuremath{121}}
\newcommand{\kmmd}{\ensuremath{106}} 
\newcommand{\kmme}{\ensuremath{p = 1.16\times10^{-7}}}
% KMM test w/o kna < 4
\newcommand{\kmmf}{\ensuremath{K_1 = 15.0\pm 5.0 \ent}}
\newcommand{\kmmg}{\ensuremath{K_2 = 129 \pm 45 \ent}}
\newcommand{\kmmh}{\ensuremath{89}}
\newcommand{\kmmi}{\ensuremath{131}}
\newcommand{\kmmj}{\ensuremath{p = 1.90\times10^{-13}}}
% Hifl stats
\newcommand{\hifla}{\ensuremath{\alpha = 1.17 \pm 0.37}}
\newcommand{\hiflb}{\ensuremath{\kna = 11.4 \pm 4.2 \ent}}
\newcommand{\hiflc}{\ensuremath{\khun = 235 \pm 89 \ent}}
\newcommand{\hifld}{\ensuremath{\alpha = 1.19 \pm 0.39 \ent}}
\newcommand{\hifle}{\ensuremath{\kna = 151 \pm 53 \ent}}
\newcommand{\hiflf}{\ensuremath{\khun = 113 \pm 43 \ent}}
% HIFL KMM for all
\newcommand{\hiflkmma}{\ensuremath{K_1 = 9.7 \pm 3.5 \ent}}
\newcommand{\hiflkmmb}{\ensuremath{K_2 = 131 \pm 46 \ent}}
\newcommand{\hiflkmmc}{\ensuremath{28}}
\newcommand{\hiflkmmd}{\ensuremath{31}}
\newcommand{\hiflkmme}{\ensuremath{p = 3.34\times10^{-3}}}
% HIFL KMM w/o kna < 4
\newcommand{\hiflkmmf}{\ensuremath{K_1 = 10.5 \pm 3.4 \ent}}
\newcommand{\hiflkmmg}{\ensuremath{K_2 = 116 \pm 42 \ent}}
\newcommand{\hiflkmmh}{\ensuremath{21}}
\newcommand{\hiflkmmi}{\ensuremath{34}}
\newcommand{\hiflkmmj}{\ensuremath{p = 1.55\times10^{-5}}}
% cooling time stats
\newcommand{\tckmma}{\ensuremath{t_{c1} = 0.60 \pm 0.24 \Gyr}}
\newcommand{\tckmmb}{\ensuremath{t_{c2} = 6.23 \pm 2.19 \Gyr}}
\newcommand{\tckmmc}{\ensuremath{130}}
\newcommand{\tckmmd}{\ensuremath{97}}
\newcommand{\tckmme}{\ensuremath{p = 8.77\times10^{-7}}}
\newcommand{\mytitle}{INTRACLUSTER MEDIUM ENTROPY PROFILES FOR A CHANDRA ARCHIVAL SAMPLE OF GALAXY CLUSTERS}

%%%%%%%%%%
% Header %
%%%%%%%%%%

%\documentclass[12pt, preprint]{aastex}
%\documentclass{aastex}
\documentclass{emulateapj}
\usepackage{apjfonts,graphicx,here,common,longtable,ifthen,amsmath,amssymb,natbib}
\usepackage[pagebackref,
  pdftitle={\mytitle},
  pdfauthor={Kenneth W. Cavagnolo},
  pdfsubject={ApJ Supplement},
  pdfkeywords={clusters galaxies X-ray Chandra entropy},
  pdfproducer={LaTeX with hyperref},
  pdfcreator={LaTeX}
  pdfdisplaydoctitle=true,
  colorlinks=true,
  citecolor=blue,
  linkcolor=blue,
  urlcolor=blue]{hyperref}
\bibliographystyle{apj}
%\bibliographystyle{astroads}
\begin{document}
\title{\mytitle}
\author{Kenneth. W. Cavagnolo\altaffilmark{1}, Megan Donahue, G. Mark
  Voit, and Ming Sun}
\affil{Department of Physics and Astronomy, BPS Building, Michigan
  State University, East Lansing, MI 48824}
\altaffiltext{1}{cavagnolo@pa.msu.edu}
\shorttitle{Archive of Entropy Profiles}
\shortauthors{K. W. Cavagnolo et al.}
\journalinfo{The Astrophysical Journal Supplement Series, 000:000-000,2008 July}
\slugcomment{Submitted to ApJ Supplement}

%%%%%%%%%%%%
% Abstract %
%%%%%%%%%%%%

\begin{abstract}
   We present radial entropy profiles of the intracluster medium (ICM)
   for a collection of \numcluster\ clusters taken from the
   \chandra\ X-ray Observatory's Data Archive. Entropy is of great
   interest because it dictates ICM global properties and records the
   thermal history of a cluster. Entropy is therefore a useful
   quantity for studying the effects of feedback on the cluster
   environment and investigating the breakdown of cluster
   self-similarity. We find that most ICM entropy profiles are
   well-fit by a model which is a power-law at large radii and
   approaches a constant value at small radii: $K(r) = \kna + \khun
   (r/100 \kpc)^{\alpha}$, where \kna\ quantifies the typical excess
   of core entropy above the best fitting power-law found at larger
   radii. We also show that the \kna\ distributions of both the full
   archival sample and the primary \hifl\ sample of \citet{hiflugcs1}
   are bimodal with a distinct gap between $\kna \approx 30-50 \ent$
   and population peaks at $\kna \sim 15 \ent$ and $\kna \sim 150
   \ent$. The effect of PSF smearing and angular resolution on
   best-fit \kna\ values are investigated using mock
   \chandra\ observations and degraded entropy profiles,
   respectively. We find that neither of these effects is sufficient
   to explain the entropy profile flattening we measure at small
   radii. The influence of profile curvature and number of radial bins
   on best-fit \kna\ is also considered, and we find no indication
   \kna\ is significantly impacted by either. All data and results
   associated with this work are publicly available via the project
   web site.
\end{abstract}

%%%%%%%%%%%%
% Keywords %
%%%%%%%%%%%%

\keywords{astronomical data bases: miscellaneous -- cooling flows --
  X-rays: general -- X-rays: galaxies: clusters}

%%%%%%%%%%%%%%%%%%%%%%
\section{Introduction}
\label{sec:intro}
%%%%%%%%%%%%%%%%%%%%%%

The general process of galaxy cluster formation through hierarchical
merging is well understood, but many details, such as the impact of
feedback sources on the cluster environment and radiative cooling in
the cluster core, are not. The nature of feedback operating within
clusters is of great interest because of the implications regarding
the formation of massive galaxies and for the cluster mass-observable
scaling relations used in cosmological studies. Early models of
structure formation which included only gravitation predicted
self-similarity among the galaxy cluster population. These
self-similar models made specific predictions for how the physical
properties of galaxy clusters, such as temperature and luminosity,
should scale with cluster redshift and mass \citep{kaiser86, kaiser91,
1991ApJ...383...95E, nfw1, nfw2, 1996ApJ...469..494E,
1997MNRAS.292..289E, 1997ApJ...480...36T, 1998ApJ...503..569E,
1998ApJ...495...80B}. However, numerous observational studies have
shown clusters do not follow the predicted simple mass-observable
scaling relations \citep{edge91, 1998MNRAS.297L..57A,
1998ApJ...504...27M, 1999MNRAS.305..631A, 1999ApJ...520...78H,
2000ApJ...536...73N, 2001A&A...368..749F}. To reconcile observation
with theory, it was realized non-gravitational effects, such as
heating and radiative cooling in cluster cores, could not be neglected
if models were to accurately replicate the process of cluster
formation \citep[\eg][]{kaiser91, 1991ApJ...383...95E,
2000ApJ...532...17L, 2002MNRAS.336..409B}.

As a consequence of radiative cooling, best-fit total cluster
temperature decreases while total cluster luminosity increases. In
addition, feedback sources such as active galactic nuclei (AGN) and
supernovae can drive cluster cores (where most of the cluster flux
originates) away from hydrostatic equilibrium. Thus, at a given mass
scale, radiative cooling and feedback conspire to create dispersion in
otherwise tight mass-observable correlations like mass-luminosity and
mass-temperature. While considerable progress has been made both
observationally and theoretically in the areas of understanding,
quantifying, and reducing scatter in cluster scaling-relations
\citep{1996ApJ...458...27B, 2005ApJ...624..606J, kravtsov06, nagai07,
VV08}, it is still important to understand how, taken as a whole,
non-gravitational processes affect cluster formation and evolution.

A related issue to the departure of clusters from self-similarity is
that of cooling flows in cluster cores. The core cooling time in
50\%-66\% of clusters is much shorter than both the Hubble time and
cluster age \citep{1984ApJ...285....1S, 1992MNRAS.258..177E, white97,
  1998MNRAS.298..416P, 2005MNRAS.359.1481B}. For such clusters (and
without compensatory heating), radiative cooling will result in the
formation of a cooling flow \citep[see][for a
  review]{fabiancfreview}. Early estimates put the mass deposition
rates from cooling flows in the range of $100-1000 \Msol \pyr$
\citep[\eg][]{1984ApJ...276...38J, 1994MNRAS.270L...1E,
  1998MNRAS.298..416P}. However, cooling flow mass deposition rates
inferred from soft X-ray spectroscopy were found to be significantly
less than predicted, with the ICM never reaching temperatures lower
than $T_{virial}/3$ \citep{tamura01, peterson01, peterson03,
  2004A&A...413..415K}. Irrespective of system mass, the expected
massive torrents of cool gas turned out to be more like cooling
trickles.

In addition to the lack of soft X-ray line emission from cooling
flows, prior methodical searches for the end products of cooling flows
(\ie\ molecular gas and emission line nebulae) revealed far less mass
is locked-up in cooled by-products than expected \citep{heckman89,
mcnamara90, odea94, voit95}. The disconnects between observation and
theory have been termed ``the cooling flow problem'' and raise the
question, ``Where has all the cool gas gone?'' The substantial amount
of observational evidence suggests some combination of energetic
feedback sources, such as AGN outbursts and supernovae explosions,
have heated the ICM to selectively remove gas with a short cooling
time and establish quasi-stable thermal balance of the ICM.

Both the breakdown of self-similarity and the cooling flow problem
point toward the need for a better understanding of cluster feedback
and radiative cooling. Recent revisions to models of how clusters form
and evolve by including feedback sources has led to better agreement
between observation and theory \citep{bower06, croton06, saro06}. The
current paradigm regarding the cluster feedback process holds that AGN
are the primary heat delivery mechanism and that an AGN outburst
deposits the requisite energy into the ICM to retard, and in some
cases quench, cooling \citep[see][for a review]{mcnamrev}. How the
feedback loop functions is still the topic of much debate, but that
AGN are interacting with the hot atmospheres of clusters is no longer
in doubt as evidenced by the prevalence of bubbles in clusters
\citep[\eg][]{birzan04}.

One robust observable which has proven useful in studying the effect
of non-gravitational processes is ICM entropy. Taken individually, ICM
temperature and density do not fully reveal a cluster's thermal
history. Gas temperature reflects the depth of the potential well,
while density reflects the capacity of the well to compress the
gas. However, at constant pressure the density of a gas is determined
by its specific entropy. Rewriting the expression for the adiabatic
index, $K \propto P\rho^{-5/3}$, in terms of temperature and electron
density, one can define a new quantity, $K=T_X n_e^{-2/3}$, where
$T_X$ is temperature and $n_e$ is electron gas density
\citep{1999Natur.397..135P, davies00}. This new definiton for $K$
captures the thermal history of the gas because only gains and losses
of heat energy can change $K$. This form for the quantity $K$ is
commonly referred to as entropy in the X-ray cluster literature, but
in actuality the classic thermodynamic specific entropy for a
monatomic ideal gas is $s = \ln K^{3/2} + \mathrm{constant}$.

One important property of gas entropy is that a gas cloud is
convectively stable when $dK/dr \geq 0$. Thus, gravitational potential
wells are giant entropy sorting devices: low entropy gas sinks to the
bottom of the potential well, while high entropy gas buoyantly rises
to a radius at which the ambient gas has equal entropy. If cluster
evolution proceeded under the influence of gravitation only, then the
radial entropy distribution of clusters would exhibit power-law
behavior for $r > 0.1 r_{200}$ with a constant, low entropy core at
small radii \citep{vkb05}. Thus, large-scale departures of the radial
entropy distribution from a power-law can be used to measure the
effect processes such as AGN heating and radiative cooling have on the
ICM. Several studies have previously found that the radial ICM entropy
distribution in some clusters flattens at $< 0.1 r_{virial}$, or that
the core entropy has much larger dispersion than the entropy at larger
radii \citep{1996ApJ...473..692D, 1999Natur.397..135P, davies00,
ponman03, piffaretti05, radioquiet, d06, morandi07}. However, these
previous studies utilized smaller, focused samples, and to expand the
utility of entropy in understanding cluster thermodynamic history and
non-gravitational processes, we have undertaken a much larger study
utilizing the \chandra\ Data Archive.

In this paper we present the data analysis and results from a
\chandra\ archival project in which we studied the ICM entropy
distribution for \numcluster\ galaxy clusters. We have named this
project the ``Archive of \chandra\ Cluster Entropy Profile Tables'' or
\accept\ for short. In contrast to the sample of nine classic cooling
flow clusters studied in \citet[][hereafter D06]{d06}, \accept\ covers
a broader range of luminosities, temperatures, and morphologies,
focusing on more than just cooling flow clusters. One of our primary
objectives for this project was to provide the research community with
an additional resource to study cluster evolution and confront current
models with a comprehensive set of entropy profiles.

We have found that the departure of entropy profiles from a power-law
at small radii is a feature of most clusters, and given high enough
angular resolution, possibly all clusters. We also find that the core
entropy distribution of both the full \accept\ collection and the
Highest X-Ray Flux Galaxy Cluster Sample (\hifl, \citealt{hiflugcs1,
hiflugcs2}) are bimodal. In a separate letter \citep{haradent}, we
present results that show indicators of feedback - radio sources
assumed to be associated with AGN and \halpha\ emission assumed to be
the result of thermal instability formation - are strongly correlated
with core entropy.

A key aspect of this project is the dissemination of all data and
results to the public. We have created a searchable, interactive web
site\footnote{\url{http://www.pa.msu.edu/astro/MC2/accept}} which
hosts all of our results. The \accept\ web site will be continually
updated as new \chandra\ cluster and group observations are archived
and analyzed. The web site provides all data tables, plots, spectra,
reduced \chandra\ data products, reduction scripts, and more. Given
the large number of clusters in our sample, figures, fits, and tables
showing/listing results for individual clusters have been omitted from
this paper and are available at the \accept\ web site.

The structure of this paper is as follows: In \S\ref{sec:sample} we
outline initial sample selection criteria and information about the
\chandra\ observations selected under these criteria. Data reduction
is discussed in \S\ref{sec:data}. Spectral extraction and analysis are
discussed in \S\ref{sec:temppr}, while our method for deriving
deprojected electron density profiles is outlined in
\S\ref{sec:dene}. A few possible sources of systematics are discussed
in \S\ref{sec:sys}. Results and discussion are presented in
\S\ref{sec:r&d}. A brief summary is given in \S\ref{sec:summary}. For
this work we have assumed a flat \LCDM\ Universe with cosmology
$\OM=0.3$, $\OL=0.7$, and $\Hn=70\km\ps\pMpc$. All quoted
uncertainties are 90\% confidence ($1.6\sigma$).

%%%%%%%%%%%%%%%%%%%%%%%%%
\section{Data Collection}
\label{sec:sample}
%%%%%%%%%%%%%%%%%%%%%%%%%

Our sample was initially collected from observations publicly
available in the \chandra\ Data Archive (CDA) as of June 2006. We
first assembled a list of targets from multiple flux-limited surveys:
the \rosat\ Brightest Cluster Sample \citep{1998MNRAS.301..881E}, RBCS
Extended Sample \citep{2000MNRAS.318..333E}, \rosat\ Brightest 55
Sample \citep{1990MNRAS.245..559E, 1998MNRAS.298..416P},
\einstein\ Extended Medium Sensitivity Survey
\citep{1990ApJS...72..567G}, North Ecliptic Pole Survey
\citep{2006ApJS..162..304H}, \rosat\ Deep Cluster Survey
\citep{1995ApJ...445L..11R}, \rosat\ Serendipitous Survey
\citep{1998ApJ...502..558V}, Massive Cluster Survey
\citep{2001ApJ...553..668E}, and {\it{REFLEX}} Survey
\citep{reflex}. After the first round of data analysis concluded, we
continued to expand our collection by adding new archival data listed
under the CDA Science Categories ``clusters of galaxies'' or ``active
galaxies.'' As of submission, we have inspected all CDA clusters of
galaxies observations and analyzed 510 of those observations (14.16
Msec). The Coma and Fornax clusters have been intentionally left out
of our sample because they are very well studied nearby clusters which
require a more intensive analysis than we undertook in this project.

The data available for some clusters limited our ability to derive an
entropy profile. Calculation of ICM entropy requires measurement of
the gas temperature and density structure as a function of radius
(discussed further in \S\ref{sec:data}). To infer a temperature which
was reasonably well constrained ($\Delta (kT_X) \approx \pm 1.0
\keV$), we imposed a minimum requirement that each cluster temperature
profile have at least three concentric radial annular bins containing
2500 counts each. A post-analysis check showed our minimum criterion
resulted in a mean $\Delta (kT_X) = 0.87$ keV for the final sample.

In section \ref{sec:hifl} we cull the \hifl\ primary sample
\citep{hiflugcs1, hiflugcs2} from our full archival collection. The
groups M49, NGC 507, NGC 4636, NGC 5044, NGC 5813, and NGC 5846 are
part of the \hifl\ primary sample but were not members of our initial
archival sample. In order to take full advantage of the \hifl\ primary
sample, we analyzed observations of these 6 groups. Note, however,
that none of these 6 groups are included in the general discussion of
\accept.

We were unable to analyze some clusters for this study because of
complications other than not meeting our minimum requirements for
analysis. These clusters were: 2PIGG J0311.8-2655, 3C 129, A168, A514,
A753, A1367, A2634, A2670, A2877, A3074, A3128, A3627, AS0463, APMCC
0421, MACS J2243.3-0935, MS J1621.5+2640, RX J1109.7+2145, RX
J1206.6+2811, RX J1423.8+2404, SDSS J198.070267-00.984433, Triangulum
Australis, and Zw5247.

After applying the constraint on the temperature profiles, adding the
6 \hifl\ groups, and removing troublesome observations, the final
sample presented in this paper contains \numobs\ observations of
\numcluster\ clusters with a total exposure time of \expt. The sample
covers the temperature range $kT_X \sim 1-20$ keV, a bolometric
luminosity range of $L_{bol} \sim 10^{42-46} \ergps$, and redshifts of
$z \sim 0.05-0.89$. Table \ref{tab:sample} lists the general
properties for each observation in \accept.

We also report previously unpublished \halpha\ observations taken by
M. Donahue while a Carnegie Fellow. These observations are not
utilized in this paper but are used in \citet{haradent}. The new
$[NII]/\halpha$ ratios and \halpha\ fluxes are listed in Table
\ref{tab:newha}. The upper-limits listed in Table \ref{tab:newha} are
$3\sigma$ significance. The observations were taken with either the 5
m Hale Telescope at the Palomar Observatory, USA, or the Du Pont 2.5 m
telescope at the Las Campanas Observatory, Chile. All observations
were made with a $2\arcs$ slit centered on the brightest cluster
galaxy (BCG) using two position angles: one along the semi-major axis
and one along the semi-minor axis of the galaxy. The overlap area was
$10$ pixels$^2$. The red light (555-798 nm) setup on the Hale Double
Spectrograph used a 316 lines/mm grating with a dispersion of 0.31
nm/pixel and an effective resolution of 0.7-0.8 nm. The Du Pont
Modular Spectrograph setup included a 1200 lines/mm grating with a
dispersion of 0.12 nm/pixel and an effective resolution of 0.3 nm. The
statistical and calibration uncertainties for the observations are
both $\sim 10\%$. The statistical uncertainty arises primarily from
variability of the spectral continuum and hence imperfect background
subtraction.

%%%%%%%%%%%%%%%%%%%%%%%
\section{Data Analysis}
\label{sec:data}
%%%%%%%%%%%%%%%%%%%%%%%

Measuring radial ICM entropy first requires measurement of radial ICM
temperature and density. As discussed in \citet{xrayband}, the ICM
X-ray peak of the point-source cleaned, exposure-corrected cluster
image was used as the cluster center, unless the iteratively
determined X-ray centroid was more than 70 kpc away from the X-ray
peak, in which case the centroid was used as the radial analysis
zero-point. The radial temperature structure of each cluster was
measured by fitting a single-temperature thermal model to spectra
extracted from concentric annuli centered on the cluster X-ray
center. To derive the gas density profile, we first deprojected an
exposure-corrected, background-subtracted, point source clean surface
brightness profile extracted in the 0.7-2.0\keV\ energy range to
attain a volume emission density. This emission density, along with
spectroscopic information (count rate and normalization in each
annulus), was then used to calculate gas density. The resulting
entropy profiles were then fit with two models: a simple model
consisting of only a radial power-law, and a model which is the sum of
a constant core entropy term, \kna, and the radial power-law.

In this paper we cover the basics of deriving gas entropy from X-ray
observables, and direct interested readers to D06 for more in-depth
discussion of our data reprocessing and reduction, and
\citet{xrayband} for details regarding determination of each cluster's
center and how the X-ray background was handled. The only difference
between the analysis presented in this paper and that of D06 and
\citet{xrayband}, is that we have used newer versions of the CXC
issued data reduction software (\ciao\ 3.4.1 and \caldb\ 3.4.0).

%%%%%%%%%%%%%%%%%%%%%%%%%%%%%%%%%
\subsection{Temperature Profiles}
\label{sec:temppr}
%%%%%%%%%%%%%%%%%%%%%%%%%%%%%%%%%

One of the two components needed to derive a gas entropy profile is
the temperature as a function of radius. We therefore constructed
radial temperature profiles for each cluster in our collection. To
reliably constrain a temperature, and allow for the detection of
temperature structure beyond isothermality, we required each
temperature profile to have a minimum of three annuli containing 2500
counts each. The annuli for each cluster were generated by first
extracting a background-subtracted cumulative counts profile using 1
pixel width annular bins originating from the cluster center and
extending to the detector edge. Temperature profiles, however, were
truncated at the radius bounded by the detector edge, or $0.5
r_{180}$, whichever was smaller. Truncation occurred at $0.5 r_{180}$
as we are most interested in the radial entropy behavior of cluster
core regions ($r \la 100$ kpc) and $0.5 r_{180}$ is the approximate
radius where temperature profiles begin to turnover
\citep{2005ApJ...628..655V}.  Additionally, analysis of diffuse gas
temperature structure at large radii, which spectroscopically is
dominated by background, requires a time consuming,
observation-specific analysis of the X-ray background \cite[see][for a
  detailed discussion on this point]{minggroups}.

Cumulative counts profiles were divided into annuli containing at
least 2500 counts. For well resolved clusters, the number of counts
per annulus was increased to reduce the resulting uncertainty of
$kT_X$ and, for simplicity, to keep the number of annuli less than
50. The method we use to derive entropy profiles is most sensitive to
the surface brightness radial bin size and not the resolution or
uncertainties of the temperature profile. Thus, the loss of resolution
in the temperature profile from increasing the number of counts per
bin, and thereby reducing the number of annuli, has an insignificant
effect on the final entropy profiles and best-fit entropy models.

Background analysis was performed using the blank-sky datasets
provided in the \caldb. Backgrounds were reprocessed and reprojected
to match each observation. Off-axis chips were used to normalize for
variations of the hard-particle background by comparing blank-sky and
observation 9.5-12\keV\ count rates. Soft residuals
\citet[see][]{2005ApJ...628..655V} were also created
and fitted for each observation to account for the spatially-varying
soft Galactic background. The best-fit spectral model for the residual
soft component (scaled for sky area) was included as an additional,
fixed background component during fitting of cluster spectra. Errors
associated with the additional soft background component were
determined by refitting cluster spectra using the $\pm 1\sigma$
temperatures of the soft background component's best fit model and
then adding the associated error in quadrature to the final total
error budget.

For each radial annular region, source and background spectra were
extracted from the target cluster and corresponding normalized
blank-sky dataset. Following standard
\ciao\ techniques\footnote{\url{http://cxc.harvard.edu/ciao/guides/esa.html}}
we created weighted response files (WARF) and redistribution matrices
(WRMF) for each cluster using a flux-weighted map (WMAP) across the
entire extraction region. These files quantify the effective area,
quantum efficiency, and imperfect resolution of the
\chandra\ instrumentation as a function of chip position. Each
spectrum was binned to contain a minimum of 25 counts per energy bin.

Spectra were fitted with \xspec\ 11.3.2ag \citep{xspec} using an
absorbed, single-temperature \mekal\ model \citep{mekal1, mekal2} over
the energy range 0.7-7.0 \keV. Neutral hydrogen column densities,
\nhi, were taken from \citet{dickeylockman}. A comparison between the
\nhi\ values of \citet{dickeylockman} and the higher-resolution LAB
Survey \citep{lab} revealed that the two surveys agree to within $\pm
20\%$ for 80\% of the clusters in our sample. For the other 20\% of
the sample, using the LAB value, or allowing \nhi\ to be free, did not
result in best-fit temperatures or metallicities which differ
significantly from fits using the \citet{dickeylockman} values.

The potentially free parameters of the absorbed thermal model are
\nhi, X-ray temperature, metal abundance normalized to solar
\citep[elemental ratios taken from][]{ag89}, and a normalization
proportional to the integrated emission measure within the extraction
region,
\begin{equation}
\label{eqn:norm}
\eta = \frac{10^{-14}}{4\pi D_A^2(1+z)^2}\int \nelec \np dV,
\end{equation}
where $D_A$ is the angular diameter distance, $z$ is cluster redshift,
\nelec\ and \np\ are the electron and proton densities respectively,
and $V$ is the volume of the emission region. In all fits the metal
abundance in each annulus was a free parameter and \nhi\ was fixed to
the Galactic value. No systematic error was added during fitting and
thus all quoted errors are statistical only. The statistic used during
fitting was $\chi^2$ (\xspec\ statistics package \textsc{chi}). All
uncertainties were calculated using 90\% confidence.

More than one observation was available in the archive for some
clusters. We utilized the combined exposure time for these clusters by
first extracting independent spectra, WARFs, WRMFs, normalized
background spectra, and soft residuals for each observation. These
independent spectra were then read into \xspec\ simultaneously and fit
with the same spectral model which had all parameters, except
normalization, tied among the spectra.

In D06 we studied a sample of nine ``classic'' cooling flow clusters,
all of which have steep temperature gradients. Spectral deprojection
of ICM temperature should result in slightly lower temperatures in the
central bins of only the clusters with the steepest temperature
gradients. For those clusters, the end result would be a slight, and
typically insignificant, lowering of the entropy for the central-most
bins. Our analysis in D06 showed that spectral deprojection did not
result in significant differences between entropy profiles derived
using projected or deprojected temperature profiles. In addition,
proper spectral deprojection is a time consuming computational
process. Given the large number of clusters, large number of
temperature bins for some clusters, and the insignificant gains, we
concluded spectral deprojection was an unnecessary step. Thus, for
this work, we quote projected temperatures only. We again stress that
spectral deprojection does not significantly change the shape of the
entropy profiles nor the best-fit \kna\ values \citep[see][]{d06}.

%%%%%%%%%%%%%%%%%%%%%%%%%%%%%%%%%%%%%%%%%%%%%%%%%%
\subsection{Deprojected Electron Density Profiles}
\label{sec:dene}
%%%%%%%%%%%%%%%%%%%%%%%%%%%%%%%%%%%%%%%%%%%%%%%%%%

For predominantly free-free emission, emissivity strongly depends on
density and only weakly on temperature, $\epsilon \propto \rho^2
T^{1/2}$. Therefore the flux measured in a narrow temperature range is
a good diagnostic of ICM density. To reconstruct the relevant gas
density as a function of physical radius, we deprojected the cluster
emission from high-resolution surface brightness profiles and
converted to electron density using normalizations and count rates
taken from the spectral analysis.

We extracted surface brightness profiles from the 0.7-2.0 keV energy
range using concentric annular bins of size $5\arcs$ originating from
the cluster center. Each surface brightness profile was corrected with
an observation specific, normalized radial exposure profile to remove
the effects of vignetting and exposure time fluctuations. Following
the recommendation in the \ciao\ guide for analyzing extended sources,
exposure maps were created using the monoenergetic value associated
with the observed count rate peak. The more sophisticated method of
creating exposure maps using spectral weights calculated for an
incident spectrum with the temperature and metallicity of the observed
cluster was also tested. For the narrow energy band we consider, the
chip response is relatively flat and we find no significant
differences between the two methods. For all clusters, the
monoenergetic value used in creating exposure maps was between
$0.8-1.7\keV$.

The 0.7-2.0 keV spectroscopic count rate and spectral normalization
were interpolated from the radial temperature profile grid to match
the surface brightness radial grid. Utilizing the deprojection
technique of \citet{kriss83}, the interpolated spectral parameters were
used to convert observed surface brightness to deprojected electron
density. Radial electron density written in terms of relevant
quantities is,
\begin{equation}
\nelec(r) = \sqrt{\frac{r_{ion}~4 \pi [D_A(1+z)]^2~C(r)~\eta(r)}{10^{-14}~f(r)}}
\end{equation}
where $r_{ion}$ is an ionization ratio ($\nelec=1.2\np$), $C(r)$ is
the radial emission density derived from eqn. A1 in \citet{kriss83},
$\eta$ is the interpolated spectral normalization from
eqn. \ref{eqn:norm}, $D_A$ is the angular diameter distance, $z$ is
cluster redshift, and $f(r)$ is the interpolated spectroscopic count
rate. Cosmic dimming of source surface brightness is accounted for by
the $D_A^2 (1+z)^2$ term. This method of deprojection takes into
account temperature and metallicity fluctuations which affect observed
gas emissivity. Errors for the gas density profile were estimated
using 5000 Monte Carlo simulations of the original surface brightness
profile. The \citet{kriss83} deprojection technique assumes spherical
symmetry, but it was shown in D06 such an assumption has little effect
on final entropy profiles.

%%%%%%%%%%%%%%%%%%%%%%%%%%%%%%%
\subsection{$\beta$-model Fits}
\label{sec:beta}
%%%%%%%%%%%%%%%%%%%%%%%%%%%%%%%

Noisy surface brightness profiles, or profiles with irregularities
such as inversions or extended flat cores, result in unstable,
unphysical quantities when using an ``onion'' deprojection technique
like that of \citet{kriss83}. For cases where deprojection of the raw
data was problematic, we resorted to fitting the surface brightness
profile with a $\beta$-model \citep{1978A&A....70..677C}. It is well
known that the $\beta$-model does not precisely represent all the
features of the ICM for clusters of high central surface brightness
\citep{2000MNRAS.311..313E, 2002ApJ...579..571L,
2007ApJ...665..911H}. However, for the profiles which required a fit,
the $\beta$-model was a suitable approximation, and the model's use was
only a means for creating a smooth function which was easily
deprojected. The single ($N=1$) and double ($N=2$) $\beta$-models were
used in fitting,
\begin{eqnarray}
S_X &=& \displaystyle\sum_{i=1}^N S_i
\left[1+\left(\frac{r}{r_{c,i}}\right)^2\right]^{-3\beta_i+\onehalf}.
\end{eqnarray}
The models were fitted using Craig Markwardt's robust non-linear least
squares minimization IDL
routines\footnote{\url{http://rsinc.com/idl/}}$^{,}$\footnote{\url{http://cow.physics.wisc.edu/~craigm/idl/}}. The
data input to the fitting routines were weighted using the inverse
square of the observational errors. Using this weighting scheme
resulted in residuals which were near unity for, on average, the inner
80\% of the radial range considered. Accuracy of errors output from
the fitting routine were checked against a bootstrap Monte Carlo
analysis of 1000 surface brightness realizations. Both the single- and
double-$\beta$ models were fit to each profile and using the F-test
functionality of
\sherpa\footnote{\url{http://cxc.harvard.edu/ciao3.4/ahelp/ftest.html}}
we determined if the addition of extra model components was justified
given the degrees of freedom and \chisq\ values of each fit. If the
significance was less than 0.05, the extra components were justified
and the double-$\beta$ model was used.

A best-fit $\beta$-model was used in place of the data when deriving
electron density for the clusters listed in Table
\ref{tab:betafits}. These clusters are also flagged in Table
\ref{tab:sample} with the note letter `a.' The best-fit $\beta$-models
and background-subtracted, exposure-corrected surface brightness
profiles are shown in Figure \ref{fig:betamods}. See Appendix
\ref{app:beta} for notes discussing individual clusters. The
disagreement between the best-fit $\beta$-model and the surface
brightness in the central regions for some clusters is also discussed
in Appendix \ref{app:beta}. In short, the discrepancy arises from the
presence of compact X-ray sources, a topic which is addressed in
\S\ref{sec:centsrc}. All clusters requiring a $\beta$-model fit have
core entropy $> 95 \ent$ and the mean best-fit parameters are listed
in Table \ref{tab:bfparams}.

%%%%%%%%%%%%%%%%%%%%%%%%%%%%%
\subsection{Entropy Profiles}
\label{sec:kpr}
%%%%%%%%%%%%%%%%%%%%%%%%%%%%%

Radial entropy profiles were calculated using the widely adopted
formulation $K(r) = kT_x(r)\nelec(r)^{-2/3}$. To create the radial
entropy profiles, the temperature and density profiles must be on the
same radial grid. This was accomplished by interpolating the
temperature profile across the higher-resolution radial grid of the
deprojected electron density profile. Because, in general, density
profiles have higher radial resolution, the central bin of a cluster
temperature profile will span several of the innermost bins of the
density profile. Since we are most interested in the behavior of the
entropy profiles in the central regions, how the interpolation was
performed for the inner regions is important. Thus, temperature
interpolation over the region of the density profile where a single
central temperature bin encompasses several density profile bins was
applied in two ways: (1) as a linear gradient consistent with the
slope of the temperature profile at radii larger than the central
$T_X$ bin ($\Delta T_{center} \ne 0$; `extr' in Table
\ref{tab:kfits}), and (2) as a constant ($\Delta T_{center}=0$; `flat'
in Table \ref{tab:kfits}). Shown in Figure \ref{fig:kcomp} is the
ratio of best-fit core entropy, \kna, using the above two methods. The
five points lying below the line of equality are clusters which are
best-fit by a power-law or have \kna\ statistically consistent with
zero. It is worth noting that both schemes yield statistically
consistent values for \kna\ except for the clusters marked by black
squares which have a ratio significantly different from unity.

The clusters for which the two methods give \kna\ values that
significantly differ all have steep temperature gradients with the
maximum and minimum radial temperatures differing by a factor of
1.3-5.0. Extrapolation of a steep temperature gradient as $r
\rightarrow 0$ results in very low central temperatures (typically
$T_X \leq T_{virial}/3$) which are inconsistent with observations,
most notably \citet{peterson03}. Most important however, is that the
flattening of entropy we observe in the cores of our sample (discussed
in \S\ref{sec:nonzerok0}) is {\bfseries\em{not}} a result of the
method chosen for interpolating the temperature profile. For this
paper, we therefore focus on the entropy results derived assuming a
constant temperature for the central density bins covered by a single
temperature bin.

Uncertainty in $K(r)$ arising from using a single-component
temperature model for each annulus during spectral analysis
contributes negligibly to our final fits and is discussed in detail in
the Appendix of D06. Briefly summarizing D06: we have primarily
measured the entropy of the lowest entropy gas because it is the most
luminous gas. For the best-fit entropy values to be significantly
changed, the volume filling fraction of a higher-entropy component
must be non-trivial ($> 50\%$). As discussed in D06, our results are
robust to the presence of multiple, low luminosity gas phases and
mostly insensitive to X-ray surface brightness decrements, such as
X-ray cavities and bubbles, although in extreme cases their influence
on an entropy profile can be detected (for an example, see the cluster
A2052).

Each entropy profile was fit with two models: a simple model which is
a power-law at large radii and approaches a constant value at small
radii (eqn. \ref{eqn:k0}), and a model which is a power-law only
(eqn. \ref{eqn:plaw}). The models were fitted using Craig Markwardt's
IDL routines in the package MPFIT. The output best-fit parameters and
associated errors were checked against a bootstrap Monte Carlo
analysis of 5000 entropy profile realizations to independently confirm
their accuracy.
\begin{eqnarray}
K(r) &=& \kna + \khun\ \left(\frac{r}{100 \kpc}\right)^{\alpha}\label{eqn:k0}\\
K(r) &=& \khun\ \left(\frac{r}{100 \kpc}\right)^{\alpha}\label{eqn:plaw}.
\end{eqnarray}
In our entropy models, \kna\ is what we call core entropy, \khun\ is a
normalization for entropy at 100 kpc, and $\alpha$ is the power-law
index. Note, however, that \kna\ does not necessarily represent the
minimum core entropy or the entropy at $r=0$. Nor does \kna\ capture
the gas entropy which would be measured immediately around an AGN or
in a compact BCG X-ray corona. Instead, \kna\ represents the typical
excess of core entropy above the best fitting power-law at larger
radii.

The radial range of fitting was truncated at a maximum radius
(determined by-eye) to avoid the influence of noisy bins and profile
turnover at large radii which result from instability of our
deprojection method. A listing of all the best-fit parameters for each
cluster are listed in Table \ref{tab:kfits}. The mean best-fit
parameters for the full \accept\ sample are given in Table
\ref{tab:bfparams}. Also given in Table \ref{tab:bfparams} are the
mean best-fit parameters for clusters below and above $\kna = 50
\ent$. We show in \S\ref{sec:bimod} that the cut at $\kna=50 \ent$ is
not completely arbitrary as it approximately demarcates the division
between two distinct populations in the \kna\ distribution.

Some clusters have a surface brightness profile which is comparable to
a double $\beta$-model. Our models for the behavior of $K(r)$ are
intentionally simplistic and are not intended to fully describe all
the features of $K(r)$. Thus, for the small number of clusters with
discernible double-$\beta$ behavior, fitting of the entropy profiles
was restricted to the innermost of the two $\beta$-like
features. These clusters have been flagged in Table \ref{tab:sample}
with the note letter `b.' The best-fit power-law index is typically
much steeper for these clusters, but the outer regions, which we do
not discuss here, have power-law indices which are typical of the rest
of the sample, \ie\ $\alpha \sim 1.2$.


%%%%%%%%%%%%%%%%%%%%%%%%%%%%%%%%%%%%%%%%%
\subsection{Exclusion of Central Sources}
\label{sec:centsrc}
%%%%%%%%%%%%%%%%%%%%%%%%%%%%%%%%%%%%%%%%%

For many clusters in our sample the ICM X-ray peak, ICM X-ray
centroid, BCG optical emission, and BCG infrared emission are
coincident or well within 70 kpc of one another. This made
identification of the cluster center robust and trivial. However, in
some clusters, there is an X-ray point source or compact X-ray source
($r \la 5$ kpc) found very near ($r < 10$ kpc) the cluster center and
always associated with a galaxy. We identified \centsrcnum\ clusters
with central sources and have flagged them in Table \ref{tab:sample}
with the note letter `d' for AGN and `e' for compact but resolved
sources. The mean best-fit parameters for these clusters are given in
Table \ref{tab:bfparams} under the sample name `CSE' for ``central
source excluded.'' These clusters cover the redshift range $z =
0.0044-0.4641$ with mean $z = 0.1196 \pm 0.1234$, and temperature
range $kT_X = 1-12$ keV with mean $kT_X = 4.43 \pm 2.53$ keV. For some
clusters -- such as 3C 295, A2052, A426, Cygnus A, Hydra A, or M87 --
the source is an AGN and there was no question the source must be
removed.

However, determining how to handle the compact X-ray sources was not
so straightforward. These compact sources are larger than the PSF,
fainter than an AGN, but typically have significantly higher surface
brightness than the surrounding ICM such that the compact source's
extent was distinguishable from the ICM. These sources are most
prominent, and thus the most troublesome, in non-cool core clusters
(\ie\ clusters which are approximately isothermal). They are
troublesome because the compact source is typically much cooler and
denser than the surrounding ICM and hence has an entropy much lower
than the ambient ICM. We consider most of these compact sources as
X-ray coronae associated with the BCG \citep[see][for discussion of
  BCG coronae]{coronae}.

Without removing the compact sources, we derived radial entropy
profiles and found, for all cases, that $K(r)$ abruptly changes at the
outer edge of the compact source. Including the compact sources
results in the central cluster region(s) appearing overdense, and at a
given temperature the region will have a much lower entropy than if
the source were excluded. Such a discontinuity in $K(r)$ results in
our simple models of $K(r)$ not being a good description of the
profiles. Aside from producing poor fits, a significantly lower
entropy influences the value of best-fit parameters because the shape
of $K(r)$ is drastically changed. Obviously, two solutions are
available: exclude or keep the compact sources during analysis.
Deciding what to do with these sources depends upon what cluster
properties we are specifically interested in quantifying.

The compact X-ray sources discussed in this section are not
representative of the cluster's core entropy; these sources are
representative of the entropy within and immediately surrounding
peculiar BCGs. Our focus for the \accept\ project was to quantify the
entropy structure of the cluster core region and surrounding
``pristine'' ICM, not to determine the minimum entropy of cluster
cores or to quantify the entropy of peculiar core objects such as BCG
coronae. Thus, we opted to exclude these compact sources during our
analysis. For a few extraordinary sources, it was simpler to ignore
the central bin of the surface brightness profile during analysis
because of imperfect exclusion of a compact source's extended
emission. These clusters have been flagged in Table \ref{tab:sample}
with the note letter `f.'

It is worth noting that when any source is excluded from the data, the
empty pixels where the source once was were not included in the
calculation of the surface brightness (counts and pixels are both
excluded). Thus, the decrease in surface brightness of a bin where a
source has been removed is not a result of the count to area ratio
being artificially reduced.

%%%%%%%%%%%%%%%%%%%%%
\section{Systematics}
\label{sec:sys}
%%%%%%%%%%%%%%%%%%%%%

Our models for $K(r)$ were designed so that the best-fit \kna\ values
are a good measure of the entropy profile flattening at small
radii. This flattening could potentially be altered through the
effects of systematics such as PSF smearing and surface brightness
profile angular resolution. To quantify the extent to which our
\kna\ values are being affected by these systematics, we have analyzed
mock \chandra\ observations created using the ray-tracing program
MARX\footnote{\url{http://space.mit.edu/CXC/MARX/}}, and also by
analyzing degraded entropy profiles generated from artificially
redshifting well-resolved clusters. In the analysis below, we show
that the lack of clusters with $\kna \la 10 \ent$ at $z \ga 0.1$ is
attributable to resolution effects, but that deviation of an entropy
profile from a power-law, even if only in the central-most bin, cannot
be accounted for by PSF effects. We also discuss the number of
profiles which are reasonably well-represented by the power-law only
profile, and establish that no more than $\sim 10\%$ of the entropy
profiles in \accept\ are consistent with a power-law.

%%%%%%%%%%%%%%%%%%%%%%%%
\subsection{PSF Effects}
\label{sec:psf}
%%%%%%%%%%%%%%%%%%%%%%%%

To assess the effect of PSF smearing on our entropy profiles, we have
updated the analysis presented in \S4.1 of D06 to use MARX
simulations. In the D06 analysis, we assumed the density and
temperature structure of the cluster core obeyed power-laws with $n_e
\propto r^{-1}$ and $T_X \propto r^{1/3}$. This results in a power-law
entropy profile with $K \propto r$. Further assuming the main emission
mechanism is thermal bremsstrahlung, \ie\ $\epsilon_X \propto
T_X^{1/2}$, yields a surface brightness profile which has the form
$S_X \propto r^{-5/6}$. A source image consistent with these
parameters was created in \idl\ and then input to MARX to create the
mock \chandra\ observations.

The MARX simulations were performed using the spectrum of a 4.0 keV,
$0.3 \Zsol$ abundance \mekal\ model. We have tested using input
spectra with $kT_X = 2-10$ keV with varying abundances and find the
effect of temperature and metallicity on the distribution of photons
in MARX to be insignificant for our discussion here. We have neglected
the X-ray background in this analysis as it is overwhelmed by cluster
emission in the core and is only important at large
radii. Observations for both ACIS-S and ACIS-I instruments were
simulated using an exposure time of 40 ksec. A surface brightness
profile was then extracted from the mock observations using the same
$5\arcs$ bins used on the real data.

For $5\arcs$ bins, we find the difference between the central bins of
the input surface brightness and the output MARX observations to be
less than the statistical uncertainty. One should expect this result,
as the on-axis \chandra\ PSF is $\la 1\arcs$ and the surface
brightness bins we have used on the data are five times this
size. What is most interesting and important though, is that our
analysis using MARX suggests any deviation of the surface brightness
-- and consequently the entropy profile -- from a power-law, even if
only in the central bin, is real and cannot be attributed to PSF
effects. Even for the most poorly resolved clusters, the deviation
away from a power-law we observe in a large majority of our entropy
profiles is not a result of our deprojection technique or systematics.

%%%%%%%%%%%%%%%%%%%%%%%%%%%%%%%%%%%%%%%
\subsection{Angular Resolution Effects}
\label{sec:angres}
%%%%%%%%%%%%%%%%%%%%%%%%%%%%%%%%%%%%%%%

Another possible limitation on evaluating \kna\ is the effect of using
fixed angular size bins for extracting surface brightness
profiles. This choice may introduce a redshift-dependence into the
best-fit \kna\ values because as redshift increases, a fixed angular
size encompasses a larger physical volume and the value of \kna\ may
increase if the bin includes a broad range of gas entropy. Shown in
Figure \ref{fig:k0res} is a plot of the best-fit \kna\ values for our
entire sample versus redshift. At low redshift ($z < 0.02$), there are
a few objects with $\kna < 10 \ent$ and only one at higher redshift:
A1991 ($\kna = 1.53 \pm 0.32$, $z = 0.0587$), which is a very peculiar
cluster \citep{2004ApJ...613..180S}. This raises the question: can the
lack of clusters with $\kna \la 10 \ent$ at $z > 0.02$ be completely
explained by resolution effects?

To answer this question we tested the effect redshift has on our
measurements of \kna\ by selecting a subsample containing all the
clusters with $\kna \leq 10 \ent$ and $z \leq 0.1$ and then degrading
their surface brightness profiles to mimic the effect of increasing
the cluster redshift. Our test is best illustrated using an example:
consider a cluster at $z = 0.1$. For this cluster, $5\arcs \approx 9$
kpc. Were the cluster at $z = 0.2$, $5\arcs$ would be $\approx 16$
kpc. To mimic moving this example cluster from $z = 0.1 \rightarrow
0.2$, we can extract a new surface brightness profile using a bin size
of 16 kpc instead of $5\arcs$. This will result in a new surface
brightness profile which has the angular resolution for a cluster at a
higher redshift.

We used the preceding procedure to degrade the profiles of the $\kna
\leq 10 \ent$ and $z \leq 0.1$ subsample. New surface brightness bin
sizes were calculated for each cluster over an evenly distributed grid
of redshifts in the range $z = 0.1-0.4$ using step sizes of 0.02. The
temperature profiles for each cluster were also degraded by starting
at the innermost temperature profile annulus and moving outward
pairing-up neighboring annuli. New spectra were extracted for these
enlarged regions and analyzed following the same procedure detailed in
\S\ref{sec:temppr}.

The ensemble of artificially redshifted clusters were analyzed using
the procedure outlined in \S\ref{sec:kpr}. The notable effects on the
entropy profiles arising from lower angular resolution are: (1) less
information about profile shape, and (2) increased entropy of the
central-most bins. Obviously, as redshift increases, the number of
radial bins decreases. Fewer radial bins translates into a less
detailed sampling of an entropy profile's curvature, \eg\ the profiles
become less ``curvy.'' On its own this effect should lead to lower
best-fit \kna\ values, but, while profile curvature is reduced, the
entropy of the central-most bins is increasing because the bins
encompass a broader range of entropy. From $z = 0.1-0.3$ this last
effect dominates, resulting in an increase of $\dkna = 2.72 \pm 1.84$
where \kna\ is the original best-fit value and $\kna^{\prime}$ is the
best-fit value of the degraded profiles. However, at $z > 0.3$, the
loss of radial resolution dominates and the degraded profiles begin to
resemble power-laws except for the innermost bin which still lies
above the power-law (the uncertainty of the best-fit \kna\ also
increases). The result is that the degraded \kna\ values for $z > 0.3$
are only slightly larger than the best-fit \kna\ of the un-degraded
data, $\dkna = 0.71 \pm 0.57$.

Our analysis of the degraded entropy profiles suggests that \kna\ is
more sensitive to the value of $K(r)$ in the central bins than it is
to the shape of the profile or the number of radial bins (systematics
we explore further in \S\ref{sec:curve}). Most importantly however, is
that low-redshift clusters with $\kna \le 10 \ent$ look like they have
$\kna \approx 10-30 \ent$ if they are placed at $z > 0.1$. Thus we
conclude that the lack of $\kna < 10 \ent$ clusters at $z \ga 0.1$ can
be attributed to resolution effects.

%%%%%%%%%%%%%%%%%%%%%%%%%%%%%%%%%%%%%%%%%%%%%%%%%
\subsection{Profile Curvature and Number of Bins}
\label{sec:curve}
%%%%%%%%%%%%%%%%%%%%%%%%%%%%%%%%%%%%%%%%%%%%%%%%%

From our analysis of the degraded entropy profiles in
\S\ref{sec:angres} we found: (1) that the best-fit \kna\ is sensitive
to the curvature of the entropy profile, and (2) that the number of
radial bins may also affect the best-fit \kna. This raises the
possibility of two troubling systematics in our analysis. To check for
a possible correlation between best-fit \kna\ and profile curvature we
first calculated average profile curvatures, $\kappa_A$. For each
profile, $\kappa_A$ was calculated using the standard formulation for
the curvature of a function, $\kappa = \|y^{''}\|/(1+y^{'2})^{3/2}$,
where we set $y = K(r) = \kna+\khun(r/100\kpc)^{\alpha}$. This
derivation yields,
\begin{equation}
\kappa_A = \frac{\int\frac{\| 100^{-\alpha} (\alpha-1) \alpha \khun
  r^{\alpha-2}\|}{[1+(100^{-\alpha} \alpha \khun
    r^{\alpha-1})^2]^{3/2}} dr}{\int dr}
\end{equation}
where $\alpha$ and \khun\ are the best-fit parameters unique to each
entropy profile. The integral over all space ensures we evaluate the
curvature of each profile in the limit where the profiles have
asymptotically approached a constant at small radii and a power law at
large radii. We find that at any value of \kna a large range of
curvatures are covered and that there is no systematic trend in
\kna\ associated with $\kappa_A$.

In \S\ref{sec:angres} we also found that profiles with fewer radial
bins may tend toward lower best-fit \kna\ values. After examining
plots of best-fit \kna\ versus the number of bins fit in each entropy
profile we found only scatter and no trends.

We do not find any systematic trends with profile shape or number of
fit bins which would significantly affect our best-fit
\kna\ values. Thus, we conclude that the \kna\ values discussed in
this paper are, as intended, an adequate measure of the core entropy,
and that any undetected dependence of \kna\ on profile shape or radial
resolution affect our results at significance levels much smaller than
the measured uncertainties.

%%%%%%%%%%%%%%%%%%%%%%%%%%%%%%%
\subsection{Power-law Profiles}
\label{sec:quality}
%%%%%%%%%%%%%%%%%%%%%%%%%%%%%%%

Equation \ref{eqn:k0} is a special case of eqn. \ref{eqn:plaw} with
$\kna = 0$, \eg\ the models we fit to $K(r)$ are nested. In addition,
the added parameter has an acceptable best-fit value, $\kna = 0$,
which lies on the boundary of the parameter space. While under these
conditions \chisq, associated p-values, and F-tests are not useful in
determining which model is the ``best'' description of $K(r)$,
comparison between the \chisq\ values (shown in Table \ref{tab:kfits})
of each cluster's best-fit models implies, even if only qualitatively,
which model shows more agreement with the data. In addition, for each
fit in Table \ref{tab:kfits} we show the number of sigma,
$\sigma_{\kna}$, the best-fit \kna\ value is from zero. We also show
in Table \ref{tab:bfparams} the number of clusters and the percentage
of the sample which have a \kna\ statistically consistent with zero at
various confidence levels. Table \ref{tab:bfparams} shows that at the
$3\sigma$ significance level only $\sim10\%$ of the full
\accept\ sample has a best-fit \kna\ value which is consistent with
zero.

Of the \numcluster\ clusters in \accept, only four clusters have a
\kna\ value which is statistically consistent with zero at $1\sigma$,
or, based on comparison of reduced \chisq, are better fit by the
power-law only model. These clusters are A2151, AS0405, MS
0116.3-0115, and NGC 507 (part of \hifl\ analysis only). Two
additional clusters, A1991 and A4059, are better fit by the power-law
model only when interpolation of the temperature profile in the core
is not constant (see \S\ref{sec:temppr}). We find that the entropy
model which approaches a constant core entropy at small radii appears
to be a better descriptor of the radial entropy distribution for most
\accept\ clusters. However, we cannot rule out the power-law only
model, but do point out that $\sim90\%$ of clusters have best-fit
\kna\ values greater than zero at $> 3\sigma$ significance. Moreover,
that there is a systematic trend for a single power-law to be a poor
fit mainly at the smallest radii suggests non-zero \kna\ is not
random.

%%%%%%%%%%%%%%%%%%%%%%%%%%%%%%%%
\section{Results and Discussion}
\label{sec:r&d}
%%%%%%%%%%%%%%%%%%%%%%%%%%%%%%%%

Presented in Figure \ref{fig:splots} is a montage of \accept\ entropy
profiles for different temperature ranges. These figures highlight the
cornerstone result of \accept: a uniformly analyzed collection of
entropy profiles covering a broad range of core entropy. Each profile
is color-coded to represent the global cluster temperature. Plotted in
each panel of Fig. \ref{fig:splots} are the mean profiles representing
$\kna \le 50 \ent$ clusters (dashed-line) and $\kna > 50 \ent$
clusters (dashed-dotted line), in addition to the pure-cooling model
of \citet{voitbryan} (solid black line). The theoretical pure-cooling
curve represents the entropy profile of a 5 keV cluster simulated with
radiative cooling but no feedback and gives us a useful baseline
against which to compare \accept\ profiles.

In the following sections we discuss results gleaned from analysis of
our library of entropy profiles. These results include the departure
of most entropy profiles from a simple radial power-law profile, the
bimodal distribution of core entropy, and the asymptotic convergence
of the entropy profiles to the self-similar $K(r) \propto r^{1.1-1.2}$
power-law at $r \geq 100\kpc$.

%%%%%%%%%%%%%%%%%%%%%%%%%%%%%%%%%%
\subsection{Non-Zero Core Entropy}
\label{sec:nonzerok0}
%%%%%%%%%%%%%%%%%%%%%%%%%%%%%%%%%%

Arguably the most striking feature of Figure \ref{fig:splots} is the
departure of most profiles from a simple power-law. Core flattening of
surface brightness profiles (and consequently density profiles) is a
well known feature of clusters (\eg\ \citealt{1984ApJ...276...38J},
\citealt{1999ApJ...517..627M} and \citealt{2000MNRAS.318..715X}). What
is notable in our work however is that, based on comparison of reduced
$\chi^2$ and significance of \kna, very few of the clusters in our
sample have an entropy distribution which is best-fit by the power-law
only model (eqn. \ref{eqn:plaw}), rather they are sufficiently
well-described by the model which flattens in the core
(eqn. \ref{eqn:k0}).

For the six clusters discussed in \S\ref{sec:quality} which are more
consistent with a power-law, it may be the case that the ICM entropy
departs from a power-law at a radial scale smaller than the $5\arcs$
bins we used for extracting surface brightness profiles. After
extracting new surface brightness profiles for these six clusters
using $2.5\arcs$ bins and repeating the analysis, we find that the
profiles for A4059 and AS0405 do flatten. This leaves A1919, A2151, MS
0116.3-0115, and NGC 507 as the only clusters in \accept\ for which
the power-law model cannot be reasonably argued against.

For clusters with central cooling times shorter than the age of the
cluster, non-zero core entropy is an expected consequence of episodic
heating of the ICM \citep{agnframework}, with AGN as one possible
heating source \citep{1997MNRAS.288..355B, 2000ApJ...532...17L,
2001Natur.414..425V, 2001ApJ...549..832S, 2002MNRAS.332..729C,
2002Natur.418..301B, 2002MNRAS.331..545B, 2002MNRAS.333..145N,
2002ApJ...581..223R, 2002MNRAS.335..610A, 2004MNRAS.348.1105O,
2004ApJ...613..811M, 2004ApJ...615..681R, 2004ApJ...617..896H,
2004MNRAS.355..995D, 2005ApJ...622..847S, pizzolato05,
2006ApJ...643..120B, 2006ApJ...638..659M}. Clusters with cooling times
of order the age of the Universe, however, require other mechanisms to
generate their core entropy, for example via mergers or extremely
energetic AGN outbursts. For the very highest \kna\ values, $\kna >
100 \ent$, the mechanism by which the core entropy came to be so large
is not well understood as it is difficult to boost the entropy of a
gas parcel to $> 100 \ent$ via merger shocks
\citep{2008MNRAS.386.1309M} and would require AGN outburst energies
which have never been observed. We are providing the data and results
of \accept\ to the public with the hope that the research community
finds it a useful new resource to further understand the processes
which result in non-zero cluster core entropy.

%%%%%%%%%%%%%%%%%%%%%%%%%%%%%%%%%%%%%%%%%%%%%%%%%%%%
\subsection{Bimodality of Core Entropy Distribution}
\label{sec:bimod}
%%%%%%%%%%%%%%%%%%%%%%%%%%%%%%%%%%%%%%%%%%%%%%%%%%%%

The time required for a gas parcel to radiate away its thermal energy
is a function of the gas entropy. Low entropy gas radiates profusely
and is thus subject to rapid cooling and vice versa for high entropy
gas. Hence, the distribution of \kna\ is of particular interest
because it is an approximate indicator of the cooling timescale in the
cluster core. The \kna\ distribution is also interesting because it
may be useful in better understanding the physical processes operating
in cluster cores. For example, if processes such as thermal conduction
and AGN feedback are important in establishing the entropy state of
cluster cores, then models which properly incorporate these processes
should approximately reproduce the observed \kna\ distribution.

In the top panel of Figure \ref{fig:k0hist} is plotted the
logarithmically binned distribution of \kna. In the bottom panel of
Figure \ref{fig:k0hist} is plotted the cumulative distribution of
\kna. One can immediately see from these distributions that there are
at least two distinct populations separated by a small number of
clusters with $\kna \approx 30-50 \ent$. If the distinct bimodality of
the \kna\ distribution seen in the binned histogram were an artifact
of binning, then the cumulative distribution should be relatively
smooth. But, there is clearly a plateau in the cumulative distribution
which coincides with the division between the two populations at $\kna
\approx 30-50 \ent$. We have tested re-binning the \kna\ histogram
using the optimized binning techniques outlined in \citet{knuthbin}
and \citet{2008arXiv0807.4820H} and find no change in the bimodality
or \kna\ range of the gap versus when using naive fixed width bins.

To further test for the presence of a bimodal population, we utilized
the KMM test of \citet{kmm1}. The KMM test estimates the probability
that a set of data points is better described by the sum of multiple
Gaussians than by a single Gaussian. We tested the unimodal case
versus the bimodal case under the assumption that the dispersion of
the two Gaussian components are not the same. We have used the updated
KMM code of \citet{kmm2} which incorporates bootstrap resampling to
determine uncertainties for all parameters. A post-analysis comparison
of fits assuming the populations have the same and different
dispersions confirms our initial guess that the dispersions are
different is a better model.

The KMM test, as with any statistical test, is very specific. At
zeroth order, the KMM test simply determines if a population is
unimodal or not, and finds the means of these populations. However,
the dispersions of these populations are subject to the quality of
sampling and the presence of outliers (\eg\ KMM must assign all data
points to a population). The outputs of the KMM test are the best-fit
populations to the data, not necessarily the best-fit populations of
the underlying distribution (hence no goodness of fit is
output). However, the KMM test does output a p-value, $p$, and with
the assumption that \chisq\ describes the distribution of the
likelihood ratio statistic, $p$ is the confidence interval for the
null hypothesis.

There are a small number of clusters with $\kna \le 4 \ent$ that when
included in the KMM test significantly change the results. Thus, we
conducted tests including and excluding $\kna \le 4 \ent$ clusters and
provide two sets of best-fit parameters. The results of the bimodal
KMM test neglecting $\kna \le 4 \ent$ clusters were two statistically
distinct peaks at \kmma\ and \kmmb. \kmmc\ clusters were assigned to
the first distribution, while \kmmd\ were assigned to the
second. Including $\kna \le 4 \ent$ clusters, the bimodal KMM test
found populations at \kmmf\ (\kmmh\ clusters) and
\kmmg\ (\kmmi\ clusters). The bimodal KMM test neglecting $\kna \le 4
\ent$ clusters returned \kmme, while the test including all clusters
returned \kmmj. These tiny $p$-values indicate the unimodal
distribution is significantly rejected as the parent distribution of
the observed \kna\ distribution.

One possible explanation for a bimodal core entropy distribution is
that it arises from the effects of episodic AGN feedback and electron
thermal conduction in the cluster core. \citet{agnframework} outlined
a model of AGN feedback whereby outbursts of $\sim 10^{45} \ergps$
occurring every $\sim 10^8 \yrs$ can maintain a quasi-steady core
entropy of $\approx 10-30 \ent$. In addition, very energetic and
infrequent AGN outbursts of $10^{61} \erg$ can increase the core
entropy into the $\approx 30-50 \ent$ range. This model satisfactorily
explains the distribution of $\kna \lesssim 50 \ent$, but depletion of
the $\kna = 30-50 \ent$ region and populating $\kna > 50 \ent$
requires more physics. \citet{conduction} have recently suggested that
the dramatic fall-off of clusters beginning at $\kna \approx 30 \ent$
may be the result of electron thermal conduction. After \kna\ has
exceeded $\approx 30 \ent$, conduction could severely slow, if not
halt, a cluster's core from appreciably cooling and returning to a
core entropy state with $\kna < 30 \ent$. This model is supported by
results presented in \citet{haradent}, \citet{2008arXiv0804.3823G},
and \citet{2008arXiv0802.1864R} which find that the formation of
thermal instabilities and signatures of ongoing feedback and star
formation are extremely sensitive to the core entropy state of a
cluster.

We acknowledge that \accept\ is not a complete, uniformly selected
sample of clusters. This raises the possibility that our sample is
biased towards clusters that have historically drawn the attention of
observers, such as cooling flows or mergers. If that were the case,
then one reasonable explanation of the \kna\ bimodality is that $\kna
= 30-50 \ent$ clusters are ``boring'' and thus go unobserved. However,
as we show in \S\ref{sec:hifl}, the unbiased, flux-limited
\hifl\ sample is also bimodal.

%%%%%%%%%%%%%%%%%%%%%%%%%%%%%%%%%%
\subsection{The \hifl\ Sub-Sample}
\label{sec:hifl}
%%%%%%%%%%%%%%%%%%%%%%%%%%%%%%%%%%

\accept\ is not a flux-limited or volume-limited sample. To ensure our
results are not affected by an unknown selection bias, we culled the
\hifl\ sample from \accept\ for separate analysis. \hifl\ is a
flux-limited sample ($f_X \ge 2 \times 10^{-11} \flux$) selected from
the {\it{REFLEX}} sample \citep{reflex} with no consideration of
morphology. Thus, at any given luminosity in \hifl\ there is a good
sampling of different morphologies, \ie\ possible bias toward
cool-cores or mergers has been removed. The sample also covers most of
the sky with holes near Virgo and the Large and Small Magellanic
Clouds, and has no known incompleteness
\citep{2007A&A...466..805C}. There are a total of 106 objects in
\hifl: 63 in the primary sample and 43 in the extended sample. Of
these 106 objects, no public \chandra\ observations were available for
16 objects (A548e, A548w, A1775, A1800, A3528n, A3530, A3532, A3560,
A3695, A3827, A3888, AS0636, HCG 94, IC 1365, NGC 499, RXCJ
2344.2-0422), 6 objects did not meet our minimum analysis requirements
and were thus insufficient for study (3C 129, A1367, A2634, A2877,
A3627, Triangulum Australis), and as discussed in \S\ref{sec:sample},
Coma and Fornax were intentionally ignored. This left a total of 82
\hifl\ objects which we analyzed, 59 from the primary sample ($\sim
94\%$ complete) and 23 from the extended sample ($\sim 50\%$
complete). The primary sample is the more complete of the two, thus we
focus our following discussion on the primary sample only.

The clusters missing from the primary \hifl\ sample are A1367, A2634,
Coma, and Fornax. The extent to which these 4 clusters can change our
analysis of the \kna\ distribution for \hifl\ is limited.  To alter or
wash-out bimodality, all 4 clusters would need to fall in the range
$\kna = 20-40 \ent$, which is certainly not the case for any of these
clusters. A1367 has been studied by \citet{1998ApJ...500..138D} and
\citet{2002ApJ...576..708S}, with both finding that two sub-clusters
are merging in the cluster. The merger process, and the potential for
associated shock formation, is known to create large increases of gas
entropy \citep{2007MNRAS.376..497M}. Given the combination of low
surface brightness, moderate temperatures ($kT_X = 3.5-5.0$ keV), lack
of a temperature gradient, ongoing merger, and presence of a shock, it
is unlikely A1367 has a core entropy $\la 40 \ent$. A2634 is a very
low surface brightness cluster with the bright radio source 3C 465 at
the center of an X-ray coronae \citep{coronae}. Clusters with
comparable properties to A2634 are not found to have $\kna \la 40
\ent$. Coma and Fornax are known to have core entropy $> 40 \ent$
\citep[][C. Scharf, private communication]{2008arXiv0802.1864R}.

Shown in Figure \ref{fig:hiflk0} are the log-binned (top panel) and
cumulative (bottom panel) \kna\ distributions of the \hifl\ primary
sample. The bimodality seen in the full \accept\ collection is also
present in the \hifl\ sub-sample. Mean best-fit parameters are given
in Table \ref{tab:bfparams}. We again performed two KMM tests: one
test with, and another test without, clusters having $\kna \le 4
\ent$. For the test including $\kna \le 4 \ent$ clusters we find
populations at \hiflkmma\ (\hiflkmmc\ clusters) and
\hiflkmmb\ (\hiflkmmd\ clusters) with \hiflkmme. Excluding clusters
with $\kna \le 4 \ent$ we find peaks at \hiflkmmf\ and \hiflkmmg, each
having \hiflkmmh\ and \hiflkmmi\ clusters, respectively, and
\hiflkmmj.

\citet{2007hvcg.conf...42H} note a similar core entropy bimodality to
the one we find here. \citet{2007hvcg.conf...42H} discuss two distinct
groupings of objects in a plot of average cluster temperature versus
core entropy, with the dividing point being $K \approx 40 \ent$. Our
results agree with the findings of \citet{2007hvcg.conf...42H}. While
the gaps of \accept\ and \hifl\ do not cover the same \kna\ range, it
is interesting that both gaps appear to be the deepest around $\kna
\approx 30 \ent$. That bimodality is present in both \accept\ and the
unbiased \hifl\ sub-sample suggests bimodality cannot be the result of
simple archival bias.

%%%%%%%%%%%%%%%%%%%%%%%%%%%%%%%%%%%%%%%%%%%%%%%
\subsection{Distribution of Core Cooling Times}
\label{sec:hifl}
%%%%%%%%%%%%%%%%%%%%%%%%%%%%%%%%%%%%%%%%%%%%%%%

In the X-ray regime, cooling time and entropy are related in that
decreasing gas entropy also means shorter cooling time. Thus, if the
\kna\ distribution is bimodal, the distribution of cooling times
should also be bimodal. We have calculated cooling time profiles from
the spectral analysis using the relation
\begin{equation}
\tcool = \frac{3nkT_X}{2\nelec \nH \Lambda(T,Z)}
\label{eqn:tcool}
\end{equation}
where $n$ is the total ion number ($\approx 2.3\nH$ for a fully
ionized plasma), \nelec\ and \nH\ are the electron and proton
densities respectively, $\Lambda(T,Z)$ is the cooling function for a
given temperature and metal abundance, and $3/2$ is a constant
associated with isochoric cooling. The values of the cooling function
for each temperature profile bin were calculated in \xspec\ using the
flux of the best-fit spectral model. Following the procedure discussed
in \S\ref{sec:kpr}, $\Lambda$ and $kT_X$ were interpolated across the
radial grid of the electron density profile. The cooling time profiles
were then fit with a simple model analogous to that used for fitting
$K(r)$:
\begin{equation}
\tcool(r) = t_{c0} + t_{100} \left(\frac{r}{100 \kpc}\right)^{\alpha}
\label{eqn:tc0}
\end{equation}
where $t_{c0}$ is core cooling time and $t_{100}$ is a normalization
at 100 kpc.

The \kna\ distribution can also be used to explore the distribution of
core cooling times. Assuming free-free interactions are the dominant
gas cooling mechanism (\ie\ $\epsilon \propto T^{1/2}$),
\citet{radioquiet} show that for free-free emission, entropy is
related to cooling time via the formulation:
\begin{equation}
t_{c0}(\kna) \approx 10^8 \yrs\ \left(\frac{\kna}{10 \keV \cmsq}\right)^{3/2} \left(\frac{kT_X}{5 \keV}\right)^{-1}.
\label{eqn:tck0}
\end{equation}

Shown in Figure \ref{fig:t0} is the logarithmically binned and
cumulative distributions of best-fit core cooling times from
eqn. \ref{eqn:tc0} (top panel) and core cooling times calculated using
eqn. \ref{eqn:tck0} (bottom panel). The bin widths in both histograms
are 0.20 in log-space. The pile-up of cluster core cooling times below
1 Gyr is well known, for example in \citet{hu85} or more recently by
\citet{dunn08}. In addition, the core cooling times we calculate are
consistent with the results of other cooling time studies, such as
\citet{1998MNRAS.298..416P} or \citet{2008arXiv0802.1864R}. However,
what is most important about Fig. \ref{fig:t0} is that the distinct
bimodality of the \kna\ distribution is also present in best-fit core
cooling time, $t_{c0}$. A KMM bimodality test using $t_{c0}$ found
peaks at \tckmma\ and \tckmmb\ with \tckmmc\ and \tckmmd\ objects in
each respective population. The probability that the unimodal
distribution is a better fit was once again exceedingly small,
\tckmme.

If entropy is more closely related to the physical processes which
cause bimodality than is cooling time, then that the cooling time
distribution does not present with the sharp, deep bimodality seen in
\kna\ suggests entropy is the fundamental quantity related to
bimodality. But, since cooling time profiles are more sensitive to the
resolution of the temperature profiles than are the entropy profiles,
it may be that resolution effects are limiting the quantification of
the true cooling time of the core. For example, if our temperature
interpolation scheme is too coarse, or averaging over many small-scale
temperature fluctuations significantly increases $t_{c0}$, then
$t_{c0}$ would not be the best approximation of true core cooling
time. In which case, the core cooling times might be shorter and the
sharpness and offsets of the distribution gaps may significantly
change.

%%%%%%%%%%%%%%%%%%%%%%%%%%%%%%%%%%%%%%%%%%%%%%%%%%%%%%%%%%%%
\subsection{Slope and Normalization of Power-law Components}
\label{sec:slopes}
%%%%%%%%%%%%%%%%%%%%%%%%%%%%%%%%%%%%%%%%%%%%%%%%%%%%%%%%%%%%

Beyond $r \approx 100 \kpc$ the entropy profiles show a striking
similarity in the slope of the power-law component which is
independent of \kna. For the full sample, the mean value of \alphafs.
For clusters with $\kna < 50 \ent$, the mean \alphaga, and for
clusters with $\kna \geq 50 \ent$, the mean \alphagb. Our mean slope
of $\alpha \approx 1.2$ is not statistically different from the
theoretical value of $\alpha = 1.1$ found by \citet{tozzi01} and
$\alpha = 1.2$ found by \citet{vkb05}. For the full sample, the mean
value of \khunfs. Again distinguishing between clusters below and
above $\kna\ = 50 \ent$, we find \khunga\ and \khungb,
respectively. Scaling each entropy profile by the cluster virial
temperature and virial radius considerably reduces the dispersion in
\khun, but we reserve detailed discussion of scaling relations for a
future paper.

%%%%%%%%%%%%%%%%%%%%%%%%%%%%%%%%%%%%%%%%%%%%%%%%%%
\subsection{Comparison with Other Entropy Studies}
\label{sec:comp}
%%%%%%%%%%%%%%%%%%%%%%%%%%%%%%%%%%%%%%%%%%%%%%%%%%

There are many published studies of ICM entropy, and in this section
we compare our results with the results of just a few other
studies. The studies which we have made comparison with are:
\begin{description}
\item \citet{davies00}: \rosat\ and \asca\ data for 20 bound galaxy
  systems in the redshift range $z \approx 0.08-0.2$ and temperature
  range $kT_X \approx 0.5-14$ keV was used in this study.
\item \citet{ponman03}: This study used a sample of 66 systems,
  observed with \rosat\ and \asca, in the redshift range $z=
  0.0036-0.208$ and temperature range $kT_X = 0.5-17$ keV and was the
  largest sample with which we compared our results.
\item \citet{piffaretti05}: Using \xmm\ data for 17 cooling flow
  clusters in the temperature range $kT_X = 1-7$ keV, this study found
  no isentropic cores and that the profiles had a mean power law index
  of $\alpha = 0.95 \pm 0.02$. However, \xmm's angular resolution
  limited the radial analysis to no less than $0.01 r_{virial}$, and
  \citet{piffaretti05} did find the dispersion of entropy in the
  central-most bins to be greater than at larger radii.
\item \citet{pratt06}: The galaxy cluster sample used in this study
  consisted of 10 relaxed systems at $z < 0.2$ with temperatures in
  the range $kT_X \approx 2.5-8$ keV. The clusters were observed with
  \xmm. Like \citet{piffaretti05}, \citet{pratt06} did not find
  isentropic cores, but this is again likely due to \xmm's inability
  to resolve the core region. \citet{pratt06} did however find $<
  20\%$ dispersion in entropy at $r > 0.1r_{200}$ and $> 60\%$
  dispersion at $r \sim 0.02r_{200}$ in addition to a mean power law
  index of $\alpha = 1.08 \pm 0.04$.
\item \citet{morandi07}: Using \chandra\ data, this study examined 24
  galaxy clusters with $kT_X > 6$ keV in the redshift range
  $z=0.14–0.82$. \citet{morandi07} found that the power law indices
  for various subsamples to be in the range $\alpha=1-1.18$, and that
  all of the entropy profiles flatten at $r < 0.5r_{2500}$. They also
  found best-fit \kna\ values in the range $20-300 \ent$.
\end{description}

In general, we find good agreement between the properties of our
entropy profiles and the profiles presented in the papers cited above,
specifically that:
\begin{enumerate}
\item The core region ($r \la 0.1 r_{virial}$) of clusters approach
  isentropic behavior as $r \rightarrow 0$, or in the cases where the
  observations do not resolve the core regions, the dispersion of
  entropy within the core region is considerably larger than the
  dispersion of the entropy at larger ($r \ga 0.1 r_{virial}$) radii.
\item Cluster entropy profiles at $r \ga 0.1 r_{virial}$ are well
  described by an entropy distribution which goes as $K(r) \propto
  r^{1.1-1.2}$.
\item The above two properties are seen in the entropy profiles of
  clusters over a large range of redshifts ($0.05 \la z \la 0.5$) and
  temperatures ($0.5 \keV \la kT_X \la 15 \keV$).
\end{enumerate}  

%%%%%%%%%%%%%%%%%%%%%%%%%%%%%%%%%
\section{Summary and Conclusions}
\label{sec:summary}
%%%%%%%%%%%%%%%%%%%%%%%%%%%%%%%%%

We have presented intracluster medium entropy profiles for a sample of
\numcluster\ galaxy clusters (\expt) taken from the \chandra\ Data
Archive. We have named this project \accept\ for ``Archive of Chandra
Cluster Entropy Profile Tables.'' The reduced data products, data
tables, figures, cluster images, and results of our analysis for all
clusters and observations are freely available at the \accept\ web
site: \url{http://www.pa.msu.edu/astro/MC2/accept}. We encourage
observers and theorists to utilize this library of entropy profiles in
their own work.

We created radial temperature profiles using spectra extracted from a
minimum of three concentric annuli containing 2500 counts each and
extending to either the chip edge or $0.5 r_{180}$, whichever was
smaller. We deprojected surface brightness profiles extracted from
$5\arcs$ bins over the energy range 0.7-2.0 keV to obtain the electron
gas density as a function of radius. Entropy profiles were calculated
from the density and temperature profiles as $K(r) =
T(r)n(r)^{-2/3}$. Two models for the entropy distribution were then
fit to each profile: a power-law only model (eqn. \ref{eqn:plaw}) and
a power-law which approaches a constant value at small radii
(eqn. \ref{eqn:k0}).

We have demonstrated that the entropy profiles for the majority of
\accept\ clusters are well-represented by the model which approaches a
constant entropy, \kna, in the core. The entropy profiles of
\accept\ are also remarkably similar at radii greater than 100 kpc,
and asymptotically approach the self-similar pure-cooling curve ($r
\propto 1.2$) with a slope of \alphafs\ (the dispersion here is in the
sample, not in the uncertainty of the measurement). We also find that
the distribution of \kna\ for the full archival sample is bimodal with
the two populations separated by a poorly populated region between
$\kna \approx 30-50 \ent$. After culling out the primary
\hifl\ sub-sample of \citet{hiflugcs1}, we find the \kna\ distribution
of this complete sub-sample to be bimodal, refuting the possibility of
archival bias.

When we compared our results with those of a few other entropy
studies, specifically \citet{davies00}, \citet{ponman03},
\citet{piffaretti05}, \citet{pratt06}, and \citet{morandi07}, we found
good agreement between the results, noting however that
\citet{piffaretti05} and \citet{pratt06} did not specifically find
isentropic cores. However, those two studies did find large dispersion
of entropy in the core region ($r < 0.1 r_{virial}$), suggesting that
\xmm\ did not resolve the flattened entropy profiles we measure with
the finer angular resolution of \chandra.

Two core cooling times were derived for each cluster: (1) cooling time
profiles were calculated using eqn. \ref{eqn:tcool} and each cooling
time profile was then fit with eqn. \ref{eqn:tc0} returning a best-fit
core cooling time, $t_{c0}$; (2) Using best-fit \kna\ values, entropy
was converted to a core cooling time, $t_{c0}(\kna)$ using
eqn. \ref{eqn:tck0}. We find the distributions of both core cooling
times to be bimodal. Comparison of the core cooling times from method
(1) and (2) reveals that the gap in the bimodal cooling time
distributions occur over different timescales, $\sim 2-3$ Gyrs for
$t_{c0}$, and $\sim 0.7-1$ for $t_{c0}(\kna)$, but this offset may be
the result of resolution limitations.

After analyzing an ensemble of artificially redshifted entropy
profiles, we find the lack of $\kna \la 10 \ent$ clusters at $z > 0.1$
is most likely a result of resolution effects. Investigation of
possible systematics affecting best-fit \kna\ values, such as profile
curvature and number of profile bins, revealed no trends which would
significantly affect our results. We came to the conclusion that
\kna\ is an acceptable measure of average core entropy and is not
overly influenced by profile shape or radial resolution. We also find
that $\sim90\%$ of the sample clusters have a best-fit \kna\ more than
$3\sigma$ away from zero.

Our results regarding non-zero core entropy and \kna\ bimodality
support the sharpening picture of how feedback and radiative cooling
in clusters alter global cluster properties and affect massive galaxy
formation. Among the many models of AGN feedback,
\citet{agnframework} outlined a model which specifically addresses how
AGN outbursts generate and sustain non-zero core entropy in the regime
of $\kna \la 30 \ent$ \citep[see also][]{kaiser03}. In addition, if
electron thermal conduction is an important process in clusters, then
there exists a critical entropy threshold below which conduction is no
longer efficient at wiping out thermal instabilities, the consequences
of which should be a bimodal core entropy distribution and a
sensitivity of cooling by-product formation (like star formation and
AGN activity) to this entropy threshold \citep{conduction,
2008arXiv0804.3823G}. We show in \citet{haradent} that indicators of
feedback like \halpha\ and radio emission are extremely sensitive to
the lower-bound of the bimodal gap at $\kna \approx 30 \ent$.

However, many details are still missing from this emerging picture,
and there are many open questions regarding the evolution of the ICM
and formation of thermal instabilities in cluster cores: How are
clusters with $\kna > 100 \ent$ produced? Is an early episode of
pre-heating necessary? What are the role of MHD instabilities, \eg\
MTI \citep{2000ApJ...534..420B, 2008ApJ...673..758Q} and HBI
\citep{2008ApJ...677L...9P}, in shaping the ICM?  Are the compact
X-ray sources we find at the cores of some BCGs truly coronae? If so,
are their properties consistent with the sample studied by
\citet{coronae}? Can BCG corona properties be used to constrain the
effects of conduction? We hope \accept\ will be a useful resource in
answering these questions.

%%%%%%%%%%%%%%%%%
\acknowledgements
%%%%%%%%%%%%%%%%%

K. W. C. was supported in this work through \chandra\ X-ray
Observatory Archive grants AR-6016X and AR-4017A. M. D. acknowledges
support from the NASA LTSA program NNG-05GD82G. The \chandra\ X-ray
Observatory Center is operated by the Smithsonian Astrophysical
Observatory for and on behalf of NASA under contract
NAS8-03060. K. W. C. thanks Chris Waters for supplying and supporting
his new KMM bimodality code. K. W. C. thanks Jim Linnemann for very
helpful suggestions regarding the error and statistical analysis
presented in this paper. This research has made use of software
provided by the Chandra X-ray Center in the application packages
\ciao, \chips, and \sherpa. This research has made use of the
NASA/IPAC Extragalactic Database which is operated by the Jet
Propulsion Laboratory, California Institute of Technology, under
contract with NASA. This research has also made use of NASA's
Astrophysics Data System. Some software was obtained from the High
Energy Astrophysics Science Archive Research Center, provided by
NASA's Goddard Space Flight Center.

%%%%%%%%%%%%%%
% Facilities %
%%%%%%%%%%%%%%

{\it Facilities:} \facility{CXO (ACIS)}, \facility{Du Pont (Modular
  Spectrograph)}, \facility{Hale (Double Spectrograph)}

%%%%%%%%%%%%%%%%
% Bibliography %
%%%%%%%%%%%%%%%%

\bibliography{cavagnolo}

%%%%%%%%%%%%%%
% Appendices %
%%%%%%%%%%%%%%

\begin{appendix}

%%%%%%%%%%%%%%%%%%%%%%%%%%%%%%%%%%%%%%%%%%%%%%%%%%%%%%%
\section{Notes on clusters requiring $\beta$-model fit}
\label{app:beta}
%%%%%%%%%%%%%%%%%%%%%%%%%%%%%%%%%%%%%%%%%%%%%%%%%%%%%%%
%% ABELL_0119 ABELL_0160 ABELL_0193 ABELL_0400 ABELL_1240 ABELL_1736
%% ABELL_2125 ABELL_2255 ABELL_2319 ABELL_2462 ABELL_2631 ABELL_3376
%% ABELL_3391 ABELL_3395 MKW_08 RBS_0461

\begin{description}
\item[Abell 119 ($z=0.0442$):] This is a highly diffuse cluster
  without a prominent cool core. The large core region and slowly
  varying surface brightness made deprojection highly unstable. We
  have excluded a small source at the very center of the BCG. The
  exclusion region for the source is $\approx 2.2\arcsec$ in radius
  which at the redshift of the cluster is $\sim 2$ kpc. This cluster
  required a double $\beta$-model.

\item[Abell 160 ($z=0.0447$):] The highly asymmetric, low surface
  brightness of this cluster resulted in a noisy surface brightness
  profile that could not be deprojected. This cluster required a
  double $\beta$-model. The BCG hosts a compact X-ray source. The
  exclusion region for the compact source has a radius of $\sim
  5\arcsec$ or $\sim 4.3$ kpc. The BCG for this cluster is not
  coincident with the X-ray centroid and hence is not at the
  zero-point of our radial analysis.

\item[Abell 193 ($z=0.0485$):] This cluster has an azimuthally
  symmetric and a very diffuse ICM centered on a BCG which is
  interacting with a companion galaxy. In Fig. \ref{fig:betamods} one
  can see that the central three bins of this cluster's surface
  brightness profile are highly discrepant from the best-fit
  $\beta$-model. This is a result of the BCG being coincident with a
  bright, compact X-ray source. As we have concluded in
  \ref{sec:centsrc}, compact X-ray sources are excluded from our
  analysis as they are not the focus of our study here. Hence we have
  used the best-fit $\beta$-model in deriving $K(r)$ instead of the
  raw surface brightness.

\item[Abell 400 ($z=0.0240$):] The two ellipticals at the center of
  this cluster have compact X-ray sources which are excluded during
  analysis. The core entropy we derive for this cluster is in
  agreement with that found by \cite{2006A&A...453..433H} which
  supports the accuracy of the $\beta$-model we have used.

\item[Abell 1060 ($z=0.0125$):] There is a distinct compact source
  associated with the BCG in this cluster. The ICM is also very faint
  and uniform in surface brightness making the compact source that
  much more obvious. Deprojection was unstable because of imperfect
  exclusion of the source.

\item[Abell 1240 ($z=0.1590$):] The surface brightness of this cluster
  is well-modeled by a $\beta$-model. There is nothing peculiar worth
  noting about the BCG or the core of this cluster.

\item[Abell 1736 ($z=0.0338$):] Another ``boring'' cluster with a very
  diffuse low surface brightness ICM, no peaky core, and no signs of
  merger activity in the X-ray. The noisy surface brightness profile
  necessitated the use of a double $\beta$-model. The BCG is
  coincident with a very compact X-ray source, but the BCG is offset
  from the X-ray centroid and thus the central bins are not adversely
  affected. The radius of the exclusion region for the compact source
  is $\approx 2.3\arcsec$ or $1.5$ kpc.

\item[Abell 2125 ($z=0.2465$):] Although the ICM of this cluster is
  very similar to the other clusters listed here (\ie\ diffuse, large
  cores), A2125 is one of the more compact clusters. The presence of
  several merging sub-clusters \citep{1997ApJ...487L..13W,
    2004ApJ...611..821W} to the NW of the main cluster form a diffuse
  mass which cannot rightly be excluded. This complication yields
  inversions of the deprojected surface brightness profile if a double
  $\beta$-model is not used.

\item[Abell 2255 ($z=0.0805$):] This is a very well studied merger
  cluster \citep{1995ApJ...446..583B, 1997A&A...317..432F}. The core
  of this cluster is very large ($r > 200$ kpc). Such large extended
  cores cannot be deprojected using our methods because if too many
  neighboring bins have approximately the same surface brightness,
  deprojection results in bins with negative or zero value. The
  surface brightness for this cluster is well modeled as a $\beta$
  function.

\item[Abell 2319 ($z=0.0562$):] A2319 is another well studied merger
  cluster \citep{1997NewA....2..501F, 1999ApJ...525L..73M} with a very
  large core region ($r > 100$ kpc) and a prominent cold front
  \citep{2004ApJ...604..604O}. Once again, the surface brightness
  profile is well-fit by a $\beta$-model.

\item[Abell 2462 ($z=0.0737$):] This cluster is very similar in
  appearance to A193: highly symmetric ICM with a bright, compact
  X-ray source embedded at the center of an extended diffuse ICM. The
  central compact source has been excluded from our analysis with a
  region of radius $\approx 1.5\arcsec$ or $\sim 3$ kpc. The central
  bin of the surface brightness profile is most likely boosted above
  the best-fit double $\beta$-model because of faint extended emission
  from the compact source which cannot be discerned from the ambient
  ICM.

\item[Abell 2631 ($z=0.2779$):] The surface brightness profile for
  this cluster is rather regular, but because the cluster has a large
  core it suffers from the same unstable deprojection as A2255 and
  A2319. The ICM is symmetric about the BCG and is incredibly uniform
  in the core region. We did not detect or exclude a source at the
  center of this cluster, but under heavy binning the cluster image
  appears to have a source coincident with the BCG, and the slightly
  higher flux in central bin of the surface brightness profile may be
  a result of an unresolved source.

\item[Abell 3376 ($z=0.0456$):] The large core of this cluster ($r >
  120$ kpc) makes deprojection unstable and a $\beta$-model must be
  used.

\item[Abell 3391 ($z=0.0560$):] The BCG is coincident with a compact
  X-ray source. The source is excluded using a region with radius
  $\approx 2\arcsec$ or $\sim 2$ kpc. The large uniform core region
  made deprojection unstable and thus required a $\beta$-model fit.

\item[Abell 3395 ($z=0.0510$):] The surface brightness profile for
  this cluster is noisy resulting in deprojection inversions and
  requiring a $\beta$-model fit. The BCG of this cluster has a compact
  X-ray source and this source was excluded using a region with radius
  $\approx 1.9\arcsec$ or $\sim 2$ kpc.

\item[MKW 08 ($z=0.0270$):] MKW 08 is a nearby large group/poor
  cluster with a pair of interacting elliptical galaxies in the
  core. The BCG falls directly in the middle of the ACIS-I detector
  gap. However, despite the lack of proper exposure, CCD dithering
  reveals that a very bright X-ray source is associated with the
  BCG. A double $\beta$-model was necessary for this cluster because
  the low surface brightness of the ICM is noisy and deprojection is
  unstable.

\item[RBS 461 ($z=0.0290$):] This is another nearby large group/poor
  cluster with an extended, diffuse, axisymmetric, featureless ICM
  centered on the BCG. The BCG is coincident with a compact source
  with size $r \approx 1.7$ kpc. This source was excluded during
  reduction. The $\beta$-model is a good fit to the surface brightness
  profile.
\end{description}

%% %%%%%%%%%%%%%%%%%%%%%%%%%%%%%%%%%%%%%%%%%%%%%%%%%%%%%%%
%% \section{Notes on clusters with central source removed}
%% \label{app:centsrc}
%% %%%%%%%%%%%%%%%%%%%%%%%%%%%%%%%%%%%%%%%%%%%%%%%%%%%%%%%

%% 2PIGG_J0011.5-2850, 3C_388, 4C_55.16, ABELL_0223, ABELL_0426, ABELL_0539, ABELL_0562,
%% ABELL_0576, ABELL_0611, ABELL_0744, ABELL_2052, ABELL_2151,
%% ABELL_2717, ABELL_3112, ABELL_3558, ABELL_3581, ABELL_3822,
%% CYGNUS_A, HYDRA_A, M87, MACS_J0547.0-3904, MACS_J1931.8-2634,
%% RBS_0797, RX_J1320.2+3308, ZwCl_0857.9+2107, ZWICKY_1742

%% The clusters A119, A160, A193, A1736, A2462, A3391, A3395, and RBS461
%% also have a central source removed during analysis, but they are
%% discussed in Appendix \ref{app:beta}.

%% \begin{description}
%% \item[3C 295 ($z=0.4641$):] The core of this cluster has been
%% studied in detail by \cite{2001MNRAS.324..842A}. In the central 50 kpc
%% \cite{2001MNRAS.324..842A} found, as we do, that the temperature drops
%% from $\sim 5.0$ keV to $\sim 3.5$ keV. \cite{2001MNRAS.324..842A} also
%% derive a mass deposition rate of $\dot{M} = 280~\Msolpy$ indicating
%% the core of this cluster has a strong cooling flow. As was done in
%% \cite{2001MNRAS.324..842A}, three sources are excluded from the core
%% during our analysis: the region surrounding the central AGN and two
%% nearby radio hot spots \citep{2000ApJ...530L..81H}.

%% \item[3C 388 ($z=0.0917$):]
%% From \cite{2006ApJ...639..753K}:
%% in process of CF quenching
%% The radio galaxy 3C 388 is classified as a Fanaroff-Riley type II (FR
%% II) radio galaxy, although its luminosity ( W Hz−1; Fanaroff \& Riley
%% 1974) lies near the FR I/II dividing line. The radio morphology of
%% this source is closer to a “fat double” (Owen \& Laing 1989) than the
%% canonical FR II “classical double” such as Cyg A and 3C 98. Optically,
%% the nucleus of 3C 388 is classified as a low-excitation radio galaxy
%% (Jackson \& Rawlings 1997). Multifrequency VLA observations of 3C 388
%% show significant structure in spectral index maps that has been
%% interpreted as evidence for multiple nuclear outbursts (Roettiger et
%% al. 1994). Previous X-ray observations of this radio galaxy have shown
%% that it is embedded in a cluster environment (Feigelson \& Berg 1983;
%% Hardcastle \& Worrall 1999; Leahy \& Gizani 2001). The local galaxy
%% environment is extremely dense (Prestage \& Peacock 1988), and the
%% central elliptical galaxy that hosts 3C 388 is one of the most
%% luminous (MB=-24.24) in the local universe (Owen \& Laing 1989; Martel et
%% al. 1999).

%% \item[4C 55.16 ($z=0.2420$):]
%% From \cite{2001MNRAS.328L...5I}:
%% 4C+55.16 is a compact powerful radio source residing in a large galaxy
%% at a redshift of 0.240 (Pearson \& Readhead 1981, 1988; Whyborn et
%% al. 1985; Hutchings, Johnson \& Pyke 1988). Recently, luminous cluster
%% emission (~1045 erg s−1) around the radio galaxy has been recognized
%% through ASCA and ROSAT High Resolution Imager (HRI) observations
%% (Iwasawa et al. 1999).  The point source at the nucleus shows a hard
%% X-ray spectrum, which can be attributed naturally to non-thermal
%% emission from the active nucleus.

%% \item[Abell 223 ($z=0.2070$):]
%% From \cite{2000A&A...355..443P}:
%% As already noticed by Sandage et al. (1976), these two neighboring
%% clusters have nearly the same redshift and probably constitute an
%% interacting system which is going to merge in the future. Both are
%% dominated by a particularly bright cD galaxy. They have a richness
%% class R=3 and are X-ray luminous with [FORMULA] and [FORMULA] for A
%% 222 and A 223, respectively (Lea \& Henry 1988). The BOW83 sample
%% covers only the central regions of these two clusters and, in order to
%% study the galaxy distribution in these systems, as well as to estimate
%% the projected density for the galaxies in our sample (see below), we
%% have built a more extensive, although shallower, galaxy catalog,
%% covering a region of [FORMULA] centered on the median position of the
%% two clusters. This catalogue, with 356 objects, was extracted from
%% Digital Sky Survey (DSS) images, using the software SExtractor (Bertin
%% \& Arnouts 1996). It is more than 90\% complete to BOW83 magnitudes
%% [FORMULA].

%% \item[Abell 426 ($z=0.0179$):]
%% Come on, it's Perseus, you don't know about this cluster? Well, it's
%% got an AGN, just ask \cite{perseus1, perseus2, perseus3}.

%% \item[Abell 539 ($z=0.0288$):]
%% From \cite{1988AJ.....96.1775O}:
%% Within 1 Mpc of the center, the physical parameters of A539 are found
%% to be typical of those of rich clusters. It is shown that early-type
%% galaxies are more concentrated toward the cluster center and that the
%% velocity distributions of early-type and late-type galaxies differ
%% marginally.

%% \item[Abell 562 ($z=0.1100$):]
%% From \cite{1997ApJ...474..580G}:
%% The X-ray emission from this cluster is elongated and shows the radio
%% source offset from the central X-ray peak by 30''. The substructure
%% test (Table A1) detects a significant X-ray excess east of the
%% WAT. The radio pressure is in rough agreement with the thermal
%% pressure. Optically, the cluster is dominated by the WAT host galaxy.

%% \item[Abell 576 ($z=0.0385$):]
%% From \cite{1996ApJ...470..724M}:
%% The central cluster region contains a nonemission galaxy population
%% and an intracluster medium which is significantly cooler (σ\_core\_ =
%% 387\_-105\_\^+250\^ km s\^-1\^ and T\_x\_ = 1.6\_-0.3\_\^+0.4\^ keV at 90\%
%% confidence) than the global populations (σ = 977\_-96\_\^+124\^ km s\^- 1\^
%% for the nonemission population and T\_X\_ > 4 keV at 90\%
%% confidence). Because (1) the low-dispersion galaxy population is no
%% more luminous than the global population and (2) the evidence for a
%% cooling flow is weak, we suggest that the core of A576 may contain the
%% remnants of a lower mass subcluster.
%% From \cite{2004ApJ...607..220K}:
%% We present data from a Chandra observation of the nearby cluster of
%% galaxies A576. The core of the cluster shows a significant departure
%% from dynamical equilibrium. We show that this core gas is most likely
%% the remnant of a merging subcluster, which has been stripped of much
%% of its gas, depositing a stream of gas behind it in the main
%% cluster. The unstripped remnant of the subcluster is characterized by
%% a different temperature, density, and metallicity than that of the
%% surrounding main cluster, suggesting its distinct origin. Continual
%% dissipation of the kinetic energy of this minor merger may be
%% sufficient to counteract most cooling in the main cluster over the
%% lifetime of the merger event.

%% \item[Abell 611 ($z=0.2880$):]
%% From \cite{2002MNRAS.337.1207G}:
%% Abell 611 is a cluster at z= 0.288 (Crawford et al. 1995) originally
%% identified by Abell (1957). It has a 0.1–2.4 keV luminosity of 8.63 ×
%% 1044 W (Böhringer et al. 2000), with a temperature of 7.95+0.56−0.52×
%% 107 K (White 2000). White derived this value from a 57-ks ASCA
%% exposure by considering both a single-phase and two-phase cooling
%% model. The temperature values found for the bulk of the gas are
%% statistically equivalent, and a mass deposit rate of 0+177−0 Mo yr−1
%% was found for the cooling model. The 17-ks ROSAT HRI observation from
%% 1996 April is shown with 8-arcsec binning in Fig. 7. The image
%% contains two bright pixels, which, on comparison with the POSS image,
%% are coincident with a large galaxy. These pixels are ignored whilst
%% fitting a model to this observation.

%% \item[Abell 744 ($z=0.0729$):]
%% From \cite{1985AJ.....90.1665K}:
%% The authors present X-ray and optical observations of the cluster of
%% galaxies Abell 744. The X-ray flux (assuming H0 = 100 km s-1Mpc-1) is
%% ≡9×1042erg s-1. The X-ray source is extended, but shows no other
%% structure. The authors present photographic photometry (in
%% Kron-Cousins R), calibrated by deep CCD frames, for all galaxies
%% brighter than 19th magnitude within 0.75 Mpc of the cluster
%% center. The luminosity function is normal, and the isopleths show
%% little evidence of substructure near the cluster center. The cluster
%% has a dominant central galaxy which the authors classify as a normal
%% brightest-cluster elliptical on the basis of its luminosity
%% profile. New redshifts were obtained for 26 galaxies in the vicinity
%% of the cluster center; 20 appear to be cluster members. The spatial
%% distribution of redshifts is peculiar; the dispersion within the 150
%% kpc core radius is much greater than outside. Abell 744 is similar to
%% the nearby cluster Abell 1060.

%% \item[Abell 2052 ($z=0.0353$):]
%% AGN \cite{2001ApJ...558L..15B, 2003ApJ...585..227B}.

%% \item[Abell 2151 ($z=0.0366$):]
%% From \cite{1995AJ....109..465M}:
%% it's a merger with three distinct pops in vel disp space.

%% \item[Abell 2717 ($z=0.0475$):]
%% From \cite{1997A&A...321...64L}:
%% We present an X-ray, radio and optical study of the cluster A
%% 2717. The central D galaxy is associated with a Wide-Angled-Tailed
%% (WAT) radio source. A Rosat PSPC observation of the cluster shows that
%% the cluster has a well constrained temperature of
%% 1.9\^+0.3\^\_-0.2\_x10\^7\^K. The pressure of the intracluster medium was
%% found to be comparable to the mininum pressure of the radio source
%% suggesting that the tails may in fact be in equipartition with the
%% surrounding hot gas.

%% \item[Abell 3112 ($z=0.0720$):]
%% It's one of Lieu's soft excess clusters \cite{2007ApJ...668..796B}
%% searching for cold gas in A3112 \cite{2004A&A...421..503L}
%% From \cite{2003ApJ...595..142T}:
%% We present the results of a Chandra observation of the central region
%% of A3112. This cluster has a powerful radio source in the center and
%% was believed to have a strong cooling flow. The X-ray image shows that
%% the intracluster medium (ICM) is distributed smoothly on large scales
%% but has significant deviations from a simple concentric elliptical
%% isophotal model near the center. Regions of excess emission appear to
%% surround two lobelike radio-emitting regions. This structure probably
%% indicates that hot X-ray gas and radio lobes are interacting. From an
%% analysis of the X-ray spectra in annuli, we found clear evidence for a
%% temperature decrease and abundance increase toward the center. The
%% X-ray spectrum of the central region is consistent with a
%% single-temperature thermal plasma model. The contribution of X-ray
%% emission from a multiphase cooling flow component with gas cooling to
%% very low temperatures locally is limited to less than 10\% of the
%% total emission. However, the whole cluster spectrum indicates that the
%% ICM is cooling significantly as a whole, but only in a limited
%% temperature range (>=2 keV). Inside the cooling radius the conduction
%% timescales based on the Spitzer conductivity are shorter than the
%% cooling timescales. We detect an X-ray point source in the cluster
%% center that is coincident with the optical nucleus of the central cD
%% galaxy and the core of the associated radio source. The X-ray spectrum
%% of the central point source can be fitted by a 1.3 keV thermal plasma
%% and a power-law component whose photon index is 1.9. The thermal
%% component is probably plasma associated with the cD galaxy. We
%% attribute the power-law component to the central active galactic
%% nucleus.

%% \item[Abell 3558 ($z=0.0480$):]
%% From \cite{2007A&A...463..839R}:
%% Combining XMM-Newton and Chandra data, we have performed a detailed
%% study of A3558. Our analysis shows that its dynamical history is more
%% complicated than previously thought. We have found some traits typical
%% of cool core clusters (surface brightness peaked at the center, peaked
%% metal abundance profile) and others that are more common in merging
%% clusters, like deviations from spherical symmetry in the thermodynamic
%% quantities of the ICM. This last result has been achieved with a new
%% technique for deriving temperature maps from images. We have also
%% detected a cold front and, with the combined use of XMM-Newton and
%% Chandra, we have characterized its properties, such as the speed and
%% the metal abundance profile across the edge. This cold front is
%% probably due to the sloshing of the core, induced by the perturbation
%% of the gravitational potential associated with a past merger. The
%% hydrodynamic processes related to this perturbation have presumably
%% produced a tail of lower entropy, higher pressure and metal rich ICM,
%% which extends behind the cold front for~500 kpc. The unique
%% characteristics of A3558 are probably due to the very peculiar
%% environment in which it is located: the core of the Shapley
%% supercluster.

%% \item[Abell 3581 ($z=0.0218$):]
%% From \cite{2007A&A...463..839R}:
%% We present results from an analysis of a Chandra observation of the
%% cluster of galaxies A3581. We discover the presence of a point-source
%% in the central dominant galaxy that is coincident with the core of the
%% radio source PKS 1404-267. The emission from the intracluster medium
%% is analysed, both as seen in projection on the sky, and after
%% correcting for projection effects, to determine the spatial
%% distribution of gas temperature, density and metallicity. We find that
%% the cluster, despite hosting a moderately powerful radio source, shows
%% a temperature decline to around 0.4 Tmax within the central 5 kpc. The
%% cluster is notable for the low entropy within its core. We test and
%% validate the XSPEC PROJCT model for determining the intrinsic cluster
%% gas properties.

%% \item[Abell 3822 ($z=0.0759$):]
%% zero literature, seriously, only mentioned in survey papers.

%% \item[Cygnus A ($z=0.0561$):]
%% AGN \cite{2002ApJ...565..195S}

%% \item[Hydra A ($z=0.0549$):]
%% AGN \cite{2000ApJ...534L.135M, 2001ApJ...557..546D,
%%   2002ApJ...568..163N}

%% \item[M87 ($z=0.0044$):]
%% AGN \cite{2002ApJ...564..683M, 2005ApJ...635..894F}

%% \item[MACS J0547.0-3904 ($z=0.2100$):]
%% no lit

%% \item[MACS J1931.8-2634 ($z=0.3520$):]
%% no lit

%% \item[RBS 797 ($z=0.3540$):]
%% From \cite{2001A&A...376L..27S}:
%% We present CHANDRA observations of the X-ray luminous, distant galaxy
%% cluster RBS797 at z=0.35. In the central region the X-ray emission
%% shows two pronounced X-ray minima, which are located opposite to each
%% other with respect to the cluster centre. These depressions suggest an
%% interaction between the central radio galaxy and the intra-cluster
%% medium, which would be the first detection in such a distant
%% cluster. The minima are symmetric relative to the cluster centre and
%% very deep compared to similar features found in a few other nearby
%% clusters. A spectral and morphological analysis of the overall cluster
%% emission shows that RBS797 is a hot cluster (T=7.7+1.2-1.0 keV) with a
%% total mass of Mtot(r500)= 6.5+1.6-1.2 *E14Msun.

%% \item[RX J1320.2+3308 ($z=0.0366$):]
%% no lit

%% \item[ZwCl 0857.9+2107 ($z=0.2350$):]
%% no lit

%% \item[Zwicky 1742 ($z=0.0757$):]
%% brand new obs

%% \end{description}

\end{appendix}


%%%%%%%%%%%%%%%%%%%%%%
% Figures  and Tables%
%%%%%%%%%%%%%%%%%%%%%%

\clearpage
\LongTables
\begin{deluxetable}{lcccccccc}
\tablewidth{0pt}
\tabletypesize{\scriptsize}
\tablecaption{Summary of Sample\label{tab:sample}}
\tablehead{\colhead{Cluster} & \colhead{Obs. ID} & \colhead{R.A.} & \colhead{Decl.} & \colhead{Exposure Time} & \colhead{ACIS} & \colhead{$z$} & \colhead{$kT_{X}$} & \colhead{Notes}\\
\colhead{} & \colhead{} & \colhead{hr:min:sec} & \colhead{$\mydeg:\arcm:\arcs$} & \colhead{ksec} & \colhead{} & \colhead{} & \colhead{keV} & \colhead{}\\
\colhead{{(1)}} & \colhead{{(2)}} & \colhead{{(3)}} & \colhead{{(4)}} & \colhead{{(5)}} & \colhead{{(6)}} & \colhead{{(7)}} & \colhead{{(8)}} & \colhead{{(9)}}}
\startdata
 \object{1E0657 56} & \dataset[ADS/Sa.CXO#obs/3184]{3184} & 06:58:29.627 & -55:56:39.79 & 87.5 & I3 & 0.2960 & 11.64 & \nodata\\
 & \dataset[ADS/Sa.CXO#obs/5356]{5356} & \nodata & \nodata & 97.2 & I2 & \nodata & \nodata & \nodata\\
 & \dataset[ADS/Sa.CXO#obs/5361]{5361} & \nodata & \nodata & 82.6 & I3 & \nodata & \nodata & \nodata\\
 \object{2A 335+096} & \dataset[ADS/Sa.CXO#obs/919]{919} & 03:38:41.105 & +09:58:00.66 & 19.7 & S3 & 0.0347 & 2.88 & \nodata\\
 \object{2PIGG J0011.5-2850} & \dataset[ADS/Sa.CXO#obs/5797]{5797} & 00:11:21.623 & -28:51:14.44 & 19.9 & I3 & 0.0753 & 5.15 &      f\\
 \object{2PIGG J2227.0-3041} & \dataset[ADS/Sa.CXO#obs/5798]{5798} & 22:27:54.560 & -30:34:34.84 & 22.3 & I2 & 0.0729 & 2.79 & \nodata\\
 \object{3C 28.0} & \dataset[ADS/Sa.CXO#obs/3233]{3233} & 00:55:50.401 & +26:24:36.47 & 49.7 & I3 & 0.1952 & 5.53 & \nodata\\
 \object{3C 295} & \dataset[ADS/Sa.CXO#obs/2254]{2254} & 14:11:20.280 & +52:12:10.55 & 90.9 & I3 & 0.4641 & 5.16 &      d\\
 \object{3C 388} & \dataset[ADS/Sa.CXO#obs/5295]{5295} & 18:44:02.365 & +45:33:29.31 & 30.7 & I3 & 0.0917 & 3.23 &      d\\
 \object{4C 55.16} & \dataset[ADS/Sa.CXO#obs/4940]{4940} & 08:34:54.923 & +55:34:21.15 & 96.0 & S3 & 0.2420 & 4.98 &      d\\
 \object{Abell 13} & \dataset[ADS/Sa.CXO#obs/4945]{4945} & 00:13:37.883 & -19:30:09.10 & 55.3 & S3 & 0.0940 & 6.84 & \nodata\\
 \object{Abell 68} & \dataset[ADS/Sa.CXO#obs/3250]{3250} & 00:37:06.475 & +09:09:32.28 & 10.0 & I3 & 0.2546 & 9.01 & \nodata\\
 \object{Abell 85} & \dataset[ADS/Sa.CXO#obs/904]{904} & 00:41:50.406 & -09:18:10.79 & 38.4 & I0 & 0.0558 & 6.40 & \nodata\\
 \object{Abell 119} & \dataset[ADS/Sa.CXO#obs/4180]{4180} & 00:56:15.150 & -01:14:59.70 & 11.9 & I3 & 0.0442 & 5.86 &    a,e\\
 \object{Abell 133} & \dataset[ADS/Sa.CXO#obs/2203]{2203} & 01:02:41.756 & -21:52:49.79 & 35.5 & S3 & 0.0558 & 4.31 & \nodata\\
 \object{Abell 141} & \dataset[ADS/Sa.CXO#obs/9410]{9410} & 01:05:34.385 & -24:37:58.78 & 19.9 & I3 & 0.2300 & 5.31 & \nodata\\
 \object{Abell 160} & \dataset[ADS/Sa.CXO#obs/3219]{3219} & 01:13:00.692 & +15:29:15.08 & 58.5 & I3 & 0.0447 & 1.88 &    a,e\\
 \object{Abell 193} & \dataset[ADS/Sa.CXO#obs/6931]{6931} & 01:25:07.660 & +08:41:57.08 & 17.9 & S3 & 0.0485 & 2.50 &    a,e\\
 \object{Abell 209} & \dataset[ADS/Sa.CXO#obs/3579]{3579} & 01:31:52.565 & -13:36:38.79 & 10.0 & I3 & 0.2060 & 8.28 & \nodata\\
 & \dataset[ADS/Sa.CXO#obs/522]{522} & \nodata & \nodata & 10.0 & I3 & \nodata & \nodata & \nodata\\
 \object{Abell 222} & \dataset[ADS/Sa.CXO#obs/4967]{4967} & 01:37:34.562 & -12:59:34.88 & 45.1 & I3 & 0.2130 & 4.60 & \nodata\\
 \object{Abell 223} & \dataset[ADS/Sa.CXO#obs/49671]{49671} & 01:37:55.963 & -12:49:10.53 & 45.1 & I0 & 0.2070 & 5.28 &      e\\
 \object{Abell 262} & \dataset[ADS/Sa.CXO#obs/2215]{2215} & 01:52:46.299 & +36:09:11.80 & 28.7 & S3 & 0.0164 & 2.18 & \nodata\\
 & \dataset[ADS/Sa.CXO#obs/7921]{7921} & \nodata & \nodata & 110.7 & S3 & \nodata & \nodata & \nodata\\
 \object{Abell 267} & \dataset[ADS/Sa.CXO#obs/1448]{1448} & 01:52:42.269 & +01:00:45.33 & 7.9 & I3 & 0.2300 & 6.79 & \nodata\\
 & \dataset[ADS/Sa.CXO#obs/3580]{3580} & \nodata & \nodata & 19.9 & I3 & \nodata & \nodata & \nodata\\
 \object{Abell 368} & \dataset[ADS/Sa.CXO#obs/9412]{9412} & 02:37:27.640 & -26:30:28.99 & 18.4 & I3 & 0.2200 & 6.23 & \nodata\\
 \object{Abell 370} & \dataset[ADS/Sa.CXO#obs/515]{515} & 02:39:53.169 & -01:34:36.96 & 88.0 & S3 & 0.3747 & 7.35 & \nodata\\
 \object{Abell 383} & \dataset[ADS/Sa.CXO#obs/2321]{2321} & 02:48:03.364 & -03:31:44.69 & 19.5 & S3 & 0.1871 & 4.91 & \nodata\\
 \object{Abell 399} & \dataset[ADS/Sa.CXO#obs/3230]{3230} & 02:57:53.382 & +13:01:30.86 & 48.6 & I0 & 0.0716 & 7.95 & \nodata\\
 \object{Abell 400} & \dataset[ADS/Sa.CXO#obs/4181]{4181} & 02:57:41.603 & +06:01:27.61 & 21.5 & I3 & 0.0240 & 2.31 &    a,e\\
 \object{Abell 401} & \dataset[ADS/Sa.CXO#obs/2309]{2309} & 02:58:56.920 & +13:34:14.51 & 11.6 & I2 & 0.0745 & 8.07 & \nodata\\
 & \dataset[ADS/Sa.CXO#obs/518]{518} & \nodata & \nodata & 18.0 & I3 & \nodata & \nodata & \nodata\\
 \object{Abell 426} & \dataset[ADS/Sa.CXO#obs/3209]{3209} & 03:19:48.194 & +41:30:40.73 & 95.8 & S3 & 0.0179 & 3.55 &      d\\
 & \dataset[ADS/Sa.CXO#obs/4289]{4289} & \nodata & \nodata & 95.4 & S3 & \nodata & \nodata & \nodata\\
 \object{Abell 478} & \dataset[ADS/Sa.CXO#obs/1669]{1669} & 04:13:25.345 & +10:27:55.15 & 42.4 & S3 & 0.0883 & 7.07 & \nodata\\
 & \dataset[ADS/Sa.CXO#obs/6102]{6102} & \nodata & \nodata & 10.0 & I3 & \nodata & \nodata & \nodata\\
 \object{Abell 496} & \dataset[ADS/Sa.CXO#obs/3361]{3361} & 04:33:38.038 & -13:15:39.65 & 10.0 & S3 & 0.0328 & 5.03 & \nodata\\
 \object{Abell 520} & \dataset[ADS/Sa.CXO#obs/4215]{4215} & 04:54:10.303 & +02:55:36.48 & 66.3 & I3 & 0.2020 & 9.29 & \nodata\\
 \object{Abell 521} & \dataset[ADS/Sa.CXO#obs/430]{430} & 04:54:06.337 & -10:13:16.88 & 39.1 & S3 & 0.2533 & 7.03 & \nodata\\
 \object{Abell 539} & \dataset[ADS/Sa.CXO#obs/5808]{5808} & 05:16:37.335 & +06:26:25.18 & 24.3 & I3 & 0.0288 & 3.24 &    b,e\\
 & \dataset[ADS/Sa.CXO#obs/7209]{7209} & \nodata & \nodata & 18.6 & I3 & \nodata & \nodata & \nodata\\
 \object{Abell 562} & \dataset[ADS/Sa.CXO#obs/6936]{6936} & 06:53:21.524 & +69:19:51.19 & 51.5 & S3 & 0.1100 & 3.04 &      e\\
 \object{Abell 576} & \dataset[ADS/Sa.CXO#obs/3289]{3289} & 07:21:30.394 & +55:45:41.95 & 38.6 & S3 & 0.0385 & 4.43 &      e\\
 \object{Abell 586} & \dataset[ADS/Sa.CXO#obs/530]{530} & 07:32:20.339 & +31:37:58.59 & 10.0 & I3 & 0.1710 & 6.47 & \nodata\\
 \object{Abell 611} & \dataset[ADS/Sa.CXO#obs/3194]{3194} & 08:00:56.832 & +36:03:24.09 & 36.1 & S3 & 0.2880 & 7.06 &      e\\
 \object{Abell 644} & \dataset[ADS/Sa.CXO#obs/2211]{2211} & 08:17:25.225 & -07:30:40.03 & 29.7 & I3 & 0.0698 & 7.73 & \nodata\\
 \object{Abell 665} & \dataset[ADS/Sa.CXO#obs/3586]{3586} & 08:30:59.226 & +65:50:20.06 & 29.7 & I3 & 0.1810 & 7.45 & \nodata\\
 \object{Abell 697} & \dataset[ADS/Sa.CXO#obs/4217]{4217} & 08:42:57.549 & +36:21:57.65 & 19.5 & I3 & 0.2820 & 9.52 & \nodata\\
 \object{Abell 744} & \dataset[ADS/Sa.CXO#obs/6947]{6947} & 09:07:20.455 & +16:39:06.18 & 39.5 & I3 & 0.0729 & 2.50 &      e\\
 \object{Abell 754} & \dataset[ADS/Sa.CXO#obs/577]{577} & 09:09:18.188 & -09:41:09.56 & 44.2 & I3 & 0.0543 & 9.94 & \nodata\\
 \object{Abell 773} & \dataset[ADS/Sa.CXO#obs/5006]{5006} & 09:17:52.566 & +51:43:38.18 & 19.8 & I3 & 0.2170 & 7.83 & \nodata\\
 \object{Abell 907} & \dataset[ADS/Sa.CXO#obs/3185]{3185} & 09:58:21.946 & -11:03:50.73 & 48.0 & I3 & 0.1527 & 5.59 & \nodata\\
 & \dataset[ADS/Sa.CXO#obs/3205]{3205} & \nodata & \nodata & 47.1 & I3 & \nodata & \nodata & \nodata\\
 & \dataset[ADS/Sa.CXO#obs/535]{535} & \nodata & \nodata & 11.0 & I3 & \nodata & \nodata & \nodata\\
 \object{Abell 963} & \dataset[ADS/Sa.CXO#obs/903]{903} & 10:17:03.744 & +39:02:49.17 & 36.3 & S3 & 0.2056 & 6.73 & \nodata\\
 \object{Abell 1060} & \dataset[ADS/Sa.CXO#obs/2220]{2220} & 10:36:42.828 & -27:31:42.06 & 31.9 & I3 & 0.0125 & 3.29 &  a,e,f\\
 \object{Abell 1063S} & \dataset[ADS/Sa.CXO#obs/4966]{4966} & 22:48:44.294 & -44:31:48.37 & 26.7 & I3 & 0.3540 & 11.96 & \nodata\\
 \object{Abell 1068} & \dataset[ADS/Sa.CXO#obs/1652]{1652} & 10:40:44.520 & +39:57:10.28 & 26.8 & S3 & 0.1375 & 4.62 & \nodata\\
 \object{Abell 1201} & \dataset[ADS/Sa.CXO#obs/4216]{4216} & 11:12:54.489 & +13:26:08.76 & 39.7 & S3 & 0.1688 & 5.61 & \nodata\\
 \object{Abell 1204} & \dataset[ADS/Sa.CXO#obs/2205]{2205} & 11:13:20.419 & +17:35:38.45 & 23.6 & I3 & 0.1706 & 3.63 & \nodata\\
 \object{Abell 1240} & \dataset[ADS/Sa.CXO#obs/4961]{4961} & 11:23:38.357 & +43:05:48.33 & 51.3 & I3 & 0.1590 & 4.77 &      a\\
 \object{Abell 1361} & \dataset[ADS/Sa.CXO#obs/2200]{2200} & 11:43:39.637 & +46:21:20.41 & 16.7 & S3 & 0.1171 & 5.32 & \nodata\\
 \object{Abell 1413} & \dataset[ADS/Sa.CXO#obs/5003]{5003} & 11:55:17.893 & +23:24:21.84 & 75.1 & I2 & 0.1426 & 7.41 & \nodata\\
 \object{Abell 1423} & \dataset[ADS/Sa.CXO#obs/538]{538} & 11:57:17.263 & +33:36:37.44 & 9.8 & I3 & 0.2130 & 6.01 & \nodata\\
 \object{Abell 1446} & \dataset[ADS/Sa.CXO#obs/4975]{4975} & 12:02:03.744 & +58:02:17.93 & 58.4 & S3 & 0.1035 & 3.96 & \nodata\\
 \object{Abell 1569} & \dataset[ADS/Sa.CXO#obs/6100]{6100} & 12:36:26.015 & +16:32:17.81 & 41.2 & I3 & 0.0735 & 2.51 & \nodata\\
 \object{Abell 1576} & \dataset[ADS/Sa.CXO#obs/7938]{7938} & 12:36:58.274 & +63:11:13.88 & 15.0 & I3 & 0.2790 & 10.10 & \nodata\\
 \object{Abell 1644} & \dataset[ADS/Sa.CXO#obs/2206]{2206} & 12:57:11.665 & -17:24:32.86 & 18.7 & I3 & 0.0471 & 4.60 &      b\\
 & \dataset[ADS/Sa.CXO#obs/7922]{7922} & \nodata & \nodata & 51.5 & I3 & \nodata & \nodata & \nodata\\
 \object{Abell 1650} & \dataset[ADS/Sa.CXO#obs/4178]{4178} & 12:58:41.499 & -01:45:44.32 & 27.3 & S3 & 0.0843 & 6.17 & \nodata\\
 \object{Abell 1651} & \dataset[ADS/Sa.CXO#obs/4185]{4185} & 12:59:22.830 & -04:11:45.86 & 9.6 & I3 & 0.0840 & 6.26 & \nodata\\
 \object{Abell 1664} & \dataset[ADS/Sa.CXO#obs/1648]{1648} & 13:03:42.622 & -24:14:41.59 & 9.8 & S3 & 0.1276 & 4.39 & \nodata\\
 & \dataset[ADS/Sa.CXO#obs/7901]{7901} & \nodata & \nodata & 36.6 & S3 & \nodata & \nodata & \nodata\\
 \object{Abell 1689} & \dataset[ADS/Sa.CXO#obs/1663]{1663} & 13:11:29.612 & -01:20:28.69 & 10.7 & I3 & 0.1843 & 10.10 & \nodata\\
 & \dataset[ADS/Sa.CXO#obs/5004]{5004} & \nodata & \nodata & 19.9 & I3 & \nodata & \nodata & \nodata\\
 & \dataset[ADS/Sa.CXO#obs/540]{540} & \nodata & \nodata & 10.3 & I3 & \nodata & \nodata & \nodata\\
 \object{Abell 1736} & \dataset[ADS/Sa.CXO#obs/4186]{4186} & 13:26:49.453 & -27:09:48.13 & 14.9 & I1 & 0.0338 & 3.45 &    a,e\\
 \object{Abell 1758} & \dataset[ADS/Sa.CXO#obs/2213]{2213} & 13:32:48.398 & +50:32:32.53 & 58.3 & S3 & 0.2792 & 12.14 & \nodata\\
 \object{Abell 1763} & \dataset[ADS/Sa.CXO#obs/3591]{3591} & 13:35:17.957 & +40:59:55.80 & 19.6 & I3 & 0.1866 & 7.78 & \nodata\\
 \object{Abell 1795} & \dataset[ADS/Sa.CXO#obs/493]{493} & 13:48:52.802 & +26:35:23.55 & 19.6 & S3 & 0.0625 & 7.80 & \nodata\\
 & \dataset[ADS/Sa.CXO#obs/5289]{5289} & \nodata & \nodata & 15.0 & I3 & \nodata & \nodata & \nodata\\
 \object{Abell 1835} & \dataset[ADS/Sa.CXO#obs/495]{495} & 14:01:01.951 & +02:52:43.18 & 19.5 & S3 & 0.2532 & 9.77 & \nodata\\
 \object{Abell 1914} & \dataset[ADS/Sa.CXO#obs/3593]{3593} & 14:26:03.060 & +37:49:27.84 & 18.9 & I3 & 0.1712 & 9.62 & \nodata\\
 \object{Abell 1942} & \dataset[ADS/Sa.CXO#obs/3290]{3290} & 14:38:21.878 & +03:40:12.97 & 57.6 & I2 & 0.2240 & 4.77 & \nodata\\
 \object{Abell 1991} & \dataset[ADS/Sa.CXO#obs/3193]{3193} & 14:54:31.620 & +18:38:41.48 & 38.3 & S3 & 0.0587 & 2.67 & \nodata\\
 \object{Abell 1995} & \dataset[ADS/Sa.CXO#obs/7021]{7021} & 14:52:57.410 & +58:02:56.84 & 48.5 & I3 & 0.3186 & 3.40 & \nodata\\
 \object{Abell 2029} & \dataset[ADS/Sa.CXO#obs/4977]{4977} & 15:10:56.139 & +05:44:40.47 & 77.9 & S3 & 0.0765 & 7.38 & \nodata\\
 & \dataset[ADS/Sa.CXO#obs/6101]{6101} & \nodata & \nodata & 9.9 & I3 & \nodata & \nodata & \nodata\\
 & \dataset[ADS/Sa.CXO#obs/891]{891} & \nodata & \nodata & 19.8 & S3 & \nodata & \nodata & \nodata\\
 \object{Abell 2034} & \dataset[ADS/Sa.CXO#obs/2204]{2204} & 15:10:12.498 & +33:30:39.57 & 53.9 & I3 & 0.1130 & 7.15 &      f\\
 \object{Abell 2052} & \dataset[ADS/Sa.CXO#obs/5807]{5807} & 15:16:44.514 & +07:01:17.02 & 127.0 & S3 & 0.0353 & 2.98 &      d\\
 & \dataset[ADS/Sa.CXO#obs/890]{890} & \nodata & \nodata & 36.8 & S3 & \nodata & \nodata & \nodata\\
 \object{Abell 2063} & \dataset[ADS/Sa.CXO#obs/4187]{4187} & 15:23:04.851 & +08:36:20.16 & 8.8 & I3 & 0.0351 & 3.61 & \nodata\\
 & \dataset[ADS/Sa.CXO#obs/6263]{6263} & \nodata & \nodata & 16.8 & S3 & \nodata & \nodata & \nodata\\
 \object{Abell 2065} & \dataset[ADS/Sa.CXO#obs/31821]{31821} & 15:22:29.517 & +27:42:22.93 & 27.7 & I3 & 0.0730 & 5.75 & \nodata\\
 \object{Abell 2069} & \dataset[ADS/Sa.CXO#obs/4965]{4965} & 15:24:11.376 & +29:52:19.02 & 55.4 & I2 & 0.1160 & 6.50 & \nodata\\
 \object{Abell 2104} & \dataset[ADS/Sa.CXO#obs/895]{895} & 15:40:08.131 & -03:18:15.02 & 49.2 & S3 & 0.1554 & 8.53 & \nodata\\
 \object{Abell 2107} & \dataset[ADS/Sa.CXO#obs/4960]{4960} & 15:39:39.113 & +21:46:57.66 & 35.6 & I3 & 0.0411 & 3.82 &      b\\
 \object{Abell 2111} & \dataset[ADS/Sa.CXO#obs/544]{544} & 15:39:40.637 & +34:25:28.01 & 10.3 & I3 & 0.2300 & 7.13 & \nodata\\
 \object{Abell 2124} & \dataset[ADS/Sa.CXO#obs/3238]{3238} & 15:44:59.131 & +36:06:34.11 & 19.4 & S3 & 0.0658 & 4.73 & \nodata\\
 \object{Abell 2125} & \dataset[ADS/Sa.CXO#obs/2207]{2207} & 15:41:14.154 & +66:15:57.20 & 81.5 & I3 & 0.2465 & 2.88 &      a\\
 \object{Abell 2142} & \dataset[ADS/Sa.CXO#obs/1196]{1196} & 15:58:20.880 & +27:13:44.21 & 11.4 & S3 & 0.0898 & 8.24 & \nodata\\
 & \dataset[ADS/Sa.CXO#obs/1228]{1228} & \nodata & \nodata & 12.1 & S3 & \nodata & \nodata & \nodata\\
 & \dataset[ADS/Sa.CXO#obs/5005]{5005} & \nodata & \nodata & 44.6 & I3 & \nodata & \nodata & \nodata\\
 \object{Abell 2147} & \dataset[ADS/Sa.CXO#obs/3211]{3211} & 16:02:17.025 & +15:58:28.32 & 17.9 & I3 & 0.0356 & 4.09 & \nodata\\
 \object{Abell 2151} & \dataset[ADS/Sa.CXO#obs/4996]{4996} & 16:04:35.887 & +17:43:17.36 & 21.8 & I3 & 0.0366 & 2.90 &      e\\
 \object{Abell 2163} & \dataset[ADS/Sa.CXO#obs/1653]{1653} & 16:15:45.705 & -06:09:00.62 & 71.1 & I1 & 0.1695 & 19.20 & \nodata\\
 \object{Abell 2199} & \dataset[ADS/Sa.CXO#obs/497]{497} & 16:28:38.249 & +39:33:04.28 & 19.5 & S3 & 0.0300 & 4.55 &      b\\
 \object{Abell 2204} & \dataset[ADS/Sa.CXO#obs/499]{499} & 16:32:46.920 & +05:34:32.86 & 10.1 & S3 & 0.1524 & 6.97 & \nodata\\
 & \dataset[ADS/Sa.CXO#obs/6104]{6104} & \nodata & \nodata & 9.6 & I3 & \nodata & \nodata & \nodata\\
 & \dataset[ADS/Sa.CXO#obs/7940]{7940} & \nodata & \nodata & 77.1 & I3 & \nodata & \nodata & \nodata\\
 \object{Abell 2218} & \dataset[ADS/Sa.CXO#obs/1666]{1666} & 16:35:50.831 & +66:12:42.31 & 48.6 & I0 & 0.1713 & 7.35 & \nodata\\
 \object{Abell 2219} & \dataset[ADS/Sa.CXO#obs/896]{896} & 16:40:20.112 & +46:42:42.84 & 42.3 & S3 & 0.2256 & 12.75 & \nodata\\
 \object{Abell 2244} & \dataset[ADS/Sa.CXO#obs/4179]{4179} & 17:02:42.579 & +34:03:37.34 & 57.0 & S3 & 0.0967 & 5.68 & \nodata\\
 \object{Abell 2255} & \dataset[ADS/Sa.CXO#obs/894]{894} & 17:12:42.935 & +64:04:10.81 & 39.4 & I3 & 0.0805 & 6.12 &      a\\
 \object{Abell 2256} & \dataset[ADS/Sa.CXO#obs/1386]{1386} & 17:03:44.567 & +78:38:11.51 & 12.4 & I3 & 0.0579 & 6.90 &      a\\
 \object{Abell 2259} & \dataset[ADS/Sa.CXO#obs/3245]{3245} & 17:20:08.299 & +27:40:11.53 & 10.0 & I3 & 0.1640 & 5.18 & \nodata\\
 \object{Abell 2261} & \dataset[ADS/Sa.CXO#obs/5007]{5007} & 17:22:27.254 & +32:07:58.60 & 24.3 & I3 & 0.2240 & 7.63 & \nodata\\
 \object{Abell 2294} & \dataset[ADS/Sa.CXO#obs/3246]{3246} & 17:24:10.149 & +85:53:09.77 & 10.0 & I3 & 0.1780 & 9.98 & \nodata\\
 \object{Abell 2319} & \dataset[ADS/Sa.CXO#obs/3231]{3231} & 19:21:09.638 & +43:57:21.53 & 14.4 & I1 & 0.0562 & 10.87 &      a\\
 \object{Abell 2384} & \dataset[ADS/Sa.CXO#obs/4202]{4202} & 21:52:21.178 & -19:32:51.90 & 31.5 & I3 & 0.0945 & 4.75 & \nodata\\
 \object{Abell 2390} & \dataset[ADS/Sa.CXO#obs/4193]{4193} & 21:53:36.825 & +17:41:44.38 & 95.1 & S3 & 0.2301 & 11.15 & \nodata\\
 \object{Abell 2409} & \dataset[ADS/Sa.CXO#obs/3247]{3247} & 22:00:52.567 & +20:58:06.55 & 10.2 & I3 & 0.1479 & 5.94 & \nodata\\
 \object{Abell 2420} & \dataset[ADS/Sa.CXO#obs/8271]{8271} & 22:10:18.792 & -12:10:13.35 & 8.1 & I3 & 0.0846 & 6.47 & \nodata\\
 \object{Abell 2462} & \dataset[ADS/Sa.CXO#obs/4159]{4159} & 22:39:11.367 & -17:20:28.33 & 39.2 & S3 & 0.0737 & 2.42 &    a,e\\
 \object{Abell 2537} & \dataset[ADS/Sa.CXO#obs/4962]{4962} & 23:08:22.313 & -02:11:29.88 & 36.2 & S3 & 0.2950 & 8.40 & \nodata\\
 \object{Abell 2554} & \dataset[ADS/Sa.CXO#obs/1696]{1696} & 23:12:19.622 & -21:30:11.32 & 19.9 & S3 & 0.1103 & 5.29 & \nodata\\
 \object{Abell 2556} & \dataset[ADS/Sa.CXO#obs/2226]{2226} & 23:13:01.413 & -21:38:04.47 & 19.9 & S3 & 0.0862 & 3.50 & \nodata\\
 \object{Abell 2589} & \dataset[ADS/Sa.CXO#obs/3210]{3210} & 23:23:57.315 & +16:46:38.43 & 13.7 & S3 & 0.0415 & 3.65 & \nodata\\
 \object{Abell 2597} & \dataset[ADS/Sa.CXO#obs/922]{922} & 23:25:19.779 & -12:07:27.63 & 39.4 & S3 & 0.0854 & 4.02 & \nodata\\
 \object{Abell 2626} & \dataset[ADS/Sa.CXO#obs/3192]{3192} & 23:36:30.452 & +21:08:47.36 & 24.8 & S3 & 0.0573 & 3.29 & \nodata\\
 \object{Abell 2631} & \dataset[ADS/Sa.CXO#obs/3248]{3248} & 23:37:38.560 & +00:16:05.02 & 9.2 & I3 & 0.2779 & 7.06 &      a\\
 \object{Abell 2657} & \dataset[ADS/Sa.CXO#obs/4941]{4941} & 23:44:57.253 & +09:11:30.74 & 16.1 & I3 & 0.0402 & 3.77 & \nodata\\
 \object{Abell 2667} & \dataset[ADS/Sa.CXO#obs/2214]{2214} & 23:51:39.395 & -26:05:02.75 & 9.6 & S3 & 0.2300 & 6.75 & \nodata\\
 \object{Abell 2717} & \dataset[ADS/Sa.CXO#obs/6974]{6974} & 00:03:12.968 & -35:56:00.13 & 19.8 & I3 & 0.0475 & 1.69 &      e\\
 \object{Abell 2744} & \dataset[ADS/Sa.CXO#obs/2212]{2212} & 00:14:19.529 & -30:23:30.24 & 24.8 & S3 & 0.3080 & 9.18 & \nodata\\
 & \dataset[ADS/Sa.CXO#obs/7915]{7915} & \nodata & \nodata & 18.6 & I3 & \nodata & \nodata & \nodata\\
 & \dataset[ADS/Sa.CXO#obs/8477]{8477} & \nodata & \nodata & 45.9 & I3 & \nodata & \nodata & \nodata\\
 & \dataset[ADS/Sa.CXO#obs/8557]{8557} & \nodata & \nodata & 27.8 & I3 & \nodata & \nodata & \nodata\\
 \object{Abell 2813} & \dataset[ADS/Sa.CXO#obs/9409]{9409} & 00:43:24.881 & -20:37:25.08 & 19.9 & I3 & 0.2924 & 8.96 & \nodata\\
 \object{Abell 3084} & \dataset[ADS/Sa.CXO#obs/9413]{9413} & 03:04:03.920 & -36:56:27.17 & 19.9 & I3 & 0.0977 & 5.30 & \nodata\\
 \object{Abell 3088} & \dataset[ADS/Sa.CXO#obs/9414]{9414} & 03:07:01.734 & -28:39:55.47 & 18.9 & I3 & 0.2534 & 6.71 & \nodata\\
 \object{Abell 3112} & \dataset[ADS/Sa.CXO#obs/2516]{2516} & 03:17:57.681 & -44:14:17.16 & 16.9 & S3 & 0.0720 & 5.17 &      d\\
 \object{Abell 3120} & \dataset[ADS/Sa.CXO#obs/6951]{6951} & 03:21:56.464 & -51:19:35.40 & 26.8 & I3 & 0.0690 & 4.40 & \nodata\\
 \object{Abell 3158} & \dataset[ADS/Sa.CXO#obs/3201]{3201} & 03:42:54.675 & -53:37:24.36 & 24.8 & I3 & 0.0580 & 4.94 & \nodata\\
 & \dataset[ADS/Sa.CXO#obs/3712]{3712} & \nodata & \nodata & 30.9 & I3 & \nodata & \nodata & \nodata\\
 \object{Abell 3266} & \dataset[ADS/Sa.CXO#obs/899]{899} & 04:31:13.304 & -61:27:12.59 & 29.8 & I1 & 0.0590 & 9.07 & \nodata\\
 \object{Abell 3364} & \dataset[ADS/Sa.CXO#obs/9419]{9419} & 05:47:37.698 & -31:52:23.61 & 19.8 & I3 & 0.1483 & 7.88 & \nodata\\
 \object{Abell 3376} & \dataset[ADS/Sa.CXO#obs/3202]{3202} & 06:02:11.756 & -39:56:59.07 & 44.3 & I3 & 0.0456 & 4.08 &      a\\
 & \dataset[ADS/Sa.CXO#obs/3450]{3450} & \nodata & \nodata & 19.8 & I3 & \nodata & \nodata & \nodata\\
 \object{Abell 3391} & \dataset[ADS/Sa.CXO#obs/4943]{4943} & 06:26:21.511 & -53:41:44.81 & 18.4 & I3 & 0.0560 & 6.07 &    a,e\\
 \object{Abell 3395} & \dataset[ADS/Sa.CXO#obs/4944]{4944} & 06:26:48.463 & -54:32:59.21 & 21.9 & I3 & 0.0510 & 5.13 &    a,e\\
 \object{Abell 3528S} & \dataset[ADS/Sa.CXO#obs/8268]{8268} & 12:54:40.897 & -29:13:38.10 & 8.1 & I3 & 0.0530 & 5.44 & \nodata\\
 \object{Abell 3558} & \dataset[ADS/Sa.CXO#obs/1646]{1646} & 13:27:56.854 & -31:29:43.78 & 14.4 & S3 & 0.0480 & 6.60 &    e,f\\
 \object{Abell 3562} & \dataset[ADS/Sa.CXO#obs/4167]{4167} & 13:33:37.800 & -31:40:12.04 & 19.3 & I2 & 0.0490 & 4.59 & \nodata\\
 \object{Abell 3571} & \dataset[ADS/Sa.CXO#obs/4203]{4203} & 13:47:28.434 & -32:51:52.45 & 34.0 & S3 & 0.0391 & 7.77 & \nodata\\
 \object{Abell 3581} & \dataset[ADS/Sa.CXO#obs/1650]{1650} & 14:07:29.777 & -27:01:05.88 & 7.2 & S3 & 0.0218 & 2.10 &      d\\
 \object{Abell 3667} & \dataset[ADS/Sa.CXO#obs/5751]{5751} & 20:12:41.231 & -56:50:35.70 & 128.9 & I3 & 0.0556 & 6.51 & \nodata\\
 & \dataset[ADS/Sa.CXO#obs/5752]{5752} & \nodata & \nodata & 60.4 & I3 & \nodata & \nodata & \nodata\\
 & \dataset[ADS/Sa.CXO#obs/5753]{5753} & \nodata & \nodata & 103.6 & I3 & \nodata & \nodata & \nodata\\
 & \dataset[ADS/Sa.CXO#obs/889]{889} & \nodata & \nodata & 50.3 & I2 & \nodata & \nodata & \nodata\\
 \object{Abell 3822} & \dataset[ADS/Sa.CXO#obs/8269]{8269} & 21:54:04.203 & -57:52:02.71 & 8.1 & I3 & 0.0759 & 4.89 &      e\\
 \object{Abell 3827} & \dataset[ADS/Sa.CXO#obs/7920]{7920} & 22:01:53.200 & -59:56:43.04 & 45.6 & S3 & 0.0984 & 8.05 & \nodata\\
 \object{Abell 3921} & \dataset[ADS/Sa.CXO#obs/4973]{4973} & 22:49:57.829 & -64:25:42.17 & 29.4 & I3 & 0.0927 & 5.69 & \nodata\\
 \object{Abell 4038} & \dataset[ADS/Sa.CXO#obs/4992]{4992} & 23:47:43.180 & -28:08:34.81 & 33.5 & I2 & 0.0300 & 3.11 & \nodata\\
 \object{Abell 4059} & \dataset[ADS/Sa.CXO#obs/5785]{5785} & 23:57:01.065 & -34:45:33.28 & 92.1 & S3 & 0.0475 & 4.69 & \nodata\\
 \object{Abell S0405} & \dataset[ADS/Sa.CXO#obs/8272]{8272} & 03:51:32.815 & -82:13:10.19 & 7.9 & I3 & 0.0613 & 4.11 & \nodata\\
 \object{Abell S0592} & \dataset[ADS/Sa.CXO#obs/9420]{9420} & 06:38:48.610 & -53:58:26.32 & 19.9 & I3 & 0.2216 & 9.08 & \nodata\\
 \object{AC 114} & \dataset[ADS/Sa.CXO#obs/1562]{1562} & 22:58:48.316 & -34:48:08.20 & 72.5 & S3 & 0.3120 & 7.53 & \nodata\\
 \object{AWM7} & \dataset[ADS/Sa.CXO#obs/908]{908} & 02:54:27.631 & +41:34:47.07 & 47.9 & I3 & 0.0172 & 3.71 &      b\\
 \object{Centaurus} & \dataset[ADS/Sa.CXO#obs/4190]{4190} & 12:48:49.267 & -41:18:39.54 & 34.3 & S3 & 0.0109 & 3.96 &      b\\
 & \dataset[ADS/Sa.CXO#obs/4191]{4191} & \nodata & \nodata & 34.0 & S3 & \nodata & \nodata & \nodata\\
 & \dataset[ADS/Sa.CXO#obs/4954]{4954} & \nodata & \nodata & 89.1 & S3 & \nodata & \nodata & \nodata\\
 & \dataset[ADS/Sa.CXO#obs/4955]{4955} & \nodata & \nodata & 44.7 & S3 & \nodata & \nodata & \nodata\\
 & \dataset[ADS/Sa.CXO#obs/504]{504} & \nodata & \nodata & 31.8 & S3 & \nodata & \nodata & \nodata\\
 & \dataset[ADS/Sa.CXO#obs/505]{505} & \nodata & \nodata & 10.0 & S3 & \nodata & \nodata & \nodata\\
 & \dataset[ADS/Sa.CXO#obs/5310]{5310} & \nodata & \nodata & 49.3 & S3 & \nodata & \nodata & \nodata\\
 \object{CID 72} & \dataset[ADS/Sa.CXO#obs/2018]{2018} & 17:33:03.247 & +43:45:37.28 & 30.7 & S3 & 0.0344 & 1.91 & \nodata\\
 & \dataset[ADS/Sa.CXO#obs/6949]{6949} & \nodata & \nodata & 38.6 & I3 & \nodata & \nodata & \nodata\\
 & \dataset[ADS/Sa.CXO#obs/7321]{7321} & \nodata & \nodata & 37.5 & I3 & \nodata & \nodata & \nodata\\
 & \dataset[ADS/Sa.CXO#obs/7322]{7322} & \nodata & \nodata & 37.5 & I3 & \nodata & \nodata & \nodata\\
 \object{CL J1226.9+3332} & \dataset[ADS/Sa.CXO#obs/3180]{3180} & 12:26:58.373 & +33:32:47.36 & 31.7 & I3 & 0.8900 & 10.00 & \nodata\\
 & \dataset[ADS/Sa.CXO#obs/5014]{5014} & \nodata & \nodata & 32.7 & I3 & \nodata & \nodata & \nodata\\
 & \dataset[ADS/Sa.CXO#obs/932]{932} & \nodata & \nodata & 9.8 & S3 & \nodata & \nodata & \nodata\\
 \object{Cygnus A} & \dataset[ADS/Sa.CXO#obs/360]{360} & 19:59:28.381 & +40:44:01.98 & 34.7 & S3 & 0.0561 & 7.68 &      d\\
 \object{ESO 3060170} & \dataset[ADS/Sa.CXO#obs/3188]{3188} & 05:40:06.687 & -40:50:12.82 & 14.0 & I3 & 0.0358 & 2.79 &      b\\
 & \dataset[ADS/Sa.CXO#obs/3189]{3189} & \nodata & \nodata & 14.1 & I0 & \nodata & \nodata & \nodata\\
 \object{ESO 5520200} & \dataset[ADS/Sa.CXO#obs/3206]{3206} & 04:54:52.318 & -18:06:56.52 & 23.9 & I3 & 0.0314 & 2.37 & \nodata\\
 \object{EXO 422-086} & \dataset[ADS/Sa.CXO#obs/4183]{4183} & 04:25:51.271 & -08:33:36.42 & 10.0 & I3 & 0.0397 & 3.40 & \nodata\\
 \object{HCG 62} & \dataset[ADS/Sa.CXO#obs/921]{921} & 12:53:05.741 & -09:12:15.64 & 48.5 & S3 & 0.0146 & 1.10 & \nodata\\
 \object{HCG 42} & \dataset[ADS/Sa.CXO#obs/3215]{3215} & 10:00:14.234 & -19:38:10.77 & 31.7 & S3 & 0.0133 & 0.70 & \nodata\\
 \object{Hercules A} & \dataset[ADS/Sa.CXO#obs/1625]{1625} & 16:51:08.161 & +04:59:32.44 & 14.8 & S3 & 0.1541 & 5.21 & \nodata\\
 & \dataset[ADS/Sa.CXO#obs/5796]{5796} & \nodata & \nodata & 47.5 & S3 & \nodata & \nodata & \nodata\\
 & \dataset[ADS/Sa.CXO#obs/6257]{6257} & \nodata & \nodata & 49.5 & S3 & \nodata & \nodata & \nodata\\
 \object{Hydra A} & \dataset[ADS/Sa.CXO#obs/4970]{4970} & 09:18:05.985 & -12:05:43.94 & 98.8 & S3 & 0.0549 & 4.00 &      d\\
 & \dataset[ADS/Sa.CXO#obs/576]{576} & \nodata & \nodata & 19.5 & S3 & \nodata & \nodata & \nodata\\
 \object{M49} & \dataset[ADS/Sa.CXO#obs/321]{321} & 12:29:46.841 & +08:00:01.98 & 39.6 & S3 & 0.0033 & 1.33 &      c\\
 \object{M87} & \dataset[ADS/Sa.CXO#obs/5826]{5826} & 12:30:49.383 & +12:23:28.67 & 126.8 & I3 & 0.0044 & 2.50 &      d\\
 & \dataset[ADS/Sa.CXO#obs/5827]{5827} & \nodata & \nodata & 156.2 & I3 & \nodata & \nodata & \nodata\\
 \object{MACS J0011.7-1523} & \dataset[ADS/Sa.CXO#obs/3261]{3261} & 00:11:42.965 & -15:23:20.79 & 21.6 & I3 & 0.3600 & 5.42 & \nodata\\
 & \dataset[ADS/Sa.CXO#obs/6105]{6105} & \nodata & \nodata & 37.3 & I3 & \nodata & \nodata & \nodata\\
 \object{MACS J0035.4-2015} & \dataset[ADS/Sa.CXO#obs/3262]{3262} & 00:35:26.573 & -20:15:46.06 & 21.4 & I3 & 0.3644 & 7.39 & \nodata\\
 \object{MACS J0159.8-0849} & \dataset[ADS/Sa.CXO#obs/3265]{3265} & 01:59:49.453 & -08:50:00.90 & 17.9 & I3 & 0.4050 & 9.59 & \nodata\\
 & \dataset[ADS/Sa.CXO#obs/6106]{6106} & \nodata & \nodata & 35.3 & I3 & \nodata & \nodata & \nodata\\
 \object{MACS J0242.5-2132} & \dataset[ADS/Sa.CXO#obs/3266]{3266} & 02:42:35.906 & -21:32:26.30 & 11.9 & I3 & 0.3140 & 5.58 & \nodata\\
 \object{MACS J0257.1-2325} & \dataset[ADS/Sa.CXO#obs/1654]{1654} & 02:57:09.130 & -23:26:05.85 & 19.8 & I3 & 0.5053 & 10.50 & \nodata\\
 & \dataset[ADS/Sa.CXO#obs/3581]{3581} & \nodata & \nodata & 18.5 & I3 & \nodata & \nodata & \nodata\\
 \object{MACS J0257.6-2209} & \dataset[ADS/Sa.CXO#obs/3267]{3267} & 02:57:41.024 & -22:09:11.12 & 20.5 & I3 & 0.3224 & 8.02 & \nodata\\
 \object{MACS J0308.9+2645} & \dataset[ADS/Sa.CXO#obs/3268]{3268} & 03:08:55.927 & +26:45:38.34 & 24.4 & I3 & 0.3240 & 10.54 & \nodata\\
 \object{MACS J0329.6-0211} & \dataset[ADS/Sa.CXO#obs/3257]{3257} & 03:29:41.681 & -02:11:47.67 & 9.9 & I3 & 0.4500 & 5.20 & \nodata\\
 & \dataset[ADS/Sa.CXO#obs/3582]{3582} & \nodata & \nodata & 19.9 & I3 & \nodata & \nodata & \nodata\\
 & \dataset[ADS/Sa.CXO#obs/6108]{6108} & \nodata & \nodata & 39.6 & I3 & \nodata & \nodata & \nodata\\
 \object{MACS J0417.5-1154} & \dataset[ADS/Sa.CXO#obs/3270]{3270} & 04:17:34.686 & -11:54:32.71 & 12.0 & I3 & 0.4400 & 11.07 & \nodata\\
 \object{MACS J0429.6-0253} & \dataset[ADS/Sa.CXO#obs/3271]{3271} & 04:29:36.088 & -02:53:09.02 & 23.2 & I3 & 0.3990 & 5.66 & \nodata\\
 \object{MACS J0520.7-1328} & \dataset[ADS/Sa.CXO#obs/3272]{3272} & 05:20:42.052 & -13:28:49.38 & 19.2 & I3 & 0.3398 & 6.27 & \nodata\\
 \object{MACS J0547.0-3904} & \dataset[ADS/Sa.CXO#obs/3273]{3273} & 05:47:01.582 & -39:04:28.24 & 21.7 & I3 & 0.2100 & 3.58 &      e\\
 \object{MACS J0717.5+3745} & \dataset[ADS/Sa.CXO#obs/1655]{1655} & 07:17:31.654 & +37:45:18.52 & 19.9 & I3 & 0.5480 & 10.50 & \nodata\\
 & \dataset[ADS/Sa.CXO#obs/4200]{4200} & \nodata & \nodata & 59.2 & I3 & \nodata & \nodata & \nodata\\
 \object{MACS J0744.8+3927} & \dataset[ADS/Sa.CXO#obs/3197]{3197} & 07:44:52.802 & +39:27:24.41 & 20.2 & I3 & 0.6860 & 11.29 & \nodata\\
 & \dataset[ADS/Sa.CXO#obs/3585]{3585} & \nodata & \nodata & 19.9 & I3 & \nodata & \nodata & \nodata\\
 & \dataset[ADS/Sa.CXO#obs/6111]{6111} & \nodata & \nodata & 49.5 & I3 & \nodata & \nodata & \nodata\\
 \object{MACS J1115.2+5320} & \dataset[ADS/Sa.CXO#obs/3253]{3253} & 11:15:15.632 & +53:20:03.31 & 8.8 & I3 & 0.4390 & 8.03 & \nodata\\
 & \dataset[ADS/Sa.CXO#obs/5008]{5008} & \nodata & \nodata & 18.0 & I3 & \nodata & \nodata & \nodata\\
 & \dataset[ADS/Sa.CXO#obs/5350]{5350} & \nodata & \nodata & 6.9 & I3 & \nodata & \nodata & \nodata\\
 \object{MACS J1115.8+0129} & \dataset[ADS/Sa.CXO#obs/3275]{3275} & 11:15:52.048 & +01:29:56.56 & 15.9 & I3 & 0.1200 & 6.78 & \nodata\\
 \object{MACS J1131.8-1955} & \dataset[ADS/Sa.CXO#obs/3276]{3276} & 11:31:54.580 & -19:55:44.54 & 13.9 & I3 & 0.3070 & 8.64 & \nodata\\
 \object{MACS J1149.5+2223} & \dataset[ADS/Sa.CXO#obs/1656]{1656} & 11:49:35.856 & +22:23:55.02 & 18.5 & I3 & 0.5440 & 8.40 & \nodata\\
 & \dataset[ADS/Sa.CXO#obs/3589]{3589} & \nodata & \nodata & 20.0 & I3 & \nodata & \nodata & \nodata\\
 \object{MACS J1206.2-0847} & \dataset[ADS/Sa.CXO#obs/3277]{3277} & 12:06:12.276 & -08:48:02.40 & 23.5 & I3 & 0.4400 & 10.21 & \nodata\\
 \object{MACS J1311.0-0310} & \dataset[ADS/Sa.CXO#obs/3258]{3258} & 13:11:01.665 & -03:10:39.50 & 14.9 & I3 & 0.4940 & 5.60 & \nodata\\
 & \dataset[ADS/Sa.CXO#obs/6110]{6110} & \nodata & \nodata & 63.2 & I3 & \nodata & \nodata & \nodata\\
 \object{MACS J1621.3+3810} & \dataset[ADS/Sa.CXO#obs/3254]{3254} & 16:21:24.801 & +38:10:08.65 & 9.8 & I3 & 0.4610 & 7.53 & \nodata\\
 & \dataset[ADS/Sa.CXO#obs/3594]{3594} & \nodata & \nodata & 19.7 & I3 & \nodata & \nodata & \nodata\\
 & \dataset[ADS/Sa.CXO#obs/6109]{6109} & \nodata & \nodata & 37.5 & I3 & \nodata & \nodata & \nodata\\
 & \dataset[ADS/Sa.CXO#obs/6172]{6172} & \nodata & \nodata & 29.8 & I3 & \nodata & \nodata & \nodata\\
 \object{MACS J1931.8-2634} & \dataset[ADS/Sa.CXO#obs/3282]{3282} & 19:31:49.656 & -26:34:33.99 & 13.6 & I3 & 0.3520 & 6.97 &      e\\
 \object{MACS J2049.9-3217} & \dataset[ADS/Sa.CXO#obs/3283]{3283} & 20:49:56.245 & -32:16:52.30 & 23.8 & I3 & 0.3254 & 6.98 & \nodata\\
 \object{MACS J2211.7-0349} & \dataset[ADS/Sa.CXO#obs/3284]{3284} & 22:11:45.856 & -03:49:37.24 & 17.7 & I3 & 0.2700 & 11.30 & \nodata\\
 \object{MACS J2214.9-1359} & \dataset[ADS/Sa.CXO#obs/3259]{3259} & 22:14:57.467 & -14:00:09.35 & 19.5 & I3 & 0.5026 & 8.80 & \nodata\\
 & \dataset[ADS/Sa.CXO#obs/5011]{5011} & \nodata & \nodata & 18.5 & I3 & \nodata & \nodata & \nodata\\
 \object{MACS J2228+2036} & \dataset[ADS/Sa.CXO#obs/3285]{3285} & 22:28:33.872 & +20:37:18.31 & 19.9 & I3 & 0.4120 & 7.86 & \nodata\\
 \object{MACS J2229.7-2755} & \dataset[ADS/Sa.CXO#obs/3286]{3286} & 22:29:45.358 & -27:55:38.41 & 16.4 & I3 & 0.3240 & 5.01 & \nodata\\
 \object{MACS J2245.0+2637} & \dataset[ADS/Sa.CXO#obs/3287]{3287} & 22:45:04.657 & +26:38:03.46 & 16.9 & I3 & 0.3040 & 6.06 & \nodata\\
 \object{MKW3S} & \dataset[ADS/Sa.CXO#obs/900]{900} & 15:21:51.930 & +07:42:31.97 & 57.3 & I3 & 0.0450 & 2.18 & \nodata\\
 \object{MKW 4} & \dataset[ADS/Sa.CXO#obs/3234]{3234} & 12:04:27.218 & +01:53:42.79 & 30.0 & S3 & 0.0198 & 2.06 & \nodata\\
 \object{MKW 8} & \dataset[ADS/Sa.CXO#obs/4942]{4942} & 14:40:39.633 & +03:28:13.61 & 23.1 & I3 & 0.0270 & 3.29 &    a,b\\
 \object{MS J0016.9+1609} & \dataset[ADS/Sa.CXO#obs/520]{520} & 00:18:33.503 & +16:26:12.99 & 67.4 & I3 & 0.5410 & 8.94 & \nodata\\
 \object{MS J0116.3-0115} & \dataset[ADS/Sa.CXO#obs/4963]{4963} & 01:18:53.944 & -01:00:07.54 & 39.3 & S3 & 0.0452 & 1.84 & \nodata\\
 \object{MS J0440.5+0204} & \dataset[ADS/Sa.CXO#obs/4196]{4196} & 04:43:09.952 & +02:10:18.70 & 59.4 & S3 & 0.1900 & 5.46 & \nodata\\
 \object{MS J0451.6-0305} & \dataset[ADS/Sa.CXO#obs/902]{902} & 04:54:11.004 & -03:00:52.19 & 44.2 & S3 & 0.5386 & 8.90 & \nodata\\
 \object{MS J0735.6+7421} & \dataset[ADS/Sa.CXO#obs/4197]{4197} & 07:41:44.245 & +74:14:38.23 & 45.5 & S3 & 0.2160 & 5.55 & \nodata\\
 \object{MS J0839.8+2938} & \dataset[ADS/Sa.CXO#obs/2224]{2224} & 08:42:55.969 & +29:27:26.97 & 29.8 & S3 & 0.1940 & 4.68 & \nodata\\
 \object{MS J0906.5+1110} & \dataset[ADS/Sa.CXO#obs/924]{924} & 09:09:12.753 & +10:58:32.00 & 29.7 & I3 & 0.1630 & 5.38 & \nodata\\
 \object{MS J1006.0+1202} & \dataset[ADS/Sa.CXO#obs/925]{925} & 10:08:47.462 & +11:47:36.31 & 29.4 & I3 & 0.2210 & 5.61 & \nodata\\
 \object{MS J1008.1-1224} & \dataset[ADS/Sa.CXO#obs/926]{926} & 10:10:32.312 & -12:39:56.80 & 44.2 & I3 & 0.3010 & 7.45 & \nodata\\
 \object{MS J1455.0+2232} & \dataset[ADS/Sa.CXO#obs/4192]{4192} & 14:57:15.088 & +22:20:32.49 & 91.9 & I3 & 0.2590 & 4.77 & \nodata\\
 \object{MS J2137.3-2353} & \dataset[ADS/Sa.CXO#obs/4974]{4974} & 21:40:15.178 & -23:39:40.71 & 57.4 & S3 & 0.3130 & 6.01 & \nodata\\
 \object{MS J1157.3+5531} & \dataset[ADS/Sa.CXO#obs/4964]{4964} & 11:59:52.295 & +55:32:05.61 & 75.1 & S3 & 0.0810 & 3.28 &      b\\
 \object{NGC 507} & \dataset[ADS/Sa.CXO#obs/2882]{2882} & 01:23:39.905 & +33:15:21.73 & 43.6 & I3 & 0.0164 & 1.40 &      c\\
 \object{NGC 4636} & \dataset[ADS/Sa.CXO#obs/3926]{3926} & 12:42:49.856 & +02:41:15.86 & 74.7 & I3 & 0.0031 & 0.66 &      c\\
 & \dataset[ADS/Sa.CXO#obs/4415]{4415} & \nodata & \nodata & 74.4 & I3 & \nodata & \nodata & \nodata\\
 \object{NGC 5044} & \dataset[ADS/Sa.CXO#obs/3225]{3225} & 13:15:23.947 & -16:23:07.62 & 83.1 & S3 & 0.0090 & 1.22 &      c\\
 & \dataset[ADS/Sa.CXO#obs/3664]{3664} & \nodata & \nodata & 61.3 & S3 & \nodata & \nodata & \nodata\\
 \object{NGC 5813} & \dataset[ADS/Sa.CXO#obs/5907]{5907} & 15:01:11.260 & +01:42:07.23 & 48.4 & S3 & 0.0066 & 0.76 &      c\\
 \object{NGC 5846} & \dataset[ADS/Sa.CXO#obs/788]{788} & 15:06:29.289 & +01:36:20.13 & 29.9 & S3 & 0.0057 & 0.64 &      c\\
 \object{Ophiuchus} & \dataset[ADS/Sa.CXO#obs/3200]{3200} & 17:12:27.731 & -23:22:06.74 & 50.5 & S3 & 0.0280 & 11.12 & \nodata\\
 \object{PKS 0745-191} & \dataset[ADS/Sa.CXO#obs/2427]{2427} & 07:47:31.436 & -19:17:39.78 & 17.9 & S3 & 0.1028 & 8.50 & \nodata\\
 & \dataset[ADS/Sa.CXO#obs/508]{508} & \nodata & \nodata & 28.0 & S3 & \nodata & \nodata & \nodata\\
 & \dataset[ADS/Sa.CXO#obs/6103]{6103} & \nodata & \nodata & 10.3 & I3 & \nodata & \nodata & \nodata\\
 \object{RBS 461} & \dataset[ADS/Sa.CXO#obs/4182]{4182} & 03:41:17.490 & +15:23:54.66 & 23.4 & I3 & 0.0290 & 2.60 &    a,e\\
 \object{RBS 533} & \dataset[ADS/Sa.CXO#obs/3186]{3186} & 04:19:38.105 & +02:24:35.54 & 10.0 & I3 & 0.0123 & 1.29 & \nodata\\
 & \dataset[ADS/Sa.CXO#obs/3187]{3187} & \nodata & \nodata & 9.6 & I3 & \nodata & \nodata & \nodata\\
 & \dataset[ADS/Sa.CXO#obs/5800]{5800} & \nodata & \nodata & 44.5 & S3 & \nodata & \nodata & \nodata\\
 & \dataset[ADS/Sa.CXO#obs/5801]{5801} & \nodata & \nodata & 44.4 & S3 & \nodata & \nodata & \nodata\\
 \object{RBS 797} & \dataset[ADS/Sa.CXO#obs/2202]{2202} & 09:47:12.693 & +76:23:13.40 & 11.7 & I3 & 0.3540 & 7.68 &      d\\
 & \dataset[ADS/Sa.CXO#obs/7902]{7902} & \nodata & \nodata & 38.3 & S3 & \nodata & \nodata & \nodata\\
 \object{RCS J2327-0204} & \dataset[ADS/Sa.CXO#obs/7355]{7355} & 23:27:27.524 & -02:04:39.01 & 24.7 & S3 & 0.2000 & 7.06 & \nodata\\
 \object{RX J0220.9-3829} & \dataset[ADS/Sa.CXO#obs/9411]{9411} & 02:20:56.582 & -38:28:51.21 & 19.9 & I3 & 0.2287 & 5.02 & \nodata\\
 \object{RX J0232.2-4420} & \dataset[ADS/Sa.CXO#obs/4993]{4993} & 02:32:18.771 & -44:20:46.68 & 23.4 & I3 & 0.2836 & 7.83 & \nodata\\
 \object{RX J0439+0520} & \dataset[ADS/Sa.CXO#obs/527]{527} & 04:39:02.218 & +05:20:43.11 & 9.6 & I3 & 0.2080 & 4.60 & \nodata\\
 \object{RX J0439.0+0715} & \dataset[ADS/Sa.CXO#obs/1449]{1449} & 04:39:00.710 & +07:16:07.65 & 6.3 & I3 & 0.2300 & 6.50 & \nodata\\
 & \dataset[ADS/Sa.CXO#obs/3583]{3583} & \nodata & \nodata & 19.2 & I3 & \nodata & \nodata & \nodata\\
 \object{RX J0528.9-3927} & \dataset[ADS/Sa.CXO#obs/4994]{4994} & 05:28:53.039 & -39:28:15.53 & 22.5 & I3 & 0.2632 & 7.89 & \nodata\\
 \object{RX J0647.7+7015} & \dataset[ADS/Sa.CXO#obs/3196]{3196} & 06:47:50.029 & +70:14:49.66 & 19.3 & I3 & 0.5840 & 9.07 & \nodata\\
 & \dataset[ADS/Sa.CXO#obs/3584]{3584} & \nodata & \nodata & 20.0 & I3 & \nodata & \nodata & \nodata\\
 \object{RX J0819.6+6336} & \dataset[ADS/Sa.CXO#obs/2199]{2199} & 08:19:26.007 & +63:37:26.53 & 14.9 & S3 & 0.1190 & 3.87 & \nodata\\
 \object{RX J1000.4+4409} & \dataset[ADS/Sa.CXO#obs/9421]{9421} & 10:00:32.024 & +44:08:39.69 & 18.5 & I3 & 0.1540 & 3.42 & \nodata\\
 \object{RX J1022.1+3830} & \dataset[ADS/Sa.CXO#obs/6942]{6942} & 10:22:10.034 & +38:31:23.54 & 41.5 & S3 & 0.0491 & 3.04 &      f\\
 \object{RX J1130.0+3637} & \dataset[ADS/Sa.CXO#obs/6945]{6945} & 11:30:02.789 & +36:38:08.26 & 49.4 & S3 & 0.0600 & 2.00 & \nodata\\
 \object{RX J1320.2+3308} & \dataset[ADS/Sa.CXO#obs/6941]{6941} & 13:20:14.650 & +33:08:33.06 & 38.6 & S3 & 0.0366 & 1.01 &      e\\
 \object{RX J1347.5-1145} & \dataset[ADS/Sa.CXO#obs/3592]{3592} & 13:47:30.593 & -11:45:10.05 & 57.7 & I3 & 0.4510 & 10.88 & \nodata\\
 & \dataset[ADS/Sa.CXO#obs/507]{507} & \nodata & \nodata & 10.0 & S3 & \nodata & \nodata & \nodata\\
 \object{RX J1423.8+2404} & \dataset[ADS/Sa.CXO#obs/1657]{1657} & 14:23:47.759 & +24:04:40.45 & 18.5 & I3 & 0.5450 & 5.92 & \nodata\\
 & \dataset[ADS/Sa.CXO#obs/4195]{4195} & \nodata & \nodata & 115.6 & S3 & \nodata & \nodata & \nodata\\
 \object{RX J1504.1-0248} & \dataset[ADS/Sa.CXO#obs/5793]{5793} & 15:04:07.415 & -02:48:15.70 & 39.2 & I3 & 0.2150 & 8.00 & \nodata\\
 \object{RX J1532.9+3021} & \dataset[ADS/Sa.CXO#obs/1649]{1649} & 15:32:53.781 & +30:20:58.72 & 9.4 & I3 & 0.3450 & 5.44 & \nodata\\
 & \dataset[ADS/Sa.CXO#obs/1665]{1665} & \nodata & \nodata & 10.0 & S3 & \nodata & \nodata & \nodata\\
 \object{RX J1539.5-8335} & \dataset[ADS/Sa.CXO#obs/8266]{8266} & 15:39:32.485 & -83:35:23.83 & 8.0 & I3 & 0.0728 & 4.29 & \nodata\\
 \object{RX J1720.1+2638} & \dataset[ADS/Sa.CXO#obs/4361]{4361} & 17:20:09.941 & +26:37:29.11 & 25.7 & I3 & 0.1640 & 6.37 & \nodata\\
 \object{RX J1720.2+3536} & \dataset[ADS/Sa.CXO#obs/3280]{3280} & 17:20:16.953 & +35:36:23.63 & 20.8 & I3 & 0.3913 & 5.65 & \nodata\\
 & \dataset[ADS/Sa.CXO#obs/6107]{6107} & \nodata & \nodata & 33.9 & I3 & \nodata & \nodata & \nodata\\
 & \dataset[ADS/Sa.CXO#obs/7225]{7225} & \nodata & \nodata & 2.0 & I3 & \nodata & \nodata & \nodata\\
 \object{RX J1852.1+5711} & \dataset[ADS/Sa.CXO#obs/5749]{5749} & 18:52:08.815 & +57:11:42.63 & 29.8 & I3 & 0.1094 & 3.66 & \nodata\\
 \object{RX J2129.6+0005} & \dataset[ADS/Sa.CXO#obs/552]{552} & 21:29:39.944 & +00:05:18.83 & 10.0 & I3 & 0.2350 & 5.91 & \nodata\\
 \object{RXCJ0331.1-2100} & \dataset[ADS/Sa.CXO#obs/10790]{10790} & 03:31:06.020 & -21:00:32.93 & 10.0 & I3 & 0.1880 & 4.61 & \nodata\\
 & \dataset[ADS/Sa.CXO#obs/9415]{9415} & \nodata & \nodata & 9.9 & I3 & \nodata & \nodata & \nodata\\
 \object{SC 1327-312} & \dataset[ADS/Sa.CXO#obs/4165]{4165} & 13:29:47.748 & -31:36:23.54 & 18.4 & I3 & 0.0531 & 3.53 &      f\\
 \object{Sersic 159-03} & \dataset[ADS/Sa.CXO#obs/1668]{1668} & 23:13:58.764 & -42:43:34.70 & 9.9 & S3 & 0.0580 & 2.65 & \nodata\\
 \object{SS2B153} & \dataset[ADS/Sa.CXO#obs/3243]{3243} & 10:50:26.125 & -12:50:41.76 & 29.5 & S3 & 0.0186 & 0.80 & \nodata\\
 \object{UGC 3957} & \dataset[ADS/Sa.CXO#obs/8265]{8265} & 07:40:58.335 & +55:25:38.30 & 7.9 & I3 & 0.0341 & 2.85 & \nodata\\
 \object{UGC 12491} & \dataset[ADS/Sa.CXO#obs/7896]{7896} & 23:18:38.311 & +42:57:29.06 & 32.7 & S3 & 0.0174 & 0.87 & \nodata\\
 \object{ZWCL 1215} & \dataset[ADS/Sa.CXO#obs/4184]{4184} & 12:17:41.708 & +03:39:15.81 & 12.1 & I3 & 0.0750 & 6.62 & \nodata\\
 \object{ZWCL 1358+6245} & \dataset[ADS/Sa.CXO#obs/516]{516} & 13:59:50.526 & +62:31:04.57 & 54.1 & S3 & 0.3280 & 10.66 & \nodata\\
 \object{ZWCL 1742} & \dataset[ADS/Sa.CXO#obs/8267]{8267} & 17:44:14.515 & +32:59:29.68 & 8.0 & I3 & 0.0757 & 4.40 & \nodata\\
 \object{ZWCL 1953} & \dataset[ADS/Sa.CXO#obs/1659]{1659} & 08:50:06.677 & +36:04:16.16 & 24.9 & I3 & 0.3800 & 7.37 & \nodata\\
 \object{ZWCL 3146} & \dataset[ADS/Sa.CXO#obs/909]{909} & 10:23:39.735 & +04:11:08.05 & 46.0 & I3 & 0.2900 & 7.48 & \nodata\\
 \object{ZWCL 7160} & \dataset[ADS/Sa.CXO#obs/543]{543} & 14:57:15.158 & +22:20:33.85 & 9.9 & I3 & 0.2578 & 4.53 & \nodata\\
 \object{Zwicky 2701} & \dataset[ADS/Sa.CXO#obs/3195]{3195} & 09:52:49.183 & +51:53:05.27 & 26.9 & S3 & 0.2100 & 5.21 & \nodata\\
 \object{ZwCl 0857.9+2107} & \dataset[ADS/Sa.CXO#obs/7897]{7897} & 09:00:36.835 & +20:53:40.36 & 9.0 & I3 & 0.2350 & 4.29 &      e
\enddata
\tablecomments{Col. (1) Cluster name; Col. (2) CXC CDA Observation Identification Number; Col. (3) R.A. of cluster center; Col. (4) Decl. of cluster center; Col. (5) exposure time; Col. (6) CCD location of cluster center; Col. (7) redshift; Col. (8) average cluster temperature; Col. (9) best-fit core entropy measured in this work; Col. (10) assigned notes: `a' Clusters analyzed using the best-fit $\beta$-model for the surface brightness profiles (discussed in \S\ref{sec:dene}); `b' Clusters with complex surface brightness of which only the central regions were used in fitting $K(r)$; `c' Clusters only used during analysis of the \hifl\ sub-sample (discussed in \S\ref{sec:hifl}); `d' Clusters with central AGN removed during analysis (discussed in \S\ref{sec:centsrc}); `e' Clusters with central compact source removed during analysis (discussed in \S\ref{sec:centsrc}); `f' Clusters with central bin ignored during fitting (discussed in \S\ref{sec:centsrc})..}
\end{deluxetable}

\begin{deluxetable}{lcccccccc}
\tablewidth{0pt}
\tabletypesize{\scriptsize}
\tablecaption{Summary of $\beta$-Model Fits\label{tab:betafits}}
\tablehead{\colhead{Cluster} & \colhead{$S_{01}$} & \colhead{$r_{c1}$} & \colhead{$\beta_{1}$} & \colhead{$S_{02}$} & \colhead{$r_{c2}$} & \colhead{$\beta_{2}$} & \colhead{D.O.F.} & \colhead{$\chi_{\mathrm{red}}^2$}\\
\colhead{} & \colhead{10$^{-6}$ cts s$^{-1}$ arcsec$^{2}$} & \colhead{\arcsec} & \colhead{} & \colhead{10$^{-6}$ cts s$^{-1}$ arcsec$^{2}$} & \colhead{\arcsec} & \colhead{} & \colhead{} & \colhead{}\\
\colhead{{(1)}} & \colhead{{(2)}} & \colhead{{(3)}} & \colhead{{(4)}} & \colhead{{(5)}} & \colhead{{(6)}} & \colhead{{(7)}} & \colhead{{(8)}} & \colhead{{(9)}}
}
\startdata
Abell 119 &  4.93 $\pm$  0.73 &  39.1 $\pm$  15.3 &  0.34 $\pm$  0.07 &  3.52 $\pm$  0.96 & 735.2 $\pm$ 479.4 &  1.27 $\pm$  1.27 &    52 &  1.76\\
Abell 160 &  2.32 $\pm$  0.27 &  53.4 $\pm$  11.1 &  0.57 $\pm$  0.12 &  1.29 $\pm$  0.22 & 284.0 $\pm$  52.2 &  0.74 $\pm$  0.10 &    90 &  1.18\\
Abell 193 & 24.72 $\pm$  1.62 &  80.8 $\pm$   2.2 &  0.43 $\pm$  0.01 & \nodata & \nodata & \nodata &    38 &  0.43\\
Abell 400 &  4.66 $\pm$  0.09 & 151.3 $\pm$   6.4 &  0.42 $\pm$  0.01 & \nodata & \nodata & \nodata &    96 &  0.57\\
Abell 1060 & 21.95 $\pm$  0.44 &  93.5 $\pm$   8.1 &  0.35 $\pm$  0.01 & \nodata & \nodata & \nodata &    42 &  1.44\\
Abell 1240 &  1.58 $\pm$  0.07 & 247.9 $\pm$  46.9 &  1.01 $\pm$  0.22 & \nodata & \nodata & \nodata &    58 &  1.58\\
Abell 1736 &  3.81 $\pm$  0.56 &  55.6 $\pm$  16.1 &  0.42 $\pm$  0.12 &  2.49 $\pm$  0.47 & 1470.0 $\pm$  87.2 &  5.00 $\pm$  0.73 &    35 &  1.58\\
Abell 2125 &  3.50 $\pm$  0.20 &  26.0 $\pm$   4.9 &  0.49 $\pm$  0.05 &  1.02 $\pm$  0.13 & 159.9 $\pm$   9.2 &  1.32 $\pm$  0.16 &    35 &  0.33\\
Abell 2255 &  8.38 $\pm$  0.15 & 222.7 $\pm$   9.8 &  0.62 $\pm$  0.02 & \nodata & \nodata & \nodata &    94 &  1.45\\
Abell 2256 & 21.69 $\pm$  0.19 & 407.8 $\pm$  17.9 &  0.99 $\pm$  0.05 & \nodata & \nodata & \nodata &    88 &  0.83\\
Abell 2319 & 47.39 $\pm$  0.61 & 128.8 $\pm$   3.1 &  0.49 $\pm$  0.01 & \nodata & \nodata & \nodata &    92 &  1.67\\
Abell 2462 &  8.19 $\pm$  1.43 &  60.8 $\pm$   9.6 &  0.64 $\pm$  0.11 &  1.87 $\pm$  0.25 & 762.7 $\pm$  39.1 &  5.00 $\pm$  0.87 &    67 &  1.54\\
Abell 2631 & 20.55 $\pm$  1.01 &  66.0 $\pm$   4.0 &  0.73 $\pm$  0.03 & \nodata & \nodata & \nodata &    58 &  1.15\\
Abell 3376 &  4.21 $\pm$  0.09 & 125.5 $\pm$   5.6 &  0.40 $\pm$  0.01 & \nodata & \nodata & \nodata &    98 &  1.42\\
Abell 3391 & 10.65 $\pm$  0.31 & 132.3 $\pm$   7.9 &  0.48 $\pm$  0.01 & \nodata & \nodata & \nodata &    84 &  1.86\\
Abell 3395 &  6.85 $\pm$  0.67 &  90.9 $\pm$   6.7 &  0.49 $\pm$  0.03 & \nodata & \nodata & \nodata &    38 &  0.96\\
MKW 8 &  7.71 $\pm$  0.62 &  25.2 $\pm$   2.5 &  0.32 $\pm$  0.01 &  1.51 $\pm$  0.08 & 1124.0 $\pm$  64.1 &  5.00 $\pm$  0.40 &    88 &  0.65\\
RBS 461 & 12.84 $\pm$  0.34 & 102.2 $\pm$   4.1 &  0.52 $\pm$  0.01 & \nodata & \nodata & \nodata &    84 &  1.56
\enddata
\tablecomments{Col. (1) Cluster name; col. (2) central surface brightness of first component; col. (3) core radius of first component; col. (4) $\beta$ parameter of first component; col. (5) central surface brightness of second component; col. (6) core radius of second component; col. (7) $\beta$ parameter of second component; col. (8) model degrees of freedom; and col. (9) reduced chi-squared statistic for best-fit model.}
\end{deluxetable}

\begin{deluxetable}{lcccc}
\tabletypesize{\scriptsize}
\tablecaption{M. Donahue's \halpha\ Observations.\label{tab:newha}}
\tablewidth{0pt}
\tablehead{
  \colhead{Cluster} & \colhead{Telescope} & \colhead{$z$} & \colhead{$[NII]$/\halpha} & \colhead{\halpha\ Flux}\\
  \colhead{}      & \colhead{}        & \colhead{}  & \colhead{}               & \colhead{$10^{-15}$ \flux}
}
\startdata
Abell 85     & PO & 0.0558 & 2.67    &    0.581\\
Abell 119    & LC & 0.0442 & \nodata & $<$0.036\\
Abell 133    & LC & 0.0558 & \nodata &    0.88\\
Abell 496    & LC & 0.0328 & 2.50    &    2.90\\
Abell 1644   & LC & 0.0471 & \nodata &    1.00\\
Abell 1650   & LC & 0.0843 & \nodata & $<$0.029\\
Abell 1689   & LC & 0.1843 & \nodata & $<$0.029\\
Abell 1736   & LC & 0.0338 & \nodata & $<$0.026\\
Abell 2597   & PO & 0.0854 & 0.85    &    29.7\\
Abell 3112   & LC & 0.0720 & 2.22    &    2.66\\
Abell 3158   & LC & 0.0586 & \nodata & $<$0.036\\
Abell 3266   & LC & 0.0590 & 1.62    & $<$0.027\\
Abell 4059   & LC & 0.0475 & 3.60    &    2.22\\
Cygnus A     & PO & 0.0561 & 1.85    &    28.4\\
EXO 0422-086 & LC & 0.0397 & \nodata & $<$0.031\\
Hydra A      & LC & 0.0522 & 0.85    &    13.4\\
PKS 0745-191 & LC & 0.1028 & 1.02    &    10.4
\enddata
\tablecomments{The abbreviation ``PO'' denotes observations taken on
  the 5 m Hale Telescope at the Palomar Observatory, USA, while ``LC''
  are observations taken on the DuPont 2.5 m telescope at the Las
  Campanas Observatory, Chile. Upperlimits for \halpha\ fluxes are
  $3\sigma$.}
\end{deluxetable}

\begin{deluxetable}{lcccccccc}
\tabletypesize{\scriptsize}
\tablecaption{Statistics of Best-Fit Parameters\label{tab:bfparams}}
\tablewidth{0pt}
\tablehead{
\colhead{Sample} & \colhead{$N_{obj}$} & \colhead{\kna} & \colhead {$K_{12}$} & \colhead{\khun} & \colhead{$\alpha$} & \multicolumn{3}{c}{$N_{\kna=0}$}\\
\colhead{} & \colhead{} & \colhead{\ent} & \colhead{\ent} & \colhead{\ent} & \colhead{} & \colhead{$1\sigma$} & \colhead{$2\sigma$} & \colhead{$3\sigma$}\\
\colhead{(1)} & \colhead{(2)} & \colhead{(3)} & \colhead{(4)} & \colhead{(5)} & \colhead{(6)} & \colhead{(7)} & \colhead{(8)} & \colhead{(9)}
}
\startdata
\multicolumn{9}{c}{All \kna}\\
\hline
\accept\       & 233 & $72.9 \pm 33.7$ & $91.6 \pm 35.7$ & $126 \pm 45$   & $1.21 \pm 0.39$ & 4 (2\%) & 12 (5\%)  & 24 (11\%)\\
\hifl\         & 59  & $62.3 \pm 32.7$ & $87.2 \pm 34.5$ & $166 \pm 65$   & $1.18 \pm 0.38$ & 1 (2\%) & 3  (5\%)  & 4  (7\%) \\
CSE            & 37  & $61.9 \pm 27.4$ & $81.6 \pm 31.3$ & $132 \pm 45$   & $1.19 \pm 0.39$ & 1 (3\%) & 2  (5\%)  & 6  (16\%)\\
$\beta$ Models & 17  & $220  \pm 74$   & $230  \pm 76.9$ & $67.4\pm 27.0$ & $1.45 \pm 0.47$ & \nodata & \nodata   & \nodata  \\[0.25cm]
\hline
\multicolumn{9}{c}{$4 \ent < \kna \le 50 \ent$}\\
\hline
\accept\       & 99  & $17.5 \pm 5.8$  & $31.2 \pm 10.3$ & $148 \pm 49$   & $1.21 \pm 0.39$ & 3 (3\%) & 5  (5\%)  & 10 (10\%)\\
\hifl\         & 25  & $13.6 \pm 4.6$  & $29.4 \pm 9.63$ & $174 \pm 57$   & $1.15 \pm 0.37$ & 0 (0\%) & 0  (0\%)  & 0  (0\%) \\
CSE            & 17  & $16.4 \pm 5.4$  & $30.9 \pm 10.2$ & $146 \pm 48$   & $1.19 \pm 0.38$ & 0 (0\%) & 0  (0\%)  & 2  (12\%)\\[0.25cm]
\hline
\multicolumn{9}{c}{$\kna \le 50 \ent$}\\
\hline
\accept\       & 107 & $16.1 \pm 5.7$  & $30.5 \pm 10.0$ & $150 \pm 50$   & $1.20 \pm 0.38$ & 4 (4\%) & 6  (6\%)  & 11 (10\%)\\
\hifl\         & 29  & $11.4 \pm 4.2$  & $31.2 \pm 10.5$ & $235 \pm 89$   & $1.17 \pm 0.37$ & 1 (4\%) & 1  (4\%)  & 1  (4\%) \\
CSE            & 19  & $15.6 \pm 5.2$  & $30.9 \pm 10.2$ & $146 \pm 48$   & $1.16 \pm 0.38$ & 1 (5\%) & 1  (5\%)  & 3  (16\%)\\[0.25cm]
\hline
\multicolumn{9}{c}{$\kna > 50 \ent$}\\
\hline
\accept\       & 126 & $156  \pm 54$   & $175  \pm 59$   & $107 \pm 39$   & $1.23 \pm 0.40$ & 0 (0\%) & 6  (5\%)  & 13 (11\%)\\
\hifl\         & 30  & $151  \pm 53$   & $172  \pm 58$   & $113 \pm 43$   & $1.19 \pm 0.39$ & 0 (0\%) & 2  (7\%)  & 3  (10\%)\\
CSE            & 18  & $148  \pm 49$   & $165  \pm 54$   & $118 \pm 42$   & $1.23 \pm 0.40$ & 0 (0\%) & 1  (6\%)  & 3  (17\%)
\enddata
\tablecomments{
Listed here are the mean best-fit parameters of the model $K(r) = \kna
+ \khun (r/100 \kpc)^{\alpha}$ for various sub-groups of the full
\accept\ sample. Each sub-group is labeled in the table. The 'CSE'
sample are the clusters with a central source excluded (discussed in
\S\ref{sec:centsrc}). The $K_{12}$ values represent the entropy at 12
kpc and are calculated from the best-fit models. Col. (1) Sample being
considered; col. (2) number of objects in the sub-group; col. (3) mean
best-fit \kna; col. (4) mean entropy at 12 kpc; col. (5) mean best-fit
\khun; and col. (6) mean best-fit power-law index; cols. (7,8,9)
number of clusters consistent with $\kna = 0 \ent$ at $1\sigma$,
$2\sigma$, and $3\sigma$ significance, respectively. Percentage of the
sub-group represented by each is also listed.}
\end{deluxetable}

\begin{deluxetable}{lcccccccccc}
\tablewidth{0pt}
\tabletypesize{\scriptsize}
\tablecaption{Summary of Entropy Profile Fits\label{tab:kfits}}
\tablehead{\colhead{Cluster} & \colhead{Method} & \colhead{$N_{bins}$} & \colhead{$r_{max}$} & \colhead{\kna} & \colhead{$\sigma_{\kna} > 0$} & \colhead{\khun} & \colhead{$\alpha$} & \colhead{DOF} & \colhead{\chisq} & \colhead{p-value}\\
\colhead{} & \colhead{} & \colhead{} & \colhead{Mpc} & \colhead{\ent} & \colhead{} & \colhead{\ent} & \colhead{} & \colhead{} & \colhead{} & \colhead{}\\
\colhead{{(1)}} & \colhead{{(2)}} & \colhead{{(3)}} & \colhead{{(4)}} & \colhead{{(5)}} & \colhead{{(6)}} & \colhead{{(7)}} & \colhead{{(8)}} & \colhead{{(9)}} & \colhead{{(10)}} & \colhead{{(11)}}
}
\startdata
1E0657 56 &   extr &     48 &   1.00 &  299.4 $\pm$   19.6 &   15.3 &   20.5 $\pm$    7.0 &   1.84 $\pm$   0.16 &     45 &  42.09 & 5.96e-01\\
 &      - & - & - &    0.0 & - &  277.9 $\pm$   14.5 &   0.60 $\pm$   0.04 &     46 & 146.18 & 2.31e-12\\
 &   flat & - & - &  307.5 $\pm$   19.3 &   15.9 &   18.6 $\pm$    6.5 &   1.88 $\pm$   0.17 &     45 &  42.87 & 5.63e-01\\
 &      - & - & - &    0.0 & - &  283.6 $\pm$   14.6 &   0.58 $\pm$   0.04 &     46 & 157.03 & 4.77e-14\\
2A 335+096 &   extr &     37 &   0.12 &    5.3 $\pm$    0.2 &   34.8 &  137.7 $\pm$    1.9 &   1.43 $\pm$   0.02 &     34 & 173.51 & 1.26e-20\\
 &      - & - & - &    0.0 & - &  117.7 $\pm$    1.5 &   1.06 $\pm$   0.01 &     35 & 1188.38 & 6.24e-227\\
 &   flat & - & - &    7.1 $\pm$    0.1 &   49.3 &  138.6 $\pm$    1.9 &   1.52 $\pm$   0.02 &     34 & 209.16 & 4.39e-27\\
 &      - & - & - &    0.0 & - &  107.4 $\pm$    1.4 &   0.97 $\pm$   0.01 &     35 & 2097.26 & 0.00e+00\\
2PIGG J0011.5-2850 &   extr &     27 &   0.20 &   75.3 $\pm$   44.8 &    1.7 &  236.9 $\pm$   53.2 &   0.82 $\pm$   0.27 &     24 &   2.01 & 1.00e+00\\
 &      - & - & - &    0.0 & - &  318.5 $\pm$   13.6 &   0.53 $\pm$   0.06 &     25 &   3.19 & 1.00e+00\\
 &   flat & - & - &  102.0 $\pm$   42.9 &    2.4 &  214.7 $\pm$   51.5 &   0.84 $\pm$   0.29 &     24 &   2.79 & 1.00e+00\\
 &      - & - & - &    0.0 & - &  323.8 $\pm$   13.7 &   0.45 $\pm$   0.05 &     25 &   4.40 & 1.00e+00\\
2PIGG J2227.0-3041 &   extr &     23 &   0.15 &   12.5 $\pm$    1.0 &   12.3 &  119.5 $\pm$    3.6 &   1.32 $\pm$   0.06 &     20 &  13.14 & 8.71e-01\\
 &      - & - & - &    0.0 & - &  118.4 $\pm$    3.3 &   0.88 $\pm$   0.02 &     21 & 132.53 & 3.43e-18\\
 &   flat & - & - &   17.1 $\pm$    1.0 &   17.4 &  113.9 $\pm$    3.6 &   1.37 $\pm$   0.06 &     20 &  11.50 & 9.32e-01\\
 &      - & - & - &    0.0 & - &  108.6 $\pm$    3.1 &   0.73 $\pm$   0.02 &     21 & 202.76 & 1.04e-31\\
3C 28.0 &   extr &     12 &   0.18 &   20.7 $\pm$    1.3 &   15.5 &  111.7 $\pm$    3.9 &   1.70 $\pm$   0.09 &      9 &  23.42 & 5.32e-03\\
 &      - & - & - &    0.0 & - &  115.3 $\pm$    3.4 &   0.82 $\pm$   0.03 &     10 & 151.06 & 2.25e-27\\
 &   flat & - & - &   23.9 $\pm$    1.3 &   18.6 &  107.8 $\pm$    3.9 &   1.79 $\pm$   0.09 &      9 &  22.93 & 6.35e-03\\
 &      - & - & - &    0.0 & - &  110.8 $\pm$    3.3 &   0.74 $\pm$   0.03 &     10 & 179.58 & 2.86e-33\\
3C 295 &   extr &     17 &   0.50 &   12.6 $\pm$    2.6 &    4.9 &   84.5 $\pm$    6.4 &   1.45 $\pm$   0.07 &     14 &   7.52 & 9.13e-01\\
 &      - & - & - &    0.0 & - &  108.2 $\pm$    3.8 &   1.20 $\pm$   0.04 &     15 &  27.39 & 2.57e-02\\
 &   flat & - & - &   14.5 $\pm$    2.5 &    5.8 &   81.9 $\pm$    6.3 &   1.47 $\pm$   0.07 &     14 &   8.36 & 8.70e-01\\
 &      - & - & - &    0.0 & - &  109.3 $\pm$    3.8 &   1.18 $\pm$   0.04 &     15 &  34.84 & 2.59e-03\\
3C 388 &   extr &     24 &   0.20 &   17.0 $\pm$    5.7 &    3.0 &  214.2 $\pm$    8.5 &   0.76 $\pm$   0.07 &     21 &  10.82 & 9.66e-01\\
 &      - & - & - &    0.0 & - &  226.3 $\pm$    7.0 &   0.60 $\pm$   0.02 &     22 &  16.13 & 8.09e-01\\
 &   flat & - & - &   17.0 $\pm$    5.8 &    3.0 &  214.3 $\pm$    8.5 &   0.76 $\pm$   0.07 &     21 &  10.90 & 9.65e-01\\
 &      - & - & - &    0.0 & - &  226.4 $\pm$    7.0 &   0.60 $\pm$   0.02 &     22 &  16.14 & 8.09e-01\\
4C 55.16 &   extr &     21 &   0.40 &   22.4 $\pm$    2.9 &    7.7 &  162.9 $\pm$    7.7 &   1.28 $\pm$   0.06 &     18 &   7.52 & 9.85e-01\\
 &      - & - & - &    0.0 & - &  197.1 $\pm$    5.6 &   0.94 $\pm$   0.03 &     19 &  46.97 & 3.61e-04\\
 &   flat & - & - &   23.3 $\pm$    2.9 &    8.1 &  161.6 $\pm$    7.7 &   1.29 $\pm$   0.06 &     18 &   7.92 & 9.80e-01\\
 &      - & - & - &    0.0 & - &  197.0 $\pm$    5.6 &   0.93 $\pm$   0.03 &     19 &  50.60 & 1.07e-04\\
Abell 13 &   extr &     35 &   0.30 &  182.6 $\pm$   26.2 &    7.0 &  182.0 $\pm$   36.8 &   1.37 $\pm$   0.22 &     32 &  11.58 & 1.00e+00\\
 &      - & - & - &    0.0 & - &  401.9 $\pm$   14.1 &   0.59 $\pm$   0.05 &     33 &  32.03 & 5.15e-01\\
 &   flat & - & - &  182.6 $\pm$   26.2 &    7.0 &  182.0 $\pm$   36.8 &   1.37 $\pm$   0.22 &     32 &  11.58 & 1.00e+00\\
 &      - & - & - &    0.0 & - &  401.9 $\pm$   14.1 &   0.59 $\pm$   0.05 &     33 &  32.03 & 5.15e-01\\
Abell 68 &   extr &     31 &   0.60 &  217.3 $\pm$   89.0 &    2.4 &  142.3 $\pm$   98.3 &   0.89 $\pm$   0.39 &     28 &   1.72 & 1.00e+00\\
 &      - & - & - &    0.0 & - &  393.4 $\pm$   36.9 &   0.40 $\pm$   0.08 &     29 &   3.45 & 1.00e+00\\
 &   flat & - & - &  217.3 $\pm$   89.0 &    2.4 &  142.3 $\pm$   98.3 &   0.89 $\pm$   0.39 &     28 &   1.72 & 1.00e+00\\
 &      - & - & - &    0.0 & - &  393.4 $\pm$   36.9 &   0.40 $\pm$   0.08 &     29 &   3.45 & 1.00e+00\\
Abell 85 &   extr &     39 &   0.20 &    7.3 $\pm$    0.6 &   12.8 &  165.5 $\pm$    1.9 &   1.05 $\pm$   0.02 &     36 &  52.57 & 3.67e-02\\
 &      - & - & - &    0.0 & - &  170.2 $\pm$    1.8 &   0.90 $\pm$   0.01 &     37 & 201.42 & 1.67e-24\\
 &   flat & - & - &   12.5 $\pm$    0.5 &   23.7 &  158.8 $\pm$    1.9 &   1.12 $\pm$   0.02 &     36 &  59.03 & 9.10e-03\\
 &      - & - & - &    0.0 & - &  165.5 $\pm$    1.8 &   0.83 $\pm$   0.01 &     37 & 492.25 & 6.48e-81\\
Abell 119 &   extr &     23 &   0.20 &  210.1 $\pm$   84.5 &    2.5 &  207.1 $\pm$  100.1 &   0.77 $\pm$   0.56 &     20 &   0.12 & 1.00e+00\\
 &      - & - & - &    0.0 & - &  418.7 $\pm$   31.2 &   0.26 $\pm$   0.07 &     21 &   1.34 & 1.00e+00\\
 &   flat & - & - &  233.9 $\pm$   87.7 &    2.7 &  191.3 $\pm$  102.8 &   0.75 $\pm$   0.61 &     20 &   0.10 & 1.00e+00\\
 &      - & - & - &    0.0 & - &  425.5 $\pm$   31.1 &   0.22 $\pm$   0.06 &     21 &   1.19 & 1.00e+00\\
Abell 133 &   extr &     20 &   0.10 &   13.3 $\pm$    0.5 &   25.1 &  170.7 $\pm$    3.9 &   1.47 $\pm$   0.04 &     17 &  44.38 & 3.01e-04\\
 &      - & - & - &    0.0 & - &  142.2 $\pm$    2.7 &   0.90 $\pm$   0.01 &     18 & 504.69 & 1.08e-95\\
 &   flat & - & - &   17.3 $\pm$    0.5 &   35.0 &  170.1 $\pm$    4.1 &   1.59 $\pm$   0.04 &     17 &  54.26 & 9.02e-06\\
 &      - & - & - &    0.0 & - &  127.5 $\pm$    2.5 &   0.79 $\pm$   0.01 &     18 & 812.02 & 8.79e-161\\
Abell 141 &   extr &     33 &   0.60 &  144.1 $\pm$   31.3 &    4.6 &   68.5 $\pm$   27.5 &   1.53 $\pm$   0.27 &     30 & 136.92 & 1.32e-15\\
 &      - & - & - &    0.0 & - &  221.9 $\pm$   18.4 &   0.77 $\pm$   0.09 &     31 & 447.75 & 2.25e-75\\
 &   flat & - & - &  205.0 $\pm$   27.4 &    7.5 &   42.6 $\pm$   20.8 &   1.78 $\pm$   0.33 &     30 & 175.31 & 1.84e-22\\
 &      - & - & - &    0.0 & - &  269.7 $\pm$   17.7 &   0.57 $\pm$   0.07 &     31 & 704.66 & 2.56e-128\\
Abell 160 &   extr &     28 &   0.12 &  155.8 $\pm$   27.7 &    5.6 &  116.3 $\pm$   29.2 &   0.98 $\pm$   0.57 &     25 &   0.33 & 1.00e+00\\
 &      - & - & - &    0.0 & - &  254.7 $\pm$   13.5 &   0.20 $\pm$   0.04 &     26 &   3.66 & 1.00e+00\\
 &   flat & - & - &  155.8 $\pm$   27.7 &    5.6 &  116.3 $\pm$   29.2 &   0.98 $\pm$   0.57 &     25 &   0.33 & 1.00e+00\\
 &      - & - & - &    0.0 & - &  254.7 $\pm$   13.5 &   0.20 $\pm$   0.04 &     26 &   3.66 & 1.00e+00\\
Abell 193 &   extr &     26 &   0.12 &  185.5 $\pm$   13.3 &   13.9 &   36.0 $\pm$   16.8 &   2.23 $\pm$   1.89 &     23 &   0.02 & 1.00e+00\\
 &      - & - & - &    0.0 & - &  213.8 $\pm$    7.3 &   0.09 $\pm$   0.04 &     24 &   2.92 & 1.00e+00\\
 &   flat & - & - &  185.5 $\pm$   13.3 &   13.9 &   36.0 $\pm$   16.8 &   2.23 $\pm$   1.89 &     23 &   0.02 & 1.00e+00\\
 &      - & - & - &    0.0 & - &  213.8 $\pm$    7.3 &   0.09 $\pm$   0.04 &     24 &   2.92 & 1.00e+00\\
Abell 209 &   extr &     19 &   0.30 &  100.7 $\pm$   26.3 &    3.8 &  150.5 $\pm$   34.5 &   0.81 $\pm$   0.21 &     16 &   2.48 & 1.00e+00\\
 &      - & - & - &    0.0 & - &  266.2 $\pm$    9.6 &   0.40 $\pm$   0.04 &     17 &   7.88 & 9.69e-01\\
 &   flat & - & - &  105.5 $\pm$   26.9 &    3.9 &  149.3 $\pm$   35.2 &   0.80 $\pm$   0.21 &     16 &   2.73 & 1.00e+00\\
 &      - & - & - &    0.0 & - &  269.5 $\pm$    9.6 &   0.38 $\pm$   0.04 &     17 &   8.03 & 9.66e-01\\
Abell 222 &   extr &     37 &   0.60 &  122.2 $\pm$   15.2 &    8.0 &   84.8 $\pm$   19.2 &   0.99 $\pm$   0.15 &     34 &   4.82 & 1.00e+00\\
 &      - & - & - &    0.0 & - &  231.9 $\pm$    7.3 &   0.40 $\pm$   0.03 &     35 &  26.22 & 8.58e-01\\
 &   flat & - & - &  126.0 $\pm$   15.0 &    8.4 &   82.2 $\pm$   19.0 &   1.00 $\pm$   0.15 &     34 &   4.94 & 1.00e+00\\
 &      - & - & - &    0.0 & - &  233.9 $\pm$    7.3 &   0.39 $\pm$   0.03 &     35 &  27.16 & 8.26e-01\\
Abell 223 &   extr &     30 &   0.50 &  183.9 $\pm$   46.1 &    4.0 &  160.7 $\pm$   59.2 &   1.24 $\pm$   0.31 &     27 &   1.35 & 1.00e+00\\
 &      - & - & - &    0.0 & - &  386.1 $\pm$   23.5 &   0.57 $\pm$   0.08 &     28 &   6.55 & 1.00e+00\\
 &   flat & - & - &  183.9 $\pm$   46.1 &    4.0 &  160.7 $\pm$   59.2 &   1.24 $\pm$   0.31 &     27 &   1.35 & 1.00e+00\\
 &      - & - & - &    0.0 & - &  386.1 $\pm$   23.5 &   0.57 $\pm$   0.08 &     28 &   6.55 & 1.00e+00\\
Abell 262 &   extr &     30 &   0.05 &    9.4 $\pm$    0.8 &   11.8 &  200.9 $\pm$    7.3 &   0.95 $\pm$   0.04 &     27 &  52.37 & 2.40e-03\\
 &      - & - & - &    0.0 & - &  166.6 $\pm$    3.3 &   0.66 $\pm$   0.01 &     28 & 159.48 & 2.36e-20\\
 &   flat & - & - &   10.6 $\pm$    0.8 &   13.8 &  205.1 $\pm$    7.9 &   0.98 $\pm$   0.04 &     27 &  60.17 & 2.50e-04\\
 &      - & - & - &    0.0 & - &  164.3 $\pm$    3.3 &   0.65 $\pm$   0.01 &     28 & 199.73 & 7.70e-28\\
Abell 267 &   extr &     22 &   0.40 &  168.3 $\pm$   17.7 &    9.5 &   52.0 $\pm$   21.1 &   1.82 $\pm$   0.38 &     19 &   0.62 & 1.00e+00\\
 &      - & - & - &    0.0 & - &  263.4 $\pm$   11.7 &   0.41 $\pm$   0.06 &     20 &  22.64 & 3.07e-01\\
 &   flat & - & - &  168.6 $\pm$   17.6 &    9.6 &   51.8 $\pm$   21.0 &   1.82 $\pm$   0.38 &     19 &   0.62 & 1.00e+00\\
 &      - & - & - &    0.0 & - &  263.5 $\pm$   11.7 &   0.40 $\pm$   0.06 &     20 &  22.71 & 3.03e-01\\
Abell 368 &   extr &     28 &   0.50 &   47.5 $\pm$    8.3 &    5.7 &  146.7 $\pm$   15.4 &   1.20 $\pm$   0.11 &     25 &   6.13 & 1.00e+00\\
 &      - & - & - &    0.0 & - &  216.8 $\pm$    8.0 &   0.77 $\pm$   0.04 &     26 &  24.09 & 5.71e-01\\
 &   flat & - & - &   50.9 $\pm$    8.2 &    6.2 &  144.1 $\pm$   15.4 &   1.21 $\pm$   0.11 &     25 &   6.18 & 1.00e+00\\
 &      - & - & - &    0.0 & - &  218.7 $\pm$    8.0 &   0.74 $\pm$   0.04 &     26 &  26.03 & 4.61e-01\\
Abell 370 &   extr &     20 &   0.50 &  321.9 $\pm$   90.8 &    3.5 &   78.7 $\pm$   89.3 &   1.24 $\pm$   0.68 &     17 &   2.41 & 1.00e+00\\
 &      - & - & - &    0.0 & - &  422.4 $\pm$   34.9 &   0.40 $\pm$   0.08 &     18 &   6.02 & 9.96e-01\\
 &   flat & - & - &  321.9 $\pm$   90.8 &    3.5 &   78.7 $\pm$   89.3 &   1.24 $\pm$   0.68 &     17 &   2.41 & 1.00e+00\\
 &      - & - & - &    0.0 & - &  422.4 $\pm$   34.9 &   0.40 $\pm$   0.08 &     18 &   6.02 & 9.96e-01\\
Abell 383 &   extr &     13 &   0.20 &   10.9 $\pm$    1.6 &    6.6 &  114.0 $\pm$    5.2 &   1.34 $\pm$   0.09 &     10 &   4.76 & 9.07e-01\\
 &      - & - & - &    0.0 & - &  121.4 $\pm$    4.9 &   0.96 $\pm$   0.04 &     11 &  40.90 & 2.50e-05\\
 &   flat & - & - &   13.0 $\pm$    1.6 &    8.3 &  110.9 $\pm$    5.2 &   1.40 $\pm$   0.09 &     10 &   6.30 & 7.89e-01\\
 &      - & - & - &    0.0 & - &  119.2 $\pm$    4.9 &   0.92 $\pm$   0.03 &     11 &  58.48 & 1.78e-08\\
Abell 399 &   extr &     31 &   0.20 &  140.3 $\pm$   19.1 &    7.3 &  215.3 $\pm$   22.7 &   0.73 $\pm$   0.12 &     28 &   4.14 & 1.00e+00\\
 &      - & - & - &    0.0 & - &  360.8 $\pm$    7.0 &   0.32 $\pm$   0.02 &     29 &  21.40 & 8.44e-01\\
 &   flat & - & - &  153.2 $\pm$   18.8 &    8.2 &  204.3 $\pm$   22.4 &   0.74 $\pm$   0.12 &     28 &   4.19 & 1.00e+00\\
 &      - & - & - &    0.0 & - &  362.5 $\pm$    7.0 &   0.30 $\pm$   0.02 &     29 &  22.24 & 8.10e-01\\
Abell 400 &   extr &     73 &   0.18 &  162.8 $\pm$    3.9 &   41.6 &   35.3 $\pm$    5.7 &   1.76 $\pm$   0.28 &     70 &   0.71 & 1.00e+00\\
 &      - & - & - &    0.0 & - &  205.9 $\pm$    2.1 &   0.17 $\pm$   0.01 &     71 &  57.23 & 8.82e-01\\
 &   flat & - & - &  162.8 $\pm$    3.9 &   41.6 &   35.3 $\pm$    5.7 &   1.76 $\pm$   0.28 &     70 &   0.71 & 1.00e+00\\
 &      - & - & - &    0.0 & - &  205.9 $\pm$    2.1 &   0.17 $\pm$   0.01 &     71 &  57.23 & 8.82e-01\\
Abell 401 &   extr &     60 &   0.40 &  162.5 $\pm$    7.9 &   20.7 &   86.0 $\pm$   10.7 &   1.37 $\pm$   0.11 &     57 &   8.70 & 1.00e+00\\
 &      - & - & - &    0.0 & - &  290.7 $\pm$    4.7 &   0.43 $\pm$   0.02 &     58 & 134.73 & 4.81e-08\\
 &   flat & - & - &  166.9 $\pm$    7.7 &   21.7 &   81.8 $\pm$   10.4 &   1.40 $\pm$   0.11 &     57 &   8.36 & 1.00e+00\\
 &      - & - & - &    0.0 & - &  292.0 $\pm$    4.7 &   0.42 $\pm$   0.02 &     58 & 142.56 & 4.50e-09\\
Abell 426 &   extr &     56 &   0.10 &   19.4 $\pm$    0.2 &  124.3 &  119.9 $\pm$    0.5 &   1.74 $\pm$   0.01 &     53 & 1040.29 & 3.10e-183\\
 &      - & - & - &    0.0 & - &  112.3 $\pm$    0.3 &   0.92 $\pm$   0.00 &     54 & 6430.00 & 0.00e+00\\
 &   flat & - & - &   19.4 $\pm$    0.2 &  124.4 &  119.9 $\pm$    0.5 &   1.74 $\pm$   0.01 &     53 & 1045.73 & 2.32e-184\\
 &      - & - & - &    0.0 & - &  112.3 $\pm$    0.3 &   0.92 $\pm$   0.00 &     54 & 6447.72 & 0.00e+00\\
Abell 478 &   extr &     49 &   0.40 &    6.9 $\pm$    0.9 &    7.5 &  123.4 $\pm$    2.6 &   0.96 $\pm$   0.02 &     46 &  20.38 & 1.00e+00\\
 &      - & - & - &    0.0 & - &  136.7 $\pm$    1.7 &   0.84 $\pm$   0.01 &     47 &  66.62 & 3.13e-02\\
 &   flat & - & - &    7.8 $\pm$    0.9 &    8.5 &  122.0 $\pm$    2.6 &   0.97 $\pm$   0.02 &     46 &  22.58 & 9.99e-01\\
 &      - & - & - &    0.0 & - &  137.0 $\pm$    1.7 &   0.84 $\pm$   0.01 &     47 &  81.79 & 1.25e-03\\
Abell 496 &   extr &     26 &   0.08 &    4.3 $\pm$    0.8 &    5.7 &  206.1 $\pm$    9.2 &   1.13 $\pm$   0.04 &     23 &   7.05 & 9.99e-01\\
 &      - & - & - &    0.0 & - &  182.9 $\pm$    6.6 &   0.94 $\pm$   0.02 &     24 &  36.09 & 5.38e-02\\
 &   flat & - & - &    8.9 $\pm$    0.7 &   13.4 &  216.3 $\pm$   10.5 &   1.27 $\pm$   0.05 &     23 &   6.95 & 1.00e+00\\
 &      - & - & - &    0.0 & - &  161.2 $\pm$    5.8 &   0.83 $\pm$   0.02 &     24 & 132.18 & 6.24e-17\\
Abell 520 &   extr &     33 &   0.55 &  325.5 $\pm$   29.2 &   11.1 &   10.2 $\pm$   11.8 &   2.09 $\pm$   0.71 &     30 &   2.86 & 1.00e+00\\
 &      - & - & - &    0.0 & - &  328.7 $\pm$   18.7 &   0.29 $\pm$   0.05 &     31 &  14.09 & 9.96e-01\\
 &   flat & - & - &  325.5 $\pm$   29.2 &   11.1 &   10.2 $\pm$   11.8 &   2.09 $\pm$   0.71 &     30 &   2.86 & 1.00e+00\\
 &      - & - & - &    0.0 & - &  328.7 $\pm$   18.7 &   0.29 $\pm$   0.05 &     31 &  14.09 & 9.96e-01\\
Abell 521 &   extr &      8 &   0.15 &  201.6 $\pm$   36.1 &    5.6 &  235.7 $\pm$   61.8 &   1.92 $\pm$   0.72 &      5 &   0.23 & 9.99e-01\\
 &      - & - & - &    0.0 & - &  420.3 $\pm$   37.9 &   0.44 $\pm$   0.10 &      6 &   9.70 & 1.38e-01\\
 &   flat & - & - &  259.9 $\pm$   36.2 &    7.2 &  245.4 $\pm$   61.8 &   1.91 $\pm$   0.69 &      5 &   0.32 & 9.97e-01\\
 &      - & - & - &    0.0 & - &  481.0 $\pm$   37.3 &   0.35 $\pm$   0.08 &      6 &  11.51 & 7.39e-02\\
Abell 539 &   extr &     11 &   0.03 &   19.6 $\pm$    4.0 &    4.9 &  552.4 $\pm$  198.3 &   1.14 $\pm$   0.21 &      8 &   1.80 & 9.86e-01\\
 &      - & - & - &    0.0 & - &  241.9 $\pm$   31.9 &   0.58 $\pm$   0.05 &      9 &  10.03 & 3.48e-01\\
 &   flat & - & - &   22.6 $\pm$    4.5 &    5.0 &  493.3 $\pm$  165.6 &   1.05 $\pm$   0.20 &      8 &   2.12 & 9.77e-01\\
 &      - & - & - &    0.0 & - &  234.5 $\pm$   27.5 &   0.53 $\pm$   0.04 &      9 &  10.08 & 3.44e-01\\
Abell 562 &   extr &     27 &   0.27 &  202.1 $\pm$   39.3 &    5.1 &   34.6 $\pm$   45.3 &   1.09 $\pm$   1.19 &     24 &   1.66 & 1.00e+00\\
 &      - & - & - &    0.0 & - &  244.4 $\pm$    9.7 &   0.13 $\pm$   0.06 &     25 &   2.41 & 1.00e+00\\
 &   flat & - & - &  202.1 $\pm$   39.3 &    5.1 &   34.6 $\pm$   45.3 &   1.09 $\pm$   1.19 &     24 &   1.66 & 1.00e+00\\
 &      - & - & - &    0.0 & - &  244.4 $\pm$    9.7 &   0.13 $\pm$   0.06 &     25 &   2.41 & 1.00e+00\\
Abell 576 &   extr &     21 &   0.08 &   78.4 $\pm$   18.7 &    4.2 &  230.6 $\pm$   26.6 &   1.19 $\pm$   0.34 &     18 &   3.81 & 1.00e+00\\
 &      - & - & - &    0.0 & - &  259.8 $\pm$   16.1 &   0.51 $\pm$   0.06 &     19 &  10.60 & 9.37e-01\\
 &   flat & - & - &   95.3 $\pm$   15.4 &    6.2 &  221.2 $\pm$   31.5 &   1.41 $\pm$   0.41 &     18 &   4.71 & 9.99e-01\\
 &      - & - & - &    0.0 & - &  247.8 $\pm$   15.2 &   0.45 $\pm$   0.06 &     19 &  15.49 & 6.91e-01\\
Abell 586 &   extr &     17 &   0.25 &   94.7 $\pm$   19.2 &    4.9 &   92.1 $\pm$   25.5 &   1.25 $\pm$   0.32 &     14 &   3.47 & 9.98e-01\\
 &      - & - & - &    0.0 & - &  201.4 $\pm$    7.2 &   0.53 $\pm$   0.06 &     15 &  10.34 & 7.98e-01\\
 &   flat & - & - &   94.7 $\pm$   19.2 &    4.9 &   92.1 $\pm$   25.5 &   1.25 $\pm$   0.32 &     14 &   3.47 & 9.98e-01\\
 &      - & - & - &    0.0 & - &  201.4 $\pm$    7.2 &   0.53 $\pm$   0.06 &     15 &  10.34 & 7.98e-01\\
Abell 611 &   extr &     19 &   0.40 &  124.9 $\pm$   18.6 &    6.7 &  164.4 $\pm$   31.5 &   1.25 $\pm$   0.20 &     16 &   1.98 & 1.00e+00\\
 &      - & - & - &    0.0 & - &  326.7 $\pm$   15.2 &   0.53 $\pm$   0.05 &     17 &  14.90 & 6.02e-01\\
 &   flat & - & - &  124.9 $\pm$   18.6 &    6.7 &  164.4 $\pm$   31.5 &   1.25 $\pm$   0.20 &     16 &   1.98 & 1.00e+00\\
 &      - & - & - &    0.0 & - &  326.7 $\pm$   15.2 &   0.53 $\pm$   0.05 &     17 &  14.90 & 6.02e-01\\
Abell 644 &   extr &     53 &   0.35 &  132.4 $\pm$    9.1 &   14.5 &   85.9 $\pm$   11.7 &   1.55 $\pm$   0.13 &     50 &  15.09 & 1.00e+00\\
 &      - & - & - &    0.0 & - &  244.8 $\pm$    4.3 &   0.68 $\pm$   0.03 &     51 &  90.43 & 5.59e-04\\
 &   flat & - & - &  132.4 $\pm$    9.1 &   14.5 &   85.9 $\pm$   11.7 &   1.55 $\pm$   0.13 &     50 &  15.09 & 1.00e+00\\
 &      - & - & - &    0.0 & - &  244.8 $\pm$    4.3 &   0.68 $\pm$   0.03 &     51 &  90.43 & 5.59e-04\\
Abell 665 &   extr &     46 &   0.70 &  134.6 $\pm$   23.5 &    5.7 &  106.3 $\pm$   25.1 &   1.06 $\pm$   0.13 &     43 &   3.79 & 1.00e+00\\
 &      - & - & - &    0.0 & - &  254.8 $\pm$   10.1 &   0.61 $\pm$   0.04 &     44 &  19.71 & 9.99e-01\\
 &   flat & - & - &  134.6 $\pm$   23.5 &    5.7 &  106.3 $\pm$   25.1 &   1.06 $\pm$   0.13 &     43 &   3.79 & 1.00e+00\\
 &      - & - & - &    0.0 & - &  254.8 $\pm$   10.1 &   0.61 $\pm$   0.04 &     44 &  19.71 & 9.99e-01\\
Abell 697 &   extr &     30 &   0.60 &  161.0 $\pm$   24.7 &    6.5 &  111.1 $\pm$   29.5 &   1.09 $\pm$   0.18 &     27 &   4.01 & 1.00e+00\\
 &      - & - & - &    0.0 & - &  310.0 $\pm$   13.4 &   0.46 $\pm$   0.04 &     28 &  19.49 & 8.82e-01\\
 &   flat & - & - &  166.7 $\pm$   24.4 &    6.8 &  108.2 $\pm$   29.1 &   1.10 $\pm$   0.18 &     27 &   4.28 & 1.00e+00\\
 &      - & - & - &    0.0 & - &  313.9 $\pm$   13.3 &   0.45 $\pm$   0.04 &     28 &  20.28 & 8.54e-01\\
Abell 744 &   extr &     18 &   0.12 &   60.3 $\pm$    9.4 &    6.4 &  227.9 $\pm$   15.4 &   0.83 $\pm$   0.13 &     15 &   1.20 & 1.00e+00\\
 &      - & - & - &    0.0 & - &  251.0 $\pm$   11.7 &   0.41 $\pm$   0.03 &     16 &  13.36 & 6.46e-01\\
 &   flat & - & - &   63.4 $\pm$   10.2 &    6.2 &  229.3 $\pm$   15.2 &   0.79 $\pm$   0.13 &     15 &   1.27 & 1.00e+00\\
 &      - & - & - &    0.0 & - &  256.9 $\pm$   11.5 &   0.39 $\pm$   0.02 &     16 &  12.56 & 7.05e-01\\
Abell 754 &   extr &     58 &   0.30 &  270.4 $\pm$   23.8 &   11.4 &   69.7 $\pm$   26.5 &   1.48 $\pm$   0.34 &     55 &  13.35 & 1.00e+00\\
 &      - & - & - &    0.0 & - &  366.4 $\pm$    8.1 &   0.34 $\pm$   0.03 &     56 &  35.36 & 9.86e-01\\
 &   flat & - & - &  270.4 $\pm$   23.8 &   11.4 &   69.7 $\pm$   26.5 &   1.48 $\pm$   0.34 &     55 &  13.35 & 1.00e+00\\
 &      - & - & - &    0.0 & - &  366.4 $\pm$    8.1 &   0.34 $\pm$   0.03 &     56 &  35.36 & 9.86e-01\\
Abell 773 &   extr &     35 &   0.60 &  244.3 $\pm$   31.7 &    7.7 &   41.1 $\pm$   22.5 &   1.60 $\pm$   0.33 &     32 &   3.28 & 1.00e+00\\
 &      - & - & - &    0.0 & - &  283.2 $\pm$   16.6 &   0.54 $\pm$   0.06 &     33 &  19.39 & 9.71e-01\\
 &   flat & - & - &  244.3 $\pm$   31.7 &    7.7 &   41.1 $\pm$   22.5 &   1.60 $\pm$   0.33 &     32 &   3.28 & 1.00e+00\\
 &      - & - & - &    0.0 & - &  283.2 $\pm$   16.6 &   0.54 $\pm$   0.06 &     33 &  19.39 & 9.71e-01\\
Abell 907 &   extr &     31 &   0.40 &   20.4 $\pm$    3.3 &    6.1 &  191.5 $\pm$    8.1 &   1.02 $\pm$   0.05 &     28 &   7.33 & 1.00e+00\\
 &      - & - & - &    0.0 & - &  223.9 $\pm$    5.4 &   0.81 $\pm$   0.02 &     29 &  32.96 & 2.79e-01\\
 &   flat & - & - &   23.4 $\pm$    3.2 &    7.3 &  187.0 $\pm$    8.1 &   1.05 $\pm$   0.05 &     28 &   7.62 & 1.00e+00\\
 &      - & - & - &    0.0 & - &  224.1 $\pm$    5.4 &   0.79 $\pm$   0.02 &     29 &  41.74 & 5.92e-02\\
Abell 963 &   extr &     24 &   0.40 &   22.0 $\pm$   15.7 &    1.4 &  205.5 $\pm$   22.9 &   0.79 $\pm$   0.09 &     21 &   2.75 & 1.00e+00\\
 &      - & - & - &    0.0 & - &  234.8 $\pm$    7.8 &   0.68 $\pm$   0.04 &     22 &   4.30 & 1.00e+00\\
 &   flat & - & - &   55.8 $\pm$   12.9 &    4.3 &  169.1 $\pm$   20.3 &   0.90 $\pm$   0.10 &     21 &   3.37 & 1.00e+00\\
 &      - & - & - &    0.0 & - &  244.6 $\pm$    7.6 &   0.61 $\pm$   0.03 &     22 &  13.86 & 9.06e-01\\
Abell 1060 &   extr &     25 &   0.03 &   58.1 $\pm$    8.8 &    6.6 &  138.8 $\pm$   40.0 &   0.80 $\pm$   0.30 &     22 &   1.55 & 1.00e+00\\
 &      - & - & - &    0.0 & - &  134.9 $\pm$    7.7 &   0.21 $\pm$   0.03 &     23 &   7.68 & 9.99e-01\\
 &   flat & - & - &   72.0 $\pm$    5.2 &   13.8 &  178.3 $\pm$  100.9 &   1.25 $\pm$   0.49 &     22 &   2.61 & 1.00e+00\\
 &      - & - & - &    0.0 & - &  121.7 $\pm$    6.6 &   0.15 $\pm$   0.02 &     23 &  13.85 & 9.31e-01\\
Abell 1063S &   extr &     24 &   0.60 &  169.6 $\pm$   19.7 &    8.6 &   42.2 $\pm$   17.7 &   1.72 $\pm$   0.27 &     21 &   2.98 & 1.00e+00\\
 &      - & - & - &    0.0 & - &  235.3 $\pm$   13.3 &   0.63 $\pm$   0.06 &     22 &  34.40 & 4.47e-02\\
 &   flat & - & - &  169.6 $\pm$   19.7 &    8.6 &   42.2 $\pm$   17.7 &   1.72 $\pm$   0.27 &     21 &   2.98 & 1.00e+00\\
 &      - & - & - &    0.0 & - &  235.3 $\pm$   13.3 &   0.63 $\pm$   0.06 &     22 &  34.40 & 4.47e-02\\
Abell 1068 &   extr &     17 &   0.20 &    9.0 $\pm$    1.0 &    8.7 &  108.9 $\pm$    3.2 &   1.31 $\pm$   0.06 &     14 &   3.45 & 9.98e-01\\
 &      - & - & - &    0.0 & - &  116.5 $\pm$    3.0 &   0.96 $\pm$   0.03 &     15 &  53.28 & 3.46e-06\\
 &   flat & - & - &    9.1 $\pm$    1.0 &    8.8 &  108.8 $\pm$    3.2 &   1.31 $\pm$   0.06 &     14 &   3.44 & 9.98e-01\\
 &      - & - & - &    0.0 & - &  116.5 $\pm$    3.0 &   0.96 $\pm$   0.03 &     15 &  54.19 & 2.44e-06\\
Abell 1201 &   extr &     14 &   0.20 &   39.2 $\pm$   14.0 &    2.8 &  200.4 $\pm$   23.8 &   1.20 $\pm$   0.21 &     11 &   1.60 & 1.00e+00\\
 &      - & - & - &    0.0 & - &  245.2 $\pm$   15.1 &   0.81 $\pm$   0.08 &     12 &   6.57 & 8.85e-01\\
 &   flat & - & - &   64.8 $\pm$   16.9 &    3.8 &  198.9 $\pm$   25.2 &   1.03 $\pm$   0.21 &     11 &   2.19 & 9.98e-01\\
 &      - & - & - &    0.0 & - &  262.1 $\pm$   15.3 &   0.56 $\pm$   0.05 &     12 &   8.39 & 7.54e-01\\
Abell 1204 &   extr &     11 &   0.15 &   14.1 $\pm$    1.5 &    9.5 &   83.1 $\pm$    3.6 &   1.35 $\pm$   0.11 &      8 &   1.62 & 9.91e-01\\
 &      - & - & - &    0.0 & - &   87.9 $\pm$    3.2 &   0.75 $\pm$   0.03 &      9 &  54.35 & 1.62e-08\\
 &   flat & - & - &   15.3 $\pm$    1.4 &   10.8 &   81.8 $\pm$    3.6 &   1.40 $\pm$   0.11 &      8 &   1.91 & 9.84e-01\\
 &      - & - & - &    0.0 & - &   86.7 $\pm$    3.2 &   0.73 $\pm$   0.03 &      9 &  65.62 & 1.09e-10\\
Abell 1240 &   extr &     37 &   0.50 &  429.4 $\pm$   46.9 &    9.1 &   16.9 $\pm$   28.8 &   1.96 $\pm$   1.14 &     34 &   0.06 & 1.00e+00\\
 &      - & - & - &    0.0 & - &  482.7 $\pm$   27.4 &   0.17 $\pm$   0.06 &     35 &   4.78 & 1.00e+00\\
 &   flat & - & - &  462.4 $\pm$   41.7 &   11.1 &    8.3 $\pm$   18.2 &   2.37 $\pm$   1.48 &     34 &   0.03 & 1.00e+00\\
 &      - & - & - &    0.0 & - &  504.2 $\pm$   26.9 &   0.13 $\pm$   0.05 &     35 &   4.76 & 1.00e+00\\
Abell 1361 &   extr &     14 &   0.15 &   14.8 $\pm$    4.3 &    3.4 &  119.2 $\pm$   10.7 &   1.15 $\pm$   0.19 &     11 &   3.47 & 9.83e-01\\
 &      - & - & - &    0.0 & - &  121.7 $\pm$    9.4 &   0.74 $\pm$   0.06 &     12 &  12.04 & 4.43e-01\\
 &   flat & - & - &   18.6 $\pm$    4.9 &    3.8 &  117.9 $\pm$   10.5 &   1.06 $\pm$   0.18 &     11 &   4.08 & 9.68e-01\\
 &      - & - & - &    0.0 & - &  122.2 $\pm$    8.9 &   0.63 $\pm$   0.05 &     12 &  13.17 & 3.57e-01\\
Abell 1413 &   extr &     10 &   0.12 &   29.8 $\pm$   13.9 &    2.1 &  158.2 $\pm$   14.7 &   0.82 $\pm$   0.20 &      7 &   5.97 & 5.43e-01\\
 &      - & - & - &    0.0 & - &  179.6 $\pm$   10.0 &   0.54 $\pm$   0.05 &      8 &  11.45 & 1.77e-01\\
 &   flat & - & - &   64.0 $\pm$    8.3 &    7.7 &  123.2 $\pm$   13.0 &   1.19 $\pm$   0.28 &      7 &   6.18 & 5.19e-01\\
 &      - & - & - &    0.0 & - &  164.1 $\pm$    9.2 &   0.38 $\pm$   0.04 &      8 &  25.44 & 1.31e-03\\
Abell 1423 &   extr &     23 &   0.40 &   58.8 $\pm$   12.6 &    4.7 &  124.8 $\pm$   20.9 &   1.22 $\pm$   0.17 &     20 &   1.75 & 1.00e+00\\
 &      - & - & - &    0.0 & - &  205.5 $\pm$    9.7 &   0.73 $\pm$   0.06 &     21 &  15.66 & 7.88e-01\\
 &   flat & - & - &   68.3 $\pm$   12.9 &    5.3 &  124.2 $\pm$   21.1 &   1.20 $\pm$   0.17 &     20 &   1.67 & 1.00e+00\\
 &      - & - & - &    0.0 & - &  215.6 $\pm$    9.7 &   0.65 $\pm$   0.05 &     21 &  17.39 & 6.87e-01\\
Abell 1446 &   extr &     34 &   0.32 &  152.4 $\pm$   43.8 &    3.5 &  119.5 $\pm$   49.5 &   0.67 $\pm$   0.27 &     31 &   6.87 & 1.00e+00\\
 &      - & - & - &    0.0 & - &  282.4 $\pm$    8.4 &   0.26 $\pm$   0.04 &     32 &   9.71 & 1.00e+00\\
 &   flat & - & - &  152.4 $\pm$   43.8 &    3.5 &  119.5 $\pm$   49.5 &   0.67 $\pm$   0.27 &     31 &   6.87 & 1.00e+00\\
 &      - & - & - &    0.0 & - &  282.4 $\pm$    8.4 &   0.26 $\pm$   0.04 &     32 &   9.71 & 1.00e+00\\
Abell 1569 &   extr &     29 &   0.20 &  110.1 $\pm$   27.8 &    4.0 &  149.1 $\pm$   28.9 &   0.51 $\pm$   0.19 &     26 &   7.39 & 1.00e+00\\
 &      - & - & - &    0.0 & - &  253.7 $\pm$    9.5 &   0.20 $\pm$   0.02 &     27 &   9.59 & 9.99e-01\\
 &   flat & - & - &  110.1 $\pm$   27.8 &    4.0 &  149.1 $\pm$   28.9 &   0.51 $\pm$   0.19 &     26 &   7.39 & 1.00e+00\\
 &      - & - & - &    0.0 & - &  253.7 $\pm$    9.5 &   0.20 $\pm$   0.02 &     27 &   9.59 & 9.99e-01\\
Abell 1576 &   extr &     33 &   0.70 &  174.1 $\pm$   49.7 &    3.5 &  102.3 $\pm$   48.5 &   1.36 $\pm$   0.29 &     30 &  41.88 & 7.32e-02\\
 &      - & - & - &    0.0 & - &  286.9 $\pm$   27.0 &   0.77 $\pm$   0.09 &     31 & 250.93 & 2.94e-36\\
 &   flat & - & - &  186.2 $\pm$   49.1 &    3.8 &   98.3 $\pm$   47.6 &   1.38 $\pm$   0.29 &     30 &  41.62 & 7.71e-02\\
 &      - & - & - &    0.0 & - &  297.3 $\pm$   26.9 &   0.74 $\pm$   0.09 &     31 & 272.38 & 2.10e-40\\
Abell 1644 &   extr &     11 &   0.05 &   10.7 $\pm$    1.3 &    8.2 &  511.4 $\pm$   61.2 &   1.54 $\pm$   0.10 &      8 &   0.50 & 1.00e+00\\
 &      - & - & - &    0.0 & - &  293.9 $\pm$   22.4 &   1.02 $\pm$   0.04 &      9 &  43.93 & 1.45e-06\\
 &   flat & - & - &   19.0 $\pm$    1.2 &   16.4 &  585.7 $\pm$   81.8 &   1.76 $\pm$   0.11 &      8 &   1.25 & 9.96e-01\\
 &      - & - & - &    0.0 & - &  177.6 $\pm$   12.5 &   0.71 $\pm$   0.03 &      9 & 108.10 & 3.58e-19\\
Abell 1650 &   extr &     15 &   0.12 &   32.7 $\pm$   10.8 &    3.0 &  164.9 $\pm$   12.3 &   0.80 $\pm$   0.16 &     12 &   1.85 & 1.00e+00\\
 &      - & - & - &    0.0 & - &  185.9 $\pm$    9.1 &   0.49 $\pm$   0.04 &     13 &   6.09 & 9.43e-01\\
 &   flat & - & - &   38.0 $\pm$   10.0 &    3.8 &  159.9 $\pm$   12.1 &   0.84 $\pm$   0.17 &     12 &   2.00 & 9.99e-01\\
 &      - & - & - &    0.0 & - &  183.7 $\pm$    9.0 &   0.47 $\pm$   0.04 &     13 &   7.85 & 8.53e-01\\
Abell 1651 &   extr &     27 &   0.20 &   87.7 $\pm$   11.2 &    7.8 &  117.3 $\pm$   15.3 &   0.96 $\pm$   0.18 &     24 &  13.05 & 9.65e-01\\
 &      - & - & - &    0.0 & - &  207.6 $\pm$    6.7 &   0.34 $\pm$   0.03 &     25 &  28.85 & 2.70e-01\\
 &   flat & - & - &   89.5 $\pm$   11.1 &    8.1 &  115.5 $\pm$   15.2 &   0.97 $\pm$   0.19 &     24 &  13.26 & 9.62e-01\\
 &      - & - & - &    0.0 & - &  207.6 $\pm$    6.7 &   0.34 $\pm$   0.03 &     25 &  29.42 & 2.47e-01\\
Abell 1664 &   extr &     13 &   0.15 &   10.0 $\pm$    1.1 &    9.1 &  142.7 $\pm$    5.9 &   1.50 $\pm$   0.08 &     10 &  27.58 & 2.11e-03\\
 &      - & - & - &    0.0 & - &  127.9 $\pm$    4.9 &   0.97 $\pm$   0.03 &     11 &  82.78 & 4.27e-13\\
 &   flat & - & - &   14.4 $\pm$    1.0 &   14.8 &  141.8 $\pm$    6.1 &   1.70 $\pm$   0.09 &     10 &  16.24 & 9.31e-02\\
 &      - & - & - &    0.0 & - &  117.2 $\pm$    4.6 &   0.85 $\pm$   0.03 &     11 & 127.13 & 6.63e-22\\
Abell 1689 &   extr &     20 &   0.30 &   78.4 $\pm$    7.6 &   10.4 &  111.8 $\pm$   13.8 &   1.35 $\pm$   0.14 &     17 &   7.34 & 9.79e-01\\
 &      - & - & - &    0.0 & - &  218.8 $\pm$    6.3 &   0.62 $\pm$   0.03 &     18 &  52.72 & 2.90e-05\\
 &   flat & - & - &   78.4 $\pm$    7.6 &   10.4 &  111.8 $\pm$   13.8 &   1.35 $\pm$   0.14 &     17 &   7.34 & 9.79e-01\\
 &      - & - & - &    0.0 & - &  218.8 $\pm$    6.3 &   0.62 $\pm$   0.03 &     18 &  52.72 & 2.90e-05\\
Abell 1736 &   extr &     15 &   0.10 &  150.4 $\pm$   38.3 &    3.9 &  127.3 $\pm$   37.9 &   0.99 $\pm$   0.83 &     12 &   0.10 & 1.00e+00\\
 &      - & - & - &    0.0 & - &  251.9 $\pm$   19.2 &   0.20 $\pm$   0.06 &     13 &   1.58 & 1.00e+00\\
 &   flat & - & - &  150.4 $\pm$   38.3 &    3.9 &  127.3 $\pm$   37.9 &   0.99 $\pm$   0.83 &     12 &   0.10 & 1.00e+00\\
 &      - & - & - &    0.0 & - &  251.9 $\pm$   19.2 &   0.20 $\pm$   0.06 &     13 &   1.58 & 1.00e+00\\
Abell 1758 &   extr &     20 &   0.40 &  116.8 $\pm$   44.3 &    2.6 &  218.0 $\pm$   58.6 &   1.03 $\pm$   0.24 &     17 &   0.61 & 1.00e+00\\
 &      - & - & - &    0.0 & - &  361.7 $\pm$   20.8 &   0.62 $\pm$   0.08 &     18 &   4.61 & 9.99e-01\\
 &   flat & - & - &  230.8 $\pm$   37.2 &    6.2 &  144.0 $\pm$   50.2 &   1.21 $\pm$   0.32 &     17 &   1.98 & 1.00e+00\\
 &      - & - & - &    0.0 & - &  417.8 $\pm$   20.2 &   0.36 $\pm$   0.06 &     18 &   9.94 & 9.34e-01\\
Abell 1763 &   extr &     39 &   0.60 &  214.7 $\pm$   32.8 &    6.5 &   70.8 $\pm$   29.1 &   1.37 $\pm$   0.25 &     36 &   2.87 & 1.00e+00\\
 &      - & - & - &    0.0 & - &  288.8 $\pm$   13.8 &   0.60 $\pm$   0.05 &     37 &  18.21 & 9.96e-01\\
 &   flat & - & - &  214.7 $\pm$   32.8 &    6.5 &   70.8 $\pm$   29.1 &   1.37 $\pm$   0.25 &     36 &   2.87 & 1.00e+00\\
 &      - & - & - &    0.0 & - &  288.8 $\pm$   13.8 &   0.60 $\pm$   0.05 &     37 &  18.21 & 9.96e-01\\
Abell 1795 &   extr &     53 &   0.30 &   18.4 $\pm$    1.1 &   17.4 &  131.4 $\pm$    2.8 &   1.17 $\pm$   0.03 &     50 &  33.33 & 9.66e-01\\
 &      - & - & - &    0.0 & - &  158.9 $\pm$    2.0 &   0.86 $\pm$   0.01 &     51 & 271.73 & 7.10e-32\\
 &   flat & - & - &   19.0 $\pm$    1.1 &   18.1 &  130.4 $\pm$    2.8 &   1.18 $\pm$   0.03 &     50 &  35.74 & 9.36e-01\\
 &      - & - & - &    0.0 & - &  158.8 $\pm$    2.0 &   0.86 $\pm$   0.01 &     51 & 292.75 & 1.18e-35\\
Abell 1835 &   extr &     16 &   0.30 &   10.9 $\pm$    2.5 &    4.4 &  112.6 $\pm$    7.9 &   1.25 $\pm$   0.09 &     13 &   8.46 & 8.12e-01\\
 &      - & - & - &    0.0 & - &  134.2 $\pm$    5.2 &   0.99 $\pm$   0.03 &     14 &  26.28 & 2.38e-02\\
 &   flat & - & - &   11.4 $\pm$    2.5 &    4.6 &  111.7 $\pm$    7.9 &   1.26 $\pm$   0.09 &     13 &   8.76 & 7.91e-01\\
 &      - & - & - &    0.0 & - &  134.3 $\pm$    5.3 &   0.98 $\pm$   0.03 &     14 &  28.26 & 1.32e-02\\
Abell 1914 &   extr &     29 &   0.40 &   63.3 $\pm$   22.3 &    2.8 &  175.5 $\pm$   32.3 &   0.88 $\pm$   0.14 &     26 &   3.91 & 1.00e+00\\
 &      - & - & - &    0.0 & - &  256.7 $\pm$   10.4 &   0.61 $\pm$   0.04 &     27 &   9.94 & 9.99e-01\\
 &   flat & - & - &  107.2 $\pm$   18.0 &    5.9 &  131.1 $\pm$   28.3 &   1.05 $\pm$   0.18 &     26 &   4.42 & 1.00e+00\\
 &      - & - & - &    0.0 & - &  269.8 $\pm$   10.3 &   0.52 $\pm$   0.04 &     27 &  21.84 & 7.45e-01\\
Abell 1942 &   extr &     12 &   0.22 &  107.7 $\pm$   77.7 &    1.4 &  194.1 $\pm$   88.7 &   0.66 $\pm$   0.41 &      9 &   1.21 & 9.99e-01\\
 &      - & - & - &    0.0 & - &  307.8 $\pm$   17.3 &   0.35 $\pm$   0.07 &     10 &   1.81 & 9.98e-01\\
 &   flat & - & - &  107.7 $\pm$   77.7 &    1.4 &  194.1 $\pm$   88.7 &   0.66 $\pm$   0.41 &      9 &   1.21 & 9.99e-01\\
 &      - & - & - &    0.0 & - &  307.8 $\pm$   17.3 &   0.35 $\pm$   0.07 &     10 &   1.81 & 9.98e-01\\
Abell 1991 &   extr &     19 &   0.10 &    1.0 $\pm$    0.3 &    3.0 &  151.4 $\pm$    4.1 &   1.04 $\pm$   0.03 &     16 &  31.46 & 1.18e-02\\
 &      - & - & - &    0.0 & - &  151.3 $\pm$    3.6 &   1.04 $\pm$   0.01 &     17 &  31.47 & 1.75e-02\\
 &   flat & - & - &    1.5 $\pm$    0.3 &    4.8 &  152.2 $\pm$    4.2 &   1.09 $\pm$   0.03 &     16 &  43.79 & 2.12e-04\\
 &      - & - & - &    0.0 & - &  143.7 $\pm$    3.4 &   0.99 $\pm$   0.01 &     17 &  64.00 & 2.26e-07\\
Abell 1995 &   extr &     26 &   0.60 &  374.3 $\pm$   60.1 &    6.2 &   26.8 $\pm$   32.9 &   2.08 $\pm$   0.81 &     23 &   0.99 & 1.00e+00\\
 &      - & - & - &    0.0 & - &  421.2 $\pm$   36.4 &   0.35 $\pm$   0.11 &     24 &   9.74 & 9.96e-01\\
 &   flat & - & - &  374.3 $\pm$   60.1 &    6.2 &   26.8 $\pm$   32.9 &   2.08 $\pm$   0.81 &     23 &   0.99 & 1.00e+00\\
 &      - & - & - &    0.0 & - &  421.2 $\pm$   36.4 &   0.35 $\pm$   0.11 &     24 &   9.74 & 9.96e-01\\
Abell 2029 &   extr &     58 &   0.40 &    6.1 $\pm$    0.7 &    8.7 &  169.9 $\pm$    2.1 &   0.92 $\pm$   0.01 &     55 &  82.78 & 9.09e-03\\
 &      - & - & - &    0.0 & - &  181.2 $\pm$    1.6 &   0.82 $\pm$   0.01 &     56 & 146.10 & 5.63e-10\\
 &   flat & - & - &   10.5 $\pm$    0.7 &   15.8 &  163.6 $\pm$    2.1 &   0.95 $\pm$   0.02 &     55 &  58.95 & 3.33e-01\\
 &      - & - & - &    0.0 & - &  182.6 $\pm$    1.6 &   0.78 $\pm$   0.01 &     56 & 235.51 & 7.10e-24\\
Abell 2034 &   extr &     67 &   0.50 &  215.8 $\pm$   25.1 &    8.6 &   99.1 $\pm$   25.3 &   1.05 $\pm$   0.16 &     64 &  11.63 & 1.00e+00\\
 &      - & - & - &    0.0 & - &  333.4 $\pm$    9.0 &   0.42 $\pm$   0.03 &     65 &  31.58 & 1.00e+00\\
 &   flat & - & - &  232.6 $\pm$   23.0 &   10.1 &   85.1 $\pm$   22.6 &   1.14 $\pm$   0.17 &     64 &  10.87 & 1.00e+00\\
 &      - & - & - &    0.0 & - &  338.1 $\pm$    8.9 &   0.41 $\pm$   0.03 &     65 &  35.48 & 9.99e-01\\
Abell 2052 &   extr &     29 &   0.10 &    8.9 $\pm$    0.7 &   13.2 &  164.8 $\pm$    2.6 &   1.23 $\pm$   0.03 &     26 & 374.86 & 1.67e-63\\
 &      - & - & - &    0.0 & - &  162.4 $\pm$    2.3 &   0.99 $\pm$   0.01 &     27 & 541.69 & 3.71e-97\\
 &   flat & - & - &    9.5 $\pm$    0.7 &   14.3 &  164.7 $\pm$    2.6 &   1.25 $\pm$   0.03 &     26 & 387.05 & 5.51e-66\\
 &      - & - & - &    0.0 & - &  162.1 $\pm$    2.3 &   0.99 $\pm$   0.01 &     27 & 580.67 & 3.03e-105\\
Abell 2063 &   extr &     52 &   0.18 &   53.5 $\pm$    2.6 &   20.6 &  129.0 $\pm$    3.9 &   1.07 $\pm$   0.05 &     49 &  37.82 & 8.77e-01\\
 &      - & - & - &    0.0 & - &  180.6 $\pm$    2.4 &   0.51 $\pm$   0.01 &     50 & 224.14 & 6.72e-24\\
 &   flat & - & - &   53.5 $\pm$    2.6 &   20.6 &  129.0 $\pm$    3.9 &   1.07 $\pm$   0.05 &     49 &  37.82 & 8.77e-01\\
 &      - & - & - &    0.0 & - &  180.6 $\pm$    2.4 &   0.51 $\pm$   0.01 &     50 & 224.14 & 6.72e-24\\
Abell 2065 &   extr &     29 &   0.20 &   33.1 $\pm$    6.9 &    4.8 &  206.9 $\pm$   10.8 &   0.97 $\pm$   0.09 &     26 &   7.99 & 1.00e+00\\
 &      - & - & - &    0.0 & - &  239.0 $\pm$    7.5 &   0.67 $\pm$   0.03 &     27 &  21.36 & 7.69e-01\\
 &   flat & - & - &   43.9 $\pm$    6.5 &    6.8 &  195.3 $\pm$   10.6 &   1.02 $\pm$   0.10 &     26 &   7.97 & 1.00e+00\\
 &      - & - & - &    0.0 & - &  236.5 $\pm$    7.5 &   0.60 $\pm$   0.03 &     27 &  29.46 & 3.39e-01\\
Abell 2069 &   extr &     39 &   0.40 &  416.2 $\pm$   41.8 &   10.0 &   82.4 $\pm$   46.0 &   1.22 $\pm$   0.41 &     36 &   5.75 & 1.00e+00\\
 &      - & - & - &    0.0 & - &  544.7 $\pm$   16.4 &   0.20 $\pm$   0.03 &     37 &  15.09 & 1.00e+00\\
 &   flat & - & - &  453.2 $\pm$   35.6 &   12.7 &   54.6 $\pm$   36.3 &   1.47 $\pm$   0.51 &     36 &   5.71 & 1.00e+00\\
 &      - & - & - &    0.0 & - &  557.2 $\pm$   16.2 &   0.17 $\pm$   0.03 &     37 &  16.52 & 9.99e-01\\
Abell 2104 &   extr &      9 &   0.12 &   98.0 $\pm$   57.6 &    1.7 &  276.2 $\pm$   59.7 &   0.94 $\pm$   0.55 &      6 &   0.64 & 9.96e-01\\
 &      - & - & - &    0.0 & - &  350.0 $\pm$   36.1 &   0.46 $\pm$   0.10 &      7 &   2.22 & 9.47e-01\\
 &   flat & - & - &  160.6 $\pm$   42.2 &    3.8 &  210.1 $\pm$   53.9 &   1.20 $\pm$   0.77 &      6 &   0.74 & 9.94e-01\\
 &      - & - & - &    0.0 & - &  331.9 $\pm$   33.4 &   0.30 $\pm$   0.08 &      7 &   3.39 & 8.47e-01\\
Abell 2107 &   extr &      6 &   0.03 &   18.0 $\pm$    4.7 &    3.8 &  473.9 $\pm$  117.3 &   1.03 $\pm$   0.16 &      3 &  13.10 & 4.42e-03\\
 &      - & - & - &    0.0 & - &  290.4 $\pm$   26.6 &   0.64 $\pm$   0.04 &      4 &  40.08 & 4.17e-08\\
 &   flat & - & - &   21.2 $\pm$    5.8 &    3.6 &  396.1 $\pm$   92.5 &   0.91 $\pm$   0.16 &      3 &  15.79 & 1.25e-03\\
 &      - & - & - &    0.0 & - &  263.6 $\pm$   21.3 &   0.55 $\pm$   0.03 &      4 &  43.05 & 1.01e-08\\
Abell 2111 &   extr &     22 &   0.40 &  107.4 $\pm$   97.3 &    1.1 &  194.0 $\pm$  118.7 &   0.65 $\pm$   0.38 &     19 &   1.06 & 1.00e+00\\
 &      - & - & - &    0.0 & - &  317.5 $\pm$   23.7 &   0.39 $\pm$   0.08 &     20 &   1.54 & 1.00e+00\\
 &   flat & - & - &  107.4 $\pm$   97.3 &    1.1 &  194.0 $\pm$  118.7 &   0.65 $\pm$   0.38 &     19 &   1.06 & 1.00e+00\\
 &      - & - & - &    0.0 & - &  317.5 $\pm$   23.7 &   0.39 $\pm$   0.08 &     20 &   1.54 & 1.00e+00\\
Abell 2124 &   extr &     19 &   0.12 &   88.7 $\pm$   24.2 &    3.7 &  272.5 $\pm$   30.8 &   0.89 $\pm$   0.27 &     16 &   2.86 & 1.00e+00\\
 &      - & - & - &    0.0 & - &  325.0 $\pm$   21.8 &   0.41 $\pm$   0.05 &     17 &   7.20 & 9.81e-01\\
 &   flat & - & - &   98.3 $\pm$   23.9 &    4.1 &  260.8 $\pm$   30.8 &   0.90 $\pm$   0.28 &     16 &   3.24 & 1.00e+00\\
 &      - & - & - &    0.0 & - &  320.8 $\pm$   21.3 &   0.37 $\pm$   0.05 &     17 &   7.78 & 9.71e-01\\
Abell 2125 &   extr &     10 &   0.20 &  225.2 $\pm$   32.0 &    7.0 &   32.9 $\pm$   41.2 &   1.35 $\pm$   1.73 &      7 &   0.06 & 1.00e+00\\
 &      - & - & - &    0.0 & - &  264.5 $\pm$   11.5 &   0.10 $\pm$   0.05 &      8 &   1.06 & 9.98e-01\\
 &   flat & - & - &  225.2 $\pm$   32.0 &    7.0 &   32.9 $\pm$   41.2 &   1.35 $\pm$   1.73 &      7 &   0.06 & 1.00e+00\\
 &      - & - & - &    0.0 & - &  264.5 $\pm$   11.5 &   0.10 $\pm$   0.05 &      8 &   1.06 & 9.98e-01\\
Abell 2142 &   extr &     75 &   0.30 &   58.5 $\pm$    2.7 &   21.7 &  132.5 $\pm$    4.5 &   1.13 $\pm$   0.04 &     72 &  17.26 & 1.00e+00\\
 &      - & - & - &    0.0 & - &  205.9 $\pm$    2.1 &   0.62 $\pm$   0.01 &     73 & 240.81 & 8.51e-20\\
 &   flat & - & - &   68.1 $\pm$    2.5 &   27.5 &  120.6 $\pm$    4.4 &   1.22 $\pm$   0.04 &     72 &  17.98 & 1.00e+00\\
 &      - & - & - &    0.0 & - &  206.1 $\pm$    2.2 &   0.58 $\pm$   0.01 &     73 & 335.00 & 3.31e-35\\
Abell 2147 &   extr &     57 &   0.20 &  151.9 $\pm$   27.2 &    5.6 &  136.2 $\pm$   30.5 &   0.55 $\pm$   0.19 &     54 &  31.13 & 9.95e-01\\
 &      - & - & - &    0.0 & - &  291.4 $\pm$    6.4 &   0.18 $\pm$   0.02 &     55 &  35.26 & 9.82e-01\\
 &   flat & - & - &  151.9 $\pm$   27.2 &    5.6 &  136.2 $\pm$   30.5 &   0.55 $\pm$   0.19 &     54 &  31.13 & 9.95e-01\\
 &      - & - & - &    0.0 & - &  291.4 $\pm$    6.4 &   0.18 $\pm$   0.02 &     55 &  35.26 & 9.82e-01\\
Abell 2151 &   extr &     20 &   0.07 &    1.7 $\pm$    3.0 &    0.6 &  137.9 $\pm$    6.0 &   0.61 $\pm$   0.06 &     17 &  36.84 & 3.54e-03\\
 &      - & - & - &    0.0 & - &  136.6 $\pm$    5.2 &   0.58 $\pm$   0.02 &     18 &  37.11 & 5.07e-03\\
 &   flat & - & - &    0.4 $\pm$    3.6 &    0.1 &  135.2 $\pm$    5.4 &   0.56 $\pm$   0.06 &     17 &  36.91 & 3.46e-03\\
 &      - & - & - &    0.0 & - &  135.0 $\pm$    5.0 &   0.55 $\pm$   0.02 &     18 &  36.92 & 5.37e-03\\
Abell 2163 &   extr &     42 &   0.60 &  437.3 $\pm$   82.7 &    5.3 &   72.5 $\pm$   50.8 &   1.86 $\pm$   0.43 &     39 &   7.08 & 1.00e+00\\
 &      - & - & - &    0.0 & - &  449.2 $\pm$   42.9 &   0.82 $\pm$   0.09 &     40 &  20.09 & 9.96e-01\\
 &   flat & - & - &  438.0 $\pm$   82.6 &    5.3 &   72.2 $\pm$   50.6 &   1.87 $\pm$   0.43 &     39 &   7.08 & 1.00e+00\\
 &      - & - & - &    0.0 & - &  449.3 $\pm$   42.9 &   0.82 $\pm$   0.09 &     40 &  20.17 & 9.96e-01\\
Abell 2199 &   extr &      7 &   0.02 &    7.6 $\pm$    0.8 &    9.1 &  423.7 $\pm$   95.3 &   1.38 $\pm$   0.12 &      4 &   3.72 & 4.45e-01\\
 &      - & - & - &    0.0 & - &  143.3 $\pm$   11.8 &   0.72 $\pm$   0.03 &      5 &  35.07 & 1.46e-06\\
 &   flat & - & - &   13.3 $\pm$    0.8 &   15.6 &  331.5 $\pm$   90.0 &   1.35 $\pm$   0.15 &      4 &  11.09 & 2.56e-02\\
 &      - & - & - &    0.0 & - &   81.8 $\pm$    5.2 &   0.44 $\pm$   0.02 &      5 &  45.17 & 1.34e-08\\
Abell 2204 &   extr &     15 &   0.20 &    9.7 $\pm$    0.9 &   11.1 &  166.2 $\pm$    6.0 &   1.41 $\pm$   0.05 &     12 &  22.73 & 3.01e-02\\
 &      - & - & - &    0.0 & - &  164.6 $\pm$    5.9 &   1.02 $\pm$   0.02 &     13 & 102.32 & 5.88e-16\\
 &   flat & - & - &    9.7 $\pm$    0.9 &   11.1 &  166.2 $\pm$    6.0 &   1.41 $\pm$   0.05 &     12 &  22.73 & 3.01e-02\\
 &      - & - & - &    0.0 & - &  164.6 $\pm$    5.9 &   1.02 $\pm$   0.02 &     13 & 102.32 & 5.88e-16\\
Abell 2218 &   extr &     42 &   0.60 &  288.6 $\pm$   20.0 &   14.4 &   10.7 $\pm$    7.1 &   2.35 $\pm$   0.41 &     39 &   4.83 & 1.00e+00\\
 &      - & - & - &    0.0 & - &  294.5 $\pm$   14.7 &   0.41 $\pm$   0.05 &     40 &  39.78 & 4.80e-01\\
 &   flat & - & - &  288.6 $\pm$   20.0 &   14.4 &   10.7 $\pm$    7.1 &   2.35 $\pm$   0.41 &     39 &   4.83 & 1.00e+00\\
 &      - & - & - &    0.0 & - &  294.5 $\pm$   14.7 &   0.41 $\pm$   0.05 &     40 &  39.78 & 4.80e-01\\
Abell 2219 &   extr &     34 &   0.60 &  411.6 $\pm$   43.2 &    9.5 &   17.0 $\pm$   19.2 &   1.97 $\pm$   0.66 &     31 &   3.70 & 1.00e+00\\
 &      - & - & - &    0.0 & - &  407.6 $\pm$   26.4 &   0.36 $\pm$   0.06 &     32 &  19.62 & 9.58e-01\\
 &   flat & - & - &  411.6 $\pm$   43.2 &    9.5 &   17.0 $\pm$   19.2 &   1.97 $\pm$   0.66 &     31 &   3.70 & 1.00e+00\\
 &      - & - & - &    0.0 & - &  407.6 $\pm$   26.4 &   0.36 $\pm$   0.06 &     32 &  19.62 & 9.58e-01\\
Abell 2244 &   extr &     34 &   0.30 &   57.6 $\pm$    4.2 &   13.6 &  109.1 $\pm$    6.0 &   1.00 $\pm$   0.05 &     31 &  14.02 & 9.96e-01\\
 &      - & - & - &    0.0 & - &  180.0 $\pm$    2.1 &   0.56 $\pm$   0.02 &     32 & 102.67 & 2.46e-09\\
 &   flat & - & - &   57.6 $\pm$    4.2 &   13.6 &  109.1 $\pm$    6.0 &   1.00 $\pm$   0.05 &     31 &  14.02 & 9.96e-01\\
 &      - & - & - &    0.0 & - &  180.0 $\pm$    2.1 &   0.56 $\pm$   0.02 &     32 & 102.67 & 2.46e-09\\
Abell 2255 &   extr &     40 &   0.30 &  529.1 $\pm$   28.2 &   18.8 &    5.8 $\pm$   16.6 &   2.63 $\pm$   2.69 &     37 &   0.24 & 1.00e+00\\
 &      - & - & - &    0.0 & - &  553.0 $\pm$   14.0 &   0.05 $\pm$   0.03 &     38 &   2.79 & 1.00e+00\\
 &   flat & - & - &  529.1 $\pm$   28.2 &   18.8 &    5.8 $\pm$   16.6 &   2.63 $\pm$   2.69 &     37 &   0.24 & 1.00e+00\\
 &      - & - & - &    0.0 & - &  553.0 $\pm$   14.0 &   0.05 $\pm$   0.03 &     38 &   2.79 & 1.00e+00\\
Abell 2256 &   extr &     63 &   0.35 &  349.6 $\pm$   11.6 &   30.2 &    7.0 $\pm$    7.6 &   2.54 $\pm$   0.93 &     60 &   2.24 & 1.00e+00\\
 &      - & - & - &    0.0 & - &  378.4 $\pm$    6.9 &   0.08 $\pm$   0.02 &     61 &  21.60 & 1.00e+00\\
 &   flat & - & - &  349.6 $\pm$   11.6 &   30.2 &    7.0 $\pm$    7.6 &   2.54 $\pm$   0.93 &     60 &   2.24 & 1.00e+00\\
 &      - & - & - &    0.0 & - &  378.4 $\pm$    6.9 &   0.08 $\pm$   0.02 &     61 &  21.60 & 1.00e+00\\
Abell 2259 &   extr &     36 &   0.50 &  114.0 $\pm$   18.9 &    6.0 &   61.0 $\pm$   20.4 &   1.36 $\pm$   0.24 &     33 &   1.37 & 1.00e+00\\
 &      - & - & - &    0.0 & - &  189.0 $\pm$    8.7 &   0.63 $\pm$   0.05 &     34 &  15.77 & 9.97e-01\\
 &   flat & - & - &  114.0 $\pm$   18.9 &    6.0 &   61.0 $\pm$   20.4 &   1.36 $\pm$   0.24 &     33 &   1.37 & 1.00e+00\\
 &      - & - & - &    0.0 & - &  189.0 $\pm$    8.7 &   0.63 $\pm$   0.05 &     34 &  15.77 & 9.97e-01\\
Abell 2261 &   extr &     18 &   0.30 &   60.5 $\pm$    8.2 &    7.4 &  106.5 $\pm$   14.1 &   1.27 $\pm$   0.16 &     15 &   3.63 & 9.99e-01\\
 &      - & - & - &    0.0 & - &  189.6 $\pm$    6.6 &   0.61 $\pm$   0.04 &     16 &  28.62 & 2.67e-02\\
 &   flat & - & - &   61.1 $\pm$    8.1 &    7.5 &  106.0 $\pm$   14.1 &   1.27 $\pm$   0.16 &     15 &   3.62 & 9.99e-01\\
 &      - & - & - &    0.0 & - &  189.7 $\pm$    6.6 &   0.61 $\pm$   0.04 &     16 &  29.00 & 2.40e-02\\
Abell 2294 &   extr &     22 &   0.32 &  128.5 $\pm$   52.0 &    2.5 &  246.7 $\pm$   75.6 &   1.04 $\pm$   0.32 &     19 &   0.60 & 1.00e+00\\
 &      - & - & - &    0.0 & - &  409.8 $\pm$   28.7 &   0.57 $\pm$   0.09 &     20 &   3.67 & 1.00e+00\\
 &   flat & - & - &  156.3 $\pm$   52.7 &    3.0 &  235.7 $\pm$   76.3 &   1.03 $\pm$   0.33 &     19 &   0.83 & 1.00e+00\\
 &      - & - & - &    0.0 & - &  428.8 $\pm$   28.6 &   0.49 $\pm$   0.08 &     20 &   4.23 & 1.00e+00\\
Abell 2319 &   extr &     74 &   0.40 &  270.2 $\pm$    4.8 &   56.0 &   39.4 $\pm$    7.1 &   1.76 $\pm$   0.15 &     71 &   9.83 & 1.00e+00\\
 &      - & - & - &    0.0 & - &  363.1 $\pm$    4.3 &   0.19 $\pm$   0.01 &     72 & 212.75 & 7.89e-16\\
 &   flat & - & - &  270.2 $\pm$    4.8 &   56.0 &   39.4 $\pm$    7.1 &   1.76 $\pm$   0.15 &     71 &   9.83 & 1.00e+00\\
 &      - & - & - &    0.0 & - &  363.1 $\pm$    4.3 &   0.19 $\pm$   0.01 &     72 & 212.75 & 7.89e-16\\
Abell 2384 &   extr &     23 &   0.20 &   17.9 $\pm$    3.3 &    5.4 &  162.9 $\pm$    7.3 &   1.31 $\pm$   0.09 &     20 &   7.54 & 9.95e-01\\
 &      - & - & - &    0.0 & - &  179.6 $\pm$    6.3 &   0.99 $\pm$   0.04 &     21 &  29.61 & 1.00e-01\\
 &   flat & - & - &   38.5 $\pm$    3.0 &   13.0 &  139.2 $\pm$    7.3 &   1.49 $\pm$   0.11 &     20 &   7.85 & 9.93e-01\\
 &      - & - & - &    0.0 & - &  163.6 $\pm$    6.1 &   0.70 $\pm$   0.03 &     21 &  87.32 & 4.67e-10\\
Abell 2390 &   extr &     11 &   0.20 &   14.7 $\pm$    7.0 &    2.1 &  202.9 $\pm$   15.6 &   1.07 $\pm$   0.15 &      8 &   0.96 & 9.99e-01\\
 &      - & - & - &    0.0 & - &  214.4 $\pm$   13.9 &   0.84 $\pm$   0.05 &      9 &   4.71 & 8.59e-01\\
 &   flat & - & - &   14.7 $\pm$    7.0 &    2.1 &  202.9 $\pm$   15.6 &   1.07 $\pm$   0.15 &      8 &   0.96 & 9.99e-01\\
 &      - & - & - &    0.0 & - &  214.4 $\pm$   13.9 &   0.84 $\pm$   0.05 &      9 &   4.71 & 8.59e-01\\
Abell 2409 &   extr &     16 &   0.20 &   69.6 $\pm$   20.9 &    3.3 &  124.1 $\pm$   27.4 &   0.96 $\pm$   0.32 &     13 &   8.79 & 7.88e-01\\
 &      - & - & - &    0.0 & - &  198.6 $\pm$   10.2 &   0.45 $\pm$   0.06 &     14 &  15.23 & 3.62e-01\\
 &   flat & - & - &   73.8 $\pm$   20.7 &    3.6 &  120.8 $\pm$   27.3 &   0.97 $\pm$   0.33 &     13 &   9.06 & 7.68e-01\\
 &      - & - & - &    0.0 & - &  199.4 $\pm$   10.3 &   0.43 $\pm$   0.06 &     14 &  15.83 & 3.24e-01\\
Abell 2420 &   extr &     64 &   0.50 &  332.6 $\pm$   67.5 &    4.9 &   64.3 $\pm$   62.6 &   1.12 $\pm$   0.58 &     61 &   5.54 & 1.00e+00\\
 &      - & - & - &    0.0 & - &  411.0 $\pm$   22.4 &   0.28 $\pm$   0.06 &     62 &   9.20 & 1.00e+00\\
 &   flat & - & - &  332.6 $\pm$   67.5 &    4.9 &   64.3 $\pm$   62.6 &   1.12 $\pm$   0.58 &     61 &   5.54 & 1.00e+00\\
 &      - & - & - &    0.0 & - &  411.0 $\pm$   22.4 &   0.28 $\pm$   0.06 &     62 &   9.20 & 1.00e+00\\
Abell 2462 &   extr &     58 &   0.40 &  129.7 $\pm$   27.0 &    4.8 &   83.2 $\pm$   31.1 &   0.77 $\pm$   0.24 &     55 &   1.23 & 1.00e+00\\
 &      - & - & - &    0.0 & - &  224.1 $\pm$    6.2 &   0.30 $\pm$   0.03 &     56 &   7.73 & 1.00e+00\\
 &   flat & - & - &  129.7 $\pm$   27.0 &    4.8 &   83.2 $\pm$   31.1 &   0.77 $\pm$   0.24 &     55 &   1.23 & 1.00e+00\\
 &      - & - & - &    0.0 & - &  224.1 $\pm$    6.2 &   0.30 $\pm$   0.03 &     56 &   7.73 & 1.00e+00\\
Abell 2537 &   extr &     14 &   0.30 &  106.7 $\pm$   19.6 &    5.4 &  127.9 $\pm$   29.2 &   1.24 $\pm$   0.26 &     11 &   1.05 & 1.00e+00\\
 &      - & - & - &    0.0 & - &  259.9 $\pm$   11.9 &   0.51 $\pm$   0.06 &     12 &  12.70 & 3.91e-01\\
 &   flat & - & - &  110.4 $\pm$   19.4 &    5.7 &  124.7 $\pm$   29.0 &   1.26 $\pm$   0.27 &     11 &   1.05 & 1.00e+00\\
 &      - & - & - &    0.0 & - &  261.0 $\pm$   11.9 &   0.50 $\pm$   0.06 &     12 &  13.23 & 3.52e-01\\
Abell 2554 &   extr &     30 &   0.30 &  105.1 $\pm$   71.8 &    1.5 &  318.4 $\pm$   86.2 &   0.66 $\pm$   0.21 &     27 &   0.87 & 1.00e+00\\
 &      - & - & - &    0.0 & - &  436.9 $\pm$   18.7 &   0.45 $\pm$   0.05 &     28 &   1.98 & 1.00e+00\\
 &   flat & - & - &  105.1 $\pm$   71.8 &    1.5 &  318.4 $\pm$   86.2 &   0.66 $\pm$   0.21 &     27 &   0.87 & 1.00e+00\\
 &      - & - & - &    0.0 & - &  436.9 $\pm$   18.7 &   0.45 $\pm$   0.05 &     28 &   1.98 & 1.00e+00\\
Abell 2556 &   extr &     17 &   0.13 &   10.6 $\pm$    1.4 &    7.7 &  117.5 $\pm$    3.9 &   1.10 $\pm$   0.06 &     14 &   4.27 & 9.94e-01\\
 &      - & - & - &    0.0 & - &  116.2 $\pm$    3.5 &   0.76 $\pm$   0.02 &     15 &  44.30 & 9.85e-05\\
 &   flat & - & - &   12.4 $\pm$    1.3 &    9.2 &  115.8 $\pm$    4.0 &   1.13 $\pm$   0.07 &     14 &   4.50 & 9.92e-01\\
 &      - & - & - &    0.0 & - &  113.8 $\pm$    3.4 &   0.72 $\pm$   0.02 &     15 &  57.13 & 7.81e-07\\
Abell 2589 &   extr &     25 &   0.10 &   52.0 $\pm$   39.2 &    1.3 &  109.6 $\pm$   34.8 &   0.61 $\pm$   0.51 &     22 &   1.06 & 1.00e+00\\
 &      - & - & - &    0.0 & - &  154.1 $\pm$   13.4 &   0.29 $\pm$   0.07 &     23 &   1.60 & 1.00e+00\\
 &   flat & - & - &   52.0 $\pm$   39.2 &    1.3 &  109.6 $\pm$   34.8 &   0.61 $\pm$   0.51 &     22 &   1.06 & 1.00e+00\\
 &      - & - & - &    0.0 & - &  154.1 $\pm$   13.4 &   0.29 $\pm$   0.07 &     23 &   1.60 & 1.00e+00\\
Abell 2597 &   extr &      8 &   0.06 &    9.6 $\pm$    1.6 &    5.9 &   96.1 $\pm$   14.0 &   1.19 $\pm$   0.18 &      5 &   4.09 & 5.37e-01\\
 &      - & - & - &    0.0 & - &   70.7 $\pm$    5.0 &   0.62 $\pm$   0.04 &      6 &  23.75 & 5.81e-04\\
 &   flat & - & - &   10.6 $\pm$    1.5 &    7.0 &   98.9 $\pm$   15.2 &   1.26 $\pm$   0.19 &      5 &   4.10 & 5.35e-01\\
 &      - & - & - &    0.0 & - &   68.5 $\pm$    4.8 &   0.59 $\pm$   0.04 &      6 &  28.70 & 6.94e-05\\
Abell 2626 &   extr &     22 &   0.12 &   23.2 $\pm$    2.9 &    8.1 &  144.1 $\pm$    6.3 &   1.05 $\pm$   0.09 &     19 &  11.88 & 8.91e-01\\
 &      - & - & - &    0.0 & - &  147.7 $\pm$    5.2 &   0.62 $\pm$   0.03 &     20 &  46.28 & 7.38e-04\\
 &   flat & - & - &   23.2 $\pm$    2.9 &    8.1 &  144.1 $\pm$    6.3 &   1.05 $\pm$   0.09 &     19 &  11.88 & 8.91e-01\\
 &      - & - & - &    0.0 & - &  147.7 $\pm$    5.2 &   0.62 $\pm$   0.03 &     20 &  46.28 & 7.38e-04\\
Abell 2631 &   extr &     38 &   0.80 &  308.8 $\pm$   37.4 &    8.3 &   29.2 $\pm$   23.4 &   1.44 $\pm$   0.41 &     35 &   0.21 & 1.00e+00\\
 &      - & - & - &    0.0 & - &  347.2 $\pm$   21.7 &   0.33 $\pm$   0.05 &     36 &  13.73 & 1.00e+00\\
 &   flat & - & - &  308.8 $\pm$   37.4 &    8.3 &   29.2 $\pm$   23.4 &   1.44 $\pm$   0.41 &     35 &   0.21 & 1.00e+00\\
 &      - & - & - &    0.0 & - &  347.2 $\pm$   21.7 &   0.33 $\pm$   0.05 &     36 &  13.73 & 1.00e+00\\
Abell 2657 &   extr &     51 &   0.20 &   65.4 $\pm$   12.0 &    5.5 &  153.5 $\pm$   15.1 &   0.91 $\pm$   0.13 &     48 &   7.69 & 1.00e+00\\
 &      - & - & - &    0.0 & - &  222.0 $\pm$    5.9 &   0.50 $\pm$   0.03 &     49 &  21.73 & 1.00e+00\\
 &   flat & - & - &   65.4 $\pm$   12.0 &    5.5 &  153.5 $\pm$   15.1 &   0.91 $\pm$   0.13 &     48 &   7.69 & 1.00e+00\\
 &      - & - & - &    0.0 & - &  222.0 $\pm$    5.9 &   0.50 $\pm$   0.03 &     49 &  21.73 & 1.00e+00\\
Abell 2667 &   extr &     11 &   0.20 &   12.3 $\pm$    4.0 &    3.1 &  102.2 $\pm$    7.7 &   1.17 $\pm$   0.15 &      8 &   1.61 & 9.91e-01\\
 &      - & - & - &    0.0 & - &  113.7 $\pm$    6.2 &   0.85 $\pm$   0.05 &      9 &   9.48 & 3.94e-01\\
 &   flat & - & - &   19.3 $\pm$    3.4 &    5.7 &   93.4 $\pm$    7.6 &   1.31 $\pm$   0.17 &      8 &   1.66 & 9.90e-01\\
 &      - & - & - &    0.0 & - &  110.5 $\pm$    6.2 &   0.75 $\pm$   0.05 &      9 &  20.81 & 1.35e-02\\
Abell 2717 &   extr &     26 &   0.12 &   26.3 $\pm$    8.2 &    3.2 &  152.2 $\pm$   10.1 &   0.76 $\pm$   0.13 &     23 &   2.19 & 1.00e+00\\
 &      - & - & - &    0.0 & - &  167.6 $\pm$    7.9 &   0.50 $\pm$   0.03 &     24 &   7.90 & 9.99e-01\\
 &   flat & - & - &   27.0 $\pm$    8.4 &    3.2 &  151.2 $\pm$   10.2 &   0.75 $\pm$   0.13 &     23 &   2.15 & 1.00e+00\\
 &      - & - & - &    0.0 & - &  167.2 $\pm$    7.8 &   0.49 $\pm$   0.03 &     24 &   7.77 & 9.99e-01\\
Abell 2744 &   extr &     27 &   0.60 &  295.1 $\pm$  113.4 &    2.6 &  152.8 $\pm$  112.7 &   0.83 $\pm$   0.37 &     24 &   8.72 & 9.98e-01\\
 &      - & - & - &    0.0 & - &  460.3 $\pm$   29.9 &   0.37 $\pm$   0.05 &     25 &  10.50 & 9.95e-01\\
 &   flat & - & - &  438.4 $\pm$   58.7 &    7.5 &   46.4 $\pm$   44.0 &   1.41 $\pm$   0.55 &     24 &   7.87 & 9.99e-01\\
 &      - & - & - &    0.0 & - &  503.6 $\pm$   29.3 &   0.30 $\pm$   0.05 &     25 &  14.15 & 9.59e-01\\
Abell 2813 &   extr &     14 &   0.30 &  216.3 $\pm$   48.9 &    4.4 &  126.0 $\pm$   74.9 &   1.52 $\pm$   0.64 &     11 &   2.29 & 9.97e-01\\
 &      - & - & - &    0.0 & - &  397.4 $\pm$   33.0 &   0.42 $\pm$   0.10 &     12 &   7.83 & 7.98e-01\\
 &   flat & - & - &  267.6 $\pm$   43.8 &    6.1 &   90.4 $\pm$   67.3 &   1.76 $\pm$   0.80 &     11 &   2.64 & 9.95e-01\\
 &      - & - & - &    0.0 & - &  417.0 $\pm$   33.5 &   0.31 $\pm$   0.09 &     12 &   8.95 & 7.07e-01\\
Abell 3084 &   extr &     34 &   0.30 &   96.7 $\pm$   13.4 &    7.2 &  193.7 $\pm$   22.8 &   1.08 $\pm$   0.17 &     31 &   4.48 & 1.00e+00\\
 &      - & - & - &    0.0 & - &  288.3 $\pm$   14.4 &   0.43 $\pm$   0.04 &     32 &  17.29 & 9.84e-01\\
 &   flat & - & - &   96.7 $\pm$   13.4 &    7.2 &  193.7 $\pm$   22.8 &   1.08 $\pm$   0.17 &     31 &   4.48 & 1.00e+00\\
 &      - & - & - &    0.0 & - &  288.3 $\pm$   14.4 &   0.43 $\pm$   0.04 &     32 &  17.29 & 9.84e-01\\
Abell 3088 &   extr &     10 &   0.20 &   32.7 $\pm$    9.5 &    3.4 &  269.7 $\pm$   25.8 &   1.51 $\pm$   0.20 &      7 &   0.21 & 1.00e+00\\
 &      - & - & - &    0.0 & - &  283.9 $\pm$   23.8 &   1.02 $\pm$   0.09 &      8 &   7.68 & 4.65e-01\\
 &   flat & - & - &   82.8 $\pm$    8.4 &    9.8 &  216.8 $\pm$   25.8 &   1.71 $\pm$   0.25 &      7 &   0.59 & 9.99e-01\\
 &      - & - & - &    0.0 & - &  230.3 $\pm$   18.8 &   0.49 $\pm$   0.06 &      8 &  18.94 & 1.52e-02\\
Abell 3112 &   extr &     18 &   0.12 &    8.2 $\pm$    1.6 &    5.3 &  170.1 $\pm$    6.8 &   1.09 $\pm$   0.06 &     15 &   3.55 & 9.99e-01\\
 &      - & - & - &    0.0 & - &  162.7 $\pm$    6.0 &   0.86 $\pm$   0.03 &     16 &  23.03 & 1.13e-01\\
 &   flat & - & - &   11.4 $\pm$    1.4 &    8.0 &  169.1 $\pm$    7.0 &   1.17 $\pm$   0.07 &     15 &   5.32 & 9.89e-01\\
 &      - & - & - &    0.0 & - &  157.3 $\pm$    5.8 &   0.82 $\pm$   0.03 &     16 &  45.16 & 1.31e-04\\
Abell 3120 &   extr &     29 &   0.20 &   15.0 $\pm$    3.3 &    4.5 &  209.1 $\pm$   10.9 &   1.02 $\pm$   0.08 &     26 &   6.41 & 1.00e+00\\
 &      - & - & - &    0.0 & - &  206.6 $\pm$   10.1 &   0.76 $\pm$   0.03 &     27 &  20.49 & 8.10e-01\\
 &   flat & - & - &   17.3 $\pm$    3.5 &    4.9 &  206.2 $\pm$   10.9 &   0.99 $\pm$   0.08 &     26 &   7.14 & 1.00e+00\\
 &      - & - & - &    0.0 & - &  202.9 $\pm$    9.8 &   0.70 $\pm$   0.03 &     27 &  22.57 & 7.08e-01\\
Abell 3158 &   extr &     72 &   0.40 &  166.0 $\pm$   11.7 &   14.1 &   80.9 $\pm$   12.9 &   0.90 $\pm$   0.10 &     69 &  22.54 & 1.00e+00\\
 &      - & - & - &    0.0 & - &  260.6 $\pm$    2.9 &   0.32 $\pm$   0.01 &     70 &  71.32 & 4.34e-01\\
 &   flat & - & - &  166.0 $\pm$   11.7 &   14.1 &   80.9 $\pm$   12.9 &   0.90 $\pm$   0.10 &     69 &  22.54 & 1.00e+00\\
 &      - & - & - &    0.0 & - &  260.6 $\pm$    2.9 &   0.32 $\pm$   0.01 &     70 &  71.32 & 4.34e-01\\
Abell 3266 &   extr &     15 &   0.08 &   63.7 $\pm$   41.9 &    1.5 &  405.3 $\pm$   51.6 &   0.71 $\pm$   0.27 &     12 &   0.79 & 1.00e+00\\
 &      - & - & - &    0.0 & - &  418.9 $\pm$   37.8 &   0.44 $\pm$   0.06 &     13 &   2.02 & 1.00e+00\\
 &   flat & - & - &   72.5 $\pm$   49.7 &    1.5 &  376.7 $\pm$   48.0 &   0.64 $\pm$   0.28 &     12 &   1.26 & 1.00e+00\\
 &      - & - & - &    0.0 & - &  404.6 $\pm$   35.2 &   0.39 $\pm$   0.05 &     13 &   2.34 & 1.00e+00\\
Abell 3364 &   extr &     55 &   0.70 &  268.6 $\pm$   33.2 &    8.1 &   34.5 $\pm$   18.0 &   1.97 $\pm$   0.32 &     52 &   3.99 & 1.00e+00\\
 &      - & - & - &    0.0 & - &  298.6 $\pm$   22.7 &   0.63 $\pm$   0.08 &     53 &  30.04 & 9.95e-01\\
 &   flat & - & - &  268.6 $\pm$   33.2 &    8.1 &   34.5 $\pm$   18.0 &   1.97 $\pm$   0.32 &     52 &   3.99 & 1.00e+00\\
 &      - & - & - &    0.0 & - &  298.6 $\pm$   22.7 &   0.63 $\pm$   0.08 &     53 &  30.04 & 9.95e-01\\
Abell 3376 &   extr &     67 &   0.30 &  282.9 $\pm$    9.3 &   30.3 &   59.0 $\pm$   10.6 &   1.71 $\pm$   0.18 &     64 &   5.46 & 1.00e+00\\
 &      - & - & - &    0.0 & - &  378.5 $\pm$    4.3 &   0.30 $\pm$   0.02 &     65 & 112.42 & 2.39e-04\\
 &   flat & - & - &  282.9 $\pm$    9.3 &   30.3 &   59.0 $\pm$   10.6 &   1.71 $\pm$   0.18 &     64 &   5.46 & 1.00e+00\\
 &      - & - & - &    0.0 & - &  378.5 $\pm$    4.3 &   0.30 $\pm$   0.02 &     65 & 112.42 & 2.39e-04\\
Abell 3391 &   extr &     75 &   0.40 &  367.5 $\pm$   16.0 &   22.9 &   23.6 $\pm$   14.8 &   1.64 $\pm$   0.47 &     72 &   3.59 & 1.00e+00\\
 &      - & - & - &    0.0 & - &  420.4 $\pm$    7.5 &   0.14 $\pm$   0.02 &     73 &  24.89 & 1.00e+00\\
 &   flat & - & - &  367.5 $\pm$   16.0 &   22.9 &   23.6 $\pm$   14.8 &   1.64 $\pm$   0.47 &     72 &   3.59 & 1.00e+00\\
 &      - & - & - &    0.0 & - &  420.4 $\pm$    7.5 &   0.14 $\pm$   0.02 &     73 &  24.89 & 1.00e+00\\
Abell 3395 &   extr &     24 &   0.12 &  213.3 $\pm$   26.2 &    8.2 &  133.5 $\pm$   30.4 &   1.58 $\pm$   0.79 &     21 &   0.00 & 1.00e+00\\
 &      - & - & - &    0.0 & - &  325.5 $\pm$   14.4 &   0.23 $\pm$   0.05 &     22 &   5.73 & 1.00e+00\\
 &   flat & - & - &  247.2 $\pm$   25.2 &    9.8 &  105.9 $\pm$   29.8 &   1.65 $\pm$   1.01 &     21 &   0.01 & 1.00e+00\\
 &      - & - & - &    0.0 & - &  332.8 $\pm$   14.0 &   0.16 $\pm$   0.05 &     22 &   4.49 & 1.00e+00\\
Abell 3528S &   extr &     24 &   0.12 &   19.4 $\pm$    2.3 &    8.6 &  288.1 $\pm$   10.2 &   1.16 $\pm$   0.05 &     21 &  32.09 & 5.73e-02\\
 &      - & - & - &    0.0 & - &  271.7 $\pm$    8.8 &   0.84 $\pm$   0.02 &     22 &  84.38 & 3.04e-09\\
 &   flat & - & - &   31.6 $\pm$    2.3 &   14.0 &  270.0 $\pm$   10.3 &   1.17 $\pm$   0.06 &     21 &  32.23 & 5.55e-02\\
 &      - & - & - &    0.0 & - &  239.2 $\pm$    7.6 &   0.65 $\pm$   0.02 &     22 & 128.53 & 4.82e-17\\
Abell 3558 &   extr &     25 &   0.12 &  126.2 $\pm$   11.8 &   10.7 &  132.5 $\pm$   17.2 &   2.11 $\pm$   0.58 &     22 &   6.87 & 9.99e-01\\
 &      - & - & - &    0.0 & - &  234.0 $\pm$   10.7 &   0.42 $\pm$   0.06 &     23 &  19.89 & 6.49e-01\\
 &   flat & - & - &  126.2 $\pm$   11.8 &   10.7 &  132.5 $\pm$   17.2 &   2.11 $\pm$   0.58 &     22 &   6.87 & 9.99e-01\\
 &      - & - & - &    0.0 & - &  234.0 $\pm$   10.7 &   0.42 $\pm$   0.06 &     23 &  19.89 & 6.49e-01\\
Abell 3562 &   extr &     26 &   0.12 &   71.4 $\pm$    9.0 &    8.0 &  166.8 $\pm$   10.4 &   0.80 $\pm$   0.13 &     23 &  33.16 & 7.84e-02\\
 &      - & - & - &    0.0 & - &  217.3 $\pm$    6.5 &   0.33 $\pm$   0.02 &     24 &  54.22 & 3.99e-04\\
 &   flat & - & - &   77.4 $\pm$    8.9 &    8.7 &  159.8 $\pm$   10.4 &   0.81 $\pm$   0.13 &     23 &  35.16 & 5.01e-02\\
 &      - & - & - &    0.0 & - &  215.4 $\pm$    6.4 &   0.31 $\pm$   0.02 &     24 &  56.31 & 2.08e-04\\
Abell 3571 &   extr &     31 &   0.12 &   79.3 $\pm$   14.8 &    5.4 &  191.3 $\pm$   14.8 &   0.82 $\pm$   0.16 &     28 & 375.69 & 1.65e-62\\
 &      - & - & - &    0.0 & - &  256.1 $\pm$    7.9 &   0.39 $\pm$   0.03 &     29 & 657.82 & 6.19e-120\\
 &   flat & - & - &   79.3 $\pm$   14.8 &    5.4 &  191.3 $\pm$   14.8 &   0.82 $\pm$   0.16 &     28 & 375.69 & 1.65e-62\\
 &      - & - & - &    0.0 & - &  256.1 $\pm$    7.9 &   0.39 $\pm$   0.03 &     29 & 657.82 & 6.19e-120\\
Abell 3581 &   extr &     46 &   0.10 &    7.1 $\pm$    0.8 &    8.4 &  138.1 $\pm$    5.5 &   1.15 $\pm$   0.05 &     43 &  20.49 & 9.99e-01\\
 &      - & - & - &    0.0 & - &  121.6 $\pm$    4.0 &   0.85 $\pm$   0.02 &     44 &  65.85 & 1.80e-02\\
 &   flat & - & - &    9.5 $\pm$    0.8 &   12.2 &  138.1 $\pm$    5.7 &   1.22 $\pm$   0.05 &     43 &  21.56 & 9.97e-01\\
 &      - & - & - &    0.0 & - &  114.3 $\pm$    3.8 &   0.79 $\pm$   0.02 &     44 & 103.30 & 1.13e-06\\
Abell 3667 &   extr &     56 &   0.30 &  149.3 $\pm$   17.2 &    8.7 &  121.9 $\pm$   18.6 &   0.72 $\pm$   0.09 &     53 &  21.14 & 1.00e+00\\
 &      - & - & - &    0.0 & - &  278.7 $\pm$    2.3 &   0.34 $\pm$   0.01 &     54 &  44.43 & 8.20e-01\\
 &   flat & - & - &  160.4 $\pm$   15.5 &   10.4 &  110.6 $\pm$   16.8 &   0.78 $\pm$   0.10 &     53 &  22.84 & 1.00e+00\\
 &      - & - & - &    0.0 & - &  279.5 $\pm$    2.3 &   0.33 $\pm$   0.01 &     54 &  52.83 & 5.19e-01\\
Abell 3822 &   extr &     42 &   0.30 &  108.7 $\pm$   76.4 &    1.4 &  200.3 $\pm$   90.8 &   0.66 $\pm$   0.33 &     39 &   3.24 & 1.00e+00\\
 &      - & - & - &    0.0 & - &  322.5 $\pm$   16.6 &   0.38 $\pm$   0.07 &     40 &   3.95 & 1.00e+00\\
 &   flat & - & - &  108.7 $\pm$   76.4 &    1.4 &  200.3 $\pm$   90.8 &   0.66 $\pm$   0.33 &     39 &   3.24 & 1.00e+00\\
 &      - & - & - &    0.0 & - &  322.5 $\pm$   16.6 &   0.38 $\pm$   0.07 &     40 &   3.95 & 1.00e+00\\
Abell 3827 &   extr &     67 &   0.60 &  144.6 $\pm$   13.4 &   10.8 &  113.1 $\pm$   15.2 &   1.23 $\pm$   0.10 &     64 & 1651.91 & 6.60e-303\\
 &      - & - & - &    0.0 & - &  287.2 $\pm$    7.4 &   0.60 $\pm$   0.03 &     65 & 4867.53 & 0.00e+00\\
 &   flat & - & - &  164.6 $\pm$   12.5 &   13.2 &   94.8 $\pm$   13.7 &   1.34 $\pm$   0.10 &     64 & 1368.56 & 6.59e-244\\
 &      - & - & - &    0.0 & - &  293.5 $\pm$    7.3 &   0.57 $\pm$   0.03 &     65 & 5896.48 & 0.00e+00\\
Abell 3921 &   extr &     47 &   0.40 &  101.2 $\pm$   17.9 &    5.7 &  151.5 $\pm$   23.0 &   0.86 $\pm$   0.11 &     44 &   7.55 & 1.00e+00\\
 &      - & - & - &    0.0 & - &  272.4 $\pm$    6.8 &   0.48 $\pm$   0.03 &     45 &  22.08 & 9.98e-01\\
 &   flat & - & - &  101.2 $\pm$   17.9 &    5.7 &  151.5 $\pm$   23.0 &   0.86 $\pm$   0.11 &     44 &   7.55 & 1.00e+00\\
 &      - & - & - &    0.0 & - &  272.4 $\pm$    6.8 &   0.48 $\pm$   0.03 &     45 &  22.08 & 9.98e-01\\
Abell 4038 &   extr &     42 &   0.12 &   37.1 $\pm$    1.2 &   30.2 &  118.5 $\pm$    2.7 &   1.10 $\pm$   0.05 &     39 &  58.69 & 2.22e-02\\
 &      - & - & - &    0.0 & - &  127.3 $\pm$    2.0 &   0.42 $\pm$   0.01 &     40 & 393.69 & 1.15e-59\\
 &   flat & - & - &   37.9 $\pm$    1.2 &   31.2 &  117.9 $\pm$    2.7 &   1.11 $\pm$   0.05 &     39 &  60.31 & 1.58e-02\\
 &      - & - & - &    0.0 & - &  126.5 $\pm$    1.9 &   0.41 $\pm$   0.01 &     40 & 410.34 & 6.07e-63\\
Abell 4059 &   extr &     33 &   0.15 &    0.0 $\pm$    1.0 &    0.0 &  210.7 $\pm$    2.2 &   0.82 $\pm$   0.01 &     30 &  44.86 & 3.98e-02\\
 &      - & - & - &    0.0 & - &  210.7 $\pm$    2.2 &   0.82 $\pm$   0.01 &     31 &  44.86 & 5.13e-02\\
 &   flat & - & - &    7.1 $\pm$    1.0 &    6.7 &  203.2 $\pm$    2.4 &   0.88 $\pm$   0.02 &     30 &  54.25 & 4.31e-03\\
 &      - & - & - &    0.0 & - &  208.3 $\pm$    2.2 &   0.77 $\pm$   0.01 &     31 &  93.59 & 3.35e-08\\
Abell S0405 &   extr &     34 &   0.20 &   23.5 $\pm$   21.0 &    1.1 &  261.1 $\pm$   22.1 &   0.52 $\pm$   0.10 &     31 &   8.24 & 1.00e+00\\
 &      - & - & - &    0.0 & - &  281.9 $\pm$   11.3 &   0.43 $\pm$   0.03 &     32 &   9.16 & 1.00e+00\\
 &   flat & - & - &   16.9 $\pm$   27.9 &    0.6 &  274.2 $\pm$   27.3 &   0.45 $\pm$   0.10 &     31 &   9.79 & 1.00e+00\\
 &      - & - & - &    0.0 & - &  289.3 $\pm$   11.2 &   0.40 $\pm$   0.02 &     32 &  10.10 & 1.00e+00\\
Abell S0592 &   extr &     23 &   0.40 &   52.2 $\pm$   14.4 &    3.6 &  199.0 $\pm$   23.6 &   0.99 $\pm$   0.12 &     20 &   9.34 & 9.79e-01\\
 &      - & - & - &    0.0 & - &  271.1 $\pm$   10.0 &   0.68 $\pm$   0.04 &     21 &  16.08 & 7.65e-01\\
 &   flat & - & - &   58.7 $\pm$   14.4 &    4.1 &  195.5 $\pm$   23.6 &   0.99 $\pm$   0.13 &     20 &   9.70 & 9.73e-01\\
 &      - & - & - &    0.0 & - &  275.6 $\pm$   10.0 &   0.65 $\pm$   0.04 &     21 &  17.46 & 6.83e-01\\
AC 114 &   flat &     20 &   0.45 &  199.8 $\pm$   28.0 &    7.1 &   70.0 $\pm$   32.6 &   1.50 $\pm$   0.36 &     17 &   3.69 & 1.00e+00\\
 &      - & - & - &    0.0 & - &  306.6 $\pm$   14.8 &   0.46 $\pm$   0.06 &     18 &  16.94 & 5.28e-01\\
 &   extr & - & - &  199.8 $\pm$   28.0 &    7.1 &   70.0 $\pm$   32.6 &   1.50 $\pm$   0.36 &     17 &   3.69 & 1.00e+00\\
 &      - & - & - &    0.0 & - &  306.6 $\pm$   14.8 &   0.46 $\pm$   0.06 &     18 &  16.94 & 5.28e-01\\
AWM7 &   extr &     13 &   0.02 &    4.8 $\pm$    1.1 &    4.5 &  290.2 $\pm$   28.4 &   0.89 $\pm$   0.06 &     10 &   7.30 & 6.96e-01\\
 &      - & - & - &    0.0 & - &  217.6 $\pm$   10.6 &   0.70 $\pm$   0.02 &     11 &  20.91 & 3.43e-02\\
 &   flat & - & - &    8.4 $\pm$    1.3 &    6.5 &  227.6 $\pm$   23.1 &   0.80 $\pm$   0.06 &     10 &  13.19 & 2.13e-01\\
 &      - & - & - &    0.0 & - &  157.1 $\pm$    6.6 &   0.54 $\pm$   0.01 &     11 &  32.84 & 5.58e-04\\
Centaurus &   extr &     27 &   0.03 &    1.4 $\pm$   0.04 &   32.1 &  421.2 $\pm$    5.4 &   1.25 $\pm$   0.01 &     24 & 253.13 & 3.95e-40\\
 &      - & - & - &    0.0 & - &  328.8 $\pm$    3.1 &   1.11 $\pm$   0.00 &     25 & 1159.86 & 6.17e-229\\
 &   flat & - & - &    2.2 $\pm$   0.04 &   56.6 &  474.9 $\pm$    6.3 &   1.33 $\pm$   0.01 &     24 & 483.38 & 4.67e-87\\
 &      - & - & - &    0.0 & - &  307.3 $\pm$    2.9 &   1.08 $\pm$   0.00 &     25 & 3151.59 & 0.00e+00\\
CID 72 &   extr &     37 &   0.12 &    4.9 $\pm$    0.3 &   14.6 &  139.2 $\pm$    2.1 &   0.95 $\pm$   0.02 &     34 & 135.51 & 4.60e-14\\
 &      - & - & - &    0.0 & - &  128.6 $\pm$    1.7 &   0.77 $\pm$   0.01 &     35 & 313.61 & 1.74e-46\\
 &   flat & - & - &    9.4 $\pm$    0.3 &   29.9 &  133.2 $\pm$    2.2 &   0.99 $\pm$   0.02 &     34 & 129.24 & 5.04e-13\\
 &      - & - & - &    0.0 & - &  111.3 $\pm$    1.5 &   0.63 $\pm$   0.01 &     35 & 634.02 & 4.82e-111\\
CL J1226.9+3332 &   extr &     10 &   0.40 &  166.0 $\pm$   45.2 &    3.7 &   99.0 $\pm$   58.7 &   1.41 $\pm$   0.50 &      7 &   0.75 & 9.98e-01\\
 &      - & - & - &    0.0 & - &  308.7 $\pm$   25.3 &   0.55 $\pm$   0.10 &      8 &   4.81 & 7.78e-01\\
 &   flat & - & - &  166.0 $\pm$   45.2 &    3.7 &   99.0 $\pm$   58.7 &   1.41 $\pm$   0.50 &      7 &   0.75 & 9.98e-01\\
 &      - & - & - &    0.0 & - &  308.7 $\pm$   25.3 &   0.55 $\pm$   0.10 &      8 &   4.81 & 7.78e-01\\
Cygnus A &   extr &     19 &   0.10 &   21.7 $\pm$    0.9 &   24.2 &  208.4 $\pm$    6.7 &   1.51 $\pm$   0.05 &     16 &  28.49 & 2.76e-02\\
 &      - & - & - &    0.0 & - &  154.4 $\pm$    3.7 &   0.73 $\pm$   0.02 &     17 & 294.72 & 1.38e-52\\
 &   flat & - & - &   23.6 $\pm$    0.9 &   27.1 &  210.1 $\pm$    6.9 &   1.57 $\pm$   0.05 &     16 &  22.48 & 1.28e-01\\
 &      - & - & - &    0.0 & - &  148.5 $\pm$    3.6 &   0.70 $\pm$   0.02 &     17 & 340.49 & 4.67e-62\\
ESO 3060170 &   extr &      5 &   0.02 &    7.8 $\pm$    1.0 &    7.8 & 1370.5 $\pm$  562.2 &   1.79 $\pm$   0.20 &      2 &   0.77 & 6.80e-01\\
 &      - & - & - &    0.0 & - &  255.8 $\pm$   37.1 &   0.90 $\pm$   0.05 &      3 &  25.78 & 1.06e-05\\
 &   flat & - & - &    8.0 $\pm$    1.0 &    8.0 & 1400.9 $\pm$  578.9 &   1.80 $\pm$   0.21 &      2 &   0.81 & 6.67e-01\\
 &      - & - & - &    0.0 & - &  251.2 $\pm$   36.3 &   0.89 $\pm$   0.05 &      3 &  26.70 & 6.81e-06\\
ESO 5520200 &   extr &     17 &   0.10 &    6.3 $\pm$    3.5 &    1.8 &  113.8 $\pm$    7.0 &   0.74 $\pm$   0.10 &     31 &   0.15 & 1.00e+00\\
 &      - & - & - &    0.0 & - &  112.0 $\pm$    6.0 &   0.60 $\pm$   0.03 &     32 &   5.57 & 1.00e+00\\
 &   flat & - & - &    5.9 $\pm$    4.2 &    1.4 &  121.8 $\pm$    6.5 &   0.67 $\pm$   0.09 &     31 &   0.52 & 1.00e+00\\
 &      - & - & - &    0.0 & - &  121.1 $\pm$    5.8 &   0.57 $\pm$   0.03 &     32 &   4.15 & 1.00e+00\\
EXO 422-086 &   extr &     19 &   0.07 &   10.1 $\pm$    0.8 &   12.5 &  199.3 $\pm$   11.4 &   1.21 $\pm$   0.06 &     16 &  11.00 & 8.10e-01\\
 &      - & - & - &    0.0 & - &  142.0 $\pm$    5.6 &   0.75 $\pm$   0.02 &     17 & 112.48 & 4.11e-16\\
 &   flat & - & - &   13.8 $\pm$    0.8 &   17.5 &  193.8 $\pm$   11.8 &   1.25 $\pm$   0.06 &     16 &  11.24 & 7.95e-01\\
 &      - & - & - &    0.0 & - &  120.4 $\pm$    4.6 &   0.62 $\pm$   0.02 &     17 & 157.52 & 8.19e-25\\
HCG 62 &   extr &     27 &   0.04 &    3.1 $\pm$   0.08 &   40.8 &  203.9 $\pm$   10.4 &   1.23 $\pm$   0.02 &     24 & 153.17 & 8.52e-21\\
 &      - & - & - &    0.0 & - &   63.4 $\pm$    1.7 &   0.63 $\pm$   0.01 &     25 & 660.63 & 2.48e-123\\
 &   flat & - & - &    3.4 $\pm$   0.07 &   47.4 &  219.0 $\pm$   11.4 &   1.28 $\pm$   0.03 &     24 & 138.50 & 4.39e-18\\
 &      - & - & - &    0.0 & - &   57.7 $\pm$    1.5 &   0.60 $\pm$   0.01 &     25 & 751.59 & 1.92e-142\\
HCG 42 &   extr &     22 &   0.03 &    1.8 $\pm$    0.3 &    5.5 &  128.5 $\pm$   12.8 &   0.88 $\pm$   0.05 &     19 &  44.38 & 8.38e-04\\
 &      - & - & - &    0.0 & - &   89.4 $\pm$    4.1 &   0.67 $\pm$   0.01 &     20 &  60.87 & 5.23e-06\\
 &   flat & - & - &    1.9 $\pm$    0.3 &    5.7 &  126.5 $\pm$   12.6 &   0.87 $\pm$   0.05 &     19 &  45.28 & 6.25e-04\\
 &      - & - & - &    0.0 & - &   87.4 $\pm$    4.0 &   0.66 $\pm$   0.01 &     20 &  62.24 & 3.19e-06\\
Hercules A &   extr &     16 &   0.20 &    2.8 $\pm$    1.5 &    1.8 &  151.8 $\pm$    3.3 &   0.99 $\pm$   0.04 &     13 &   2.34 & 1.00e+00\\
 &      - & - & - &    0.0 & - &  154.1 $\pm$    3.1 &   0.94 $\pm$   0.02 &     14 &   6.23 & 9.60e-01\\
 &   flat & - & - &    9.2 $\pm$    1.3 &    6.8 &  143.9 $\pm$    3.3 &   1.07 $\pm$   0.04 &     13 &   6.24 & 9.37e-01\\
 &      - & - & - &    0.0 & - &  151.0 $\pm$    3.1 &   0.87 $\pm$   0.02 &     14 &  46.89 & 2.01e-05\\
Hydra A &   extr &     57 &   0.30 &   13.0 $\pm$    0.7 &   19.5 &  115.3 $\pm$    1.4 &   1.02 $\pm$   0.02 &     54 &  71.44 & 5.62e-02\\
 &      - & - & - &    0.0 & - &  134.0 $\pm$    1.0 &   0.81 $\pm$   0.01 &     55 & 364.39 & 3.36e-47\\
 &   flat & - & - &   13.3 $\pm$    0.7 &   20.0 &  114.9 $\pm$    1.4 &   1.03 $\pm$   0.02 &     54 &  72.66 & 4.60e-02\\
 &      - & - & - &    0.0 & - &  134.0 $\pm$    1.0 &   0.80 $\pm$   0.01 &     55 & 379.86 & 4.40e-50\\
M49 &   extr &     54 &   1.00 &    0.9 $\pm$   0.05 &   18.1 &  486.7 $\pm$   32.2 &   1.14 $\pm$   0.02 &     51 &  74.03 & 1.92e-02\\
 &      - & - & - &    0.0 & - &  231.3 $\pm$   10.1 &   0.89 $\pm$   0.01 &     52 & 327.07 & 1.58e-41\\
 &   flat & - & - &    0.9 $\pm$   0.05 &   18.9 &  495.3 $\pm$   32.9 &   1.14 $\pm$   0.02 &     51 &  75.65 & 1.41e-02\\
 &      - & - & - &    0.0 & - &  227.4 $\pm$   10.0 &   0.88 $\pm$   0.01 &     52 & 349.43 & 1.14e-45\\
M87 &   extr &     88 &   0.04 &    3.5 $\pm$   0.08 &   43.1 &  146.4 $\pm$    1.0 &   0.80 $\pm$   0.00 &     85 & 749.92 & 4.94e-107\\
 &      - & - & - &    0.0 & - &  123.8 $\pm$    0.5 &   0.64 $\pm$   0.00 &     86 & 2083.55 & 0.00e+00\\
 &   flat & - & - &    3.5 $\pm$   0.08 &   43.7 &  146.6 $\pm$    1.0 &   0.80 $\pm$   0.00 &     85 & 763.71 & 1.06e-109\\
 &      - & - & - &    0.0 & - &  123.7 $\pm$    0.5 &   0.64 $\pm$   0.00 &     86 & 2130.02 & 0.00e+00\\
MACS J0011.7-1523 &   extr &     16 &   0.40 &   14.9 $\pm$    6.4 &    2.3 &  111.3 $\pm$   11.6 &   1.03 $\pm$   0.10 &     13 &   1.95 & 1.00e+00\\
 &      - & - & - &    0.0 & - &  134.7 $\pm$    5.1 &   0.86 $\pm$   0.04 &     14 &   5.88 & 9.70e-01\\
 &   flat & - & - &   18.8 $\pm$    6.3 &    3.0 &  109.1 $\pm$   11.5 &   1.04 $\pm$   0.10 &     13 &   2.28 & 1.00e+00\\
 &      - & - & - &    0.0 & - &  138.4 $\pm$    5.0 &   0.81 $\pm$   0.04 &     14 &   7.99 & 8.90e-01\\
MACS J0035.4-2015 &   extr &     29 &   0.70 &   69.5 $\pm$   17.1 &    4.1 &   93.9 $\pm$   23.0 &   1.15 $\pm$   0.16 &     26 &   0.70 & 1.00e+00\\
 &      - & - & - &    0.0 & - &  183.2 $\pm$   11.5 &   0.74 $\pm$   0.06 &     27 &  11.72 & 9.95e-01\\
 &   flat & - & - &   93.4 $\pm$   15.7 &    6.0 &   76.4 $\pm$   20.8 &   1.26 $\pm$   0.17 &     26 &   1.00 & 1.00e+00\\
 &      - & - & - &    0.0 & - &  198.2 $\pm$   11.1 &   0.66 $\pm$   0.05 &     27 &  20.41 & 8.13e-01\\
MACS J0159.8-0849 &   extr &     15 &   0.40 &   11.9 $\pm$    4.0 &    3.0 &  133.7 $\pm$   10.0 &   1.25 $\pm$   0.08 &     12 &   2.47 & 9.98e-01\\
 &      - & - & - &    0.0 & - &  155.7 $\pm$    5.8 &   1.06 $\pm$   0.04 &     13 &   9.44 & 7.39e-01\\
 &   flat & - & - &   18.8 $\pm$    3.7 &    5.0 &  123.9 $\pm$    9.9 &   1.31 $\pm$   0.09 &     12 &   3.68 & 9.89e-01\\
 &      - & - & - &    0.0 & - &  158.3 $\pm$    5.9 &   1.01 $\pm$   0.04 &     13 &  21.08 & 7.13e-02\\
MACS J0242.5-2132 &   extr &     22 &   0.50 &    9.7 $\pm$    1.9 &    5.0 &   76.3 $\pm$    5.1 &   1.27 $\pm$   0.07 &     19 &  11.73 & 8.97e-01\\
 &      - & - & - &    0.0 & - &   94.0 $\pm$    3.2 &   1.01 $\pm$   0.04 &     20 &  29.52 & 7.81e-02\\
 &   flat & - & - &   10.9 $\pm$    1.9 &    5.7 &   74.6 $\pm$    5.0 &   1.29 $\pm$   0.07 &     19 &  11.84 & 8.93e-01\\
 &      - & - & - &    0.0 & - &   94.4 $\pm$    3.2 &   0.99 $\pm$   0.03 &     20 &  34.37 & 2.37e-02\\
MACS J0257.1-2325 &   extr &     13 &   0.40 &  234.5 $\pm$   68.2 &    3.4 &  195.8 $\pm$  107.3 &   1.39 $\pm$   0.57 &     10 &   0.24 & 1.00e+00\\
 &      - & - & - &    0.0 & - &  489.1 $\pm$   50.9 &   0.47 $\pm$   0.12 &     11 &   3.07 & 9.90e-01\\
 &   flat & - & - &  234.5 $\pm$   68.2 &    3.4 &  195.8 $\pm$  107.3 &   1.39 $\pm$   0.57 &     10 &   0.24 & 1.00e+00\\
 &      - & - & - &    0.0 & - &  489.1 $\pm$   50.9 &   0.47 $\pm$   0.12 &     11 &   3.07 & 9.90e-01\\
MACS J0257.6-2209 &   extr &     17 &   0.40 &  155.1 $\pm$   25.1 &    6.2 &   82.7 $\pm$   32.5 &   1.55 $\pm$   0.34 &     14 &   1.00 & 1.00e+00\\
 &      - & - & - &    0.0 & - &  277.1 $\pm$   15.3 &   0.56 $\pm$   0.07 &     15 &  18.10 & 2.57e-01\\
 &   flat & - & - &  155.9 $\pm$   25.0 &    6.2 &   82.1 $\pm$   32.4 &   1.55 $\pm$   0.34 &     14 &   1.01 & 1.00e+00\\
 &      - & - & - &    0.0 & - &  277.6 $\pm$   15.2 &   0.56 $\pm$   0.07 &     15 &  18.25 & 2.49e-01\\
MACS J0308.9+2645 &   extr &     30 &   0.70 &  212.8 $\pm$   53.9 &    3.9 &   70.1 $\pm$   42.2 &   1.43 $\pm$   0.35 &     27 &   0.86 & 1.00e+00\\
 &      - & - & - &    0.0 & - &  290.5 $\pm$   34.0 &   0.66 $\pm$   0.10 &     28 &   7.88 & 1.00e+00\\
 &   flat & - & - &  212.8 $\pm$   53.9 &    3.9 &   70.1 $\pm$   42.2 &   1.43 $\pm$   0.35 &     27 &   0.86 & 1.00e+00\\
 &      - & - & - &    0.0 & - &  290.5 $\pm$   34.0 &   0.66 $\pm$   0.10 &     28 &   7.88 & 1.00e+00\\
MACS J0329.6-0211 &   extr &     14 &   0.40 &    6.6 $\pm$    2.7 &    2.4 &  102.9 $\pm$    6.5 &   1.21 $\pm$   0.07 &     11 &   9.63 & 5.64e-01\\
 &      - & - & - &    0.0 & - &  115.4 $\pm$    3.6 &   1.08 $\pm$   0.03 &     12 &  14.83 & 2.51e-01\\
 &   flat & - & - &   11.1 $\pm$    2.5 &    4.4 &   96.7 $\pm$    6.4 &   1.26 $\pm$   0.07 &     11 &  11.91 & 3.71e-01\\
 &      - & - & - &    0.0 & - &  117.5 $\pm$    3.6 &   1.03 $\pm$   0.03 &     12 &  26.77 & 8.33e-03\\
MACS J0417.5-1154 &   extr &     11 &   0.30 &    9.5 $\pm$    6.7 &    1.4 &  101.6 $\pm$   14.8 &   1.52 $\pm$   0.22 &      8 &   0.88 & 9.99e-01\\
 &      - & - & - &    0.0 & - &  117.2 $\pm$    9.2 &   1.29 $\pm$   0.13 &      9 &   2.51 & 9.81e-01\\
 &   flat & - & - &   27.1 $\pm$    7.3 &    3.7 &   99.7 $\pm$   15.1 &   1.42 $\pm$   0.23 &      8 &   1.16 & 9.97e-01\\
 &      - & - & - &    0.0 & - &  136.1 $\pm$    9.4 &   0.85 $\pm$   0.08 &      9 &   7.22 & 6.14e-01\\
MACS J0429.6-0253 &   extr &     15 &   0.40 &   14.8 $\pm$    4.4 &    3.4 &   91.4 $\pm$    9.0 &   1.21 $\pm$   0.11 &     12 &   2.46 & 9.98e-01\\
 &      - & - & - &    0.0 & - &  115.3 $\pm$    4.7 &   0.95 $\pm$   0.05 &     13 &  10.52 & 6.51e-01\\
 &   flat & - & - &   17.2 $\pm$    4.3 &    4.0 &   88.9 $\pm$    9.0 &   1.23 $\pm$   0.11 &     12 &   2.52 & 9.98e-01\\
 &      - & - & - &    0.0 & - &  116.5 $\pm$    4.7 &   0.92 $\pm$   0.05 &     13 &  13.22 & 4.31e-01\\
MACS J0520.7-1328 &   extr &     21 &   0.50 &   88.6 $\pm$   22.0 &    4.0 &   84.9 $\pm$   28.2 &   1.20 $\pm$   0.24 &     18 &   0.75 & 1.00e+00\\
 &      - & - & - &    0.0 & - &  194.8 $\pm$   12.0 &   0.64 $\pm$   0.07 &     19 &   8.63 & 9.79e-01\\
 &   flat & - & - &   88.6 $\pm$   22.0 &    4.0 &   84.9 $\pm$   28.2 &   1.20 $\pm$   0.24 &     18 &   0.75 & 1.00e+00\\
 &      - & - & - &    0.0 & - &  194.8 $\pm$   12.0 &   0.64 $\pm$   0.07 &     19 &   8.63 & 9.79e-01\\
MACS J0547.0-3904 &   extr &     24 &   0.40 &   22.0 $\pm$    4.4 &    5.0 &  122.6 $\pm$   10.2 &   1.19 $\pm$   0.10 &     21 &   7.76 & 9.96e-01\\
 &      - & - & - &    0.0 & - &  153.5 $\pm$    6.9 &   0.84 $\pm$   0.04 &     22 &  23.85 & 3.55e-01\\
 &   flat & - & - &   23.1 $\pm$    4.4 &    5.2 &  121.6 $\pm$   10.2 &   1.20 $\pm$   0.10 &     21 &   7.65 & 9.96e-01\\
 &      - & - & - &    0.0 & - &  153.7 $\pm$    7.0 &   0.83 $\pm$   0.04 &     22 &  25.01 & 2.97e-01\\
MACS J0717.5+3745 &   extr &     16 &   0.50 &  158.7 $\pm$  111.6 &    1.4 &  202.0 $\pm$  128.8 &   0.69 $\pm$   0.35 &     13 &   1.31 & 1.00e+00\\
 &      - & - & - &    0.0 & - &  378.6 $\pm$   26.0 &   0.40 $\pm$   0.07 &     14 &   2.63 & 1.00e+00\\
 &   flat & - & - &  220.1 $\pm$   96.4 &    2.3 &  160.1 $\pm$  112.2 &   0.76 $\pm$   0.40 &     13 &   1.03 & 1.00e+00\\
 &      - & - & - &    0.0 & - &  404.8 $\pm$   25.2 &   0.33 $\pm$   0.06 &     14 &   3.02 & 9.99e-01\\
MACS J0744.8+3927 &   extr &     17 &   0.60 &   39.5 $\pm$   11.0 &    3.6 &  113.9 $\pm$   17.4 &   1.10 $\pm$   0.11 &     14 &   3.84 & 9.96e-01\\
 &      - & - & - &    0.0 & - &  170.4 $\pm$    7.6 &   0.81 $\pm$   0.05 &     15 &  11.91 & 6.86e-01\\
 &   flat & - & - &   42.4 $\pm$   10.9 &    3.9 &  112.0 $\pm$   17.2 &   1.11 $\pm$   0.12 &     14 &   3.88 & 9.96e-01\\
 &      - & - & - &    0.0 & - &  172.6 $\pm$    7.5 &   0.79 $\pm$   0.04 &     15 &  12.98 & 6.04e-01\\
MACS J1115.2+5320 &   extr &     18 &   0.50 &  292.3 $\pm$   60.5 &    4.8 &   27.6 $\pm$   42.3 &   1.73 $\pm$   1.01 &     15 &   3.47 & 9.99e-01\\
 &      - & - & - &    0.0 & - &  334.8 $\pm$   32.1 &   0.33 $\pm$   0.10 &     16 &   6.98 & 9.74e-01\\
 &   flat & - & - &  292.3 $\pm$   60.5 &    4.8 &   27.6 $\pm$   42.3 &   1.73 $\pm$   1.01 &     15 &   3.47 & 9.99e-01\\
 &      - & - & - &    0.0 & - &  334.8 $\pm$   32.1 &   0.33 $\pm$   0.10 &     16 &   6.98 & 9.74e-01\\
MACS J1115.8+0129 &   extr &     20 &   0.20 &   14.1 $\pm$    5.1 &    2.8 &  265.5 $\pm$   18.4 &   1.26 $\pm$   0.11 &     17 &   5.12 & 9.97e-01\\
 &      - & - & - &    0.0 & - &  278.8 $\pm$   17.7 &   1.05 $\pm$   0.06 &     18 &  13.05 & 7.89e-01\\
 &   flat & - & - &   22.7 $\pm$    4.9 &    4.7 &  253.8 $\pm$   18.4 &   1.32 $\pm$   0.12 &     17 &   5.50 & 9.96e-01\\
 &      - & - & - &    0.0 & - &  270.5 $\pm$   17.7 &   0.96 $\pm$   0.05 &     18 &  24.02 & 1.54e-01\\
MACS J1131.8-1955 &   extr &     23 &   0.50 &   62.1 $\pm$   22.3 &    2.8 &  160.9 $\pm$   33.8 &   1.18 $\pm$   0.18 &     20 &   0.40 & 1.00e+00\\
 &      - & - & - &    0.0 & - &  246.2 $\pm$   16.5 &   0.84 $\pm$   0.08 &     21 &   6.22 & 9.99e-01\\
 &   flat & - & - &   97.3 $\pm$   23.0 &    4.2 &  156.3 $\pm$   34.7 &   1.15 $\pm$   0.19 &     20 &   0.69 & 1.00e+00\\
 &      - & - & - &    0.0 & - &  287.7 $\pm$   15.5 &   0.64 $\pm$   0.06 &     21 &   9.81 & 9.81e-01\\
MACS J1149.5+2223 &   extr &     32 &   1.00 &  280.7 $\pm$   39.2 &    7.2 &   33.1 $\pm$   20.6 &   1.47 $\pm$   0.30 &     29 &   1.62 & 1.00e+00\\
 &      - & - & - &    0.0 & - &  282.3 $\pm$   22.1 &   0.52 $\pm$   0.06 &     30 &  15.32 & 9.88e-01\\
 &   flat & - & - &  280.7 $\pm$   39.2 &    7.2 &   33.1 $\pm$   20.6 &   1.47 $\pm$   0.30 &     29 &   1.62 & 1.00e+00\\
 &      - & - & - &    0.0 & - &  282.3 $\pm$   22.1 &   0.52 $\pm$   0.06 &     30 &  15.32 & 9.88e-01\\
MACS J1206.2-0847 &   extr &     30 &   0.80 &   61.0 $\pm$   10.1 &    6.0 &   97.1 $\pm$   14.6 &   1.27 $\pm$   0.11 &     27 &   1.38 & 1.00e+00\\
 &      - & - & - &    0.0 & - &  181.0 $\pm$    8.5 &   0.84 $\pm$   0.05 &     28 &  25.36 & 6.08e-01\\
 &   flat & - & - &   69.0 $\pm$   10.1 &    6.8 &   94.7 $\pm$   14.5 &   1.28 $\pm$   0.11 &     27 &   1.87 & 1.00e+00\\
 &      - & - & - &    0.0 & - &  190.5 $\pm$    8.3 &   0.78 $\pm$   0.05 &     28 &  30.00 & 3.63e-01\\
MACS J1311.0-0310 &   extr &     14 &   0.40 &   42.5 $\pm$    4.2 &   10.1 &   67.1 $\pm$    7.4 &   1.58 $\pm$   0.12 &     11 &   2.47 & 9.96e-01\\
 &      - & - & - &    0.0 & - &  127.7 $\pm$    3.9 &   0.84 $\pm$   0.04 &     12 &  67.11 & 1.11e-09\\
 &   flat & - & - &   47.4 $\pm$    4.1 &   11.5 &   63.5 $\pm$    7.3 &   1.62 $\pm$   0.12 &     11 &   2.39 & 9.97e-01\\
 &      - & - & - &    0.0 & - &  130.2 $\pm$    3.9 &   0.77 $\pm$   0.04 &     12 &  77.77 & 1.10e-11\\
MACS J1621.3+3810 &   extr &     17 &   0.50 &   13.9 $\pm$    5.6 &    2.5 &  135.0 $\pm$   11.6 &   1.16 $\pm$   0.08 &     14 &   6.71 & 9.45e-01\\
 &      - & - & - &    0.0 & - &  158.9 $\pm$    5.8 &   1.01 $\pm$   0.04 &     15 &  11.72 & 7.00e-01\\
 &   flat & - & - &   20.1 $\pm$    5.4 &    3.7 &  129.8 $\pm$   11.4 &   1.18 $\pm$   0.08 &     14 &   7.04 & 9.33e-01\\
 &      - & - & - &    0.0 & - &  164.4 $\pm$    5.8 &   0.96 $\pm$   0.04 &     15 &  16.97 & 3.21e-01\\
MACS J1931.8-2634 &   extr &     16 &   0.40 &   10.3 $\pm$    3.8 &    2.7 &   93.7 $\pm$    9.3 &   1.22 $\pm$   0.10 &     13 &   4.58 & 9.83e-01\\
 &      - & - & - &    0.0 & - &  112.9 $\pm$    5.1 &   1.01 $\pm$   0.05 &     14 &  10.52 & 7.23e-01\\
 &   flat & - & - &   14.6 $\pm$    3.6 &    4.1 &   87.5 $\pm$    9.2 &   1.27 $\pm$   0.11 &     13 &   5.80 & 9.53e-01\\
 &      - & - & - &    0.0 & - &  114.6 $\pm$    5.1 &   0.97 $\pm$   0.04 &     14 &  17.89 & 2.12e-01\\
MACS J2049.9-3217 &   extr &     21 &   0.50 &  195.8 $\pm$   67.6 &    2.9 &   92.7 $\pm$   71.5 &   1.06 $\pm$   0.48 &     18 &   0.87 & 1.00e+00\\
 &      - & - & - &    0.0 & - &  309.0 $\pm$   25.4 &   0.43 $\pm$   0.08 &     19 &   3.69 & 1.00e+00\\
 &   flat & - & - &  195.8 $\pm$   67.6 &    2.9 &   92.7 $\pm$   71.5 &   1.06 $\pm$   0.48 &     18 &   0.87 & 1.00e+00\\
 &      - & - & - &    0.0 & - &  309.0 $\pm$   25.4 &   0.43 $\pm$   0.08 &     19 &   3.69 & 1.00e+00\\
MACS J2211.7-0349 &   extr &     29 &   0.60 &  165.5 $\pm$   25.5 &    6.5 &   78.3 $\pm$   26.3 &   1.59 $\pm$   0.24 &     26 &   0.89 & 1.00e+00\\
 &      - & - & - &    0.0 & - &  270.5 $\pm$   16.5 &   0.74 $\pm$   0.07 &     27 &  20.58 & 8.06e-01\\
 &   flat & - & - &  165.5 $\pm$   25.5 &    6.5 &   78.3 $\pm$   26.3 &   1.59 $\pm$   0.24 &     26 &   0.89 & 1.00e+00\\
 &      - & - & - &    0.0 & - &  270.5 $\pm$   16.5 &   0.74 $\pm$   0.07 &     27 &  20.58 & 8.06e-01\\
MACS J2214.9-1359 &   extr &     13 &   0.40 &  238.6 $\pm$   88.3 &    2.7 &  203.6 $\pm$  152.6 &   1.38 $\pm$   0.66 &     10 &   0.08 & 1.00e+00\\
 &      - & - & - &    0.0 & - &  507.6 $\pm$   70.9 &   0.52 $\pm$   0.16 &     11 &   2.25 & 9.97e-01\\
 &   flat & - & - &  297.7 $\pm$   83.2 &    3.6 &  172.0 $\pm$  147.7 &   1.46 $\pm$   0.76 &     10 &   0.10 & 1.00e+00\\
 &      - & - & - &    0.0 & - &  534.0 $\pm$   73.0 &   0.40 $\pm$   0.14 &     11 &   2.62 & 9.95e-01\\
MACS J2228+2036 &   extr &     22 &   0.60 &  118.8 $\pm$   39.2 &    3.0 &  107.2 $\pm$   45.9 &   1.00 $\pm$   0.26 &     19 &   0.60 & 1.00e+00\\
 &      - & - & - &    0.0 & - &  246.7 $\pm$   17.6 &   0.55 $\pm$   0.07 &     20 &   4.67 & 1.00e+00\\
 &   flat & - & - &  118.8 $\pm$   39.2 &    3.0 &  107.2 $\pm$   45.9 &   1.00 $\pm$   0.26 &     19 &   0.60 & 1.00e+00\\
 &      - & - & - &    0.0 & - &  246.7 $\pm$   17.6 &   0.55 $\pm$   0.07 &     20 &   4.67 & 1.00e+00\\
MACS J2229.7-2755 &   extr &     17 &   0.40 &   10.2 $\pm$    2.1 &    4.8 &   78.1 $\pm$    5.2 &   1.32 $\pm$   0.08 &     14 &  12.45 & 5.70e-01\\
 &      - & - & - &    0.0 & - &   95.0 $\pm$    3.4 &   1.04 $\pm$   0.04 &     15 &  30.08 & 1.16e-02\\
 &   flat & - & - &   12.4 $\pm$    2.0 &    6.1 &   75.0 $\pm$    5.2 &   1.36 $\pm$   0.08 &     14 &  13.61 & 4.79e-01\\
 &      - & - & - &    0.0 & - &   95.4 $\pm$    3.4 &   1.01 $\pm$   0.04 &     15 &  39.96 & 4.60e-04\\
MACS J2245.0+2637 &   extr &     23 &   0.50 &   39.0 $\pm$    6.6 &    5.9 &  108.5 $\pm$   13.1 &   1.31 $\pm$   0.12 &     20 &   0.54 & 1.00e+00\\
 &      - & - & - &    0.0 & - &  166.7 $\pm$    7.2 &   0.82 $\pm$   0.05 &     21 &  23.13 & 3.37e-01\\
 &   flat & - & - &   42.0 $\pm$    6.5 &    6.5 &  105.9 $\pm$   13.1 &   1.33 $\pm$   0.13 &     20 &   0.53 & 1.00e+00\\
 &      - & - & - &    0.0 & - &  168.1 $\pm$    7.2 &   0.79 $\pm$   0.05 &     21 &  25.90 & 2.10e-01\\
MKW3S &   extr &     46 &   0.20 &   20.7 $\pm$    1.7 &   12.1 &  134.8 $\pm$    2.6 &   0.93 $\pm$   0.03 &     43 &  26.23 & 9.80e-01\\
 &      - & - & - &    0.0 & - &  154.3 $\pm$    1.8 &   0.66 $\pm$   0.01 &     44 & 121.79 & 3.16e-09\\
 &   flat & - & - &   23.9 $\pm$    1.6 &   14.7 &  131.1 $\pm$    2.5 &   0.96 $\pm$   0.03 &     43 &  27.65 & 9.67e-01\\
 &      - & - & - &    0.0 & - &  153.5 $\pm$    1.8 &   0.65 $\pm$   0.01 &     44 & 159.12 & 6.08e-15\\
MKW 4 &   extr &     16 &   0.03 &    5.9 $\pm$    0.3 &   18.9 &  368.4 $\pm$   26.7 &   1.21 $\pm$   0.04 &     13 &  17.01 & 1.99e-01\\
 &      - & - & - &    0.0 & - &  164.0 $\pm$    6.7 &   0.74 $\pm$   0.01 &     14 & 233.26 & 8.23e-42\\
 &   flat & - & - &    6.9 $\pm$    0.3 &   23.0 &  392.7 $\pm$   29.4 &   1.26 $\pm$   0.04 &     13 &  19.05 & 1.21e-01\\
 &      - & - & - &    0.0 & - &  146.6 $\pm$    5.9 &   0.70 $\pm$   0.01 &     14 & 305.78 & 7.37e-57\\
MKW 8 &   extr &     19 &   0.05 &  130.7 $\pm$   22.4 &    5.8 &  228.5 $\pm$   54.2 &   0.87 $\pm$   0.40 &     16 &   0.44 & 1.00e+00\\
 &      - & - & - &    0.0 & - &  275.3 $\pm$   16.3 &   0.22 $\pm$   0.03 &     17 &   4.86 & 9.98e-01\\
 &   flat & - & - &  130.7 $\pm$   22.4 &    5.8 &  228.5 $\pm$   54.2 &   0.87 $\pm$   0.40 &     16 &   0.44 & 1.00e+00\\
 &      - & - & - &    0.0 & - &  275.3 $\pm$   16.3 &   0.22 $\pm$   0.03 &     17 &   4.86 & 9.98e-01\\
MS J0016.9+1609 &   extr &     16 &   0.50 &  160.7 $\pm$   22.6 &    7.1 &   65.0 $\pm$   26.7 &   1.28 $\pm$   0.30 &     13 &   3.17 & 9.97e-01\\
 &      - & - & - &    0.0 & - &  258.5 $\pm$   11.8 &   0.40 $\pm$   0.05 &     14 &  15.63 & 3.37e-01\\
 &   flat & - & - &  162.1 $\pm$   22.5 &    7.2 &   64.2 $\pm$   26.5 &   1.29 $\pm$   0.30 &     13 &   3.17 & 9.97e-01\\
 &      - & - & - &    0.0 & - &  259.3 $\pm$   11.7 &   0.40 $\pm$   0.05 &     14 &  15.74 & 3.30e-01\\
MS J0116.3-0115 &   extr &     22 &   0.10 &   17.2 $\pm$   32.0 &    0.5 &  214.2 $\pm$   24.7 &   0.62 $\pm$   0.23 &     19 &   2.51 & 1.00e+00\\
 &      - & - & - &    0.0 & - &  225.3 $\pm$   14.8 &   0.52 $\pm$   0.05 &     20 &   3.02 & 1.00e+00\\
 &   flat & - & - &   12.8 $\pm$   31.0 &    0.4 &  220.8 $\pm$   24.1 &   0.63 $\pm$   0.22 &     19 &   2.53 & 1.00e+00\\
 &      - & - & - &    0.0 & - &  228.7 $\pm$   15.1 &   0.55 $\pm$   0.05 &     20 &   2.96 & 1.00e+00\\
MS J0440.5+0204 &   extr &     19 &   0.30 &   22.8 $\pm$    7.6 &    3.0 &  165.5 $\pm$   15.1 &   1.11 $\pm$   0.13 &     16 &   5.73 & 9.91e-01\\
 &      - & - & - &    0.0 & - &  196.6 $\pm$    9.6 &   0.82 $\pm$   0.06 &     17 &  11.13 & 8.50e-01\\
 &   flat & - & - &   25.5 $\pm$    7.6 &    3.4 &  164.0 $\pm$   15.2 &   1.11 $\pm$   0.13 &     16 &   6.15 & 9.86e-01\\
 &      - & - & - &    0.0 & - &  198.0 $\pm$    9.6 &   0.79 $\pm$   0.05 &     17 &  12.34 & 7.79e-01\\
MS J0451.6-0305 &   extr &     16 &   0.50 &  568.1 $\pm$  115.6 &    4.9 &   15.6 $\pm$   49.9 &   2.81 $\pm$   2.27 &     13 &   0.56 & 1.00e+00\\
 &      - & - & - &    0.0 & - &  643.5 $\pm$   79.7 &   0.21 $\pm$   0.16 &     14 &   3.73 & 9.97e-01\\
 &   flat & - & - &  568.1 $\pm$  115.6 &    4.9 &   15.6 $\pm$   49.9 &   2.81 $\pm$   2.27 &     13 &   0.56 & 1.00e+00\\
 &      - & - & - &    0.0 & - &  643.5 $\pm$   79.7 &   0.21 $\pm$   0.16 &     14 &   3.73 & 9.97e-01\\
MS J0735.6+7421 &   extr &     18 &   0.30 &   13.8 $\pm$    2.2 &    6.3 &  109.9 $\pm$    4.6 &   1.12 $\pm$   0.05 &     15 &  22.06 & 1.06e-01\\
 &      - & - & - &    0.0 & - &  131.3 $\pm$    2.7 &   0.89 $\pm$   0.02 &     16 &  60.72 & 3.95e-07\\
 &   flat & - & - &   16.0 $\pm$    2.1 &    7.5 &  106.8 $\pm$    4.6 &   1.14 $\pm$   0.05 &     15 &  25.59 & 4.26e-02\\
 &      - & - & - &    0.0 & - &  131.5 $\pm$    2.7 &   0.87 $\pm$   0.02 &     16 &  77.93 & 3.92e-10\\
MS J0839.8+2938 &   extr &     16 &   0.25 &   15.5 $\pm$    3.1 &    5.1 &  110.7 $\pm$    6.3 &   1.26 $\pm$   0.11 &     13 &   3.12 & 9.98e-01\\
 &      - & - & - &    0.0 & - &  127.3 $\pm$    4.9 &   0.88 $\pm$   0.04 &     14 &  21.16 & 9.75e-02\\
 &   flat & - & - &   19.2 $\pm$    2.9 &    6.7 &  105.8 $\pm$    6.3 &   1.33 $\pm$   0.11 &     13 &   2.50 & 9.99e-01\\
 &      - & - & - &    0.0 & - &  126.1 $\pm$    4.9 &   0.84 $\pm$   0.04 &     14 &  30.67 & 6.17e-03\\
MS J0906.5+1110 &   extr &     29 &   0.40 &  104.2 $\pm$   14.9 &    7.0 &   97.3 $\pm$   19.6 &   1.15 $\pm$   0.17 &     26 &   1.25 & 1.00e+00\\
 &      - & - & - &    0.0 & - &  222.7 $\pm$    6.4 &   0.54 $\pm$   0.04 &     27 &  19.62 & 8.46e-01\\
 &   flat & - & - &  104.2 $\pm$   14.9 &    7.0 &   97.3 $\pm$   19.6 &   1.15 $\pm$   0.17 &     26 &   1.25 & 1.00e+00\\
 &      - & - & - &    0.0 & - &  222.7 $\pm$    6.4 &   0.54 $\pm$   0.04 &     27 &  19.62 & 8.46e-01\\
MS J1006.0+1202 &   extr &     29 &   0.50 &  175.8 $\pm$   20.1 &    8.7 &   71.7 $\pm$   25.0 &   1.40 $\pm$   0.26 &     26 &   7.00 & 1.00e+00\\
 &      - & - & - &    0.0 & - &  285.4 $\pm$   12.1 &   0.41 $\pm$   0.05 &     27 &  29.77 & 3.25e-01\\
 &   flat & - & - &  160.3 $\pm$   21.3 &    7.5 &   82.8 $\pm$   26.9 &   1.32 $\pm$   0.24 &     26 &   6.68 & 1.00e+00\\
 &      - & - & - &    0.0 & - &  278.4 $\pm$   12.2 &   0.46 $\pm$   0.05 &     27 &  26.32 & 5.01e-01\\
MS J1008.1-1224 &   extr &     23 &   0.50 &   96.0 $\pm$   40.7 &    2.4 &  260.2 $\pm$   56.0 &   0.77 $\pm$   0.18 &     20 &   1.45 & 1.00e+00\\
 &      - & - & - &    0.0 & - &  373.9 $\pm$   18.0 &   0.49 $\pm$   0.05 &     21 &   4.07 & 1.00e+00\\
 &   flat & - & - &   97.6 $\pm$   41.5 &    2.4 &  262.0 $\pm$   56.8 &   0.76 $\pm$   0.18 &     20 &   1.50 & 1.00e+00\\
 &      - & - & - &    0.0 & - &  377.0 $\pm$   18.1 &   0.48 $\pm$   0.05 &     21 &   4.07 & 1.00e+00\\
MS J1455.0+2232 &   extr &     16 &   0.30 &   16.9 $\pm$    1.5 &   11.1 &   81.5 $\pm$    4.0 &   1.39 $\pm$   0.07 &     13 &  10.09 & 6.86e-01\\
 &      - & - & - &    0.0 & - &  107.3 $\pm$    2.7 &   0.86 $\pm$   0.03 &     14 &  80.05 & 2.76e-11\\
 &   flat & - & - &   16.9 $\pm$    1.5 &   11.1 &   81.5 $\pm$    4.0 &   1.39 $\pm$   0.07 &     13 &  10.09 & 6.86e-01\\
 &      - & - & - &    0.0 & - &  107.3 $\pm$    2.7 &   0.86 $\pm$   0.03 &     14 &  80.05 & 2.76e-11\\
MS J2137.3-2353 &   extr &     22 &   0.50 &   12.3 $\pm$    1.9 &    6.5 &   93.5 $\pm$    5.3 &   1.36 $\pm$   0.06 &     19 &   5.01 & 9.99e-01\\
 &      - & - & - &    0.0 & - &  116.9 $\pm$    3.4 &   1.08 $\pm$   0.03 &     20 &  36.15 & 1.47e-02\\
 &   flat & - & - &   14.7 $\pm$    1.8 &    7.9 &   89.9 $\pm$    5.3 &   1.39 $\pm$   0.06 &     19 &   5.76 & 9.98e-01\\
 &      - & - & - &    0.0 & - &  117.6 $\pm$    3.4 &   1.05 $\pm$   0.03 &     20 &  50.37 & 1.96e-04\\
MS J1157.3+5531 &   extr &     13 &   0.10 &    4.1 $\pm$    0.4 &    9.7 &  283.8 $\pm$   17.7 &   1.44 $\pm$   0.05 &     10 &   7.54 & 6.74e-01\\
 &      - & - & - &    0.0 & - &  196.2 $\pm$    9.6 &   1.09 $\pm$   0.02 &     11 &  64.85 & 1.15e-09\\
 &   flat & - & - &    5.9 $\pm$    0.4 &   13.9 &  277.0 $\pm$   17.7 &   1.45 $\pm$   0.05 &     10 &   7.22 & 7.04e-01\\
 &      - & - & - &    0.0 & - &  160.6 $\pm$    7.7 &   0.95 $\pm$   0.02 &     11 &  96.24 & 9.86e-16\\
NGC 507 &   extr &     61 &   0.05 &    0.0 $\pm$    2.1 &    0.0 &  101.7 $\pm$    2.8 &   0.67 $\pm$   0.01 &     58 &  42.84 & 9.32e-01\\
 &      - & - & - &    0.0 & - &  101.7 $\pm$    2.8 &   0.67 $\pm$   0.01 &     59 &  42.84 & 9.44e-01\\
 &   flat & - & - &    0.0 $\pm$    2.1 &    0.0 &   99.9 $\pm$    2.7 &   0.65 $\pm$   0.01 &     58 &  46.55 & 8.60e-01\\
 &      - & - & - &    0.0 & - &   99.9 $\pm$    2.7 &   0.65 $\pm$   0.01 &     59 &  46.55 & 8.80e-01\\
NGC 4636 &   extr &     12 &   0.00 &    1.4 $\pm$    0.1 &   13.4 & 10674.9 $\pm$ 7937.9 &   1.93 $\pm$   0.18 &      9 &   8.12 & 5.22e-01\\
 &      - & - & - &    0.0 & - &  108.2 $\pm$   19.2 &   0.77 $\pm$   0.04 &     10 &  56.25 & 1.84e-08\\
 &   flat & - & - &    1.4 $\pm$    0.1 &   13.9 & 11962.1 $\pm$ 8977.0 &   1.96 $\pm$   0.18 &      9 &   8.95 & 4.42e-01\\
 &      - & - & - &    0.0 & - &  104.9 $\pm$   18.6 &   0.77 $\pm$   0.04 &     10 &  60.03 & 3.58e-09\\
NGC 5044 &   extr &     66 &   0.03 &    1.9 $\pm$    0.3 &    7.2 &   79.6 $\pm$    6.7 &   0.93 $\pm$   0.05 &     63 &  49.49 & 8.93e-01\\
 &      - & - & - &    0.0 & - &   55.1 $\pm$    2.4 &   0.67 $\pm$   0.02 &     64 &  77.04 & 1.27e-01\\
 &   flat & - & - &    2.3 $\pm$    0.3 &    8.9 &   82.2 $\pm$    7.2 &   0.96 $\pm$   0.05 &     63 &  48.05 & 9.18e-01\\
 &      - & - & - &    0.0 & - &   52.3 $\pm$    2.2 &   0.64 $\pm$   0.02 &     64 &  86.52 & 3.19e-02\\
NGC 5813 &   extr &     60 &   0.02 &    1.4 $\pm$    0.2 &    8.9 &  102.5 $\pm$    7.1 &   0.91 $\pm$   0.03 &     57 & 107.52 & 6.00e-05\\
 &      - & - & - &    0.0 & - &   69.3 $\pm$    2.1 &   0.70 $\pm$   0.01 &     58 & 161.30 & 1.14e-11\\
 &   flat & - & - &    1.4 $\pm$    0.2 &    8.9 &  102.5 $\pm$    7.1 &   0.91 $\pm$   0.03 &     57 & 107.52 & 6.00e-05\\
 &      - & - & - &    0.0 & - &   69.3 $\pm$    2.1 &   0.70 $\pm$   0.01 &     58 & 161.30 & 1.14e-11\\
NGC 5846 &   extr &     16 &   0.00 &    1.8 $\pm$    0.2 &   10.7 &  685.8 $\pm$  344.9 &   1.44 $\pm$   0.15 &     13 &   1.16 & 1.00e+00\\
 &      - & - & - &    0.0 & - &   52.7 $\pm$    7.3 &   0.63 $\pm$   0.03 &     14 &  40.72 & 1.97e-04\\
 &   flat & - & - &    1.8 $\pm$    0.2 &   10.7 &  685.8 $\pm$  344.9 &   1.44 $\pm$   0.15 &     13 &   1.16 & 1.00e+00\\
 &      - & - & - &    0.0 & - &   52.7 $\pm$    7.3 &   0.63 $\pm$   0.03 &     14 &  40.72 & 1.97e-04\\
Ophiuchus &   extr &     18 &   0.05 &    4.0 $\pm$    0.6 &    6.3 &  375.1 $\pm$   12.8 &   1.06 $\pm$   0.03 &     15 &   9.75 & 8.35e-01\\
 &      - & - & - &    0.0 & - &  328.4 $\pm$    7.8 &   0.92 $\pm$   0.01 &     16 &  42.24 & 3.63e-04\\
 &   flat & - & - &    8.9 $\pm$    1.2 &    7.5 &  247.5 $\pm$    7.6 &   0.73 $\pm$   0.03 &     15 &  95.06 & 1.12e-13\\
 &      - & - & - &    0.0 & - &  217.0 $\pm$    3.9 &   0.58 $\pm$   0.01 &     16 & 127.43 & 2.02e-19\\
PKS 0745-191 &   extr &     34 &   0.30 &   11.9 $\pm$    0.7 &   17.4 &  111.7 $\pm$    2.7 &   1.38 $\pm$   0.04 &     31 &  17.17 & 9.79e-01\\
 &      - & - & - &    0.0 & - &  129.2 $\pm$    2.4 &   0.98 $\pm$   0.02 &     32 & 245.68 & 8.53e-35\\
 &   flat & - & - &   12.4 $\pm$    0.7 &   18.3 &  110.7 $\pm$    2.7 &   1.39 $\pm$   0.04 &     31 &  19.54 & 9.45e-01\\
 &      - & - & - &    0.0 & - &  128.9 $\pm$    2.4 &   0.97 $\pm$   0.02 &     32 & 270.30 & 1.59e-39\\
RBS 461 &   extr &     70 &   0.20 &   95.7 $\pm$    3.0 &   31.4 &   68.8 $\pm$    4.5 &   1.39 $\pm$   0.10 &     67 &  22.14 & 1.00e+00\\
 &      - & - & - &    0.0 & - &  173.2 $\pm$    1.8 &   0.35 $\pm$   0.01 &     68 & 217.68 & 1.45e-17\\
 &   flat & - & - &   95.7 $\pm$    3.0 &   31.4 &   68.8 $\pm$    4.5 &   1.39 $\pm$   0.10 &     67 &  22.14 & 1.00e+00\\
 &      - & - & - &    0.0 & - &  173.2 $\pm$    1.8 &   0.35 $\pm$   0.01 &     68 & 217.68 & 1.45e-17\\
RBS 533 &   extr &     44 &   0.06 &    2.0 $\pm$   0.05 &   39.5 &  162.8 $\pm$    2.5 &   0.99 $\pm$   0.01 &     41 & 202.89 & 2.65e-23\\
 &      - & - & - &    0.0 & - &  113.5 $\pm$    1.3 &   0.76 $\pm$   0.00 &     42 & 1282.66 & 1.75e-241\\
 &   flat & - & - &    2.2 $\pm$   0.05 &   43.7 &  164.3 $\pm$    2.5 &   1.00 $\pm$   0.01 &     41 & 215.65 & 1.46e-25\\
 &      - & - & - &    0.0 & - &  110.0 $\pm$    1.3 &   0.75 $\pm$   0.00 &     42 & 1490.02 & 3.27e-285\\
RBS 797 &   extr &     24 &   0.30 &   20.0 $\pm$    2.4 &    8.3 &   95.2 $\pm$    9.0 &   1.72 $\pm$   0.14 &     21 &  89.64 & 1.86e-10\\
 &      - & - & - &    0.0 & - &  116.2 $\pm$    8.0 &   0.98 $\pm$   0.06 &     22 & 1061.58 & 1.51e-210\\
 &   flat & - & - &   20.9 $\pm$    2.4 &    8.9 &   93.2 $\pm$    9.1 &   1.75 $\pm$   0.15 &     21 & 104.70 & 4.22e-13\\
 &      - & - & - &    0.0 & - &  114.6 $\pm$    8.0 &   0.96 $\pm$   0.06 &     22 & 1188.56 & 1.25e-237\\
RCS J2327-0204 &   extr &     18 &   0.30 &   65.5 $\pm$   20.2 &    3.2 &  220.6 $\pm$   37.0 &   1.27 $\pm$   0.25 &     15 &  31.21 & 8.24e-03\\
 &      - & - & - &    0.0 & - &  300.3 $\pm$   22.5 &   0.74 $\pm$   0.09 &     16 & 119.10 & 8.17e-18\\
 &   flat & - & - &   68.5 $\pm$   19.9 &    3.4 &  217.2 $\pm$   36.9 &   1.28 $\pm$   0.26 &     15 &  31.00 & 8.80e-03\\
 &      - & - & - &    0.0 & - &  300.1 $\pm$   22.6 &   0.73 $\pm$   0.09 &     16 & 126.00 & 3.83e-19\\
RXCJ0331.1-2100 &   extr &     25 &   0.20 &    6.4 $\pm$    1.6 &    4.1 &  141.0 $\pm$    5.8 &   1.23 $\pm$   0.06 &     22 & 325.76 & 7.05e-56\\
 &      - & - & - &    0.0 & - &  145.9 $\pm$    5.7 &   1.05 $\pm$   0.03 &     23 & 677.70 & 2.20e-128\\
 &   flat & - & - &   11.4 $\pm$    1.5 &    7.7 &  134.1 $\pm$    5.8 &   1.30 $\pm$   0.07 &     22 & 356.18 & 4.25e-62\\
 &      - & - & - &    0.0 & - &  140.5 $\pm$    5.7 &   0.95 $\pm$   0.03 &     23 & 1408.70 & 8.65e-284\\
RX J0220.9-3829 &   extr &     22 &   0.40 &   33.1 $\pm$    6.2 &    5.3 &  163.7 $\pm$   14.0 &   1.25 $\pm$   0.11 &     19 &   3.90 & 1.00e+00\\
 &      - & - & - &    0.0 & - &  211.1 $\pm$    9.0 &   0.84 $\pm$   0.05 &     20 &  20.59 & 4.22e-01\\
 &   flat & - & - &   43.0 $\pm$    6.3 &    6.8 &  159.9 $\pm$   14.0 &   1.23 $\pm$   0.12 &     19 &   4.20 & 1.00e+00\\
 &      - & - & - &    0.0 & - &  216.2 $\pm$    9.2 &   0.73 $\pm$   0.04 &     20 &  25.95 & 1.68e-01\\
RX J0232.2-4420 &   extr &     14 &   0.30 &   34.2 $\pm$   13.0 &    2.6 &  176.3 $\pm$   25.0 &   1.12 $\pm$   0.18 &     11 &   0.85 & 1.00e+00\\
 &      - & - & - &    0.0 & - &  225.4 $\pm$   13.1 &   0.80 $\pm$   0.06 &     12 &   5.16 & 9.53e-01\\
 &   flat & - & - &   44.6 $\pm$   12.4 &    3.6 &  166.5 $\pm$   24.7 &   1.16 $\pm$   0.18 &     11 &   0.71 & 1.00e+00\\
 &      - & - & - &    0.0 & - &  228.9 $\pm$   13.2 &   0.74 $\pm$   0.06 &     12 &   7.42 & 8.28e-01\\
RX J0439+0520 &   extr &     18 &   0.30 &   12.8 $\pm$    2.9 &    4.5 &   97.1 $\pm$    6.2 &   1.18 $\pm$   0.10 &     15 &   6.80 & 9.63e-01\\
 &      - & - & - &    0.0 & - &  112.8 $\pm$    4.6 &   0.86 $\pm$   0.04 &     16 &  19.20 & 2.59e-01\\
 &   flat & - & - &   14.9 $\pm$    2.9 &    5.2 &   95.5 $\pm$    6.2 &   1.19 $\pm$   0.10 &     15 &   6.64 & 9.67e-01\\
 &      - & - & - &    0.0 & - &  113.0 $\pm$    4.6 &   0.82 $\pm$   0.04 &     16 &  21.93 & 1.45e-01\\
RX J0439.0+0715 &   extr &     22 &   0.40 &   61.2 $\pm$   21.3 &    2.9 &  152.0 $\pm$   31.1 &   0.95 $\pm$   0.18 &     19 &   5.54 & 9.99e-01\\
 &      - & - & - &    0.0 & - &  212.0 $\pm$   10.6 &   0.68 $\pm$   0.06 &     20 &   8.75 & 9.86e-01\\
 &   flat & - & - &   66.8 $\pm$   18.5 &    3.6 &  129.6 $\pm$   28.4 &   1.06 $\pm$   0.20 &     19 &   6.20 & 9.97e-01\\
 &      - & - & - &    0.0 & - &  217.0 $\pm$   10.5 &   0.63 $\pm$   0.06 &     20 &  13.41 & 8.59e-01\\
RX J0528.9-3927 &   extr &     21 &   0.40 &   69.9 $\pm$   13.9 &    5.0 &  102.2 $\pm$   22.6 &   1.45 $\pm$   0.23 &     18 &   1.71 & 1.00e+00\\
 &      - & - & - &    0.0 & - &  201.5 $\pm$   11.3 &   0.74 $\pm$   0.08 &     19 &  15.10 & 7.16e-01\\
 &   flat & - & - &   72.9 $\pm$   13.8 &    5.3 &   99.8 $\pm$   22.4 &   1.47 $\pm$   0.23 &     18 &   1.67 & 1.00e+00\\
 &      - & - & - &    0.0 & - &  203.1 $\pm$   11.3 &   0.72 $\pm$   0.07 &     19 &  15.94 & 6.61e-01\\
RX J0647.7+7015 &   extr &     24 &   0.80 &  225.1 $\pm$   47.1 &    4.8 &   48.8 $\pm$   31.9 &   1.70 $\pm$   0.39 &     21 &   0.42 & 1.00e+00\\
 &      - & - & - &    0.0 & - &  275.6 $\pm$   32.0 &   0.71 $\pm$   0.10 &     22 &   9.72 & 9.89e-01\\
 &   flat & - & - &  225.1 $\pm$   47.1 &    4.8 &   48.8 $\pm$   31.9 &   1.70 $\pm$   0.39 &     21 &   0.42 & 1.00e+00\\
 &      - & - & - &    0.0 & - &  275.6 $\pm$   32.0 &   0.71 $\pm$   0.10 &     22 &   9.72 & 9.89e-01\\
RX J0819.6+6336 &   extr &     28 &   0.30 &   20.7 $\pm$   14.3 &    1.5 &  170.6 $\pm$   19.4 &   0.68 $\pm$   0.12 &     25 &  10.13 & 9.96e-01\\
 &      - & - & - &    0.0 & - &  194.0 $\pm$    8.8 &   0.55 $\pm$   0.04 &     26 &  11.55 & 9.93e-01\\
 &   flat & - & - &   20.7 $\pm$   14.3 &    1.5 &  170.6 $\pm$   19.4 &   0.68 $\pm$   0.12 &     25 &  10.13 & 9.96e-01\\
 &      - & - & - &    0.0 & - &  194.0 $\pm$    8.8 &   0.55 $\pm$   0.04 &     26 &  11.55 & 9.93e-01\\
RX J1000.4+4409 &   extr &     23 &   0.30 &   23.1 $\pm$    4.3 &    5.4 &  151.7 $\pm$    9.9 &   1.12 $\pm$   0.09 &     20 &   1.85 & 1.00e+00\\
 &      - & - & - &    0.0 & - &  182.2 $\pm$    7.1 &   0.77 $\pm$   0.04 &     21 &  18.65 & 6.07e-01\\
 &   flat & - & - &   27.7 $\pm$    4.4 &    6.3 &  151.1 $\pm$    9.9 &   1.09 $\pm$   0.09 &     20 &   1.94 & 1.00e+00\\
 &      - & - & - &    0.0 & - &  184.9 $\pm$    7.2 &   0.71 $\pm$   0.03 &     21 &  21.59 & 4.24e-01\\
RX J1022.1+3830 &   extr &     18 &   0.09 &   44.0 $\pm$   10.0 &    4.4 &  206.8 $\pm$   18.5 &   1.03 $\pm$   0.21 &     15 &   7.73 & 9.34e-01\\
 &      - & - & - &    0.0 & - &  208.7 $\pm$   11.4 &   0.54 $\pm$   0.04 &     16 &  13.56 & 6.32e-01\\
 &   flat & - & - &   51.6 $\pm$    9.8 &    5.3 &  194.8 $\pm$   18.7 &   1.04 $\pm$   0.22 &     15 &   8.26 & 9.13e-01\\
 &      - & - & - &    0.0 & - &  201.1 $\pm$   10.7 &   0.48 $\pm$   0.04 &     16 &  14.68 & 5.48e-01\\
RX J1130.0+3637 &   extr &     26 &   0.15 &   23.4 $\pm$    2.2 &   10.7 &  158.7 $\pm$    9.3 &   1.19 $\pm$   0.09 &     23 &   2.01 & 1.00e+00\\
 &      - & - & - &    0.0 & - &  140.8 $\pm$    6.7 &   0.60 $\pm$   0.03 &     24 &  54.32 & 3.86e-04\\
 &   flat & - & - &   29.9 $\pm$    2.3 &   12.9 &  149.6 $\pm$    9.2 &   1.14 $\pm$   0.10 &     23 &   2.81 & 1.00e+00\\
 &      - & - & - &    0.0 & - &  133.0 $\pm$    6.0 &   0.48 $\pm$   0.02 &     24 &  58.11 & 1.18e-04\\
RX J1320.2+3308 &   extr &     11 &   0.04 &    7.6 $\pm$    0.6 &   12.1 &  162.6 $\pm$   26.6 &   1.36 $\pm$   0.12 &      8 &   5.25 & 7.31e-01\\
 &      - & - & - &    0.0 & - &   67.6 $\pm$    4.2 &   0.61 $\pm$   0.03 &      9 &  50.82 & 7.56e-08\\
 &   flat & - & - &    8.8 $\pm$    0.7 &   13.1 &  140.3 $\pm$   23.4 &   1.28 $\pm$   0.12 &      8 &   7.01 & 5.36e-01\\
 &      - & - & - &    0.0 & - &   59.9 $\pm$    3.4 &   0.53 $\pm$   0.02 &      9 &  49.88 & 1.13e-07\\
RX J1347.5-1145 &   extr &      8 &   0.22 &   12.5 $\pm$   20.7 &    0.6 &  179.9 $\pm$   35.3 &   1.06 $\pm$   0.34 &      5 &   4.00 & 5.49e-01\\
 &      - & - & - &    0.0 & - &  196.4 $\pm$   18.3 &   0.90 $\pm$   0.08 &      6 &   4.23 & 6.46e-01\\
 &   flat & - & - &   12.5 $\pm$   20.7 &    0.6 &  179.9 $\pm$   35.3 &   1.06 $\pm$   0.34 &      5 &   4.00 & 5.49e-01\\
 &      - & - & - &    0.0 & - &  196.4 $\pm$   18.3 &   0.90 $\pm$   0.08 &      6 &   4.23 & 6.46e-01\\
RX J1423.8+2404 &   extr &      7 &   0.22 &   10.2 $\pm$    5.0 &    2.0 &  119.9 $\pm$   10.8 &   1.27 $\pm$   0.17 &      4 &   1.75 & 7.82e-01\\
 &      - & - & - &    0.0 & - &  133.8 $\pm$    7.3 &   1.02 $\pm$   0.05 &      5 &  15.01 & 1.03e-02\\
 &   flat & - & - &   10.2 $\pm$    5.0 &    2.0 &  119.9 $\pm$   10.8 &   1.27 $\pm$   0.17 &      4 &   1.75 & 7.82e-01\\
 &      - & - & - &    0.0 & - &  133.8 $\pm$    7.3 &   1.02 $\pm$   0.05 &      5 &  15.01 & 1.03e-02\\
RX J1504.1-0248 &   extr &     27 &   0.45 &   13.1 $\pm$    0.9 &   13.9 &   95.6 $\pm$    3.5 &   1.50 $\pm$   0.04 &     24 &   2.89 & 1.00e+00\\
 &      - & - & - &    0.0 & - &  121.2 $\pm$    2.7 &   1.09 $\pm$   0.02 &     25 & 154.86 & 1.07e-20\\
 &   flat & - & - &   13.1 $\pm$    0.9 &   13.9 &   95.6 $\pm$    3.5 &   1.50 $\pm$   0.04 &     24 &   2.89 & 1.00e+00\\
 &      - & - & - &    0.0 & - &  121.2 $\pm$    2.7 &   1.09 $\pm$   0.02 &     25 & 154.86 & 1.07e-20\\
RX J1532.9+3021 &   extr &     21 &   0.50 &   14.3 $\pm$    1.9 &    7.6 &   80.3 $\pm$    5.0 &   1.46 $\pm$   0.07 &     18 &   2.24 & 1.00e+00\\
 &      - & - & - &    0.0 & - &  105.6 $\pm$    3.3 &   1.08 $\pm$   0.04 &     19 &  48.03 & 2.54e-04\\
 &   flat & - & - &   16.9 $\pm$    1.8 &    9.3 &   76.3 $\pm$    5.0 &   1.51 $\pm$   0.07 &     18 &   2.38 & 1.00e+00\\
 &      - & - & - &    0.0 & - &  106.1 $\pm$    3.3 &   1.04 $\pm$   0.04 &     19 &  67.16 & 2.71e-07\\
RX J1539.5-8335 &   extr &     29 &   0.20 &   21.8 $\pm$    3.1 &    7.1 &  115.1 $\pm$    5.8 &   1.32 $\pm$   0.11 &     26 &  13.29 & 9.81e-01\\
 &      - & - & - &    0.0 & - &  135.3 $\pm$    4.5 &   0.83 $\pm$   0.04 &     27 &  40.39 & 4.71e-02\\
 &   flat & - & - &   25.9 $\pm$    2.9 &    9.1 &  110.0 $\pm$    5.8 &   1.41 $\pm$   0.12 &     26 &  13.52 & 9.79e-01\\
 &      - & - & - &    0.0 & - &  133.7 $\pm$    4.5 &   0.79 $\pm$   0.04 &     27 &  54.08 & 1.49e-03\\
RX J1720.1+2638 &   extr &     30 &   0.40 &   20.7 $\pm$    1.9 &   10.7 &  109.7 $\pm$    5.4 &   1.38 $\pm$   0.06 &     27 &   5.34 & 1.00e+00\\
 &      - & - & - &    0.0 & - &  145.3 $\pm$    3.6 &   0.98 $\pm$   0.03 &     28 &  94.37 & 4.06e-09\\
 &   flat & - & - &   21.0 $\pm$    1.9 &   10.9 &  109.1 $\pm$    5.4 &   1.39 $\pm$   0.06 &     27 &   5.56 & 1.00e+00\\
 &      - & - & - &    0.0 & - &  145.3 $\pm$    3.6 &   0.98 $\pm$   0.03 &     28 &  97.94 & 1.09e-09\\
RX J1720.2+3536 &   extr &     13 &   0.32 &   17.5 $\pm$    3.5 &    4.9 &  101.8 $\pm$    7.9 &   1.35 $\pm$   0.10 &     10 &   2.47 & 9.91e-01\\
 &      - & - & - &    0.0 & - &  129.4 $\pm$    4.7 &   1.00 $\pm$   0.04 &     11 &  23.76 & 1.38e-02\\
 &   flat & - & - &   24.0 $\pm$    3.3 &    7.2 &   94.4 $\pm$    7.8 &   1.42 $\pm$   0.11 &     10 &   2.67 & 9.88e-01\\
 &      - & - & - &    0.0 & - &  131.3 $\pm$    4.7 &   0.92 $\pm$   0.04 &     11 &  40.43 & 3.02e-05\\
RX J1852.1+5711 &   extr &     12 &   0.12 &   13.7 $\pm$    6.3 &    2.2 &  184.3 $\pm$   12.8 &   0.96 $\pm$   0.15 &      9 &   2.63 & 9.77e-01\\
 &      - & - & - &    0.0 & - &  182.4 $\pm$   10.9 &   0.73 $\pm$   0.05 &     10 &   5.31 & 8.70e-01\\
 &   flat & - & - &   18.7 $\pm$    8.3 &    2.3 &  170.4 $\pm$   11.8 &   0.83 $\pm$   0.16 &      9 &   5.06 & 8.29e-01\\
 &      - & - & - &    0.0 & - &  173.3 $\pm$    9.8 &   0.58 $\pm$   0.04 &     10 &   7.26 & 7.01e-01\\
RX J2129.6+0005 &   extr &     22 &   0.40 &   18.0 $\pm$    3.8 &    4.7 &  100.8 $\pm$    8.1 &   1.24 $\pm$   0.10 &     19 &   7.01 & 9.94e-01\\
 &      - & - & - &    0.0 & - &  129.2 $\pm$    4.8 &   0.91 $\pm$   0.05 &     20 &  21.36 & 3.76e-01\\
 &   flat & - & - &   21.1 $\pm$    3.7 &    5.7 &   97.9 $\pm$    8.0 &   1.26 $\pm$   0.10 &     19 &   7.16 & 9.93e-01\\
 &      - & - & - &    0.0 & - &  130.8 $\pm$    4.8 &   0.87 $\pm$   0.04 &     20 &  26.01 & 1.66e-01\\
SC 1327-312 &   extr &     31 &   0.15 &   65.5 $\pm$   10.1 &    6.5 &  160.4 $\pm$   12.5 &   0.80 $\pm$   0.14 &     28 &   1.08 & 1.00e+00\\
 &      - & - & - &    0.0 & - &  212.5 $\pm$    8.1 &   0.36 $\pm$   0.03 &     29 &  15.85 & 9.77e-01\\
 &   flat & - & - &   64.6 $\pm$    9.9 &    6.5 &  160.8 $\pm$   12.5 &   0.81 $\pm$   0.14 &     28 &   1.03 & 1.00e+00\\
 &      - & - & - &    0.0 & - &  212.0 $\pm$    8.1 &   0.37 $\pm$   0.03 &     29 &  16.01 & 9.75e-01\\
Sersic 159-03 &   extr &     23 &   0.12 &    7.5 $\pm$    0.8 &    9.7 &   79.7 $\pm$    2.3 &   1.06 $\pm$   0.05 &     20 &  15.95 & 7.20e-01\\
 &      - & - & - &    0.0 & - &   77.9 $\pm$    2.0 &   0.72 $\pm$   0.02 &     21 &  77.11 & 2.44e-08\\
 &   flat & - & - &   10.5 $\pm$    0.7 &   15.0 &   77.8 $\pm$    2.4 &   1.17 $\pm$   0.06 &     20 &  16.81 & 6.65e-01\\
 &      - & - & - &    0.0 & - &   74.0 $\pm$    1.9 &   0.65 $\pm$   0.02 &     21 & 136.22 & 7.00e-19\\
SS2B153 &   extr &     38 &   0.07 &    1.1 $\pm$    0.2 &    6.9 &   71.4 $\pm$    2.1 &   0.80 $\pm$   0.02 &     35 &  24.19 & 9.15e-01\\
 &      - & - & - &    0.0 & - &   63.4 $\pm$    1.4 &   0.69 $\pm$   0.01 &     36 &  59.46 & 8.24e-03\\
 &   flat & - & - &    1.1 $\pm$    0.2 &    6.9 &   71.4 $\pm$    2.1 &   0.80 $\pm$   0.02 &     35 &  24.19 & 9.15e-01\\
 &      - & - & - &    0.0 & - &   63.4 $\pm$    1.4 &   0.69 $\pm$   0.01 &     36 &  59.46 & 8.24e-03\\
UGC 3957 &   extr &     36 &   0.12 &   11.0 $\pm$    1.0 &   11.2 &  180.8 $\pm$    7.3 &   1.01 $\pm$   0.04 &     33 &   6.63 & 1.00e+00\\
 &      - & - & - &    0.0 & - &  151.9 $\pm$    5.1 &   0.68 $\pm$   0.02 &     34 &  84.60 & 3.37e-06\\
 &   flat & - & - &   12.9 $\pm$    1.0 &   12.5 &  175.1 $\pm$    7.1 &   0.98 $\pm$   0.04 &     33 &   6.95 & 1.00e+00\\
 &      - & - & - &    0.0 & - &  144.2 $\pm$    4.7 &   0.62 $\pm$   0.02 &     34 &  91.61 & 3.48e-07\\
UGC 12491 &   extr &     23 &   0.04 &    3.0 $\pm$    0.2 &   13.8 &  148.5 $\pm$   11.7 &   1.12 $\pm$   0.04 &     20 & 445.44 & 7.29e-82\\
 &      - & - & - &    0.0 & - &   77.4 $\pm$    3.4 &   0.70 $\pm$   0.02 &     21 & 2353.02 & 0.00e+00\\
 &   flat & - & - &    3.0 $\pm$    0.2 &   13.8 &  148.5 $\pm$   11.7 &   1.12 $\pm$   0.04 &     20 & 445.44 & 7.29e-82\\
 &      - & - & - &    0.0 & - &   77.4 $\pm$    3.4 &   0.70 $\pm$   0.02 &     21 & 2353.02 & 0.00e+00\\
ZWCL 1215 &   extr &     36 &   0.25 &  163.2 $\pm$   35.6 &    4.6 &  131.3 $\pm$   43.6 &   1.00 $\pm$   0.32 &     33 &   2.94 & 1.00e+00\\
 &      - & - & - &    0.0 & - &  314.8 $\pm$   10.9 &   0.37 $\pm$   0.05 &     34 &   7.69 & 1.00e+00\\
 &   flat & - & - &  163.2 $\pm$   35.6 &    4.6 &  131.3 $\pm$   43.6 &   1.00 $\pm$   0.32 &     33 &   2.94 & 1.00e+00\\
 &      - & - & - &    0.0 & - &  314.8 $\pm$   10.9 &   0.37 $\pm$   0.05 &     34 &   7.69 & 1.00e+00\\
ZWCL 1358+6245 &   extr &     26 &   0.60 &   13.8 $\pm$    3.3 &    4.2 &  102.3 $\pm$    9.5 &   1.40 $\pm$   0.08 &     23 &   5.58 & 1.00e+00\\
 &      - & - & - &    0.0 & - &  130.6 $\pm$    6.1 &   1.15 $\pm$   0.05 &     24 &  19.02 & 7.51e-01\\
 &   flat & - & - &   20.7 $\pm$    3.2 &    6.4 &   98.0 $\pm$    9.4 &   1.43 $\pm$   0.09 &     23 &   5.65 & 1.00e+00\\
 &      - & - & - &    0.0 & - &  138.5 $\pm$    6.1 &   1.04 $\pm$   0.05 &     24 &  32.17 & 1.23e-01\\
ZWCL 1742 &   extr &     17 &   0.12 &   13.8 $\pm$    1.5 &    9.0 &  147.7 $\pm$    9.4 &   1.39 $\pm$   0.11 &     14 &  14.80 & 3.92e-01\\
 &      - & - & - &    0.0 & - &  122.0 $\pm$    6.1 &   0.78 $\pm$   0.04 &     15 &  55.08 & 1.73e-06\\
 &   flat & - & - &   23.8 $\pm$    1.7 &   14.4 &  126.5 $\pm$    9.0 &   1.30 $\pm$   0.12 &     14 &  24.08 & 4.49e-02\\
 &      - & - & - &    0.0 & - &  100.7 $\pm$    4.5 &   0.48 $\pm$   0.03 &     15 &  69.54 & 5.39e-09\\
ZWCL 1953 &   extr &     17 &   0.45 &  194.5 $\pm$   56.6 &    3.4 &   62.1 $\pm$   57.0 &   1.39 $\pm$   0.65 &     14 &   0.99 & 1.00e+00\\
 &      - & - & - &    0.0 & - &  283.3 $\pm$   27.3 &   0.45 $\pm$   0.11 &     15 &   4.39 & 9.96e-01\\
 &   flat & - & - &  194.5 $\pm$   56.6 &    3.4 &   62.1 $\pm$   57.0 &   1.39 $\pm$   0.65 &     14 &   0.99 & 1.00e+00\\
 &      - & - & - &    0.0 & - &  283.3 $\pm$   27.3 &   0.45 $\pm$   0.11 &     15 &   4.39 & 9.96e-01\\
ZWCL 3146 &   extr &     15 &   0.30 &   11.4 $\pm$    2.0 &    5.7 &  105.5 $\pm$    6.4 &   1.29 $\pm$   0.08 &     12 &   5.24 & 9.49e-01\\
 &      - & - & - &    0.0 & - &  126.3 $\pm$    4.5 &   0.98 $\pm$   0.03 &     13 &  31.82 & 2.55e-03\\
 &   flat & - & - &   11.4 $\pm$    2.0 &    5.7 &  105.5 $\pm$    6.4 &   1.29 $\pm$   0.08 &     12 &   5.24 & 9.49e-01\\
 &      - & - & - &    0.0 & - &  126.3 $\pm$    4.5 &   0.98 $\pm$   0.03 &     13 &  31.82 & 2.55e-03\\
ZWCL 7160 &   extr &     21 &   0.40 &   18.8 $\pm$    3.2 &    5.9 &   89.3 $\pm$    7.3 &   1.34 $\pm$   0.10 &     18 &   2.43 & 1.00e+00\\
 &      - & - & - &    0.0 & - &  117.0 $\pm$    4.8 &   0.93 $\pm$   0.05 &     19 &  29.31 & 6.13e-02\\
 &   flat & - & - &   21.1 $\pm$    3.1 &    6.8 &   86.3 $\pm$    7.2 &   1.37 $\pm$   0.10 &     18 &   2.82 & 1.00e+00\\
 &      - & - & - &    0.0 & - &  116.9 $\pm$    4.8 &   0.90 $\pm$   0.05 &     19 &  36.37 & 9.49e-03\\
Zwicky 2701 &   extr &     24 &   0.40 &   34.0 $\pm$    4.2 &    8.2 &  135.1 $\pm$   10.3 &   1.37 $\pm$   0.10 &     21 &   4.79 & 1.00e+00\\
 &      - & - & - &    0.0 & - &  187.1 $\pm$    6.6 &   0.87 $\pm$   0.04 &     22 &  43.01 & 4.71e-03\\
 &   flat & - & - &   39.7 $\pm$    3.9 &   10.1 &  126.0 $\pm$   10.2 &   1.45 $\pm$   0.10 &     21 &   5.67 & 1.00e+00\\
 &      - & - & - &    0.0 & - &  186.4 $\pm$    6.7 &   0.82 $\pm$   0.04 &     22 &  60.27 & 2.04e-05\\
ZwCl 0857.9+2107 &   extr &     16 &   0.30 &   23.6 $\pm$    5.0 &    4.8 &   89.6 $\pm$   10.4 &   1.40 $\pm$   0.17 &     13 &   0.92 & 1.00e+00\\
 &      - & - & - &    0.0 & - &  116.8 $\pm$    7.3 &   0.86 $\pm$   0.07 &     14 &  14.36 & 4.24e-01\\
 &   flat & - & - &   24.2 $\pm$    5.0 &    4.9 &   89.3 $\pm$   10.4 &   1.40 $\pm$   0.18 &     13 &   0.88 & 1.00e+00\\
 &      - & - & - &    0.0 & - &  116.9 $\pm$    7.4 &   0.85 $\pm$   0.07 &     14 &  14.76 & 3.95e-01
\enddata
\tablecomments{Col. (1) Cluster name; col. (2) CDA observation identification number; col. (3) method of $T_X$ interpolation (discussed in \S\ref{sec:kpr}); col. (4) maximum radius for fit; col. (5) number of radial bins included in fit; col. (6) best-fit core entropy; col. (7) number of sigma \kna\ is away from zero; col. (9) best-fit entropy at 100 kpc; col. (10) best-fit power-law index; col. (11) degrees of freedom in fit; col. (12) \chisq\ statistic of best-fit model; and col. (13) probability of worse fit given \chisq\ and degrees of freedom.}
\end{deluxetable}

\clearpage
\begin{figure}
  \begin{center}
    \begin{minipage}{\linewidth}
      \includegraphics*[width=\textwidth, trim=0mm 0mm 0mm 0mm, clip]{rbs797.ps}
    \end{minipage}
    \caption{Fluxed, unsmoothed 0.7--2.0 keV clean image of \rbs\ in
      units of ph \pcmsq\ \ps\ pix$^{-1}$. Image is $\approx 250$ kpc
      on a side and coordinates are J2000 epoch. Black contours in the
      nucleus are 2.5--9.0 keV X-ray emission of the nuclear point
      source; the outer contour approximately traces the 90\% enclosed
      energy fraction (EEF) of the \cxo\ point spread function. The
      dashed green ellipse is centered on the nuclear point source,
      encloses both cavities, and was drawn by-eye to pass through the
      X-ray ridge/rims.}
    \label{fig:img}
  \end{center}
\end{figure}

\begin{figure}
  \begin{center}
    \begin{minipage}{0.495\linewidth}
      \includegraphics*[width=\textwidth, trim=0mm 0mm 0mm 0mm, clip]{325.ps}
    \end{minipage}
   \begin{minipage}{0.495\linewidth}
      \includegraphics*[width=\textwidth, trim=0mm 0mm 0mm 0mm, clip]{8.4.ps}
   \end{minipage}
   \begin{minipage}{0.495\linewidth}
      \includegraphics*[width=\textwidth, trim=0mm 0mm 0mm 0mm, clip]{1.4.ps}
    \end{minipage}
    \begin{minipage}{0.495\linewidth}
      \includegraphics*[width=\textwidth, trim=0mm 0mm 0mm 0mm, clip]{4.8.ps}
    \end{minipage}
     \caption{Radio images of \rbs\ overlaid with black contours
       tracing ICM X-ray emission. Images are in mJy beam$^{-1}$ with
       intensity beginning at $3\sigma_{\rm{rms}}$ and ending at the
       peak flux, and are arranged by decreasing size of the
       significant, projected radio structure. X-ray contours are from
       $2.3 \times 10^{-6}$ to $1.3 \times 10^{-7}$ ph
       \pcmsq\ \ps\ pix$^{-1}$ in 12 square-root steps. {\it{Clockwise
           from top left}}: 325 MHz \vla\ A-array, 8.4 GHz
       \vla\ D-array, 4.8 GHz \vla\ A-array, and 1.4 GHz
       \vla\ A-array.}
    \label{fig:composite}
  \end{center}
\end{figure}

\begin{figure}
  \begin{center}
    \begin{minipage}{0.495\linewidth}
      \includegraphics*[width=\textwidth, trim=0mm 0mm 0mm 0mm, clip]{sub_inner.ps}
    \end{minipage}
    \begin{minipage}{0.495\linewidth}
      \includegraphics*[width=\textwidth, trim=0mm 0mm 0mm 0mm, clip]{sub_outer.ps}
    \end{minipage}
    \caption{Red text point-out regions of interest discussed in
      Section \ref{sec:cavities}. {\it{Left:}} Residual 0.3-10.0 keV
      X-ray image smoothed with $1\arcs$ Gaussian. Yellow contours are
      1.4 GHz emission (\vla\ A-array), orange contours are 4.8 GHz
      emission (\vla\ A-array), orange vector is 4.8 GHz jet axis, and
      red ellipses outline definite cavities. {\it{Bottom:}} Residual
      0.3-10.0 keV X-ray image smoothed with $3\arcs$ Gaussian. Green
      contours are 325 MHz emission (\vla\ A-array), blue contours are
      8.4 GHz emission (\vla\ D-array), and orange vector is 4.8 GHz
      jet axis.}
    \label{fig:subxray}
  \end{center}
\end{figure}

\begin{figure}
  \begin{center}
    \begin{minipage}{\linewidth}
      \includegraphics*[width=\textwidth]{r797_nhfro.eps}
      \caption{Gallery of radial ICM profiles. Vertical black dashed
        lines mark the approximate end-points of both
        cavities. Horizontal dashed line on cooling time profile marks
        age of the Universe at redshift of \rbs. For X-ray luminosity
        profile, dashed line marks \lcool, and dashed-dotted line
        marks \pcav.}
      \label{fig:gallery}
    \end{minipage}
  \end{center}
\end{figure}

\begin{figure}
  \begin{center}
    \begin{minipage}{\linewidth}
      \setlength\fboxsep{0pt}
      \setlength\fboxrule{0.5pt}
      \fbox{\includegraphics*[width=\textwidth]{cav_config.eps}}
    \end{minipage}
    \caption{Cartoon of possible cavity configurations. Arrows denote
      direction of AGN outflow, ellipses outline cavities, \rlos\ is
      line-of-sight cavity depth, and $z$ is the height of a cavity's
      center above the plane of the sky. {\it{Left:}} Cavities which
      are symmetric about the plane of the sky, have $z=0$, and are
      inflating perpendicular to the line-of-sight. {\it{Right:}}
      Cavities which are larger than left panel, have non-zero $z$,
      and are inflating along an axis close to our line-of-sight.}
    \label{fig:config}
  \end{center}
\end{figure}

\begin{figure}
  \begin{center}
    \begin{minipage}{0.495\linewidth}
      \includegraphics*[width=\textwidth, trim=25mm 0mm 40mm 10mm, clip]{edec.eps}
    \end{minipage}
    \begin{minipage}{0.495\linewidth}
      \includegraphics*[width=\textwidth, trim=25mm 0mm 40mm 10mm, clip]{wdec.eps}
    \end{minipage}
    \caption{Surface brightness decrement as a function of height
      above the plane of the sky for a variety of cavity radii. Each
      curve is labeled with the corresponding \rlos. The curves
      furthest to the left are for the minimum \rlos\ needed to
      reproduce $y_{\rm{min}}$, \ie\ the case of $z = 0$, and the
      horizontal dashed line denotes the minimum decrement for each
      cavity. {\it{Left}} Cavity E1; {\it{Right}} Cavity W1.}
    \label{fig:decs}
  \end{center}
\end{figure}


\begin{figure}
  \begin{center}
    \begin{minipage}{\linewidth}
      \includegraphics*[width=\textwidth, trim=15mm 5mm 5mm 10mm, clip]{pannorm.eps}
      \caption{Histograms of normalized surface brightness variation
        in wedges of a $2.5\arcs$ wide annulus centered on the X-ray
        peak and passing through the cavity midpoints. {\it{Left:}}
        $36\mydeg$ wedges; {\it{Middle:}} $14.4\mydeg$ wedges;
        {\it{Right:}} $7.2\mydeg$ wedges. The depth of the cavities
        and prominence of the rims can be clearly seen in this plot.}
      \label{fig:pannorm}
    \end{minipage}
  \end{center}
\end{figure}

\begin{figure}
  \begin{center}
    \begin{minipage}{0.5\linewidth}
      \includegraphics*[width=\textwidth, angle=-90]{nucspec.ps}
    \end{minipage}
    \caption{X-ray spectrum of nuclear point source. Black denotes
      year 2000 \cxo\ data (points) and best-fit model (line), and red
      denotes year 2007 \cxo\ data (points) and best-fit model (line).
      The residuals of the fit for both datasets are given below.}
    \label{fig:nucspec}
  \end{center}
\end{figure}

\begin{figure}
  \begin{center}
    \begin{minipage}{\linewidth}
      \includegraphics*[width=\textwidth, trim=10mm 5mm 10mm 10mm, clip]{radiofit.eps}
    \end{minipage}
    \caption{Best-fit continuous injection (CI) synchrotron model to
      the nuclear 1.4 GHz, 4.8 GHz, and 7.0 keV X-ray emission. The
      two triangles are \galex\ UV fluxes showing the emission is
      boosted above the power-law attributable to the nucleus.}
    \label{fig:sync}
    \end{center}
\end{figure}

\begin{figure}
  \begin{center}
    \begin{minipage}{\linewidth}
      \includegraphics*[width=\textwidth, trim=0mm 0mm 0mm 0mm, clip]{rbs797_opt.ps}
    \end{minipage}
    \caption{\hst\ \myi+\myv\ image of the \rbs\ BCG with units e$^-$
      s$^{-1}$. Green, dashed contour is the \cxo\ 90\% EEF. Emission
      features discussed in the text are labeled.}
    \label{fig:hst}
  \end{center}
\end{figure}

\begin{figure}
  \begin{center}
    \begin{minipage}{0.495\linewidth}
      \includegraphics*[width=\textwidth, trim=0mm 0mm 0mm 0mm, clip]{suboptcolor.ps}
    \end{minipage}
    \begin{minipage}{0.495\linewidth}
      \includegraphics*[width=\textwidth, trim=0mm 0mm 0mm 0mm, clip]{suboptrad.ps}
    \end{minipage}
    \caption{{\it{Left:}} Residual \hst\ \myv\ image. White regions
      (numbered 1--8) are areas with greatest color difference with
      \rbs\ halo. {\it{Right:}} Residual \hst\ \myi\ image. Green
      contours are 4.8 GHz radio emission down to
      $1\sigma_{\rm{rms}}$, white dashed circle has radius $2\arcs$,
      edge of ACS ghost is show in yellow, and southern whiskers are
      numbered 9--11 with corresponding white lines.}
    \label{fig:subopt}
  \end{center}
\end{figure}


%%%%%%%%%%%%%%%%%%%%
% End the document %
%%%%%%%%%%%%%%%%%%%%
\end{document}
