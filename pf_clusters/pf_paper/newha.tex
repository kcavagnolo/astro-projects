\begin{deluxetable}{lcccc}
\tabletypesize{\scriptsize}
\tablecaption{M. Donahue's \halpha\ Observations.\label{tab:newha}}
\tablewidth{0pt}
\tablehead{
  \colhead{Cluster} & \colhead{Telescope} & \colhead{$z$} & \colhead{$[NII]$/\halpha} & \colhead{\halpha\ Flux}\\
  \colhead{}      & \colhead{}        & \colhead{}  & \colhead{}               & \colhead{$10^{-15}$ \flux}
}
\startdata
Abell 85     & PO & 0.0558 & 2.67    &    0.581\\
Abell 119    & LC & 0.0442 & \nodata & $<$0.036\\
Abell 133    & LC & 0.0558 & \nodata &    0.88\\
Abell 496    & LC & 0.0328 & 2.50    &    2.90\\
Abell 1644   & LC & 0.0471 & \nodata &    1.00\\
Abell 1650   & LC & 0.0843 & \nodata & $<$0.029\\
Abell 1689   & LC & 0.1843 & \nodata & $<$0.029\\
Abell 1736   & LC & 0.0338 & \nodata & $<$0.026\\
Abell 2597   & PO & 0.0854 & 0.85    &    29.7\\
Abell 3112   & LC & 0.0720 & 2.22    &    2.66\\
Abell 3158   & LC & 0.0586 & \nodata & $<$0.036\\
Abell 3266   & LC & 0.0590 & 1.62    & $<$0.027\\
Abell 4059   & LC & 0.0475 & 3.60    &    2.22\\
Cygnus A     & PO & 0.0561 & 1.85    &    28.4\\
EXO 0422-086 & LC & 0.0397 & \nodata & $<$0.031\\
Hydra A      & LC & 0.0522 & 0.85    &    13.4\\
PKS 0745-191 & LC & 0.1028 & 1.02    &    10.4
\enddata
\tablecomments{The abbreviation ``PO'' denotes observations taken on
  the 5 m Hale Telescope at the Palomar Observatory, USA, while ``LC''
  are observations taken on the DuPont 2.5 m telescope at the Las
  Campanas Observatory, Chile. Upperlimits for \halpha\ fluxes are
  $3\sigma$.}
\end{deluxetable}
