\documentclass[11pt]{article}
\setlength{\topmargin}{-.3in}
\setlength{\oddsidemargin}{-0.1in}
\setlength{\evensidemargin}{-0.1in}
\setlength{\textwidth}{6.7in}
\setlength{\headheight}{0in}
\setlength{\headsep}{0in}
\setlength{\topskip}{0.5in}
\setlength{\textheight}{9.25in}
\setlength{\parindent}{0.0in}
\setlength{\parskip}{1em}
\usepackage{common,graphicx,hyperref,epsfig}

\pagestyle{empty}

\begin{document}

\today

{\bf{Article:}} ApJS 295532

{\bf{Title:}} Intracluster Medium Entropy Profiles for a Chandra Archival
Sample of Galaxy Clusters

{\bf{Authors:}} Kenneth W. Cavagnolo, Megan Donahue, G. Mark Voit, and Ming
Sun

Dear Editor,

Below is our reply to the Referee's Report for ApJS Article 295532.
The referee's comments are in quotes and {\it{italicized}}, while our
replies are in regular font. We have found the referee's comments to
be very helpful in making the focus and discussion of our paper more
concise and thorough.

---------------------------------------------------------------------

{\it{``1) Section 3.4, describes how the values of K0 were derived and
    shows in Figure 2 the ratio of the core entropy using the two
    methods. I count about 40/239 systems for which the ratio is
    significantly different from unity, which is a non-negligible
    fraction. While it's pointed out in the text that the different
    interpolation schemes do not affect the conclusion of non-sero
    core entropy, I am wondering how the different temperature
    interpolation schemes affect the bimodality in the K0
    distribution. Since the bimodality is the principal result of the
    paper, I think the effect of the temperature interpolation on this
    result should be investigated.''}}

Shown in Figure \ref{fig:k0hist} is the histogram of \kna\ values
derived for the temperature interpolation scheme which assumes the
temperature is not a constant in the centralmost bins. The bimodal
behavior is still present, but with a second peak which is broader
compared to the same peak shown in the paper. Our KMM test using these
\kna\ values does not yield results which significantly differ from
those presented in the paper. That the \kna\ bimodality is not
significantly affected by the different \Tx\ interpolation schemes is
an important point, and we have added paragraph six in \S 5.2 to
emphasize this point.

\begin{figure}[htp]
  \begin{center}
    \begin{minipage}[htp]{0.9\linewidth}
      \includegraphics*[width=\textwidth, trim=20mm 10mm 10mm 10mm, clip]{k0hist_itpl.eps}
      \caption{{\it{Top panel:}} Histogram of best-fit \kna\ derived
      using the alternate temperature interpolation scheme. Bin widths
      are 0.15 in log space. {\it{Bottom panel:}} Cumulative
      distribution of \kna\ values.}
      \label{fig:k0hist}
    \end{minipage}
  \end{center}
\end{figure}

---------------------------------------------------------------------

{\it{``2) Related to this, Section 4, on systematics, is very
    interesting. However, the entire Section is based on the
    systematics introduced by angular resolution effects on the
    density profiles, yet there are clusters in the sample for which
    there are only three temperature bins. I would like to see some
    discussion added on the effect of finite binning of temperature
    profiles. This is especially important because the temperature
    bins are (rightly) defined according to a total count criterion,
    meaning that clusters with centrally peaked emission will yield
    entropy profiles of generally better resolution than those with
    flat central emission. For instance, what happens when the
    temperature profile of a cluster with centrally peaked emission is
    binned to the typical resolution of a cluster with flat central
    emission?''}}

This is a good point, one we considered during our analysis (paragraph
four of \S 4.2) but did not discuss at length in the manuscript. We
have segregated out the discussion of the degraded temperature
profiles and added text to further explain our method. Inspired by this
referee comment, we have also added a few sentences to the end of \S
4.3 to point out that our use of counts per annulus in creating
temperature profiles has not resulted in a relationship between
\kna\ and exposure time or number of bins in $\Tx(r)$.

---------------------------------------------------------------------

{\it{``3) I'm not sure Section 4.2, on the subject of XMM profiles, is
    entirely correct. Taking the example of the samples of Piffaretti
    et al (2005) and Pratt et al. (2006), it is important to note
    significant differences in the respective analyses. My
    understanding of Piffaretti et al's analysis is that they derived
    their gas density profiles from spectral fits to relatively coarse
    annular bins defined for the temperature profile analysis.
    However, I understand that Pratt et al. derived their density
    profiles from deprojected, PSF deconvolved analytic model fits to
    surface brightness profiles, corrected for radial variations in
    emissivity due to temperature and abundance gradients, and their
    temperature profiles are also corrected for PSF effects (as
    described in Pointecouteau et al 2005). Furthermore, Croston et al
    (2006) show that their analytic surface brightness models recover
    the density distribution exceptionally well in comparison to
    Chandra results, with differences occurring only in the single
    central surface brightness bin. Thus the Pratt et al analysis is
    undertaken on a radial surface brightness binning scale that is
    equivalent to that of the present paper, while the Piffaretti et
    all analysis is not.  Combining a discussion of resolution effects
    in the context of redshift with a comparison with XMM profiles
    confuses the issue further. I thus suggest that Section 4.2 should
    deal only with resolution effects in the context of cluster
    redshift, and that a separate Section should be added comparing
    with XMM results. The latter Section will have to discuss the
    effect of the different analysis methods employed. My belief is
    that since the XMM profiles were never fitted to anything other
    than a power law, it is disingenuous to conclude (especially in
    the case of Pratt et al.) that these results did not find core
    flattening because of resolution effects. It is more likely
    because, unlike the present work, they did not think to
    investigate core flattening.''}}

We have pulled the discussion of angular resolution as it relates to
XMM from \S 4.2 and moved it to a subsection in \S 5.6 with the rest
of the discussion regarding XMM. We have expanded discussion of
Piffaretti 2005 and Pratt 2006 to highlight their analysis methods and
how they differ from our own. We have also pointed out, more heavily
in this revised manuscript than the original, that some of the XMM
profiles appear to have flattened cores, but nothing but a power law
was used in fitting.

---------------------------------------------------------------------

{\it{``4) What are the implications of the bimodality? The existence
    of a gap in the distribution of core entropies implies that either
    the gap is intrinsic to the cluster population, or that the
    entropy in the cluster cores is a constantly evolving property.
    Both of these possibilities open up intriguing areas for further
    discussion. If the gap is intrinsic, how is it that some clusters
    have a high core entropy and some clusters have a low core
    entropy? The idea suggested by McCarthy and collaborators, that
    different clusters experience a different level of preheating, may
    be one explanation. I'm sure the authors could think of others.
    If the gap is a function of evolving core entropy distributions,
    the fraction of clusters on one side or the other, and
    particularly the evolution of that fraction over time, might be
    indicative of a characteristic timescale. If that timescale is of
    the order of Gyr, then it would suggest that dynamical
    (gravitational) processes are modifying the core entropy, through
    mixing of high and low entropy gas during mergers. If the
    timescale is of order Myr, then it is possible that AGN-driven
    feedback is the dominant driver of the distribution of core
    entropies. Is it possible to investigate this with the current
    sample? If not, what kind of sample would be needed to do this
    kind of thing?  Is there anything special about the clusters in
    the gap?  I think the paper would benefit from a bit more
    discussion along these lines added to the text.''}}

Our discussion of the physical meaning of \kna\ bimodality is parsed
out to several sections (paragraph 7 of \S 5.2 and paragraph 7 of \S
6) of the manuscript and most specifically we reserved the bulk of the
discussion to Voit et al (ApJ 2008, 681 5L). In Voit et al we suggest
that the gap is a result of the effects of thermal electron conduction
slowing cooling with subsequent mergers possibly depopulating the
\kna\ = 30-50 \ent\ region. Most of the questions the referee has
offered are actually the focus of our ongoing projects, and hence we
have been cautious in this manuscript to discuss ongoing work.

---------------------------------------------------------------------

{\it{``5) This may be a little more work, but I think the paper would
    also benefit from a cross correlation of K0 with the presence of a
    compact central radio source. Since AGN are thought to be key to
    feedback in the ICM, such a cross-correlation seems to be the
    obvious thing to do with such a large and well-observed sample,
    especially in the context of an ApJS paper.''}}

This is a very keen observation, one we addressed in the ApJ Letter
Cavagnolo et al. 2008 (ApJ 683, 107). We found that \kna\ does
correlate with the presence of AGN activity (as evidenced by radio
emission) and possible star formation (as evidence by \halpha\
emission).

---------------------------------------------------------------------

{\it{* Sect 1, para 2 (page 2): ``...otherwise tight mass observable
    relations...'', I would say ``otherwise theoretically tight''.
    O'Hara et al. (2006) could be added to the list of references
    concerned with reduction of scatter about scaling relation.}}

We agree that it is true the expectation for tight scaling relations
comes from theory, hence we have altered the wording and added the
reference.

---------------------------------------------------------------------

{\it{* Sect 1, para 7 (page 4) Pratt et al. (2006) could be added to
    studies showing core entropy has larger dispersion than at large
    radii.}}

An oversight on our part, the reference has been added.

---------------------------------------------------------------------

{\it{* Sect 3, para 1 (page 7) The effect of the choice of center is
    another source of systematics. Clusters with a large-scale
    centroid far away from the surface brightness peak will have
    flatter central entropies, or central entropies that yield
    physically impossible values on deprojection. Also, within what
    radius was the centroid defined?}}

The centroid is defined via an iterative process which is detailed in
Cavagnolo et al. 2008 (ApJ 682, 821). We have added this reference for
interested readers and quote the relevant section from Cavagnolo et
al. 2008 (ApJ 682, 821) below for the referee:

-- snip --\\
Defining the cluster ``center'' is essential for the later purpose of
excluding cool cores from our spectral analysis (see
\S\ref{sec:extraction}). To determine the cluster center, we
calculated the centroid of the flare cleaned, point-source free
level-2 events file filtered to include only photons in the $0.7-7.0$
keV range. Before centroiding, the events file was exposure-corrected
and ``holes'' created by excluding point sources were filled using
interpolated values taken from a narrow annular region just outside
the hole (holes are not filled during spectral extraction discussed in
\S\ref{sec:extraction}). Prior to centroiding, we defined the emission
peak by heavily binning the image, finding the peak value within a
circular region extending from the peak to the chip edge (defined by
the radius $R_{max}$), reducing $R_{max}$ by 5\%, reducing the binning
by a factor of two, and finding the peak again. This process was
repeated until the image was unbinned (binning factor of one). We then
returned to an unbinned image with an aperture centered on the
emission peak with a radius $R_{max}$ and found the centroid using
{\textsc{CIAO}}'s {\textsc{dmstat}}. The centroid, ($x_c$,$y_c$), for
a distribution of $N$ good pixels with coordinates ($x_i$,$y_j$) and
values f($x_i$,$y_j$) is defined as:
\begin{eqnarray}
Q &=& \sum_{i,j=1}^N f(x_i,y_i) \\
x_c &=& \frac{\sum_{i,j=1}^N x_i \cdot f(x_i,y_i)}{Q} \nonumber\\
y_c &=& \frac{\sum_{i,j=1}^N y_i \cdot f(x_i,y_i)}{Q}. \nonumber
\end{eqnarray}

If the centroid was within 70 kpc of the emission peak, the emission
peak was selected as the center, otherwise the centroid was used
as the center. This selection was made to ensure all ``peaky'' cool
cores coincided with the cluster center, thus maximizing their
exclusion later in our analysis. All cluster centers were additionally
verified by eye.
-- snip --

---------------------------------------------------------------------

{\it{* Sect 3.1, para 8 (page 9) Regarding deprojection, steeply
    declining T gradients would have lower central entropy, but
    increasing T gradients would have higher central entropy.}}

The wording in the paper was ambiguous, as we meant to convey exactly
what the referee has stated above. We have altered the wording to be
clearer.

---------------------------------------------------------------------

{\it{* Sect 3.2, para 3 (page 10) The T profile interpolation scheme
    is not particularly well described, particularly outside the core
    regions. Was there a model used? How well did this perform for the
    case of T profiles with 3 radial bins? Also, were the interpolated
    abundances also used to correct for emissivity variations or was
    it just the T variations? Abundance variations have more of an
    effect on emissivity than T variations.}}

We used IDL's native linear interpolation routine {\it{interpol}} to
place the temperature profile and density profile on the same radial
grid. We have added this note to the first paragraph of \S 3.4 to make
it clear that our method was very simplistic.

We did not use a model for the temperature interpolation as one of our
goals for this project was to avoid the use of models to minimize the
number of assumptions embedded in the final entropy profiles.

The deprojection technique we use, Kriss et al. 1983, produces an
emission profile (count rate per unit volume) which is used to convert
from surface brightness to gas density. The spectral count rate and
normalization for each annulus of the temperature profile are part of
the conversion from surface brightness to density. Since abundance is
intrinsically linked with the count rate and normalization, our
deprojection method does take into account emissivity variations. We
have added text above Eqn. 2 to point this out.

---------------------------------------------------------------------

{\it{* Sect 3.3, para 1 (page 11) Quantitatively, how suitable an
    approximation were the beta models, i.e., in terms of chi
    squared?}}

The reduced \chisq\ values and degrees of freedom for each
$\beta$-model fit are listed in Table 2. Given the number of degrees
of freedom for each fit, the \chisq\ values are not unacceptable.

---------------------------------------------------------------------

{\it{* Sect 5.2, para 5 (page 23) The fact that the KMM test indicates
    that two statistically different populations are not present at
    $z>0.4$ warrants more discussion. Is this due to resolution effects,
    or is it an evolutionary effect? If the latter, what does this
    tell us?}}

Unfortunately, the lack of bimodality for $z>0.4$ clusters is most
likely a result of poor statistics: there are 20 clusters with $z >
0.4$. We checked this hypothesis by randomly selecting 20 clusters
from our full archival sample 1000 times and running the KMM test
again with each subset of clusters. Of the 1000 subsets, two
statistically distinct populations were found 18 times ($~2\%$). This
suggests 20 clusters is too few to detect a bimodal population. We
have added this discussion to the end of paragraph five of \S 5.2.

---------------------------------------------------------------------

{\it{* Sect 5.5, para 1 (page 26) I would specify that the Tozzi \&
    Norman results were for semi-analytic models and the Voit et al
    results were for cosmological simulations with only gravitational
    effects included.}}

We have amended the section to point out this difference.

---------------------------------------------------------------------

{\it{* Sect 5.6, (page 27) This section should be revised as per point
    3 above.}}

See comments above and the manuscript for revisions.

---------------------------------------------------------------------

{\it{- Fig4: The legend ($T_x < 4.0; 8.0 < T_x$) is a bit confusing.}}

For clarity we have changed \Tx\ to $T_{cluster}$, added units, and
rearranged the legend to read $T_{cluster} > 8.0$ keV.

---------------------------------------------------------------------

{\it{- Fig 4: I suggest addition of two more panels showing the
    subsamples with $z<0.4$ and $z>0.4$}}

As addressed in the section above, the lack of bimodality at $z > 0.4$
cannot be attributed to a physical cause as the statistics are
poor. Hence examining the two populations, $z<0.4$ and $z>0.4$, loses
its appeal and we have decided not to add the two panels to Fig. 4.

---------------------------------------------------------------------

\end{document}
