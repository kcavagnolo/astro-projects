Things to fix/do:

-- sbr degradation
	+ effect on K0 as a function of z
	+ is the number of low-K0 decreasing w/ redshift faster than
	  resolution limit?
-- what do I mean by K0?
	+ not gas ent around smbh
	+ not lowest state of gas ent
	+ define as the "average" central gas entropy at ~ 10 kpc
	  scale, beyond any possible central core?
	+ define as beyond 5x(?) resolution element (5 or 20 kpc
	  depends on redshift)?
	+ simply call it ``core entropy''
-- which clusters have a central source excluded? flag these
	+ could be coronae
	+ could be AGN
	+ easy to hide high density, small things with an AGN or not
	+ AGN don't heat all the way to the core, so it's okay to
	ignore these areas
-- say it this way, we're investigating the fuel depot of the feedback
   mechanisms, and how that feedback energy interacts with the depot
-- discuss temperature profs at large radii
	+ we don't pay much attention there, the core we care about
	+ background changing anyway
-- add discussion of reflex, hiflugcs, B55 subsamples
	+ are trends in full sample present in subsamples?
	+ what's our completion level?
-- don't call it universal ent pedestal

%%%%%%%%%%
% Header %
%%%%%%%%%%

\documentclass[12pt, preprint]{aastex}
\usepackage{common}
\newcommand{\accept}{\textit{ACCEPT}}
\begin{document}
\title{Athenaeum of Chandra Cluster Entropy Profile Tables -- I.\\
A Bimodal Core Entropy Distribution}
\author{Kenneth W. Cavagnolo\altaffilmark{1,2},
	Megan Donahue\altaffilmark{1},
	G. Mark Voit\altaffilmark{1}, and
	Ming Sun\altaffilmark{1}}
\altaffiltext{1}{Michigan State University, Department of Physics and
Astronomy, BPS Building, East Lansing, MI 48824}
\altaffiltext{2}{cavagnolo@pa.msu.edu}
\shorttitle{ACCEPT I.}
\shortauthors{K. W. Cavagnolo et al.}
\bibliographystyle{apj}

%%%%%%%%%%%%
% Abstract %
%%%%%%%%%%%%

\begin{abstract}
We present radial entropy distributions of the intracluster
medium (ICM) for 208 galaxy clusters collected from the \Chandra\ data
archive. Radiative cooling of the ICM sets an entropy scale which is
responsible for breaking the self-similarity which would otherwise be
set by adiabatic hierarchical galaxy cluster formation. The entropy of
the ICM can thus be used as a tool for studying the thermal history of
a cluster, and as importantly, the feedback mechanisms, such as active
galactic nuclei (AGN) or star formation, which naturally occur from
the prodigious cooling of the ICM. We show that the entropy
``pedestal'' which has been observed in classic cooling flow clusters
is indeed a universal feature of clusters regardless of dynamical
state. We also show that the distribution of central entropy, \kna,
for our collection is consistent with models of ICM heating via AGN
feedback, and that the distribution is bimodal. We suggest the
observed distribution of \kna\ is a reflection of AGN feedback,
thermal electron conduction, and mergers. This paper focuses on
the data reduction, methodology, and initial results of \accept.
We have named this project \accept\ for Athenaeum of Chandra Cluster
Entropy Profiles Tables. The data and results for \accept\ can be
found online at http://www.pa.msu.edu/astro/MC2/accept/.
\end{abstract}

%%%%%%%%%%%%
% Keywords %
%%%%%%%%%%%%

\keywords{I'm -- a -- keyword!}

%%%%%%%%%%%%%%%%%%%%%%
\section{Introduction}
\label{sec:intro}
%%%%%%%%%%%%%%%%%%%%%%

The general process of galaxy cluster formation through hierarchical
merging is well understood, but many details, such as the impact of
feedback sources on the cluster environment and radiative cooling in
the cluster core, are not. The nature of feedback operating within
clusters is of great interest because of the implications for using
mass-observable scaling relations in cluster cosmological studies, and
for better understanding galaxy formation. The adiabatic model of
hierarchical structure formation predicts clusters of galaxies which
are scaled versions of each other. This model also predicts the most
massive galaxies in the Universe should be rife with young stellar
populations. Observations have long shown \citep{1999ApJ...520...78H,
2000ApJ...536...73N, 2001A&A...368..749F} however that clusters adhere
to steeper scaling relations with larger intrinsic dispersion than
theory predicts \citep{1996ApJ...469..494E, 1997MNRAS.292..289E}. In
addition, the most massive galaxies are composed of old stars (the
``old, red, and dead'' quandry) and have much less mass than predicted.
Moreover these massive galaxies formed by $\red \sim 1-2$
\citep{1996MNRAS.283.1388M, 1996Natur.384..439S} -- the so called
down-sizing problem \citep{1996AJ....112..839C}.

Because the cooling time in the cores of many clusters is much shorter
than the Hubble time \citep{1994ARA&A..32..277F, 1998MNRAS.298..416P},
the core region of clusters have been subject to cooling at some point
in their past and most likely heating from feedback mechanisms which
may initiate as a result of cooling. This secondary heating and
cooling effectively decouples baryons from the dark matter and results
in the breaking of self-similarity. As a consequence of radiative
cooling, global cluster temperature decreases while global cluster
luminosity increases. Thus, at a given mass scale, radiative cooling
conspires to create dispersion in otherwise tight correlations between
mass-luminosity and mass-temperature.

Additional effects of radiative cooling, such as massive cooling
flows ($100-1000 \Msol \pyr$), have been suggested but never
observed \citep{tamura2001, peterson2001}. Methodical searches for the
end products of this gas cooling have never yielded the enormous mass
sinks expected to reside in the cores of clusters \citep{heckman89,
mcnamara90, odea94, voit95} . On the contrary, the torrents of cool
gas flowing to a cluster center are more like cooling trickles
\citep{peterson2003}, depositing at most a few solar masses per year
in the core and never reaching temperatures below approximately $1/3
T_{virial}$. Obviously some source(s) of energetic feedback has heated
the ICM to selectively remove gas with a short cooling time.

Thanks to the high-resolution optics aboard \Chandra, previously
unobserved bubbles and cavities in the ICM have been found in numerous
clusters with short central cooling times. A consensus has been
reached that these bubbles are blown by active galactic nuclei (AGN)
and the outflow energies are sufficient to heat the ICM such that
energy loss from cooling is balanced. This has made AGN the
cornerstone for feedback models of ICM heating. Taken as a whole, the
non-gravitational processes operating in clusters are proving to be
very important for understanding how clusters form and evolve in
addition to being important for understanding the formation of
galaxies. These feedback processes are also interesting because
the mechanisms operating present-day in the most massive galaxies are
likely also the mechanisms which were important in galaxy formation
early in the Universe.

In the work presented in this paper, and Paper II, we have undertaken
a 'close to the data' study of the ICM to better understand the
physical state of gas below the cooling threshold. By studying the
sub-threshold gas, we aim to gain insight to how ICM properties
correlate with cluster feedback state, what can be learned of how
feedback mechanisms operate, and how this feedback alters global
cluster properties. To this end, we have paid particular attention to
ICM entropy.

Taken individually, ICM temperature and density do not fully
illuminate the cluster thermal history because they are most
influenced by the underlying dark matter potential. Gas temperature
reflects the depth of the potential well, while density reflects the
capacity of the well to compress the gas. However, recall that at
constant pressure the density of a gas is determined by its specific
entropy. So using the expression for the adiabatic constant
$K=P\rho^{-5/3}$, rewriting pressure and density in terms of
temperature and electron density, one can define a new quantity, $K =
Tn_e^{-2/3}$, where $T$ is temperature and $n_e)$ is electron gas
density, one captures the complete thermal history of the gas because
only heating and cooling can change $K$. This quantity is typically
referred to as entropy, but in actuality $K$ is only a pseudo-entropy
as relation to true thermodynamic entropy is $s = \ln K^{3/2} +
\mathrm{constant}$.

Entropy imparts the wonderfully useful property of making a gas
convectively stable only when $dK/dr \geq 0$. Thus, gravitational
potential wells are giant entropy sorting devices: low entropy gas
sinks to the bottom of the potential well, while high entropy gas
buoyantly rises to a radius of equal entropy. If a cluster were a
sealed box of gravitation-only processes, then the radial entropy
distribution of the ICM would strictly follow a power-law relation
across all radii. Thus, any departures of the radial entropy
distribution from a power-law is indicative of past heating and
cooling, \ie\ from AGN.

If feedback activity in the core of a cluster is unrelated to the
gas entropy, then the expectation is that central entropy and
robust indicators of feedback, \eg\ radio-loud AGN and star formation,
will be uncorrelated. But if ICM entropy and feedback are connected,
then departures of $K(r)$ from a power-law can be used to better
understand feedback timescales, deposition of AGN outflow energy,
and ultimately the effects of feedback on cluster and massive galaxy
evolution (\eg\ truncating star formation).

In our previous work (\citealt{2006ApJ...643..730D}, hereafter DHC06),
we analyzed a sample of nine nearby ($\red < 0.1$) classic cool core
clusters selected for their potential to produce high-quality radial
entropy profiles. In this paper, we present the results of an
exhaustive study of radial entropy profiles for an archive-limited
collection of observations taken from the \Chandra\ Data Archive. We have
conveniently named this project \accept\ which is short for Athenaeum
of \Chandra\ Cluster Entropy Profile Tables. For this project we cover
a much broader range of \Lx, \Tx, and morphologies than was used in
DHC06. The advantage being a uniformly analyzed collection of entropy
profiles, covering a broad range of central entropys, which can then be
utilized to study carefully selected sub-samples of clusters.

\accept\ as a whole is not an unbiased, complete sample of clusters as
it is drawn from the \Chandra\ archive and is thus subject to selection
biases. But \accept\ does provide us with a powerful baseline to study
any entropy-feedback connections which may exist. From our analysis we
have found that the entropy pedestal observed in DHC06, and by many
others previously (\eg\ \citealt{1999Natur.397..135P},
\citealt{2000MNRAS.315..689L}, \citealt{2005A&A...433..101P}), is not
isolated to just cooling flow clusters, but is a feature of all
clusters. We also have found two indicators of feedback -- radio-loud
AGN found using the NRAO VLA Sky Survey (NVSS) and star formation as
evidenced by \halpha\ emission from the central dominant (cD) galaxy
-- are strongly anti-correlated with central entropy. For the first
time we also note two interesting results:
\begin{enumerate}
\item The central entropy distribution of the full \accept\ collection
shows strong bimodality, while the distribution for the HIFLUGCS
sub-sample shows robust multimodality. 
\item There appears to exist a characteristic entropy threshold which
coincides with the electron thermal conduction stability limit below
which feedback proceeds and above which feedback is abated. The
clusters below the limit are what would be called ``active'' clusters
with star formation, AGN, and infrared excesses.
\end{enumerate}
We defer detailed scientific discussion of the above two points,
feedback timescales, and entropy scaling relations for Papers II and
III of the series.

As part of this work, we have plans to make all data publicly available
through two access points: 1) the NASA High Energy Space Archive
(HEASARC) under the Chandra section of
W3Browse\footnote{http://heasarc.gsfc.nasa.gov/W3Browse/chandra}, and
2) our own searchable, interactive
database\footnote{http://www.pa.msu.edu/astro/MC2/accept}. All data
tables, plots, spectra, reduced \Chandra\ data products, scripts, and
so on, will be available for any and all interested parties. We
particularly encourage theorists to take advantage of the entropy
profile library, a resource which to our knowledge is not available
anywhere else as of right now.

This paper focuses on the analysis methods and statistical trends of
\accept. The structure of this paper is as follows:\\
In \S\ref{sec:sample} we outline sample-selection criteria and
\Chandra\ observations selected under these criteria. Data reduction
is discussed in \S\ref{sec:data}. Spectral extraction and analysis are
discussed in \S\ref{sec:specan}. Our method for deriving deprojected
electron density profiles is outlined in \S\ref{sec:dene}. Results and
discussion of our final analysis are presented in \S\ref{sec:r&d}. A
summary this project is presented in \S\ref{sec:summary}. For this
work we have assumed a flat \LCDM\ Universe with cosmogony $\OM=0.3$,
$\OL=0.7$, and $\Hn=70\km\ps\pMpc$. All quoted uncertainties are at
the $1.6\sigma$ level (90\% confidence).

%%%%%%%%%%%%%%%%%%%%%%%%%
\section{Data Collection}
\label{sec:sample}
%%%%%%%%%%%%%%%%%%%%%%%%%

Our sample was initially collected from observations publicly
available in the \Chandra\ X-ray Telescope's Data Archive (CDA) as of
June 2007. We first assembled a list of possible targets from multiple
flux-limited surveys: \Rosat\ Brightest Cluster Sample
\citep{1998MNRAS.301..881E}, RBCS Extended Sample
\citep{2000MNRAS.318..333E}, \Rosat\ Brightest 55 Sample
\citep{1990MNRAS.245..559E, 1998MNRAS.298..416P}, \Einstein\ Extended
Medium Sensitivity Survey \citep{1990ApJS...72..567G}, North Ecliptic
Pole Survey \citep{2006ApJS..162..304H}, \Rosat\ Deep Cluster Survey
\citep{1995ApJ...445L..11R}, \Rosat\ Serendipitous Survey
\citep{1998ApJ...502..558V}, Massive Cluster Survey
\citep{2001ApJ...553..668E}, and REFLEX Survey
\citep{2004A&A...425..367B}. After the first pass of data collection
concluded, we continued expanding our collection by adding new archival
data listed under the CDA Science Category as ``clusters of galaxies''
or ``active galaxies''. As of submitting this paper, the result of our
CDA scouring is a total of 454 observations amounting to a nominal
exposure time of 15.05 Msec. 

- Temperature, luminosity, and redshift ranges?

Calculation of radial entropy necessitates measurement of the radial
temperature structure (discussed further in \S\ref{sec:data}). To
infer a temperature which is reasonably well constrained ($\pm 1.0
\keV$) we require a minimum of 2500 counts per radial bin. We also
require a minimum of three radial bins for each cluster. We find the
last criterion necessary to avoid isothermal temperature profiles
(within the uncertainties) which are a result of under-resolving
temperature structure; the consequence of which is artificially
flattened entropy profiles. After applying these constraints to all
the observations in our collection we are left with 262 observations
for 198 clusters totaling 8.40 Msec. Table \ref{tab:sample} lists the
general properties for each cluster in \accept.

% bad kprofs:
% 36 clusters
% 47 obs
% 1.3 Msec
% :(

%%%%%%%%%%%%%%%%%%%%%%%
\section{Data Analysis}
\label{sec:data}
%%%%%%%%%%%%%%%%%%%%%%%

The purpose of undertaking the archival entropy study which resulted in
\accept\ was to create a library of uniformly analyzed galaxy cluster
entropy profiles covering a broad range of masses, luminosities,
temperatures, morphological states, and central cooling times. As
presented in DHC06 and \cite{2002ApJ...576..601V}, we have defined
used the adiabatic constant, $K=P\rho^{-5/3}$, to define entropy in
terms of the gas temperature, $T(r)$, and electron density,
$\nelec(r)$. Using these observables in place of the pressure and
density yields entropy, $K(r) = T(r) \nelec(r)^{-2/3}$. The following
sections outline our method for deriving $K(r)$.

The radial temperature structure of each cluster was measured by
fitting a single-temperature thermal model to spectra extracted from
concentric annuli centered on the cluster X-ray peak. To acquire the
gas density profile, we deprojected an exposure-corrected,
background-subtracted, point source clean surface brightness profile
(extracted in the 0.7-2.0\keV\ energy range) for each cluster and
converted from observed surface brightness to emission density using a
spatially dependent count rate conversion taken from the spectral
analysis.

The temperature and deprojected density profiles were then used to
derive entropy profiles. The resulting profiles were fit with two
models: The first model assumes power-law only behavior while the
second adds a constant central entropy, \kna\, to the power-law.

Details of our reprocessing and reduction of {\it Chandra} data, along
with discussion of our background analysis, are covered more
thoroughly in DHC06 and \citealt{xrayband} (hereafter CDV08). We
direct interested reader's to those papers for in-depth
discussion. In this paper we simply cover the basics of deriving gas
entropy from X-ray observables. The primary difference between the
present analysis and that of DHC06 and CDV08, is that we have used {\tt
CIAO 3.4.1} and {\tt CALDB 3.4.0} when reducing data for this
project. All quoted uncertainties are 90\% confidence ($\Delta\chisq=
2.71 = 1.6\sigma$).

%%%%%%%%%%%%%%%%%%%%%%%%%%%%%%%%%
\subsection{Temperature Profiles}
\label{sec:temppr}
%%%%%%%%%%%%%%%%%%%%%%%%%%%%%%%%%

One of the two components needed to derive gas entropy is the
temperature as a function of radius. We therefore constructed radial
temperature profiles for each cluster in our collection. A minimum of
three annuli containing 2500 counts were required to reliably constrain
a temperature and detect temperature structure beyond simple
isothermality. To construct the annuli for each cluster, we extracted
a background-subtracted cumulative counts profile using 1 pixel width
annular bins originating from the cluster surface brightness peak and
extending to a radius bounded by the detector edge, or $1/2 R_{180}$,
whichever is smaller. Profiles are truncated at $1/2 R_{180}$ as this
is the approximate radius where temperature profiles begin to turnover
\citep{2005ApJ...628..655V}, and we are most interested in the radial
entropy behavior in the cluster core. Additionally, analysis of
cluster temperature structure at large radii requires a more time
consuming and customized analysis of the X-ray background as the
hard-particle background has been changing rapidly since 2004.

The cumulative counts profiles were then divided into annuli
containing at least 2500 counts (exact counts per annulus are listed
in Table \ref{tab:specfits}). For well resolved clusters, the number
of counts per annulus was arbitrarily increased to reduce the
resulting uncertainty of $T_X$ and, for simplicity, to keep the number
of annuli less than 50. The loss of temperature resolution from
reducing the number of annuli and mixing of different gas phases in a
given bin has an insignificant effect on the final entropy profiles.

Background analysis was performed using the blank-sky datasets provided
in the \Caldb. Backgrounds were reprocessed and reprojected to match
each observation. Off-axis chips were used to normalize for variations
of the hard-particle background by comparing blank-sky and observation
9.5-12\keV\ count rates. Soft residuals were also created and fitted
for each observation to account for the spatially-varying soft
Galactic background. This component was added as an additional, fixed
background component during spectral fitting. Errors associated with
the soft background are estimated and added in quadrature to the final
error budget.

After defining annular apertures, we extracted source spectra from the
target cluster and background spectra from the corresponding
normalized blank-sky dataset. By standard \Ciao\ means we created
response files (ARF) and redistribution matrices (RMF) for each
cluster using a flux-weighted map (WMAP) across the entire extraction
region. These files quantify the effective area, quantum efficiency,
and imperfect resolution of the \Chandra\ instrumentation. The WMAP
was calculated over the energy range 0.3-2.0\keV\ to weight
calibration which varies as a function of position on the chip. Each
spectrum was binned to contain a minimum of 25 counts per energy bin.

Spectra were fitted with {\tt XSPEC 11.3.2ag}
\citep{1996ASPC..101...17A} using an absorbed, single-temperature
thermal model over the energy range 0.7-7.0 \keV. Galactic absorption
values, $N_{HI}$, are taken from \cite{1990ARA&A..28..215D}. The
potentially free parameters of the absorbed thermal model ({\tt
WABS$\cdot$MeKaL}) are $N_{HI}$, X-ray temperature, metal abundance
normalized to Solar (elemental ratios taken from
\cite{1989GeCoA..53..197A}), and a normalization proportional to the
integrated emission measure within the extraction region:
\begin{equation}
\label{eqn:norm}
\eta = \frac{10^{-14}}{4\pi D_A^2(1+z)^2}\int \nelec \np dV\\
\end{equation}
where $D_A$ is the angular diameter distance, \red\ is cluster
redshift, \nelec\ and \np\ are the electron and proton densities
respectively, and $dV$ is the volume of the emission region. In all
fits the metal abundance in each annulus was a free parameter and
$N_{HI}$ was fixed to the Galactic value. No systematic error is added
during fitting and thus all quoted errors are statistical only.

For some clusters, more than one observation was available in the
archive. We utilized the power of the combined exposure time by first
extracting independent spectra, WARFs, WRMFs, normalized background
spectra, and soft residuals for each observation. Then, these
independent spectra were read into \textsc{XSPEC} simultaneously and
fit with one spectral model which has all parameters, except
normalization, tied among the spectra. The simultaneous fit is what is
reported for these clusters, denoted by a ($\ddagger$), in Table
\ref{tab:specfits}.

Results from the spectral analysis are presented in Table
\ref{tab:specfits} and corresponding plots of temperature profiles are
presented in the first pane of the profile gallery. Our \accept\
website\footnote{http://www.pa.msu.edu/astro/MC2/accept} also houses
plots of spectra -- and their fits -- for the annuli of every cluster
temperature profile. As in DHC06, we find spectral deprojection does
not result in significant differences between best-fit temperatures
inferred for projected or deprojected quantities. Thus, for this work,
we quote projected temperatures only. Deprojection of temperature
should result in slightly lower temperatures in the central bins of
the clusters with the steepest temperature gradients. For these
clusters, the end result would be a negligible lowering of the entropy
for the central-most bins. We stress that deprojection does not
significantly change the shape or \kna\ values we derive in this
work.

%%%%%%%%%%%%%%%%%%%%%%%%%%%%%%%%%%%%%%%%%%%%%%%%%%
\subsection{Deprojected Electron Density Profiles}
\label{sec:dene}
%%%%%%%%%%%%%%%%%%%%%%%%%%%%%%%%%%%%%%%%%%%%%%%%%%

For predominantly free-free emission, as is the case for the cluster
ICM, gas emissivity strongly depends on gas density and only weakly on
temperature, $\epsilon \propto \rho^2 T^{1/2}$. Therefore the measured
flux in a narrow bandpass is an excellent measure of ICM density. To
reconstruct the relevant gas density as a function of physical radius
we deprojected the cluster emission from high-resolution surface
brightness profiles and converted to electron density using
normalizations and count rates taken from the spectral analysis.

We began by extracting surface brightness profiles (second pane of the
profile gallery) from the 0.7-2.0\keV\ energy range using concentric
annular bins of size $\approx 5\arcsec$ (10 ACIS pixels) originating from
the X-ray emission peak. To remove the effects of vignetting and
exposure time fluctuations, we corrected each surface brightness
profile by an observation specific, normalized radial exposure
profile. Following the recommendation in the \Ciao\ guide for
analyzing extended sources, exposure maps were created using the
monoenergetic value associated with the observation count rate
peak. We also tested the more sophisticated method of using spectral
weights calculated for an incident spectrum with the temperature and
metallicity of the observed cluster. But in the narrow band we
consider, the response is relatively flat and we find no significant
differences between using spectral weights or the monochromatic
assumption. For all clusters the monoenergetic value used was $\sim
1.0-1.7\keV$.

The spectroscopic count rate and normalization in each temperature bin
were then interpolated from the temperature radial grid to match the
surface brightness radial grid. These interpolated values are then
used to convert observed surface brightness to a deprojected electron
density utilizing the deprojection technique of
\citealt{1983ApJ...272..439K}. Radial electron density can be written
in terms of relevant quantities as,
\begin{equation}
\nelec(r) = \sqrt{\frac{1.2 C(r) \eta(r) 4 \pi [D_A(1+z)]^2}{f(r) 10^{-14}}}
\end{equation}
where 1.2 comes from the ionization ratio \nelec=1.2\np, $C(r)$ is the
radial emission density derived using equation A1 of
\citealt{1983ApJ...272..439K}, $\eta$ is the spectral normalization from
Eqn. \ref{eqn:norm}, $D_A$ is angular diameter distance, \red\ is
redshift, and $f(r)$ is the 0.7-2.0\keV\ spectroscopic count rate. This
method of density determination accounts for temperature and
metallicity fluctuations which affect observed gas emissivity.

Plots of the gas density as a function of radius are presented in pane
three of the profile gallery. Fundamentally, this deprojection
technique requires an assumption of geometric symmetry, and while we
limit ourselves to the case of spherical symmetry, we also note this
assumption has little effect on our results (see Appendix A of DHC06).
Errors for the gas density profile were estimated using 5000 Monte
Carlo simulations of the original surface brightness profile.

Surface brightness irregularities, such as small inversions or
extended flat cores, result in unstable, unphysical quantities when
using the ``onion'' technique for deprojection. For cases where
deprojection of the raw data was problematic, we resorted to fitting
the surface brightness profile with the classic $\beta$-model. It is
well known that the $beta$-model is only an isothermal approximation
of the gas distribution and does not precisely represent all the
features of the ICM
\citep{2000MNRAS.311..313E,2002ApJ...579..571L,2007ApJ...665..911H}.
However, we used the $\beta$-model as a means for generating a smooth
analytic function which is easily deprojected. A more sophisticated
method of modeling the gas density \cite{2005ApJ...628..655V} would
avoid the problem of deprojection all together, but for our purposes
here, we find such elegant schemes unnecessary. The models used in
fitting were the single and double models:
of \cite{1978A&A....70..677C}:
\begin{eqnarray}
S_X(r,\beta) &=& S_0 \left[1+\left(\frac{r}{r_c}\right)^2\right]^{-3\beta+0.5} \nonumber \\
S_X(r,\beta_1,\beta_2) &=& S_{01} \left[1+\left(\frac{r}{r_{c1}}\right)^2\right]^{-3\beta_1+0.5} + S_{02} \left[1+\left(\frac{r}{r_{c2}}\right)^2\right]^{-3\beta_2+0.5} \nonumber
\end{eqnarray}
The models were fitted using Craig Markwardt's robust least-squares
minimization IDL routines\footnote{available at
http://cow.physics.wisc.edu/~craigm/idl/}.  The fits were weighted
using the inverse square of the surface brightness errors. Using this
weighting scheme results in residuals which are near unity for, on
average, the inner 80\% of the radial range considered. In all cases, the
$\beta$-model was an excellent fit to the data. Accuracy of errors
output from the fitting routine are checked against a bootstrap Monte
Carlo analysis of 1000 surface brightness realizations. Both models
were considered and we established which model better represented the
surface brightness profile by making a model comparison via an
F-test. We considered the addition of free model components to be
statistically necessary when the F-statistic was less than 0.05. A
by-eye assessment was also made.

For the clusters listed in Table \ref{tab:betafits}, we used the
smooth analytic solution of the $\beta$-model, instead of the raw
data, in deriving electron density and ultimately entropy. When
necessary, the best-fit model is also plotted in pane two of the
profile gallery. We emphasis that our use of a $\beta$-model fit for
some clusters has not artificially introduced flattened cores into
otherwise peaky surface brightness profiles. A visual comparison of
the best-fit $\beta$-model and the raw surface brightness profile
reinforces this point as there is little to no disagreement between
the size and shape of the core region in the $beta$-models and the
surface brightness.

%%%%%%%%%%%%%%%%%%%%%%%%%%%%%
\subsection{Entropy Profiles}
\label{sec:kpr}
%%%%%%%%%%%%%%%%%%%%%%%%%%%%%

We derived radial entropy profiles for the clusters which comprise
\accept\ by taking $T(r)$ and \nelec$(r)$ and calculating $K(r) =
T(r)\nelec(r)^{-2/3}$. To construct the radial entropy profiles, we
interpolated the cluster temperature profile across the
high-resolution radial grid of the deprojected electron density
profile. Temperature interpolation across the central bin is applied
in two ways: 1) as a linear gradient consistent with the slope of the
other central bins (referred to as the 'flat' model) or 2) assumed to
be constant across the central bin (referred to as the 'EXTR'
model). Shown in Figure {\ref{fig:kcomp} is the ratio of central
entropy, \kna, derived when the central temperature, $T_{X,0}$, is
assumed to be constant or interpolated. It is worth noting that both
schemes yield statistically consistent values for \kna\ except for the
instances marked by red points.

These special cases all have steep temperature gradients in their cores
with the maximum and minimum radial temperatures differing by a factor of
1.5-4.0. Extrapolating a steep temperature gradient to $r \rightarrow
0$ results in very low central temperatures ($T_X \leq 1/3
T_{virial}$) which are inconsistent with observations, most notably
\citealt{2003ApJ...590..207P}. Most importantly however, is that the
flattening of entropy we observe in the cores of our sample (discussed
in \S\ref{sec:uped}) is {\bfseries\em{not}} a result of the method
chosen for interpolating the temperature profile. For this paper we
therefore focus on the results derived assuming a constant temperature
across the central-most bin.

Uncertainties generated by having assumed a single-component
temperature gas in each annular bin and from neglecting the effect of
ICM inhomogeneity (\eg\ cavities or bubbles) are discussed in detail
in the Appendix of DHC06. Succinctly, the volume filling fraction of a
second, cooler gas phase or high-entropy component must be non-trivial
($> 50\%$) in order for our entropy profiles to be significantly
altered. As is discussed in DHC06, our results are robust to the
presence of unresolved cool gas and mostly insensitive to X-ray
surface brightness decrements.

With the exception of 24 clusters which required a $\beta$-model fit
to their surface brightness profile, we have made no theoretical
assumptions -- besides spherical symmetry and one observationally
motivated assumption for $T(r\rightarrow0)$ -- regarding the
equilibrium state of the ICM, shape or depth of the underlying dark
matter potential, or temperature distributions. Thus we consider these
entropy profiles to be the most genuine representation of the data.

%%%%%%%%%%%%%%%%%%%%%%%%%%%%%%%%
\section{Results and Discussion}
\label{sec:r&d}
%%%%%%%%%%%%%%%%%%%%%%%%%%%%%%%%

Each entropy profile was fit with two models, one which assumed a
constant entropy pedestal (1) and another which assumed only power-law
behavior (2):
\begin{eqnarray}
(1)~K(r) &=& \kna + \khun\ \left(\frac{r}{100 \kpc}\right)^{\alpha}\nonumber \\
(2)~K(r) &=& \khun\ \left(\frac{r}{100 \kpc}\right)^{\alpha}.\nonumber
\end{eqnarray}
where \kna\ is central entropy, \khun\ is a normalization for entropy
at 100\kpc, and $\alpha$ is the power-law index. Listed in Table
\ref{tab:kfits} are the results of the fitting. The table lists two
sets of values for each model: 'EXTR' which was calculated using
temperature profile which was allowed to got to zero in the central
bin, and 'FLAT' which was calculated using a constant temperature in
the central bin. Not the entire radial range was used for fitting as
the outermost bins are noisy from the ``onion'' deprojection scheme
used. The minimum and maximum radii used during fitting are also
listed. For clusters which required a double $\beta$-model fit, we
restrict the fit to the innermost region as we are most
interested in the entropy of the core region. The power-law index is
typically much steeper for these cases, but the outer regions have
power-law indices which are typical of the rest of the sample.

Shown in Figure \ref{fig:splots} is a composite of all the entropy
profiles in \accept plotted in physical units. This figure represents
the cornerstone purpose of \accept: a uniformly analyzed collection of
entropy profiles covering a broad range of central entropies. Each
profile is color-coded in representation of the global cluster
temperature. Overlaid on the figure is the theoretical pure cooling
curve calculated by \cite{2002ApJ...576..601V}. This curve represents
the entropy profile which would result from gravitational heating
alone.

In the following sections we discuss the results gleaned from analysis
of \accept. Results such as the universal entropy pedestal, the
bimodal distribution of central entropy, and the asymptotic
convergence of the entropy profiles to the self-similar $K(r) \propto
r^{-1.1}$ power-law at $r \geq 100\kpc$.

%%%%%%%%%%%%%%%%%%%%%%%%%%%%%%%%%%%%%%%
\subsection{Universal Entropy Pedestal}
\label{sec:uped}
%%%%%%%%%%%%%%%%%%%%%%%%%%%%%%%%%%%%%%%

Arguably the most striking feature of Figure \ref{fig:splots} is the
ubiquity of a central entropy pedestal. Core flattening of surface
brightness profiles (and conversely gas density) in clusters is a well
established feature which led to the advent and wide usage of
$\beta$-models (see \S\ref{sec:dene}). What is notable here however is
that based on comparison of reduced $\chi^2$, none of the clusters in
\accept\ have an entropy distribution which is best-fit by the power-law
only model. For even the most highly peaked surface brightness
profiles (e.g. Abell 478, Abell 2022, Abell 2029, Abell 2129) the
incredible resolving power of \Chandra\ allows us to discern the
subtle flattening of the gas density at small radii ($r < 10
\kpc$). Only four clusters have a central entropy which is not
statistically distinguishable from zero: 2PIGG 0011.5-2850, Abell
4059, MS 0116.3-0115, and MS 1008.1-1224. For these clusters it may be
the case that the core region of the cluster is very small
($r_{\mathrm{core}} \lesssim 5-10\kpc$) and is thus not resolved.

The departure of core entropy from the self-similar power-law
behavior is a primary signature of past heating and cooling which has
taken place inside the cluster during its formation. That this is a
ubiquitous feature of cluster entropy profiles is telling of just how
important feedback mechanisms are to understanding the global
properties of clusters and their evolution. For the whole collection,
the mean of \kna\ is $86.1 \pm 89.1 \ent$. Subdividing the collection
into two arbitrary classes -- clusters with \kna\ below and above $50
\ent$ -- we find means of $\kna\ = 19.2 \pm 11.3 \ent$ and $\kna\ =
155.5 \pm 80.9 \ent$, respectively. We show section \S\ref{sec:bimod}
that this arbitrary cut in \kna\ space is not so arbitrary, and that
the low dispersion in these two subgroups of the full sample is
related to the processes of AGN heating and possibly thermal electron
conduction.

While one may be tempted to explain the entropy pedestal feature
as a resolution effect, we direct the reader again to DHC06 where we
showed using a simulated surface brightness profile for a power-law
entropy distribution, that the \Chandra\ PSF is not creating an
envelope which limits resolution of low entropy ICM gas. Intuitively,
one can see from the gallery of profiles that an entropy core of a few
to tens of \kpc\ cannot possibly arise from instrument resolution
effects alone.

The PSF of course is not the only limitation on resolving ICM core
entropy structure. There is the issue of our choice to use fixed
angular size bins ($5\arcsec$) when extracting surface brightness
profiles. This choice may introduce an artificial resolution limit which
is redshift dependent. As redshift increases a fixed bin size
incorporates a larger physical volume and the value of \kna\ might
artificially increase because each radial bin possibly contains a
broader range of gas entropys. This can be understood as a type of
inherited resolution bias on the central entropy. Shown in Figure
{\ref{fig:k0res} is a plot of \kna\ versus redshift. The trend of
increasing central entropy with increasing redshift seen in the figure
can take on several interpretations. One interpretation is that the
trend is evidence of a redshift dependant entropy lower-boundary
introduced from resolution effects. Another explanation is that the
number of low entropy systems is decreasing as a function of redshift.

As a test of the redshift dependence on measured central entropy, we
selected all clusters with $\kna \leq 20 \ent$ and $\red \leq 0.3$
and degraded their surface brightness profiles to mimic the effect of
increasing the cluster redshift. This is best illustrated using an
example: consider a cluster at $\red = 0.1$. For this cluster,
$5\arcsec = 9 \kpc$. Were the cluster to be moved out to $\red = 0.2$,
$5\arcsec$ would equal $16 \kpc$. To mimic moving this example cluster
from $\red = 0.1$ to $\red = 0.2$, we can extract a new surface
brightness profile using a bin size of $16\kpc$ instead of
$5\arcsec$. This will result in a new surface brightness profile which
is analogous to a cluster at a higher redshift.

This procedure of artificially redshifting clusters was used on the
subsample and repeated over an evenly distributed grid of redshifts
thereby creating an ensemble of degraded surface brightness profiles
and hence entropy profiles. Using this degradation method we traced
the relative change in central entropy as a function of redshift for
each cluster with $\kna \leq 20 \ent$ and $\red \leq 0.3$. Each
cluster was stepped from $\red =$ 0.1 to  0.38 in increments of
0.02. If the nominal redshift of the cluster was less than the next
step, then that step was skipped. The effect of decreasing surface
brightness due to cosmic reddening, $(1+z)^{-4}$, was also included in
the degradation, however the effect of decreased signal-to-noise is
not considered here.

[the work for following paragraphs is not finished to my
satisfaction]:

After repeating the analysis from \S\ref{dene} and \S\ref{kpr} using
the degraded entropy profiles we find the effect of redshift on our
central entropy measure is... ???

From our analysis of the degraded entropy profiles, we find the slope
of the lower boundary in Figure \ref{fig:k0res} is steeper than it is
in the degraded profiles [TRUE?]. This points to a physical
mechanism as an explanation and not an observational effect. It has
been suggested recently that the number of cool core clusters
decreases with increasing redshift [REFS?]. We are most likely seeing
this effect in our entropy profiles. If this is the case or not, we
certainly find that the redshift resolution dependence is not
significantly inflating the central entropy values we have measured
for the clusters in \accept [TRUE?].

%%%%%%%%%%%%%%%%%%%%%%%%%%%%%%%%%%%%%%%%%%%%%%%%%%%%%%%%%%%%%%%%%%%
\subsection{Bimodality of Central Entropy and Central Cooling Time}
\label{sec:bimod}
%%%%%%%%%%%%%%%%%%%%%%%%%%%%%%%%%%%%%%%%%%%%%%%%%%%%%%%%%%%%%%%%%%%

The time required for a gas parcel to radiate away its thermal energy
is a function of the gas entropy. Low entropy gas radiates profusely
and is thus subject to rapid cooling and vice versa for high
entropy gas. Thus, the distribution of \kna\ is of particular interest
because it is an indicator of the cooling timescale in the cluster
core. Consequently, the \kna\ distribution for a carefully selected
subsample (\eg\ HIFLUGCS or REXCESS) of \accept\ can be utilized to
study how, and on what timescales, feedback mechanisms operate.

In the top panel of Figure \ref{fig:k0hist} is plotted the
logarithmically binned distribution of \kna. One can immediately see
the distribution has at a minimum two distinct populations. In the
bottom panel of Figure \ref{fig:k0hist} is plotted the cumulative
distribution of \kna. If the distinct bimodality of the \kna\
distribution were an artifact of binning, then one should expect the
cumulative distribution to be relatively smooth. But there are clearly
plateaus in the cumulative distribution, one of which coincides with
the division between the two populations at $\kna \approx 40-60
\ent$. To verify the presence of this bimodality we have performed two
additional checks: a KMM test and examination of the central cooling
time distribution.

First, we applied the KMM bimodality test of \cite{1994AJ....108.2348A}
to check for the presence of two statistically distinct \kna\
distributions. The KMM test estimates the probability that a set of
datapoints is better described by the sum of multiple Gaussians than
by a single Gaussian. We tested two cases: the first assumed that the
\kna\ distribution has two Gaussian components and the dispersions of
these distributions are different (i.e. the heteroscedastic case). The
second case was identical except with three Gaussian components. The
results of the KMM test on the first case were peaks at $54.7 \pm XXX$
and $253.8 \pm YYY \ent$. 163 clusters were assigned to the first
distribution, while 36 were assigned to the second. The KMM code also
outputs a P-value, $p$, for each fit. Under the assumption that
\chisq\ describes the distribution of the likelihood ratio statistic,
$1-p$ is the confidence interval for the model used. Our bimodal KMM
test returned $p = 0.000$, or $100\%$ confidence that the bimodal
distribution was a significantly better model than the single Gaussian
model. We found a similar result for the second case which used three
Gaussians, with peaks at $35.4 \ent$, $175.9 \ent$, and $337.5
\ent$. 140 clusters were assigned to the first distribution, 48 to the
second, and 11 to the third. The $p$-value was again 0.000. There is
little to no possibility that the \kna\ distribution we observe in
\accept\ arises from a single Gaussian population.

As a second check, we converted entropy to cooling time and compared
those values with central cooling times calculated from only the
spectral results. Cooling times were calculated using the equation,
\begin{equation}
\tcool = \frac{5nkT}{2\nelec \nH\Lambda(T,Z)}
\label{eqn:tcool}
\end{equation}
where \tcool is in seconds, $n$ is the total ion number ($\approx
2.3\nH$ for a fully ionized plasma), \nelec and \nH are the electron
and proton densities respectively, $\Lambda(T,Z)$ is the cooling
function at a particular temperature and metal abundance, and $5/2$ is
a constant appropriate for isobaric cooling. $\Lambda(T,Z)$ is
calculated for each temperature bin by inputing the best-fit
temperature and metal abundance into \xspec\ to get a flux. The
flux is then converted into $\Lambda$. The $\Lambda$ values were
interpolated across the radial grid of the electron density
profile. Shown in Figure \ref{fig:tcool} is the logarithmically binned
distribution of central cooling times.

To make a nearly one-to-one comparison, we again calculate cooling
times but using the entropy as input. Assuming free-free
interactions are the dominant gas cooling mechanism, entropy is
related to cooling time by recasting Eqn. 5.23 in \cite{sarazinbook}
as, 
\begin{equation}
\tcool \approx 10^8 \yrs \left(\frac{K}{10 \ent}\right)^{3/2} \left(\frac{kT}{5 \keV}\right)^{-1}.
\label{eqn:kcool}
\end{equation}
Because the central cooling times from Eqn. \ref{eqn:tcool} utilize
only spectroscopic information, they are mostly independent of the
entropy derivation and models we fit to the data. We can thus compare
the distributions of cooling time and \kna\ as complementary checks
for a bimodal distribution. After converting \kna\ into a cooling time
(bottom panel of Fig. \ref{fig:tcool}) and comparing the two
distributions, we find that the cooling time gaps from
Eqn. \ref{eqn:tcool} and Eqn. \ref{eqn:kcool} are consistent with each
other and occur at \tcool\ $\sim 0.7-1.0$ \Gyrs. The bimodality of the
\kna\ distribution is clearly not an artifact of our analysis
techniques.

A bimodal entropy distribution is expected to arise in the model of
episodic AGN feedback. \cite{2005ApJ...634..955V} put forth a model of
ICM heating through AGN feedback whereby outbursts of $10^{45} \ergps$
occurring every $\approx 10^8 \yrs$ can maintain a quasi-steady entropy
pedestal of $\approx 10-30 \ent$. Rare, large AGN outbursts of total
energy $10^{61} \ergps$ should push the central entropy level into the
$\approx 30-50 \ent$ range. This model predicts quite well the
distribution at $\kna \lesssim 50 \ent$, but depletion of the $40-60
\ent$ region and populating $> 60 \ent$ requires an additional piece
of physics.

If after a cluster experiences a large AGN outburst, and the central
entropy is boosted to $> 30 \ent$, some mechanism were to prevent the
ICM from cooling, the cluster core should remain at its present
entropy configuration since cooling can no longer reduce the core
entropy level. Being unable to cool, these systems will never move
back toward the mean of the \kna\ distribution. They have been
isolated beyond a hardened lower entropy limit. Subsequent mergers
could continue to raise the ICM entropy until most $40-60 \ent$
clusters are gone and nothing but $> 60 \ent$ clusters remain. We do
note however, that it is very difficult to generate enough entropy
through AGN feedback and mergers alone to produce clusters with $\kna
> 300 \ent$. To make these clusters may require some form of
``pre-heating'' or it may be the case that these clusters were never
cool.

We propose that the mechanism by which clusters in the $> 30 \ent$
regime are stabilized against cooling is electron thermal
conduction. This has been proposed before by
\cite{2005ApJ...630L..13D} in the case of two radio-quiet galaxy
clusters with \kna\ $\approx 30-60 \ent$, and is discussed in much
more detail in the forthcoming paper Voit \etal\ 2008. The premise of
the model for conductively stabilized cluster cores is quite simply
that above a critical entropy threshold, the energetic losses from
radiative cooling are balanced by thermal conduction. The model relies
on a ``coincidence of scaling'' whereby the Field length, a descriptor
for the size of a gas cloud which can condense in the presence of
conduction, is a function of entropy alone. It so happens that,
assuming reasonable suppression, the entropy scale at which conduction
balances cooling is $\kna \approx 20-30 \ent$. This is rather
auspicious, as it is the same entropy scale above which the \kna\
distribution falls-off rapidly. In Paper II we also present the result
that below this critical entropy threshold of $20-30 \ent$, signatures
of feedback (which rely on multi-phase gas) such as star formation and
radio-loud AGN are almost universally present, while above this
threshold they are not.

We must mention that \accept\ is not a complete, uniformly selected
sample of clusters, it is simply an archive-limited analysis of 
publicly available \Chandra\ data. As such, one may suspect that a
particular type of cluster (classic cooling flows or big mergers) draw
the attention of proposers and TACs alike. \kna\ bimodality could be a
manifestation of $40-60 \ent$ clusters being ``boring'' and thus
underobserved. This possibility can be easily addressed through the
study of either a complete sample of clusters or a carefully selected
sample of clusters like those in REXCESS \citep{2007A&A...469..363B}
or HIFLUGCS \citep{2002ApJ...567..716R}. We stress though that this
sociological explanation is highly unlikely.

%%%%%%%%%%%%%%%%%%%%%%%%%%%%%%%%%%%%%%%%%%%%%%%%%%%%%%%%%%%%
\subsection{Slope and Normalization of Power-law Components}
\label{sec:slopes}
%%%%%%%%%%%%%%%%%%%%%%%%%%%%%%%%%%%%%%%%%%%%%%%%%%%%%%%%%%%%

Beyond $\approx 100 \kpc$, the entropy profiles show a striking
similarity in the slope of the power-law component which is
independent of the central entropy value. For the entire collection,
$\alpha = 1.22 \pm 0.31$ while for clusters with $\kna < 50 \ent$,
$\alpha = 1.22 \pm 0.26$, and for clusters with $\kna \geq 50 \ent$,
$\alpha = 1.21 \pm 0.35$. Scaling the entropy profiles by virial
temperature and virial radius does reduce the scatter in Figure
\ref{fig:splots}, but we reserve detailed discussion of scaling
relations for a future paper. This mean slope of $\alpha = 1.22$, is
not statistically different from the theoretical value of 1.1
\citep{2001ApJ...546...63T} which arises from hierarchical cluster
formation.

The scaling factor used in the power-law component of model-1 and
model-2, $\khun$, also shows statistical similarity but with much
larger dispersion. For the whole collection, the mean value of
$\khun$ is $131.9 \pm 65.3 \ent$. Again distinguishing
between clusters below and above $\kna\ = 50 \ent$, we find
$\khun = 146.8 \pm 68.1 \ent$ and $\khun = 116.4 \pm 58.7 \ent$,
respectively. Additional entropy scaling relations and their
importance will be explored more thoroughly in Paper III of the
\accept\ series.

%%%%%%%%%%%%%%%%%%%%%%%%%%%%%%%%%
\section{Summary and Conclusions}
\label{sec:summary}
%%%%%%%%%%%%%%%%%%%%%%%%%%%%%%%%%

We have presented our analysis of 199 galaxy cluster entropy
profiles derived using archival \Chandra\ data, a project which we
have named \accept\ for Athenaeum of Chandra Cluster Entropy Profile
Tables. We deprojected surface brightness profiles extracted from
$5\arcsec$ annuli to obtain the electron gas density as a function of
radius. We also generated temperature profiles using spectra extracted
from a minimum of three concentric annuli containing 2500 counts each
and extending to either the chip edge or $0.5 R_{180}$, whichever was
smaller. Entropy profiles were then calculated from gas density and
temperature using $K(r) = T(r)n(r)^{-2/3}$. Two models for the entropy
distribution were then fit to each profile: power-law only and
power-law plus constant entropy pedestal, \kna.

We have demonstrated that for all clusters, the entropy profile is
best described by the model which behaves like a power-law at large
radii and has a constant entropy pedestal in the core. The profiles
also show a remarkable similarity at radii greater than 100 kpc, and
asymptotically approach the self-similar pure-cooling curve. We have
also shown that the distribution of central entropy in \accept\ is at
a minimum bimodal. This bimodality has a distinct peak at $\approx 20
\ent$ with a dispersion of $\approx 10 \ent$. This part of the
\kna\ distribution are clusters which would be dubbed ``classic cooling
flows'' as they have central cooling times shorter than a Hubble
time. The second broader peak of the distribution is at $\approx 150
\pm 80 \ent$. These clusters can be mostly classified as having
disturbed morphologies or as being highly diffuse or ``fluffy'' with
no dense, cool cores in the lot.

We propose that the observed range and distribution of central
entropys in \accept\ are consistent with the picture of an entropy
life-cycle which at it's heart is run by AGN feedback and
conduction. Our position is strongly supported by a number of AGN
feedback models and the continuing stream of literature detailing
voids and bubbles in the ICM filled with radio emission associated
with AGN activity. In our picture of this entropy life-cycle, clusters
undergo a set of steps as follows:
\begin{enumerate}
\item Clusters with a central cooling time shorter than a Hubble time
have been subject to prodigious core cooling at some point in their
history.
\item A very small fraction of this gas condenses and flows onto the
supermassive black hole in the core of the galaxy residing at the
bottom of the cluster potential well.
\item Resulting AGN activity with a duty cycle of a few tens of Myrs
and occurring every few hundred Myrs is capable of maintaining an entropy
pedestal of $\approx 10-30 \ent$ as is observed in \accept.
\item On rare occasion, an AGN outburst can supply enough energy to
push the central entropy to $> 30 \ent$. However, at these entropy
scales we suggest electron thermal conduction is capable of
suppressing future cooling and thus setting a hard-limit above which a
cluster will no longer cool.
\item After becoming conductively stable, the cluster is then subject
to continual mergers which boost the gas entropy ever higher.
\end{enumerate}
We suggest the above steps in a clusters entropy life-cycle describe
well the observed range of central entropys and bimodality of the
\kna\ distribution.

There are additional consequences of this entropy life-cycle which are
threshed out in Paper II. Foremost of which is that below the
conductive threshold signatures of feedback which require a multiphase
medium, such as AGN activity and star formation, should be present, as
where above the threshold they should be unilaterally absent. In Paper
II we show that below the theoretical conduction limit of $\approx 30
\ent$, AGN activity and star formation in the central cluster galaxy, as
demonstrated by radio-loud sources and \halpha\ emission, are
universally ``on'' below the limit and ``off'' above it. We suggest
this as a strong indication that the entropy life-cycle we have
pieced together in this paper is most likely how feedback is regulated
in clusters.

In the effort toward developing a self-regulating model of cluster
feedback, we believe a better understanding of ICM entropy has proven
to be pivotal. Moving forward, developers of models of massive galaxy
formation, and possibly galaxy formation in general, may find the
online database associated with \accept\ a highly useful tool.

%%%%%%%%%%%%%%%%%%%%%%%%%%%%%%%%%%%%%%%%%%%%%%%%%%%%
%%%%%%%%%%%%%%%%%%%%%%%%%%%%%%%%%%%%%%%%%%%%%%%%%%%%

\acknowledgements
Kenneth Cavagnolo was supported in this work by the National
Aeronautics and Space Administration through \Chandra\ X-ray
Observatory Archive grants AR-6016X and AR-4017A, with additional
support from a start-up grant for Megan Donahue from Michigan State
University. Megan Donahue and Michigan State University acknowledge
support from the NASA LTSA program NNG-05GD82G. The \Chandra\ X-ray
Observatory Center is operated by the Smithsonian Astrophysical
Observatory for and on behalf of the National Aeronautics Space
Administration under contract NAS8-03060. This research has made use
of software provided by the Chandra X-ray Center (CXC) in the
application packages \Ciao, \chips, and \sherpa. This research has
made use of the NASA/IPAC Extragalactic Database (NED) which is
operated by the Jet Propulsion Laboratory, California Institute of
Technology, under contract with the National Aeronautics and Space
Administration. This research has also made use of NASA's Astrophysics
Data System. Some software was obtained from the High Energy
Astrophysics Science Archive Research Center (HEASARC), provided by
NASA's Goddard Space Flight Center.

%%%%%%%%%%%%%%%%
% Bibliography %
%%%%%%%%%%%%%%%%

\bibliography{cavagnolo}

%%%%%%%%%%%%%%%%%%%%%%
% Figures  and Tables%
%%%%%%%%%%%%%%%%%%%%%%

\clearpage
\clearpage
\begin{figure}[htp]
  \begin{center}
    \begin{minipage}[htp]{0.9\linewidth}
      \includegraphics*[width=\textwidth, trim=15mm 10mm 10mm 10mm, clip]{beta.eps}
      \caption{Surface brightness profiles for clusters requiring a
        $\beta$-model fit for deprojection (discussed in
        \S\ref{sec:beta}). The best-fit $\beta$-model for each cluster
        is overplotted as a dashed line. The discrepancy between the
        data and best-fit model for some clusters results from the
        presence of a compact X-ray source at the center of the
        cluster. These cases are discussed in Appendix
        \ref{app:beta}.}
      \label{fig:betamods}
    \end{minipage}
  \end{center}
\end{figure}
\clearpage
\begin{figure}[htp]
  \begin{center}
    \begin{minipage}[htp]{0.9\linewidth}
      \includegraphics*[width=\textwidth, trim=5mm 0mm 5mm 5mm, clip]{itplflat_rat.eps}
      \caption{Ratio of best-fit \kna\ for the two treatments of
        central temperature interpolation (see \S\ref{sec:temppr}):
        (1) temperature is free to decline across the central density
        bins ($\Delta T_{center} \ne 0$), and (2) the temperature
        across the central density bins is isothermal ($\Delta
        T_{center} = 0$). Filled black squares are clusters for which
        the \kna\ ratio is inconsistent with unity.}
      \label{fig:kcomp}
    \end{minipage}
  \end{center}
\end{figure}
\clearpage
\begin{figure}[htp]
  \begin{center}
    \begin{minipage}[htp]{0.9\linewidth}
      \includegraphics*[width=\textwidth, trim=5mm 0mm 5mm 5mm, clip]{k0res.eps}
      \caption{Best-fit \kna\ vs. redshift. Some clusters have
        \kna\ error bars smaller than the point. The clusters with
        upper-limits ({\it{black points with downward arrows}}) are:
        A2151, AS0405, MS 0116.3-0115, and RX J1347.5-1145. The
        numerically labeled clusters are: (1) M87, (2) Centaurus
        Cluster, (3) RBS 533, (4) HCG 42, (5) HCG 62, (6) SS2B153, (7)
        A1991, (8) MACS0744.8+3927, and (9) CL J1226.9+3332. For
        CLJ1226, \cite{2007ApJ...659.1125M} found best-fit $\kna = 132
        \pm 24 \ent$ which is not significantly different from our
        value of $\kna = 166 \pm 45 \ent$. The lack of $\kna < 10
        \ent$ clusters at $z > 0.1$ is most likely the result of
        insufficient angular resolution (see \S\ref{sec:angres}).}
      \label{fig:k0res}
    \end{minipage}
  \end{center}
\end{figure}
\clearpage
\begin{center}
  \begin{figure}[htp]
    \begin{minipage}[htp]{0.5\linewidth}
      \includegraphics*[width=\textwidth, trim=28mm 7mm 30mm 17mm, clip]{curvk0.eps}
    \end{minipage}
    \begin{minipage}[htp]{0.5\linewidth}
      \includegraphics*[width=\textwidth, trim=28mm 7mm 30mm 17mm, clip]{nbins_k0.eps}
    \end{minipage}
    \begin{minipage}[htp]{0.5\linewidth}
      \includegraphics*[width=\textwidth, trim=28mm 7mm 30mm 17mm, clip]{texpk0.eps}
    \end{minipage}
    \begin{minipage}[htp]{0.5\linewidth}
      \includegraphics*[width=\textwidth, trim=28mm 7mm 30mm 17mm, clip]{ntxbins_k0.eps}
    \end{minipage}
    \caption{Plots of possible systematics versus best-fit \kna.
      {\it{Top left:}} Best-fit \kna\ plotted versus average curvature
      of the corresponding entropy profile (see eq. \ref{eqn:avgcurv})
      There is no trend between these two quantities suggesting that
      \kna\ is not heavily influenced by the total shape of the
      entropy profile. {\it{Top right:}} Best-fit \kna\ plotted versus
      number of bins in the entropy profile which were used during
      fitting. Again, no trend is found. {\it{Bottom left:}} Best-fit
      \kna\ plotted versus the total used exposure time for each
      cluster. No trend is found. {\it{Bottom right:}} Best-fit
      \kna\ plotted versus the number of bins in the temperature
      profile for each cluster. As expected, fewer $\Tx(r)$ does not
      correlate with \kna.}
    \label{fig:sys}
  \end{figure}
\end{center}
\clearpage
\begin{center}
  \begin{figure}[htp]
    \begin{minipage}[htp]{0.5\linewidth}
      \includegraphics*[width=\textwidth, trim=28mm 7mm 30mm 17mm, clip]{splots_allt.eps}
    \end{minipage}
    \begin{minipage}[htp]{0.5\linewidth}
      \includegraphics*[width=\textwidth, trim=28mm 7mm 30mm 17mm, clip]{splots_tle4.eps}
    \end{minipage}
    \begin{minipage}[htp]{0.5\linewidth}
      \includegraphics*[width=\textwidth, trim=28mm 7mm 30mm 17mm, clip]{splots_gt4tle8.eps}
    \end{minipage}
    \begin{minipage}[htp]{0.5\linewidth}
      \includegraphics*[width=\textwidth, trim=28mm 7mm 30mm 17mm, clip]{splots_tgt8.eps}
    \end{minipage}
    \caption{Composite plots of entropy profiles for varying cluster
      temperature ranges. Profiles are color-coded based on average
      cluster temperature. Units of the color bars are keV. The solid
      line is the pure-cooling model of \cite{voitbryan}, the dashed
      line is the mean profile for clusters with $\kna \le 50 \ent$,
      and the dashed-dotted line is the mean profile for clusters with
      $\kna > 50 \ent$. {\it{Top left:}} This panel contains all the
      entropy profiles in our study. {\it{Top right:}} Clusters with
      $kT_X < 4$ keV. {\it{Bottom left:}} Clusters with $4\keV < kT_X
      < 8\keV$. {\it{Bottom right:}} Clusters with $kT_X > 8$
      keV. Note that while the dispersion of core entropy for each
      temperature range is large, as the $kT_X$ range increases so to
      does the mean core entropy.}
    \label{fig:splots}
  \end{figure}
\end{center}
\clearpage
\begin{figure}[htp]
  \begin{center}
    \begin{minipage}[htp]{0.9\linewidth}
      \includegraphics*[width=\textwidth, trim=20mm 10mm 10mm 10mm, clip]{k0hist.eps}
      \caption{{\it{Top panel:}} Histogram of best-fit \kna\ for all
        the clusters in \accept. Bin widths are 0.15 in log space.
        {\it{Bottom panel:}} Cumulative distribution of \kna\ values
        for the full sample. The distinct bimodality in \kna\ is
        present in both distributions, which would not be seen if it
        were an artifact of the histogram binning. A KMM test finds
        the \kna\ distribution cannot arise from a simple unimodal
        Gaussian.}
      \label{fig:k0hist}
    \end{minipage}
  \end{center}
\end{figure}
\clearpage
\begin{figure}[htp]
  \begin{center}
    \begin{minipage}[htp]{0.9\linewidth}
      \includegraphics*[width=\textwidth, trim=20mm 10mm 10mm 10mm, clip]{hifl_k0hist.eps}
      \caption{{\it{Top panel:}} Histogram of best-fit \kna\ values
        for the primary \hifl\ sample. Bin widths are 0.15 in log
        space.  {\it{Bottom panel:}} Cumulative distribution of
        best-fit \kna\ values. The distinct bimodality seen in the
        full \accept\ sample (Fig. \ref{fig:k0hist}) is also present
        in the \hifl\ subsample and shares the same gap between the
        low-entropy peak at 10-20 \ent\ and the high-entropy peak at
        100-200 \ent. That bimodality is present in both samples is
        strong evidence it is not a result of an unknown archival
        bias.}
      \label{fig:hiflk0}
    \end{minipage}
  \end{center}
\end{figure}
\clearpage
\begin{figure}[htp]
  \begin{center}
    \begin{minipage}[htp]{0.8\linewidth}
      \includegraphics*[width=\textwidth, trim=20mm 10mm 10mm 10mm, clip]{t0.eps}
    \end{minipage}
    \begin{minipage}[htp]{0.8\linewidth}
      \includegraphics*[width=\textwidth, trim=20mm 10mm 10mm 10mm, clip]{k0cool.eps}
    \end{minipage}
    \caption{{\it{Top panel:}} Log-binned histogram and cumulative
      distribution of best-fit core cooling times, $t_{c0}$
      (eqn. \ref{eqn:tc0}), for all the clusters in \accept. Histogram
      bin widths are 0.2 in log space. {\it{Bottom panel:}} Log-binned
      histogram and cumulative distribution of core cooling times
      calculated from best-fit \kna\ values, $t_{c0}(\kna)$
      (eqn. \ref{eqn:tck0}), for all the clusters in
      \accept. Histogram bin widths are 0.2 in log space. The
      bimodality we observe in the \kna\ distribution is also present
      in best-fit $t_{c0}$. However, the gaps between the two
      populations of $t_{c0}$ and $t_{c0}(\kna)$ differ by $\sim 0.3$
      Gyrs which may be an artifact of the binning.}
    \label{fig:t0}
  \end{center}
\end{figure}


%\clearpage
%\LongTables
%\begin{deluxetable}{lcccccccc}
\tablewidth{0pt}
\tabletypesize{\scriptsize}
\tablecaption{Summary of Sample\label{tab:sample}}
\tablehead{\colhead{Cluster} & \colhead{Obs.ID} & \colhead{R.A.} & \colhead{Dec.} & \colhead{ExpT} & \colhead{Mode} & \colhead{ACIS} & \colhead{$z$} & \colhead{$L_{bol.}$}\\
\colhead{ } & \colhead{ } & \colhead{hr:min:sec} & \colhead{$\degr:\arcmin:\arcsec$} & \colhead{ksec} & \colhead{ } & \colhead{ } & \colhead{ } & \colhead{$10^{44}$ ergs s$^{-1}$}\\
\colhead{{(1)}} & \colhead{{(2)}} & \colhead{{(3)}} & \colhead{{(4)}} & \colhead{{(5)}} & \colhead{{(6)}} & \colhead{{(7)}} & \colhead{{(8)}} & \colhead{{(9)}}
}
\startdata
1E0657 56 & \dataset [ADS/Sa.CXO\#obs/03184] {3184} & 06:58:29.627 & -55:56:39.79 & 87.5 & VF & I3 & 0.296 & 52.48\\
1E0657 56 & \dataset [ADS/Sa.CXO\#obs/05356] {5356} & 06:58:29.619 & -55:56:39.35 & 97.2 & VF & I2 & 0.296 & 52.48\\
1E0657 56 & \dataset [ADS/Sa.CXO\#obs/05361] {5361} & 06:58:29.670 & -55:56:39.80 & 82.6 & VF & I3 & 0.296 & 52.48\\
1RXS J2129.4-0741 & \dataset [ADS/Sa.CXO\#obs/03199] {3199} & 21:29:26.274 & -07:41:29.18 & 19.9 & VF & I3 & 0.570 & 20.58\\
1RXS J2129.4-0741 & \dataset [ADS/Sa.CXO\#obs/03595] {3595} & 21:29:26.281 & -07:41:29.36 & 19.9 & VF & I3 & 0.570 & 20.58\\
2PIGG J0011.5-2850 & \dataset [ADS/Sa.CXO\#obs/05797] {5797} & 00:11:21.623 & -28:51:14.44 & 19.9 & VF & I3 & 0.075 &  2.15\\
2PIGG J0311.8-2655 $\dagger$ & \dataset [ADS/Sa.CXO\#obs/05799] {5799} & 03:11:33.904 & -26:54:16.48 & 39.6 & VF & I3 & 0.062 &  0.25\\
2PIGG J2227.0-3041 & \dataset [ADS/Sa.CXO\#obs/05798] {5798} & 22:27:54.560 & -30:34:34.84 & 22.3 & VF & I2 & 0.073 &  0.81\\
3C 220.1 & \dataset [ADS/Sa.CXO\#obs/00839] {839} & 09:32:40.218 & +79:06:29.46 & 18.9 &  F & S3 & 0.610 &  3.25\\
3C 28.0 & \dataset [ADS/Sa.CXO\#obs/03233] {3233} & 00:55:50.401 & +26:24:36.47 & 49.7 & VF & I3 & 0.195 &  4.78\\
3C 295 & \dataset [ADS/Sa.CXO\#obs/02254] {2254} & 14:11:20.280 & +52:12:10.55 & 90.9 & VF & I3 & 0.464 &  6.92\\
3C 388 & \dataset [ADS/Sa.CXO\#obs/05295] {5295} & 18:44:02.365 & +45:33:29.31 & 30.7 & VF & I3 & 0.092 &  0.52\\
4C 55.16 & \dataset [ADS/Sa.CXO\#obs/04940] {4940} & 08:34:54.923 & +55:34:21.15 & 96.0 & VF & S3 & 0.242 &  5.90\\
ABELL 0013 $\dagger$ & \dataset [ADS/Sa.CXO\#obs/04945] {4945} & 00:13:37.883 & -19:30:09.10 & 55.3 & VF & S3 & 0.094 &  1.41\\
ABELL 0068 & \dataset [ADS/Sa.CXO\#obs/03250] {3250} & 00:37:06.309 & +09:09:32.28 & 10.0 & VF & I3 & 0.255 & 12.70\\
ABELL 0119 $\dagger$ & \dataset [ADS/Sa.CXO\#obs/04180] {4180} & 00:56:15.150 & -01:14:59.70 & 11.9 & VF & I3 & 0.044 &  1.39\\
ABELL 0168 & \dataset [ADS/Sa.CXO\#obs/03203] {3203} & 01:14:57.909 & +00:24:42.55 & 40.6 & VF & I3 & 0.045 &  0.23\\
ABELL 0168 & \dataset [ADS/Sa.CXO\#obs/03204] {3204} & 01:14:57.925 & +00:24:42.73 & 37.6 & VF & I3 & 0.045 &  0.23\\
ABELL 0209 & \dataset [ADS/Sa.CXO\#obs/03579] {3579} & 01:31:52.565 & -13:36:39.29 & 10.0 & VF & I3 & 0.206 & 10.96\\
ABELL 0209 & \dataset [ADS/Sa.CXO\#obs/00522] {522} & 01:31:52.595 & -13:36:39.25 & 10.0 & VF & I3 & 0.206 & 10.96\\
ABELL 0267 & \dataset [ADS/Sa.CXO\#obs/01448] {1448} & 01:52:29.181 & +00:57:34.43 & 7.9 &  F & I3 & 0.230 &  8.62\\
ABELL 0267 & \dataset [ADS/Sa.CXO\#obs/03580] {3580} & 01:52:29.180 & +00:57:34.23 & 19.9 & VF & I3 & 0.230 &  8.62\\
ABELL 0370 & \dataset [ADS/Sa.CXO\#obs/00515] {515} & 02:39:53.169 & -01:34:36.96 & 88.0 &  F & S3 & 0.375 & 11.95\\
ABELL 0383 & \dataset [ADS/Sa.CXO\#obs/02321] {2321} & 02:48:03.364 & -03:31:44.69 & 19.5 &  F & S3 & 0.187 &  5.32\\
ABELL 0399 & \dataset [ADS/Sa.CXO\#obs/03230] {3230} & 02:57:54.931 & +13:01:58.41 & 48.6 & VF & I0 & 0.072 &  4.37\\
ABELL 0401 & \dataset [ADS/Sa.CXO\#obs/00518] {518} & 02:58:56.896 & +13:34:14.48 & 18.0 &  F & I3 & 0.074 &  8.39\\
ABELL 0478 & \dataset [ADS/Sa.CXO\#obs/06102] {6102} & 04:13:25.347 & +10:27:55.62 & 10.0 & VF & I3 & 0.088 & 16.39\\
ABELL 0514 & \dataset [ADS/Sa.CXO\#obs/03578] {3578} & 04:48:19.229 & -20:30:28.79 & 44.5 & VF & I3 & 0.072 &  0.66\\
ABELL 0520 & \dataset [ADS/Sa.CXO\#obs/04215] {4215} & 04:54:09.711 & +02:55:23.69 & 66.3 & VF & I3 & 0.202 & 12.97\\
ABELL 0521 & \dataset [ADS/Sa.CXO\#obs/00430] {430} & 04:54:07.004 & -10:13:26.72 & 39.1 & VF & S3 & 0.253 &  9.77\\
ABELL 0586 & \dataset [ADS/Sa.CXO\#obs/00530] {530} & 07:32:20.339 & +31:37:58.59 & 10.0 & VF & I3 & 0.171 &  8.54\\
ABELL 0611 & \dataset [ADS/Sa.CXO\#obs/03194] {3194} & 08:00:56.832 & +36:03:24.09 & 36.1 & VF & S3 & 0.288 & 10.78\\
ABELL 0644 $\dagger$ & \dataset [ADS/Sa.CXO\#obs/02211] {2211} & 08:17:25.225 & -07:30:40.03 & 29.7 & VF & I3 & 0.070 &  6.95\\
ABELL 0665 & \dataset [ADS/Sa.CXO\#obs/03586] {3586} & 08:30:59.231 & +65:50:37.78 & 29.7 & VF & I3 & 0.181 & 13.37\\
ABELL 0697 & \dataset [ADS/Sa.CXO\#obs/04217] {4217} & 08:42:57.549 & +36:21:57.65 & 19.5 & VF & I3 & 0.282 & 26.10\\
ABELL 0773 & \dataset [ADS/Sa.CXO\#obs/05006] {5006} & 09:17:52.566 & +51:43:38.18 & 19.8 & VF & I3 & 0.217 & 12.87\\
ABELL 0781 & \dataset [ADS/Sa.CXO\#obs/00534] {534} & 09:20:25.431 & +30:30:07.56 & 9.9 & VF & I3 & 0.298 &  0.00\\
ABELL 0907 & \dataset [ADS/Sa.CXO\#obs/03185] {3185} & 09:58:21.880 & -11:03:52.20 & 48.0 & VF & I3 & 0.153 &  6.19\\
ABELL 0963 & \dataset [ADS/Sa.CXO\#obs/00903] {903} & 10:17:03.744 & +39:02:49.17 & 36.3 &  F & S3 & 0.206 & 10.65\\
ABELL 1063S & \dataset [ADS/Sa.CXO\#obs/04966] {4966} & 22:48:44.294 & -44:31:48.37 & 26.7 & VF & I3 & 0.354 & 71.09\\
ABELL 1068 $\dagger$ & \dataset [ADS/Sa.CXO\#obs/01652] {1652} & 10:40:44.520 & +39:57:10.28 & 26.8 &  F & S3 & 0.138 &  4.19\\
ABELL 1201 $\dagger$ & \dataset [ADS/Sa.CXO\#obs/04216] {4216} & 11:12:54.489 & +13:26:08.76 & 39.7 & VF & S3 & 0.169 &  3.52\\
ABELL 1204 & \dataset [ADS/Sa.CXO\#obs/02205] {2205} & 11:13:20.419 & +17:35:38.45 & 23.6 & VF & I3 & 0.171 &  3.92\\
ABELL 1361 $\dagger$ & \dataset [ADS/Sa.CXO\#obs/02200] {2200} & 11:43:39.827 & +46:21:21.40 & 16.7 &  F & S3 & 0.117 &  2.16\\
ABELL 1423 & \dataset [ADS/Sa.CXO\#obs/00538] {538} & 11:57:17.026 & +33:36:37.44 & 9.8 & VF & I3 & 0.213 &  7.01\\
ABELL 1651 & \dataset [ADS/Sa.CXO\#obs/04185] {4185} & 12:59:22.830 & -04:11:45.86 & 9.6 & VF & I3 & 0.084 &  6.66\\
ABELL 1664 $\dagger$ & \dataset [ADS/Sa.CXO\#obs/01648] {1648} & 13:03:42.478 & -24:14:44.55 & 9.8 & VF & S3 & 0.128 &  2.59\\
ABELL 1682 & \dataset [ADS/Sa.CXO\#obs/03244] {3244} & 13:06:50.764 & +46:33:19.86 & 9.8 & VF & I3 & 0.226 &  0.00\\
ABELL 1689 & \dataset [ADS/Sa.CXO\#obs/01663] {1663} & 13:11:29.612 & -01:20:28.69 & 10.7 &  F & I3 & 0.184 & 24.71\\
ABELL 1689 & \dataset [ADS/Sa.CXO\#obs/05004] {5004} & 13:11:29.606 & -01:20:28.61 & 19.9 & VF & I3 & 0.184 & 24.71\\
ABELL 1689 & \dataset [ADS/Sa.CXO\#obs/00540] {540} & 13:11:29.595 & -01:20:28.47 & 10.3 &  F & I3 & 0.184 & 24.71\\
ABELL 1758 & \dataset [ADS/Sa.CXO\#obs/02213] {2213} & 13:32:42.978 & +50:32:44.83 & 58.3 & VF & S3 & 0.279 & 21.01\\
ABELL 1763 & \dataset [ADS/Sa.CXO\#obs/03591] {3591} & 13:35:17.957 & +40:59:55.80 & 19.6 & VF & I3 & 0.187 &  9.26\\
ABELL 1795 $\dagger$ & \dataset [ADS/Sa.CXO\#obs/05289] {5289} & 13:48:52.829 & +26:35:24.01 & 15.0 & VF & I3 & 0.062 &  7.59\\
ABELL 1835 & \dataset [ADS/Sa.CXO\#obs/00495] {495} & 14:01:01.951 & +02:52:43.18 & 19.5 &  F & S3 & 0.253 & 39.38\\
ABELL 1914 & \dataset [ADS/Sa.CXO\#obs/03593] {3593} & 14:26:01.399 & +37:49:27.83 & 18.9 & VF & I3 & 0.171 & 26.25\\
ABELL 1942 & \dataset [ADS/Sa.CXO\#obs/03290] {3290} & 14:38:21.878 & +03:40:12.97 & 57.6 & VF & I2 & 0.224 &  2.27\\
ABELL 1995 & \dataset [ADS/Sa.CXO\#obs/00906] {906} & 14:52:57.758 & +58:02:51.34 & 0.0 &  F & S3 & 0.319 & 10.19\\
ABELL 2029 $\dagger$ & \dataset [ADS/Sa.CXO\#obs/06101] {6101} & 15:10:56.163 & +05:44:40.89 & 9.9 & VF & I3 & 0.076 & 13.90\\
ABELL 2034 & \dataset [ADS/Sa.CXO\#obs/02204] {2204} & 15:10:11.003 & +33:30:46.46 & 53.9 & VF & I3 & 0.113 &  6.45\\
ABELL 2065 $\dagger$ & \dataset [ADS/Sa.CXO\#obs/031821] {31821} & 15:22:29.220 & +27:42:46.54 & 0.0 & VF & I3 & 0.073 &  2.92\\
ABELL 2069 & \dataset [ADS/Sa.CXO\#obs/04965] {4965} & 15:24:09.181 & +29:53:18.05 & 55.4 & VF & I2 & 0.116 &  3.82\\
ABELL 2111 & \dataset [ADS/Sa.CXO\#obs/00544] {544} & 15:39:41.432 & +34:25:12.26 & 10.3 &  F & I3 & 0.230 &  7.45\\
ABELL 2125 & \dataset [ADS/Sa.CXO\#obs/02207] {2207} & 15:41:14.154 & +66:15:57.20 & 81.5 & VF & I3 & 0.246 &  0.77\\
ABELL 2163 & \dataset [ADS/Sa.CXO\#obs/01653] {1653} & 16:15:45.705 & -06:09:00.62 & 71.1 & VF & I1 & 0.170 & 49.11\\
ABELL 2204 $\dagger$ & \dataset [ADS/Sa.CXO\#obs/0499] {499} & 16:32:45.437 & +05:34:21.05 & 10.1 &  F & S3 & 0.152 & 20.77\\
ABELL 2204 & \dataset [ADS/Sa.CXO\#obs/06104] {6104} & 16:32:45.428 & +05:34:20.89 & 9.6 & VF & I3 & 0.152 & 22.03\\
ABELL 2218 & \dataset [ADS/Sa.CXO\#obs/01666] {1666} & 16:35:50.831 & +66:12:42.31 & 48.6 & VF & I0 & 0.171 &  8.39\\
ABELL 2219 $\dagger$ & \dataset [ADS/Sa.CXO\#obs/0896] {896} & 16:40:21.069 & +46:42:29.07 & 42.3 &  F & S3 & 0.226 & 33.15\\
ABELL 2255 & \dataset [ADS/Sa.CXO\#obs/00894] {894} & 17:12:40.385 & +64:03:50.63 & 39.4 &  F & I3 & 0.081 &  3.67\\
ABELL 2256 $\dagger$ & \dataset [ADS/Sa.CXO\#obs/01386] {1386} & 17:03:44.567 & +78:38:11.51 & 12.4 &  F & I3 & 0.058 &  4.65\\
ABELL 2259 & \dataset [ADS/Sa.CXO\#obs/03245] {3245} & 17:20:08.299 & +27:40:11.53 & 10.0 & VF & I3 & 0.164 &  5.37\\
ABELL 2261 & \dataset [ADS/Sa.CXO\#obs/05007] {5007} & 17:22:27.254 & +32:07:58.60 & 24.3 & VF & I3 & 0.224 & 17.49\\
ABELL 2294 & \dataset [ADS/Sa.CXO\#obs/03246] {3246} & 17:24:10.149 & +85:53:09.77 & 10.0 & VF & I3 & 0.178 & 10.35\\
ABELL 2384 & \dataset [ADS/Sa.CXO\#obs/04202] {4202} & 21:52:21.178 & -19:32:51.90 & 31.5 & VF & I3 & 0.095 &  1.95\\
ABELL 2390 $\dagger$ & \dataset [ADS/Sa.CXO\#obs/04193] {4193} & 21:53:36.825 & +17:41:44.38 & 95.1 & VF & S3 & 0.230 & 31.02\\
ABELL 2409 & \dataset [ADS/Sa.CXO\#obs/03247] {3247} & 22:00:52.567 & +20:58:34.11 & 10.2 & VF & I3 & 0.148 &  7.01\\
ABELL 2537 & \dataset [ADS/Sa.CXO\#obs/04962] {4962} & 23:08:22.313 & -02:11:29.88 & 36.2 & VF & S3 & 0.295 & 10.16\\
ABELL 2550 & \dataset [ADS/Sa.CXO\#obs/02225] {2225} & 23:11:35.806 & -21:44:46.70 & 59.0 & VF & S3 & 0.154 &  0.58\\
ABELL 2554 $\dagger$ & \dataset [ADS/Sa.CXO\#obs/01696] {1696} & 23:12:19.939 & -21:30:09.84 & 19.9 & VF & S3 & 0.110 &  1.57\\
ABELL 2556 $\dagger$ & \dataset [ADS/Sa.CXO\#obs/02226] {2226} & 23:13:01.413 & -21:38:04.47 & 19.9 & VF & S3 & 0.086 &  1.43\\
ABELL 2631 & \dataset [ADS/Sa.CXO\#obs/03248] {3248} & 23:37:38.560 & +00:16:28.64 & 9.2 & VF & I3 & 0.278 & 12.59\\
ABELL 2667 & \dataset [ADS/Sa.CXO\#obs/02214] {2214} & 23:51:39.395 & -26:05:02.75 & 9.6 & VF & S3 & 0.230 & 19.91\\
ABELL 2670 & \dataset [ADS/Sa.CXO\#obs/04959] {4959} & 23:54:13.687 & -10:25:08.85 & 39.6 & VF & I3 & 0.076 &  1.39\\
ABELL 2717 & \dataset [ADS/Sa.CXO\#obs/06974] {6974} & 00:03:11.996 & -35:56:08.01 & 19.8 & VF & I3 & 0.048 &  0.26\\
ABELL 2744 & \dataset [ADS/Sa.CXO\#obs/02212] {2212} & 00:14:14.396 & -30:22:40.04 & 24.8 & VF & S3 & 0.308 & 29.00\\
ABELL 3128 $\dagger$ & \dataset [ADS/Sa.CXO\#obs/00893] {893} & 03:29:50.918 & -52:34:51.04 & 19.6 &  F & I3 & 0.062 &  0.35\\
ABELL 3158 $\dagger$ & \dataset [ADS/Sa.CXO\#obs/03201] {3201} & 03:42:54.675 & -53:37:24.36 & 24.8 & VF & I3 & 0.059 &  3.01\\
ABELL 3158 $\dagger$ & \dataset [ADS/Sa.CXO\#obs/03712] {3712} & 03:42:54.683 & -53:37:24.37 & 30.9 & VF & I3 & 0.059 &  3.01\\
ABELL 3164 & \dataset [ADS/Sa.CXO\#obs/06955] {6955} & 03:46:16.839 & -57:02:11.38 & 13.5 & VF & I3 & 0.057 &  0.19\\
ABELL 3376 & \dataset [ADS/Sa.CXO\#obs/03202] {3202} & 06:02:05.122 & -39:57:42.82 & 44.3 & VF & I3 & 0.046 &  0.75\\
ABELL 3376 & \dataset [ADS/Sa.CXO\#obs/03450] {3450} & 06:02:05.162 & -39:57:42.87 & 19.8 & VF & I3 & 0.046 &  0.75\\
ABELL 3391 $\dagger$ & \dataset [ADS/Sa.CXO\#obs/04943] {4943} & 06:26:21.511 & -53:41:44.81 & 18.4 & VF & I3 & 0.056 &  1.44\\
ABELL 3921 & \dataset [ADS/Sa.CXO\#obs/04973] {4973} & 22:49:57.829 & -64:25:42.17 & 29.4 & VF & I3 & 0.093 &  3.37\\
AC 114 & \dataset [ADS/Sa.CXO\#obs/01562] {1562} & 22:58:48.196 & -34:47:56.89 & 72.5 &  F & S3 & 0.312 & 10.90\\
CL 0024+17 & \dataset [ADS/Sa.CXO\#obs/00929] {929} & 00:26:35.996 & +17:09:45.37 & 39.8 &  F & S3 & 0.394 &  2.88\\
CL 1221+4918 & \dataset [ADS/Sa.CXO\#obs/01662] {1662} & 12:21:26.709 & +49:18:21.60 & 79.1 & VF & I3 & 0.700 &  8.65\\
CL J0030+2618 & \dataset [ADS/Sa.CXO\#obs/05762] {5762} & 00:30:34.339 & +26:18:01.58 & 17.9 & VF & I3 & 0.500 &  3.41\\
CL J0152-1357 & \dataset [ADS/Sa.CXO\#obs/00913] {913} & 01:52:42.141 & -13:57:59.71 & 36.5 &  F & I3 & 0.831 & 13.30\\
CL J0542.8-4100 & \dataset [ADS/Sa.CXO\#obs/00914] {914} & 05:42:49.994 & -40:59:58.50 & 50.4 &  F & I3 & 0.630 &  6.18\\
CL J0848+4456 & \dataset [ADS/Sa.CXO\#obs/01708] {1708} & 08:48:48.235 & +44:56:17.11 & 61.4 & VF & I1 & 0.574 &  0.62\\
CL J0848+4456 & \dataset [ADS/Sa.CXO\#obs/00927] {927} & 08:48:48.252 & +44:56:17.13 & 125.1 & VF & I1 & 0.574 &  0.62\\
CL J1113.1-2615 & \dataset [ADS/Sa.CXO\#obs/00915] {915} & 11:13:05.167 & -26:15:40.43 & 104.6 &  F & I3 & 0.730 &  2.22\\
CL J1213+0253 & \dataset [ADS/Sa.CXO\#obs/04934] {4934} & 12:13:34.948 & +02:53:45.45 & 18.9 & VF & I3 & 0.409 &  0.00\\
CL J1226.9+3332 & \dataset [ADS/Sa.CXO\#obs/03180] {3180} & 12:26:58.373 & +33:32:47.36 & 31.7 & VF & I3 & 0.890 & 30.76\\
CL J1226.9+3332 & \dataset [ADS/Sa.CXO\#obs/05014] {5014} & 12:26:58.372 & +33:32:47.18 & 32.7 & VF & I3 & 0.890 & 30.76\\
CL J1641+4001 & \dataset [ADS/Sa.CXO\#obs/03575] {3575} & 16:41:53.704 & +40:01:44.40 & 46.5 & VF & I3 & 0.464 &  0.00\\
CL J2302.8+0844 & \dataset [ADS/Sa.CXO\#obs/00918] {918} & 23:02:48.156 & +08:43:52.74 & 108.6 &  F & I3 & 0.730 &  2.93\\
DLS J0514-4904 & \dataset [ADS/Sa.CXO\#obs/04980] {4980} & 05:14:40.037 & -49:03:15.07 & 19.9 & VF & I3 & 0.091 &  0.68\\
EXO 0422-086 $\dagger$ & \dataset [ADS/Sa.CXO\#obs/04183] {4183} & 04:25:51.271 & -08:33:36.42 & 10.0 & VF & I3 & 0.040 &  0.65\\
HERCULES A $\dagger$ & \dataset [ADS/Sa.CXO\#obs/01625] {1625} & 16:51:08.161 & +04:59:32.44 & 14.8 & VF & S3 & 0.154 &  3.27\\
IRAS 09104+4109 & \dataset [ADS/Sa.CXO\#obs/00509] {509} & 09:13:45.481 & +40:56:27.49 & 9.1 &  F & S3 & 0.442 &  0.00\\
LYNX E & \dataset [ADS/Sa.CXO\#obs/017081] {17081} & 08:48:58.851 & +44:51:51.44 & 61.4 & VF & I2 & 1.260 &  0.00\\
LYNX E & \dataset [ADS/Sa.CXO\#obs/09271] {9271} & 08:48:58.858 & +44:51:51.46 & 125.1 & VF & I2 & 1.260 &  0.00\\
MACS J0011.7-1523 & \dataset [ADS/Sa.CXO\#obs/03261] {3261} & 00:11:42.965 & -15:23:20.79 & 21.6 & VF & I3 & 0.360 & 10.75\\
MACS J0011.7-1523 & \dataset [ADS/Sa.CXO\#obs/06105] {6105} & 00:11:42.957 & -15:23:20.76 & 37.3 & VF & I3 & 0.360 & 10.75\\
MACS J0025.4-1222 & \dataset [ADS/Sa.CXO\#obs/03251] {3251} & 00:25:29.368 & -12:22:38.05 & 19.3 & VF & I3 & 0.584 & 13.00\\
MACS J0025.4-1222 & \dataset [ADS/Sa.CXO\#obs/05010] {5010} & 00:25:29.399 & -12:22:38.10 & 24.8 & VF & I3 & 0.584 & 13.00\\
MACS J0035.4-2015 & \dataset [ADS/Sa.CXO\#obs/03262] {3262} & 00:35:26.573 & -20:15:46.06 & 21.4 & VF & I3 & 0.364 & 19.79\\
MACS J0111.5+0855 & \dataset [ADS/Sa.CXO\#obs/03256] {3256} & 01:11:31.515 & +08:55:39.21 & 19.4 & VF & I3 & 0.263 &  0.64\\
MACS J0152.5-2852 & \dataset [ADS/Sa.CXO\#obs/03264] {3264} & 01:52:34.479 & -28:53:38.01 & 17.5 & VF & I3 & 0.341 &  6.33\\
MACS J0159.0-3412 & \dataset [ADS/Sa.CXO\#obs/05818] {5818} & 01:59:00.366 & -34:13:00.23 & 9.4 & VF & I3 & 0.458 & 18.92\\
MACS J0159.8-0849 & \dataset [ADS/Sa.CXO\#obs/03265] {3265} & 01:59:49.453 & -08:50:00.90 & 17.9 & VF & I3 & 0.405 & 26.31\\
MACS J0159.8-0849 & \dataset [ADS/Sa.CXO\#obs/06106] {6106} & 01:59:49.422 & -08:50:00.42 & 35.3 & VF & I3 & 0.405 & 26.31\\
MACS J0242.5-2132 & \dataset [ADS/Sa.CXO\#obs/03266] {3266} & 02:42:35.906 & -21:32:26.30 & 11.9 & VF & I3 & 0.314 & 12.74\\
MACS J0257.1-2325 & \dataset [ADS/Sa.CXO\#obs/01654] {1654} & 02:57:09.130 & -23:26:06.25 & 19.8 &  F & I3 & 0.505 & 21.72\\
MACS J0257.1-2325 & \dataset [ADS/Sa.CXO\#obs/03581] {3581} & 02:57:09.152 & -23:26:06.21 & 18.5 & VF & I3 & 0.505 & 21.72\\
MACS J0257.6-2209 & \dataset [ADS/Sa.CXO\#obs/03267] {3267} & 02:57:41.024 & -22:09:11.12 & 20.5 & VF & I3 & 0.322 & 10.77\\
MACS J0308.9+2645 & \dataset [ADS/Sa.CXO\#obs/03268] {3268} & 03:08:55.927 & +26:45:38.34 & 24.4 & VF & I3 & 0.324 & 20.42\\
MACS J0329.6-0211 & \dataset [ADS/Sa.CXO\#obs/03257] {3257} & 03:29:41.681 & -02:11:47.67 & 9.9 & VF & I3 & 0.450 & 12.82\\
MACS J0329.6-0211 & \dataset [ADS/Sa.CXO\#obs/03582] {3582} & 03:29:41.688 & -02:11:47.81 & 19.9 & VF & I3 & 0.450 & 12.82\\
MACS J0329.6-0211 & \dataset [ADS/Sa.CXO\#obs/06108] {6108} & 03:29:41.681 & -02:11:47.57 & 39.6 & VF & I3 & 0.450 & 12.82\\
MACS J0404.6+1109 & \dataset [ADS/Sa.CXO\#obs/03269] {3269} & 04:04:32.491 & +11:08:02.10 & 21.8 & VF & I3 & 0.355 &  3.90\\
MACS J0417.5-1154 & \dataset [ADS/Sa.CXO\#obs/03270] {3270} & 04:17:34.686 & -11:54:32.71 & 12.0 & VF & I3 & 0.440 & 37.99\\
MACS J0429.6-0253 & \dataset [ADS/Sa.CXO\#obs/03271] {3271} & 04:29:36.088 & -02:53:09.02 & 23.2 & VF & I3 & 0.399 & 11.58\\
MACS J0451.9+0006 & \dataset [ADS/Sa.CXO\#obs/05815] {5815} & 04:51:54.291 & +00:06:20.20 & 10.2 & VF & I3 & 0.430 &  8.20\\
MACS J0455.2+0657 & \dataset [ADS/Sa.CXO\#obs/05812] {5812} & 04:55:17.426 & +06:57:47.15 & 9.9 & VF & I3 & 0.425 &  9.77\\
MACS J0520.7-1328 & \dataset [ADS/Sa.CXO\#obs/03272] {3272} & 05:20:42.052 & -13:28:49.38 & 19.2 & VF & I3 & 0.340 &  9.63\\
MACS J0547.0-3904 & \dataset [ADS/Sa.CXO\#obs/03273] {3273} & 05:47:01.582 & -39:04:28.24 & 21.7 & VF & I3 & 0.210 &  1.59\\
MACS J0553.4-3342 & \dataset [ADS/Sa.CXO\#obs/05813] {5813} & 05:53:27.200 & -33:42:53.02 & 9.9 & VF & I3 & 0.407 & 32.68\\
MACS J0717.5+3745 & \dataset [ADS/Sa.CXO\#obs/01655] {1655} & 07:17:31.654 & +37:45:18.52 & 19.9 &  F & I3 & 0.548 & 46.58\\
MACS J0717.5+3745 & \dataset [ADS/Sa.CXO\#obs/04200] {4200} & 07:17:31.651 & +37:45:18.46 & 59.2 & VF & I3 & 0.548 & 46.58\\
MACS J0744.8+3927 & \dataset [ADS/Sa.CXO\#obs/03197] {3197} & 07:44:52.802 & +39:27:24.41 & 20.2 & VF & I3 & 0.686 & 24.67\\
MACS J0744.8+3927 & \dataset [ADS/Sa.CXO\#obs/03585] {3585} & 07:44:52.779 & +39:27:24.41 & 19.9 & VF & I3 & 0.686 & 24.67\\
MACS J0744.8+3927 & \dataset [ADS/Sa.CXO\#obs/06111] {6111} & 07:44:52.800 & +39:27:24.41 & 49.5 & VF & I3 & 0.686 & 24.67\\
MACS J0911.2+1746 & \dataset [ADS/Sa.CXO\#obs/03587] {3587} & 09:11:11.325 & +17:46:31.06 & 17.9 & VF & I3 & 0.541 & 10.52\\
MACS J0911.2+1746 & \dataset [ADS/Sa.CXO\#obs/05012] {5012} & 09:11:11.309 & +17:46:30.92 & 23.8 & VF & I3 & 0.541 & 10.52\\
MACS J0949+1708   & \dataset [ADS/Sa.CXO\#obs/03274] {3274} & 09:49:51.824 & +17:07:05.62 & 14.3 & VF & I3 & 0.382 & 19.19\\
MACS J1006.9+3200 & \dataset [ADS/Sa.CXO\#obs/05819] {5819} & 10:06:54.668 & +32:01:34.61 & 10.9 & VF & I3 & 0.359 &  6.06\\
MACS J1105.7-1014 & \dataset [ADS/Sa.CXO\#obs/05817] {5817} & 11:05:46.462 & -10:14:37.20 & 10.3 & VF & I3 & 0.466 & 11.29\\
MACS J1108.8+0906 & \dataset [ADS/Sa.CXO\#obs/03252] {3252} & 11:08:55.393 & +09:05:51.16 & 9.9 & VF & I3 & 0.449 &  8.96\\
MACS J1108.8+0906 & \dataset [ADS/Sa.CXO\#obs/05009] {5009} & 11:08:55.402 & +09:05:51.14 & 24.5 & VF & I3 & 0.449 &  8.96\\
MACS J1115.2+5320 & \dataset [ADS/Sa.CXO\#obs/03253] {3253} & 11:15:15.632 & +53:20:03.71 & 8.8 & VF & I3 & 0.439 & 14.29\\
MACS J1115.2+5320 & \dataset [ADS/Sa.CXO\#obs/05008] {5008} & 11:15:15.646 & +53:20:03.74 & 18.0 & VF & I3 & 0.439 & 14.29\\
MACS J1115.2+5320 & \dataset [ADS/Sa.CXO\#obs/05350] {5350} & 11:15:15.632 & +53:20:03.37 & 6.9 & VF & I3 & 0.439 & 14.29\\
MACS J1115.8+0129 & \dataset [ADS/Sa.CXO\#obs/03275] {3275} & 11:15:52.048 & +01:29:56.56 & 15.9 & VF & I3 & 0.120 &  1.47\\
MACS J1131.8-1955 & \dataset [ADS/Sa.CXO\#obs/03276] {3276} & 11:31:56.011 & -19:55:55.85 & 13.9 & VF & I3 & 0.307 & 17.45\\
MACS J1149.5+2223 & \dataset [ADS/Sa.CXO\#obs/01656] {1656} & 11:49:35.856 & +22:23:55.02 & 18.5 & VF & I3 & 0.544 & 21.60\\
MACS J1149.5+2223 & \dataset [ADS/Sa.CXO\#obs/03589] {3589} & 11:49:35.848 & +22:23:55.05 & 20.0 & VF & I3 & 0.544 & 21.60\\
MACS J1206.2-0847 & \dataset [ADS/Sa.CXO\#obs/03277] {3277} & 12:06:12.276 & -08:48:02.40 & 23.5 & VF & I3 & 0.440 & 37.02\\
MACS J1226.8+2153 & \dataset [ADS/Sa.CXO\#obs/03590] {3590} & 12:26:51.207 & +21:49:55.22 & 19.0 & VF & I3 & 0.370 &  2.63\\
MACS J1311.0-0310 & \dataset [ADS/Sa.CXO\#obs/03258] {3258} & 13:11:01.665 & -03:10:39.50 & 14.9 & VF & I3 & 0.494 & 10.03\\
MACS J1311.0-0310 & \dataset [ADS/Sa.CXO\#obs/06110] {6110} & 13:11:01.680 & -03:10:39.75 & 63.2 & VF & I3 & 0.494 & 10.03\\
MACS J1319+7003   & \dataset [ADS/Sa.CXO\#obs/03278] {3278} & 13:20:08.370 & +70:04:33.81 & 21.6 & VF & I3 & 0.328 &  7.03\\
MACS J1427.2+4407 & \dataset [ADS/Sa.CXO\#obs/06112] {6112} & 14:27:16.175 & +44:07:30.33 & 9.4 & VF & I3 & 0.477 & 14.18\\
MACS J1427.6-2521 & \dataset [ADS/Sa.CXO\#obs/03279] {3279} & 14:27:39.389 & -25:21:04.66 & 16.9 & VF & I3 & 0.220 &  1.55\\
MACS J1621.3+3810 & \dataset [ADS/Sa.CXO\#obs/03254] {3254} & 16:21:25.552 & +38:09:43.56 & 9.8 & VF & I3 & 0.461 & 11.49\\
MACS J1621.3+3810 & \dataset [ADS/Sa.CXO\#obs/03594] {3594} & 16:21:25.558 & +38:09:43.44 & 19.7 & VF & I3 & 0.461 & 11.49\\
MACS J1621.3+3810 & \dataset [ADS/Sa.CXO\#obs/06109] {6109} & 16:21:25.535 & +38:09:43.34 & 37.5 & VF & I3 & 0.461 & 11.49\\
MACS J1621.3+3810 & \dataset [ADS/Sa.CXO\#obs/06172] {6172} & 16:21:25.559 & +38:09:43.63 & 29.8 & VF & I3 & 0.461 & 11.49\\
MACS J1731.6+2252 & \dataset [ADS/Sa.CXO\#obs/03281] {3281} & 17:31:39.902 & +22:52:00.55 & 20.5 & VF & I3 & 0.366 &  9.32\\
MACS J1824.3+4309 & \dataset [ADS/Sa.CXO\#obs/03255] {3255} & 18:24:18.444 & +43:09:43.39 & 14.9 & VF & I3 & 0.487 &  0.00\\
MACS J1931.8-2634 & \dataset [ADS/Sa.CXO\#obs/03282] {3282} & 19:31:49.656 & -26:34:33.99 & 13.6 & VF & I3 & 0.352 & 23.14\\
MACS J2046.0-3430 & \dataset [ADS/Sa.CXO\#obs/05816] {5816} & 20:46:00.522 & -34:30:15.50 & 10.0 & VF & I3 & 0.413 &  5.79\\
MACS J2049.9-3217 & \dataset [ADS/Sa.CXO\#obs/03283] {3283} & 20:49:56.245 & -32:16:52.30 & 23.8 & VF & I3 & 0.325 &  8.71\\
MACS J2211.7-0349 & \dataset [ADS/Sa.CXO\#obs/03284] {3284} & 22:11:45.856 & -03:49:37.24 & 17.7 & VF & I3 & 0.270 & 22.11\\
MACS J2214.9-1359 & \dataset [ADS/Sa.CXO\#obs/03259] {3259} & 22:14:57.467 & -14:00:09.35 & 19.5 & VF & I3 & 0.503 & 24.05\\
MACS J2214.9-1359 & \dataset [ADS/Sa.CXO\#obs/05011] {5011} & 22:14:57.481 & -14:00:09.39 & 18.5 & VF & I3 & 0.503 & 24.05\\
MACS J2228+2036   & \dataset [ADS/Sa.CXO\#obs/03285] {3285} & 22:28:33.241 & +20:37:11.42 & 19.9 & VF & I3 & 0.412 & 17.92\\
MACS J2229.7-2755 & \dataset [ADS/Sa.CXO\#obs/03286] {3286} & 22:29:45.358 & -27:55:38.41 & 16.4 & VF & I3 & 0.324 &  9.49\\
MACS J2243.3-0935 & \dataset [ADS/Sa.CXO\#obs/03260] {3260} & 22:43:21.537 & -09:35:44.30 & 20.5 & VF & I3 & 0.101 &  0.78\\
MACS J2245.0+2637 & \dataset [ADS/Sa.CXO\#obs/03287] {3287} & 22:45:04.547 & +26:38:07.88 & 16.9 & VF & I3 & 0.304 &  9.36\\
MACS J2311+0338   & \dataset [ADS/Sa.CXO\#obs/03288] {3288} & 23:11:33.213 & +03:38:06.51 & 13.6 & VF & I3 & 0.300 & 10.98\\
MKW3S & \dataset [ADS/Sa.CXO\#obs/0900] {900} & 15:21:51.930 & +07:42:31.97 & 57.3 & VF & I3 & 0.045 &  1.14\\
MS 0016.9+1609 & \dataset [ADS/Sa.CXO\#obs/00520] {520} & 00:18:33.503 & +16:26:12.99 & 67.4 & VF & I3 & 0.541 & 32.94\\
MS 0302.7+1658 & \dataset [ADS/Sa.CXO\#obs/00525] {525} & 03:05:31.614 & +17:10:02.06 & 10.0 & VF & I3 & 0.424 &  0.00\\
MS 0440.5+0204 $\dagger$ & \dataset [ADS/Sa.CXO\#obs/04196] {4196} & 04:43:09.952 & +02:10:18.70 & 59.4 & VF & S3 & 0.190 &  2.17\\
MS 0451.6-0305 & \dataset [ADS/Sa.CXO\#obs/00902] {902} & 04:54:11.004 & -03:00:52.19 & 44.2 &  F & S3 & 0.539 & 33.32\\
MS 0735.6+7421 & \dataset [ADS/Sa.CXO\#obs/04197] {4197} & 07:41:44.245 & +74:14:38.23 & 45.5 & VF & S3 & 0.216 &  7.57\\
MS 0839.8+2938 & \dataset [ADS/Sa.CXO\#obs/02224] {2224} & 08:42:55.969 & +29:27:26.97 & 29.8 &  F & S3 & 0.194 &  3.10\\
MS 0906.5+1110 & \dataset [ADS/Sa.CXO\#obs/00924] {924} & 09:09:12.753 & +10:58:32.00 & 29.7 & VF & I3 & 0.163 &  4.64\\
MS 1006.0+1202 & \dataset [ADS/Sa.CXO\#obs/00925] {925} & 10:08:47.194 & +11:47:55.99 & 29.4 & VF & I3 & 0.221 &  4.75\\
MS 1008.1-1224 & \dataset [ADS/Sa.CXO\#obs/00926] {926} & 10:10:32.312 & -12:39:56.80 & 44.2 & VF & I3 & 0.301 &  6.44\\
MS 1054.5-0321 & \dataset [ADS/Sa.CXO\#obs/00512] {512} & 10:56:58.499 & -03:37:32.76 & 89.1 &  F & S3 & 0.830 & 27.22\\
MS 1455.0+2232 & \dataset [ADS/Sa.CXO\#obs/04192] {4192} & 14:57:15.088 & +22:20:32.49 & 91.9 & VF & I3 & 0.259 & 10.25\\
MS 1621.5+2640 & \dataset [ADS/Sa.CXO\#obs/00546] {546} & 16:23:35.522 & +26:34:25.67 & 30.1 &  F & I3 & 0.426 &  6.49\\
MS 2053.7-0449 & \dataset [ADS/Sa.CXO\#obs/01667] {1667} & 20:56:21.295 & -04:37:46.81 & 44.5 & VF & I3 & 0.583 &  2.96\\
MS 2053.7-0449 & \dataset [ADS/Sa.CXO\#obs/00551] {551} & 20:56:21.297 & -04:37:46.80 & 44.3 &  F & I3 & 0.583 &  2.96\\
MS 2137.3-2353 & \dataset [ADS/Sa.CXO\#obs/04974] {4974} & 21:40:15.178 & -23:39:40.71 & 57.4 & VF & S3 & 0.313 & 11.28\\
MS J1157.3+5531 $\dagger$ & \dataset [ADS/Sa.CXO\#obs/04964] {4964} & 11:59:52.295 & +55:32:05.61 & 75.1 & VF & S3 & 0.081 &  0.12\\
NGC 6338 $\dagger$ & \dataset [ADS/Sa.CXO\#obs/04194] {4194} & 17:15:23.036 & +57:24:40.29 & 47.3 & VF & I3 & 0.028 &  0.13\\
PKS 0745-191 & \dataset [ADS/Sa.CXO\#obs/06103] {6103} & 07:47:31.469 & -19:17:40.01 & 10.3 & VF & I3 & 0.103 & 18.41\\
RBS 0797 & \dataset [ADS/Sa.CXO\#obs/02202] {2202} & 09:47:12.971 & +76:23:13.90 & 11.7 & VF & I3 & 0.354 & 26.07\\
RDCS 1252-29    & \dataset [ADS/Sa.CXO\#obs/04198] {4198} & 12:52:54.221 & -29:27:21.01 & 163.4 & VF & I3 & 1.237 &  2.28\\
RX J0232.2-4420 & \dataset [ADS/Sa.CXO\#obs/04993] {4993} & 02:32:18.771 & -44:20:46.68 & 23.4 & VF & I3 & 0.284 & 18.17\\
RX J0340-4542   & \dataset [ADS/Sa.CXO\#obs/06954] {6954} & 03:40:44.765 & -45:41:18.41 & 17.9 & VF & I3 & 0.082 &  0.33\\
RX J0439+0520   & \dataset [ADS/Sa.CXO\#obs/00527] {527} & 04:39:02.218 & +05:20:43.11 & 9.6 & VF & I3 & 0.208 &  3.57\\
RX J0439.0+0715 & \dataset [ADS/Sa.CXO\#obs/01449] {1449} & 04:39:00.710 & +07:16:07.65 & 6.3 &  F & I3 & 0.230 &  9.44\\
RX J0439.0+0715 & \dataset [ADS/Sa.CXO\#obs/03583] {3583} & 04:39:00.710 & +07:16:07.63 & 19.2 & VF & I3 & 0.230 &  9.44\\
RX J0528.9-3927 & \dataset [ADS/Sa.CXO\#obs/04994] {4994} & 05:28:53.039 & -39:28:15.53 & 22.5 & VF & I3 & 0.263 & 12.99\\
RX J0647.7+7015 & \dataset [ADS/Sa.CXO\#obs/03196] {3196} & 06:47:50.029 & +70:14:49.66 & 19.3 & VF & I3 & 0.584 & 26.48\\
RX J0647.7+7015 & \dataset [ADS/Sa.CXO\#obs/03584] {3584} & 06:47:50.014 & +70:14:49.69 & 20.0 & VF & I3 & 0.584 & 26.48\\
RX J0819.6+6336 $\dagger$ & \dataset [ADS/Sa.CXO\#obs/02199] {2199} & 08:19:26.007 & +63:37:26.53 & 14.9 &  F & S3 & 0.119 &  0.98\\
RX J0910+5422   & \dataset [ADS/Sa.CXO\#obs/02452] {2452} & 09:10:44.478 & +54:22:03.77 & 65.3 & VF & I3 & 1.100 &  1.33\\
RX J1053+5735   & \dataset [ADS/Sa.CXO\#obs/04936] {4936} & 10:53:39.844 & +57:35:18.42 & 92.2 &  F & S3 & 1.140 &  0.00\\
RX J1347.5-1145 & \dataset [ADS/Sa.CXO\#obs/03592] {3592} & 13:47:30.593 & -11:45:10.05 & 57.7 & VF & I3 & 0.451 & 100.36\\
RX J1347.5-1145 & \dataset [ADS/Sa.CXO\#obs/00507] {507} & 13:47:30.598 & -11:45:10.27 & 10.0 &  F & S3 & 0.451 & 100.36\\
RX J1350+6007   & \dataset [ADS/Sa.CXO\#obs/02229] {2229} & 13:50:48.038 & +60:07:08.39 & 58.3 & VF & I3 & 0.804 &  2.19\\
RX J1423.8+2404 & \dataset [ADS/Sa.CXO\#obs/01657] {1657} & 14:23:47.759 & +24:04:40.45 & 18.5 & VF & I3 & 0.545 & 15.84\\
RX J1423.8+2404 & \dataset [ADS/Sa.CXO\#obs/04195] {4195} & 14:23:47.763 & +24:04:40.63 & 115.6 & VF & S3 & 0.545 & 15.84\\
RX J1504.1-0248 & \dataset [ADS/Sa.CXO\#obs/05793] {5793} & 15:04:07.415 & -02:48:15.70 & 39.2 & VF & I3 & 0.215 & 34.64\\
RX J1525+0958   & \dataset [ADS/Sa.CXO\#obs/01664] {1664} & 15:24:39.729 & +09:57:44.42 & 50.9 & VF & I3 & 0.516 &  3.29\\
RX J1532.9+3021 & \dataset [ADS/Sa.CXO\#obs/01649] {1649} & 15:32:55.642 & +30:18:57.69 & 9.4 & VF & S3 & 0.345 & 20.77\\
RX J1532.9+3021 & \dataset [ADS/Sa.CXO\#obs/01665] {1665} & 15:32:55.641 & +30:18:57.31 & 10.0 & VF & I3 & 0.345 & 20.77\\
RX J1716.9+6708 & \dataset [ADS/Sa.CXO\#obs/00548] {548} & 17:16:49.015 & +67:08:25.80 & 51.7 &  F & I3 & 0.810 &  8.04\\
RX J1720.1+2638 & \dataset [ADS/Sa.CXO\#obs/04361] {4361} & 17:20:09.941 & +26:37:29.11 & 25.7 & VF & I3 & 0.164 & 11.39\\
RX J1720.2+3536 & \dataset [ADS/Sa.CXO\#obs/03280] {3280} & 17:20:16.953 & +35:36:23.63 & 20.8 & VF & I3 & 0.391 & 13.02\\
RX J1720.2+3536 & \dataset [ADS/Sa.CXO\#obs/06107] {6107} & 17:20:16.949 & +35:36:23.68 & 33.9 & VF & I3 & 0.391 & 13.02\\
RX J1720.2+3536 & \dataset [ADS/Sa.CXO\#obs/07225] {7225} & 17:20:16.947 & +35:36:23.69 & 2.0 & VF & I3 & 0.391 & 13.02\\
RX J2011.3-5725 & \dataset [ADS/Sa.CXO\#obs/04995] {4995} & 20:11:26.889 & -57:25:09.08 & 24.0 & VF & I3 & 0.279 &  2.77\\
RX J2129.6+0005 & \dataset [ADS/Sa.CXO\#obs/00552] {552} & 21:29:39.944 & +00:05:18.83 & 10.0 & VF & I3 & 0.235 & 12.56\\
S0463 & \dataset [ADS/Sa.CXO\#obs/06956] {6956} & 04:29:07.040 & -53:49:38.02 & 29.3 & VF & I3 & 0.099 & 22.19\\
S0463 & \dataset [ADS/Sa.CXO\#obs/07250] {7250} & 04:29:07.063 & -53:49:38.11 & 29.1 & VF & I3 & 0.099 & 22.19\\
TRIANG AUSTR $\dagger$ & \dataset [ADS/Sa.CXO\#obs/01281] {1281} & 16:38:22.712 & -64:21:19.70 & 11.4 &  F & I3 & 0.051 &  9.41\\
V 1121.0+2327 & \dataset [ADS/Sa.CXO\#obs/01660] {1660} & 11:20:57.195 & +23:26:27.60 & 71.3 & VF & I3 & 0.560 &  3.28\\
ZWCL 1215 & \dataset [ADS/Sa.CXO\#obs/04184] {4184} & 12:17:40.787 & +03:39:39.42 & 12.1 & VF & I3 & 0.075 &  3.49\\
ZWCL 1358+6245 & \dataset [ADS/Sa.CXO\#obs/00516] {516} & 13:59:50.526 & +62:31:04.57 & 54.1 &  F & S3 & 0.328 & 12.42\\
ZWCL 1953 & \dataset [ADS/Sa.CXO\#obs/01659] {1659} & 08:50:06.677 & +36:04:16.16 & 24.9 &  F & I3 & 0.380 & 17.11\\
ZWCL 3146 & \dataset [ADS/Sa.CXO\#obs/00909] {909} & 10:23:39.735 & +04:11:08.05 & 46.0 &  F & I3 & 0.290 & 29.59\\
ZWCL 5247 & \dataset [ADS/Sa.CXO\#obs/00539] {539} & 12:34:21.928 & +09:47:02.83 & 9.3 & VF & I3 & 0.229 &  4.87\\
ZWCL 7160 & \dataset [ADS/Sa.CXO\#obs/00543] {543} & 14:57:15.158 & +22:20:33.85 & 9.9 &  F & I3 & 0.258 & 10.14\\
ZWICKY 2701 & \dataset [ADS/Sa.CXO\#obs/03195] {3195} & 09:52:49.183 & +51:53:05.27 & 26.9 & VF & S3 & 0.210 &  5.19\\
ZwCL 1332.8+5043 & \dataset [ADS/Sa.CXO\#obs/05772] {5772} & 13:34:20.698 & +50:31:04.64 & 19.5 & VF & I3 & 0.620 &  4.46\\
ZwCl 0848.5+3341 & \dataset [ADS/Sa.CXO\#obs/04205] {4205} & 08:51:38.873 & +33:31:08.00 & 11.4 & VF & S3 & 0.371 &  4.58
\enddata
\tablecomments{(1) Cluster name, (2) CDA observation identification number, (3) R.A. of cluster center, (4) Dec. of cluster center, (5) nominal exposure time, (6) observing mode, (7) CCD location of centroid, (8) redshift, (9) NRAO absorbing Galactic neutral hydrogen column density, (10) bolometric luminosity. $\dagger$ indicates clusters analyzed within R$_{5000}$ only.}
\end{deluxetable}

%\clearpage
%\begin{deluxetable}{lcccccccc}
\tablewidth{0pt}
\tabletypesize{\scriptsize}
\tablecaption{Summary of $\beta$-Model Fits\label{tab:betafits}}
\tablehead{\colhead{Cluster} & \colhead{$S_{01}$} & \colhead{$r_{c1}$} & \colhead{$\beta_{1}$} & \colhead{$S_{02}$} & \colhead{$r_{c2}$} & \colhead{$\beta_{2}$} & \colhead{D.O.F.} & \colhead{$\chi_{\mathrm{red}}^2$}\\
\colhead{} & \colhead{10$^{-6}$ cts s$^{-1}$ arcsec$^{2}$} & \colhead{\arcsec} & \colhead{} & \colhead{10$^{-6}$ cts s$^{-1}$ arcsec$^{2}$} & \colhead{\arcsec} & \colhead{} & \colhead{} & \colhead{}\\
\colhead{{(1)}} & \colhead{{(2)}} & \colhead{{(3)}} & \colhead{{(4)}} & \colhead{{(5)}} & \colhead{{(6)}} & \colhead{{(7)}} & \colhead{{(8)}} & \colhead{{(9)}}
}
\startdata
Abell 119 &  4.93 $\pm$  0.73 &  39.1 $\pm$  15.3 &  0.34 $\pm$  0.07 &  3.52 $\pm$  0.96 & 735.2 $\pm$ 479.4 &  1.27 $\pm$  1.27 &    52 &  1.76\\
Abell 160 &  2.32 $\pm$  0.27 &  53.4 $\pm$  11.1 &  0.57 $\pm$  0.12 &  1.29 $\pm$  0.22 & 284.0 $\pm$  52.2 &  0.74 $\pm$  0.10 &    90 &  1.18\\
Abell 193 & 24.72 $\pm$  1.62 &  80.8 $\pm$   2.2 &  0.43 $\pm$  0.01 & \nodata & \nodata & \nodata &    38 &  0.43\\
Abell 400 &  4.66 $\pm$  0.09 & 151.3 $\pm$   6.4 &  0.42 $\pm$  0.01 & \nodata & \nodata & \nodata &    96 &  0.57\\
Abell 1060 & 21.95 $\pm$  0.44 &  93.5 $\pm$   8.1 &  0.35 $\pm$  0.01 & \nodata & \nodata & \nodata &    42 &  1.44\\
Abell 1240 &  1.58 $\pm$  0.07 & 247.9 $\pm$  46.9 &  1.01 $\pm$  0.22 & \nodata & \nodata & \nodata &    58 &  1.58\\
Abell 1736 &  3.81 $\pm$  0.56 &  55.6 $\pm$  16.1 &  0.42 $\pm$  0.12 &  2.49 $\pm$  0.47 & 1470.0 $\pm$  87.2 &  5.00 $\pm$  0.73 &    35 &  1.58\\
Abell 2125 &  3.50 $\pm$  0.20 &  26.0 $\pm$   4.9 &  0.49 $\pm$  0.05 &  1.02 $\pm$  0.13 & 159.9 $\pm$   9.2 &  1.32 $\pm$  0.16 &    35 &  0.33\\
Abell 2255 &  8.38 $\pm$  0.15 & 222.7 $\pm$   9.8 &  0.62 $\pm$  0.02 & \nodata & \nodata & \nodata &    94 &  1.45\\
Abell 2256 & 21.69 $\pm$  0.19 & 407.8 $\pm$  17.9 &  0.99 $\pm$  0.05 & \nodata & \nodata & \nodata &    88 &  0.83\\
Abell 2319 & 47.39 $\pm$  0.61 & 128.8 $\pm$   3.1 &  0.49 $\pm$  0.01 & \nodata & \nodata & \nodata &    92 &  1.67\\
Abell 2462 &  8.19 $\pm$  1.43 &  60.8 $\pm$   9.6 &  0.64 $\pm$  0.11 &  1.87 $\pm$  0.25 & 762.7 $\pm$  39.1 &  5.00 $\pm$  0.87 &    67 &  1.54\\
Abell 2631 & 20.55 $\pm$  1.01 &  66.0 $\pm$   4.0 &  0.73 $\pm$  0.03 & \nodata & \nodata & \nodata &    58 &  1.15\\
Abell 3376 &  4.21 $\pm$  0.09 & 125.5 $\pm$   5.6 &  0.40 $\pm$  0.01 & \nodata & \nodata & \nodata &    98 &  1.42\\
Abell 3391 & 10.65 $\pm$  0.31 & 132.3 $\pm$   7.9 &  0.48 $\pm$  0.01 & \nodata & \nodata & \nodata &    84 &  1.86\\
Abell 3395 &  6.85 $\pm$  0.67 &  90.9 $\pm$   6.7 &  0.49 $\pm$  0.03 & \nodata & \nodata & \nodata &    38 &  0.96\\
MKW 8 &  7.71 $\pm$  0.62 &  25.2 $\pm$   2.5 &  0.32 $\pm$  0.01 &  1.51 $\pm$  0.08 & 1124.0 $\pm$  64.1 &  5.00 $\pm$  0.40 &    88 &  0.65\\
RBS 461 & 12.84 $\pm$  0.34 & 102.2 $\pm$   4.1 &  0.52 $\pm$  0.01 & \nodata & \nodata & \nodata &    84 &  1.56\\
\enddata
\tablecomments{Col. (1) Cluster name; col. (2) central surface brightness of first component; col. (3) core radius of first component; col. (4) $\beta$ parameter of first component; col. (5) central surface brightness of second component; col. (6) core radius of second component; col. (7) $\beta$ parameter of second component; col. (8) model degrees of freedom; and col. (9) reduced chi-squared statistic for best-fit model.}
\end{deluxetable}

%\clearpage
%\begin{table*}
\caption{\sc Summary of Global ICM Spectral Fits. \label{tab:specfits}}
\begin{tabular}{lcccccccccc}
\hline
\hline
Region & $R_{\mathrm{in}}$ & $R_{\mathrm{out}}$  & \tx & \lbol & $Z$ & \redchisq & D.O.F. & \% Source & $\eta$ & Ct. Rate\\
- & kpc & kpc & keV & $10^{44}$ erg s$^{-1}$ & $Z_{\sun}$ & - & - & - & $10^{-4}$ cm$^{-5}$ & ct s$^{-1}$\\
(1) & (2) & (3) & (4) & (5) & (6) & (7) & (8) & (9) & (10) & (11)\\
\hline
$R_{500-\mathrm{Core}}$ & 174 & 1160 & 7.54$^{+1.76}_{-1.15}$  & 6.90$^{+0.61}_{-0.59}$  & 0.38$^{+0.31}_{-0.17}$  & 1.01 & 277 &  27 & 8.24   $^{+6\%}_{-6\%}$  & 0.063\\
$R_{1000-\mathrm{Core}}$ & 174 & 820 & 6.80$^{+1.14}_{-0.88}$  & 6.17$^{+0.41}_{-0.57}$  & 0.38$\dagger$ & 1.05 & 219 &  38 & 7.90   $^{+3\%}_{-3\%}$  & 0.058\\
$R_{2500-\mathrm{Core}}$ & 174 & 519 & 7.18$^{+1.25}_{-0.93}$  & 5.18$^{+0.41}_{-0.38}$  & 0.38$\dagger$ & 1.06 & 150 &  56 & 6.48   $^{+3\%}_{-3\%}$  & 0.048\\
$R_{500}$ & 13 & 1160 & 5.61$^{+0.32}_{-0.30}$  & 24.8$^{+2.9}_{-2.5}$  & 0.43$^{+0.09}_{-0.08}$  & 0.78 & 357 &  54 & 34.1  $^{+2\%}_{-2\%}$  & 0.237\\
$R_{1000}$ & 13 & 820 & 5.49$^{+0.28}_{-0.26}$  & 24.2$^{+2.6}_{-2.4}$  & 0.40$^{+0.08}_{-0.08}$  & 0.80 & 306 &  68 & 33.8  $^{+2\%}_{-2\%}$  & 0.232\\
$R_{2500}$ & 13 & 519 & 5.50$^{+0.27}_{-0.25}$  & 23.1$^{+2.5}_{-2.0}$  & 0.39$^{+0.07}_{-0.07}$  & 0.82 & 265 &  83 & 32.4  $^{+2\%}_{-2\%}$  & 0.222\\
$R_{\mathrm{cool}}$ & 13 & 128 & 4.94$^{+0.24}_{-0.22}$  & 16.1$^{+2.5}_{-2.0}$  & 0.42$^{+0.08}_{-0.08}$  & 0.89 & 208 &  98 & 22.9  $^{+3\%}_{-3\%}$  & 0.155\\
\hline
\end{tabular}
\begin{quote}
A dagger ($\dagger$) indicates core-excised regions fit with $Z$ fixed
at the iteratively determined value for
$R_{500-\mathrm{Core}}$. Bolometric luminosities were determined using
a diagonalized response function over the energy range 0.01-100.0 keV
with 5000 linearly spaced energy channels. Col. (1) Spectral
extraction region; Col. (2) Inner radius; Col. (3) Outer radius;
Col. (4) Gas temperature; Col. (5) Unabsorbed bolometric luminosity;
Col. (6) Gas abundance; Col. (7) Reduced \chisq; Col. (8) Degrees of
freedom; Col. (9) Percentage of emission attributable to source;
Col. (10) Model normalization; Col. (11) Background-subtracted count
rate.
\end{quote}
\end{table*}

%% $R_{200-\mathrm{Core}}$ & 174 & 1835 & 10.22$^{+5.38}_{-2.65}$ & 7.95$^{+1.50}_{-1.80}$  & 0.38$\dagger$ & 1.13 & 429 &  15 & 8.50   $^{+4\%}_{-4\%}$  & 0.066\\
%% $R_{5000-\mathrm{Core}}$ & 174 & 367 & 6.40$^{+1.01}_{-0.80}$  & 3.80$^{+0.44}_{-0.26}$  & 0.38$\dagger$ & 1.01 & 104 &  67 & 4.99   $^{+4\%}_{-3\%}$  & 0.036\\
%% $R_{7500-\mathrm{Core}}$ & 174 & 300 & 6.52$^{+1.16}_{-0.89}$  & 2.92$^{+0.23}_{-0.22}$  & 0.38$\dagger$ & 1.24 &  73 &  74 & 3.81   $^{+4\%}_{-4\%}$  & 0.027\\
%% $R_{200}$ & 13 & 1835 & 6.02$^{+0.45}_{-0.40}$  & 25.6$^{+2.7}_{-2.4}$  & 0.41$^{+0.10}_{-0.11}$  & 1.00 & 498 &  34 & 34.4  $^{+2\%}_{-2\%}$  & 0.240\\
%% $R_{5000}$ & 13 & 367 & 5.34$^{+0.25}_{-0.23}$  & 21.8$^{+2.1}_{-1.9}$  & 0.39$^{+0.07}_{-0.07}$  & 0.80 & 241 &  90 & 30.9  $^{+2\%}_{-2\%}$  & 0.210\\
%% $R_{7500}$ & 13 & 300 & 5.30$^{+0.24}_{-0.24}$  & 20.9$^{+1.7}_{-1.7}$  & 0.40$^{+0.07}_{-0.07}$  & 0.86 & 233 &  93 & 29.6  $^{+2\%}_{-2\%}$  & 0.202\\

%\clearpage
%\begin{deluxetable}{lcccccccccc}
\tablewidth{0pt}
\tabletypesize{\scriptsize}
\tablecaption{Summary of Entropy Profile Fits\label{tab:kfits}}
\tablehead{\colhead{Cluster} & \colhead{Method} & \colhead{$N_{bins}$} & \colhead{$r_{max}$} & \colhead{\kna} & \colhead{$\sigma_{\kna} > 0$} & \colhead{\khun} & \colhead{$\alpha$} & \colhead{DOF} & \colhead{\chisq} & \colhead{p-value}\\
\colhead{} & \colhead{} & \colhead{} & \colhead{Mpc} & \colhead{\ent} & \colhead{} & \colhead{\ent} & \colhead{} & \colhead{} & \colhead{} & \colhead{}\\
\colhead{{(1)}} & \colhead{{(2)}} & \colhead{{(3)}} & \colhead{{(4)}} & \colhead{{(5)}} & \colhead{{(6)}} & \colhead{{(7)}} & \colhead{{(8)}} & \colhead{{(9)}} & \colhead{{(10)}} & \colhead{{(11)}}
}
\startdata
1E0657 56 &   extr &     48 &   1.00 &  299.4 $\pm$   19.6 &   15.3 &   20.5 $\pm$    7.0 &   1.84 $\pm$   0.16 &     45 &  42.09 & 5.96e-01\\
 &      - & - & - &    0.0 & - &  277.9 $\pm$   14.5 &   0.60 $\pm$   0.04 &     46 & 146.18 & 2.31e-12\\
 &   flat & - & - &  307.5 $\pm$   19.3 &   15.9 &   18.6 $\pm$    6.5 &   1.88 $\pm$   0.17 &     45 &  42.87 & 5.63e-01\\
 &      - & - & - &    0.0 & - &  283.6 $\pm$   14.6 &   0.58 $\pm$   0.04 &     46 & 157.03 & 4.77e-14\\
2A 335+096 &   extr &     37 &   0.12 &    5.3 $\pm$    0.2 &   34.8 &  137.7 $\pm$    1.9 &   1.43 $\pm$   0.02 &     34 & 173.51 & 1.26e-20\\
 &      - & - & - &    0.0 & - &  117.7 $\pm$    1.5 &   1.06 $\pm$   0.01 &     35 & 1188.38 & 6.24e-227\\
 &   flat & - & - &    7.1 $\pm$    0.1 &   49.3 &  138.6 $\pm$    1.9 &   1.52 $\pm$   0.02 &     34 & 209.16 & 4.39e-27\\
 &      - & - & - &    0.0 & - &  107.4 $\pm$    1.4 &   0.97 $\pm$   0.01 &     35 & 2097.26 & 0.00e+00\\
2PIGG J0011.5-2850 &   extr &     27 &   0.20 &   75.3 $\pm$   44.8 &    1.7 &  236.9 $\pm$   53.2 &   0.82 $\pm$   0.27 &     24 &   2.01 & 1.00e+00\\
 &      - & - & - &    0.0 & - &  318.5 $\pm$   13.6 &   0.53 $\pm$   0.06 &     25 &   3.19 & 1.00e+00\\
 &   flat & - & - &  102.0 $\pm$   42.9 &    2.4 &  214.7 $\pm$   51.5 &   0.84 $\pm$   0.29 &     24 &   2.79 & 1.00e+00\\
 &      - & - & - &    0.0 & - &  323.8 $\pm$   13.7 &   0.45 $\pm$   0.05 &     25 &   4.40 & 1.00e+00\\
2PIGG J2227.0-3041 &   extr &     23 &   0.15 &   12.5 $\pm$    1.0 &   12.3 &  119.5 $\pm$    3.6 &   1.32 $\pm$   0.06 &     20 &  13.14 & 8.71e-01\\
 &      - & - & - &    0.0 & - &  118.4 $\pm$    3.3 &   0.88 $\pm$   0.02 &     21 & 132.53 & 3.43e-18\\
 &   flat & - & - &   17.1 $\pm$    1.0 &   17.4 &  113.9 $\pm$    3.6 &   1.37 $\pm$   0.06 &     20 &  11.50 & 9.32e-01\\
 &      - & - & - &    0.0 & - &  108.6 $\pm$    3.1 &   0.73 $\pm$   0.02 &     21 & 202.76 & 1.04e-31\\
3C 28.0 &   extr &     12 &   0.18 &   20.7 $\pm$    1.3 &   15.5 &  111.7 $\pm$    3.9 &   1.70 $\pm$   0.09 &      9 &  23.42 & 5.32e-03\\
 &      - & - & - &    0.0 & - &  115.3 $\pm$    3.4 &   0.82 $\pm$   0.03 &     10 & 151.06 & 2.25e-27\\
 &   flat & - & - &   23.9 $\pm$    1.3 &   18.6 &  107.8 $\pm$    3.9 &   1.79 $\pm$   0.09 &      9 &  22.93 & 6.35e-03\\
 &      - & - & - &    0.0 & - &  110.8 $\pm$    3.3 &   0.74 $\pm$   0.03 &     10 & 179.58 & 2.86e-33\\
3C 295 &   extr &     17 &   0.50 &   12.6 $\pm$    2.6 &    4.9 &   84.5 $\pm$    6.4 &   1.45 $\pm$   0.07 &     14 &   7.52 & 9.13e-01\\
 &      - & - & - &    0.0 & - &  108.2 $\pm$    3.8 &   1.20 $\pm$   0.04 &     15 &  27.39 & 2.57e-02\\
 &   flat & - & - &   14.5 $\pm$    2.5 &    5.8 &   81.9 $\pm$    6.3 &   1.47 $\pm$   0.07 &     14 &   8.36 & 8.70e-01\\
 &      - & - & - &    0.0 & - &  109.3 $\pm$    3.8 &   1.18 $\pm$   0.04 &     15 &  34.84 & 2.59e-03\\
3C 388 &   extr &     24 &   0.20 &   17.0 $\pm$    5.7 &    3.0 &  214.2 $\pm$    8.5 &   0.76 $\pm$   0.07 &     21 &  10.82 & 9.66e-01\\
 &      - & - & - &    0.0 & - &  226.3 $\pm$    7.0 &   0.60 $\pm$   0.02 &     22 &  16.13 & 8.09e-01\\
 &   flat & - & - &   17.0 $\pm$    5.8 &    3.0 &  214.3 $\pm$    8.5 &   0.76 $\pm$   0.07 &     21 &  10.90 & 9.65e-01\\
 &      - & - & - &    0.0 & - &  226.4 $\pm$    7.0 &   0.60 $\pm$   0.02 &     22 &  16.14 & 8.09e-01\\
4C 55.16 &   extr &     21 &   0.40 &   22.4 $\pm$    2.9 &    7.7 &  162.9 $\pm$    7.7 &   1.28 $\pm$   0.06 &     18 &   7.52 & 9.85e-01\\
 &      - & - & - &    0.0 & - &  197.1 $\pm$    5.6 &   0.94 $\pm$   0.03 &     19 &  46.97 & 3.61e-04\\
 &   flat & - & - &   23.3 $\pm$    2.9 &    8.1 &  161.6 $\pm$    7.7 &   1.29 $\pm$   0.06 &     18 &   7.92 & 9.80e-01\\
 &      - & - & - &    0.0 & - &  197.0 $\pm$    5.6 &   0.93 $\pm$   0.03 &     19 &  50.60 & 1.07e-04\\
Abell 13 &   extr &     35 &   0.30 &  182.6 $\pm$   26.2 &    7.0 &  182.0 $\pm$   36.8 &   1.37 $\pm$   0.22 &     32 &  11.58 & 1.00e+00\\
 &      - & - & - &    0.0 & - &  401.9 $\pm$   14.1 &   0.59 $\pm$   0.05 &     33 &  32.03 & 5.15e-01\\
 &   flat & - & - &  182.6 $\pm$   26.2 &    7.0 &  182.0 $\pm$   36.8 &   1.37 $\pm$   0.22 &     32 &  11.58 & 1.00e+00\\
 &      - & - & - &    0.0 & - &  401.9 $\pm$   14.1 &   0.59 $\pm$   0.05 &     33 &  32.03 & 5.15e-01\\
Abell 68 &   extr &     31 &   0.60 &  217.3 $\pm$   89.0 &    2.4 &  142.3 $\pm$   98.3 &   0.89 $\pm$   0.39 &     28 &   1.72 & 1.00e+00\\
 &      - & - & - &    0.0 & - &  393.4 $\pm$   36.9 &   0.40 $\pm$   0.08 &     29 &   3.45 & 1.00e+00\\
 &   flat & - & - &  217.3 $\pm$   89.0 &    2.4 &  142.3 $\pm$   98.3 &   0.89 $\pm$   0.39 &     28 &   1.72 & 1.00e+00\\
 &      - & - & - &    0.0 & - &  393.4 $\pm$   36.9 &   0.40 $\pm$   0.08 &     29 &   3.45 & 1.00e+00\\
Abell 85 &   extr &     39 &   0.20 &    7.3 $\pm$    0.6 &   12.8 &  165.5 $\pm$    1.9 &   1.05 $\pm$   0.02 &     36 &  52.57 & 3.67e-02\\
 &      - & - & - &    0.0 & - &  170.2 $\pm$    1.8 &   0.90 $\pm$   0.01 &     37 & 201.42 & 1.67e-24\\
 &   flat & - & - &   12.5 $\pm$    0.5 &   23.7 &  158.8 $\pm$    1.9 &   1.12 $\pm$   0.02 &     36 &  59.03 & 9.10e-03\\
 &      - & - & - &    0.0 & - &  165.5 $\pm$    1.8 &   0.83 $\pm$   0.01 &     37 & 492.25 & 6.48e-81\\
Abell 119 &   extr &     23 &   0.20 &  210.1 $\pm$   84.5 &    2.5 &  207.1 $\pm$  100.1 &   0.77 $\pm$   0.56 &     20 &   0.12 & 1.00e+00\\
 &      - & - & - &    0.0 & - &  418.7 $\pm$   31.2 &   0.26 $\pm$   0.07 &     21 &   1.34 & 1.00e+00\\
 &   flat & - & - &  233.9 $\pm$   87.7 &    2.7 &  191.3 $\pm$  102.8 &   0.75 $\pm$   0.61 &     20 &   0.10 & 1.00e+00\\
 &      - & - & - &    0.0 & - &  425.5 $\pm$   31.1 &   0.22 $\pm$   0.06 &     21 &   1.19 & 1.00e+00\\
Abell 133 &   extr &     20 &   0.10 &   13.3 $\pm$    0.5 &   25.1 &  170.7 $\pm$    3.9 &   1.47 $\pm$   0.04 &     17 &  44.38 & 3.01e-04\\
 &      - & - & - &    0.0 & - &  142.2 $\pm$    2.7 &   0.90 $\pm$   0.01 &     18 & 504.69 & 1.08e-95\\
 &   flat & - & - &   17.3 $\pm$    0.5 &   35.0 &  170.1 $\pm$    4.1 &   1.59 $\pm$   0.04 &     17 &  54.26 & 9.02e-06\\
 &      - & - & - &    0.0 & - &  127.5 $\pm$    2.5 &   0.79 $\pm$   0.01 &     18 & 812.02 & 8.79e-161\\
Abell 141 &   extr &     33 &   0.60 &  144.1 $\pm$   31.3 &    4.6 &   68.5 $\pm$   27.5 &   1.53 $\pm$   0.27 &     30 & 136.92 & 1.32e-15\\
 &      - & - & - &    0.0 & - &  221.9 $\pm$   18.4 &   0.77 $\pm$   0.09 &     31 & 447.75 & 2.25e-75\\
 &   flat & - & - &  205.0 $\pm$   27.4 &    7.5 &   42.6 $\pm$   20.8 &   1.78 $\pm$   0.33 &     30 & 175.31 & 1.84e-22\\
 &      - & - & - &    0.0 & - &  269.7 $\pm$   17.7 &   0.57 $\pm$   0.07 &     31 & 704.66 & 2.56e-128\\
Abell 160 &   extr &     28 &   0.12 &  155.8 $\pm$   27.7 &    5.6 &  116.3 $\pm$   29.2 &   0.98 $\pm$   0.57 &     25 &   0.33 & 1.00e+00\\
 &      - & - & - &    0.0 & - &  254.7 $\pm$   13.5 &   0.20 $\pm$   0.04 &     26 &   3.66 & 1.00e+00\\
 &   flat & - & - &  155.8 $\pm$   27.7 &    5.6 &  116.3 $\pm$   29.2 &   0.98 $\pm$   0.57 &     25 &   0.33 & 1.00e+00\\
 &      - & - & - &    0.0 & - &  254.7 $\pm$   13.5 &   0.20 $\pm$   0.04 &     26 &   3.66 & 1.00e+00\\
Abell 193 &   extr &     26 &   0.12 &  185.5 $\pm$   13.3 &   13.9 &   36.0 $\pm$   16.8 &   2.23 $\pm$   1.89 &     23 &   0.02 & 1.00e+00\\
 &      - & - & - &    0.0 & - &  213.8 $\pm$    7.3 &   0.09 $\pm$   0.04 &     24 &   2.92 & 1.00e+00\\
 &   flat & - & - &  185.5 $\pm$   13.3 &   13.9 &   36.0 $\pm$   16.8 &   2.23 $\pm$   1.89 &     23 &   0.02 & 1.00e+00\\
 &      - & - & - &    0.0 & - &  213.8 $\pm$    7.3 &   0.09 $\pm$   0.04 &     24 &   2.92 & 1.00e+00\\
Abell 209 &   extr &     19 &   0.30 &  100.7 $\pm$   26.3 &    3.8 &  150.5 $\pm$   34.5 &   0.81 $\pm$   0.21 &     16 &   2.48 & 1.00e+00\\
 &      - & - & - &    0.0 & - &  266.2 $\pm$    9.6 &   0.40 $\pm$   0.04 &     17 &   7.88 & 9.69e-01\\
 &   flat & - & - &  105.5 $\pm$   26.9 &    3.9 &  149.3 $\pm$   35.2 &   0.80 $\pm$   0.21 &     16 &   2.73 & 1.00e+00\\
 &      - & - & - &    0.0 & - &  269.5 $\pm$    9.6 &   0.38 $\pm$   0.04 &     17 &   8.03 & 9.66e-01\\
Abell 222 &   extr &     37 &   0.60 &  122.2 $\pm$   15.2 &    8.0 &   84.8 $\pm$   19.2 &   0.99 $\pm$   0.15 &     34 &   4.82 & 1.00e+00\\
 &      - & - & - &    0.0 & - &  231.9 $\pm$    7.3 &   0.40 $\pm$   0.03 &     35 &  26.22 & 8.58e-01\\
 &   flat & - & - &  126.0 $\pm$   15.0 &    8.4 &   82.2 $\pm$   19.0 &   1.00 $\pm$   0.15 &     34 &   4.94 & 1.00e+00\\
 &      - & - & - &    0.0 & - &  233.9 $\pm$    7.3 &   0.39 $\pm$   0.03 &     35 &  27.16 & 8.26e-01\\
Abell 223 &   extr &     30 &   0.50 &  183.9 $\pm$   46.1 &    4.0 &  160.7 $\pm$   59.2 &   1.24 $\pm$   0.31 &     27 &   1.35 & 1.00e+00\\
 &      - & - & - &    0.0 & - &  386.1 $\pm$   23.5 &   0.57 $\pm$   0.08 &     28 &   6.55 & 1.00e+00\\
 &   flat & - & - &  183.9 $\pm$   46.1 &    4.0 &  160.7 $\pm$   59.2 &   1.24 $\pm$   0.31 &     27 &   1.35 & 1.00e+00\\
 &      - & - & - &    0.0 & - &  386.1 $\pm$   23.5 &   0.57 $\pm$   0.08 &     28 &   6.55 & 1.00e+00\\
Abell 262 &   extr &     30 &   0.05 &    9.4 $\pm$    0.8 &   11.8 &  200.9 $\pm$    7.3 &   0.95 $\pm$   0.04 &     27 &  52.37 & 2.40e-03\\
 &      - & - & - &    0.0 & - &  166.6 $\pm$    3.3 &   0.66 $\pm$   0.01 &     28 & 159.48 & 2.36e-20\\
 &   flat & - & - &   10.6 $\pm$    0.8 &   13.8 &  205.1 $\pm$    7.9 &   0.98 $\pm$   0.04 &     27 &  60.17 & 2.50e-04\\
 &      - & - & - &    0.0 & - &  164.3 $\pm$    3.3 &   0.65 $\pm$   0.01 &     28 & 199.73 & 7.70e-28\\
Abell 267 &   extr &     22 &   0.40 &  168.3 $\pm$   17.7 &    9.5 &   52.0 $\pm$   21.1 &   1.82 $\pm$   0.38 &     19 &   0.62 & 1.00e+00\\
 &      - & - & - &    0.0 & - &  263.4 $\pm$   11.7 &   0.41 $\pm$   0.06 &     20 &  22.64 & 3.07e-01\\
 &   flat & - & - &  168.6 $\pm$   17.6 &    9.6 &   51.8 $\pm$   21.0 &   1.82 $\pm$   0.38 &     19 &   0.62 & 1.00e+00\\
 &      - & - & - &    0.0 & - &  263.5 $\pm$   11.7 &   0.40 $\pm$   0.06 &     20 &  22.71 & 3.03e-01\\
Abell 368 &   extr &     28 &   0.50 &   47.5 $\pm$    8.3 &    5.7 &  146.7 $\pm$   15.4 &   1.20 $\pm$   0.11 &     25 &   6.13 & 1.00e+00\\
 &      - & - & - &    0.0 & - &  216.8 $\pm$    8.0 &   0.77 $\pm$   0.04 &     26 &  24.09 & 5.71e-01\\
 &   flat & - & - &   50.9 $\pm$    8.2 &    6.2 &  144.1 $\pm$   15.4 &   1.21 $\pm$   0.11 &     25 &   6.18 & 1.00e+00\\
 &      - & - & - &    0.0 & - &  218.7 $\pm$    8.0 &   0.74 $\pm$   0.04 &     26 &  26.03 & 4.61e-01\\
Abell 370 &   extr &     20 &   0.50 &  321.9 $\pm$   90.8 &    3.5 &   78.7 $\pm$   89.3 &   1.24 $\pm$   0.68 &     17 &   2.41 & 1.00e+00\\
 &      - & - & - &    0.0 & - &  422.4 $\pm$   34.9 &   0.40 $\pm$   0.08 &     18 &   6.02 & 9.96e-01\\
 &   flat & - & - &  321.9 $\pm$   90.8 &    3.5 &   78.7 $\pm$   89.3 &   1.24 $\pm$   0.68 &     17 &   2.41 & 1.00e+00\\
 &      - & - & - &    0.0 & - &  422.4 $\pm$   34.9 &   0.40 $\pm$   0.08 &     18 &   6.02 & 9.96e-01\\
Abell 383 &   extr &     13 &   0.20 &   10.9 $\pm$    1.6 &    6.6 &  114.0 $\pm$    5.2 &   1.34 $\pm$   0.09 &     10 &   4.76 & 9.07e-01\\
 &      - & - & - &    0.0 & - &  121.4 $\pm$    4.9 &   0.96 $\pm$   0.04 &     11 &  40.90 & 2.50e-05\\
 &   flat & - & - &   13.0 $\pm$    1.6 &    8.3 &  110.9 $\pm$    5.2 &   1.40 $\pm$   0.09 &     10 &   6.30 & 7.89e-01\\
 &      - & - & - &    0.0 & - &  119.2 $\pm$    4.9 &   0.92 $\pm$   0.03 &     11 &  58.48 & 1.78e-08\\
Abell 399 &   extr &     31 &   0.20 &  140.3 $\pm$   19.1 &    7.3 &  215.3 $\pm$   22.7 &   0.73 $\pm$   0.12 &     28 &   4.14 & 1.00e+00\\
 &      - & - & - &    0.0 & - &  360.8 $\pm$    7.0 &   0.32 $\pm$   0.02 &     29 &  21.40 & 8.44e-01\\
 &   flat & - & - &  153.2 $\pm$   18.8 &    8.2 &  204.3 $\pm$   22.4 &   0.74 $\pm$   0.12 &     28 &   4.19 & 1.00e+00\\
 &      - & - & - &    0.0 & - &  362.5 $\pm$    7.0 &   0.30 $\pm$   0.02 &     29 &  22.24 & 8.10e-01\\
Abell 400 &   extr &     73 &   0.18 &  162.8 $\pm$    3.9 &   41.6 &   35.3 $\pm$    5.7 &   1.76 $\pm$   0.28 &     70 &   0.71 & 1.00e+00\\
 &      - & - & - &    0.0 & - &  205.9 $\pm$    2.1 &   0.17 $\pm$   0.01 &     71 &  57.23 & 8.82e-01\\
 &   flat & - & - &  162.8 $\pm$    3.9 &   41.6 &   35.3 $\pm$    5.7 &   1.76 $\pm$   0.28 &     70 &   0.71 & 1.00e+00\\
 &      - & - & - &    0.0 & - &  205.9 $\pm$    2.1 &   0.17 $\pm$   0.01 &     71 &  57.23 & 8.82e-01\\
Abell 401 &   extr &     60 &   0.40 &  162.5 $\pm$    7.9 &   20.7 &   86.0 $\pm$   10.7 &   1.37 $\pm$   0.11 &     57 &   8.70 & 1.00e+00\\
 &      - & - & - &    0.0 & - &  290.7 $\pm$    4.7 &   0.43 $\pm$   0.02 &     58 & 134.73 & 4.81e-08\\
 &   flat & - & - &  166.9 $\pm$    7.7 &   21.7 &   81.8 $\pm$   10.4 &   1.40 $\pm$   0.11 &     57 &   8.36 & 1.00e+00\\
 &      - & - & - &    0.0 & - &  292.0 $\pm$    4.7 &   0.42 $\pm$   0.02 &     58 & 142.56 & 4.50e-09\\
Abell 426 &   extr &     56 &   0.10 &   19.4 $\pm$    0.2 &  124.3 &  119.9 $\pm$    0.5 &   1.74 $\pm$   0.01 &     53 & 1040.29 & 3.10e-183\\
 &      - & - & - &    0.0 & - &  112.3 $\pm$    0.3 &   0.92 $\pm$   0.00 &     54 & 6430.00 & 0.00e+00\\
 &   flat & - & - &   19.4 $\pm$    0.2 &  124.4 &  119.9 $\pm$    0.5 &   1.74 $\pm$   0.01 &     53 & 1045.73 & 2.32e-184\\
 &      - & - & - &    0.0 & - &  112.3 $\pm$    0.3 &   0.92 $\pm$   0.00 &     54 & 6447.72 & 0.00e+00\\
Abell 478 &   extr &     49 &   0.40 &    6.9 $\pm$    0.9 &    7.5 &  123.4 $\pm$    2.6 &   0.96 $\pm$   0.02 &     46 &  20.38 & 1.00e+00\\
 &      - & - & - &    0.0 & - &  136.7 $\pm$    1.7 &   0.84 $\pm$   0.01 &     47 &  66.62 & 3.13e-02\\
 &   flat & - & - &    7.8 $\pm$    0.9 &    8.5 &  122.0 $\pm$    2.6 &   0.97 $\pm$   0.02 &     46 &  22.58 & 9.99e-01\\
 &      - & - & - &    0.0 & - &  137.0 $\pm$    1.7 &   0.84 $\pm$   0.01 &     47 &  81.79 & 1.25e-03\\
Abell 496 &   extr &     26 &   0.08 &    4.3 $\pm$    0.8 &    5.7 &  206.1 $\pm$    9.2 &   1.13 $\pm$   0.04 &     23 &   7.05 & 9.99e-01\\
 &      - & - & - &    0.0 & - &  182.9 $\pm$    6.6 &   0.94 $\pm$   0.02 &     24 &  36.09 & 5.38e-02\\
 &   flat & - & - &    8.9 $\pm$    0.7 &   13.4 &  216.3 $\pm$   10.5 &   1.27 $\pm$   0.05 &     23 &   6.95 & 1.00e+00\\
 &      - & - & - &    0.0 & - &  161.2 $\pm$    5.8 &   0.83 $\pm$   0.02 &     24 & 132.18 & 6.24e-17\\
Abell 520 &   extr &     33 &   0.55 &  325.5 $\pm$   29.2 &   11.1 &   10.2 $\pm$   11.8 &   2.09 $\pm$   0.71 &     30 &   2.86 & 1.00e+00\\
 &      - & - & - &    0.0 & - &  328.7 $\pm$   18.7 &   0.29 $\pm$   0.05 &     31 &  14.09 & 9.96e-01\\
 &   flat & - & - &  325.5 $\pm$   29.2 &   11.1 &   10.2 $\pm$   11.8 &   2.09 $\pm$   0.71 &     30 &   2.86 & 1.00e+00\\
 &      - & - & - &    0.0 & - &  328.7 $\pm$   18.7 &   0.29 $\pm$   0.05 &     31 &  14.09 & 9.96e-01\\
Abell 521 &   extr &      8 &   0.15 &  201.6 $\pm$   36.1 &    5.6 &  235.7 $\pm$   61.8 &   1.92 $\pm$   0.72 &      5 &   0.23 & 9.99e-01\\
 &      - & - & - &    0.0 & - &  420.3 $\pm$   37.9 &   0.44 $\pm$   0.10 &      6 &   9.70 & 1.38e-01\\
 &   flat & - & - &  259.9 $\pm$   36.2 &    7.2 &  245.4 $\pm$   61.8 &   1.91 $\pm$   0.69 &      5 &   0.32 & 9.97e-01\\
 &      - & - & - &    0.0 & - &  481.0 $\pm$   37.3 &   0.35 $\pm$   0.08 &      6 &  11.51 & 7.39e-02\\
Abell 539 &   extr &     11 &   0.03 &   19.6 $\pm$    4.0 &    4.9 &  552.4 $\pm$  198.3 &   1.14 $\pm$   0.21 &      8 &   1.80 & 9.86e-01\\
 &      - & - & - &    0.0 & - &  241.9 $\pm$   31.9 &   0.58 $\pm$   0.05 &      9 &  10.03 & 3.48e-01\\
 &   flat & - & - &   22.6 $\pm$    4.5 &    5.0 &  493.3 $\pm$  165.6 &   1.05 $\pm$   0.20 &      8 &   2.12 & 9.77e-01\\
 &      - & - & - &    0.0 & - &  234.5 $\pm$   27.5 &   0.53 $\pm$   0.04 &      9 &  10.08 & 3.44e-01\\
Abell 562 &   extr &     27 &   0.27 &  202.1 $\pm$   39.3 &    5.1 &   34.6 $\pm$   45.3 &   1.09 $\pm$   1.19 &     24 &   1.66 & 1.00e+00\\
 &      - & - & - &    0.0 & - &  244.4 $\pm$    9.7 &   0.13 $\pm$   0.06 &     25 &   2.41 & 1.00e+00\\
 &   flat & - & - &  202.1 $\pm$   39.3 &    5.1 &   34.6 $\pm$   45.3 &   1.09 $\pm$   1.19 &     24 &   1.66 & 1.00e+00\\
 &      - & - & - &    0.0 & - &  244.4 $\pm$    9.7 &   0.13 $\pm$   0.06 &     25 &   2.41 & 1.00e+00\\
Abell 576 &   extr &     21 &   0.08 &   78.4 $\pm$   18.7 &    4.2 &  230.6 $\pm$   26.6 &   1.19 $\pm$   0.34 &     18 &   3.81 & 1.00e+00\\
 &      - & - & - &    0.0 & - &  259.8 $\pm$   16.1 &   0.51 $\pm$   0.06 &     19 &  10.60 & 9.37e-01\\
 &   flat & - & - &   95.3 $\pm$   15.4 &    6.2 &  221.2 $\pm$   31.5 &   1.41 $\pm$   0.41 &     18 &   4.71 & 9.99e-01\\
 &      - & - & - &    0.0 & - &  247.8 $\pm$   15.2 &   0.45 $\pm$   0.06 &     19 &  15.49 & 6.91e-01\\
Abell 586 &   extr &     17 &   0.25 &   94.7 $\pm$   19.2 &    4.9 &   92.1 $\pm$   25.5 &   1.25 $\pm$   0.32 &     14 &   3.47 & 9.98e-01\\
 &      - & - & - &    0.0 & - &  201.4 $\pm$    7.2 &   0.53 $\pm$   0.06 &     15 &  10.34 & 7.98e-01\\
 &   flat & - & - &   94.7 $\pm$   19.2 &    4.9 &   92.1 $\pm$   25.5 &   1.25 $\pm$   0.32 &     14 &   3.47 & 9.98e-01\\
 &      - & - & - &    0.0 & - &  201.4 $\pm$    7.2 &   0.53 $\pm$   0.06 &     15 &  10.34 & 7.98e-01\\
Abell 611 &   extr &     19 &   0.40 &  124.9 $\pm$   18.6 &    6.7 &  164.4 $\pm$   31.5 &   1.25 $\pm$   0.20 &     16 &   1.98 & 1.00e+00\\
 &      - & - & - &    0.0 & - &  326.7 $\pm$   15.2 &   0.53 $\pm$   0.05 &     17 &  14.90 & 6.02e-01\\
 &   flat & - & - &  124.9 $\pm$   18.6 &    6.7 &  164.4 $\pm$   31.5 &   1.25 $\pm$   0.20 &     16 &   1.98 & 1.00e+00\\
 &      - & - & - &    0.0 & - &  326.7 $\pm$   15.2 &   0.53 $\pm$   0.05 &     17 &  14.90 & 6.02e-01\\
Abell 644 &   extr &     53 &   0.35 &  132.4 $\pm$    9.1 &   14.5 &   85.9 $\pm$   11.7 &   1.55 $\pm$   0.13 &     50 &  15.09 & 1.00e+00\\
 &      - & - & - &    0.0 & - &  244.8 $\pm$    4.3 &   0.68 $\pm$   0.03 &     51 &  90.43 & 5.59e-04\\
 &   flat & - & - &  132.4 $\pm$    9.1 &   14.5 &   85.9 $\pm$   11.7 &   1.55 $\pm$   0.13 &     50 &  15.09 & 1.00e+00\\
 &      - & - & - &    0.0 & - &  244.8 $\pm$    4.3 &   0.68 $\pm$   0.03 &     51 &  90.43 & 5.59e-04\\
Abell 665 &   extr &     46 &   0.70 &  134.6 $\pm$   23.5 &    5.7 &  106.3 $\pm$   25.1 &   1.06 $\pm$   0.13 &     43 &   3.79 & 1.00e+00\\
 &      - & - & - &    0.0 & - &  254.8 $\pm$   10.1 &   0.61 $\pm$   0.04 &     44 &  19.71 & 9.99e-01\\
 &   flat & - & - &  134.6 $\pm$   23.5 &    5.7 &  106.3 $\pm$   25.1 &   1.06 $\pm$   0.13 &     43 &   3.79 & 1.00e+00\\
 &      - & - & - &    0.0 & - &  254.8 $\pm$   10.1 &   0.61 $\pm$   0.04 &     44 &  19.71 & 9.99e-01\\
Abell 697 &   extr &     30 &   0.60 &  161.0 $\pm$   24.7 &    6.5 &  111.1 $\pm$   29.5 &   1.09 $\pm$   0.18 &     27 &   4.01 & 1.00e+00\\
 &      - & - & - &    0.0 & - &  310.0 $\pm$   13.4 &   0.46 $\pm$   0.04 &     28 &  19.49 & 8.82e-01\\
 &   flat & - & - &  166.7 $\pm$   24.4 &    6.8 &  108.2 $\pm$   29.1 &   1.10 $\pm$   0.18 &     27 &   4.28 & 1.00e+00\\
 &      - & - & - &    0.0 & - &  313.9 $\pm$   13.3 &   0.45 $\pm$   0.04 &     28 &  20.28 & 8.54e-01\\
Abell 744 &   extr &     18 &   0.12 &   60.3 $\pm$    9.4 &    6.4 &  227.9 $\pm$   15.4 &   0.83 $\pm$   0.13 &     15 &   1.20 & 1.00e+00\\
 &      - & - & - &    0.0 & - &  251.0 $\pm$   11.7 &   0.41 $\pm$   0.03 &     16 &  13.36 & 6.46e-01\\
 &   flat & - & - &   63.4 $\pm$   10.2 &    6.2 &  229.3 $\pm$   15.2 &   0.79 $\pm$   0.13 &     15 &   1.27 & 1.00e+00\\
 &      - & - & - &    0.0 & - &  256.9 $\pm$   11.5 &   0.39 $\pm$   0.02 &     16 &  12.56 & 7.05e-01\\
Abell 754 &   extr &     58 &   0.30 &  270.4 $\pm$   23.8 &   11.4 &   69.7 $\pm$   26.5 &   1.48 $\pm$   0.34 &     55 &  13.35 & 1.00e+00\\
 &      - & - & - &    0.0 & - &  366.4 $\pm$    8.1 &   0.34 $\pm$   0.03 &     56 &  35.36 & 9.86e-01\\
 &   flat & - & - &  270.4 $\pm$   23.8 &   11.4 &   69.7 $\pm$   26.5 &   1.48 $\pm$   0.34 &     55 &  13.35 & 1.00e+00\\
 &      - & - & - &    0.0 & - &  366.4 $\pm$    8.1 &   0.34 $\pm$   0.03 &     56 &  35.36 & 9.86e-01\\
Abell 773 &   extr &     35 &   0.60 &  244.3 $\pm$   31.7 &    7.7 &   41.1 $\pm$   22.5 &   1.60 $\pm$   0.33 &     32 &   3.28 & 1.00e+00\\
 &      - & - & - &    0.0 & - &  283.2 $\pm$   16.6 &   0.54 $\pm$   0.06 &     33 &  19.39 & 9.71e-01\\
 &   flat & - & - &  244.3 $\pm$   31.7 &    7.7 &   41.1 $\pm$   22.5 &   1.60 $\pm$   0.33 &     32 &   3.28 & 1.00e+00\\
 &      - & - & - &    0.0 & - &  283.2 $\pm$   16.6 &   0.54 $\pm$   0.06 &     33 &  19.39 & 9.71e-01\\
Abell 907 &   extr &     31 &   0.40 &   20.4 $\pm$    3.3 &    6.1 &  191.5 $\pm$    8.1 &   1.02 $\pm$   0.05 &     28 &   7.33 & 1.00e+00\\
 &      - & - & - &    0.0 & - &  223.9 $\pm$    5.4 &   0.81 $\pm$   0.02 &     29 &  32.96 & 2.79e-01\\
 &   flat & - & - &   23.4 $\pm$    3.2 &    7.3 &  187.0 $\pm$    8.1 &   1.05 $\pm$   0.05 &     28 &   7.62 & 1.00e+00\\
 &      - & - & - &    0.0 & - &  224.1 $\pm$    5.4 &   0.79 $\pm$   0.02 &     29 &  41.74 & 5.92e-02\\
Abell 963 &   extr &     24 &   0.40 &   22.0 $\pm$   15.7 &    1.4 &  205.5 $\pm$   22.9 &   0.79 $\pm$   0.09 &     21 &   2.75 & 1.00e+00\\
 &      - & - & - &    0.0 & - &  234.8 $\pm$    7.8 &   0.68 $\pm$   0.04 &     22 &   4.30 & 1.00e+00\\
 &   flat & - & - &   55.8 $\pm$   12.9 &    4.3 &  169.1 $\pm$   20.3 &   0.90 $\pm$   0.10 &     21 &   3.37 & 1.00e+00\\
 &      - & - & - &    0.0 & - &  244.6 $\pm$    7.6 &   0.61 $\pm$   0.03 &     22 &  13.86 & 9.06e-01\\
Abell 1060 &   extr &     25 &   0.03 &   58.1 $\pm$    8.8 &    6.6 &  138.8 $\pm$   40.0 &   0.80 $\pm$   0.30 &     22 &   1.55 & 1.00e+00\\
 &      - & - & - &    0.0 & - &  134.9 $\pm$    7.7 &   0.21 $\pm$   0.03 &     23 &   7.68 & 9.99e-01\\
 &   flat & - & - &   72.0 $\pm$    5.2 &   13.8 &  178.3 $\pm$  100.9 &   1.25 $\pm$   0.49 &     22 &   2.61 & 1.00e+00\\
 &      - & - & - &    0.0 & - &  121.7 $\pm$    6.6 &   0.15 $\pm$   0.02 &     23 &  13.85 & 9.31e-01\\
Abell 1063S &   extr &     24 &   0.60 &  169.6 $\pm$   19.7 &    8.6 &   42.2 $\pm$   17.7 &   1.72 $\pm$   0.27 &     21 &   2.98 & 1.00e+00\\
 &      - & - & - &    0.0 & - &  235.3 $\pm$   13.3 &   0.63 $\pm$   0.06 &     22 &  34.40 & 4.47e-02\\
 &   flat & - & - &  169.6 $\pm$   19.7 &    8.6 &   42.2 $\pm$   17.7 &   1.72 $\pm$   0.27 &     21 &   2.98 & 1.00e+00\\
 &      - & - & - &    0.0 & - &  235.3 $\pm$   13.3 &   0.63 $\pm$   0.06 &     22 &  34.40 & 4.47e-02\\
Abell 1068 &   extr &     17 &   0.20 &    9.0 $\pm$    1.0 &    8.7 &  108.9 $\pm$    3.2 &   1.31 $\pm$   0.06 &     14 &   3.45 & 9.98e-01\\
 &      - & - & - &    0.0 & - &  116.5 $\pm$    3.0 &   0.96 $\pm$   0.03 &     15 &  53.28 & 3.46e-06\\
 &   flat & - & - &    9.1 $\pm$    1.0 &    8.8 &  108.8 $\pm$    3.2 &   1.31 $\pm$   0.06 &     14 &   3.44 & 9.98e-01\\
 &      - & - & - &    0.0 & - &  116.5 $\pm$    3.0 &   0.96 $\pm$   0.03 &     15 &  54.19 & 2.44e-06\\
Abell 1201 &   extr &     14 &   0.20 &   39.2 $\pm$   14.0 &    2.8 &  200.4 $\pm$   23.8 &   1.20 $\pm$   0.21 &     11 &   1.60 & 1.00e+00\\
 &      - & - & - &    0.0 & - &  245.2 $\pm$   15.1 &   0.81 $\pm$   0.08 &     12 &   6.57 & 8.85e-01\\
 &   flat & - & - &   64.8 $\pm$   16.9 &    3.8 &  198.9 $\pm$   25.2 &   1.03 $\pm$   0.21 &     11 &   2.19 & 9.98e-01\\
 &      - & - & - &    0.0 & - &  262.1 $\pm$   15.3 &   0.56 $\pm$   0.05 &     12 &   8.39 & 7.54e-01\\
Abell 1204 &   extr &     11 &   0.15 &   14.1 $\pm$    1.5 &    9.5 &   83.1 $\pm$    3.6 &   1.35 $\pm$   0.11 &      8 &   1.62 & 9.91e-01\\
 &      - & - & - &    0.0 & - &   87.9 $\pm$    3.2 &   0.75 $\pm$   0.03 &      9 &  54.35 & 1.62e-08\\
 &   flat & - & - &   15.3 $\pm$    1.4 &   10.8 &   81.8 $\pm$    3.6 &   1.40 $\pm$   0.11 &      8 &   1.91 & 9.84e-01\\
 &      - & - & - &    0.0 & - &   86.7 $\pm$    3.2 &   0.73 $\pm$   0.03 &      9 &  65.62 & 1.09e-10\\
Abell 1240 &   extr &     37 &   0.50 &  429.4 $\pm$   46.9 &    9.1 &   16.9 $\pm$   28.8 &   1.96 $\pm$   1.14 &     34 &   0.06 & 1.00e+00\\
 &      - & - & - &    0.0 & - &  482.7 $\pm$   27.4 &   0.17 $\pm$   0.06 &     35 &   4.78 & 1.00e+00\\
 &   flat & - & - &  462.4 $\pm$   41.7 &   11.1 &    8.3 $\pm$   18.2 &   2.37 $\pm$   1.48 &     34 &   0.03 & 1.00e+00\\
 &      - & - & - &    0.0 & - &  504.2 $\pm$   26.9 &   0.13 $\pm$   0.05 &     35 &   4.76 & 1.00e+00\\
Abell 1361 &   extr &     14 &   0.15 &   14.8 $\pm$    4.3 &    3.4 &  119.2 $\pm$   10.7 &   1.15 $\pm$   0.19 &     11 &   3.47 & 9.83e-01\\
 &      - & - & - &    0.0 & - &  121.7 $\pm$    9.4 &   0.74 $\pm$   0.06 &     12 &  12.04 & 4.43e-01\\
 &   flat & - & - &   18.6 $\pm$    4.9 &    3.8 &  117.9 $\pm$   10.5 &   1.06 $\pm$   0.18 &     11 &   4.08 & 9.68e-01\\
 &      - & - & - &    0.0 & - &  122.2 $\pm$    8.9 &   0.63 $\pm$   0.05 &     12 &  13.17 & 3.57e-01\\
Abell 1413 &   extr &     10 &   0.12 &   29.8 $\pm$   13.9 &    2.1 &  158.2 $\pm$   14.7 &   0.82 $\pm$   0.20 &      7 &   5.97 & 5.43e-01\\
 &      - & - & - &    0.0 & - &  179.6 $\pm$   10.0 &   0.54 $\pm$   0.05 &      8 &  11.45 & 1.77e-01\\
 &   flat & - & - &   64.0 $\pm$    8.3 &    7.7 &  123.2 $\pm$   13.0 &   1.19 $\pm$   0.28 &      7 &   6.18 & 5.19e-01\\
 &      - & - & - &    0.0 & - &  164.1 $\pm$    9.2 &   0.38 $\pm$   0.04 &      8 &  25.44 & 1.31e-03\\
Abell 1423 &   extr &     23 &   0.40 &   58.8 $\pm$   12.6 &    4.7 &  124.8 $\pm$   20.9 &   1.22 $\pm$   0.17 &     20 &   1.75 & 1.00e+00\\
 &      - & - & - &    0.0 & - &  205.5 $\pm$    9.7 &   0.73 $\pm$   0.06 &     21 &  15.66 & 7.88e-01\\
 &   flat & - & - &   68.3 $\pm$   12.9 &    5.3 &  124.2 $\pm$   21.1 &   1.20 $\pm$   0.17 &     20 &   1.67 & 1.00e+00\\
 &      - & - & - &    0.0 & - &  215.6 $\pm$    9.7 &   0.65 $\pm$   0.05 &     21 &  17.39 & 6.87e-01\\
Abell 1446 &   extr &     34 &   0.32 &  152.4 $\pm$   43.8 &    3.5 &  119.5 $\pm$   49.5 &   0.67 $\pm$   0.27 &     31 &   6.87 & 1.00e+00\\
 &      - & - & - &    0.0 & - &  282.4 $\pm$    8.4 &   0.26 $\pm$   0.04 &     32 &   9.71 & 1.00e+00\\
 &   flat & - & - &  152.4 $\pm$   43.8 &    3.5 &  119.5 $\pm$   49.5 &   0.67 $\pm$   0.27 &     31 &   6.87 & 1.00e+00\\
 &      - & - & - &    0.0 & - &  282.4 $\pm$    8.4 &   0.26 $\pm$   0.04 &     32 &   9.71 & 1.00e+00\\
Abell 1569 &   extr &     29 &   0.20 &  110.1 $\pm$   27.8 &    4.0 &  149.1 $\pm$   28.9 &   0.51 $\pm$   0.19 &     26 &   7.39 & 1.00e+00\\
 &      - & - & - &    0.0 & - &  253.7 $\pm$    9.5 &   0.20 $\pm$   0.02 &     27 &   9.59 & 9.99e-01\\
 &   flat & - & - &  110.1 $\pm$   27.8 &    4.0 &  149.1 $\pm$   28.9 &   0.51 $\pm$   0.19 &     26 &   7.39 & 1.00e+00\\
 &      - & - & - &    0.0 & - &  253.7 $\pm$    9.5 &   0.20 $\pm$   0.02 &     27 &   9.59 & 9.99e-01\\
Abell 1576 &   extr &     33 &   0.70 &  174.1 $\pm$   49.7 &    3.5 &  102.3 $\pm$   48.5 &   1.36 $\pm$   0.29 &     30 &  41.88 & 7.32e-02\\
 &      - & - & - &    0.0 & - &  286.9 $\pm$   27.0 &   0.77 $\pm$   0.09 &     31 & 250.93 & 2.94e-36\\
 &   flat & - & - &  186.2 $\pm$   49.1 &    3.8 &   98.3 $\pm$   47.6 &   1.38 $\pm$   0.29 &     30 &  41.62 & 7.71e-02\\
 &      - & - & - &    0.0 & - &  297.3 $\pm$   26.9 &   0.74 $\pm$   0.09 &     31 & 272.38 & 2.10e-40\\
Abell 1644 &   extr &     11 &   0.05 &   10.7 $\pm$    1.3 &    8.2 &  511.4 $\pm$   61.2 &   1.54 $\pm$   0.10 &      8 &   0.50 & 1.00e+00\\
 &      - & - & - &    0.0 & - &  293.9 $\pm$   22.4 &   1.02 $\pm$   0.04 &      9 &  43.93 & 1.45e-06\\
 &   flat & - & - &   19.0 $\pm$    1.2 &   16.4 &  585.7 $\pm$   81.8 &   1.76 $\pm$   0.11 &      8 &   1.25 & 9.96e-01\\
 &      - & - & - &    0.0 & - &  177.6 $\pm$   12.5 &   0.71 $\pm$   0.03 &      9 & 108.10 & 3.58e-19\\
Abell 1650 &   extr &     15 &   0.12 &   32.7 $\pm$   10.8 &    3.0 &  164.9 $\pm$   12.3 &   0.80 $\pm$   0.16 &     12 &   1.85 & 1.00e+00\\
 &      - & - & - &    0.0 & - &  185.9 $\pm$    9.1 &   0.49 $\pm$   0.04 &     13 &   6.09 & 9.43e-01\\
 &   flat & - & - &   38.0 $\pm$   10.0 &    3.8 &  159.9 $\pm$   12.1 &   0.84 $\pm$   0.17 &     12 &   2.00 & 9.99e-01\\
 &      - & - & - &    0.0 & - &  183.7 $\pm$    9.0 &   0.47 $\pm$   0.04 &     13 &   7.85 & 8.53e-01\\
Abell 1651 &   extr &     27 &   0.20 &   87.7 $\pm$   11.2 &    7.8 &  117.3 $\pm$   15.3 &   0.96 $\pm$   0.18 &     24 &  13.05 & 9.65e-01\\
 &      - & - & - &    0.0 & - &  207.6 $\pm$    6.7 &   0.34 $\pm$   0.03 &     25 &  28.85 & 2.70e-01\\
 &   flat & - & - &   89.5 $\pm$   11.1 &    8.1 &  115.5 $\pm$   15.2 &   0.97 $\pm$   0.19 &     24 &  13.26 & 9.62e-01\\
 &      - & - & - &    0.0 & - &  207.6 $\pm$    6.7 &   0.34 $\pm$   0.03 &     25 &  29.42 & 2.47e-01\\
Abell 1664 &   extr &     13 &   0.15 &   10.0 $\pm$    1.1 &    9.1 &  142.7 $\pm$    5.9 &   1.50 $\pm$   0.08 &     10 &  27.58 & 2.11e-03\\
 &      - & - & - &    0.0 & - &  127.9 $\pm$    4.9 &   0.97 $\pm$   0.03 &     11 &  82.78 & 4.27e-13\\
 &   flat & - & - &   14.4 $\pm$    1.0 &   14.8 &  141.8 $\pm$    6.1 &   1.70 $\pm$   0.09 &     10 &  16.24 & 9.31e-02\\
 &      - & - & - &    0.0 & - &  117.2 $\pm$    4.6 &   0.85 $\pm$   0.03 &     11 & 127.13 & 6.63e-22\\
Abell 1689 &   extr &     20 &   0.30 &   78.4 $\pm$    7.6 &   10.4 &  111.8 $\pm$   13.8 &   1.35 $\pm$   0.14 &     17 &   7.34 & 9.79e-01\\
 &      - & - & - &    0.0 & - &  218.8 $\pm$    6.3 &   0.62 $\pm$   0.03 &     18 &  52.72 & 2.90e-05\\
 &   flat & - & - &   78.4 $\pm$    7.6 &   10.4 &  111.8 $\pm$   13.8 &   1.35 $\pm$   0.14 &     17 &   7.34 & 9.79e-01\\
 &      - & - & - &    0.0 & - &  218.8 $\pm$    6.3 &   0.62 $\pm$   0.03 &     18 &  52.72 & 2.90e-05\\
Abell 1736 &   extr &     15 &   0.10 &  150.4 $\pm$   38.3 &    3.9 &  127.3 $\pm$   37.9 &   0.99 $\pm$   0.83 &     12 &   0.10 & 1.00e+00\\
 &      - & - & - &    0.0 & - &  251.9 $\pm$   19.2 &   0.20 $\pm$   0.06 &     13 &   1.58 & 1.00e+00\\
 &   flat & - & - &  150.4 $\pm$   38.3 &    3.9 &  127.3 $\pm$   37.9 &   0.99 $\pm$   0.83 &     12 &   0.10 & 1.00e+00\\
 &      - & - & - &    0.0 & - &  251.9 $\pm$   19.2 &   0.20 $\pm$   0.06 &     13 &   1.58 & 1.00e+00\\
Abell 1758 &   extr &     20 &   0.40 &  116.8 $\pm$   44.3 &    2.6 &  218.0 $\pm$   58.6 &   1.03 $\pm$   0.24 &     17 &   0.61 & 1.00e+00\\
 &      - & - & - &    0.0 & - &  361.7 $\pm$   20.8 &   0.62 $\pm$   0.08 &     18 &   4.61 & 9.99e-01\\
 &   flat & - & - &  230.8 $\pm$   37.2 &    6.2 &  144.0 $\pm$   50.2 &   1.21 $\pm$   0.32 &     17 &   1.98 & 1.00e+00\\
 &      - & - & - &    0.0 & - &  417.8 $\pm$   20.2 &   0.36 $\pm$   0.06 &     18 &   9.94 & 9.34e-01\\
Abell 1763 &   extr &     39 &   0.60 &  214.7 $\pm$   32.8 &    6.5 &   70.8 $\pm$   29.1 &   1.37 $\pm$   0.25 &     36 &   2.87 & 1.00e+00\\
 &      - & - & - &    0.0 & - &  288.8 $\pm$   13.8 &   0.60 $\pm$   0.05 &     37 &  18.21 & 9.96e-01\\
 &   flat & - & - &  214.7 $\pm$   32.8 &    6.5 &   70.8 $\pm$   29.1 &   1.37 $\pm$   0.25 &     36 &   2.87 & 1.00e+00\\
 &      - & - & - &    0.0 & - &  288.8 $\pm$   13.8 &   0.60 $\pm$   0.05 &     37 &  18.21 & 9.96e-01\\
Abell 1795 &   extr &     53 &   0.30 &   18.4 $\pm$    1.1 &   17.4 &  131.4 $\pm$    2.8 &   1.17 $\pm$   0.03 &     50 &  33.33 & 9.66e-01\\
 &      - & - & - &    0.0 & - &  158.9 $\pm$    2.0 &   0.86 $\pm$   0.01 &     51 & 271.73 & 7.10e-32\\
 &   flat & - & - &   19.0 $\pm$    1.1 &   18.1 &  130.4 $\pm$    2.8 &   1.18 $\pm$   0.03 &     50 &  35.74 & 9.36e-01\\
 &      - & - & - &    0.0 & - &  158.8 $\pm$    2.0 &   0.86 $\pm$   0.01 &     51 & 292.75 & 1.18e-35\\
Abell 1835 &   extr &     16 &   0.30 &   10.9 $\pm$    2.5 &    4.4 &  112.6 $\pm$    7.9 &   1.25 $\pm$   0.09 &     13 &   8.46 & 8.12e-01\\
 &      - & - & - &    0.0 & - &  134.2 $\pm$    5.2 &   0.99 $\pm$   0.03 &     14 &  26.28 & 2.38e-02\\
 &   flat & - & - &   11.4 $\pm$    2.5 &    4.6 &  111.7 $\pm$    7.9 &   1.26 $\pm$   0.09 &     13 &   8.76 & 7.91e-01\\
 &      - & - & - &    0.0 & - &  134.3 $\pm$    5.3 &   0.98 $\pm$   0.03 &     14 &  28.26 & 1.32e-02\\
Abell 1914 &   extr &     29 &   0.40 &   63.3 $\pm$   22.3 &    2.8 &  175.5 $\pm$   32.3 &   0.88 $\pm$   0.14 &     26 &   3.91 & 1.00e+00\\
 &      - & - & - &    0.0 & - &  256.7 $\pm$   10.4 &   0.61 $\pm$   0.04 &     27 &   9.94 & 9.99e-01\\
 &   flat & - & - &  107.2 $\pm$   18.0 &    5.9 &  131.1 $\pm$   28.3 &   1.05 $\pm$   0.18 &     26 &   4.42 & 1.00e+00\\
 &      - & - & - &    0.0 & - &  269.8 $\pm$   10.3 &   0.52 $\pm$   0.04 &     27 &  21.84 & 7.45e-01\\
Abell 1942 &   extr &     12 &   0.22 &  107.7 $\pm$   77.7 &    1.4 &  194.1 $\pm$   88.7 &   0.66 $\pm$   0.41 &      9 &   1.21 & 9.99e-01\\
 &      - & - & - &    0.0 & - &  307.8 $\pm$   17.3 &   0.35 $\pm$   0.07 &     10 &   1.81 & 9.98e-01\\
 &   flat & - & - &  107.7 $\pm$   77.7 &    1.4 &  194.1 $\pm$   88.7 &   0.66 $\pm$   0.41 &      9 &   1.21 & 9.99e-01\\
 &      - & - & - &    0.0 & - &  307.8 $\pm$   17.3 &   0.35 $\pm$   0.07 &     10 &   1.81 & 9.98e-01\\
Abell 1991 &   extr &     19 &   0.10 &    1.0 $\pm$    0.3 &    3.0 &  151.4 $\pm$    4.1 &   1.04 $\pm$   0.03 &     16 &  31.46 & 1.18e-02\\
 &      - & - & - &    0.0 & - &  151.3 $\pm$    3.6 &   1.04 $\pm$   0.01 &     17 &  31.47 & 1.75e-02\\
 &   flat & - & - &    1.5 $\pm$    0.3 &    4.8 &  152.2 $\pm$    4.2 &   1.09 $\pm$   0.03 &     16 &  43.79 & 2.12e-04\\
 &      - & - & - &    0.0 & - &  143.7 $\pm$    3.4 &   0.99 $\pm$   0.01 &     17 &  64.00 & 2.26e-07\\
Abell 1995 &   extr &     26 &   0.60 &  374.3 $\pm$   60.1 &    6.2 &   26.8 $\pm$   32.9 &   2.08 $\pm$   0.81 &     23 &   0.99 & 1.00e+00\\
 &      - & - & - &    0.0 & - &  421.2 $\pm$   36.4 &   0.35 $\pm$   0.11 &     24 &   9.74 & 9.96e-01\\
 &   flat & - & - &  374.3 $\pm$   60.1 &    6.2 &   26.8 $\pm$   32.9 &   2.08 $\pm$   0.81 &     23 &   0.99 & 1.00e+00\\
 &      - & - & - &    0.0 & - &  421.2 $\pm$   36.4 &   0.35 $\pm$   0.11 &     24 &   9.74 & 9.96e-01\\
Abell 2029 &   extr &     58 &   0.40 &    6.1 $\pm$    0.7 &    8.7 &  169.9 $\pm$    2.1 &   0.92 $\pm$   0.01 &     55 &  82.78 & 9.09e-03\\
 &      - & - & - &    0.0 & - &  181.2 $\pm$    1.6 &   0.82 $\pm$   0.01 &     56 & 146.10 & 5.63e-10\\
 &   flat & - & - &   10.5 $\pm$    0.7 &   15.8 &  163.6 $\pm$    2.1 &   0.95 $\pm$   0.02 &     55 &  58.95 & 3.33e-01\\
 &      - & - & - &    0.0 & - &  182.6 $\pm$    1.6 &   0.78 $\pm$   0.01 &     56 & 235.51 & 7.10e-24\\
Abell 2034 &   extr &     67 &   0.50 &  215.8 $\pm$   25.1 &    8.6 &   99.1 $\pm$   25.3 &   1.05 $\pm$   0.16 &     64 &  11.63 & 1.00e+00\\
 &      - & - & - &    0.0 & - &  333.4 $\pm$    9.0 &   0.42 $\pm$   0.03 &     65 &  31.58 & 1.00e+00\\
 &   flat & - & - &  232.6 $\pm$   23.0 &   10.1 &   85.1 $\pm$   22.6 &   1.14 $\pm$   0.17 &     64 &  10.87 & 1.00e+00\\
 &      - & - & - &    0.0 & - &  338.1 $\pm$    8.9 &   0.41 $\pm$   0.03 &     65 &  35.48 & 9.99e-01\\
Abell 2052 &   extr &     29 &   0.10 &    8.9 $\pm$    0.7 &   13.2 &  164.8 $\pm$    2.6 &   1.23 $\pm$   0.03 &     26 & 374.86 & 1.67e-63\\
 &      - & - & - &    0.0 & - &  162.4 $\pm$    2.3 &   0.99 $\pm$   0.01 &     27 & 541.69 & 3.71e-97\\
 &   flat & - & - &    9.5 $\pm$    0.7 &   14.3 &  164.7 $\pm$    2.6 &   1.25 $\pm$   0.03 &     26 & 387.05 & 5.51e-66\\
 &      - & - & - &    0.0 & - &  162.1 $\pm$    2.3 &   0.99 $\pm$   0.01 &     27 & 580.67 & 3.03e-105\\
Abell 2063 &   extr &     52 &   0.18 &   53.5 $\pm$    2.6 &   20.6 &  129.0 $\pm$    3.9 &   1.07 $\pm$   0.05 &     49 &  37.82 & 8.77e-01\\
 &      - & - & - &    0.0 & - &  180.6 $\pm$    2.4 &   0.51 $\pm$   0.01 &     50 & 224.14 & 6.72e-24\\
 &   flat & - & - &   53.5 $\pm$    2.6 &   20.6 &  129.0 $\pm$    3.9 &   1.07 $\pm$   0.05 &     49 &  37.82 & 8.77e-01\\
 &      - & - & - &    0.0 & - &  180.6 $\pm$    2.4 &   0.51 $\pm$   0.01 &     50 & 224.14 & 6.72e-24\\
Abell 2065 &   extr &     29 &   0.20 &   33.1 $\pm$    6.9 &    4.8 &  206.9 $\pm$   10.8 &   0.97 $\pm$   0.09 &     26 &   7.99 & 1.00e+00\\
 &      - & - & - &    0.0 & - &  239.0 $\pm$    7.5 &   0.67 $\pm$   0.03 &     27 &  21.36 & 7.69e-01\\
 &   flat & - & - &   43.9 $\pm$    6.5 &    6.8 &  195.3 $\pm$   10.6 &   1.02 $\pm$   0.10 &     26 &   7.97 & 1.00e+00\\
 &      - & - & - &    0.0 & - &  236.5 $\pm$    7.5 &   0.60 $\pm$   0.03 &     27 &  29.46 & 3.39e-01\\
Abell 2069 &   extr &     39 &   0.40 &  416.2 $\pm$   41.8 &   10.0 &   82.4 $\pm$   46.0 &   1.22 $\pm$   0.41 &     36 &   5.75 & 1.00e+00\\
 &      - & - & - &    0.0 & - &  544.7 $\pm$   16.4 &   0.20 $\pm$   0.03 &     37 &  15.09 & 1.00e+00\\
 &   flat & - & - &  453.2 $\pm$   35.6 &   12.7 &   54.6 $\pm$   36.3 &   1.47 $\pm$   0.51 &     36 &   5.71 & 1.00e+00\\
 &      - & - & - &    0.0 & - &  557.2 $\pm$   16.2 &   0.17 $\pm$   0.03 &     37 &  16.52 & 9.99e-01\\
Abell 2104 &   extr &      9 &   0.12 &   98.0 $\pm$   57.6 &    1.7 &  276.2 $\pm$   59.7 &   0.94 $\pm$   0.55 &      6 &   0.64 & 9.96e-01\\
 &      - & - & - &    0.0 & - &  350.0 $\pm$   36.1 &   0.46 $\pm$   0.10 &      7 &   2.22 & 9.47e-01\\
 &   flat & - & - &  160.6 $\pm$   42.2 &    3.8 &  210.1 $\pm$   53.9 &   1.20 $\pm$   0.77 &      6 &   0.74 & 9.94e-01\\
 &      - & - & - &    0.0 & - &  331.9 $\pm$   33.4 &   0.30 $\pm$   0.08 &      7 &   3.39 & 8.47e-01\\
Abell 2107 &   extr &      6 &   0.03 &   18.0 $\pm$    4.7 &    3.8 &  473.9 $\pm$  117.3 &   1.03 $\pm$   0.16 &      3 &  13.10 & 4.42e-03\\
 &      - & - & - &    0.0 & - &  290.4 $\pm$   26.6 &   0.64 $\pm$   0.04 &      4 &  40.08 & 4.17e-08\\
 &   flat & - & - &   21.2 $\pm$    5.8 &    3.6 &  396.1 $\pm$   92.5 &   0.91 $\pm$   0.16 &      3 &  15.79 & 1.25e-03\\
 &      - & - & - &    0.0 & - &  263.6 $\pm$   21.3 &   0.55 $\pm$   0.03 &      4 &  43.05 & 1.01e-08\\
Abell 2111 &   extr &     22 &   0.40 &  107.4 $\pm$   97.3 &    1.1 &  194.0 $\pm$  118.7 &   0.65 $\pm$   0.38 &     19 &   1.06 & 1.00e+00\\
 &      - & - & - &    0.0 & - &  317.5 $\pm$   23.7 &   0.39 $\pm$   0.08 &     20 &   1.54 & 1.00e+00\\
 &   flat & - & - &  107.4 $\pm$   97.3 &    1.1 &  194.0 $\pm$  118.7 &   0.65 $\pm$   0.38 &     19 &   1.06 & 1.00e+00\\
 &      - & - & - &    0.0 & - &  317.5 $\pm$   23.7 &   0.39 $\pm$   0.08 &     20 &   1.54 & 1.00e+00\\
Abell 2124 &   extr &     19 &   0.12 &   88.7 $\pm$   24.2 &    3.7 &  272.5 $\pm$   30.8 &   0.89 $\pm$   0.27 &     16 &   2.86 & 1.00e+00\\
 &      - & - & - &    0.0 & - &  325.0 $\pm$   21.8 &   0.41 $\pm$   0.05 &     17 &   7.20 & 9.81e-01\\
 &   flat & - & - &   98.3 $\pm$   23.9 &    4.1 &  260.8 $\pm$   30.8 &   0.90 $\pm$   0.28 &     16 &   3.24 & 1.00e+00\\
 &      - & - & - &    0.0 & - &  320.8 $\pm$   21.3 &   0.37 $\pm$   0.05 &     17 &   7.78 & 9.71e-01\\
Abell 2125 &   extr &     10 &   0.20 &  225.2 $\pm$   32.0 &    7.0 &   32.9 $\pm$   41.2 &   1.35 $\pm$   1.73 &      7 &   0.06 & 1.00e+00\\
 &      - & - & - &    0.0 & - &  264.5 $\pm$   11.5 &   0.10 $\pm$   0.05 &      8 &   1.06 & 9.98e-01\\
 &   flat & - & - &  225.2 $\pm$   32.0 &    7.0 &   32.9 $\pm$   41.2 &   1.35 $\pm$   1.73 &      7 &   0.06 & 1.00e+00\\
 &      - & - & - &    0.0 & - &  264.5 $\pm$   11.5 &   0.10 $\pm$   0.05 &      8 &   1.06 & 9.98e-01\\
Abell 2142 &   extr &     75 &   0.30 &   58.5 $\pm$    2.7 &   21.7 &  132.5 $\pm$    4.5 &   1.13 $\pm$   0.04 &     72 &  17.26 & 1.00e+00\\
 &      - & - & - &    0.0 & - &  205.9 $\pm$    2.1 &   0.62 $\pm$   0.01 &     73 & 240.81 & 8.51e-20\\
 &   flat & - & - &   68.1 $\pm$    2.5 &   27.5 &  120.6 $\pm$    4.4 &   1.22 $\pm$   0.04 &     72 &  17.98 & 1.00e+00\\
 &      - & - & - &    0.0 & - &  206.1 $\pm$    2.2 &   0.58 $\pm$   0.01 &     73 & 335.00 & 3.31e-35\\
Abell 2147 &   extr &     57 &   0.20 &  151.9 $\pm$   27.2 &    5.6 &  136.2 $\pm$   30.5 &   0.55 $\pm$   0.19 &     54 &  31.13 & 9.95e-01\\
 &      - & - & - &    0.0 & - &  291.4 $\pm$    6.4 &   0.18 $\pm$   0.02 &     55 &  35.26 & 9.82e-01\\
 &   flat & - & - &  151.9 $\pm$   27.2 &    5.6 &  136.2 $\pm$   30.5 &   0.55 $\pm$   0.19 &     54 &  31.13 & 9.95e-01\\
 &      - & - & - &    0.0 & - &  291.4 $\pm$    6.4 &   0.18 $\pm$   0.02 &     55 &  35.26 & 9.82e-01\\
Abell 2151 &   extr &     20 &   0.07 &    1.7 $\pm$    3.0 &    0.6 &  137.9 $\pm$    6.0 &   0.61 $\pm$   0.06 &     17 &  36.84 & 3.54e-03\\
 &      - & - & - &    0.0 & - &  136.6 $\pm$    5.2 &   0.58 $\pm$   0.02 &     18 &  37.11 & 5.07e-03\\
 &   flat & - & - &    0.4 $\pm$    3.6 &    0.1 &  135.2 $\pm$    5.4 &   0.56 $\pm$   0.06 &     17 &  36.91 & 3.46e-03\\
 &      - & - & - &    0.0 & - &  135.0 $\pm$    5.0 &   0.55 $\pm$   0.02 &     18 &  36.92 & 5.37e-03\\
Abell 2163 &   extr &     42 &   0.60 &  437.3 $\pm$   82.7 &    5.3 &   72.5 $\pm$   50.8 &   1.86 $\pm$   0.43 &     39 &   7.08 & 1.00e+00\\
 &      - & - & - &    0.0 & - &  449.2 $\pm$   42.9 &   0.82 $\pm$   0.09 &     40 &  20.09 & 9.96e-01\\
 &   flat & - & - &  438.0 $\pm$   82.6 &    5.3 &   72.2 $\pm$   50.6 &   1.87 $\pm$   0.43 &     39 &   7.08 & 1.00e+00\\
 &      - & - & - &    0.0 & - &  449.3 $\pm$   42.9 &   0.82 $\pm$   0.09 &     40 &  20.17 & 9.96e-01\\
Abell 2199 &   extr &      7 &   0.02 &    7.6 $\pm$    0.8 &    9.1 &  423.7 $\pm$   95.3 &   1.38 $\pm$   0.12 &      4 &   3.72 & 4.45e-01\\
 &      - & - & - &    0.0 & - &  143.3 $\pm$   11.8 &   0.72 $\pm$   0.03 &      5 &  35.07 & 1.46e-06\\
 &   flat & - & - &   13.3 $\pm$    0.8 &   15.6 &  331.5 $\pm$   90.0 &   1.35 $\pm$   0.15 &      4 &  11.09 & 2.56e-02\\
 &      - & - & - &    0.0 & - &   81.8 $\pm$    5.2 &   0.44 $\pm$   0.02 &      5 &  45.17 & 1.34e-08\\
Abell 2204 &   extr &     15 &   0.20 &    9.7 $\pm$    0.9 &   11.1 &  166.2 $\pm$    6.0 &   1.41 $\pm$   0.05 &     12 &  22.73 & 3.01e-02\\
 &      - & - & - &    0.0 & - &  164.6 $\pm$    5.9 &   1.02 $\pm$   0.02 &     13 & 102.32 & 5.88e-16\\
 &   flat & - & - &    9.7 $\pm$    0.9 &   11.1 &  166.2 $\pm$    6.0 &   1.41 $\pm$   0.05 &     12 &  22.73 & 3.01e-02\\
 &      - & - & - &    0.0 & - &  164.6 $\pm$    5.9 &   1.02 $\pm$   0.02 &     13 & 102.32 & 5.88e-16\\
Abell 2218 &   extr &     42 &   0.60 &  288.6 $\pm$   20.0 &   14.4 &   10.7 $\pm$    7.1 &   2.35 $\pm$   0.41 &     39 &   4.83 & 1.00e+00\\
 &      - & - & - &    0.0 & - &  294.5 $\pm$   14.7 &   0.41 $\pm$   0.05 &     40 &  39.78 & 4.80e-01\\
 &   flat & - & - &  288.6 $\pm$   20.0 &   14.4 &   10.7 $\pm$    7.1 &   2.35 $\pm$   0.41 &     39 &   4.83 & 1.00e+00\\
 &      - & - & - &    0.0 & - &  294.5 $\pm$   14.7 &   0.41 $\pm$   0.05 &     40 &  39.78 & 4.80e-01\\
Abell 2219 &   extr &     34 &   0.60 &  411.6 $\pm$   43.2 &    9.5 &   17.0 $\pm$   19.2 &   1.97 $\pm$   0.66 &     31 &   3.70 & 1.00e+00\\
 &      - & - & - &    0.0 & - &  407.6 $\pm$   26.4 &   0.36 $\pm$   0.06 &     32 &  19.62 & 9.58e-01\\
 &   flat & - & - &  411.6 $\pm$   43.2 &    9.5 &   17.0 $\pm$   19.2 &   1.97 $\pm$   0.66 &     31 &   3.70 & 1.00e+00\\
 &      - & - & - &    0.0 & - &  407.6 $\pm$   26.4 &   0.36 $\pm$   0.06 &     32 &  19.62 & 9.58e-01\\
Abell 2244 &   extr &     34 &   0.30 &   57.6 $\pm$    4.2 &   13.6 &  109.1 $\pm$    6.0 &   1.00 $\pm$   0.05 &     31 &  14.02 & 9.96e-01\\
 &      - & - & - &    0.0 & - &  180.0 $\pm$    2.1 &   0.56 $\pm$   0.02 &     32 & 102.67 & 2.46e-09\\
 &   flat & - & - &   57.6 $\pm$    4.2 &   13.6 &  109.1 $\pm$    6.0 &   1.00 $\pm$   0.05 &     31 &  14.02 & 9.96e-01\\
 &      - & - & - &    0.0 & - &  180.0 $\pm$    2.1 &   0.56 $\pm$   0.02 &     32 & 102.67 & 2.46e-09\\
Abell 2255 &   extr &     40 &   0.30 &  529.1 $\pm$   28.2 &   18.8 &    5.8 $\pm$   16.6 &   2.63 $\pm$   2.69 &     37 &   0.24 & 1.00e+00\\
 &      - & - & - &    0.0 & - &  553.0 $\pm$   14.0 &   0.05 $\pm$   0.03 &     38 &   2.79 & 1.00e+00\\
 &   flat & - & - &  529.1 $\pm$   28.2 &   18.8 &    5.8 $\pm$   16.6 &   2.63 $\pm$   2.69 &     37 &   0.24 & 1.00e+00\\
 &      - & - & - &    0.0 & - &  553.0 $\pm$   14.0 &   0.05 $\pm$   0.03 &     38 &   2.79 & 1.00e+00\\
Abell 2256 &   extr &     63 &   0.35 &  349.6 $\pm$   11.6 &   30.2 &    7.0 $\pm$    7.6 &   2.54 $\pm$   0.93 &     60 &   2.24 & 1.00e+00\\
 &      - & - & - &    0.0 & - &  378.4 $\pm$    6.9 &   0.08 $\pm$   0.02 &     61 &  21.60 & 1.00e+00\\
 &   flat & - & - &  349.6 $\pm$   11.6 &   30.2 &    7.0 $\pm$    7.6 &   2.54 $\pm$   0.93 &     60 &   2.24 & 1.00e+00\\
 &      - & - & - &    0.0 & - &  378.4 $\pm$    6.9 &   0.08 $\pm$   0.02 &     61 &  21.60 & 1.00e+00\\
Abell 2259 &   extr &     36 &   0.50 &  114.0 $\pm$   18.9 &    6.0 &   61.0 $\pm$   20.4 &   1.36 $\pm$   0.24 &     33 &   1.37 & 1.00e+00\\
 &      - & - & - &    0.0 & - &  189.0 $\pm$    8.7 &   0.63 $\pm$   0.05 &     34 &  15.77 & 9.97e-01\\
 &   flat & - & - &  114.0 $\pm$   18.9 &    6.0 &   61.0 $\pm$   20.4 &   1.36 $\pm$   0.24 &     33 &   1.37 & 1.00e+00\\
 &      - & - & - &    0.0 & - &  189.0 $\pm$    8.7 &   0.63 $\pm$   0.05 &     34 &  15.77 & 9.97e-01\\
Abell 2261 &   extr &     18 &   0.30 &   60.5 $\pm$    8.2 &    7.4 &  106.5 $\pm$   14.1 &   1.27 $\pm$   0.16 &     15 &   3.63 & 9.99e-01\\
 &      - & - & - &    0.0 & - &  189.6 $\pm$    6.6 &   0.61 $\pm$   0.04 &     16 &  28.62 & 2.67e-02\\
 &   flat & - & - &   61.1 $\pm$    8.1 &    7.5 &  106.0 $\pm$   14.1 &   1.27 $\pm$   0.16 &     15 &   3.62 & 9.99e-01\\
 &      - & - & - &    0.0 & - &  189.7 $\pm$    6.6 &   0.61 $\pm$   0.04 &     16 &  29.00 & 2.40e-02\\
Abell 2294 &   extr &     22 &   0.32 &  128.5 $\pm$   52.0 &    2.5 &  246.7 $\pm$   75.6 &   1.04 $\pm$   0.32 &     19 &   0.60 & 1.00e+00\\
 &      - & - & - &    0.0 & - &  409.8 $\pm$   28.7 &   0.57 $\pm$   0.09 &     20 &   3.67 & 1.00e+00\\
 &   flat & - & - &  156.3 $\pm$   52.7 &    3.0 &  235.7 $\pm$   76.3 &   1.03 $\pm$   0.33 &     19 &   0.83 & 1.00e+00\\
 &      - & - & - &    0.0 & - &  428.8 $\pm$   28.6 &   0.49 $\pm$   0.08 &     20 &   4.23 & 1.00e+00\\
Abell 2319 &   extr &     74 &   0.40 &  270.2 $\pm$    4.8 &   56.0 &   39.4 $\pm$    7.1 &   1.76 $\pm$   0.15 &     71 &   9.83 & 1.00e+00\\
 &      - & - & - &    0.0 & - &  363.1 $\pm$    4.3 &   0.19 $\pm$   0.01 &     72 & 212.75 & 7.89e-16\\
 &   flat & - & - &  270.2 $\pm$    4.8 &   56.0 &   39.4 $\pm$    7.1 &   1.76 $\pm$   0.15 &     71 &   9.83 & 1.00e+00\\
 &      - & - & - &    0.0 & - &  363.1 $\pm$    4.3 &   0.19 $\pm$   0.01 &     72 & 212.75 & 7.89e-16\\
Abell 2384 &   extr &     23 &   0.20 &   17.9 $\pm$    3.3 &    5.4 &  162.9 $\pm$    7.3 &   1.31 $\pm$   0.09 &     20 &   7.54 & 9.95e-01\\
 &      - & - & - &    0.0 & - &  179.6 $\pm$    6.3 &   0.99 $\pm$   0.04 &     21 &  29.61 & 1.00e-01\\
 &   flat & - & - &   38.5 $\pm$    3.0 &   13.0 &  139.2 $\pm$    7.3 &   1.49 $\pm$   0.11 &     20 &   7.85 & 9.93e-01\\
 &      - & - & - &    0.0 & - &  163.6 $\pm$    6.1 &   0.70 $\pm$   0.03 &     21 &  87.32 & 4.67e-10\\
Abell 2390 &   extr &     11 &   0.20 &   14.7 $\pm$    7.0 &    2.1 &  202.9 $\pm$   15.6 &   1.07 $\pm$   0.15 &      8 &   0.96 & 9.99e-01\\
 &      - & - & - &    0.0 & - &  214.4 $\pm$   13.9 &   0.84 $\pm$   0.05 &      9 &   4.71 & 8.59e-01\\
 &   flat & - & - &   14.7 $\pm$    7.0 &    2.1 &  202.9 $\pm$   15.6 &   1.07 $\pm$   0.15 &      8 &   0.96 & 9.99e-01\\
 &      - & - & - &    0.0 & - &  214.4 $\pm$   13.9 &   0.84 $\pm$   0.05 &      9 &   4.71 & 8.59e-01\\
Abell 2409 &   extr &     16 &   0.20 &   69.6 $\pm$   20.9 &    3.3 &  124.1 $\pm$   27.4 &   0.96 $\pm$   0.32 &     13 &   8.79 & 7.88e-01\\
 &      - & - & - &    0.0 & - &  198.6 $\pm$   10.2 &   0.45 $\pm$   0.06 &     14 &  15.23 & 3.62e-01\\
 &   flat & - & - &   73.8 $\pm$   20.7 &    3.6 &  120.8 $\pm$   27.3 &   0.97 $\pm$   0.33 &     13 &   9.06 & 7.68e-01\\
 &      - & - & - &    0.0 & - &  199.4 $\pm$   10.3 &   0.43 $\pm$   0.06 &     14 &  15.83 & 3.24e-01\\
Abell 2420 &   extr &     64 &   0.50 &  332.6 $\pm$   67.5 &    4.9 &   64.3 $\pm$   62.6 &   1.12 $\pm$   0.58 &     61 &   5.54 & 1.00e+00\\
 &      - & - & - &    0.0 & - &  411.0 $\pm$   22.4 &   0.28 $\pm$   0.06 &     62 &   9.20 & 1.00e+00\\
 &   flat & - & - &  332.6 $\pm$   67.5 &    4.9 &   64.3 $\pm$   62.6 &   1.12 $\pm$   0.58 &     61 &   5.54 & 1.00e+00\\
 &      - & - & - &    0.0 & - &  411.0 $\pm$   22.4 &   0.28 $\pm$   0.06 &     62 &   9.20 & 1.00e+00\\
Abell 2462 &   extr &     58 &   0.40 &  129.7 $\pm$   27.0 &    4.8 &   83.2 $\pm$   31.1 &   0.77 $\pm$   0.24 &     55 &   1.23 & 1.00e+00\\
 &      - & - & - &    0.0 & - &  224.1 $\pm$    6.2 &   0.30 $\pm$   0.03 &     56 &   7.73 & 1.00e+00\\
 &   flat & - & - &  129.7 $\pm$   27.0 &    4.8 &   83.2 $\pm$   31.1 &   0.77 $\pm$   0.24 &     55 &   1.23 & 1.00e+00\\
 &      - & - & - &    0.0 & - &  224.1 $\pm$    6.2 &   0.30 $\pm$   0.03 &     56 &   7.73 & 1.00e+00\\
Abell 2537 &   extr &     14 &   0.30 &  106.7 $\pm$   19.6 &    5.4 &  127.9 $\pm$   29.2 &   1.24 $\pm$   0.26 &     11 &   1.05 & 1.00e+00\\
 &      - & - & - &    0.0 & - &  259.9 $\pm$   11.9 &   0.51 $\pm$   0.06 &     12 &  12.70 & 3.91e-01\\
 &   flat & - & - &  110.4 $\pm$   19.4 &    5.7 &  124.7 $\pm$   29.0 &   1.26 $\pm$   0.27 &     11 &   1.05 & 1.00e+00\\
 &      - & - & - &    0.0 & - &  261.0 $\pm$   11.9 &   0.50 $\pm$   0.06 &     12 &  13.23 & 3.52e-01\\
Abell 2554 &   extr &     30 &   0.30 &  105.1 $\pm$   71.8 &    1.5 &  318.4 $\pm$   86.2 &   0.66 $\pm$   0.21 &     27 &   0.87 & 1.00e+00\\
 &      - & - & - &    0.0 & - &  436.9 $\pm$   18.7 &   0.45 $\pm$   0.05 &     28 &   1.98 & 1.00e+00\\
 &   flat & - & - &  105.1 $\pm$   71.8 &    1.5 &  318.4 $\pm$   86.2 &   0.66 $\pm$   0.21 &     27 &   0.87 & 1.00e+00\\
 &      - & - & - &    0.0 & - &  436.9 $\pm$   18.7 &   0.45 $\pm$   0.05 &     28 &   1.98 & 1.00e+00\\
Abell 2556 &   extr &     17 &   0.13 &   10.6 $\pm$    1.4 &    7.7 &  117.5 $\pm$    3.9 &   1.10 $\pm$   0.06 &     14 &   4.27 & 9.94e-01\\
 &      - & - & - &    0.0 & - &  116.2 $\pm$    3.5 &   0.76 $\pm$   0.02 &     15 &  44.30 & 9.85e-05\\
 &   flat & - & - &   12.4 $\pm$    1.3 &    9.2 &  115.8 $\pm$    4.0 &   1.13 $\pm$   0.07 &     14 &   4.50 & 9.92e-01\\
 &      - & - & - &    0.0 & - &  113.8 $\pm$    3.4 &   0.72 $\pm$   0.02 &     15 &  57.13 & 7.81e-07\\
Abell 2589 &   extr &     25 &   0.10 &   52.0 $\pm$   39.2 &    1.3 &  109.6 $\pm$   34.8 &   0.61 $\pm$   0.51 &     22 &   1.06 & 1.00e+00\\
 &      - & - & - &    0.0 & - &  154.1 $\pm$   13.4 &   0.29 $\pm$   0.07 &     23 &   1.60 & 1.00e+00\\
 &   flat & - & - &   52.0 $\pm$   39.2 &    1.3 &  109.6 $\pm$   34.8 &   0.61 $\pm$   0.51 &     22 &   1.06 & 1.00e+00\\
 &      - & - & - &    0.0 & - &  154.1 $\pm$   13.4 &   0.29 $\pm$   0.07 &     23 &   1.60 & 1.00e+00\\
Abell 2597 &   extr &      8 &   0.06 &    9.6 $\pm$    1.6 &    5.9 &   96.1 $\pm$   14.0 &   1.19 $\pm$   0.18 &      5 &   4.09 & 5.37e-01\\
 &      - & - & - &    0.0 & - &   70.7 $\pm$    5.0 &   0.62 $\pm$   0.04 &      6 &  23.75 & 5.81e-04\\
 &   flat & - & - &   10.6 $\pm$    1.5 &    7.0 &   98.9 $\pm$   15.2 &   1.26 $\pm$   0.19 &      5 &   4.10 & 5.35e-01\\
 &      - & - & - &    0.0 & - &   68.5 $\pm$    4.8 &   0.59 $\pm$   0.04 &      6 &  28.70 & 6.94e-05\\
Abell 2626 &   extr &     22 &   0.12 &   23.2 $\pm$    2.9 &    8.1 &  144.1 $\pm$    6.3 &   1.05 $\pm$   0.09 &     19 &  11.88 & 8.91e-01\\
 &      - & - & - &    0.0 & - &  147.7 $\pm$    5.2 &   0.62 $\pm$   0.03 &     20 &  46.28 & 7.38e-04\\
 &   flat & - & - &   23.2 $\pm$    2.9 &    8.1 &  144.1 $\pm$    6.3 &   1.05 $\pm$   0.09 &     19 &  11.88 & 8.91e-01\\
 &      - & - & - &    0.0 & - &  147.7 $\pm$    5.2 &   0.62 $\pm$   0.03 &     20 &  46.28 & 7.38e-04\\
Abell 2631 &   extr &     38 &   0.80 &  308.8 $\pm$   37.4 &    8.3 &   29.2 $\pm$   23.4 &   1.44 $\pm$   0.41 &     35 &   0.21 & 1.00e+00\\
 &      - & - & - &    0.0 & - &  347.2 $\pm$   21.7 &   0.33 $\pm$   0.05 &     36 &  13.73 & 1.00e+00\\
 &   flat & - & - &  308.8 $\pm$   37.4 &    8.3 &   29.2 $\pm$   23.4 &   1.44 $\pm$   0.41 &     35 &   0.21 & 1.00e+00\\
 &      - & - & - &    0.0 & - &  347.2 $\pm$   21.7 &   0.33 $\pm$   0.05 &     36 &  13.73 & 1.00e+00\\
Abell 2657 &   extr &     51 &   0.20 &   65.4 $\pm$   12.0 &    5.5 &  153.5 $\pm$   15.1 &   0.91 $\pm$   0.13 &     48 &   7.69 & 1.00e+00\\
 &      - & - & - &    0.0 & - &  222.0 $\pm$    5.9 &   0.50 $\pm$   0.03 &     49 &  21.73 & 1.00e+00\\
 &   flat & - & - &   65.4 $\pm$   12.0 &    5.5 &  153.5 $\pm$   15.1 &   0.91 $\pm$   0.13 &     48 &   7.69 & 1.00e+00\\
 &      - & - & - &    0.0 & - &  222.0 $\pm$    5.9 &   0.50 $\pm$   0.03 &     49 &  21.73 & 1.00e+00\\
Abell 2667 &   extr &     11 &   0.20 &   12.3 $\pm$    4.0 &    3.1 &  102.2 $\pm$    7.7 &   1.17 $\pm$   0.15 &      8 &   1.61 & 9.91e-01\\
 &      - & - & - &    0.0 & - &  113.7 $\pm$    6.2 &   0.85 $\pm$   0.05 &      9 &   9.48 & 3.94e-01\\
 &   flat & - & - &   19.3 $\pm$    3.4 &    5.7 &   93.4 $\pm$    7.6 &   1.31 $\pm$   0.17 &      8 &   1.66 & 9.90e-01\\
 &      - & - & - &    0.0 & - &  110.5 $\pm$    6.2 &   0.75 $\pm$   0.05 &      9 &  20.81 & 1.35e-02\\
Abell 2717 &   extr &     26 &   0.12 &   26.3 $\pm$    8.2 &    3.2 &  152.2 $\pm$   10.1 &   0.76 $\pm$   0.13 &     23 &   2.19 & 1.00e+00\\
 &      - & - & - &    0.0 & - &  167.6 $\pm$    7.9 &   0.50 $\pm$   0.03 &     24 &   7.90 & 9.99e-01\\
 &   flat & - & - &   27.0 $\pm$    8.4 &    3.2 &  151.2 $\pm$   10.2 &   0.75 $\pm$   0.13 &     23 &   2.15 & 1.00e+00\\
 &      - & - & - &    0.0 & - &  167.2 $\pm$    7.8 &   0.49 $\pm$   0.03 &     24 &   7.77 & 9.99e-01\\
Abell 2744 &   extr &     27 &   0.60 &  295.1 $\pm$  113.4 &    2.6 &  152.8 $\pm$  112.7 &   0.83 $\pm$   0.37 &     24 &   8.72 & 9.98e-01\\
 &      - & - & - &    0.0 & - &  460.3 $\pm$   29.9 &   0.37 $\pm$   0.05 &     25 &  10.50 & 9.95e-01\\
 &   flat & - & - &  438.4 $\pm$   58.7 &    7.5 &   46.4 $\pm$   44.0 &   1.41 $\pm$   0.55 &     24 &   7.87 & 9.99e-01\\
 &      - & - & - &    0.0 & - &  503.6 $\pm$   29.3 &   0.30 $\pm$   0.05 &     25 &  14.15 & 9.59e-01\\
Abell 2813 &   extr &     14 &   0.30 &  216.3 $\pm$   48.9 &    4.4 &  126.0 $\pm$   74.9 &   1.52 $\pm$   0.64 &     11 &   2.29 & 9.97e-01\\
 &      - & - & - &    0.0 & - &  397.4 $\pm$   33.0 &   0.42 $\pm$   0.10 &     12 &   7.83 & 7.98e-01\\
 &   flat & - & - &  267.6 $\pm$   43.8 &    6.1 &   90.4 $\pm$   67.3 &   1.76 $\pm$   0.80 &     11 &   2.64 & 9.95e-01\\
 &      - & - & - &    0.0 & - &  417.0 $\pm$   33.5 &   0.31 $\pm$   0.09 &     12 &   8.95 & 7.07e-01\\
Abell 3084 &   extr &     34 &   0.30 &   96.7 $\pm$   13.4 &    7.2 &  193.7 $\pm$   22.8 &   1.08 $\pm$   0.17 &     31 &   4.48 & 1.00e+00\\
 &      - & - & - &    0.0 & - &  288.3 $\pm$   14.4 &   0.43 $\pm$   0.04 &     32 &  17.29 & 9.84e-01\\
 &   flat & - & - &   96.7 $\pm$   13.4 &    7.2 &  193.7 $\pm$   22.8 &   1.08 $\pm$   0.17 &     31 &   4.48 & 1.00e+00\\
 &      - & - & - &    0.0 & - &  288.3 $\pm$   14.4 &   0.43 $\pm$   0.04 &     32 &  17.29 & 9.84e-01\\
Abell 3088 &   extr &     10 &   0.20 &   32.7 $\pm$    9.5 &    3.4 &  269.7 $\pm$   25.8 &   1.51 $\pm$   0.20 &      7 &   0.21 & 1.00e+00\\
 &      - & - & - &    0.0 & - &  283.9 $\pm$   23.8 &   1.02 $\pm$   0.09 &      8 &   7.68 & 4.65e-01\\
 &   flat & - & - &   82.8 $\pm$    8.4 &    9.8 &  216.8 $\pm$   25.8 &   1.71 $\pm$   0.25 &      7 &   0.59 & 9.99e-01\\
 &      - & - & - &    0.0 & - &  230.3 $\pm$   18.8 &   0.49 $\pm$   0.06 &      8 &  18.94 & 1.52e-02\\
Abell 3112 &   extr &     18 &   0.12 &    8.2 $\pm$    1.6 &    5.3 &  170.1 $\pm$    6.8 &   1.09 $\pm$   0.06 &     15 &   3.55 & 9.99e-01\\
 &      - & - & - &    0.0 & - &  162.7 $\pm$    6.0 &   0.86 $\pm$   0.03 &     16 &  23.03 & 1.13e-01\\
 &   flat & - & - &   11.4 $\pm$    1.4 &    8.0 &  169.1 $\pm$    7.0 &   1.17 $\pm$   0.07 &     15 &   5.32 & 9.89e-01\\
 &      - & - & - &    0.0 & - &  157.3 $\pm$    5.8 &   0.82 $\pm$   0.03 &     16 &  45.16 & 1.31e-04\\
Abell 3120 &   extr &     29 &   0.20 &   15.0 $\pm$    3.3 &    4.5 &  209.1 $\pm$   10.9 &   1.02 $\pm$   0.08 &     26 &   6.41 & 1.00e+00\\
 &      - & - & - &    0.0 & - &  206.6 $\pm$   10.1 &   0.76 $\pm$   0.03 &     27 &  20.49 & 8.10e-01\\
 &   flat & - & - &   17.3 $\pm$    3.5 &    4.9 &  206.2 $\pm$   10.9 &   0.99 $\pm$   0.08 &     26 &   7.14 & 1.00e+00\\
 &      - & - & - &    0.0 & - &  202.9 $\pm$    9.8 &   0.70 $\pm$   0.03 &     27 &  22.57 & 7.08e-01\\
Abell 3158 &   extr &     72 &   0.40 &  166.0 $\pm$   11.7 &   14.1 &   80.9 $\pm$   12.9 &   0.90 $\pm$   0.10 &     69 &  22.54 & 1.00e+00\\
 &      - & - & - &    0.0 & - &  260.6 $\pm$    2.9 &   0.32 $\pm$   0.01 &     70 &  71.32 & 4.34e-01\\
 &   flat & - & - &  166.0 $\pm$   11.7 &   14.1 &   80.9 $\pm$   12.9 &   0.90 $\pm$   0.10 &     69 &  22.54 & 1.00e+00\\
 &      - & - & - &    0.0 & - &  260.6 $\pm$    2.9 &   0.32 $\pm$   0.01 &     70 &  71.32 & 4.34e-01\\
Abell 3266 &   extr &     15 &   0.08 &   63.7 $\pm$   41.9 &    1.5 &  405.3 $\pm$   51.6 &   0.71 $\pm$   0.27 &     12 &   0.79 & 1.00e+00\\
 &      - & - & - &    0.0 & - &  418.9 $\pm$   37.8 &   0.44 $\pm$   0.06 &     13 &   2.02 & 1.00e+00\\
 &   flat & - & - &   72.5 $\pm$   49.7 &    1.5 &  376.7 $\pm$   48.0 &   0.64 $\pm$   0.28 &     12 &   1.26 & 1.00e+00\\
 &      - & - & - &    0.0 & - &  404.6 $\pm$   35.2 &   0.39 $\pm$   0.05 &     13 &   2.34 & 1.00e+00\\
Abell 3364 &   extr &     55 &   0.70 &  268.6 $\pm$   33.2 &    8.1 &   34.5 $\pm$   18.0 &   1.97 $\pm$   0.32 &     52 &   3.99 & 1.00e+00\\
 &      - & - & - &    0.0 & - &  298.6 $\pm$   22.7 &   0.63 $\pm$   0.08 &     53 &  30.04 & 9.95e-01\\
 &   flat & - & - &  268.6 $\pm$   33.2 &    8.1 &   34.5 $\pm$   18.0 &   1.97 $\pm$   0.32 &     52 &   3.99 & 1.00e+00\\
 &      - & - & - &    0.0 & - &  298.6 $\pm$   22.7 &   0.63 $\pm$   0.08 &     53 &  30.04 & 9.95e-01\\
Abell 3376 &   extr &     67 &   0.30 &  282.9 $\pm$    9.3 &   30.3 &   59.0 $\pm$   10.6 &   1.71 $\pm$   0.18 &     64 &   5.46 & 1.00e+00\\
 &      - & - & - &    0.0 & - &  378.5 $\pm$    4.3 &   0.30 $\pm$   0.02 &     65 & 112.42 & 2.39e-04\\
 &   flat & - & - &  282.9 $\pm$    9.3 &   30.3 &   59.0 $\pm$   10.6 &   1.71 $\pm$   0.18 &     64 &   5.46 & 1.00e+00\\
 &      - & - & - &    0.0 & - &  378.5 $\pm$    4.3 &   0.30 $\pm$   0.02 &     65 & 112.42 & 2.39e-04\\
Abell 3391 &   extr &     75 &   0.40 &  367.5 $\pm$   16.0 &   22.9 &   23.6 $\pm$   14.8 &   1.64 $\pm$   0.47 &     72 &   3.59 & 1.00e+00\\
 &      - & - & - &    0.0 & - &  420.4 $\pm$    7.5 &   0.14 $\pm$   0.02 &     73 &  24.89 & 1.00e+00\\
 &   flat & - & - &  367.5 $\pm$   16.0 &   22.9 &   23.6 $\pm$   14.8 &   1.64 $\pm$   0.47 &     72 &   3.59 & 1.00e+00\\
 &      - & - & - &    0.0 & - &  420.4 $\pm$    7.5 &   0.14 $\pm$   0.02 &     73 &  24.89 & 1.00e+00\\
Abell 3395 &   extr &     24 &   0.12 &  213.3 $\pm$   26.2 &    8.2 &  133.5 $\pm$   30.4 &   1.58 $\pm$   0.79 &     21 &   0.00 & 1.00e+00\\
 &      - & - & - &    0.0 & - &  325.5 $\pm$   14.4 &   0.23 $\pm$   0.05 &     22 &   5.73 & 1.00e+00\\
 &   flat & - & - &  247.2 $\pm$   25.2 &    9.8 &  105.9 $\pm$   29.8 &   1.65 $\pm$   1.01 &     21 &   0.01 & 1.00e+00\\
 &      - & - & - &    0.0 & - &  332.8 $\pm$   14.0 &   0.16 $\pm$   0.05 &     22 &   4.49 & 1.00e+00\\
Abell 3528S &   extr &     24 &   0.12 &   19.4 $\pm$    2.3 &    8.6 &  288.1 $\pm$   10.2 &   1.16 $\pm$   0.05 &     21 &  32.09 & 5.73e-02\\
 &      - & - & - &    0.0 & - &  271.7 $\pm$    8.8 &   0.84 $\pm$   0.02 &     22 &  84.38 & 3.04e-09\\
 &   flat & - & - &   31.6 $\pm$    2.3 &   14.0 &  270.0 $\pm$   10.3 &   1.17 $\pm$   0.06 &     21 &  32.23 & 5.55e-02\\
 &      - & - & - &    0.0 & - &  239.2 $\pm$    7.6 &   0.65 $\pm$   0.02 &     22 & 128.53 & 4.82e-17\\
Abell 3558 &   extr &     25 &   0.12 &  126.2 $\pm$   11.8 &   10.7 &  132.5 $\pm$   17.2 &   2.11 $\pm$   0.58 &     22 &   6.87 & 9.99e-01\\
 &      - & - & - &    0.0 & - &  234.0 $\pm$   10.7 &   0.42 $\pm$   0.06 &     23 &  19.89 & 6.49e-01\\
 &   flat & - & - &  126.2 $\pm$   11.8 &   10.7 &  132.5 $\pm$   17.2 &   2.11 $\pm$   0.58 &     22 &   6.87 & 9.99e-01\\
 &      - & - & - &    0.0 & - &  234.0 $\pm$   10.7 &   0.42 $\pm$   0.06 &     23 &  19.89 & 6.49e-01\\
Abell 3562 &   extr &     26 &   0.12 &   71.4 $\pm$    9.0 &    8.0 &  166.8 $\pm$   10.4 &   0.80 $\pm$   0.13 &     23 &  33.16 & 7.84e-02\\
 &      - & - & - &    0.0 & - &  217.3 $\pm$    6.5 &   0.33 $\pm$   0.02 &     24 &  54.22 & 3.99e-04\\
 &   flat & - & - &   77.4 $\pm$    8.9 &    8.7 &  159.8 $\pm$   10.4 &   0.81 $\pm$   0.13 &     23 &  35.16 & 5.01e-02\\
 &      - & - & - &    0.0 & - &  215.4 $\pm$    6.4 &   0.31 $\pm$   0.02 &     24 &  56.31 & 2.08e-04\\
Abell 3571 &   extr &     31 &   0.12 &   79.3 $\pm$   14.8 &    5.4 &  191.3 $\pm$   14.8 &   0.82 $\pm$   0.16 &     28 & 375.69 & 1.65e-62\\
 &      - & - & - &    0.0 & - &  256.1 $\pm$    7.9 &   0.39 $\pm$   0.03 &     29 & 657.82 & 6.19e-120\\
 &   flat & - & - &   79.3 $\pm$   14.8 &    5.4 &  191.3 $\pm$   14.8 &   0.82 $\pm$   0.16 &     28 & 375.69 & 1.65e-62\\
 &      - & - & - &    0.0 & - &  256.1 $\pm$    7.9 &   0.39 $\pm$   0.03 &     29 & 657.82 & 6.19e-120\\
Abell 3581 &   extr &     46 &   0.10 &    7.1 $\pm$    0.8 &    8.4 &  138.1 $\pm$    5.5 &   1.15 $\pm$   0.05 &     43 &  20.49 & 9.99e-01\\
 &      - & - & - &    0.0 & - &  121.6 $\pm$    4.0 &   0.85 $\pm$   0.02 &     44 &  65.85 & 1.80e-02\\
 &   flat & - & - &    9.5 $\pm$    0.8 &   12.2 &  138.1 $\pm$    5.7 &   1.22 $\pm$   0.05 &     43 &  21.56 & 9.97e-01\\
 &      - & - & - &    0.0 & - &  114.3 $\pm$    3.8 &   0.79 $\pm$   0.02 &     44 & 103.30 & 1.13e-06\\
Abell 3667 &   extr &     56 &   0.30 &  149.3 $\pm$   17.2 &    8.7 &  121.9 $\pm$   18.6 &   0.72 $\pm$   0.09 &     53 &  21.14 & 1.00e+00\\
 &      - & - & - &    0.0 & - &  278.7 $\pm$    2.3 &   0.34 $\pm$   0.01 &     54 &  44.43 & 8.20e-01\\
 &   flat & - & - &  160.4 $\pm$   15.5 &   10.4 &  110.6 $\pm$   16.8 &   0.78 $\pm$   0.10 &     53 &  22.84 & 1.00e+00\\
 &      - & - & - &    0.0 & - &  279.5 $\pm$    2.3 &   0.33 $\pm$   0.01 &     54 &  52.83 & 5.19e-01\\
Abell 3822 &   extr &     42 &   0.30 &  108.7 $\pm$   76.4 &    1.4 &  200.3 $\pm$   90.8 &   0.66 $\pm$   0.33 &     39 &   3.24 & 1.00e+00\\
 &      - & - & - &    0.0 & - &  322.5 $\pm$   16.6 &   0.38 $\pm$   0.07 &     40 &   3.95 & 1.00e+00\\
 &   flat & - & - &  108.7 $\pm$   76.4 &    1.4 &  200.3 $\pm$   90.8 &   0.66 $\pm$   0.33 &     39 &   3.24 & 1.00e+00\\
 &      - & - & - &    0.0 & - &  322.5 $\pm$   16.6 &   0.38 $\pm$   0.07 &     40 &   3.95 & 1.00e+00\\
Abell 3827 &   extr &     67 &   0.60 &  144.6 $\pm$   13.4 &   10.8 &  113.1 $\pm$   15.2 &   1.23 $\pm$   0.10 &     64 & 1651.91 & 6.60e-303\\
 &      - & - & - &    0.0 & - &  287.2 $\pm$    7.4 &   0.60 $\pm$   0.03 &     65 & 4867.53 & 0.00e+00\\
 &   flat & - & - &  164.6 $\pm$   12.5 &   13.2 &   94.8 $\pm$   13.7 &   1.34 $\pm$   0.10 &     64 & 1368.56 & 6.59e-244\\
 &      - & - & - &    0.0 & - &  293.5 $\pm$    7.3 &   0.57 $\pm$   0.03 &     65 & 5896.48 & 0.00e+00\\
Abell 3921 &   extr &     47 &   0.40 &  101.2 $\pm$   17.9 &    5.7 &  151.5 $\pm$   23.0 &   0.86 $\pm$   0.11 &     44 &   7.55 & 1.00e+00\\
 &      - & - & - &    0.0 & - &  272.4 $\pm$    6.8 &   0.48 $\pm$   0.03 &     45 &  22.08 & 9.98e-01\\
 &   flat & - & - &  101.2 $\pm$   17.9 &    5.7 &  151.5 $\pm$   23.0 &   0.86 $\pm$   0.11 &     44 &   7.55 & 1.00e+00\\
 &      - & - & - &    0.0 & - &  272.4 $\pm$    6.8 &   0.48 $\pm$   0.03 &     45 &  22.08 & 9.98e-01\\
Abell 4038 &   extr &     42 &   0.12 &   37.1 $\pm$    1.2 &   30.2 &  118.5 $\pm$    2.7 &   1.10 $\pm$   0.05 &     39 &  58.69 & 2.22e-02\\
 &      - & - & - &    0.0 & - &  127.3 $\pm$    2.0 &   0.42 $\pm$   0.01 &     40 & 393.69 & 1.15e-59\\
 &   flat & - & - &   37.9 $\pm$    1.2 &   31.2 &  117.9 $\pm$    2.7 &   1.11 $\pm$   0.05 &     39 &  60.31 & 1.58e-02\\
 &      - & - & - &    0.0 & - &  126.5 $\pm$    1.9 &   0.41 $\pm$   0.01 &     40 & 410.34 & 6.07e-63\\
Abell 4059 &   extr &     33 &   0.15 &    0.0 $\pm$    1.0 &    0.0 &  210.7 $\pm$    2.2 &   0.82 $\pm$   0.01 &     30 &  44.86 & 3.98e-02\\
 &      - & - & - &    0.0 & - &  210.7 $\pm$    2.2 &   0.82 $\pm$   0.01 &     31 &  44.86 & 5.13e-02\\
 &   flat & - & - &    7.1 $\pm$    1.0 &    6.7 &  203.2 $\pm$    2.4 &   0.88 $\pm$   0.02 &     30 &  54.25 & 4.31e-03\\
 &      - & - & - &    0.0 & - &  208.3 $\pm$    2.2 &   0.77 $\pm$   0.01 &     31 &  93.59 & 3.35e-08\\
Abell S0405 &   extr &     34 &   0.20 &   23.5 $\pm$   21.0 &    1.1 &  261.1 $\pm$   22.1 &   0.52 $\pm$   0.10 &     31 &   8.24 & 1.00e+00\\
 &      - & - & - &    0.0 & - &  281.9 $\pm$   11.3 &   0.43 $\pm$   0.03 &     32 &   9.16 & 1.00e+00\\
 &   flat & - & - &   16.9 $\pm$   27.9 &    0.6 &  274.2 $\pm$   27.3 &   0.45 $\pm$   0.10 &     31 &   9.79 & 1.00e+00\\
 &      - & - & - &    0.0 & - &  289.3 $\pm$   11.2 &   0.40 $\pm$   0.02 &     32 &  10.10 & 1.00e+00\\
Abell S0592 &   extr &     23 &   0.40 &   52.2 $\pm$   14.4 &    3.6 &  199.0 $\pm$   23.6 &   0.99 $\pm$   0.12 &     20 &   9.34 & 9.79e-01\\
 &      - & - & - &    0.0 & - &  271.1 $\pm$   10.0 &   0.68 $\pm$   0.04 &     21 &  16.08 & 7.65e-01\\
 &   flat & - & - &   58.7 $\pm$   14.4 &    4.1 &  195.5 $\pm$   23.6 &   0.99 $\pm$   0.13 &     20 &   9.70 & 9.73e-01\\
 &      - & - & - &    0.0 & - &  275.6 $\pm$   10.0 &   0.65 $\pm$   0.04 &     21 &  17.46 & 6.83e-01\\
AC 114 &   flat &     20 &   0.45 &  199.8 $\pm$   28.0 &    7.1 &   70.0 $\pm$   32.6 &   1.50 $\pm$   0.36 &     17 &   3.69 & 1.00e+00\\
 &      - & - & - &    0.0 & - &  306.6 $\pm$   14.8 &   0.46 $\pm$   0.06 &     18 &  16.94 & 5.28e-01\\
 &   extr & - & - &  199.8 $\pm$   28.0 &    7.1 &   70.0 $\pm$   32.6 &   1.50 $\pm$   0.36 &     17 &   3.69 & 1.00e+00\\
 &      - & - & - &    0.0 & - &  306.6 $\pm$   14.8 &   0.46 $\pm$   0.06 &     18 &  16.94 & 5.28e-01\\
AWM7 &   extr &     13 &   0.02 &    4.8 $\pm$    1.1 &    4.5 &  290.2 $\pm$   28.4 &   0.89 $\pm$   0.06 &     10 &   7.30 & 6.96e-01\\
 &      - & - & - &    0.0 & - &  217.6 $\pm$   10.6 &   0.70 $\pm$   0.02 &     11 &  20.91 & 3.43e-02\\
 &   flat & - & - &    8.4 $\pm$    1.3 &    6.5 &  227.6 $\pm$   23.1 &   0.80 $\pm$   0.06 &     10 &  13.19 & 2.13e-01\\
 &      - & - & - &    0.0 & - &  157.1 $\pm$    6.6 &   0.54 $\pm$   0.01 &     11 &  32.84 & 5.58e-04\\
Centaurus &   extr &     27 &   0.03 &    1.4 $\pm$   0.04 &   32.1 &  421.2 $\pm$    5.4 &   1.25 $\pm$   0.01 &     24 & 253.13 & 3.95e-40\\
 &      - & - & - &    0.0 & - &  328.8 $\pm$    3.1 &   1.11 $\pm$   0.00 &     25 & 1159.86 & 6.17e-229\\
 &   flat & - & - &    2.2 $\pm$   0.04 &   56.6 &  474.9 $\pm$    6.3 &   1.33 $\pm$   0.01 &     24 & 483.38 & 4.67e-87\\
 &      - & - & - &    0.0 & - &  307.3 $\pm$    2.9 &   1.08 $\pm$   0.00 &     25 & 3151.59 & 0.00e+00\\
CID 72 &   extr &     37 &   0.12 &    4.9 $\pm$    0.3 &   14.6 &  139.2 $\pm$    2.1 &   0.95 $\pm$   0.02 &     34 & 135.51 & 4.60e-14\\
 &      - & - & - &    0.0 & - &  128.6 $\pm$    1.7 &   0.77 $\pm$   0.01 &     35 & 313.61 & 1.74e-46\\
 &   flat & - & - &    9.4 $\pm$    0.3 &   29.9 &  133.2 $\pm$    2.2 &   0.99 $\pm$   0.02 &     34 & 129.24 & 5.04e-13\\
 &      - & - & - &    0.0 & - &  111.3 $\pm$    1.5 &   0.63 $\pm$   0.01 &     35 & 634.02 & 4.82e-111\\
CL J1226.9+3332 &   extr &     10 &   0.40 &  166.0 $\pm$   45.2 &    3.7 &   99.0 $\pm$   58.7 &   1.41 $\pm$   0.50 &      7 &   0.75 & 9.98e-01\\
 &      - & - & - &    0.0 & - &  308.7 $\pm$   25.3 &   0.55 $\pm$   0.10 &      8 &   4.81 & 7.78e-01\\
 &   flat & - & - &  166.0 $\pm$   45.2 &    3.7 &   99.0 $\pm$   58.7 &   1.41 $\pm$   0.50 &      7 &   0.75 & 9.98e-01\\
 &      - & - & - &    0.0 & - &  308.7 $\pm$   25.3 &   0.55 $\pm$   0.10 &      8 &   4.81 & 7.78e-01\\
Cygnus A &   extr &     19 &   0.10 &   21.7 $\pm$    0.9 &   24.2 &  208.4 $\pm$    6.7 &   1.51 $\pm$   0.05 &     16 &  28.49 & 2.76e-02\\
 &      - & - & - &    0.0 & - &  154.4 $\pm$    3.7 &   0.73 $\pm$   0.02 &     17 & 294.72 & 1.38e-52\\
 &   flat & - & - &   23.6 $\pm$    0.9 &   27.1 &  210.1 $\pm$    6.9 &   1.57 $\pm$   0.05 &     16 &  22.48 & 1.28e-01\\
 &      - & - & - &    0.0 & - &  148.5 $\pm$    3.6 &   0.70 $\pm$   0.02 &     17 & 340.49 & 4.67e-62\\
ESO 3060170 &   extr &      5 &   0.02 &    7.8 $\pm$    1.0 &    7.8 & 1370.5 $\pm$  562.2 &   1.79 $\pm$   0.20 &      2 &   0.77 & 6.80e-01\\
 &      - & - & - &    0.0 & - &  255.8 $\pm$   37.1 &   0.90 $\pm$   0.05 &      3 &  25.78 & 1.06e-05\\
 &   flat & - & - &    8.0 $\pm$    1.0 &    8.0 & 1400.9 $\pm$  578.9 &   1.80 $\pm$   0.21 &      2 &   0.81 & 6.67e-01\\
 &      - & - & - &    0.0 & - &  251.2 $\pm$   36.3 &   0.89 $\pm$   0.05 &      3 &  26.70 & 6.81e-06\\
ESO 5520200 &   extr &     17 &   0.10 &    6.3 $\pm$    3.5 &    1.8 &  113.8 $\pm$    7.0 &   0.74 $\pm$   0.10 &     31 &   0.15 & 1.00e+00\\
 &      - & - & - &    0.0 & - &  112.0 $\pm$    6.0 &   0.60 $\pm$   0.03 &     32 &   5.57 & 1.00e+00\\
 &   flat & - & - &    5.9 $\pm$    4.2 &    1.4 &  121.8 $\pm$    6.5 &   0.67 $\pm$   0.09 &     31 &   0.52 & 1.00e+00\\
 &      - & - & - &    0.0 & - &  121.1 $\pm$    5.8 &   0.57 $\pm$   0.03 &     32 &   4.15 & 1.00e+00\\
EXO 422-086 &   extr &     19 &   0.07 &   10.1 $\pm$    0.8 &   12.5 &  199.3 $\pm$   11.4 &   1.21 $\pm$   0.06 &     16 &  11.00 & 8.10e-01\\
 &      - & - & - &    0.0 & - &  142.0 $\pm$    5.6 &   0.75 $\pm$   0.02 &     17 & 112.48 & 4.11e-16\\
 &   flat & - & - &   13.8 $\pm$    0.8 &   17.5 &  193.8 $\pm$   11.8 &   1.25 $\pm$   0.06 &     16 &  11.24 & 7.95e-01\\
 &      - & - & - &    0.0 & - &  120.4 $\pm$    4.6 &   0.62 $\pm$   0.02 &     17 & 157.52 & 8.19e-25\\
HCG 62 &   extr &     27 &   0.04 &    3.1 $\pm$   0.08 &   40.8 &  203.9 $\pm$   10.4 &   1.23 $\pm$   0.02 &     24 & 153.17 & 8.52e-21\\
 &      - & - & - &    0.0 & - &   63.4 $\pm$    1.7 &   0.63 $\pm$   0.01 &     25 & 660.63 & 2.48e-123\\
 &   flat & - & - &    3.4 $\pm$   0.07 &   47.4 &  219.0 $\pm$   11.4 &   1.28 $\pm$   0.03 &     24 & 138.50 & 4.39e-18\\
 &      - & - & - &    0.0 & - &   57.7 $\pm$    1.5 &   0.60 $\pm$   0.01 &     25 & 751.59 & 1.92e-142\\
HCG 42 &   extr &     22 &   0.03 &    1.8 $\pm$    0.3 &    5.5 &  128.5 $\pm$   12.8 &   0.88 $\pm$   0.05 &     19 &  44.38 & 8.38e-04\\
 &      - & - & - &    0.0 & - &   89.4 $\pm$    4.1 &   0.67 $\pm$   0.01 &     20 &  60.87 & 5.23e-06\\
 &   flat & - & - &    1.9 $\pm$    0.3 &    5.7 &  126.5 $\pm$   12.6 &   0.87 $\pm$   0.05 &     19 &  45.28 & 6.25e-04\\
 &      - & - & - &    0.0 & - &   87.4 $\pm$    4.0 &   0.66 $\pm$   0.01 &     20 &  62.24 & 3.19e-06\\
Hercules A &   extr &     16 &   0.20 &    2.8 $\pm$    1.5 &    1.8 &  151.8 $\pm$    3.3 &   0.99 $\pm$   0.04 &     13 &   2.34 & 1.00e+00\\
 &      - & - & - &    0.0 & - &  154.1 $\pm$    3.1 &   0.94 $\pm$   0.02 &     14 &   6.23 & 9.60e-01\\
 &   flat & - & - &    9.2 $\pm$    1.3 &    6.8 &  143.9 $\pm$    3.3 &   1.07 $\pm$   0.04 &     13 &   6.24 & 9.37e-01\\
 &      - & - & - &    0.0 & - &  151.0 $\pm$    3.1 &   0.87 $\pm$   0.02 &     14 &  46.89 & 2.01e-05\\
Hydra A &   extr &     57 &   0.30 &   13.0 $\pm$    0.7 &   19.5 &  115.3 $\pm$    1.4 &   1.02 $\pm$   0.02 &     54 &  71.44 & 5.62e-02\\
 &      - & - & - &    0.0 & - &  134.0 $\pm$    1.0 &   0.81 $\pm$   0.01 &     55 & 364.39 & 3.36e-47\\
 &   flat & - & - &   13.3 $\pm$    0.7 &   20.0 &  114.9 $\pm$    1.4 &   1.03 $\pm$   0.02 &     54 &  72.66 & 4.60e-02\\
 &      - & - & - &    0.0 & - &  134.0 $\pm$    1.0 &   0.80 $\pm$   0.01 &     55 & 379.86 & 4.40e-50\\
M49 &   extr &     54 &   1.00 &    0.9 $\pm$   0.05 &   18.1 &  486.7 $\pm$   32.2 &   1.14 $\pm$   0.02 &     51 &  74.03 & 1.92e-02\\
 &      - & - & - &    0.0 & - &  231.3 $\pm$   10.1 &   0.89 $\pm$   0.01 &     52 & 327.07 & 1.58e-41\\
 &   flat & - & - &    0.9 $\pm$   0.05 &   18.9 &  495.3 $\pm$   32.9 &   1.14 $\pm$   0.02 &     51 &  75.65 & 1.41e-02\\
 &      - & - & - &    0.0 & - &  227.4 $\pm$   10.0 &   0.88 $\pm$   0.01 &     52 & 349.43 & 1.14e-45\\
M87 &   extr &     88 &   0.04 &    3.5 $\pm$   0.08 &   43.1 &  146.4 $\pm$    1.0 &   0.80 $\pm$   0.00 &     85 & 749.92 & 4.94e-107\\
 &      - & - & - &    0.0 & - &  123.8 $\pm$    0.5 &   0.64 $\pm$   0.00 &     86 & 2083.55 & 0.00e+00\\
 &   flat & - & - &    3.5 $\pm$   0.08 &   43.7 &  146.6 $\pm$    1.0 &   0.80 $\pm$   0.00 &     85 & 763.71 & 1.06e-109\\
 &      - & - & - &    0.0 & - &  123.7 $\pm$    0.5 &   0.64 $\pm$   0.00 &     86 & 2130.02 & 0.00e+00\\
MACS J0011.7-1523 &   extr &     16 &   0.40 &   14.9 $\pm$    6.4 &    2.3 &  111.3 $\pm$   11.6 &   1.03 $\pm$   0.10 &     13 &   1.95 & 1.00e+00\\
 &      - & - & - &    0.0 & - &  134.7 $\pm$    5.1 &   0.86 $\pm$   0.04 &     14 &   5.88 & 9.70e-01\\
 &   flat & - & - &   18.8 $\pm$    6.3 &    3.0 &  109.1 $\pm$   11.5 &   1.04 $\pm$   0.10 &     13 &   2.28 & 1.00e+00\\
 &      - & - & - &    0.0 & - &  138.4 $\pm$    5.0 &   0.81 $\pm$   0.04 &     14 &   7.99 & 8.90e-01\\
MACS J0035.4-2015 &   extr &     29 &   0.70 &   69.5 $\pm$   17.1 &    4.1 &   93.9 $\pm$   23.0 &   1.15 $\pm$   0.16 &     26 &   0.70 & 1.00e+00\\
 &      - & - & - &    0.0 & - &  183.2 $\pm$   11.5 &   0.74 $\pm$   0.06 &     27 &  11.72 & 9.95e-01\\
 &   flat & - & - &   93.4 $\pm$   15.7 &    6.0 &   76.4 $\pm$   20.8 &   1.26 $\pm$   0.17 &     26 &   1.00 & 1.00e+00\\
 &      - & - & - &    0.0 & - &  198.2 $\pm$   11.1 &   0.66 $\pm$   0.05 &     27 &  20.41 & 8.13e-01\\
MACS J0159.8-0849 &   extr &     15 &   0.40 &   11.9 $\pm$    4.0 &    3.0 &  133.7 $\pm$   10.0 &   1.25 $\pm$   0.08 &     12 &   2.47 & 9.98e-01\\
 &      - & - & - &    0.0 & - &  155.7 $\pm$    5.8 &   1.06 $\pm$   0.04 &     13 &   9.44 & 7.39e-01\\
 &   flat & - & - &   18.8 $\pm$    3.7 &    5.0 &  123.9 $\pm$    9.9 &   1.31 $\pm$   0.09 &     12 &   3.68 & 9.89e-01\\
 &      - & - & - &    0.0 & - &  158.3 $\pm$    5.9 &   1.01 $\pm$   0.04 &     13 &  21.08 & 7.13e-02\\
MACS J0242.5-2132 &   extr &     22 &   0.50 &    9.7 $\pm$    1.9 &    5.0 &   76.3 $\pm$    5.1 &   1.27 $\pm$   0.07 &     19 &  11.73 & 8.97e-01\\
 &      - & - & - &    0.0 & - &   94.0 $\pm$    3.2 &   1.01 $\pm$   0.04 &     20 &  29.52 & 7.81e-02\\
 &   flat & - & - &   10.9 $\pm$    1.9 &    5.7 &   74.6 $\pm$    5.0 &   1.29 $\pm$   0.07 &     19 &  11.84 & 8.93e-01\\
 &      - & - & - &    0.0 & - &   94.4 $\pm$    3.2 &   0.99 $\pm$   0.03 &     20 &  34.37 & 2.37e-02\\
MACS J0257.1-2325 &   extr &     13 &   0.40 &  234.5 $\pm$   68.2 &    3.4 &  195.8 $\pm$  107.3 &   1.39 $\pm$   0.57 &     10 &   0.24 & 1.00e+00\\
 &      - & - & - &    0.0 & - &  489.1 $\pm$   50.9 &   0.47 $\pm$   0.12 &     11 &   3.07 & 9.90e-01\\
 &   flat & - & - &  234.5 $\pm$   68.2 &    3.4 &  195.8 $\pm$  107.3 &   1.39 $\pm$   0.57 &     10 &   0.24 & 1.00e+00\\
 &      - & - & - &    0.0 & - &  489.1 $\pm$   50.9 &   0.47 $\pm$   0.12 &     11 &   3.07 & 9.90e-01\\
MACS J0257.6-2209 &   extr &     17 &   0.40 &  155.1 $\pm$   25.1 &    6.2 &   82.7 $\pm$   32.5 &   1.55 $\pm$   0.34 &     14 &   1.00 & 1.00e+00\\
 &      - & - & - &    0.0 & - &  277.1 $\pm$   15.3 &   0.56 $\pm$   0.07 &     15 &  18.10 & 2.57e-01\\
 &   flat & - & - &  155.9 $\pm$   25.0 &    6.2 &   82.1 $\pm$   32.4 &   1.55 $\pm$   0.34 &     14 &   1.01 & 1.00e+00\\
 &      - & - & - &    0.0 & - &  277.6 $\pm$   15.2 &   0.56 $\pm$   0.07 &     15 &  18.25 & 2.49e-01\\
MACS J0308.9+2645 &   extr &     30 &   0.70 &  212.8 $\pm$   53.9 &    3.9 &   70.1 $\pm$   42.2 &   1.43 $\pm$   0.35 &     27 &   0.86 & 1.00e+00\\
 &      - & - & - &    0.0 & - &  290.5 $\pm$   34.0 &   0.66 $\pm$   0.10 &     28 &   7.88 & 1.00e+00\\
 &   flat & - & - &  212.8 $\pm$   53.9 &    3.9 &   70.1 $\pm$   42.2 &   1.43 $\pm$   0.35 &     27 &   0.86 & 1.00e+00\\
 &      - & - & - &    0.0 & - &  290.5 $\pm$   34.0 &   0.66 $\pm$   0.10 &     28 &   7.88 & 1.00e+00\\
MACS J0329.6-0211 &   extr &     14 &   0.40 &    6.6 $\pm$    2.7 &    2.4 &  102.9 $\pm$    6.5 &   1.21 $\pm$   0.07 &     11 &   9.63 & 5.64e-01\\
 &      - & - & - &    0.0 & - &  115.4 $\pm$    3.6 &   1.08 $\pm$   0.03 &     12 &  14.83 & 2.51e-01\\
 &   flat & - & - &   11.1 $\pm$    2.5 &    4.4 &   96.7 $\pm$    6.4 &   1.26 $\pm$   0.07 &     11 &  11.91 & 3.71e-01\\
 &      - & - & - &    0.0 & - &  117.5 $\pm$    3.6 &   1.03 $\pm$   0.03 &     12 &  26.77 & 8.33e-03\\
MACS J0417.5-1154 &   extr &     11 &   0.30 &    9.5 $\pm$    6.7 &    1.4 &  101.6 $\pm$   14.8 &   1.52 $\pm$   0.22 &      8 &   0.88 & 9.99e-01\\
 &      - & - & - &    0.0 & - &  117.2 $\pm$    9.2 &   1.29 $\pm$   0.13 &      9 &   2.51 & 9.81e-01\\
 &   flat & - & - &   27.1 $\pm$    7.3 &    3.7 &   99.7 $\pm$   15.1 &   1.42 $\pm$   0.23 &      8 &   1.16 & 9.97e-01\\
 &      - & - & - &    0.0 & - &  136.1 $\pm$    9.4 &   0.85 $\pm$   0.08 &      9 &   7.22 & 6.14e-01\\
MACS J0429.6-0253 &   extr &     15 &   0.40 &   14.8 $\pm$    4.4 &    3.4 &   91.4 $\pm$    9.0 &   1.21 $\pm$   0.11 &     12 &   2.46 & 9.98e-01\\
 &      - & - & - &    0.0 & - &  115.3 $\pm$    4.7 &   0.95 $\pm$   0.05 &     13 &  10.52 & 6.51e-01\\
 &   flat & - & - &   17.2 $\pm$    4.3 &    4.0 &   88.9 $\pm$    9.0 &   1.23 $\pm$   0.11 &     12 &   2.52 & 9.98e-01\\
 &      - & - & - &    0.0 & - &  116.5 $\pm$    4.7 &   0.92 $\pm$   0.05 &     13 &  13.22 & 4.31e-01\\
MACS J0520.7-1328 &   extr &     21 &   0.50 &   88.6 $\pm$   22.0 &    4.0 &   84.9 $\pm$   28.2 &   1.20 $\pm$   0.24 &     18 &   0.75 & 1.00e+00\\
 &      - & - & - &    0.0 & - &  194.8 $\pm$   12.0 &   0.64 $\pm$   0.07 &     19 &   8.63 & 9.79e-01\\
 &   flat & - & - &   88.6 $\pm$   22.0 &    4.0 &   84.9 $\pm$   28.2 &   1.20 $\pm$   0.24 &     18 &   0.75 & 1.00e+00\\
 &      - & - & - &    0.0 & - &  194.8 $\pm$   12.0 &   0.64 $\pm$   0.07 &     19 &   8.63 & 9.79e-01\\
MACS J0547.0-3904 &   extr &     24 &   0.40 &   22.0 $\pm$    4.4 &    5.0 &  122.6 $\pm$   10.2 &   1.19 $\pm$   0.10 &     21 &   7.76 & 9.96e-01\\
 &      - & - & - &    0.0 & - &  153.5 $\pm$    6.9 &   0.84 $\pm$   0.04 &     22 &  23.85 & 3.55e-01\\
 &   flat & - & - &   23.1 $\pm$    4.4 &    5.2 &  121.6 $\pm$   10.2 &   1.20 $\pm$   0.10 &     21 &   7.65 & 9.96e-01\\
 &      - & - & - &    0.0 & - &  153.7 $\pm$    7.0 &   0.83 $\pm$   0.04 &     22 &  25.01 & 2.97e-01\\
MACS J0717.5+3745 &   extr &     16 &   0.50 &  158.7 $\pm$  111.6 &    1.4 &  202.0 $\pm$  128.8 &   0.69 $\pm$   0.35 &     13 &   1.31 & 1.00e+00\\
 &      - & - & - &    0.0 & - &  378.6 $\pm$   26.0 &   0.40 $\pm$   0.07 &     14 &   2.63 & 1.00e+00\\
 &   flat & - & - &  220.1 $\pm$   96.4 &    2.3 &  160.1 $\pm$  112.2 &   0.76 $\pm$   0.40 &     13 &   1.03 & 1.00e+00\\
 &      - & - & - &    0.0 & - &  404.8 $\pm$   25.2 &   0.33 $\pm$   0.06 &     14 &   3.02 & 9.99e-01\\
MACS J0744.8+3927 &   extr &     17 &   0.60 &   39.5 $\pm$   11.0 &    3.6 &  113.9 $\pm$   17.4 &   1.10 $\pm$   0.11 &     14 &   3.84 & 9.96e-01\\
 &      - & - & - &    0.0 & - &  170.4 $\pm$    7.6 &   0.81 $\pm$   0.05 &     15 &  11.91 & 6.86e-01\\
 &   flat & - & - &   42.4 $\pm$   10.9 &    3.9 &  112.0 $\pm$   17.2 &   1.11 $\pm$   0.12 &     14 &   3.88 & 9.96e-01\\
 &      - & - & - &    0.0 & - &  172.6 $\pm$    7.5 &   0.79 $\pm$   0.04 &     15 &  12.98 & 6.04e-01\\
MACS J1115.2+5320 &   extr &     18 &   0.50 &  292.3 $\pm$   60.5 &    4.8 &   27.6 $\pm$   42.3 &   1.73 $\pm$   1.01 &     15 &   3.47 & 9.99e-01\\
 &      - & - & - &    0.0 & - &  334.8 $\pm$   32.1 &   0.33 $\pm$   0.10 &     16 &   6.98 & 9.74e-01\\
 &   flat & - & - &  292.3 $\pm$   60.5 &    4.8 &   27.6 $\pm$   42.3 &   1.73 $\pm$   1.01 &     15 &   3.47 & 9.99e-01\\
 &      - & - & - &    0.0 & - &  334.8 $\pm$   32.1 &   0.33 $\pm$   0.10 &     16 &   6.98 & 9.74e-01\\
MACS J1115.8+0129 &   extr &     20 &   0.20 &   14.1 $\pm$    5.1 &    2.8 &  265.5 $\pm$   18.4 &   1.26 $\pm$   0.11 &     17 &   5.12 & 9.97e-01\\
 &      - & - & - &    0.0 & - &  278.8 $\pm$   17.7 &   1.05 $\pm$   0.06 &     18 &  13.05 & 7.89e-01\\
 &   flat & - & - &   22.7 $\pm$    4.9 &    4.7 &  253.8 $\pm$   18.4 &   1.32 $\pm$   0.12 &     17 &   5.50 & 9.96e-01\\
 &      - & - & - &    0.0 & - &  270.5 $\pm$   17.7 &   0.96 $\pm$   0.05 &     18 &  24.02 & 1.54e-01\\
MACS J1131.8-1955 &   extr &     23 &   0.50 &   62.1 $\pm$   22.3 &    2.8 &  160.9 $\pm$   33.8 &   1.18 $\pm$   0.18 &     20 &   0.40 & 1.00e+00\\
 &      - & - & - &    0.0 & - &  246.2 $\pm$   16.5 &   0.84 $\pm$   0.08 &     21 &   6.22 & 9.99e-01\\
 &   flat & - & - &   97.3 $\pm$   23.0 &    4.2 &  156.3 $\pm$   34.7 &   1.15 $\pm$   0.19 &     20 &   0.69 & 1.00e+00\\
 &      - & - & - &    0.0 & - &  287.7 $\pm$   15.5 &   0.64 $\pm$   0.06 &     21 &   9.81 & 9.81e-01\\
MACS J1149.5+2223 &   extr &     32 &   1.00 &  280.7 $\pm$   39.2 &    7.2 &   33.1 $\pm$   20.6 &   1.47 $\pm$   0.30 &     29 &   1.62 & 1.00e+00\\
 &      - & - & - &    0.0 & - &  282.3 $\pm$   22.1 &   0.52 $\pm$   0.06 &     30 &  15.32 & 9.88e-01\\
 &   flat & - & - &  280.7 $\pm$   39.2 &    7.2 &   33.1 $\pm$   20.6 &   1.47 $\pm$   0.30 &     29 &   1.62 & 1.00e+00\\
 &      - & - & - &    0.0 & - &  282.3 $\pm$   22.1 &   0.52 $\pm$   0.06 &     30 &  15.32 & 9.88e-01\\
MACS J1206.2-0847 &   extr &     30 &   0.80 &   61.0 $\pm$   10.1 &    6.0 &   97.1 $\pm$   14.6 &   1.27 $\pm$   0.11 &     27 &   1.38 & 1.00e+00\\
 &      - & - & - &    0.0 & - &  181.0 $\pm$    8.5 &   0.84 $\pm$   0.05 &     28 &  25.36 & 6.08e-01\\
 &   flat & - & - &   69.0 $\pm$   10.1 &    6.8 &   94.7 $\pm$   14.5 &   1.28 $\pm$   0.11 &     27 &   1.87 & 1.00e+00\\
 &      - & - & - &    0.0 & - &  190.5 $\pm$    8.3 &   0.78 $\pm$   0.05 &     28 &  30.00 & 3.63e-01\\
MACS J1311.0-0310 &   extr &     14 &   0.40 &   42.5 $\pm$    4.2 &   10.1 &   67.1 $\pm$    7.4 &   1.58 $\pm$   0.12 &     11 &   2.47 & 9.96e-01\\
 &      - & - & - &    0.0 & - &  127.7 $\pm$    3.9 &   0.84 $\pm$   0.04 &     12 &  67.11 & 1.11e-09\\
 &   flat & - & - &   47.4 $\pm$    4.1 &   11.5 &   63.5 $\pm$    7.3 &   1.62 $\pm$   0.12 &     11 &   2.39 & 9.97e-01\\
 &      - & - & - &    0.0 & - &  130.2 $\pm$    3.9 &   0.77 $\pm$   0.04 &     12 &  77.77 & 1.10e-11\\
MACS J1621.3+3810 &   extr &     17 &   0.50 &   13.9 $\pm$    5.6 &    2.5 &  135.0 $\pm$   11.6 &   1.16 $\pm$   0.08 &     14 &   6.71 & 9.45e-01\\
 &      - & - & - &    0.0 & - &  158.9 $\pm$    5.8 &   1.01 $\pm$   0.04 &     15 &  11.72 & 7.00e-01\\
 &   flat & - & - &   20.1 $\pm$    5.4 &    3.7 &  129.8 $\pm$   11.4 &   1.18 $\pm$   0.08 &     14 &   7.04 & 9.33e-01\\
 &      - & - & - &    0.0 & - &  164.4 $\pm$    5.8 &   0.96 $\pm$   0.04 &     15 &  16.97 & 3.21e-01\\
MACS J1931.8-2634 &   extr &     16 &   0.40 &   10.3 $\pm$    3.8 &    2.7 &   93.7 $\pm$    9.3 &   1.22 $\pm$   0.10 &     13 &   4.58 & 9.83e-01\\
 &      - & - & - &    0.0 & - &  112.9 $\pm$    5.1 &   1.01 $\pm$   0.05 &     14 &  10.52 & 7.23e-01\\
 &   flat & - & - &   14.6 $\pm$    3.6 &    4.1 &   87.5 $\pm$    9.2 &   1.27 $\pm$   0.11 &     13 &   5.80 & 9.53e-01\\
 &      - & - & - &    0.0 & - &  114.6 $\pm$    5.1 &   0.97 $\pm$   0.04 &     14 &  17.89 & 2.12e-01\\
MACS J2049.9-3217 &   extr &     21 &   0.50 &  195.8 $\pm$   67.6 &    2.9 &   92.7 $\pm$   71.5 &   1.06 $\pm$   0.48 &     18 &   0.87 & 1.00e+00\\
 &      - & - & - &    0.0 & - &  309.0 $\pm$   25.4 &   0.43 $\pm$   0.08 &     19 &   3.69 & 1.00e+00\\
 &   flat & - & - &  195.8 $\pm$   67.6 &    2.9 &   92.7 $\pm$   71.5 &   1.06 $\pm$   0.48 &     18 &   0.87 & 1.00e+00\\
 &      - & - & - &    0.0 & - &  309.0 $\pm$   25.4 &   0.43 $\pm$   0.08 &     19 &   3.69 & 1.00e+00\\
MACS J2211.7-0349 &   extr &     29 &   0.60 &  165.5 $\pm$   25.5 &    6.5 &   78.3 $\pm$   26.3 &   1.59 $\pm$   0.24 &     26 &   0.89 & 1.00e+00\\
 &      - & - & - &    0.0 & - &  270.5 $\pm$   16.5 &   0.74 $\pm$   0.07 &     27 &  20.58 & 8.06e-01\\
 &   flat & - & - &  165.5 $\pm$   25.5 &    6.5 &   78.3 $\pm$   26.3 &   1.59 $\pm$   0.24 &     26 &   0.89 & 1.00e+00\\
 &      - & - & - &    0.0 & - &  270.5 $\pm$   16.5 &   0.74 $\pm$   0.07 &     27 &  20.58 & 8.06e-01\\
MACS J2214.9-1359 &   extr &     13 &   0.40 &  238.6 $\pm$   88.3 &    2.7 &  203.6 $\pm$  152.6 &   1.38 $\pm$   0.66 &     10 &   0.08 & 1.00e+00\\
 &      - & - & - &    0.0 & - &  507.6 $\pm$   70.9 &   0.52 $\pm$   0.16 &     11 &   2.25 & 9.97e-01\\
 &   flat & - & - &  297.7 $\pm$   83.2 &    3.6 &  172.0 $\pm$  147.7 &   1.46 $\pm$   0.76 &     10 &   0.10 & 1.00e+00\\
 &      - & - & - &    0.0 & - &  534.0 $\pm$   73.0 &   0.40 $\pm$   0.14 &     11 &   2.62 & 9.95e-01\\
MACS J2228+2036 &   extr &     22 &   0.60 &  118.8 $\pm$   39.2 &    3.0 &  107.2 $\pm$   45.9 &   1.00 $\pm$   0.26 &     19 &   0.60 & 1.00e+00\\
 &      - & - & - &    0.0 & - &  246.7 $\pm$   17.6 &   0.55 $\pm$   0.07 &     20 &   4.67 & 1.00e+00\\
 &   flat & - & - &  118.8 $\pm$   39.2 &    3.0 &  107.2 $\pm$   45.9 &   1.00 $\pm$   0.26 &     19 &   0.60 & 1.00e+00\\
 &      - & - & - &    0.0 & - &  246.7 $\pm$   17.6 &   0.55 $\pm$   0.07 &     20 &   4.67 & 1.00e+00\\
MACS J2229.7-2755 &   extr &     17 &   0.40 &   10.2 $\pm$    2.1 &    4.8 &   78.1 $\pm$    5.2 &   1.32 $\pm$   0.08 &     14 &  12.45 & 5.70e-01\\
 &      - & - & - &    0.0 & - &   95.0 $\pm$    3.4 &   1.04 $\pm$   0.04 &     15 &  30.08 & 1.16e-02\\
 &   flat & - & - &   12.4 $\pm$    2.0 &    6.1 &   75.0 $\pm$    5.2 &   1.36 $\pm$   0.08 &     14 &  13.61 & 4.79e-01\\
 &      - & - & - &    0.0 & - &   95.4 $\pm$    3.4 &   1.01 $\pm$   0.04 &     15 &  39.96 & 4.60e-04\\
MACS J2245.0+2637 &   extr &     23 &   0.50 &   39.0 $\pm$    6.6 &    5.9 &  108.5 $\pm$   13.1 &   1.31 $\pm$   0.12 &     20 &   0.54 & 1.00e+00\\
 &      - & - & - &    0.0 & - &  166.7 $\pm$    7.2 &   0.82 $\pm$   0.05 &     21 &  23.13 & 3.37e-01\\
 &   flat & - & - &   42.0 $\pm$    6.5 &    6.5 &  105.9 $\pm$   13.1 &   1.33 $\pm$   0.13 &     20 &   0.53 & 1.00e+00\\
 &      - & - & - &    0.0 & - &  168.1 $\pm$    7.2 &   0.79 $\pm$   0.05 &     21 &  25.90 & 2.10e-01\\
MKW3S &   extr &     46 &   0.20 &   20.7 $\pm$    1.7 &   12.1 &  134.8 $\pm$    2.6 &   0.93 $\pm$   0.03 &     43 &  26.23 & 9.80e-01\\
 &      - & - & - &    0.0 & - &  154.3 $\pm$    1.8 &   0.66 $\pm$   0.01 &     44 & 121.79 & 3.16e-09\\
 &   flat & - & - &   23.9 $\pm$    1.6 &   14.7 &  131.1 $\pm$    2.5 &   0.96 $\pm$   0.03 &     43 &  27.65 & 9.67e-01\\
 &      - & - & - &    0.0 & - &  153.5 $\pm$    1.8 &   0.65 $\pm$   0.01 &     44 & 159.12 & 6.08e-15\\
MKW 4 &   extr &     16 &   0.03 &    5.9 $\pm$    0.3 &   18.9 &  368.4 $\pm$   26.7 &   1.21 $\pm$   0.04 &     13 &  17.01 & 1.99e-01\\
 &      - & - & - &    0.0 & - &  164.0 $\pm$    6.7 &   0.74 $\pm$   0.01 &     14 & 233.26 & 8.23e-42\\
 &   flat & - & - &    6.9 $\pm$    0.3 &   23.0 &  392.7 $\pm$   29.4 &   1.26 $\pm$   0.04 &     13 &  19.05 & 1.21e-01\\
 &      - & - & - &    0.0 & - &  146.6 $\pm$    5.9 &   0.70 $\pm$   0.01 &     14 & 305.78 & 7.37e-57\\
MKW 8 &   extr &     19 &   0.05 &  130.7 $\pm$   22.4 &    5.8 &  228.5 $\pm$   54.2 &   0.87 $\pm$   0.40 &     16 &   0.44 & 1.00e+00\\
 &      - & - & - &    0.0 & - &  275.3 $\pm$   16.3 &   0.22 $\pm$   0.03 &     17 &   4.86 & 9.98e-01\\
 &   flat & - & - &  130.7 $\pm$   22.4 &    5.8 &  228.5 $\pm$   54.2 &   0.87 $\pm$   0.40 &     16 &   0.44 & 1.00e+00\\
 &      - & - & - &    0.0 & - &  275.3 $\pm$   16.3 &   0.22 $\pm$   0.03 &     17 &   4.86 & 9.98e-01\\
MS J0016.9+1609 &   extr &     16 &   0.50 &  160.7 $\pm$   22.6 &    7.1 &   65.0 $\pm$   26.7 &   1.28 $\pm$   0.30 &     13 &   3.17 & 9.97e-01\\
 &      - & - & - &    0.0 & - &  258.5 $\pm$   11.8 &   0.40 $\pm$   0.05 &     14 &  15.63 & 3.37e-01\\
 &   flat & - & - &  162.1 $\pm$   22.5 &    7.2 &   64.2 $\pm$   26.5 &   1.29 $\pm$   0.30 &     13 &   3.17 & 9.97e-01\\
 &      - & - & - &    0.0 & - &  259.3 $\pm$   11.7 &   0.40 $\pm$   0.05 &     14 &  15.74 & 3.30e-01\\
MS J0116.3-0115 &   extr &     22 &   0.10 &   17.2 $\pm$   32.0 &    0.5 &  214.2 $\pm$   24.7 &   0.62 $\pm$   0.23 &     19 &   2.51 & 1.00e+00\\
 &      - & - & - &    0.0 & - &  225.3 $\pm$   14.8 &   0.52 $\pm$   0.05 &     20 &   3.02 & 1.00e+00\\
 &   flat & - & - &   12.8 $\pm$   31.0 &    0.4 &  220.8 $\pm$   24.1 &   0.63 $\pm$   0.22 &     19 &   2.53 & 1.00e+00\\
 &      - & - & - &    0.0 & - &  228.7 $\pm$   15.1 &   0.55 $\pm$   0.05 &     20 &   2.96 & 1.00e+00\\
MS J0440.5+0204 &   extr &     19 &   0.30 &   22.8 $\pm$    7.6 &    3.0 &  165.5 $\pm$   15.1 &   1.11 $\pm$   0.13 &     16 &   5.73 & 9.91e-01\\
 &      - & - & - &    0.0 & - &  196.6 $\pm$    9.6 &   0.82 $\pm$   0.06 &     17 &  11.13 & 8.50e-01\\
 &   flat & - & - &   25.5 $\pm$    7.6 &    3.4 &  164.0 $\pm$   15.2 &   1.11 $\pm$   0.13 &     16 &   6.15 & 9.86e-01\\
 &      - & - & - &    0.0 & - &  198.0 $\pm$    9.6 &   0.79 $\pm$   0.05 &     17 &  12.34 & 7.79e-01\\
MS J0451.6-0305 &   extr &     16 &   0.50 &  568.1 $\pm$  115.6 &    4.9 &   15.6 $\pm$   49.9 &   2.81 $\pm$   2.27 &     13 &   0.56 & 1.00e+00\\
 &      - & - & - &    0.0 & - &  643.5 $\pm$   79.7 &   0.21 $\pm$   0.16 &     14 &   3.73 & 9.97e-01\\
 &   flat & - & - &  568.1 $\pm$  115.6 &    4.9 &   15.6 $\pm$   49.9 &   2.81 $\pm$   2.27 &     13 &   0.56 & 1.00e+00\\
 &      - & - & - &    0.0 & - &  643.5 $\pm$   79.7 &   0.21 $\pm$   0.16 &     14 &   3.73 & 9.97e-01\\
MS J0735.6+7421 &   extr &     18 &   0.30 &   13.8 $\pm$    2.2 &    6.3 &  109.9 $\pm$    4.6 &   1.12 $\pm$   0.05 &     15 &  22.06 & 1.06e-01\\
 &      - & - & - &    0.0 & - &  131.3 $\pm$    2.7 &   0.89 $\pm$   0.02 &     16 &  60.72 & 3.95e-07\\
 &   flat & - & - &   16.0 $\pm$    2.1 &    7.5 &  106.8 $\pm$    4.6 &   1.14 $\pm$   0.05 &     15 &  25.59 & 4.26e-02\\
 &      - & - & - &    0.0 & - &  131.5 $\pm$    2.7 &   0.87 $\pm$   0.02 &     16 &  77.93 & 3.92e-10\\
MS J0839.8+2938 &   extr &     16 &   0.25 &   15.5 $\pm$    3.1 &    5.1 &  110.7 $\pm$    6.3 &   1.26 $\pm$   0.11 &     13 &   3.12 & 9.98e-01\\
 &      - & - & - &    0.0 & - &  127.3 $\pm$    4.9 &   0.88 $\pm$   0.04 &     14 &  21.16 & 9.75e-02\\
 &   flat & - & - &   19.2 $\pm$    2.9 &    6.7 &  105.8 $\pm$    6.3 &   1.33 $\pm$   0.11 &     13 &   2.50 & 9.99e-01\\
 &      - & - & - &    0.0 & - &  126.1 $\pm$    4.9 &   0.84 $\pm$   0.04 &     14 &  30.67 & 6.17e-03\\
MS J0906.5+1110 &   extr &     29 &   0.40 &  104.2 $\pm$   14.9 &    7.0 &   97.3 $\pm$   19.6 &   1.15 $\pm$   0.17 &     26 &   1.25 & 1.00e+00\\
 &      - & - & - &    0.0 & - &  222.7 $\pm$    6.4 &   0.54 $\pm$   0.04 &     27 &  19.62 & 8.46e-01\\
 &   flat & - & - &  104.2 $\pm$   14.9 &    7.0 &   97.3 $\pm$   19.6 &   1.15 $\pm$   0.17 &     26 &   1.25 & 1.00e+00\\
 &      - & - & - &    0.0 & - &  222.7 $\pm$    6.4 &   0.54 $\pm$   0.04 &     27 &  19.62 & 8.46e-01\\
MS J1006.0+1202 &   extr &     29 &   0.50 &  175.8 $\pm$   20.1 &    8.7 &   71.7 $\pm$   25.0 &   1.40 $\pm$   0.26 &     26 &   7.00 & 1.00e+00\\
 &      - & - & - &    0.0 & - &  285.4 $\pm$   12.1 &   0.41 $\pm$   0.05 &     27 &  29.77 & 3.25e-01\\
 &   flat & - & - &  160.3 $\pm$   21.3 &    7.5 &   82.8 $\pm$   26.9 &   1.32 $\pm$   0.24 &     26 &   6.68 & 1.00e+00\\
 &      - & - & - &    0.0 & - &  278.4 $\pm$   12.2 &   0.46 $\pm$   0.05 &     27 &  26.32 & 5.01e-01\\
MS J1008.1-1224 &   extr &     23 &   0.50 &   96.0 $\pm$   40.7 &    2.4 &  260.2 $\pm$   56.0 &   0.77 $\pm$   0.18 &     20 &   1.45 & 1.00e+00\\
 &      - & - & - &    0.0 & - &  373.9 $\pm$   18.0 &   0.49 $\pm$   0.05 &     21 &   4.07 & 1.00e+00\\
 &   flat & - & - &   97.6 $\pm$   41.5 &    2.4 &  262.0 $\pm$   56.8 &   0.76 $\pm$   0.18 &     20 &   1.50 & 1.00e+00\\
 &      - & - & - &    0.0 & - &  377.0 $\pm$   18.1 &   0.48 $\pm$   0.05 &     21 &   4.07 & 1.00e+00\\
MS J1455.0+2232 &   extr &     16 &   0.30 &   16.9 $\pm$    1.5 &   11.1 &   81.5 $\pm$    4.0 &   1.39 $\pm$   0.07 &     13 &  10.09 & 6.86e-01\\
 &      - & - & - &    0.0 & - &  107.3 $\pm$    2.7 &   0.86 $\pm$   0.03 &     14 &  80.05 & 2.76e-11\\
 &   flat & - & - &   16.9 $\pm$    1.5 &   11.1 &   81.5 $\pm$    4.0 &   1.39 $\pm$   0.07 &     13 &  10.09 & 6.86e-01\\
 &      - & - & - &    0.0 & - &  107.3 $\pm$    2.7 &   0.86 $\pm$   0.03 &     14 &  80.05 & 2.76e-11\\
MS J2137.3-2353 &   extr &     22 &   0.50 &   12.3 $\pm$    1.9 &    6.5 &   93.5 $\pm$    5.3 &   1.36 $\pm$   0.06 &     19 &   5.01 & 9.99e-01\\
 &      - & - & - &    0.0 & - &  116.9 $\pm$    3.4 &   1.08 $\pm$   0.03 &     20 &  36.15 & 1.47e-02\\
 &   flat & - & - &   14.7 $\pm$    1.8 &    7.9 &   89.9 $\pm$    5.3 &   1.39 $\pm$   0.06 &     19 &   5.76 & 9.98e-01\\
 &      - & - & - &    0.0 & - &  117.6 $\pm$    3.4 &   1.05 $\pm$   0.03 &     20 &  50.37 & 1.96e-04\\
MS J1157.3+5531 &   extr &     13 &   0.10 &    4.1 $\pm$    0.4 &    9.7 &  283.8 $\pm$   17.7 &   1.44 $\pm$   0.05 &     10 &   7.54 & 6.74e-01\\
 &      - & - & - &    0.0 & - &  196.2 $\pm$    9.6 &   1.09 $\pm$   0.02 &     11 &  64.85 & 1.15e-09\\
 &   flat & - & - &    5.9 $\pm$    0.4 &   13.9 &  277.0 $\pm$   17.7 &   1.45 $\pm$   0.05 &     10 &   7.22 & 7.04e-01\\
 &      - & - & - &    0.0 & - &  160.6 $\pm$    7.7 &   0.95 $\pm$   0.02 &     11 &  96.24 & 9.86e-16\\
NGC 507 &   extr &     61 &   0.05 &    0.0 $\pm$    2.1 &    0.0 &  101.7 $\pm$    2.8 &   0.67 $\pm$   0.01 &     58 &  42.84 & 9.32e-01\\
 &      - & - & - &    0.0 & - &  101.7 $\pm$    2.8 &   0.67 $\pm$   0.01 &     59 &  42.84 & 9.44e-01\\
 &   flat & - & - &    0.0 $\pm$    2.1 &    0.0 &   99.9 $\pm$    2.7 &   0.65 $\pm$   0.01 &     58 &  46.55 & 8.60e-01\\
 &      - & - & - &    0.0 & - &   99.9 $\pm$    2.7 &   0.65 $\pm$   0.01 &     59 &  46.55 & 8.80e-01\\
NGC 4636 &   extr &     12 &   0.00 &    1.4 $\pm$    0.1 &   13.4 & 10674.9 $\pm$ 7937.9 &   1.93 $\pm$   0.18 &      9 &   8.12 & 5.22e-01\\
 &      - & - & - &    0.0 & - &  108.2 $\pm$   19.2 &   0.77 $\pm$   0.04 &     10 &  56.25 & 1.84e-08\\
 &   flat & - & - &    1.4 $\pm$    0.1 &   13.9 & 11962.1 $\pm$ 8977.0 &   1.96 $\pm$   0.18 &      9 &   8.95 & 4.42e-01\\
 &      - & - & - &    0.0 & - &  104.9 $\pm$   18.6 &   0.77 $\pm$   0.04 &     10 &  60.03 & 3.58e-09\\
NGC 5044 &   extr &     66 &   0.03 &    1.9 $\pm$    0.3 &    7.2 &   79.6 $\pm$    6.7 &   0.93 $\pm$   0.05 &     63 &  49.49 & 8.93e-01\\
 &      - & - & - &    0.0 & - &   55.1 $\pm$    2.4 &   0.67 $\pm$   0.02 &     64 &  77.04 & 1.27e-01\\
 &   flat & - & - &    2.3 $\pm$    0.3 &    8.9 &   82.2 $\pm$    7.2 &   0.96 $\pm$   0.05 &     63 &  48.05 & 9.18e-01\\
 &      - & - & - &    0.0 & - &   52.3 $\pm$    2.2 &   0.64 $\pm$   0.02 &     64 &  86.52 & 3.19e-02\\
NGC 5813 &   extr &     60 &   0.02 &    1.4 $\pm$    0.2 &    8.9 &  102.5 $\pm$    7.1 &   0.91 $\pm$   0.03 &     57 & 107.52 & 6.00e-05\\
 &      - & - & - &    0.0 & - &   69.3 $\pm$    2.1 &   0.70 $\pm$   0.01 &     58 & 161.30 & 1.14e-11\\
 &   flat & - & - &    1.4 $\pm$    0.2 &    8.9 &  102.5 $\pm$    7.1 &   0.91 $\pm$   0.03 &     57 & 107.52 & 6.00e-05\\
 &      - & - & - &    0.0 & - &   69.3 $\pm$    2.1 &   0.70 $\pm$   0.01 &     58 & 161.30 & 1.14e-11\\
NGC 5846 &   extr &     16 &   0.00 &    1.8 $\pm$    0.2 &   10.7 &  685.8 $\pm$  344.9 &   1.44 $\pm$   0.15 &     13 &   1.16 & 1.00e+00\\
 &      - & - & - &    0.0 & - &   52.7 $\pm$    7.3 &   0.63 $\pm$   0.03 &     14 &  40.72 & 1.97e-04\\
 &   flat & - & - &    1.8 $\pm$    0.2 &   10.7 &  685.8 $\pm$  344.9 &   1.44 $\pm$   0.15 &     13 &   1.16 & 1.00e+00\\
 &      - & - & - &    0.0 & - &   52.7 $\pm$    7.3 &   0.63 $\pm$   0.03 &     14 &  40.72 & 1.97e-04\\
Ophiuchus &   extr &     18 &   0.05 &    4.0 $\pm$    0.6 &    6.3 &  375.1 $\pm$   12.8 &   1.06 $\pm$   0.03 &     15 &   9.75 & 8.35e-01\\
 &      - & - & - &    0.0 & - &  328.4 $\pm$    7.8 &   0.92 $\pm$   0.01 &     16 &  42.24 & 3.63e-04\\
 &   flat & - & - &    8.9 $\pm$    1.2 &    7.5 &  247.5 $\pm$    7.6 &   0.73 $\pm$   0.03 &     15 &  95.06 & 1.12e-13\\
 &      - & - & - &    0.0 & - &  217.0 $\pm$    3.9 &   0.58 $\pm$   0.01 &     16 & 127.43 & 2.02e-19\\
PKS 0745-191 &   extr &     34 &   0.30 &   11.9 $\pm$    0.7 &   17.4 &  111.7 $\pm$    2.7 &   1.38 $\pm$   0.04 &     31 &  17.17 & 9.79e-01\\
 &      - & - & - &    0.0 & - &  129.2 $\pm$    2.4 &   0.98 $\pm$   0.02 &     32 & 245.68 & 8.53e-35\\
 &   flat & - & - &   12.4 $\pm$    0.7 &   18.3 &  110.7 $\pm$    2.7 &   1.39 $\pm$   0.04 &     31 &  19.54 & 9.45e-01\\
 &      - & - & - &    0.0 & - &  128.9 $\pm$    2.4 &   0.97 $\pm$   0.02 &     32 & 270.30 & 1.59e-39\\
RBS 461 &   extr &     70 &   0.20 &   95.7 $\pm$    3.0 &   31.4 &   68.8 $\pm$    4.5 &   1.39 $\pm$   0.10 &     67 &  22.14 & 1.00e+00\\
 &      - & - & - &    0.0 & - &  173.2 $\pm$    1.8 &   0.35 $\pm$   0.01 &     68 & 217.68 & 1.45e-17\\
 &   flat & - & - &   95.7 $\pm$    3.0 &   31.4 &   68.8 $\pm$    4.5 &   1.39 $\pm$   0.10 &     67 &  22.14 & 1.00e+00\\
 &      - & - & - &    0.0 & - &  173.2 $\pm$    1.8 &   0.35 $\pm$   0.01 &     68 & 217.68 & 1.45e-17\\
RBS 533 &   extr &     44 &   0.06 &    2.0 $\pm$   0.05 &   39.5 &  162.8 $\pm$    2.5 &   0.99 $\pm$   0.01 &     41 & 202.89 & 2.65e-23\\
 &      - & - & - &    0.0 & - &  113.5 $\pm$    1.3 &   0.76 $\pm$   0.00 &     42 & 1282.66 & 1.75e-241\\
 &   flat & - & - &    2.2 $\pm$   0.05 &   43.7 &  164.3 $\pm$    2.5 &   1.00 $\pm$   0.01 &     41 & 215.65 & 1.46e-25\\
 &      - & - & - &    0.0 & - &  110.0 $\pm$    1.3 &   0.75 $\pm$   0.00 &     42 & 1490.02 & 3.27e-285\\
RBS 797 &   extr &     24 &   0.30 &   20.0 $\pm$    2.4 &    8.3 &   95.2 $\pm$    9.0 &   1.72 $\pm$   0.14 &     21 &  89.64 & 1.86e-10\\
 &      - & - & - &    0.0 & - &  116.2 $\pm$    8.0 &   0.98 $\pm$   0.06 &     22 & 1061.58 & 1.51e-210\\
 &   flat & - & - &   20.9 $\pm$    2.4 &    8.9 &   93.2 $\pm$    9.1 &   1.75 $\pm$   0.15 &     21 & 104.70 & 4.22e-13\\
 &      - & - & - &    0.0 & - &  114.6 $\pm$    8.0 &   0.96 $\pm$   0.06 &     22 & 1188.56 & 1.25e-237\\
RCS J2327-0204 &   extr &     18 &   0.30 &   65.5 $\pm$   20.2 &    3.2 &  220.6 $\pm$   37.0 &   1.27 $\pm$   0.25 &     15 &  31.21 & 8.24e-03\\
 &      - & - & - &    0.0 & - &  300.3 $\pm$   22.5 &   0.74 $\pm$   0.09 &     16 & 119.10 & 8.17e-18\\
 &   flat & - & - &   68.5 $\pm$   19.9 &    3.4 &  217.2 $\pm$   36.9 &   1.28 $\pm$   0.26 &     15 &  31.00 & 8.80e-03\\
 &      - & - & - &    0.0 & - &  300.1 $\pm$   22.6 &   0.73 $\pm$   0.09 &     16 & 126.00 & 3.83e-19\\
RXCJ0331.1-2100 &   extr &     25 &   0.20 &    6.4 $\pm$    1.6 &    4.1 &  141.0 $\pm$    5.8 &   1.23 $\pm$   0.06 &     22 & 325.76 & 7.05e-56\\
 &      - & - & - &    0.0 & - &  145.9 $\pm$    5.7 &   1.05 $\pm$   0.03 &     23 & 677.70 & 2.20e-128\\
 &   flat & - & - &   11.4 $\pm$    1.5 &    7.7 &  134.1 $\pm$    5.8 &   1.30 $\pm$   0.07 &     22 & 356.18 & 4.25e-62\\
 &      - & - & - &    0.0 & - &  140.5 $\pm$    5.7 &   0.95 $\pm$   0.03 &     23 & 1408.70 & 8.65e-284\\
RX J0220.9-3829 &   extr &     22 &   0.40 &   33.1 $\pm$    6.2 &    5.3 &  163.7 $\pm$   14.0 &   1.25 $\pm$   0.11 &     19 &   3.90 & 1.00e+00\\
 &      - & - & - &    0.0 & - &  211.1 $\pm$    9.0 &   0.84 $\pm$   0.05 &     20 &  20.59 & 4.22e-01\\
 &   flat & - & - &   43.0 $\pm$    6.3 &    6.8 &  159.9 $\pm$   14.0 &   1.23 $\pm$   0.12 &     19 &   4.20 & 1.00e+00\\
 &      - & - & - &    0.0 & - &  216.2 $\pm$    9.2 &   0.73 $\pm$   0.04 &     20 &  25.95 & 1.68e-01\\
RX J0232.2-4420 &   extr &     14 &   0.30 &   34.2 $\pm$   13.0 &    2.6 &  176.3 $\pm$   25.0 &   1.12 $\pm$   0.18 &     11 &   0.85 & 1.00e+00\\
 &      - & - & - &    0.0 & - &  225.4 $\pm$   13.1 &   0.80 $\pm$   0.06 &     12 &   5.16 & 9.53e-01\\
 &   flat & - & - &   44.6 $\pm$   12.4 &    3.6 &  166.5 $\pm$   24.7 &   1.16 $\pm$   0.18 &     11 &   0.71 & 1.00e+00\\
 &      - & - & - &    0.0 & - &  228.9 $\pm$   13.2 &   0.74 $\pm$   0.06 &     12 &   7.42 & 8.28e-01\\
RX J0439+0520 &   extr &     18 &   0.30 &   12.8 $\pm$    2.9 &    4.5 &   97.1 $\pm$    6.2 &   1.18 $\pm$   0.10 &     15 &   6.80 & 9.63e-01\\
 &      - & - & - &    0.0 & - &  112.8 $\pm$    4.6 &   0.86 $\pm$   0.04 &     16 &  19.20 & 2.59e-01\\
 &   flat & - & - &   14.9 $\pm$    2.9 &    5.2 &   95.5 $\pm$    6.2 &   1.19 $\pm$   0.10 &     15 &   6.64 & 9.67e-01\\
 &      - & - & - &    0.0 & - &  113.0 $\pm$    4.6 &   0.82 $\pm$   0.04 &     16 &  21.93 & 1.45e-01\\
RX J0439.0+0715 &   extr &     22 &   0.40 &   61.2 $\pm$   21.3 &    2.9 &  152.0 $\pm$   31.1 &   0.95 $\pm$   0.18 &     19 &   5.54 & 9.99e-01\\
 &      - & - & - &    0.0 & - &  212.0 $\pm$   10.6 &   0.68 $\pm$   0.06 &     20 &   8.75 & 9.86e-01\\
 &   flat & - & - &   66.8 $\pm$   18.5 &    3.6 &  129.6 $\pm$   28.4 &   1.06 $\pm$   0.20 &     19 &   6.20 & 9.97e-01\\
 &      - & - & - &    0.0 & - &  217.0 $\pm$   10.5 &   0.63 $\pm$   0.06 &     20 &  13.41 & 8.59e-01\\
RX J0528.9-3927 &   extr &     21 &   0.40 &   69.9 $\pm$   13.9 &    5.0 &  102.2 $\pm$   22.6 &   1.45 $\pm$   0.23 &     18 &   1.71 & 1.00e+00\\
 &      - & - & - &    0.0 & - &  201.5 $\pm$   11.3 &   0.74 $\pm$   0.08 &     19 &  15.10 & 7.16e-01\\
 &   flat & - & - &   72.9 $\pm$   13.8 &    5.3 &   99.8 $\pm$   22.4 &   1.47 $\pm$   0.23 &     18 &   1.67 & 1.00e+00\\
 &      - & - & - &    0.0 & - &  203.1 $\pm$   11.3 &   0.72 $\pm$   0.07 &     19 &  15.94 & 6.61e-01\\
RX J0647.7+7015 &   extr &     24 &   0.80 &  225.1 $\pm$   47.1 &    4.8 &   48.8 $\pm$   31.9 &   1.70 $\pm$   0.39 &     21 &   0.42 & 1.00e+00\\
 &      - & - & - &    0.0 & - &  275.6 $\pm$   32.0 &   0.71 $\pm$   0.10 &     22 &   9.72 & 9.89e-01\\
 &   flat & - & - &  225.1 $\pm$   47.1 &    4.8 &   48.8 $\pm$   31.9 &   1.70 $\pm$   0.39 &     21 &   0.42 & 1.00e+00\\
 &      - & - & - &    0.0 & - &  275.6 $\pm$   32.0 &   0.71 $\pm$   0.10 &     22 &   9.72 & 9.89e-01\\
RX J0819.6+6336 &   extr &     28 &   0.30 &   20.7 $\pm$   14.3 &    1.5 &  170.6 $\pm$   19.4 &   0.68 $\pm$   0.12 &     25 &  10.13 & 9.96e-01\\
 &      - & - & - &    0.0 & - &  194.0 $\pm$    8.8 &   0.55 $\pm$   0.04 &     26 &  11.55 & 9.93e-01\\
 &   flat & - & - &   20.7 $\pm$   14.3 &    1.5 &  170.6 $\pm$   19.4 &   0.68 $\pm$   0.12 &     25 &  10.13 & 9.96e-01\\
 &      - & - & - &    0.0 & - &  194.0 $\pm$    8.8 &   0.55 $\pm$   0.04 &     26 &  11.55 & 9.93e-01\\
RX J1000.4+4409 &   extr &     23 &   0.30 &   23.1 $\pm$    4.3 &    5.4 &  151.7 $\pm$    9.9 &   1.12 $\pm$   0.09 &     20 &   1.85 & 1.00e+00\\
 &      - & - & - &    0.0 & - &  182.2 $\pm$    7.1 &   0.77 $\pm$   0.04 &     21 &  18.65 & 6.07e-01\\
 &   flat & - & - &   27.7 $\pm$    4.4 &    6.3 &  151.1 $\pm$    9.9 &   1.09 $\pm$   0.09 &     20 &   1.94 & 1.00e+00\\
 &      - & - & - &    0.0 & - &  184.9 $\pm$    7.2 &   0.71 $\pm$   0.03 &     21 &  21.59 & 4.24e-01\\
RX J1022.1+3830 &   extr &     18 &   0.09 &   44.0 $\pm$   10.0 &    4.4 &  206.8 $\pm$   18.5 &   1.03 $\pm$   0.21 &     15 &   7.73 & 9.34e-01\\
 &      - & - & - &    0.0 & - &  208.7 $\pm$   11.4 &   0.54 $\pm$   0.04 &     16 &  13.56 & 6.32e-01\\
 &   flat & - & - &   51.6 $\pm$    9.8 &    5.3 &  194.8 $\pm$   18.7 &   1.04 $\pm$   0.22 &     15 &   8.26 & 9.13e-01\\
 &      - & - & - &    0.0 & - &  201.1 $\pm$   10.7 &   0.48 $\pm$   0.04 &     16 &  14.68 & 5.48e-01\\
RX J1130.0+3637 &   extr &     26 &   0.15 &   23.4 $\pm$    2.2 &   10.7 &  158.7 $\pm$    9.3 &   1.19 $\pm$   0.09 &     23 &   2.01 & 1.00e+00\\
 &      - & - & - &    0.0 & - &  140.8 $\pm$    6.7 &   0.60 $\pm$   0.03 &     24 &  54.32 & 3.86e-04\\
 &   flat & - & - &   29.9 $\pm$    2.3 &   12.9 &  149.6 $\pm$    9.2 &   1.14 $\pm$   0.10 &     23 &   2.81 & 1.00e+00\\
 &      - & - & - &    0.0 & - &  133.0 $\pm$    6.0 &   0.48 $\pm$   0.02 &     24 &  58.11 & 1.18e-04\\
RX J1320.2+3308 &   extr &     11 &   0.04 &    7.6 $\pm$    0.6 &   12.1 &  162.6 $\pm$   26.6 &   1.36 $\pm$   0.12 &      8 &   5.25 & 7.31e-01\\
 &      - & - & - &    0.0 & - &   67.6 $\pm$    4.2 &   0.61 $\pm$   0.03 &      9 &  50.82 & 7.56e-08\\
 &   flat & - & - &    8.8 $\pm$    0.7 &   13.1 &  140.3 $\pm$   23.4 &   1.28 $\pm$   0.12 &      8 &   7.01 & 5.36e-01\\
 &      - & - & - &    0.0 & - &   59.9 $\pm$    3.4 &   0.53 $\pm$   0.02 &      9 &  49.88 & 1.13e-07\\
RX J1347.5-1145 &   extr &      8 &   0.22 &   12.5 $\pm$   20.7 &    0.6 &  179.9 $\pm$   35.3 &   1.06 $\pm$   0.34 &      5 &   4.00 & 5.49e-01\\
 &      - & - & - &    0.0 & - &  196.4 $\pm$   18.3 &   0.90 $\pm$   0.08 &      6 &   4.23 & 6.46e-01\\
 &   flat & - & - &   12.5 $\pm$   20.7 &    0.6 &  179.9 $\pm$   35.3 &   1.06 $\pm$   0.34 &      5 &   4.00 & 5.49e-01\\
 &      - & - & - &    0.0 & - &  196.4 $\pm$   18.3 &   0.90 $\pm$   0.08 &      6 &   4.23 & 6.46e-01\\
RX J1423.8+2404 &   extr &      7 &   0.22 &   10.2 $\pm$    5.0 &    2.0 &  119.9 $\pm$   10.8 &   1.27 $\pm$   0.17 &      4 &   1.75 & 7.82e-01\\
 &      - & - & - &    0.0 & - &  133.8 $\pm$    7.3 &   1.02 $\pm$   0.05 &      5 &  15.01 & 1.03e-02\\
 &   flat & - & - &   10.2 $\pm$    5.0 &    2.0 &  119.9 $\pm$   10.8 &   1.27 $\pm$   0.17 &      4 &   1.75 & 7.82e-01\\
 &      - & - & - &    0.0 & - &  133.8 $\pm$    7.3 &   1.02 $\pm$   0.05 &      5 &  15.01 & 1.03e-02\\
RX J1504.1-0248 &   extr &     27 &   0.45 &   13.1 $\pm$    0.9 &   13.9 &   95.6 $\pm$    3.5 &   1.50 $\pm$   0.04 &     24 &   2.89 & 1.00e+00\\
 &      - & - & - &    0.0 & - &  121.2 $\pm$    2.7 &   1.09 $\pm$   0.02 &     25 & 154.86 & 1.07e-20\\
 &   flat & - & - &   13.1 $\pm$    0.9 &   13.9 &   95.6 $\pm$    3.5 &   1.50 $\pm$   0.04 &     24 &   2.89 & 1.00e+00\\
 &      - & - & - &    0.0 & - &  121.2 $\pm$    2.7 &   1.09 $\pm$   0.02 &     25 & 154.86 & 1.07e-20\\
RX J1532.9+3021 &   extr &     21 &   0.50 &   14.3 $\pm$    1.9 &    7.6 &   80.3 $\pm$    5.0 &   1.46 $\pm$   0.07 &     18 &   2.24 & 1.00e+00\\
 &      - & - & - &    0.0 & - &  105.6 $\pm$    3.3 &   1.08 $\pm$   0.04 &     19 &  48.03 & 2.54e-04\\
 &   flat & - & - &   16.9 $\pm$    1.8 &    9.3 &   76.3 $\pm$    5.0 &   1.51 $\pm$   0.07 &     18 &   2.38 & 1.00e+00\\
 &      - & - & - &    0.0 & - &  106.1 $\pm$    3.3 &   1.04 $\pm$   0.04 &     19 &  67.16 & 2.71e-07\\
RX J1539.5-8335 &   extr &     29 &   0.20 &   21.8 $\pm$    3.1 &    7.1 &  115.1 $\pm$    5.8 &   1.32 $\pm$   0.11 &     26 &  13.29 & 9.81e-01\\
 &      - & - & - &    0.0 & - &  135.3 $\pm$    4.5 &   0.83 $\pm$   0.04 &     27 &  40.39 & 4.71e-02\\
 &   flat & - & - &   25.9 $\pm$    2.9 &    9.1 &  110.0 $\pm$    5.8 &   1.41 $\pm$   0.12 &     26 &  13.52 & 9.79e-01\\
 &      - & - & - &    0.0 & - &  133.7 $\pm$    4.5 &   0.79 $\pm$   0.04 &     27 &  54.08 & 1.49e-03\\
RX J1720.1+2638 &   extr &     30 &   0.40 &   20.7 $\pm$    1.9 &   10.7 &  109.7 $\pm$    5.4 &   1.38 $\pm$   0.06 &     27 &   5.34 & 1.00e+00\\
 &      - & - & - &    0.0 & - &  145.3 $\pm$    3.6 &   0.98 $\pm$   0.03 &     28 &  94.37 & 4.06e-09\\
 &   flat & - & - &   21.0 $\pm$    1.9 &   10.9 &  109.1 $\pm$    5.4 &   1.39 $\pm$   0.06 &     27 &   5.56 & 1.00e+00\\
 &      - & - & - &    0.0 & - &  145.3 $\pm$    3.6 &   0.98 $\pm$   0.03 &     28 &  97.94 & 1.09e-09\\
RX J1720.2+3536 &   extr &     13 &   0.32 &   17.5 $\pm$    3.5 &    4.9 &  101.8 $\pm$    7.9 &   1.35 $\pm$   0.10 &     10 &   2.47 & 9.91e-01\\
 &      - & - & - &    0.0 & - &  129.4 $\pm$    4.7 &   1.00 $\pm$   0.04 &     11 &  23.76 & 1.38e-02\\
 &   flat & - & - &   24.0 $\pm$    3.3 &    7.2 &   94.4 $\pm$    7.8 &   1.42 $\pm$   0.11 &     10 &   2.67 & 9.88e-01\\
 &      - & - & - &    0.0 & - &  131.3 $\pm$    4.7 &   0.92 $\pm$   0.04 &     11 &  40.43 & 3.02e-05\\
RX J1852.1+5711 &   extr &     12 &   0.12 &   13.7 $\pm$    6.3 &    2.2 &  184.3 $\pm$   12.8 &   0.96 $\pm$   0.15 &      9 &   2.63 & 9.77e-01\\
 &      - & - & - &    0.0 & - &  182.4 $\pm$   10.9 &   0.73 $\pm$   0.05 &     10 &   5.31 & 8.70e-01\\
 &   flat & - & - &   18.7 $\pm$    8.3 &    2.3 &  170.4 $\pm$   11.8 &   0.83 $\pm$   0.16 &      9 &   5.06 & 8.29e-01\\
 &      - & - & - &    0.0 & - &  173.3 $\pm$    9.8 &   0.58 $\pm$   0.04 &     10 &   7.26 & 7.01e-01\\
RX J2129.6+0005 &   extr &     22 &   0.40 &   18.0 $\pm$    3.8 &    4.7 &  100.8 $\pm$    8.1 &   1.24 $\pm$   0.10 &     19 &   7.01 & 9.94e-01\\
 &      - & - & - &    0.0 & - &  129.2 $\pm$    4.8 &   0.91 $\pm$   0.05 &     20 &  21.36 & 3.76e-01\\
 &   flat & - & - &   21.1 $\pm$    3.7 &    5.7 &   97.9 $\pm$    8.0 &   1.26 $\pm$   0.10 &     19 &   7.16 & 9.93e-01\\
 &      - & - & - &    0.0 & - &  130.8 $\pm$    4.8 &   0.87 $\pm$   0.04 &     20 &  26.01 & 1.66e-01\\
SC 1327-312 &   extr &     31 &   0.15 &   65.5 $\pm$   10.1 &    6.5 &  160.4 $\pm$   12.5 &   0.80 $\pm$   0.14 &     28 &   1.08 & 1.00e+00\\
 &      - & - & - &    0.0 & - &  212.5 $\pm$    8.1 &   0.36 $\pm$   0.03 &     29 &  15.85 & 9.77e-01\\
 &   flat & - & - &   64.6 $\pm$    9.9 &    6.5 &  160.8 $\pm$   12.5 &   0.81 $\pm$   0.14 &     28 &   1.03 & 1.00e+00\\
 &      - & - & - &    0.0 & - &  212.0 $\pm$    8.1 &   0.37 $\pm$   0.03 &     29 &  16.01 & 9.75e-01\\
Sersic 159-03 &   extr &     23 &   0.12 &    7.5 $\pm$    0.8 &    9.7 &   79.7 $\pm$    2.3 &   1.06 $\pm$   0.05 &     20 &  15.95 & 7.20e-01\\
 &      - & - & - &    0.0 & - &   77.9 $\pm$    2.0 &   0.72 $\pm$   0.02 &     21 &  77.11 & 2.44e-08\\
 &   flat & - & - &   10.5 $\pm$    0.7 &   15.0 &   77.8 $\pm$    2.4 &   1.17 $\pm$   0.06 &     20 &  16.81 & 6.65e-01\\
 &      - & - & - &    0.0 & - &   74.0 $\pm$    1.9 &   0.65 $\pm$   0.02 &     21 & 136.22 & 7.00e-19\\
SS2B153 &   extr &     38 &   0.07 &    1.1 $\pm$    0.2 &    6.9 &   71.4 $\pm$    2.1 &   0.80 $\pm$   0.02 &     35 &  24.19 & 9.15e-01\\
 &      - & - & - &    0.0 & - &   63.4 $\pm$    1.4 &   0.69 $\pm$   0.01 &     36 &  59.46 & 8.24e-03\\
 &   flat & - & - &    1.1 $\pm$    0.2 &    6.9 &   71.4 $\pm$    2.1 &   0.80 $\pm$   0.02 &     35 &  24.19 & 9.15e-01\\
 &      - & - & - &    0.0 & - &   63.4 $\pm$    1.4 &   0.69 $\pm$   0.01 &     36 &  59.46 & 8.24e-03\\
UGC 3957 &   extr &     36 &   0.12 &   11.0 $\pm$    1.0 &   11.2 &  180.8 $\pm$    7.3 &   1.01 $\pm$   0.04 &     33 &   6.63 & 1.00e+00\\
 &      - & - & - &    0.0 & - &  151.9 $\pm$    5.1 &   0.68 $\pm$   0.02 &     34 &  84.60 & 3.37e-06\\
 &   flat & - & - &   12.9 $\pm$    1.0 &   12.5 &  175.1 $\pm$    7.1 &   0.98 $\pm$   0.04 &     33 &   6.95 & 1.00e+00\\
 &      - & - & - &    0.0 & - &  144.2 $\pm$    4.7 &   0.62 $\pm$   0.02 &     34 &  91.61 & 3.48e-07\\
UGC 12491 &   extr &     23 &   0.04 &    3.0 $\pm$    0.2 &   13.8 &  148.5 $\pm$   11.7 &   1.12 $\pm$   0.04 &     20 & 445.44 & 7.29e-82\\
 &      - & - & - &    0.0 & - &   77.4 $\pm$    3.4 &   0.70 $\pm$   0.02 &     21 & 2353.02 & 0.00e+00\\
 &   flat & - & - &    3.0 $\pm$    0.2 &   13.8 &  148.5 $\pm$   11.7 &   1.12 $\pm$   0.04 &     20 & 445.44 & 7.29e-82\\
 &      - & - & - &    0.0 & - &   77.4 $\pm$    3.4 &   0.70 $\pm$   0.02 &     21 & 2353.02 & 0.00e+00\\
ZWCL 1215 &   extr &     36 &   0.25 &  163.2 $\pm$   35.6 &    4.6 &  131.3 $\pm$   43.6 &   1.00 $\pm$   0.32 &     33 &   2.94 & 1.00e+00\\
 &      - & - & - &    0.0 & - &  314.8 $\pm$   10.9 &   0.37 $\pm$   0.05 &     34 &   7.69 & 1.00e+00\\
 &   flat & - & - &  163.2 $\pm$   35.6 &    4.6 &  131.3 $\pm$   43.6 &   1.00 $\pm$   0.32 &     33 &   2.94 & 1.00e+00\\
 &      - & - & - &    0.0 & - &  314.8 $\pm$   10.9 &   0.37 $\pm$   0.05 &     34 &   7.69 & 1.00e+00\\
ZWCL 1358+6245 &   extr &     26 &   0.60 &   13.8 $\pm$    3.3 &    4.2 &  102.3 $\pm$    9.5 &   1.40 $\pm$   0.08 &     23 &   5.58 & 1.00e+00\\
 &      - & - & - &    0.0 & - &  130.6 $\pm$    6.1 &   1.15 $\pm$   0.05 &     24 &  19.02 & 7.51e-01\\
 &   flat & - & - &   20.7 $\pm$    3.2 &    6.4 &   98.0 $\pm$    9.4 &   1.43 $\pm$   0.09 &     23 &   5.65 & 1.00e+00\\
 &      - & - & - &    0.0 & - &  138.5 $\pm$    6.1 &   1.04 $\pm$   0.05 &     24 &  32.17 & 1.23e-01\\
ZWCL 1742 &   extr &     17 &   0.12 &   13.8 $\pm$    1.5 &    9.0 &  147.7 $\pm$    9.4 &   1.39 $\pm$   0.11 &     14 &  14.80 & 3.92e-01\\
 &      - & - & - &    0.0 & - &  122.0 $\pm$    6.1 &   0.78 $\pm$   0.04 &     15 &  55.08 & 1.73e-06\\
 &   flat & - & - &   23.8 $\pm$    1.7 &   14.4 &  126.5 $\pm$    9.0 &   1.30 $\pm$   0.12 &     14 &  24.08 & 4.49e-02\\
 &      - & - & - &    0.0 & - &  100.7 $\pm$    4.5 &   0.48 $\pm$   0.03 &     15 &  69.54 & 5.39e-09\\
ZWCL 1953 &   extr &     17 &   0.45 &  194.5 $\pm$   56.6 &    3.4 &   62.1 $\pm$   57.0 &   1.39 $\pm$   0.65 &     14 &   0.99 & 1.00e+00\\
 &      - & - & - &    0.0 & - &  283.3 $\pm$   27.3 &   0.45 $\pm$   0.11 &     15 &   4.39 & 9.96e-01\\
 &   flat & - & - &  194.5 $\pm$   56.6 &    3.4 &   62.1 $\pm$   57.0 &   1.39 $\pm$   0.65 &     14 &   0.99 & 1.00e+00\\
 &      - & - & - &    0.0 & - &  283.3 $\pm$   27.3 &   0.45 $\pm$   0.11 &     15 &   4.39 & 9.96e-01\\
ZWCL 3146 &   extr &     15 &   0.30 &   11.4 $\pm$    2.0 &    5.7 &  105.5 $\pm$    6.4 &   1.29 $\pm$   0.08 &     12 &   5.24 & 9.49e-01\\
 &      - & - & - &    0.0 & - &  126.3 $\pm$    4.5 &   0.98 $\pm$   0.03 &     13 &  31.82 & 2.55e-03\\
 &   flat & - & - &   11.4 $\pm$    2.0 &    5.7 &  105.5 $\pm$    6.4 &   1.29 $\pm$   0.08 &     12 &   5.24 & 9.49e-01\\
 &      - & - & - &    0.0 & - &  126.3 $\pm$    4.5 &   0.98 $\pm$   0.03 &     13 &  31.82 & 2.55e-03\\
ZWCL 7160 &   extr &     21 &   0.40 &   18.8 $\pm$    3.2 &    5.9 &   89.3 $\pm$    7.3 &   1.34 $\pm$   0.10 &     18 &   2.43 & 1.00e+00\\
 &      - & - & - &    0.0 & - &  117.0 $\pm$    4.8 &   0.93 $\pm$   0.05 &     19 &  29.31 & 6.13e-02\\
 &   flat & - & - &   21.1 $\pm$    3.1 &    6.8 &   86.3 $\pm$    7.2 &   1.37 $\pm$   0.10 &     18 &   2.82 & 1.00e+00\\
 &      - & - & - &    0.0 & - &  116.9 $\pm$    4.8 &   0.90 $\pm$   0.05 &     19 &  36.37 & 9.49e-03\\
Zwicky 2701 &   extr &     24 &   0.40 &   34.0 $\pm$    4.2 &    8.2 &  135.1 $\pm$   10.3 &   1.37 $\pm$   0.10 &     21 &   4.79 & 1.00e+00\\
 &      - & - & - &    0.0 & - &  187.1 $\pm$    6.6 &   0.87 $\pm$   0.04 &     22 &  43.01 & 4.71e-03\\
 &   flat & - & - &   39.7 $\pm$    3.9 &   10.1 &  126.0 $\pm$   10.2 &   1.45 $\pm$   0.10 &     21 &   5.67 & 1.00e+00\\
 &      - & - & - &    0.0 & - &  186.4 $\pm$    6.7 &   0.82 $\pm$   0.04 &     22 &  60.27 & 2.04e-05\\
ZwCl 0857.9+2107 &   extr &     16 &   0.30 &   23.6 $\pm$    5.0 &    4.8 &   89.6 $\pm$   10.4 &   1.40 $\pm$   0.17 &     13 &   0.92 & 1.00e+00\\
 &      - & - & - &    0.0 & - &  116.8 $\pm$    7.3 &   0.86 $\pm$   0.07 &     14 &  14.36 & 4.24e-01\\
 &   flat & - & - &   24.2 $\pm$    5.0 &    4.9 &   89.3 $\pm$   10.4 &   1.40 $\pm$   0.18 &     13 &   0.88 & 1.00e+00\\
 &      - & - & - &    0.0 & - &  116.9 $\pm$    7.4 &   0.85 $\pm$   0.07 &     14 &  14.76 & 3.95e-01\\
\enddata
\tablecomments{Col. (1) Cluster name; col. (2) CDA observation identification number; col. (3) method of $T_X$ interpolation (discussed in \S\ref{sec:kpr}); col. (4) maximum radius for fit; col. (5) number of radial bins included in fit; col. (6) best-fit core entropy; col. (7) number of sigma \kna\ is away from zero; col. (9) best-fit entropy at 100 kpc; col. (10) best-fit power-law index; col. (11) degrees of freedom in fit; col. (12) \chisq\ statistic of best-fit model; and col. (13) probability of worse fit given \chisq\ and degrees of freedom.}
\end{deluxetable}


%%%%%%%%%%%%%%%%%%%%
% End the document %
%%%%%%%%%%%%%%%%%%%%
\end{document}
