\large
\begin{center}
\noindent{\bf{How does the population of cool core clusters evolve
    with cosmic time?}}
\end{center}
\normalsize
Typically the zeroth order classification schema for the cluster
population is to divide clusters into those which have temperature
gradients that are positively increasing outward from the cluster
center, the cool core clusters (CCs), and those that do not have this
property, the non-cool core clusters (NCCs). A physical explanation
for the apparent dichotomy between CC and NCC clusters is still
unresolved, and recent theoretical studies have focused on some form
of pre-heating or specific merger histories as the solution [for
  example 1, 12]. In addition, the relationship between CC and NCC
clusters is ambiguous and so is the expected fraction of each type in
the cluster population. Since the incidence of both mergers and AGN
activity are expected to be higher at earlier cosmic times, the
population of CC clusters as a function of redshift can be used to
probe the origins of the CC-NCC dichotomy. With constraints on the
evolution of the CC population, models can then be tested for their
formation, and additional knowledge of processes like mergers and AGN
feedback can be acquired.

In [3] we showed that signatures of feedback from the region
immediately surrounding the BCG, such as \halpha\ emission indicating
star formation and radio emission indicating AGN activity, are tightly
correlated with a core entropy of $\kna \la 30 \ent$. [14] also
demonstrated that the presence of excess blue light in BCGs (an strong
indicator of star formation) is also tightly correlated with clusters
having core entropy $\la 30 \ent$. The \accept\ database also reveals
that all clusters with $\kna \la 30 \ent$ are CC clusters. The
parameter space which can be explored to define the thermodynamic
state of a cluster's core appears to be well described by simply
measuring \kna. In addition, measuring \kna\ gives insight to the
possible state of star formation in the BCG, the thermal stability of
the gas surrounding the BCG, and the recent accretion state of the
supermassive black hole within the BCG.

\kna\ is a very useful parameter, arguably moreso than simply defining
a cluster as having a CC or NCC. Hence the evolution of \kna\ with
redshift can be supplanted for the evolution of CCs with
redshift. But, \kna\ suffers from the limitation of needing
measurements of radial density and temperature. This makes measuring
\kna\ for clusters at high redshift ($z > 0.5$) no simpler than
measuring for a CC or NCC. Thus, take advantage of \kna\ and robustly
probe the population of CC clusters as a function of redshift, it is
important to determine if there exist easily measurable surrogates for
\kna, \eg\ the integrated entropy, or a ratio of integrated entropy,
within some fixed fraction of the virial radius such as
$R_{2500}$. {\bf{Thus, as a part of our archival study we propose to
    investigate if any simple X-ray observables form a tight
    correlation with \kna.}} If a suitable surrogate is found, then we
can check for changes in the fraction of CC clusters as a function of
cosmic time, and possibly draw conclusions about the role of mergers
and feedback in shaping the cluster population.
