\begin{center}
  \begin{figure}[htp]
    \begin{minipage}[htp]{0.5\linewidth}
      \includegraphics*[width=\textwidth, trim=30mm 5mm 40mm 15mm, clip]{pcav-lrad_1400.eps}
    \end{minipage}
    \begin{minipage}[htp]{0.5\linewidth}
      \includegraphics*[width=\textwidth, trim=30mm 5mm 40mm 15mm, clip]{pcav-lrad_300.eps}
    \end{minipage}
    \caption{Cavity power vs. radio power. Orange triangles represent
      the B08 galaxy clusters and groups sample. Filled circles are
      our gE sample, color-coded for the cavity system figure of
      merit: green = `A,' blue = `B,' and red = `C.' The dotted red
      lines are the B08 power-law relations. The dashed black lines
      are our best-fit power-law relations. {\it{Left:}} Cavity power
      vs. 1.4 GHz radio power. {\it{Right:}} Cavity power vs. 200-400
      MHz radio power.}
    \label{fig:pcav}
  \end{figure}
\end{center}

\begin{figure}[htp]
  \begin{center}
    \begin{minipage}[htp]{\linewidth}
      \includegraphics*[width=\textwidth, trim=30mm 5mm 40mm 15mm, clip]{radeff.eps}
      \caption{Comparison of scaling relations between jet power and
        radio luminosity. The solid red line is the W99 model with
        $k=1$. The dashed black line is our best-fit
        \pjet-\phigh\ relation. The dotted black lines denote the
        upper and lower limits of our best-fit relation including
        intrinsic scatter of $\epsilon_{\mathrm{int}} = 1.3$ dex. The
        unfilled black circles denote the poorly confined sources
        discussed in Section \ref{sec:jet}, and the down-facing black
        triangles are FR-I sources taken from the sample in C08.}
      \label{fig:radeff}
    \end{minipage}
  \end{center}
\end{figure}

\begin{figure}[htp]
  \begin{center}
    \begin{minipage}[htp]{\linewidth}
      \includegraphics*[width=\textwidth, trim=0mm 0mm 0mm 0mm, clip]{composite2.eps}
      \caption{{\it{Left:}} \chandra\ X-ray image of M84 \citep[NGC
          4374;][]{2001ApJ...547L.107F, 2008ApJ...686..911F}, which
        exemplifies the typical interaction between an AGN outflow and
        a hot gaseous halo. Contours trace out VLA 1.4 GHz radio
        emission (green is C-configuration; white is AB-configuration)
        ranging from $\approx 0.5-50$ mJy in log spaced steps of 10
        mJy. Note the displacement of the X-ray gas around the bipolar
        AGN jet outflows. {\it{Right:}} VLA B-configuration 1.4 GHz
        radio image of NGC 4261. N4261 is typical of the sub-set of
        poorly confined sources which have compact X-ray halos, small
        centralized cavities along the jets, FR-I-like radio
        morphology, and pluming beyond the ``edge'' of the X-ray
        halo. White contours trace \chandra\ observed X-ray emission
        over the surface brightness range $\approx 5-50$ ct
        arcsec$^{-2}$ in linear spaced steps of 5 ct arcsec$^{-2}$.}
      \label{fig:pics}
    \end{minipage}
  \end{center}
\end{figure}
