%%%%%%%%%%%%%%%%%%%
% Custom commands %
%%%%%%%%%%%%%%%%%%%

\newcommand{\bces}{{\textsc{BCES}}}
\newcommand{\birzan}{B\^irzan}
\newcommand{\mytitle}{A Relationship Between AGN Jet Power and Radio Luminosity}
\newcommand{\samp}{21}
\newcommand{\ecav}{\ensuremath{E_{cav}}}
\newcommand{\pjet}{\ensuremath{P_{jet}}}
\newcommand{\pcav}{\ensuremath{P_{cav}}}
\newcommand{\prad}{\ensuremath{P_{radio}}}
\newcommand{\pbolo}{\ensuremath{P_{bolo.}}}
\newcommand{\phigh}{\ensuremath{P_{\mathrm{1.4}}}}
\newcommand{\pthree}{\ensuremath{P_{\mathrm{327}}}}
\newcommand{\plow}{\ensuremath{P_{\mathrm{200-400}}}}
\newcommand{\shigh}{\ensuremath{\sigma_{\mathrm{1.4}}}}
\newcommand{\sthree}{\ensuremath{\sigma_{\mathrm{327}}}}
\newcommand{\slow}{\ensuremath{\sigma_{\mathrm{200-400}}}}
\newcommand{\sbolo}{\ensuremath{\sigma_{bolo.}}}
\newcommand{\rhigh}{\ensuremath{r_{\mathrm{1.4}}}}
\newcommand{\rlow}{\ensuremath{r_{\mathrm{200-400}}}}

%%%%%%%%%%
% Header %
%%%%%%%%%%

%\documentclass[12pt, preprint]{aastex}
%\documentclass{aastex}
\documentclass{emulateapj}
\usepackage{apjfonts,graphicx,here,common,longtable,ifthen,amsmath,amssymb,natbib}
%\usepackage[pagebackref,
% pdftitle={\mytitle},
% pdfauthor={Kenneth W. Cavagnolo},
% pdfsubject={ApJ},
% pdfkeywords={},
% pdfproducer={LaTeX with hyperref},
% pdfcreator={LaTeX}
% pdfdisplaydoctitle=true,
% colorlinks=true,
% citecolor=blue,
% linkcolor=blue,
% urlcolor=blue]{hyperref}
\bibliographystyle{apj}
\begin{document}
\title{\mytitle}
\shorttitle{AGN Scaling Relations for Giant Ellipticals}
\author{
K. W. Cavagnolo\altaffilmark{1,7},
B. R. McNamara\altaffilmark{1,2,3},
P. E. J. Nulsen\altaffilmark{3},\\
C. L. Carilli\altaffilmark{4},
C. Jones\altaffilmark{3},
W. Forman\altaffilmark{3},\\
L. \birzan\altaffilmark{5,6}, \&
S. Murray\altaffilmark{3}
}
\altaffiltext{1}{Department of Physics and Astronomy, University of
  Waterloo, Waterloo, ON N2L 3G1, Canada.}
\altaffiltext{2}{Perimeter Institute for Theoretical Physics, 31
  Caroline Street N, Waterloo, ON N2L 2Y5, Canada.}
\altaffiltext{3}{Harvard-Smithsonian Center for Astrophysics, 60
  Garden Street, Cambridge, MA 01238, USA.}
\altaffiltext{4}{National Radio Astronomy Observatory, P.O. Box 0,
  Socorro, NM 87801-0387, USA.}
\altaffiltext{5}{Department of Astronomy and Astrophysics,
  Pennsylvania State University, 525 Davey Lab, University Park, PA
  16802, USA.}
\altaffiltext{6}{Astronomical Institute Anton Pannekoek, University of
  Amsterdam, Kruislaan 403, 1098 SJ Amsterdam, Netherlands.}
\altaffiltext{7}{kcavagno@uwaterloo.ca}
\shortauthors{K. W. Cavagnolo et al.}
\journalinfo{}
\slugcomment{For submission to ApJ}

%%%%%%%%%%%%
% Abstract %
%%%%%%%%%%%%

\begin{abstract}
Using X-ray data collected with the \cxo\ and multi-frequency VLA
radio data, we investigate the scaling relationship between active
galactic nuclei (AGN) jet power (\pjet) and radio power (\prad). The
goal of our study is to determine if the \pjet-\prad\ scaling relation
found by \citet{birzan04, birzan08} for brightest cluster galaxies
(BCGs) is continuous in form and scatter from isolated giant
elliptical galaxies (gEs) up to BCGs. We expand the sample presented
in \citet{birzan04, birzan08} to much lower radio power by
incorporating measurements for \samp\ relatively isolated
gEs. Combining the results presented in this paper with the results
from \citet{birzan04, birzan08}, we find a common scaling relation of
$\pjet \propto \prad^{0.7}$ with a scatter of $\approx 1.0$ dex using
monochromatic radio powers at 200-400 MHz and 1.4 GHz. We discuss the
utility and broader consequences of a universal relation between jet
power and radio luminosity as it pertains to AGN feedback and the
growth of black holes.
\end{abstract} 

%%%%%%%%%%%%
% Keywords %
%%%%%%%%%%%%

\keywords{galaxies: active -- galaxies: clusters: general -- X-rays:
  galaxies -- radio continuum: galaxies}

%%%%%%%%%%%%%%%%%%%%%%
\section{Introduction}
\label{sec:intro}
%%%%%%%%%%%%%%%%%%%%%%

The astrophysical influence of energy deposition into an environment
hosting a supermassive black hole (SMBH), on physical scales varying
from sub-pc to Mpc, has been under intense observational and
theoretical study for decades \citep[\eg][]{1969Natur.223..690L,
1974MNRAS.166..513S, cowie77, 1981ApJ...248...55B,
1982MNRAS.199..883B, 1982MNRAS.200..115S, 1984ARA&A..22..471R,
burns90, kaiser91, 1993MNRAS.263..323T, 1995MNRAS.276..663B,
1997MNRAS.288..355B, magorrian, 1998A&A...331L...1S,
1999MNRAS.303..188K, 2000ApJ...534L.135M, voitbryan, best05,
haradent}. More specifically, the energy returned to an environment
via relativistic jets emerging from an AGN (the result of a SMBH
accreting matter) is now understood to be a very important mode of
feedback which impacts the shape and normalization of the galaxy
luminosity function \citep{croton06, bower06, saro06,
sijacki07}. Moreover, better models and observations of the AGN
feedback process have also proved important in furthering our
understanding of the formation and evolution of large-scale structure,
such as galaxy clusters and groups \citep[see, for
example,][]{2008MNRAS.386.1309M, minggroups}. But understanding the
details of how AGN feedback is coupled to heating an environment are
still hotly debated \citep[\eg][]{2008ASPC..386..343D}.

The high-resolution optics of \chandra\ have enabled recent
observational advances in studying AGN feedback. \chandra\
observations have unambiguously revealed ``cavities'' excavated in the
X-ray emitting gas which surrounds many massive galaxies
\citep[\eg][]{2000ApJ...534L.135M, perseus1, schindler01}. These
cavities result from the direct interaction of AGN jets with a hot
halo, and the cavity properties (\ie\ morphology and location) act as
a ledger for the AGN energy budget. From X-ray images of cavities,
measurements of the $pV$ work necessary to inflate a cavity are used
to calculate enthalpy, which in turn is an approximation of total jet
power. Observations have shown that cavities contain a substantial
amount of energy, $10^{54}-10^{62}$ erg, and often cavity energy is
comparable to, or exceeds, the cooling luminosity of the environment
with which the AGN has interacted \citep{birzan04, rafferty06}, though
this is not the case for all systems \citep{dunn08}. In addition to
confirming that AGN are a viable source for the heat necessary to
impact galaxy and structure formation, the use of cavities as
calorimeters opened a new pathway to study the AGN feedback process
and properties of SMBHs \citep{2009ApJ...698..594M}.

%% In addition to jets, rpisodes of AGN feedback release energy through
%% shocks, radiation, and powerful winds. In this sense then, jet powers
%% estimated from X-ray observations are lower-limits to the total AGN
%% energy output. In addition, bubble morphology, projection effects, and
%% the definition of the bubbles boundaries all affect the enthalpy
%% measurement. Thus, it is reasonable to ask if jet powers estimated
%% from X-ray observations are representative of the total AGN energy
%% output. Using mock X-ray observations generated from 3-D
%% magnetohydrodynamic simulations in conjunction with the widely
%% accepted methodology of measuring jet power,
%% \citet{2009arXiv0909.0722M} have demonstrated that the jet powers
%% estimated from the mock observations were overestimated by
%% approximately a factor of two with respect to the true jet powers
%% measured directly from the simulations. The close agreement suggests
%% that the observational techniques currently used are reliable.

Looking beyond the analysis of individual cavity systems, we may be
able to develop a more fundamental understanding of AGN feedback as it
relates to structure formation by using large samples covering
interesting ranges of host system mass, environment, and
redshift. However, selecting representative or statistically
significant samples of cavity-bearing systems is non-trivial, and
arguably not possible. This is primarily because the study of X-ray
cavities is limited by the ability and capacity of the current
generation of X-ray telescopes to obtain data of the quality necessary
to resolve cavities \citep{2009arXiv0909.0397B}. Factors such as
morphology, distance from the AGN host galaxy, and projection,
combined with observing constraints, mean that even in a sample
specifically selected to find cavities, they may go
undetected. Lacking the leverage of a large statistical study, it
behooves observers and theorists alike to find relationships between
simple observables and AGN energetics if we wish to study feedback in
broader terms.

It may be possible to circumvent the need for deep X-ray observing
programs by quantifying and calibrating how cavity properties scale
with readily available AGN observables, a strategy similar to that
used for studying dark matter halo masses \citep[\eg][]{kravtsov06,
2007ApJ...668..772M, 2009A&A...498..361P}. This can be accomplished
if, for example, there exists a universal scaling relationship between
jet power and the radio emission associated with those jets. A very
simple quantity which has been used in previous investigations is the
monochromatic radio power. A universal \pjet-\prad\ scaling relation
could then be applied to measurements from all-sky radio surveys,
thereby installing the more readily available radio observations as a
surrogate for jet power \citep{croton06, 2006MNRAS.366..397S}.

An observationally established relationship between jet power and
radio power was first determined by \citet{birzan04, birzan08} using a
sample of BCGs. Also discussed in \citet{2008ApJ...680..897D} is a
related connection between gE X-ray luminosity and radio
luminosity. The \citet{birzan04, birzan08} studies presented scaling
relations between \pjet\ and \prad\ for the 327 MHz and 1.4 GHz radio
frequencies, in addition to the bolometric radio luminosity. Several
groups have since applied the \birzan\ scaling relations to study the
effects of feedback from elliptical galaxies, most notably
\citet{best07}, \citet{2007MNRAS.379..260M}, and
\citet{2009arXiv0908.3158H}. However, up to now, no observational
evidence has suggested that the \pjet-\prad\ relations for galaxy
clusters are continuous, of comparable scatter, or applicable for
lower power radio sources like ellipticals.

To address this disparity, the work presented in this paper
investigates the relationships \pjet-\phigh\ and \pjet-\plow\ (see
Section \S\ref{sec:radio}) for a sample of relatively isolated giant
elliptical galaxies, thus extending the work of \citet{birzan08} to
AGN host system masses orders of magnitude smaller than the BCGs
studied in previous studies. Combining the \citet{birzan08} results
and the results in this paper, we find a relationship between jet
power and radio power which is similar for both the high and low radio
frequency domains. The relationships span 6-8 orders of magnitude in
\pjet\ and \prad, and have the general form $\pjet \propto
\prad^{0.7}$ with dominantly intrinsic scatter of $\sigma \sim 1.0$
dex. Our results suggest that a universal relationship between AGN jet
power and radio power exists.

The structure of this paper is as follows: \S\ref{sec:sample} outlines
the sample selected. X-ray and radio data reduction is discussed in
\S\ref{sec:data}. Results and discussion are presented in
\S\ref{sec:r&d}. A brief summary is given in
\S\ref{sec:summary}. \LCDM\ All quoted uncertainties are 68\%
confidence. Hereafter, the terminology ``cavity power'' and ``jet
power'' are used interchangeably, and are denoted as \pcav\ and \pjet,
respectively.

%%%%%%%%%%%%%%%%%%
\section{Sample}
\label{sec:sample}
%%%%%%%%%%%%%%%%%%

%% The sample we present here is clearly incomplete, and we defer
%% discussion of possible selection biases to Section \S\ref{sec:r&d}.

Our sample of \samp\ gEs are taken from the larger sample of 160 gEs
compiled by \citet{jonesge}. The \citet{jonesge} compilation is drawn
from the samples of \citet{1999MNRAS.302..209B} and
\citet{2003MNRAS.340.1375O} using the criteria that $L_K >
10^{10}~\Lsol$ and the object has been observed with \chandra. Of the
160 gEs, extended X-ray emission was detected in 109 objects. AGN
activity was suspected in 27 objects based solely on the presence of
surface brightness depressions in the X-ray emitting gas. We have
further excluded dwarf galaxies ($M_V < -19.5$) from the sample since
it is not clear that the substructure in the X-ray gas is associated
with an AGN, leaving \samp\ objects. The \samp\ gEs in our sample are
in relatively low density environments, meaning these are not the
central dominant or brightest galaxies in clusters or groups. The
X-ray emission associated with the gEs in our sample is dominated by
the gas gravitationally bound to the gE halo and not a diffuse
atmosphere of some larger environment. The comoving distance range of
our sample is 0.4-117 Mpc, with an X-ray luminosity range of $\sim
10^{31}-10^{36} \lum$. General properties of the sample are listed in
Table \ref{tab:sample}.

\begin{deluxetable}{lcccccccc}
\tablewidth{0pt}
\tabletypesize{\scriptsize}
\tablecaption{Summary of Sample\label{tab:sample}}
\tablehead{\colhead{Cluster} & \colhead{Obs.ID} & \colhead{R.A.} & \colhead{Dec.} & \colhead{ExpT} & \colhead{Mode} & \colhead{ACIS} & \colhead{$z$} & \colhead{$L_{bol.}$}\\
\colhead{ } & \colhead{ } & \colhead{hr:min:sec} & \colhead{$\degr:\arcmin:\arcsec$} & \colhead{ksec} & \colhead{ } & \colhead{ } & \colhead{ } & \colhead{$10^{44}$ ergs s$^{-1}$}\\
\colhead{{(1)}} & \colhead{{(2)}} & \colhead{{(3)}} & \colhead{{(4)}} & \colhead{{(5)}} & \colhead{{(6)}} & \colhead{{(7)}} & \colhead{{(8)}} & \colhead{{(9)}}
}
\startdata
1E0657 56 & \dataset [ADS/Sa.CXO\#obs/03184] {3184} & 06:58:29.627 & -55:56:39.79 & 87.5 & VF & I3 & 0.296 & 52.48\\
1E0657 56 & \dataset [ADS/Sa.CXO\#obs/05356] {5356} & 06:58:29.619 & -55:56:39.35 & 97.2 & VF & I2 & 0.296 & 52.48\\
1E0657 56 & \dataset [ADS/Sa.CXO\#obs/05361] {5361} & 06:58:29.670 & -55:56:39.80 & 82.6 & VF & I3 & 0.296 & 52.48\\
1RXS J2129.4-0741 & \dataset [ADS/Sa.CXO\#obs/03199] {3199} & 21:29:26.274 & -07:41:29.18 & 19.9 & VF & I3 & 0.570 & 20.58\\
1RXS J2129.4-0741 & \dataset [ADS/Sa.CXO\#obs/03595] {3595} & 21:29:26.281 & -07:41:29.36 & 19.9 & VF & I3 & 0.570 & 20.58\\
2PIGG J0011.5-2850 & \dataset [ADS/Sa.CXO\#obs/05797] {5797} & 00:11:21.623 & -28:51:14.44 & 19.9 & VF & I3 & 0.075 &  2.15\\
2PIGG J0311.8-2655 $\dagger$ & \dataset [ADS/Sa.CXO\#obs/05799] {5799} & 03:11:33.904 & -26:54:16.48 & 39.6 & VF & I3 & 0.062 &  0.25\\
2PIGG J2227.0-3041 & \dataset [ADS/Sa.CXO\#obs/05798] {5798} & 22:27:54.560 & -30:34:34.84 & 22.3 & VF & I2 & 0.073 &  0.81\\
3C 220.1 & \dataset [ADS/Sa.CXO\#obs/00839] {839} & 09:32:40.218 & +79:06:29.46 & 18.9 &  F & S3 & 0.610 &  3.25\\
3C 28.0 & \dataset [ADS/Sa.CXO\#obs/03233] {3233} & 00:55:50.401 & +26:24:36.47 & 49.7 & VF & I3 & 0.195 &  4.78\\
3C 295 & \dataset [ADS/Sa.CXO\#obs/02254] {2254} & 14:11:20.280 & +52:12:10.55 & 90.9 & VF & I3 & 0.464 &  6.92\\
3C 388 & \dataset [ADS/Sa.CXO\#obs/05295] {5295} & 18:44:02.365 & +45:33:29.31 & 30.7 & VF & I3 & 0.092 &  0.52\\
4C 55.16 & \dataset [ADS/Sa.CXO\#obs/04940] {4940} & 08:34:54.923 & +55:34:21.15 & 96.0 & VF & S3 & 0.242 &  5.90\\
ABELL 0013 $\dagger$ & \dataset [ADS/Sa.CXO\#obs/04945] {4945} & 00:13:37.883 & -19:30:09.10 & 55.3 & VF & S3 & 0.094 &  1.41\\
ABELL 0068 & \dataset [ADS/Sa.CXO\#obs/03250] {3250} & 00:37:06.309 & +09:09:32.28 & 10.0 & VF & I3 & 0.255 & 12.70\\
ABELL 0119 $\dagger$ & \dataset [ADS/Sa.CXO\#obs/04180] {4180} & 00:56:15.150 & -01:14:59.70 & 11.9 & VF & I3 & 0.044 &  1.39\\
ABELL 0168 & \dataset [ADS/Sa.CXO\#obs/03203] {3203} & 01:14:57.909 & +00:24:42.55 & 40.6 & VF & I3 & 0.045 &  0.23\\
ABELL 0168 & \dataset [ADS/Sa.CXO\#obs/03204] {3204} & 01:14:57.925 & +00:24:42.73 & 37.6 & VF & I3 & 0.045 &  0.23\\
ABELL 0209 & \dataset [ADS/Sa.CXO\#obs/03579] {3579} & 01:31:52.565 & -13:36:39.29 & 10.0 & VF & I3 & 0.206 & 10.96\\
ABELL 0209 & \dataset [ADS/Sa.CXO\#obs/00522] {522} & 01:31:52.595 & -13:36:39.25 & 10.0 & VF & I3 & 0.206 & 10.96\\
ABELL 0267 & \dataset [ADS/Sa.CXO\#obs/01448] {1448} & 01:52:29.181 & +00:57:34.43 & 7.9 &  F & I3 & 0.230 &  8.62\\
ABELL 0267 & \dataset [ADS/Sa.CXO\#obs/03580] {3580} & 01:52:29.180 & +00:57:34.23 & 19.9 & VF & I3 & 0.230 &  8.62\\
ABELL 0370 & \dataset [ADS/Sa.CXO\#obs/00515] {515} & 02:39:53.169 & -01:34:36.96 & 88.0 &  F & S3 & 0.375 & 11.95\\
ABELL 0383 & \dataset [ADS/Sa.CXO\#obs/02321] {2321} & 02:48:03.364 & -03:31:44.69 & 19.5 &  F & S3 & 0.187 &  5.32\\
ABELL 0399 & \dataset [ADS/Sa.CXO\#obs/03230] {3230} & 02:57:54.931 & +13:01:58.41 & 48.6 & VF & I0 & 0.072 &  4.37\\
ABELL 0401 & \dataset [ADS/Sa.CXO\#obs/00518] {518} & 02:58:56.896 & +13:34:14.48 & 18.0 &  F & I3 & 0.074 &  8.39\\
ABELL 0478 & \dataset [ADS/Sa.CXO\#obs/06102] {6102} & 04:13:25.347 & +10:27:55.62 & 10.0 & VF & I3 & 0.088 & 16.39\\
ABELL 0514 & \dataset [ADS/Sa.CXO\#obs/03578] {3578} & 04:48:19.229 & -20:30:28.79 & 44.5 & VF & I3 & 0.072 &  0.66\\
ABELL 0520 & \dataset [ADS/Sa.CXO\#obs/04215] {4215} & 04:54:09.711 & +02:55:23.69 & 66.3 & VF & I3 & 0.202 & 12.97\\
ABELL 0521 & \dataset [ADS/Sa.CXO\#obs/00430] {430} & 04:54:07.004 & -10:13:26.72 & 39.1 & VF & S3 & 0.253 &  9.77\\
ABELL 0586 & \dataset [ADS/Sa.CXO\#obs/00530] {530} & 07:32:20.339 & +31:37:58.59 & 10.0 & VF & I3 & 0.171 &  8.54\\
ABELL 0611 & \dataset [ADS/Sa.CXO\#obs/03194] {3194} & 08:00:56.832 & +36:03:24.09 & 36.1 & VF & S3 & 0.288 & 10.78\\
ABELL 0644 $\dagger$ & \dataset [ADS/Sa.CXO\#obs/02211] {2211} & 08:17:25.225 & -07:30:40.03 & 29.7 & VF & I3 & 0.070 &  6.95\\
ABELL 0665 & \dataset [ADS/Sa.CXO\#obs/03586] {3586} & 08:30:59.231 & +65:50:37.78 & 29.7 & VF & I3 & 0.181 & 13.37\\
ABELL 0697 & \dataset [ADS/Sa.CXO\#obs/04217] {4217} & 08:42:57.549 & +36:21:57.65 & 19.5 & VF & I3 & 0.282 & 26.10\\
ABELL 0773 & \dataset [ADS/Sa.CXO\#obs/05006] {5006} & 09:17:52.566 & +51:43:38.18 & 19.8 & VF & I3 & 0.217 & 12.87\\
\bf{ABELL 0781} & \dataset [ADS/Sa.CXO\#obs/00534] {534} & 09:20:25.431 & +30:30:07.56 & 9.9 & VF & I3 & 0.298 &  0.00\\
ABELL 0907 & \dataset [ADS/Sa.CXO\#obs/03185] {3185} & 09:58:21.880 & -11:03:52.20 & 48.0 & VF & I3 & 0.153 &  6.19\\
ABELL 0963 & \dataset [ADS/Sa.CXO\#obs/00903] {903} & 10:17:03.744 & +39:02:49.17 & 36.3 &  F & S3 & 0.206 & 10.65\\
ABELL 1063S & \dataset [ADS/Sa.CXO\#obs/04966] {4966} & 22:48:44.294 & -44:31:48.37 & 26.7 & VF & I3 & 0.354 & 71.09\\
ABELL 1068 $\dagger$ & \dataset [ADS/Sa.CXO\#obs/01652] {1652} & 10:40:44.520 & +39:57:10.28 & 26.8 &  F & S3 & 0.138 &  4.19\\
ABELL 1201 $\dagger$ & \dataset [ADS/Sa.CXO\#obs/04216] {4216} & 11:12:54.489 & +13:26:08.76 & 39.7 & VF & S3 & 0.169 &  3.52\\
ABELL 1204 & \dataset [ADS/Sa.CXO\#obs/02205] {2205} & 11:13:20.419 & +17:35:38.45 & 23.6 & VF & I3 & 0.171 &  3.92\\
ABELL 1361 $\dagger$ & \dataset [ADS/Sa.CXO\#obs/02200] {2200} & 11:43:39.827 & +46:21:21.40 & 16.7 &  F & S3 & 0.117 &  2.16\\
ABELL 1423 & \dataset [ADS/Sa.CXO\#obs/00538] {538} & 11:57:17.026 & +33:36:37.44 & 9.8 & VF & I3 & 0.213 &  7.01\\
ABELL 1651 & \dataset [ADS/Sa.CXO\#obs/04185] {4185} & 12:59:22.830 & -04:11:45.86 & 9.6 & VF & I3 & 0.084 &  6.66\\
ABELL 1664 $\dagger$ & \dataset [ADS/Sa.CXO\#obs/01648] {1648} & 13:03:42.478 & -24:14:44.55 & 9.8 & VF & S3 & 0.128 &  2.59\\
\bf{ABELL 1682} & \dataset [ADS/Sa.CXO\#obs/03244] {3244} & 13:06:50.764 & +46:33:19.86 & 9.8 & VF & I3 & 0.226 &  0.00\\
ABELL 1689 & \dataset [ADS/Sa.CXO\#obs/01663] {1663} & 13:11:29.612 & -01:20:28.69 & 10.7 &  F & I3 & 0.184 & 24.71\\
ABELL 1689 & \dataset [ADS/Sa.CXO\#obs/05004] {5004} & 13:11:29.606 & -01:20:28.61 & 19.9 & VF & I3 & 0.184 & 24.71\\
ABELL 1689 & \dataset [ADS/Sa.CXO\#obs/00540] {540} & 13:11:29.595 & -01:20:28.47 & 10.3 &  F & I3 & 0.184 & 24.71\\
ABELL 1758 & \dataset [ADS/Sa.CXO\#obs/02213] {2213} & 13:32:42.978 & +50:32:44.83 & 58.3 & VF & S3 & 0.279 & 21.01\\
ABELL 1763 & \dataset [ADS/Sa.CXO\#obs/03591] {3591} & 13:35:17.957 & +40:59:55.80 & 19.6 & VF & I3 & 0.187 &  9.26\\
ABELL 1795 $\dagger$ & \dataset [ADS/Sa.CXO\#obs/05289] {5289} & 13:48:52.829 & +26:35:24.01 & 15.0 & VF & I3 & 0.062 &  7.59\\
ABELL 1835 & \dataset [ADS/Sa.CXO\#obs/00495] {495} & 14:01:01.951 & +02:52:43.18 & 19.5 &  F & S3 & 0.253 & 39.38\\
ABELL 1914 & \dataset [ADS/Sa.CXO\#obs/03593] {3593} & 14:26:01.399 & +37:49:27.83 & 18.9 & VF & I3 & 0.171 & 26.25\\
ABELL 1942 & \dataset [ADS/Sa.CXO\#obs/03290] {3290} & 14:38:21.878 & +03:40:12.97 & 57.6 & VF & I2 & 0.224 &  2.27\\
ABELL 1995 & \dataset [ADS/Sa.CXO\#obs/00906] {906} & 14:52:57.758 & +58:02:51.34 & 0.0 &  F & S3 & 0.319 & 10.19\\
ABELL 2029 $\dagger$ & \dataset [ADS/Sa.CXO\#obs/06101] {6101} & 15:10:56.163 & +05:44:40.89 & 9.9 & VF & I3 & 0.076 & 13.90\\
ABELL 2034 & \dataset [ADS/Sa.CXO\#obs/02204] {2204} & 15:10:11.003 & +33:30:46.46 & 53.9 & VF & I3 & 0.113 &  6.45\\
ABELL 2065 $\dagger$ & \dataset [ADS/Sa.CXO\#obs/031821] {31821} & 15:22:29.220 & +27:42:46.54 & 0.0 & VF & I3 & 0.073 &  2.92\\
ABELL 2069 & \dataset [ADS/Sa.CXO\#obs/04965] {4965} & 15:24:09.181 & +29:53:18.05 & 55.4 & VF & I2 & 0.116 &  3.82\\
ABELL 2111 & \dataset [ADS/Sa.CXO\#obs/00544] {544} & 15:39:41.432 & +34:25:12.26 & 10.3 &  F & I3 & 0.230 &  7.45\\
ABELL 2125 & \dataset [ADS/Sa.CXO\#obs/02207] {2207} & 15:41:14.154 & +66:15:57.20 & 81.5 & VF & I3 & 0.246 &  0.77\\
ABELL 2163 & \dataset [ADS/Sa.CXO\#obs/01653] {1653} & 16:15:45.705 & -06:09:00.62 & 71.1 & VF & I1 & 0.170 & 49.11\\
ABELL 2204 $\dagger$ & \dataset [ADS/Sa.CXO\#obs/0499] {499} & 16:32:45.437 & +05:34:21.05 & 10.1 &  F & S3 & 0.152 & 20.77\\
ABELL 2204 & \dataset [ADS/Sa.CXO\#obs/06104] {6104} & 16:32:45.428 & +05:34:20.89 & 9.6 & VF & I3 & 0.152 & 22.03\\
ABELL 2218 & \dataset [ADS/Sa.CXO\#obs/01666] {1666} & 16:35:50.831 & +66:12:42.31 & 48.6 & VF & I0 & 0.171 &  8.39\\
ABELL 2219 $\dagger$ & \dataset [ADS/Sa.CXO\#obs/0896] {896} & 16:40:21.069 & +46:42:29.07 & 42.3 &  F & S3 & 0.226 & 33.15\\
ABELL 2255 & \dataset [ADS/Sa.CXO\#obs/00894] {894} & 17:12:40.385 & +64:03:50.63 & 39.4 &  F & I3 & 0.081 &  3.67\\
ABELL 2256 $\dagger$ & \dataset [ADS/Sa.CXO\#obs/01386] {1386} & 17:03:44.567 & +78:38:11.51 & 12.4 &  F & I3 & 0.058 &  4.65\\
ABELL 2259 & \dataset [ADS/Sa.CXO\#obs/03245] {3245} & 17:20:08.299 & +27:40:11.53 & 10.0 & VF & I3 & 0.164 &  5.37\\
ABELL 2261 & \dataset [ADS/Sa.CXO\#obs/05007] {5007} & 17:22:27.254 & +32:07:58.60 & 24.3 & VF & I3 & 0.224 & 17.49\\
ABELL 2294 & \dataset [ADS/Sa.CXO\#obs/03246] {3246} & 17:24:10.149 & +85:53:09.77 & 10.0 & VF & I3 & 0.178 & 10.35\\
ABELL 2384 & \dataset [ADS/Sa.CXO\#obs/04202] {4202} & 21:52:21.178 & -19:32:51.90 & 31.5 & VF & I3 & 0.095 &  1.95\\
ABELL 2390 $\dagger$ & \dataset [ADS/Sa.CXO\#obs/04193] {4193} & 21:53:36.825 & +17:41:44.38 & 95.1 & VF & S3 & 0.230 & 31.02\\
ABELL 2409 & \dataset [ADS/Sa.CXO\#obs/03247] {3247} & 22:00:52.567 & +20:58:34.11 & 10.2 & VF & I3 & 0.148 &  7.01\\
ABELL 2537 & \dataset [ADS/Sa.CXO\#obs/04962] {4962} & 23:08:22.313 & -02:11:29.88 & 36.2 & VF & S3 & 0.295 & 10.16\\
\bf{ABELL 2550} & \dataset [ADS/Sa.CXO\#obs/02225] {2225} & 23:11:35.806 & -21:44:46.70 & 59.0 & VF & S3 & 0.154 &  0.58\\
ABELL 2554 $\dagger$ & \dataset [ADS/Sa.CXO\#obs/01696] {1696} & 23:12:19.939 & -21:30:09.84 & 19.9 & VF & S3 & 0.110 &  1.57\\
ABELL 2556 $\dagger$ & \dataset [ADS/Sa.CXO\#obs/02226] {2226} & 23:13:01.413 & -21:38:04.47 & 19.9 & VF & S3 & 0.086 &  1.43\\
ABELL 2631 & \dataset [ADS/Sa.CXO\#obs/03248] {3248} & 23:37:38.560 & +00:16:28.64 & 9.2 & VF & I3 & 0.278 & 12.59\\
ABELL 2667 & \dataset [ADS/Sa.CXO\#obs/02214] {2214} & 23:51:39.395 & -26:05:02.75 & 9.6 & VF & S3 & 0.230 & 19.91\\
ABELL 2670 & \dataset [ADS/Sa.CXO\#obs/04959] {4959} & 23:54:13.687 & -10:25:08.85 & 39.6 & VF & I3 & 0.076 &  1.39\\
ABELL 2717 & \dataset [ADS/Sa.CXO\#obs/06974] {6974} & 00:03:11.996 & -35:56:08.01 & 19.8 & VF & I3 & 0.048 &  0.26\\
ABELL 2744 & \dataset [ADS/Sa.CXO\#obs/02212] {2212} & 00:14:14.396 & -30:22:40.04 & 24.8 & VF & S3 & 0.308 & 29.00\\
ABELL 3128 $\dagger$ & \dataset [ADS/Sa.CXO\#obs/00893] {893} & 03:29:50.918 & -52:34:51.04 & 19.6 &  F & I3 & 0.062 &  0.35\\
ABELL 3158 $\dagger$ & \dataset [ADS/Sa.CXO\#obs/03201] {3201} & 03:42:54.675 & -53:37:24.36 & 24.8 & VF & I3 & 0.059 &  3.01\\
ABELL 3158 $\dagger$ & \dataset [ADS/Sa.CXO\#obs/03712] {3712} & 03:42:54.683 & -53:37:24.37 & 30.9 & VF & I3 & 0.059 &  3.01\\
ABELL 3164 & \dataset [ADS/Sa.CXO\#obs/06955] {6955} & 03:46:16.839 & -57:02:11.38 & 13.5 & VF & I3 & 0.057 &  0.19\\
ABELL 3376 & \dataset [ADS/Sa.CXO\#obs/03202] {3202} & 06:02:05.122 & -39:57:42.82 & 44.3 & VF & I3 & 0.046 &  0.75\\
ABELL 3376 & \dataset [ADS/Sa.CXO\#obs/03450] {3450} & 06:02:05.162 & -39:57:42.87 & 19.8 & VF & I3 & 0.046 &  0.75\\
ABELL 3391 $\dagger$ & \dataset [ADS/Sa.CXO\#obs/04943] {4943} & 06:26:21.511 & -53:41:44.81 & 18.4 & VF & I3 & 0.056 &  1.44\\
ABELL 3921 & \dataset [ADS/Sa.CXO\#obs/04973] {4973} & 22:49:57.829 & -64:25:42.17 & 29.4 & VF & I3 & 0.093 &  3.37\\
AC 114 & \dataset [ADS/Sa.CXO\#obs/01562] {1562} & 22:58:48.196 & -34:47:56.89 & 72.5 &  F & S3 & 0.312 & 10.90\\
CL 0024+17 & \dataset [ADS/Sa.CXO\#obs/00929] {929} & 00:26:35.996 & +17:09:45.37 & 39.8 &  F & S3 & 0.394 &  2.88\\
CL 1221+4918 & \dataset [ADS/Sa.CXO\#obs/01662] {1662} & 12:21:26.709 & +49:18:21.60 & 79.1 & VF & I3 & 0.700 &  8.65\\
CL J0030+2618 & \dataset [ADS/Sa.CXO\#obs/05762] {5762} & 00:30:34.339 & +26:18:01.58 & 17.9 & VF & I3 & 0.500 &  3.41\\
CL J0152-1357 & \dataset [ADS/Sa.CXO\#obs/00913] {913} & 01:52:42.141 & -13:57:59.71 & 36.5 &  F & I3 & 0.831 & 13.30\\
CL J0542.8-4100 & \dataset [ADS/Sa.CXO\#obs/00914] {914} & 05:42:49.994 & -40:59:58.50 & 50.4 &  F & I3 & 0.630 &  6.18\\
CL J0848+4456 & \dataset [ADS/Sa.CXO\#obs/01708] {1708} & 08:48:48.235 & +44:56:17.11 & 61.4 & VF & I1 & 0.574 &  0.62\\
CL J0848+4456 & \dataset [ADS/Sa.CXO\#obs/00927] {927} & 08:48:48.252 & +44:56:17.13 & 125.1 & VF & I1 & 0.574 &  0.62\\
CL J1113.1-2615 & \dataset [ADS/Sa.CXO\#obs/00915] {915} & 11:13:05.167 & -26:15:40.43 & 104.6 &  F & I3 & 0.730 &  2.22\\
\bf{CL J1213+0253} & \dataset [ADS/Sa.CXO\#obs/04934] {4934} & 12:13:34.948 & +02:53:45.45 & 18.9 & VF & I3 & 0.409 &  0.00\\
CL J1226.9+3332 & \dataset [ADS/Sa.CXO\#obs/03180] {3180} & 12:26:58.373 & +33:32:47.36 & 31.7 & VF & I3 & 0.890 & 30.76\\
CL J1226.9+3332 & \dataset [ADS/Sa.CXO\#obs/05014] {5014} & 12:26:58.372 & +33:32:47.18 & 32.7 & VF & I3 & 0.890 & 30.76\\
\bf{CL J1641+4001} & \dataset [ADS/Sa.CXO\#obs/03575] {3575} & 16:41:53.704 & +40:01:44.40 & 46.5 & VF & I3 & 0.464 &  0.00\\
CL J2302.8+0844 & \dataset [ADS/Sa.CXO\#obs/00918] {918} & 23:02:48.156 & +08:43:52.74 & 108.6 &  F & I3 & 0.730 &  2.93\\
DLS J0514-4904 & \dataset [ADS/Sa.CXO\#obs/04980] {4980} & 05:14:40.037 & -49:03:15.07 & 19.9 & VF & I3 & 0.091 &  0.68\\
EXO 0422-086 $\dagger$ & \dataset [ADS/Sa.CXO\#obs/04183] {4183} & 04:25:51.271 & -08:33:36.42 & 10.0 & VF & I3 & 0.040 &  0.65\\
HERCULES A $\dagger$ & \dataset [ADS/Sa.CXO\#obs/01625] {1625} & 16:51:08.161 & +04:59:32.44 & 14.8 & VF & S3 & 0.154 &  3.27\\
\bf{IRAS 09104+4109} & \dataset [ADS/Sa.CXO\#obs/00509] {509} & 09:13:45.481 & +40:56:27.49 & 9.1 &  F & S3 & 0.442 &  0.00\\
\bf{LYNX E} & \dataset [ADS/Sa.CXO\#obs/017081] {17081} & 08:48:58.851 & +44:51:51.44 & 61.4 & VF & I2 & 1.260 &  0.00\\
\bf{LYNX E} & \dataset [ADS/Sa.CXO\#obs/09271] {9271} & 08:48:58.858 & +44:51:51.46 & 125.1 & VF & I2 & 1.260 &  0.00\\
MACS J0011.7-1523 & \dataset [ADS/Sa.CXO\#obs/03261] {3261} & 00:11:42.965 & -15:23:20.79 & 21.6 & VF & I3 & 0.360 & 10.75\\
MACS J0011.7-1523 & \dataset [ADS/Sa.CXO\#obs/06105] {6105} & 00:11:42.957 & -15:23:20.76 & 37.3 & VF & I3 & 0.360 & 10.75\\
MACS J0025.4-1222 & \dataset [ADS/Sa.CXO\#obs/03251] {3251} & 00:25:29.368 & -12:22:38.05 & 19.3 & VF & I3 & 0.584 & 13.00\\
MACS J0025.4-1222 & \dataset [ADS/Sa.CXO\#obs/05010] {5010} & 00:25:29.399 & -12:22:38.10 & 24.8 & VF & I3 & 0.584 & 13.00\\
MACS J0035.4-2015 & \dataset [ADS/Sa.CXO\#obs/03262] {3262} & 00:35:26.573 & -20:15:46.06 & 21.4 & VF & I3 & 0.364 & 19.79\\
MACS J0111.5+0855 & \dataset [ADS/Sa.CXO\#obs/03256] {3256} & 01:11:31.515 & +08:55:39.21 & 19.4 & VF & I3 & 0.263 &  0.64\\
MACS J0152.5-2852 & \dataset [ADS/Sa.CXO\#obs/03264] {3264} & 01:52:34.479 & -28:53:38.01 & 17.5 & VF & I3 & 0.341 &  6.33\\
MACS J0159.0-3412 & \dataset [ADS/Sa.CXO\#obs/05818] {5818} & 01:59:00.366 & -34:13:00.23 & 9.4 & VF & I3 & 0.458 & 18.92\\
MACS J0159.8-0849 & \dataset [ADS/Sa.CXO\#obs/03265] {3265} & 01:59:49.453 & -08:50:00.90 & 17.9 & VF & I3 & 0.405 & 26.31\\
MACS J0159.8-0849 & \dataset [ADS/Sa.CXO\#obs/06106] {6106} & 01:59:49.422 & -08:50:00.42 & 35.3 & VF & I3 & 0.405 & 26.31\\
MACS J0242.5-2132 & \dataset [ADS/Sa.CXO\#obs/03266] {3266} & 02:42:35.906 & -21:32:26.30 & 11.9 & VF & I3 & 0.314 & 12.74\\
MACS J0257.1-2325 & \dataset [ADS/Sa.CXO\#obs/01654] {1654} & 02:57:09.130 & -23:26:06.25 & 19.8 &  F & I3 & 0.505 & 21.72\\
MACS J0257.1-2325 & \dataset [ADS/Sa.CXO\#obs/03581] {3581} & 02:57:09.152 & -23:26:06.21 & 18.5 & VF & I3 & 0.505 & 21.72\\
MACS J0257.6-2209 & \dataset [ADS/Sa.CXO\#obs/03267] {3267} & 02:57:41.024 & -22:09:11.12 & 20.5 & VF & I3 & 0.322 & 10.77\\
MACS J0308.9+2645 & \dataset [ADS/Sa.CXO\#obs/03268] {3268} & 03:08:55.927 & +26:45:38.34 & 24.4 & VF & I3 & 0.324 & 20.42\\
MACS J0329.6-0211 & \dataset [ADS/Sa.CXO\#obs/03257] {3257} & 03:29:41.681 & -02:11:47.67 & 9.9 & VF & I3 & 0.450 & 12.82\\
MACS J0329.6-0211 & \dataset [ADS/Sa.CXO\#obs/03582] {3582} & 03:29:41.688 & -02:11:47.81 & 19.9 & VF & I3 & 0.450 & 12.82\\
MACS J0329.6-0211 & \dataset [ADS/Sa.CXO\#obs/06108] {6108} & 03:29:41.681 & -02:11:47.57 & 39.6 & VF & I3 & 0.450 & 12.82\\
MACS J0404.6+1109 & \dataset [ADS/Sa.CXO\#obs/03269] {3269} & 04:04:32.491 & +11:08:02.10 & 21.8 & VF & I3 & 0.355 &  3.90\\
MACS J0417.5-1154 & \dataset [ADS/Sa.CXO\#obs/03270] {3270} & 04:17:34.686 & -11:54:32.71 & 12.0 & VF & I3 & 0.440 & 37.99\\
MACS J0429.6-0253 & \dataset [ADS/Sa.CXO\#obs/03271] {3271} & 04:29:36.088 & -02:53:09.02 & 23.2 & VF & I3 & 0.399 & 11.58\\
MACS J0451.9+0006 & \dataset [ADS/Sa.CXO\#obs/05815] {5815} & 04:51:54.291 & +00:06:20.20 & 10.2 & VF & I3 & 0.430 &  8.20\\
MACS J0455.2+0657 & \dataset [ADS/Sa.CXO\#obs/05812] {5812} & 04:55:17.426 & +06:57:47.15 & 9.9 & VF & I3 & 0.425 &  9.77\\
MACS J0520.7-1328 & \dataset [ADS/Sa.CXO\#obs/03272] {3272} & 05:20:42.052 & -13:28:49.38 & 19.2 & VF & I3 & 0.340 &  9.63\\
MACS J0547.0-3904 & \dataset [ADS/Sa.CXO\#obs/03273] {3273} & 05:47:01.582 & -39:04:28.24 & 21.7 & VF & I3 & 0.210 &  1.59\\
MACS J0553.4-3342 & \dataset [ADS/Sa.CXO\#obs/05813] {5813} & 05:53:27.200 & -33:42:53.02 & 9.9 & VF & I3 & 0.407 & 32.68\\
MACS J0717.5+3745 & \dataset [ADS/Sa.CXO\#obs/01655] {1655} & 07:17:31.654 & +37:45:18.52 & 19.9 &  F & I3 & 0.548 & 46.58\\
MACS J0717.5+3745 & \dataset [ADS/Sa.CXO\#obs/04200] {4200} & 07:17:31.651 & +37:45:18.46 & 59.2 & VF & I3 & 0.548 & 46.58\\
MACS J0744.8+3927 & \dataset [ADS/Sa.CXO\#obs/03197] {3197} & 07:44:52.802 & +39:27:24.41 & 20.2 & VF & I3 & 0.686 & 24.67\\
MACS J0744.8+3927 & \dataset [ADS/Sa.CXO\#obs/03585] {3585} & 07:44:52.779 & +39:27:24.41 & 19.9 & VF & I3 & 0.686 & 24.67\\
MACS J0744.8+3927 & \dataset [ADS/Sa.CXO\#obs/06111] {6111} & 07:44:52.800 & +39:27:24.41 & 49.5 & VF & I3 & 0.686 & 24.67\\
MACS J0911.2+1746 & \dataset [ADS/Sa.CXO\#obs/03587] {3587} & 09:11:11.325 & +17:46:31.06 & 17.9 & VF & I3 & 0.541 & 10.52\\
MACS J0911.2+1746 & \dataset [ADS/Sa.CXO\#obs/05012] {5012} & 09:11:11.309 & +17:46:30.92 & 23.8 & VF & I3 & 0.541 & 10.52\\
MACS J0949+1708   & \dataset [ADS/Sa.CXO\#obs/03274] {3274} & 09:49:51.824 & +17:07:05.62 & 14.3 & VF & I3 & 0.382 & 19.19\\
MACS J1006.9+3200 & \dataset [ADS/Sa.CXO\#obs/05819] {5819} & 10:06:54.668 & +32:01:34.61 & 10.9 & VF & I3 & 0.359 &  6.06\\
MACS J1105.7-1014 & \dataset [ADS/Sa.CXO\#obs/05817] {5817} & 11:05:46.462 & -10:14:37.20 & 10.3 & VF & I3 & 0.466 & 11.29\\
MACS J1108.8+0906 & \dataset [ADS/Sa.CXO\#obs/03252] {3252} & 11:08:55.393 & +09:05:51.16 & 9.9 & VF & I3 & 0.449 &  8.96\\
MACS J1108.8+0906 & \dataset [ADS/Sa.CXO\#obs/05009] {5009} & 11:08:55.402 & +09:05:51.14 & 24.5 & VF & I3 & 0.449 &  8.96\\
MACS J1115.2+5320 & \dataset [ADS/Sa.CXO\#obs/03253] {3253} & 11:15:15.632 & +53:20:03.71 & 8.8 & VF & I3 & 0.439 & 14.29\\
MACS J1115.2+5320 & \dataset [ADS/Sa.CXO\#obs/05008] {5008} & 11:15:15.646 & +53:20:03.74 & 18.0 & VF & I3 & 0.439 & 14.29\\
MACS J1115.2+5320 & \dataset [ADS/Sa.CXO\#obs/05350] {5350} & 11:15:15.632 & +53:20:03.37 & 6.9 & VF & I3 & 0.439 & 14.29\\
MACS J1115.8+0129 & \dataset [ADS/Sa.CXO\#obs/03275] {3275} & 11:15:52.048 & +01:29:56.56 & 15.9 & VF & I3 & 0.120 &  1.47\\
MACS J1131.8-1955 & \dataset [ADS/Sa.CXO\#obs/03276] {3276} & 11:31:56.011 & -19:55:55.85 & 13.9 & VF & I3 & 0.307 & 17.45\\
MACS J1149.5+2223 & \dataset [ADS/Sa.CXO\#obs/01656] {1656} & 11:49:35.856 & +22:23:55.02 & 18.5 & VF & I3 & 0.544 & 21.60\\
MACS J1149.5+2223 & \dataset [ADS/Sa.CXO\#obs/03589] {3589} & 11:49:35.848 & +22:23:55.05 & 20.0 & VF & I3 & 0.544 & 21.60\\
MACS J1206.2-0847 & \dataset [ADS/Sa.CXO\#obs/03277] {3277} & 12:06:12.276 & -08:48:02.40 & 23.5 & VF & I3 & 0.440 & 37.02\\
MACS J1226.8+2153 & \dataset [ADS/Sa.CXO\#obs/03590] {3590} & 12:26:51.207 & +21:49:55.22 & 19.0 & VF & I3 & 0.370 &  2.63\\
MACS J1311.0-0310 & \dataset [ADS/Sa.CXO\#obs/03258] {3258} & 13:11:01.665 & -03:10:39.50 & 14.9 & VF & I3 & 0.494 & 10.03\\
MACS J1311.0-0310 & \dataset [ADS/Sa.CXO\#obs/06110] {6110} & 13:11:01.680 & -03:10:39.75 & 63.2 & VF & I3 & 0.494 & 10.03\\
MACS J1319+7003   & \dataset [ADS/Sa.CXO\#obs/03278] {3278} & 13:20:08.370 & +70:04:33.81 & 21.6 & VF & I3 & 0.328 &  7.03\\
MACS J1427.2+4407 & \dataset [ADS/Sa.CXO\#obs/06112] {6112} & 14:27:16.175 & +44:07:30.33 & 9.4 & VF & I3 & 0.477 & 14.18\\
MACS J1427.6-2521 & \dataset [ADS/Sa.CXO\#obs/03279] {3279} & 14:27:39.389 & -25:21:04.66 & 16.9 & VF & I3 & 0.220 &  1.55\\
MACS J1621.3+3810 & \dataset [ADS/Sa.CXO\#obs/03254] {3254} & 16:21:25.552 & +38:09:43.56 & 9.8 & VF & I3 & 0.461 & 11.49\\
MACS J1621.3+3810 & \dataset [ADS/Sa.CXO\#obs/03594] {3594} & 16:21:25.558 & +38:09:43.44 & 19.7 & VF & I3 & 0.461 & 11.49\\
MACS J1621.3+3810 & \dataset [ADS/Sa.CXO\#obs/06109] {6109} & 16:21:25.535 & +38:09:43.34 & 37.5 & VF & I3 & 0.461 & 11.49\\
MACS J1621.3+3810 & \dataset [ADS/Sa.CXO\#obs/06172] {6172} & 16:21:25.559 & +38:09:43.63 & 29.8 & VF & I3 & 0.461 & 11.49\\
MACS J1731.6+2252 & \dataset [ADS/Sa.CXO\#obs/03281] {3281} & 17:31:39.902 & +22:52:00.55 & 20.5 & VF & I3 & 0.366 &  9.32\\
\bf{MACS J1824.3+4309} & \dataset [ADS/Sa.CXO\#obs/03255] {3255} & 18:24:18.444 & +43:09:43.39 & 14.9 & VF & I3 & 0.487 &  0.00\\
MACS J1931.8-2634 & \dataset [ADS/Sa.CXO\#obs/03282] {3282} & 19:31:49.656 & -26:34:33.99 & 13.6 & VF & I3 & 0.352 & 23.14\\
MACS J2046.0-3430 & \dataset [ADS/Sa.CXO\#obs/05816] {5816} & 20:46:00.522 & -34:30:15.50 & 10.0 & VF & I3 & 0.413 &  5.79\\
MACS J2049.9-3217 & \dataset [ADS/Sa.CXO\#obs/03283] {3283} & 20:49:56.245 & -32:16:52.30 & 23.8 & VF & I3 & 0.325 &  8.71\\
MACS J2211.7-0349 & \dataset [ADS/Sa.CXO\#obs/03284] {3284} & 22:11:45.856 & -03:49:37.24 & 17.7 & VF & I3 & 0.270 & 22.11\\
MACS J2214.9-1359 & \dataset [ADS/Sa.CXO\#obs/03259] {3259} & 22:14:57.467 & -14:00:09.35 & 19.5 & VF & I3 & 0.503 & 24.05\\
MACS J2214.9-1359 & \dataset [ADS/Sa.CXO\#obs/05011] {5011} & 22:14:57.481 & -14:00:09.39 & 18.5 & VF & I3 & 0.503 & 24.05\\
MACS J2228+2036   & \dataset [ADS/Sa.CXO\#obs/03285] {3285} & 22:28:33.241 & +20:37:11.42 & 19.9 & VF & I3 & 0.412 & 17.92\\
MACS J2229.7-2755 & \dataset [ADS/Sa.CXO\#obs/03286] {3286} & 22:29:45.358 & -27:55:38.41 & 16.4 & VF & I3 & 0.324 &  9.49\\
MACS J2243.3-0935 & \dataset [ADS/Sa.CXO\#obs/03260] {3260} & 22:43:21.537 & -09:35:44.30 & 20.5 & VF & I3 & 0.101 &  0.78\\
MACS J2245.0+2637 & \dataset [ADS/Sa.CXO\#obs/03287] {3287} & 22:45:04.547 & +26:38:07.88 & 16.9 & VF & I3 & 0.304 &  9.36\\
MACS J2311+0338   & \dataset [ADS/Sa.CXO\#obs/03288] {3288} & 23:11:33.213 & +03:38:06.51 & 13.6 & VF & I3 & 0.300 & 10.98\\
MKW3S & \dataset [ADS/Sa.CXO\#obs/0900] {900} & 15:21:51.930 & +07:42:31.97 & 57.3 & VF & I3 & 0.045 &  1.14\\
MS 0016.9+1609 & \dataset [ADS/Sa.CXO\#obs/00520] {520} & 00:18:33.503 & +16:26:12.99 & 67.4 & VF & I3 & 0.541 & 32.94\\
\bf{MS 0302.7+1658} & \dataset [ADS/Sa.CXO\#obs/00525] {525} & 03:05:31.614 & +17:10:02.06 & 10.0 & VF & I3 & 0.424 &  0.00\\
MS 0440.5+0204 $\dagger$ & \dataset [ADS/Sa.CXO\#obs/04196] {4196} & 04:43:09.952 & +02:10:18.70 & 59.4 & VF & S3 & 0.190 &  2.17\\
MS 0451.6-0305 & \dataset [ADS/Sa.CXO\#obs/00902] {902} & 04:54:11.004 & -03:00:52.19 & 44.2 &  F & S3 & 0.539 & 33.32\\
MS 0735.6+7421 & \dataset [ADS/Sa.CXO\#obs/04197] {4197} & 07:41:44.245 & +74:14:38.23 & 45.5 & VF & S3 & 0.216 &  7.57\\
MS 0839.8+2938 & \dataset [ADS/Sa.CXO\#obs/02224] {2224} & 08:42:55.969 & +29:27:26.97 & 29.8 &  F & S3 & 0.194 &  3.10\\
MS 0906.5+1110 & \dataset [ADS/Sa.CXO\#obs/00924] {924} & 09:09:12.753 & +10:58:32.00 & 29.7 & VF & I3 & 0.163 &  4.64\\
MS 1006.0+1202 & \dataset [ADS/Sa.CXO\#obs/00925] {925} & 10:08:47.194 & +11:47:55.99 & 29.4 & VF & I3 & 0.221 &  4.75\\
MS 1008.1-1224 & \dataset [ADS/Sa.CXO\#obs/00926] {926} & 10:10:32.312 & -12:39:56.80 & 44.2 & VF & I3 & 0.301 &  6.44\\
MS 1054.5-0321 & \dataset [ADS/Sa.CXO\#obs/00512] {512} & 10:56:58.499 & -03:37:32.76 & 89.1 &  F & S3 & 0.830 & 27.22\\
MS 1455.0+2232 & \dataset [ADS/Sa.CXO\#obs/04192] {4192} & 14:57:15.088 & +22:20:32.49 & 91.9 & VF & I3 & 0.259 & 10.25\\
MS 1621.5+2640 & \dataset [ADS/Sa.CXO\#obs/00546] {546} & 16:23:35.522 & +26:34:25.67 & 30.1 &  F & I3 & 0.426 &  6.49\\
MS 2053.7-0449 & \dataset [ADS/Sa.CXO\#obs/01667] {1667} & 20:56:21.295 & -04:37:46.81 & 44.5 & VF & I3 & 0.583 &  2.96\\
MS 2053.7-0449 & \dataset [ADS/Sa.CXO\#obs/00551] {551} & 20:56:21.297 & -04:37:46.80 & 44.3 &  F & I3 & 0.583 &  2.96\\
MS 2137.3-2353 & \dataset [ADS/Sa.CXO\#obs/04974] {4974} & 21:40:15.178 & -23:39:40.71 & 57.4 & VF & S3 & 0.313 & 11.28\\
MS J1157.3+5531 $\dagger$ & \dataset [ADS/Sa.CXO\#obs/04964] {4964} & 11:59:52.295 & +55:32:05.61 & 75.1 & VF & S3 & 0.081 &  0.12\\
NGC 6338 $\dagger$ & \dataset [ADS/Sa.CXO\#obs/04194] {4194} & 17:15:23.036 & +57:24:40.29 & 47.3 & VF & I3 & 0.028 &  0.13\\
PKS 0745-191 & \dataset [ADS/Sa.CXO\#obs/06103] {6103} & 07:47:31.469 & -19:17:40.01 & 10.3 & VF & I3 & 0.103 & 18.41\\
RBS 0797 & \dataset [ADS/Sa.CXO\#obs/02202] {2202} & 09:47:12.971 & +76:23:13.90 & 11.7 & VF & I3 & 0.354 & 26.07\\
RDCS 1252-29    & \dataset [ADS/Sa.CXO\#obs/04198] {4198} & 12:52:54.221 & -29:27:21.01 & 163.4 & VF & I3 & 1.237 &  2.28\\
RX J0232.2-4420 & \dataset [ADS/Sa.CXO\#obs/04993] {4993} & 02:32:18.771 & -44:20:46.68 & 23.4 & VF & I3 & 0.284 & 18.17\\
RX J0340-4542   & \dataset [ADS/Sa.CXO\#obs/06954] {6954} & 03:40:44.765 & -45:41:18.41 & 17.9 & VF & I3 & 0.082 &  0.33\\
RX J0439+0520   & \dataset [ADS/Sa.CXO\#obs/00527] {527} & 04:39:02.218 & +05:20:43.11 & 9.6 & VF & I3 & 0.208 &  3.57\\
RX J0439.0+0715 & \dataset [ADS/Sa.CXO\#obs/01449] {1449} & 04:39:00.710 & +07:16:07.65 & 6.3 &  F & I3 & 0.230 &  9.44\\
RX J0439.0+0715 & \dataset [ADS/Sa.CXO\#obs/03583] {3583} & 04:39:00.710 & +07:16:07.63 & 19.2 & VF & I3 & 0.230 &  9.44\\
RX J0528.9-3927 & \dataset [ADS/Sa.CXO\#obs/04994] {4994} & 05:28:53.039 & -39:28:15.53 & 22.5 & VF & I3 & 0.263 & 12.99\\
RX J0647.7+7015 & \dataset [ADS/Sa.CXO\#obs/03196] {3196} & 06:47:50.029 & +70:14:49.66 & 19.3 & VF & I3 & 0.584 & 26.48\\
RX J0647.7+7015 & \dataset [ADS/Sa.CXO\#obs/03584] {3584} & 06:47:50.014 & +70:14:49.69 & 20.0 & VF & I3 & 0.584 & 26.48\\
RX J0819.6+6336 $\dagger$ & \dataset [ADS/Sa.CXO\#obs/02199] {2199} & 08:19:26.007 & +63:37:26.53 & 14.9 &  F & S3 & 0.119 &  0.98\\
RX J0910+5422   & \dataset [ADS/Sa.CXO\#obs/02452] {2452} & 09:10:44.478 & +54:22:03.77 & 65.3 & VF & I3 & 1.100 &  1.33\\
\bf{RX J1053+5735}   & \dataset [ADS/Sa.CXO\#obs/04936] {4936} & 10:53:39.844 & +57:35:18.42 & 92.2 &  F & S3 & 1.140 &  0.00\\
RX J1347.5-1145 & \dataset [ADS/Sa.CXO\#obs/03592] {3592} & 13:47:30.593 & -11:45:10.05 & 57.7 & VF & I3 & 0.451 & 100.36\\
RX J1347.5-1145 & \dataset [ADS/Sa.CXO\#obs/00507] {507} & 13:47:30.598 & -11:45:10.27 & 10.0 &  F & S3 & 0.451 & 100.36\\
RX J1350+6007   & \dataset [ADS/Sa.CXO\#obs/02229] {2229} & 13:50:48.038 & +60:07:08.39 & 58.3 & VF & I3 & 0.804 &  2.19\\
RX J1423.8+2404 & \dataset [ADS/Sa.CXO\#obs/01657] {1657} & 14:23:47.759 & +24:04:40.45 & 18.5 & VF & I3 & 0.545 & 15.84\\
RX J1423.8+2404 & \dataset [ADS/Sa.CXO\#obs/04195] {4195} & 14:23:47.763 & +24:04:40.63 & 115.6 & VF & S3 & 0.545 & 15.84\\
RX J1504.1-0248 & \dataset [ADS/Sa.CXO\#obs/05793] {5793} & 15:04:07.415 & -02:48:15.70 & 39.2 & VF & I3 & 0.215 & 34.64\\
RX J1525+0958   & \dataset [ADS/Sa.CXO\#obs/01664] {1664} & 15:24:39.729 & +09:57:44.42 & 50.9 & VF & I3 & 0.516 &  3.29\\
RX J1532.9+3021 & \dataset [ADS/Sa.CXO\#obs/01649] {1649} & 15:32:55.642 & +30:18:57.69 & 9.4 & VF & S3 & 0.345 & 20.77\\
RX J1532.9+3021 & \dataset [ADS/Sa.CXO\#obs/01665] {1665} & 15:32:55.641 & +30:18:57.31 & 10.0 & VF & I3 & 0.345 & 20.77\\
RX J1716.9+6708 & \dataset [ADS/Sa.CXO\#obs/00548] {548} & 17:16:49.015 & +67:08:25.80 & 51.7 &  F & I3 & 0.810 &  8.04\\
RX J1720.1+2638 & \dataset [ADS/Sa.CXO\#obs/04361] {4361} & 17:20:09.941 & +26:37:29.11 & 25.7 & VF & I3 & 0.164 & 11.39\\
RX J1720.2+3536 & \dataset [ADS/Sa.CXO\#obs/03280] {3280} & 17:20:16.953 & +35:36:23.63 & 20.8 & VF & I3 & 0.391 & 13.02\\
RX J1720.2+3536 & \dataset [ADS/Sa.CXO\#obs/06107] {6107} & 17:20:16.949 & +35:36:23.68 & 33.9 & VF & I3 & 0.391 & 13.02\\
RX J1720.2+3536 & \dataset [ADS/Sa.CXO\#obs/07225] {7225} & 17:20:16.947 & +35:36:23.69 & 2.0 & VF & I3 & 0.391 & 13.02\\
RX J2011.3-5725 & \dataset [ADS/Sa.CXO\#obs/04995] {4995} & 20:11:26.889 & -57:25:09.08 & 24.0 & VF & I3 & 0.279 &  2.77\\
RX J2129.6+0005 & \dataset [ADS/Sa.CXO\#obs/00552] {552} & 21:29:39.944 & +00:05:18.83 & 10.0 & VF & I3 & 0.235 & 12.56\\
S0463 & \dataset [ADS/Sa.CXO\#obs/06956] {6956} & 04:29:07.040 & -53:49:38.02 & 29.3 & VF & I3 & 0.099 & 22.19\\
S0463 & \dataset [ADS/Sa.CXO\#obs/07250] {7250} & 04:29:07.063 & -53:49:38.11 & 29.1 & VF & I3 & 0.099 & 22.19\\
TRIANG AUSTR $\dagger$ & \dataset [ADS/Sa.CXO\#obs/01281] {1281} & 16:38:22.712 & -64:21:19.70 & 11.4 &  F & I3 & 0.051 &  9.41\\
V 1121.0+2327 & \dataset [ADS/Sa.CXO\#obs/01660] {1660} & 11:20:57.195 & +23:26:27.60 & 71.3 & VF & I3 & 0.560 &  3.28\\
ZWCL 1215 & \dataset [ADS/Sa.CXO\#obs/04184] {4184} & 12:17:40.787 & +03:39:39.42 & 12.1 & VF & I3 & 0.075 &  3.49\\
ZWCL 1358+6245 & \dataset [ADS/Sa.CXO\#obs/00516] {516} & 13:59:50.526 & +62:31:04.57 & 54.1 &  F & S3 & 0.328 & 12.42\\
ZWCL 1953 & \dataset [ADS/Sa.CXO\#obs/01659] {1659} & 08:50:06.677 & +36:04:16.16 & 24.9 &  F & I3 & 0.380 & 17.11\\
ZWCL 3146 & \dataset [ADS/Sa.CXO\#obs/00909] {909} & 10:23:39.735 & +04:11:08.05 & 46.0 &  F & I3 & 0.290 & 29.59\\
ZWCL 5247 & \dataset [ADS/Sa.CXO\#obs/00539] {539} & 12:34:21.928 & +09:47:02.83 & 9.3 & VF & I3 & 0.229 &  4.87\\
ZWCL 7160 & \dataset [ADS/Sa.CXO\#obs/00543] {543} & 14:57:15.158 & +22:20:33.85 & 9.9 &  F & I3 & 0.258 & 10.14\\
ZWICKY 2701 & \dataset [ADS/Sa.CXO\#obs/03195] {3195} & 09:52:49.183 & +51:53:05.27 & 26.9 & VF & S3 & 0.210 &  5.19\\
ZwCL 1332.8+5043 & \dataset [ADS/Sa.CXO\#obs/05772] {5772} & 13:34:20.698 & +50:31:04.64 & 19.5 & VF & I3 & 0.620 &  4.46\\
ZwCl 0848.5+3341 & \dataset [ADS/Sa.CXO\#obs/04205] {4205} & 08:51:38.873 & +33:31:08.00 & 11.4 & VF & S3 & 0.371 &  4.58
\enddata
\tablecomments{(1) Cluster name, (2) CDA observation identification number, (3) R.A. of cluster center, (4) Dec. of cluster center, (5) nominal exposure time, (6) observing mode, (7) CCD location of centroid, (8) redshift, (9) bolometric luminosity. A ($\dagger$) indicates a cluster analyzed within R$_{5000}$ only. Clusters listed with zero bolometric luminosity are clusters excluded from our analysis and discussed in \S\ref{sec:fitting}.}
\end{deluxetable}


%%%%%%%%%%%%%%%%%%%%%%%%%%%%%%%%%%%%%%%%
\section{Observations and Data Analysis}
\label{sec:data}
%%%%%%%%%%%%%%%%%%%%%%%%%%%%%%%%%%%%%%%%

%\citet{birzan08} used 1.4 GHz and 327 MHz monochromatic radio powers
%in their comparison with X-ray cavity powers to establish the scaling
%relations discussed in Section \S\ref{sec:intro}. To ensure a reliable
%comparison between the results presented here and those of
%\citet{birzan08}, we also focus on monochromatic radio fluxes at 1.4
%GHz. At lower frequencies, we use the frequency range 200-400 MHz
%since 327 MHz observations were not available for most systems. Our
%methodologies for calculating the cavity and radio powers for each
%system are presented below.

%%%%%%%%%%%%%%%%%%
\subsection{X-ray}
\label{sec:xray}
%%%%%%%%%%%%%%%%%%

%% All data used in measuring the X-ray cavity powers was taken with the
%% \cxo. All datasets were analyzed using the Chandra Interactive
%% Analysis of Observations (\ciao) software version XXX with the
%% calibrations of \caldb\ XXX. The level 1 events files were reprocessed
%% to apply the most current gain and correct for charge transfer
%% inefficiency. Additionally, the events lists were filtered for bad
%% grades, and cosmic rays were further rejected by using VFAINT
%% filtering when possible. Time periods contaminated by background
%% flaring were excluded from the data. Point sources were detected and
%% removed from the observations using the output of the routine
%% {\textsc{wavdetect}}. For the X-ray spectral analysis a low-energy
%% cut-off of 0.7 keV was used in addition to a high-energy cut-off of
%% 7.0 keV. Background spectral analysis was performed using the
%% \caldb\ blank-sky observations included in the CALDB and tailored to
%% match each targeted observation.

Details of the X-ray data analysis will be presented in
\citet{nulsen09}. \chandra\ data was reduced following standard
procedures: gain correction, CTI correction, bad grade filtering,
flare filtering, and point source exclusion. Background spectral
analysis was performed using the \caldb\ blank-sky observations
included in the CALDB and tailored to match each targeted observation.
Measurement of cavity power is performed identically to
\citet{rafferty06}. For each observation, an azimuthally averaged
deprojected gas density and temperature profile were used to calculate
the total gas pressure at each radius. Then, using cavity volumes
calculated from a by-eye measurement of a cavity's projected
cross-section, the total energy contained in each cavity is estimated
as $\ecav = pV[\gamma/(\gamma-1)]$ where $p$ is pressure, $V$ is
volume, and we assume 4/3 for the ratio of specific heat capacities,
$\gamma$. The total cavity energy is then divided by an age estimate
for the AGN outburst, yielding $\pcav = \ecav/t$ (see \citet{mcnamrev}
for discussion of how timescales are estimated). Uncertainties for
each calculation were determined by propogating errors and summing in
quadrature.

%%%%%%%%%%%%%%%%%%
\subsection{Radio}
\label{sec:radio}
%%%%%%%%%%%%%%%%%%

%% NVSS is complete down to $\approx 2.5$ mJy, while SUMSS has a
%% declination dependent completeness limit of $\approx 10$ mJy for
%% $\delta > -50\degr$ and $\approx 6$ mJy for $\delta \leq
%% -50\degr$. The NVSS right ascension and declination positional
%% uncertainties for sources with flux $>$ 15 mJy are $\la 1''$ and
%% increases to $\approx 7''$ at the survey flux-limit \citep{nvss}.

The 1.4 GHz continuum radio flux for each source was taken from the
flux-limited NRAO VLA Sky Survey (NVSS, \citealt{nvss}). For NGC 1553,
which is outside the NVSS survey area, the 843 MHz continuum radio
flux was taken from the flux-limited Sydney University Molonglo Sky
Survey (SUMSS, \citealt{sumss1, sumss2}). We calculate the
monochromatic radio power for each source using the relation $\radpow
= 4 \pi D_L^2 (1+z)^{-1} S_{\nu} f_0$ where $S_{\nu}$ is the NVSS or
SUMSS flux, $D_L$ is the luminosity distance, and $f_0$ is the central
beam frequency of the observations.

The morphologies of the radio sources associated with our sample of
gEs is heterogeneous, some sources are very large and extended, while
some sources are very compact. To ensure the entire radio source was
measured, a fixed angular distance of $1200\arcsec$ was searched
around the X-ray centroid of each target. The probability of finding a
radio source within such a large search radius is 100\% for both NVSS
and SUMSS. Thus, for each target field, all detected radio sources
were overlaid on a composite image made of the X-ray emission as seen
with \chandra, optical emission (using DSS
I/II\footnote{http://archive.stsci.edu/dss/}), infrared emission
(using 2MASS\footnote{http://www.ipac.caltech.edu/2mass/}), and when
available, the deeper and higher resolution radio data from VLA
FIRST\footnote{http://sundog.stsci.edu}. A visual inspection was then
performed to establish which detected radio sources were associated
with the target gE. After confirming which radio sources within the
search region were associated with the target gE, the fluxes of the
individual sources were added and the associated uncertainties summed
in quadrature.

As an extension of the NVSS search, we also reduced and analyzed
archival VLA data for each source in the sample. As will be discussed
in Section \S\ref{sec:r&d}, the purpose of this additional step is to
establish which sources are jet dominated and which are not. In the
cases where high-resolution VLA archival data is available, we used
the reduced images as an additional diagnostic for confirming the
connection between what we considered to be the relevant detected NVSS
emission and the host gE. The fluxes measured from the VLA archival
data also served as a check of the fluxes taken from NVSS. We found
good agreement for most sources, the exceptions being IC 4296 and NGC
4782 where the NVSS flux is approximately a factor of 2 lower. These
sources are unique because the radio lobes contain non-negligible
power in diffuse, extended emission which is not detected in NVSS. For
these sources, the fluxes measured from the archival VLA data are
used.

\citet{birzan08} found that the \pcav-\pthree\ relation had lower
scatter than the \pcav-\phigh\ relation. For our sample of gEs, the
quality and availability of 327 MHz data were poor, and thus we
resorted to gathering low-frequency radio fluxes from the CATS
Database\footnote{http://www.sao.ru/cats/} \citep{cats}. The CATS
Database is a compilation of over 350 radio catalogs and was queried
in the frequency range 200-400 MHz using the list of radio source
coordinates found from the NVSS and SUMSS searches. Of the \samp\ gEs
in our sample, 17 objects were found in the CATS database with fluxes
in the 200-400 MHz range. The search approach used means that
low-frequency radio emission which does not have a 1.4 GHz counterpart
(or vice versa) is missed, but, since CATS does not provide source
images for visual inspection, this method also ensures that
non-relevant detections are less likely to be included. Therefore, the
200-400 MHz radio powers shown in Figure \ref{fig:pcav} likely
underestimate the 200-400 MHz fluxes for these gEs. However, because
of the large range of radio powers, a systematic shift by a factor of
a few along the \radpow\ axis for all the gE points alters the
best-fit relations within the uncertainties.

%% Both the high- and low-frequency radio data analysis presented in this
%% paper differs from the work of \citet{birzan08} in that they used
%% archival VLA data as the source for measuring fluxes for all their
%% sources. Thus, neither the level of uncertainty nor the flux limit in
%% \citet{birzan08} are uniform as a result of variation in observing
%% setups, while our use of NVSS and other catalogs results in more
%% uniform measurements. Nonetheless, given the large range of radio
%% powers, it is reasonable to assume that the differences in the
%% systematic uncertainties are negligible compared to the intrinsic and
%% statistical scatter.

%%%%%%%%%%%%%%%%%%%%%%%%%%%%%%%%
\section{Results and Discussion}
\label{sec:r&d}
%%%%%%%%%%%%%%%%%%%%%%%%%%%%%%%%

%%%%%%%%%%%%%%%%%%%%%%%%%%%%%%%%%%%%%%%%%%
\subsection{\pjet-\prad\ Scaling Relation}
\label{sec:relation}
%%%%%%%%%%%%%%%%%%%%%%%%%%%%%%%%%%%%%%%%%%

The results from the X-ray and radio data analysis are shown in the
two plots of \pcav-\phigh\ and \pcav-\plow\ presented in Figure
\ref{fig:pcav}. A subjective grade of quality was assigned for each
set of cavities -- shown as color coding in Figure
\ref{fig:pcav} and listed in Table \ref{tab:sample}. We assigned
grades because, currently, no standard algorithm for detecting
cavities and determining cavity morphology exists, and the grades
quantify our by-eye assessment. Grade A cavities are decidedly
associated with AGN radio activity and have well-defined boundaries;
grade B cavities are also decidedly associated with AGN radio
activity, but lack well-defined boundaries; grade C cavities have
poorly-defined boundaries, and their connection to AGN radio activity
is uncertain. Grade C cavities are excluded from all fitting, as are
jet dominated sources (see Section \ref{sec:jet}).

Figure \ref{fig:pcav} reveals that the power law relationship between
cavity power and monochromatic radio power is continuous over
approximately 8 orders of magnitude in radio power and 6 orders of
magnitude in cavity power. To determine the form of the power-law
relation, we performed log-space linear fits to the data for each
frequency regime using the bivariate correlated error and intrinsic
scatter (\bces) algorithm \citep{bces}. The orthogonal \bces\
algorithm takes in asymmetric uncertainties for both variables,
assumes the presence of intrinsic scatter, and performs a linear
least-squares regression. The best-fit parameter uncertainties are
calculated using 5000 Monte Carlo bootstrap resampling trials.

%(discussed further in Section \S\ref{sec:sys}).
The scatter in the data is large relative to the size of the
statistical uncertainties for the individual points, and is clearly
dominated by intrinsic scatter. So while we assumed \bces\ was the
best fitting method, extensive testing with other fitting methods
(\ie\ Craig Markwardt's {\textsc{mpfit}}, IDL's {\textsc{linfit}} and
{\textsc{fitexy}}) was performed using synthetic idealized data. The
\bces\ algorithm proved to be the best fitting method for our data
because the addition of an intrinsic scatter term into the fitting
produces uncertainties which are least likely to underestimate the
90\% confidence intervals of an underlying relationship.

The best-fit linear function in log-space output by the \bces\ fitting
method for the \pcav-\phigh\ and \pcav-\plow\ relations are:
\begin{eqnarray}
\log~\pcav &=& 0.70~(\pm 0.10)~\log~\phigh + 1.69~(\pm 0.18)\\
\log~\pcav &=& 0.67~(\pm 0.08)~\log~\plow + 1.30~(\pm 0.14).
\end{eqnarray}
The scatter for each relation is \shigh\ = 1.07 dex and \slow\ = 0.88
dex, and the respective correlation coefficients are \rhigh\ = 0.74
and \rlow\ = 0.81. We have quantified log-space scatter using the
average size of the residuals about the best-fit relation. For
comparison, the scaling relations found by \citet{birzan08} are
\begin{eqnarray}
\pjet &\propto& \pthree^{0.62 \pm 0.08}\\
\pjet &\propto& \phigh^{0.35 \pm 0.07}\\
\pjet &\propto& \pbolo^{0.48 \pm 0.07}
\end{eqnarray}
with scatters of $\sthree = 0.65$ dex, $\shigh = 0.69$ dex, and
$\sbolo = 0.85$ dex. Note that high and low frequency relations
presented in \citet{birzan08} have power-law indices which
significantly differ from each other by a factor of two: $m = 0.62 \pm
0.08$ for \pcav-\pthree\ and $m = 0.35 \pm 0.07$ for
\pcav-\phigh. Inclusion of the new gE dataset shows that the
discrepancy arose from undersampling, \ie\ the 1.4 GHz relationship
simply had too few data points covering too small a range in
\pcav\ and \prad\ to reveal a steeper relationship. The similarity of
the power-laws for both frequency regimes is encouraging, it strongly
suggests an underlying universal relationship between jet power and
radio luminosity.

As discussed in \citet{birzan04, birzan08}, the intrinsic scatter in
the relation likely originates from differences between the properties
of the individual systems. \citet{birzan08} also demonstrated that
correcting for the effect of radio aging reduces the scatter of the
relations. Additionally, as demonstrated with systems like Hydra A
\citep{hydraa} and MS 0735.6+7421 \citep{ms0735}, the energetics in a
system at any given time can be dominated by AGN outbursts which are
more powerful than average. This may be particularly true for gEs,
which have lower pressure halos and are more susceptible to disruption
by AGN outbursts \citep{minggroups}. The large scatter also highlights
that radio luminosity is a poor surrogate for determining the total
energy released via an AGN.

%%%%%%%%%%%%%%%%%%%%%%%%%%%%%%%%%%
\subsection{Jet Dominated Sources}
\label{sec:jet}
%%%%%%%%%%%%%%%%%%%%%%%%%%%%%%%%%%


In Figure \ref{fig:pcav} we have enclosed a subset of points in open
circles to highlight the systems which we have deemed as being jet
dominated. Using archival VLA radio data (see Section
\S\ref{sec:radio}), if 50\% or more of the radio flux from a source
originates in the jets than the lobes, then the source is deemed jet
dominated. The subjective nature of the classification arises in
setting the boundary between what is the lobe and what is the jet. The
objects we have identified as jetted are: IC 4296
\citep{2003ApJ...585..677P}, NGC 315 \citep{1979ApJ...228L...9B,
  1981A&A....95..250W}, NGC 4261 \citep{1997ApJ...484..186J,
  2000ApJ...534..165J}, NGC 4782 \citep{2007ApJ...664..804M}, and NGC
7626 \citep{1985ApJ...291...32B}. Interestingly, as can be seen in
Figure \ref{fig:pcav}, jetted sources cluster in a particular region
of the \pjet-\prad\ plane. The jetted sources also have radio
morphologies which are distinctly different from the rest of the gEs,
but are similar to other jetted sources. In addition, for these jetted
sources, the morphologies of the X-ray halo surface brightness
decrements are reminiscent of ``tunnels'' rather than bubbles.

Typically, cavity systems are coincident with the general region of
the radio outflow's termination point, \ie\ the location where a
bubble is being inflated {\it{in situ}}, see Figure \ref{fig:pics} for
an example in M84. However, for these jetted sources, the X-ray
tunnels are found at the base of the radio outflows, approximately
along the jets, see Figure \ref{fig:pics} for an example in NGC
4261. It may be that jetted sources also have classical cavities
associated with the radio lobes, but for these systems the X-ray gas
at those radii is too faint to yield the contrast needed to detect the
cavities. To investigate this possibility, and the extent to which
\pcav\ may be underestimated in these systems, we calculated \pcav\
values for an idealized X-ray atmosphere and an adiabatically evolving
bubble.

To investigate the maximum possible increase in \pcav, we consider a
gas cloud with a slightly peaked $\beta$-model density profile ($\beta
= 2/3$, $r_{core} = 1$ kpc, $\rho_0 = 0.1 \pcc$), and a temperature
profile which starts at 0.8 keV and asymptotically approaches 1.0 keV
within 100 kpc. The bubble starts at a distance of 10 kpc with a
radius of 5 kpc and evolves adiabatically such that $r_{cav} = (D
r_{cav,0}/D_0) (D/D_0)^{-1} [p(D)/p(D_0)]^{-1/3\gamma}$
\citep{mcnamrev} where $D$ is distance, $r_{cav}$ is the cavity
radius, $p(D)$ is the pressure at some distance, $\gamma$ is the ratio
of specific heats, and naught subscripts indicate initial
values. Under these conditions, bubble enthalpy as a function of
radius, and relative to the initial value, changes by less than an
order of magnitude. Only under the extreme conditions that the bubble
is formed, or evolves, rapidly, \eg\ non-adiabatic expansion or
supersonic inflation, can the \pcav\ values be revised upward by more
than an order of magnitude. Thus, for the jetted systems it is
unlikely that larger bubbles at greater radii would move the points up
to a location where they are consistent with the \pcav-\prad\
trend. For this reason, jetted systems, which are included in the
figures for completeness, are not included in any of the fitting
because it appears they are not representative of the \pjet-\prad\
relation under study.

%% %%%%%%%%%%%%%%%%%%%%%%%%
%% \subsection{Systematics}
%% \label{sec:sys}
%% %%%%%%%%%%%%%%%%%%%%%%%%

%% The \pjet-\prad\ trend has considerable scatter, and the samples
%% presented in this paper and in \citet{birzan04} were not statistically
%% selected or intended to be representative of some parent
%% population. It is therefore useful to consider a few sources of
%% possible systematics or biases. Below we discuss how underestimating
%% \pcav\ affects our results and if one should worry about a possible
%% selection bias in the presented sample. We find that the influence of
%% systematics on our results are negligible given the large range of
%% \pcav\ and \prad\ covered.

%% On short timescales, the energetics of a system can be dominated by an
%% AGN outburst, where one mode of energy deposition is via
%% cavities. Many other channels of energy deposition may be important,
%% for example shocks \citep{ms0735, hydraa, herca, 2005ApJ...635..894F,
%%   2006MNRAS.371L..65S} or sound waves (\eg\ Perseus
%% \citep{perseus3}). In the limiting circumstance where the energy in
%% these other channels is of the order the cavity power -- or possibly
%% greater than the cavity power \citep{2006MNRAS.373..739N} -- then the
%% jet power is a poor representation of the total energy output. The
%% extent to which these additional energy channels alter the slope and
%% normalization of the \pcav-\prad\ relation is not obvious, but, it is
%% reasonable to assume the corrections to the total jet power would only
%% be significant in the very rare cases like MS0735 where the shock
%% energy is a few times the cavity energy. In which case the
%% \pcav-\prad\ relation would not significantly change. A strong scaling
%% of shock energy or sound wave energy with host system mass would also
%% be another way of altering the shape and normalization of the
%% \pcav-\prad\ relation. As of now, there is no observational evidence,
%% nor literature, to suggest such a strong effect exists.

%% There is also the issue of drawing the conclusion that a relation
%% between jet power and radio power is universal based upon samples
%% which are not statistically selected or representative. The systems
%% which were used in \citet{birzan04, birzan08}, and this paper were
%% obviously chosen because it is possible to measure \pcav, \eg\ the
%% samples are explicitly biased towards only systems with cavities in
%% their X-ray atmospheres. This raises the question: does the scaling
%% relation of $\pjet \propto \prad^{0.7}$ hold for {\it{all}} systems?
%% More succinctly, is there something special about how AGN interact
%% with an environment which makes a scaling relation between mechanical
%% power and radio power a function of the host galaxy's surroundings?
%% And...???

%%%%%%%%%%%%%%%%%%%%%%%%%%%%%%%%%
\section{Summary and Conclusions}
\label{sec:summary}
%%%%%%%%%%%%%%%%%%%%%%%%%%%%%%%%%

We have presented a sample of \samp\ giant elliptical galaxies
observed with the \cxo\ and which all have suspected AGN induced
cavity systems in the X-ray halo surrounding the host galaxy. Cavity
powers, \pcav, were calculated for each set of cavities using the
methods outline in \citet{rafferty06}. Monochromatic radio powers at
1.4 GHz and between 200-400 MHz were also calculated using
measurements from the NVSS and SUMMS surveys, and the CATS database.
Each set of cavities was graded subjectively with a letter: `A', `B',
or `C,' with `A' being the most reliable cavities and `C' being the
least reliable. For the purposes of isolating those systems which may
not be representative of the true \pcav-\prad\ relation durnig
fitting, we also subjectively determined which of the gE systems were
``jet dominated'' by using jet and lobe fluxes taken from archival,
high-resolution VLA observations. If $\ge 50\%$ of the flux emerges
from the jets than the lobes, then the object was deemed a jet
dominated system.

After incorporating the galaxy cluster \pcav-\prad\ data from
\citep{birzan08}, we found a continuous power-law relation between
\pcav-\prad\ in both frequency regimes covering 6 decades in
\prad\ and 8 decades in \pcav (see Figure \ref{fig:pcav}). Using a
log-space \bces\ fit, we found similar power laws for the
\pcav-\prad\ relations for the 1.4 GHz and 200-400 MHz data with the
form $\pcav \propto \prad^{0.7}$. This relation comes with the caveat
of $\approx 1.0$ dex cosmic scatter in both parameters, and hence the
relation is weakly applicable in determining the {\it{total}} energy
output by an AGN.

Utility and consequences...?

%%%%%%%%%%%%%%%%%
\acknowledgements
%%%%%%%%%%%%%%%%%

KWC and BRM acknowledge generous support from Canadian National
Science and Engineering Research Council grants NSERC-**LIST** and
\cxo\ grants CXO-**LIST**. The \chandra\ X-ray Observatory Center is
operated by the Smithsonian Astrophysical Observatory for and on
behalf of NASA under contract NAS8-03060. The National Radio Astronomy
Observatory is a facility of the National Science Foundation operated
under cooperative agreement by Associated Universities, Inc.

%%%%%%%%%%%%%%
% Facilities %
%%%%%%%%%%%%%%

{\it Facilities:} \facility{CXO (ACIS)} \facility{VLA}

%%%%%%%%%%%%%%%%
% Bibliography %
%%%%%%%%%%%%%%%%

\bibliography{cavagnolo}

%%%%%%%%%%%%%%%%%%%%%%
% Figures  and Tables%
%%%%%%%%%%%%%%%%%%%%%%

\begin{figure}
  \begin{center}
    \begin{minipage}{\linewidth}
      \includegraphics*[width=\textwidth, trim=0mm 0mm 0mm 0mm, clip]{rbs797.ps}
    \end{minipage}
    \caption{Fluxed, unsmoothed 0.7--2.0 keV clean image of \rbs\ in
      units of ph \pcmsq\ \ps\ pix$^{-1}$. Image is $\approx 250$ kpc
      on a side and coordinates are J2000 epoch. Black contours in the
      nucleus are 2.5--9.0 keV X-ray emission of the nuclear point
      source; the outer contour approximately traces the 90\% enclosed
      energy fraction (EEF) of the \cxo\ point spread function. The
      dashed green ellipse is centered on the nuclear point source,
      encloses both cavities, and was drawn by-eye to pass through the
      X-ray ridge/rims.}
    \label{fig:img}
  \end{center}
\end{figure}

\begin{figure}
  \begin{center}
    \begin{minipage}{0.495\linewidth}
      \includegraphics*[width=\textwidth, trim=0mm 0mm 0mm 0mm, clip]{325.ps}
    \end{minipage}
   \begin{minipage}{0.495\linewidth}
      \includegraphics*[width=\textwidth, trim=0mm 0mm 0mm 0mm, clip]{8.4.ps}
   \end{minipage}
   \begin{minipage}{0.495\linewidth}
      \includegraphics*[width=\textwidth, trim=0mm 0mm 0mm 0mm, clip]{1.4.ps}
    \end{minipage}
    \begin{minipage}{0.495\linewidth}
      \includegraphics*[width=\textwidth, trim=0mm 0mm 0mm 0mm, clip]{4.8.ps}
    \end{minipage}
     \caption{Radio images of \rbs\ overlaid with black contours
       tracing ICM X-ray emission. Images are in mJy beam$^{-1}$ with
       intensity beginning at $3\sigma_{\rm{rms}}$ and ending at the
       peak flux, and are arranged by decreasing size of the
       significant, projected radio structure. X-ray contours are from
       $2.3 \times 10^{-6}$ to $1.3 \times 10^{-7}$ ph
       \pcmsq\ \ps\ pix$^{-1}$ in 12 square-root steps. {\it{Clockwise
           from top left}}: 325 MHz \vla\ A-array, 8.4 GHz
       \vla\ D-array, 4.8 GHz \vla\ A-array, and 1.4 GHz
       \vla\ A-array.}
    \label{fig:composite}
  \end{center}
\end{figure}

\begin{figure}
  \begin{center}
    \begin{minipage}{0.495\linewidth}
      \includegraphics*[width=\textwidth, trim=0mm 0mm 0mm 0mm, clip]{sub_inner.ps}
    \end{minipage}
    \begin{minipage}{0.495\linewidth}
      \includegraphics*[width=\textwidth, trim=0mm 0mm 0mm 0mm, clip]{sub_outer.ps}
    \end{minipage}
    \caption{Red text point-out regions of interest discussed in
      Section \ref{sec:cavities}. {\it{Left:}} Residual 0.3-10.0 keV
      X-ray image smoothed with $1\arcs$ Gaussian. Yellow contours are
      1.4 GHz emission (\vla\ A-array), orange contours are 4.8 GHz
      emission (\vla\ A-array), orange vector is 4.8 GHz jet axis, and
      red ellipses outline definite cavities. {\it{Bottom:}} Residual
      0.3-10.0 keV X-ray image smoothed with $3\arcs$ Gaussian. Green
      contours are 325 MHz emission (\vla\ A-array), blue contours are
      8.4 GHz emission (\vla\ D-array), and orange vector is 4.8 GHz
      jet axis.}
    \label{fig:subxray}
  \end{center}
\end{figure}

\begin{figure}
  \begin{center}
    \begin{minipage}{\linewidth}
      \includegraphics*[width=\textwidth]{r797_nhfro.eps}
      \caption{Gallery of radial ICM profiles. Vertical black dashed
        lines mark the approximate end-points of both
        cavities. Horizontal dashed line on cooling time profile marks
        age of the Universe at redshift of \rbs. For X-ray luminosity
        profile, dashed line marks \lcool, and dashed-dotted line
        marks \pcav.}
      \label{fig:gallery}
    \end{minipage}
  \end{center}
\end{figure}

\begin{figure}
  \begin{center}
    \begin{minipage}{\linewidth}
      \setlength\fboxsep{0pt}
      \setlength\fboxrule{0.5pt}
      \fbox{\includegraphics*[width=\textwidth]{cav_config.eps}}
    \end{minipage}
    \caption{Cartoon of possible cavity configurations. Arrows denote
      direction of AGN outflow, ellipses outline cavities, \rlos\ is
      line-of-sight cavity depth, and $z$ is the height of a cavity's
      center above the plane of the sky. {\it{Left:}} Cavities which
      are symmetric about the plane of the sky, have $z=0$, and are
      inflating perpendicular to the line-of-sight. {\it{Right:}}
      Cavities which are larger than left panel, have non-zero $z$,
      and are inflating along an axis close to our line-of-sight.}
    \label{fig:config}
  \end{center}
\end{figure}

\begin{figure}
  \begin{center}
    \begin{minipage}{0.495\linewidth}
      \includegraphics*[width=\textwidth, trim=25mm 0mm 40mm 10mm, clip]{edec.eps}
    \end{minipage}
    \begin{minipage}{0.495\linewidth}
      \includegraphics*[width=\textwidth, trim=25mm 0mm 40mm 10mm, clip]{wdec.eps}
    \end{minipage}
    \caption{Surface brightness decrement as a function of height
      above the plane of the sky for a variety of cavity radii. Each
      curve is labeled with the corresponding \rlos. The curves
      furthest to the left are for the minimum \rlos\ needed to
      reproduce $y_{\rm{min}}$, \ie\ the case of $z = 0$, and the
      horizontal dashed line denotes the minimum decrement for each
      cavity. {\it{Left}} Cavity E1; {\it{Right}} Cavity W1.}
    \label{fig:decs}
  \end{center}
\end{figure}


\begin{figure}
  \begin{center}
    \begin{minipage}{\linewidth}
      \includegraphics*[width=\textwidth, trim=15mm 5mm 5mm 10mm, clip]{pannorm.eps}
      \caption{Histograms of normalized surface brightness variation
        in wedges of a $2.5\arcs$ wide annulus centered on the X-ray
        peak and passing through the cavity midpoints. {\it{Left:}}
        $36\mydeg$ wedges; {\it{Middle:}} $14.4\mydeg$ wedges;
        {\it{Right:}} $7.2\mydeg$ wedges. The depth of the cavities
        and prominence of the rims can be clearly seen in this plot.}
      \label{fig:pannorm}
    \end{minipage}
  \end{center}
\end{figure}

\begin{figure}
  \begin{center}
    \begin{minipage}{0.5\linewidth}
      \includegraphics*[width=\textwidth, angle=-90]{nucspec.ps}
    \end{minipage}
    \caption{X-ray spectrum of nuclear point source. Black denotes
      year 2000 \cxo\ data (points) and best-fit model (line), and red
      denotes year 2007 \cxo\ data (points) and best-fit model (line).
      The residuals of the fit for both datasets are given below.}
    \label{fig:nucspec}
  \end{center}
\end{figure}

\begin{figure}
  \begin{center}
    \begin{minipage}{\linewidth}
      \includegraphics*[width=\textwidth, trim=10mm 5mm 10mm 10mm, clip]{radiofit.eps}
    \end{minipage}
    \caption{Best-fit continuous injection (CI) synchrotron model to
      the nuclear 1.4 GHz, 4.8 GHz, and 7.0 keV X-ray emission. The
      two triangles are \galex\ UV fluxes showing the emission is
      boosted above the power-law attributable to the nucleus.}
    \label{fig:sync}
    \end{center}
\end{figure}

\begin{figure}
  \begin{center}
    \begin{minipage}{\linewidth}
      \includegraphics*[width=\textwidth, trim=0mm 0mm 0mm 0mm, clip]{rbs797_opt.ps}
    \end{minipage}
    \caption{\hst\ \myi+\myv\ image of the \rbs\ BCG with units e$^-$
      s$^{-1}$. Green, dashed contour is the \cxo\ 90\% EEF. Emission
      features discussed in the text are labeled.}
    \label{fig:hst}
  \end{center}
\end{figure}

\begin{figure}
  \begin{center}
    \begin{minipage}{0.495\linewidth}
      \includegraphics*[width=\textwidth, trim=0mm 0mm 0mm 0mm, clip]{suboptcolor.ps}
    \end{minipage}
    \begin{minipage}{0.495\linewidth}
      \includegraphics*[width=\textwidth, trim=0mm 0mm 0mm 0mm, clip]{suboptrad.ps}
    \end{minipage}
    \caption{{\it{Left:}} Residual \hst\ \myv\ image. White regions
      (numbered 1--8) are areas with greatest color difference with
      \rbs\ halo. {\it{Right:}} Residual \hst\ \myi\ image. Green
      contours are 4.8 GHz radio emission down to
      $1\sigma_{\rm{rms}}$, white dashed circle has radius $2\arcs$,
      edge of ACS ghost is show in yellow, and southern whiskers are
      numbered 9--11 with corresponding white lines.}
    \label{fig:subopt}
  \end{center}
\end{figure}


%%%%%%%%%%%%%%%%%%%%
% End the document %
%%%%%%%%%%%%%%%%%%%%
\end{document}
