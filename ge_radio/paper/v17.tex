%%%%%%%%%%%%%%%%%%%
% Custom commands %
%%%%%%%%%%%%%%%%%%%

\newcommand{\birzan}{B\^irzan}
\newcommand{\samp}{21}
\newcommand{\phigh}{\ensuremath{P_{\mathrm{1.4}}}}
\newcommand{\pthree}{\ensuremath{P_{\mathrm{327}}}}
\newcommand{\plow}{\ensuremath{P_{\mathrm{200-400}}}}
\newcommand{\shigh}{\ensuremath{\sigma_{\mathrm{1.4}}}}
\newcommand{\sthree}{\ensuremath{\sigma_{\mathrm{327}}}}
\newcommand{\slow}{\ensuremath{\sigma_{\mathrm{200-400}}}}
\newcommand{\rhigh}{\ensuremath{r_{\mathrm{1.4}}}}
\newcommand{\rlow}{\ensuremath{r_{\mathrm{200-400}}}}
\newcommand{\mytitle}{A RELATIONSHIP BETWEEN AGN JET POWER AND RADIO POWER}
\newcommand{\mystitle}{AGN \pjet-\prad\ Relation}

%%%%%%%%%%
% Header %
%%%%%%%%%%

%\documentclass[12pt, preprint]{aastex}
%\documentclass{aastex}
\documentclass{emulateapj}
\usepackage{apjfonts,graphicx,here,common,longtable,ifthen,amsmath,amssymb,natbib}
\usepackage[pagebackref,
  pdftitle={\mytitle},
  pdfauthor={Dr. Kenneth W. Cavagnolo},
  pdfsubject={Astrophysical Journal},
  pdfkeywords={galaxies: active -- galaxies: clusters: general --
    X-rays: galaxies -- radio continuum: galaxies},
  pdfproducer={LaTeX with hyperref},
  pdfcreator={LaTeX with hyperref}
  pdfdisplaydoctitle=true,
  colorlinks=true,
  citecolor=blue,
  linkcolor=blue,
  urlcolor=blue]{hyperref}
\bibliographystyle{apj}
\begin{document}
\title{\mytitle}
\shorttitle{\mystitle}
\author{
K. Cavagnolo\altaffilmark{1,6},
B. McNamara\altaffilmark{1,2,3},
P. Nulsen\altaffilmark{3},\\
C. Carilli\altaffilmark{4},
C. Jones\altaffilmark{3},
\& L. \birzan\altaffilmark{5},
}
\altaffiltext{1}{Department of Physics and Astronomy, University of
  Waterloo, Waterloo, ON N2L 3G1, Canada.}
\altaffiltext{2}{Perimeter Institute for Theoretical Physics, 31
  Caroline Street N, Waterloo, ON N2L 2Y5, Canada.}
\altaffiltext{3}{Harvard-Smithsonian Center for Astrophysics, 60
  Garden Street, Cambridge, MA 01238, USA.}
\altaffiltext{4}{National Radio Astronomy Observatory, P.O. Box 0,
  Socorro, NM 87801-0387, USA.}
\altaffiltext{5}{Leiden Observatory, University of Leiden, P.O. 9513,
  2300 RA Leiden, The Netherlands}
\altaffiltext{6}{kcavagno@uwaterloo.ca}
\shortauthors{K. W. Cavagnolo et al.}
\journalinfo{}
\slugcomment{For submission to ApJ}

%%%%%%%%%%%%
% Abstract %
%%%%%%%%%%%%

\begin{abstract}
  Utilizing \chandra\ X-ray and VLA radio data, we investigate the
  scaling relationship between jet power, \pjet, and synchrotron
  luminosity, \prad. We expand the sample presented in
  \citet{birzan08} to lower radio power by incorporating measurements
  for \samp\ gEs to determine if the \citet{birzan08}
  \pjet-\prad\ scaling relations are continuous in form and scatter
  from giant elliptical galaxies (gEs) up to brightest cluster
  galaxies (BCGs). We find a mean scaling relation of $\pjet \approx
  5.8 \times 10^{43} (\prad/10^{40})^{0.70} ~\lum$ which is continuous
  over $\sim 6-8$ decades in \pjet\ and \prad\ with a scatter of
  $\approx 0.7$ dex. Our mean scaling relationship is consistent with
  the model presented in \citet{w99} when the ratio of lobe energy in
  non-radiating particles to that in relativistic electrons is $\ga
  100$. Focusing on a sub-set of gEs with what we suspect are poorly
  confined radio sources, we discuss the importance of environment
  when measuring a \pjet-\prad\ relation, and a possible connection to
  the process of entrainment.
\end{abstract} 

%%%%%%%%%%%%
% Keywords %
%%%%%%%%%%%%

\keywords{galaxies: active -- galaxies: clusters: general -- X-rays:
  galaxies -- radio continuum: galaxies}

%%%%%%%%%%%%%%%%%%%%%%
\section{Introduction}
\label{sec:intro}
%%%%%%%%%%%%%%%%%%%%%%

Observational evidence indicates that most galaxies harbor a central
supermassive black hole (SMBH) which likely co-evolved with the host
galaxy, giving rise to correlations between bulge luminosity, stellar
velocity dispersion, and central black hole mass
\citep{1995ARA&A..33..581K, magorrian}. Models suggest these
correlations were imprinted via galaxy mergers and the influence of
feedback from active galactic nuclei (AGN)
\citep[\eg][]{1998A&A...331L...1S, 2000MNRAS.311..576K}. Around the
time of these discoveries, the \cxo\ found direct evidence for AGN
feedback when observations revealed cavities and shock fronts in the
X-ray emitting gas surrounding many massive galaxies
\citep[\eg][]{2000ApJ...534L.135M, perseus1,
  2007ApJ...665.1057F}.

Studies of X-ray cavities have shown that AGN feedback supplies enough
energy to regulate star formation and suppress cooling of the hot
halos of galaxies and clusters \citep{birzan04, 2005MNRAS.364.1343D,
  rafferty06}. Further supported by numerical simulations
\citep[\eg][]{croton06, bower06}, a consensus has emerged that AGN
feedback plays an important role in regulating galaxy evolution at
late times. However, one long-standing problem which is central to
understanding AGN feedback is finding reliably observational methods
for measuring total AGN mechanical output.

X-ray cavities provide a good gauge of total AGN mechanical energy and
mean jet power, \pjet\ \citep{2000ApJ...534L.135M}. Jet powers are
approximated using estimates of cavity age and of the energy required
to create the cavity \citep{2001ApJ...554..261C,
  mcnamrev}. Relationships between \pjet\ and associated synchrotron
luminosity, \prad, were presented in \citet[][hereafter B04]{birzan04}
and \citet[][hereafter B08]{birzan08}. In B08, scaling relations
between \pjet\ and 327 MHz, 1.4 GHz, and bolometric radio luminosities
were discussed. B08 found that, depending on frequency, $\pjet \propto
\prad^{0.5-0.7}$. However, there are few objects in the B08 study with
$\prad \lesssim 10^{38} ~\lum$ and $\pjet \lesssim 10^{43} ~\lum$.

In this paper we extend the study of B08 with the inclusion of
\samp\ gEs from systems with lower X-ray luminosities and jet powers
than rich clusters. Combining the B08 results and the results in this
paper, we find a relationship between jet power and radio power which
is independent of radio frequency, spans 6-8 orders of magnitude in
\pjet\ and \prad, and has the general form $\pjet \approx 5.8 \times
10^{43} (\prad/10^{40})^{0.70} ~\lum$ with dominantly intrinsic
scatter of $\sim 0.7$ dex. We also find encouraging similarity between
our relations and AGN models, specifically \citet{w99}.

We outline the sample of gEs in \S\ref{sec:sample}. X-ray and radio
measurements are discussed in \S\ref{sec:data}. Results and discussion
are presented in \S\ref{sec:r&d}. The summary and concluding remarks
are given in \S\ref{sec:summary}. \LCDM\ All quoted uncertainties are
68\% confidence.

%%%%%%%%%%%%%%%%%%
\section{Sample}
\label{sec:sample}
%%%%%%%%%%%%%%%%%%

The \samp\ gEs in this study were taken from the sample of 160 gEs
compiled by Jones et al. \citetext{in preparation}. The B08 sample is
taken from \citet[][hereafter R06]{rafferty06}. Information regarding
our gE sample is listed in Table \ref{tab:sample}. The Jones et
al. compilation is drawn from the samples of
\citet{1999MNRAS.302..209B} and \citet{2003MNRAS.340.1375O} using the
criteria that the $K$-band luminosity is $> 10^{10}~\Lsol$ and the
object has been observed with \chandra. Of the 160 gEs, AGN activity
was suspected in \samp\ objects based solely on the presence of
surface brightness depressions in the X-ray emitting gas. Most of the
gEs studied here have X-ray halos and radio sources with luminosities
lower than are typically found for cDs and BCGs.

%%%%%%%%%%%%%%%%%%%%%%%%%%%%%%%%%%%%%%%%
\section{Observations and Data Analysis}
\label{sec:data}
%%%%%%%%%%%%%%%%%%%%%%%%%%%%%%%%%%%%%%%%

%%%%%%%%%%%%%%%%%%
\subsection{X-ray}
\label{sec:xray}
%%%%%%%%%%%%%%%%%%

Jet powers were determined in the usual manner from the X-ray data
(see B04 and R06). Gas properties used here are taken from the
analysis by Jones et al. (in preparation). Cavity locations and sizes
are from Nulsen et al. (in preparation), with cavity volumes and their
errors calculated by the method of B04. The energy of each cavity was
estimated as $4pV$, the enthalpy of a cavity filled with relativistic
gas. To estimate average cavity power, \pcav, the energy of each
cavity was divided by an estimate of its age. For compatibility with
B08, we give results derived assuming the age of the cavities is
approximated by the buoyant rise time, $t_{\mathrm{buoy}}$
\citep{2001ApJ...554..261C, 2003ApJ...592..839B}. It is important to
note that \pcav\ is the measured quantity and is assumed to be a good
estimate of the physical quantity \pjet. Further, \pcav\ does not
include the energy which may go into shocks, which at times may exceed
the energy in cavities. The cavity power estimates were summed for all
the cavities in each system to obtain an estimate of the average AGN
outburst power. Because the uncertainties are large (chiefly due to
the uncertain volume estimates), errors are propagated in log space
assuming that the errors in the observable inputs ambient pressure,
cavity volume, and cavity age are independent of one another.

%%%%%%%%%%%%%%%%%%
\subsection{Radio}
\label{sec:radio}
%%%%%%%%%%%%%%%%%%

Radio powers were estimated using the relation $\nu_0 L_{\nu_0} = 4
\pi D_L^2 (1+z)^{\alpha-1} S_{\nu_0} \nu_0$, where $S_{\nu_0}$ is the
flux density at the observed frequency, $\nu_0$, over the integrated
area of the source, $z$ is redshift, $D_L$ is luminosity distance, and
$\alpha$ is radio spectral index. The radio powers estimated using
200-400 MHz and 1.4 GHz fluxes are denoted as \plow\ and \phigh,
respectively. The redshift dependent $\alpha$ correction in $\nu_0
L_{\nu_0}$ is small for our sample since the objects are nearby. We
have assumed the radio spectra behave as $S_{\nu} \propto
\nu^{-\alpha}$ with a spectral index of $\alpha = 0.8$, typical for
extragalactic radio galaxies \citep{1992ARA&A..30..575C}. The 1.4 GHz
radio flux for each source was taken from the NRAO VLA Sky Survey
(NVSS, \citealt{nvss}). NGC 1553 lies outside the NVSS survey area, so
the 1.4 GHz flux was estimated using the 843 MHz Sydney University
Molonglo Sky Survey (SUMSS, \citealt{sumss1}) flux and $\alpha =
0.89$; $\alpha$ was derived using the SUMSS and 5 GHz Parkes
\citep{1970ApL.....5...29W} fluxes.

The radio morphologies for our sample are heterogeneous: some are
large and extended, while others are compact. As a result, most
compact sources have a single catalog entry, while large sources are
divided among multiple entries. To ensure the entire radio source was
measured, a fixed physical aperture of 1 Mpc was searched around the
X-ray centroid of each gE. For each target field, all detected radio
sources were overlaid on a composite image of X-ray, optical (DSS
I/II\footnote{http://archive.stsci.edu/dss/}), and infrared emission
(2MASS\footnote{http://www.ipac.caltech.edu/2mass/}). When available,
the deeper and higher resolution radio data from VLA
FIRST\footnote{http://sundog.stsci.edu} was included. A visual
inspection was performed to establish which radio sources were
associated with the target gE. After confirming which catalog sources
are associated with the target gE, the fluxes of the individual
catalog sources were added and the associated uncertainties summed in
quadrature.

Archival VLA data for each source was also reduced and analyzed. The
continuum VLA data were reduced using a customized version of the NRAO
VLA Archive Survey\footnote{http://www.aoc.nrao.edu/$\sim$vlbacald}
reduction pipeline. In the cases where high-resolution VLA archival
data is available, multifrequency images were used to confirm the
connection between NVSS detections and the host gE. Images at 1.4 GHz
were further used to check NVSS fluxes. We found flux agreement for
most sources, the exceptions being IC 4296 and NGC 4782, where the
NVSS flux is approximately a factor of 2 lower. The radio lobes for IC
4296 and NGC 4782 contain significant power in diffuse, extended
emission which is not detected in NVSS because of the higher flux
limit. For these sources, the fluxes measured from the archival VLA
data are used in our analysis. For the systems where nuclear radio
emission was resolved, we found, on average
$S_{\nu_0,\mathrm{nucleus}} / S_{\nu_0,\mathrm{total}} \la 0.1$,
suggesting the nuclear contribution to the low-resolution NVSS
measurements has a small impact on our results.

B08 found that using lower frequency radio data, \ie\ 327 MHz versus
1400 MHz, resulted in a lower scatter \pjet-\prad\ relation. The
quality and availability of 327 MHz data for our gE sample were not
ideal, thus we gathered low-frequency radio fluxes from the CATS
Database\footnote{http://www.sao.ru/cats/} \citep{cats}. The CATS
Database is a compilation of more than 350 radio catalogs (\eg\ WENSS,
WISH, TXS, B3). Low-frequency (200-400 MHz) counterparts to the NVSS
and SUMSS sources were retrieved from CATS. Of the \samp\ gEs in our
sample, 17 of them were found to have radio sources in the CATS
database with fluxes in the 200-400 MHz range. CATS does not provide
images for visual inspection and is composed of catalogs having a
variety of spatial resolutions and flux limits. Thus, the 200-400 MHz
radio powers shown in Figure \ref{fig:pcav} may include some
contribution from background sources.

%%%%%%%%%%%%%%%%%%%%%%%%%%%%%%%%
\section{Results and Discussion}
\label{sec:r&d}
%%%%%%%%%%%%%%%%%%%%%%%%%%%%%%%%

%%%%%%%%%%%%%%%%%%%%%%%%%%%%%%%%%%%%%%%%%%%
\subsection{\pjet-\prad\ Scaling Relations}
\label{sec:relation}
%%%%%%%%%%%%%%%%%%%%%%%%%%%%%%%%%%%%%%%%%%%

The results from the X-ray and radio data analysis are shown in the
plots of \pcav-\phigh\ and \pcav-\plow\ presented in Figure
\ref{fig:pcav}. A figure of merit (FM) was assigned to each set of
cavities through visual inspection-- shown as color coding in Figure
\ref{fig:pcav} and listed in Table \ref{tab:sample}. FM-A cavities are
associated with AGN radio activity and have well-defined boundaries;
FM-B cavities are associated with AGN radio activity but lack
well-defined boundaries; FM-C cavities have poorly-defined boundaries
and their connection to AGN radio activity is unclear. FM-C cavities
are excluded from all fitting, as are a subset of objects we have
defined as being poorly confined (discussed in Section \ref{sec:jet}
and excluded from Figure \ref{fig:pcav}).

Figure \ref{fig:pcav} shows a continuous, power-law relationship
between cavity power and radio power spanning 8 orders of magnitude in
radio power and 6 orders of magnitude in cavity power. To determine
the form of the power-law relation, we performed linear fits in
log-space for each frequency regime using the bivariate correlated
error and intrinsic scatter (\bces) algorithm \citep{bces}. The
orthogonal \bces\ algorithm takes in asymmetric uncertainties for both
variables, assumes the presence of intrinsic scatter, and performs a
linear least-squares regression which minimizes the squared orthogonal
distance to the best-fit relation. Parameter uncertainties were
calculated using 10,000 Monte Carlo bootstrap resampling trials. Our
fits differ from the method used in B08 which minimized the distance
in the \pcav\ coordinate.

The best-fit log-space orthogonal \bces\ relations are:
\begin{eqnarray}
  \log~\pcav &=& 0.75~(\pm 0.14)~\log~\phigh + 1.91~(\pm 0.18) \label{eqn:high}\\
  \log~\pcav &=& 0.64~(\pm 0.09)~\log~\plow + 1.54~(\pm 0.12) \label{eqn:low}
\end{eqnarray}
where \pcav\ is in units $10^{42} ~\lum$, and \phigh\ and \plow\ are
in units $10^{40} ~\lum$. The scatter for each relation is $\shigh =
0.78$ dex and $\slow = 0.61$ dex, and the respective correlation
coefficients are \rhigh\ = 0.72 and \rlow\ = 0.81. We have quantified
the total scatter about the best-fit relation using a weighted
estimate of the orthogonal distances to the best-fit line
\citep[see][]{2009A&A...498..361P}. For comparison, the B08 scaling
relations are
\begin{eqnarray}
  \log~\pcav &=& 0.35~(\pm 0.07)~\log~\phigh + 1.85~(\pm 0.10) \\
  \log~\pcav &=& 0.51~(\pm 0.07)~\log~\pthree + 1.51~(\pm 0.12) \label{eqn:err}
\end{eqnarray}
where \pcav\ is in units $10^{42} ~\lum$, and \phigh\ and \pthree\ are
in units $10^{24}$ W Hz$^{-1}$ (or $\approx 10^{40} ~\lum$). The B08
relations have scatters of $\shigh = 0.85$ dex and $\sthree = 0.81$
dex. Equation 15 of B08 contains an error \citep{birzan08err}, and the
correct version is given in Equation \ref{eqn:err} above.

In contrast to B08, the slopes of the relations in this work now agree
to within their uncertainties. Note that we find a steeper
relationship at 1.4 GHz than B08. The difference in slope at 1.4 GHz
between our work and B08 is due to the additional data points at lower
\pjet\ and the different fitting method. The B08 points tend to be
clumped in a fairly narrow power range, which gave the few points at
the upper and lower power extremes excessive leverage over the
slope. The new data extends to lower jet powers, giving a more uniform
sampling and an improved measurement of the slope and zero point.

B08 showed that correcting for the effect of radio aging by including
a scaling with break frequency (\ie\ source age) reduces the scatter
within the relations by $\approx 50\%$. The substantial scatter in the
\pjet-\prad\ relations suggest radio lobe properties which effect
synchrotron emission, \eg\ age, composition, or magnetic field
configuration, make AGN power determination for an individual system
via \prad\ difficult. The scatter in \pjet-\prad\ may be particularly
important for gEs, which have steeper pressure profiles and are more
susceptible to disruption by AGN outbursts \citep{2006MNRAS.372.1161W,
  2008ApJ...687L..53P}.

%%%%%%%%%%%%%%%%%%%%%%%%%%%%%%%%%%%%%%%%%%%%%%%%%%%%%%%%%%%%%
\subsection{Comparison with Models and Observational Studies}
\label{sec:models}
%%%%%%%%%%%%%%%%%%%%%%%%%%%%%%%%%%%%%%%%%%%%%%%%%%%%%%%%%%%%%

Relations presented in \citet[][hereafter W99]{w99} are commonly
utilized to estimate total AGN kinetic power from observed radio
power. It is therefore useful to compare our results with those of
W99. For simplicity, in this section we use the parameterization
$\pjet = \eta \prad^{\Gamma}$, where \pjet\ is total kinetic jet
power, $\eta$ is some normalization, $\Gamma$ is a scaling index, and
\prad\ is emergent synchrotron power.

W99 derive $\Gamma$ and $\eta$ using the hypersonic jet model of
\citet{1991MNRAS.250..581F} and assuming the radio lobes are at
minimum energy density \citep[see][for
  details]{1980ARA&A..18..165M}. W99 derived $\Gamma = 6/7 ~(\approx
0.86)$ with $\eta \approx f^{3/2}~4.61 \times 10^{41} ~\lum$ when
\prad\ is in units of $10^{40} ~\lum$. We have adjusted the fiducial
W99 normalization from 151 MHz to 1.4 GHz assuming $S_{\nu} \propto
\nu^{-0.8}$. The factor $f$ consolidates a variety of unknowns (see
W99 for details). The fiducial W99 model ($f=1$) yields $\eta$ two
orders of magnitude below our normalizations, but the slopes formally
agree (see Figure \ref{fig:radeff}).

The W99 normalization has a weak dependence on ambient gas density.
Using observationally consistent shallower and lower density gas
profiles than those used to derive the fiducial W99 model results in
faster jet outflow velocities, which in turn increases $\eta$ by
factors of $\sim 2-5$. Similarly, the fractional deviation from the
minimum-energy condition is assumed to be small. More importantly, the
W99 model depends strongly on $k$, which is the ratio of energy in
non-radiating particles to relativistic electrons. We find that for
$k$ lying in the range of tens to thousands, values consistent with
observational findings \citep{2005MNRAS.364.1343D,
  2006MNRAS.372.1741D, 2006ApJ...648..200D, birzan08}, the W99
normalization is brought into agreement with our work. W99 find that
to fit their model to AGN narrow-line region luminosities (assuming
$L_{\mathrm{NLR}} \propto \pjet$) for sources in the 7C and 3CRR
surveys, $f$ must equal 20, which is the upper-limit of $f$ in their
model and implies $k \approx 20$. The scatter in our relations may
arise from a dependence on $k$, \ie\ from intrinsic differences in
radio sources (light and heavy jets), or because confined jets are
born light and become heavy on large scales due to entrainment.

For flat-spectrum compact radio cores (\ie\ small scale jets and not
radio lobes), several jet models predict $\Gamma = 12/17 ~(\approx
0.71)$ \citep{1979ApJ...232...34B, 1995A&A...293..665F,
  2003MNRAS.343L..59H}. Observational studies by
\citet{2005ApJ...633..384H} and \citet{2007MNRAS.381..589M} using
nuclear radio powers and \pjet\ estimates based on the Galactic X-ray
binary mass-radio-X-ray fundamental plane \citep{2003MNRAS.344...60G,
  2003MNRAS.345.1057M} found $\Gamma$'s and $\eta$'s consistent with
our relations. The similarity of our \pjet-\prad\ relations with the
aforementioned studies may be coincidental given that our measurements
are for integrated radio emission and not just nuclear radio
emission. But, perhaps they agree because nuclear flux and total flux
are correlated in these systems.

%%%%%%%%%%%%%%%%%%%%%%%%%%%%%%%%%%%%
\subsection{Poorly Confined Sources}
\label{sec:jet}
%%%%%%%%%%%%%%%%%%%%%%%%%%%%%%%%%%%%

The X-ray data are too shallow to image the extent of the cavities for
IC 4296, NGC 315, NGC 4261, NGC 4782, and NGC 7626. For these objects,
the radio lobes extend beyond the observed X-ray halo and the radio
morphologies are distinctly different from the rest of the gE and B08
objects. X-ray cavities typically enclose the radio emission, for
example in M84 \citep{2008ApJ...686..911F} where the interaction takes
on some complexity (see Figure \ref{fig:pics}). However, for the gEs
listed above, the AGN interaction with the X-ray halo appears to end
beyond a certain point (see Figure \ref{fig:pics} for an example in
NGC 4261). The appearance of breaking-out from the X-ray halo is our
reason for suggesting these radio sources are poorly confined
(PC). Note that the PC systems have been excluded from analysis of the
\pjet-\prad\ relations.

To compare the properties of PC sources with the rest of our sample,
we calculated \pcav\ values for these systems assuming radio lobe
volume equals the volume of cavities potentially at larger
radii. Pressure profiles were extrapolated to large radii using
$\beta$-models \citep{betamodel} fitted to the surface brightness
profiles. We assumed isothermal atmospheres and a background gas
pressure of $10^{-13}$ dyne \pcmsq. The assumed background pressure is
based on the mean value observed in the outskirts of clusters and
groups (see the
\accept\ database\footnote{http://www.pa.msu.edu/astro/MC2/accept} for
a catalog of such pressure profiles). Cavity buoyancy ages cannot be
directly calculated from the data, but to maintain continuity with the
rest of the study, $t_{\mathrm{buoy}}$ was estimated by scaling the
gas sound speed by 0.65 which is the mean value of the ratio
$t_{\mathrm{cs}}/t_{\mathrm{buoy}}$ for the B04 sample. As a result of
the radio lobes extending into regions where the pressure profiles are
steep and approaching the background pressure, the large lobe volumes
are offset by low pressures and long ages, resulting in modest values
of \pcav.

Two of the PC sources, NGC 315 and NGC 4261, are in a sample of nine
FR-I objects analyzed by \citet[][hereafter C08]{2008MNRAS.386.1709C}
using \xmm\ X-ray observations. C08 provides the $4pV$ cavity energy
and the mean temperature of the lobe environments for each FR-I
source. As with the PC sources, we calculated \pcav\ for each of the
C08 FR-I sources using scaled sound speed. For N315 and N4261 we find
no significant difference between \pcav\ calculated using the
\chandra\ data and \xmm\ data. \phigh\ was calculated for each C08
source using the method outlined in Section \ref{sec:radio}. The PC
and C08 objects are plotted in Figure \ref{fig:radeff}.

In Figure \ref{fig:radeff} we highlight the location of the PC sources
relative to our best-fit 1.4 GHz relation and the fiducial W99
relation. Figure \ref{fig:radeff} shows that the PC and C08 FR-I
sources reside well below our best-fit relation. This discrepancy
implies that these sources have the lowest jet power per unit radio
power of all objects in the sample. One possible explanation is that
these sources may have lower $k$ values than the rest of our
sample. Another explanation is that PC sources are systematically more
powerful than our measurements indicate due to energy being imparted
to shocks. On average, shock energy is a modest correction to \pcav,
factor of a few in clusters \citep{mcnamrev}, but the frequency and
variety of AGN driven shocks is broad
\citep[\eg][]{2003ApJ...592..129K, hydraa, herca,
  2003ApJ...592..129K}. However, a shock explanation would require
that the fraction of jet power going into shocks is preferentially
higher for systems with relatively high implied radiative
efficiencies.

%%%%%%%%%%%%%%%%%%%%%%%%%%%%%%%%%
\section{Summary and Conclusions}
\label{sec:summary}
%%%%%%%%%%%%%%%%%%%%%%%%%%%%%%%%%

We have presented analysis of the jet power versus radio power scaling
relation for the B08 sample and a sample of \samp\ giant elliptical
galaxies observed with the \cxo. Cavity powers were calculated for
each set of cavities using similar methods to those outlined in
R06. Radio powers for our sample were estimated using 1.4 GHz and
200-400 MHz fluxes taken from the NVSS/SUMSS surveys and the CATS
database, respectively. We find a continuous power-law relation
between \pjet-\prad\ covering 6 decades in \prad\ and 8 decades in
\pjet\ (Figure \ref{fig:pcav}). We find the power laws describing the
\pjet-\prad\ trend have the mean form $\pjet \approx 5.8 \times
10^{43} (\prad/10^{40})^{0.70} ~\lum$, and a scatter about the fit of
$\approx 0.7$ dex. Our relations agree reasonably well with previous
observational studies and predictions from theoretical jet models.

Several groups have applied the \birzan\ scaling relations to study
the effects of AGN feedback on structure formation,
\eg\ \citet{best07} and \citet{2007MNRAS.379..260M}, with some groups
now suggesting that distributed low-power radio galaxies may dominate
heating of the intracluster medium,
\eg\ \citet{2009ApJ...705..854H}. Up to now, the available
observational results, which were primarily calibrated to high-power
radio sources, did not clearly indicate if a \pjet-\prad\ relation
would be continuous, or of comparable scatter, for lower power radio
sources. Assuming there is no redshift evolution of \pjet-\prad, our
relations suggest higher mass galaxies dominate over lower mass
galaxies in the process of mechanical heating within clusters and
groups. The next step is to study the process of mechanical heating in
a cosmological context, \eg\ evaluating the contribution of
mechanically dominated AGN on the bolometric AGN luminosity function
over cosmic time \citep[\ie][]{2009MNRAS.395..518C}.

The subset of objects with radio sources that are poorly confined by
their hot halo have \pjet/\prad\ ratios which are large relative to
the rest of our sample (see Figure \ref{fig:radeff}). In addition, PC
sources reside in the same region of the \pjet-\prad\ plane as the
FR-I sources taken from C08. Possible explanations for the nature of
these sources may lie in the long-standing effort to understand the
connection between properties of radio galaxies and their
environment. Radio emission from lobes depends on their composition,
so that the large scatter in the \pjet-\prad\ relationship may result
from processes such as gas entrainment and shocks. It seems likely
that some of the scatter arises as radio sources age, but it remains
unclear what other factors are important.

With tighter, self-consistent observational constraints on the
\pjet-\prad\ relation across several decades in both parameters, and
host system mass, the need to rely on theoretical models to relate jet
power to observed radio properties has weakened. With the aide of
\pjet-\prad\ and statistical populations of radio galaxies found from
monochromatic all-sky radio surveys, constraints on the kinetic
heating of the Universe over vast swathes of cosmic time can be made.
As a consequence, inferences can be drawn about AGN duty cycles, the
total accretion history of SMBHs, and the growth of SMBHs as a
function of redshift. Moreover, future AGN feedback models, when used
in simulations, must replicate the observed relations studied in this
work, possibly yielding a better understanding of the power plants
which underlie AGN activity.

%%%%%%%%%%%%%%%%%
\acknowledgements
%%%%%%%%%%%%%%%%%

KWC and BRM acknowledge generous support from the Natural Sciences and
Engineering Research Council of Canada and grants from the \cxo. CJ
thanks the Smithsonian Institution for generous support. PN thanks the
\cxo\ Center for supporting this work. KWC thanks Judith Croston,
David Rafferty, Lorant Sjouwerman, and Chris Willott for helpful
discussions. The \cxo\ Center is operated by the Smithsonian
Astrophysical Observatory for and on behalf of NASA under contract
NAS8-03060. The National Radio Astronomy Observatory is a facility of
the National Science Foundation operated under cooperative agreement
by Associated Universities, Inc. This research has made use of
software provided by the \chandra\ X-ray Center, the NASA/IPAC
Extragalactic Database operated by the California Institute of
Technology Jet Propulsion Laboratory, and NASA's Astrophysics Data
System. Some software was obtained from the High Energy Astrophysics
Science Archive Research Center, provided by NASA's Goddard Space
Flight Center.

%%%%%%%%%%%%%%
% Facilities %
%%%%%%%%%%%%%%

{\it Facilities:} \facility{CXO (ACIS)} \facility{VLA}

%%%%%%%%%%%%%%%%
% Bibliography %
%%%%%%%%%%%%%%%%

\bibliography{cavagnolo}

%%%%%%%%%%%%%%%%%%%%%%
% Figures  and Tables%
%%%%%%%%%%%%%%%%%%%%%%

\clearpage
\begin{deluxetable}{lcccccccc}
\tablewidth{0pt}
\tabletypesize{\scriptsize}
\tablecaption{Summary of Sample\label{tab:sample}}
\tablehead{\colhead{Cluster} & \colhead{Obs.ID} & \colhead{R.A.} & \colhead{Dec.} & \colhead{ExpT} & \colhead{Mode} & \colhead{ACIS} & \colhead{$z$} & \colhead{$L_{bol.}$}\\
\colhead{ } & \colhead{ } & \colhead{hr:min:sec} & \colhead{$\degr:\arcmin:\arcsec$} & \colhead{ksec} & \colhead{ } & \colhead{ } & \colhead{ } & \colhead{$10^{44}$ ergs s$^{-1}$}\\
\colhead{{(1)}} & \colhead{{(2)}} & \colhead{{(3)}} & \colhead{{(4)}} & \colhead{{(5)}} & \colhead{{(6)}} & \colhead{{(7)}} & \colhead{{(8)}} & \colhead{{(9)}}
}
\startdata
1E0657 56 & \dataset [ADS/Sa.CXO\#obs/03184] {3184} & 06:58:29.627 & -55:56:39.79 & 87.5 & VF & I3 & 0.296 & 52.48\\
1E0657 56 & \dataset [ADS/Sa.CXO\#obs/05356] {5356} & 06:58:29.619 & -55:56:39.35 & 97.2 & VF & I2 & 0.296 & 52.48\\
1E0657 56 & \dataset [ADS/Sa.CXO\#obs/05361] {5361} & 06:58:29.670 & -55:56:39.80 & 82.6 & VF & I3 & 0.296 & 52.48\\
1RXS J2129.4-0741 & \dataset [ADS/Sa.CXO\#obs/03199] {3199} & 21:29:26.274 & -07:41:29.18 & 19.9 & VF & I3 & 0.570 & 20.58\\
1RXS J2129.4-0741 & \dataset [ADS/Sa.CXO\#obs/03595] {3595} & 21:29:26.281 & -07:41:29.36 & 19.9 & VF & I3 & 0.570 & 20.58\\
2PIGG J0011.5-2850 & \dataset [ADS/Sa.CXO\#obs/05797] {5797} & 00:11:21.623 & -28:51:14.44 & 19.9 & VF & I3 & 0.075 &  2.15\\
2PIGG J0311.8-2655 $\dagger$ & \dataset [ADS/Sa.CXO\#obs/05799] {5799} & 03:11:33.904 & -26:54:16.48 & 39.6 & VF & I3 & 0.062 &  0.25\\
2PIGG J2227.0-3041 & \dataset [ADS/Sa.CXO\#obs/05798] {5798} & 22:27:54.560 & -30:34:34.84 & 22.3 & VF & I2 & 0.073 &  0.81\\
3C 220.1 & \dataset [ADS/Sa.CXO\#obs/00839] {839} & 09:32:40.218 & +79:06:29.46 & 18.9 &  F & S3 & 0.610 &  3.25\\
3C 28.0 & \dataset [ADS/Sa.CXO\#obs/03233] {3233} & 00:55:50.401 & +26:24:36.47 & 49.7 & VF & I3 & 0.195 &  4.78\\
3C 295 & \dataset [ADS/Sa.CXO\#obs/02254] {2254} & 14:11:20.280 & +52:12:10.55 & 90.9 & VF & I3 & 0.464 &  6.92\\
3C 388 & \dataset [ADS/Sa.CXO\#obs/05295] {5295} & 18:44:02.365 & +45:33:29.31 & 30.7 & VF & I3 & 0.092 &  0.52\\
4C 55.16 & \dataset [ADS/Sa.CXO\#obs/04940] {4940} & 08:34:54.923 & +55:34:21.15 & 96.0 & VF & S3 & 0.242 &  5.90\\
ABELL 0013 $\dagger$ & \dataset [ADS/Sa.CXO\#obs/04945] {4945} & 00:13:37.883 & -19:30:09.10 & 55.3 & VF & S3 & 0.094 &  1.41\\
ABELL 0068 & \dataset [ADS/Sa.CXO\#obs/03250] {3250} & 00:37:06.309 & +09:09:32.28 & 10.0 & VF & I3 & 0.255 & 12.70\\
ABELL 0119 $\dagger$ & \dataset [ADS/Sa.CXO\#obs/04180] {4180} & 00:56:15.150 & -01:14:59.70 & 11.9 & VF & I3 & 0.044 &  1.39\\
ABELL 0168 & \dataset [ADS/Sa.CXO\#obs/03203] {3203} & 01:14:57.909 & +00:24:42.55 & 40.6 & VF & I3 & 0.045 &  0.23\\
ABELL 0168 & \dataset [ADS/Sa.CXO\#obs/03204] {3204} & 01:14:57.925 & +00:24:42.73 & 37.6 & VF & I3 & 0.045 &  0.23\\
ABELL 0209 & \dataset [ADS/Sa.CXO\#obs/03579] {3579} & 01:31:52.565 & -13:36:39.29 & 10.0 & VF & I3 & 0.206 & 10.96\\
ABELL 0209 & \dataset [ADS/Sa.CXO\#obs/00522] {522} & 01:31:52.595 & -13:36:39.25 & 10.0 & VF & I3 & 0.206 & 10.96\\
ABELL 0267 & \dataset [ADS/Sa.CXO\#obs/01448] {1448} & 01:52:29.181 & +00:57:34.43 & 7.9 &  F & I3 & 0.230 &  8.62\\
ABELL 0267 & \dataset [ADS/Sa.CXO\#obs/03580] {3580} & 01:52:29.180 & +00:57:34.23 & 19.9 & VF & I3 & 0.230 &  8.62\\
ABELL 0370 & \dataset [ADS/Sa.CXO\#obs/00515] {515} & 02:39:53.169 & -01:34:36.96 & 88.0 &  F & S3 & 0.375 & 11.95\\
ABELL 0383 & \dataset [ADS/Sa.CXO\#obs/02321] {2321} & 02:48:03.364 & -03:31:44.69 & 19.5 &  F & S3 & 0.187 &  5.32\\
ABELL 0399 & \dataset [ADS/Sa.CXO\#obs/03230] {3230} & 02:57:54.931 & +13:01:58.41 & 48.6 & VF & I0 & 0.072 &  4.37\\
ABELL 0401 & \dataset [ADS/Sa.CXO\#obs/00518] {518} & 02:58:56.896 & +13:34:14.48 & 18.0 &  F & I3 & 0.074 &  8.39\\
ABELL 0478 & \dataset [ADS/Sa.CXO\#obs/06102] {6102} & 04:13:25.347 & +10:27:55.62 & 10.0 & VF & I3 & 0.088 & 16.39\\
ABELL 0514 & \dataset [ADS/Sa.CXO\#obs/03578] {3578} & 04:48:19.229 & -20:30:28.79 & 44.5 & VF & I3 & 0.072 &  0.66\\
ABELL 0520 & \dataset [ADS/Sa.CXO\#obs/04215] {4215} & 04:54:09.711 & +02:55:23.69 & 66.3 & VF & I3 & 0.202 & 12.97\\
ABELL 0521 & \dataset [ADS/Sa.CXO\#obs/00430] {430} & 04:54:07.004 & -10:13:26.72 & 39.1 & VF & S3 & 0.253 &  9.77\\
ABELL 0586 & \dataset [ADS/Sa.CXO\#obs/00530] {530} & 07:32:20.339 & +31:37:58.59 & 10.0 & VF & I3 & 0.171 &  8.54\\
ABELL 0611 & \dataset [ADS/Sa.CXO\#obs/03194] {3194} & 08:00:56.832 & +36:03:24.09 & 36.1 & VF & S3 & 0.288 & 10.78\\
ABELL 0644 $\dagger$ & \dataset [ADS/Sa.CXO\#obs/02211] {2211} & 08:17:25.225 & -07:30:40.03 & 29.7 & VF & I3 & 0.070 &  6.95\\
ABELL 0665 & \dataset [ADS/Sa.CXO\#obs/03586] {3586} & 08:30:59.231 & +65:50:37.78 & 29.7 & VF & I3 & 0.181 & 13.37\\
ABELL 0697 & \dataset [ADS/Sa.CXO\#obs/04217] {4217} & 08:42:57.549 & +36:21:57.65 & 19.5 & VF & I3 & 0.282 & 26.10\\
ABELL 0773 & \dataset [ADS/Sa.CXO\#obs/05006] {5006} & 09:17:52.566 & +51:43:38.18 & 19.8 & VF & I3 & 0.217 & 12.87\\
ABELL 0781 & \dataset [ADS/Sa.CXO\#obs/00534] {534} & 09:20:25.431 & +30:30:07.56 & 9.9 & VF & I3 & 0.298 &  0.00\\
ABELL 0907 & \dataset [ADS/Sa.CXO\#obs/03185] {3185} & 09:58:21.880 & -11:03:52.20 & 48.0 & VF & I3 & 0.153 &  6.19\\
ABELL 0963 & \dataset [ADS/Sa.CXO\#obs/00903] {903} & 10:17:03.744 & +39:02:49.17 & 36.3 &  F & S3 & 0.206 & 10.65\\
ABELL 1063S & \dataset [ADS/Sa.CXO\#obs/04966] {4966} & 22:48:44.294 & -44:31:48.37 & 26.7 & VF & I3 & 0.354 & 71.09\\
ABELL 1068 $\dagger$ & \dataset [ADS/Sa.CXO\#obs/01652] {1652} & 10:40:44.520 & +39:57:10.28 & 26.8 &  F & S3 & 0.138 &  4.19\\
ABELL 1201 $\dagger$ & \dataset [ADS/Sa.CXO\#obs/04216] {4216} & 11:12:54.489 & +13:26:08.76 & 39.7 & VF & S3 & 0.169 &  3.52\\
ABELL 1204 & \dataset [ADS/Sa.CXO\#obs/02205] {2205} & 11:13:20.419 & +17:35:38.45 & 23.6 & VF & I3 & 0.171 &  3.92\\
ABELL 1361 $\dagger$ & \dataset [ADS/Sa.CXO\#obs/02200] {2200} & 11:43:39.827 & +46:21:21.40 & 16.7 &  F & S3 & 0.117 &  2.16\\
ABELL 1423 & \dataset [ADS/Sa.CXO\#obs/00538] {538} & 11:57:17.026 & +33:36:37.44 & 9.8 & VF & I3 & 0.213 &  7.01\\
ABELL 1651 & \dataset [ADS/Sa.CXO\#obs/04185] {4185} & 12:59:22.830 & -04:11:45.86 & 9.6 & VF & I3 & 0.084 &  6.66\\
ABELL 1664 $\dagger$ & \dataset [ADS/Sa.CXO\#obs/01648] {1648} & 13:03:42.478 & -24:14:44.55 & 9.8 & VF & S3 & 0.128 &  2.59\\
ABELL 1682 & \dataset [ADS/Sa.CXO\#obs/03244] {3244} & 13:06:50.764 & +46:33:19.86 & 9.8 & VF & I3 & 0.226 &  0.00\\
ABELL 1689 & \dataset [ADS/Sa.CXO\#obs/01663] {1663} & 13:11:29.612 & -01:20:28.69 & 10.7 &  F & I3 & 0.184 & 24.71\\
ABELL 1689 & \dataset [ADS/Sa.CXO\#obs/05004] {5004} & 13:11:29.606 & -01:20:28.61 & 19.9 & VF & I3 & 0.184 & 24.71\\
ABELL 1689 & \dataset [ADS/Sa.CXO\#obs/00540] {540} & 13:11:29.595 & -01:20:28.47 & 10.3 &  F & I3 & 0.184 & 24.71\\
ABELL 1758 & \dataset [ADS/Sa.CXO\#obs/02213] {2213} & 13:32:42.978 & +50:32:44.83 & 58.3 & VF & S3 & 0.279 & 21.01\\
ABELL 1763 & \dataset [ADS/Sa.CXO\#obs/03591] {3591} & 13:35:17.957 & +40:59:55.80 & 19.6 & VF & I3 & 0.187 &  9.26\\
ABELL 1795 $\dagger$ & \dataset [ADS/Sa.CXO\#obs/05289] {5289} & 13:48:52.829 & +26:35:24.01 & 15.0 & VF & I3 & 0.062 &  7.59\\
ABELL 1835 & \dataset [ADS/Sa.CXO\#obs/00495] {495} & 14:01:01.951 & +02:52:43.18 & 19.5 &  F & S3 & 0.253 & 39.38\\
ABELL 1914 & \dataset [ADS/Sa.CXO\#obs/03593] {3593} & 14:26:01.399 & +37:49:27.83 & 18.9 & VF & I3 & 0.171 & 26.25\\
ABELL 1942 & \dataset [ADS/Sa.CXO\#obs/03290] {3290} & 14:38:21.878 & +03:40:12.97 & 57.6 & VF & I2 & 0.224 &  2.27\\
ABELL 1995 & \dataset [ADS/Sa.CXO\#obs/00906] {906} & 14:52:57.758 & +58:02:51.34 & 0.0 &  F & S3 & 0.319 & 10.19\\
ABELL 2029 $\dagger$ & \dataset [ADS/Sa.CXO\#obs/06101] {6101} & 15:10:56.163 & +05:44:40.89 & 9.9 & VF & I3 & 0.076 & 13.90\\
ABELL 2034 & \dataset [ADS/Sa.CXO\#obs/02204] {2204} & 15:10:11.003 & +33:30:46.46 & 53.9 & VF & I3 & 0.113 &  6.45\\
ABELL 2065 $\dagger$ & \dataset [ADS/Sa.CXO\#obs/031821] {31821} & 15:22:29.220 & +27:42:46.54 & 0.0 & VF & I3 & 0.073 &  2.92\\
ABELL 2069 & \dataset [ADS/Sa.CXO\#obs/04965] {4965} & 15:24:09.181 & +29:53:18.05 & 55.4 & VF & I2 & 0.116 &  3.82\\
ABELL 2111 & \dataset [ADS/Sa.CXO\#obs/00544] {544} & 15:39:41.432 & +34:25:12.26 & 10.3 &  F & I3 & 0.230 &  7.45\\
ABELL 2125 & \dataset [ADS/Sa.CXO\#obs/02207] {2207} & 15:41:14.154 & +66:15:57.20 & 81.5 & VF & I3 & 0.246 &  0.77\\
ABELL 2163 & \dataset [ADS/Sa.CXO\#obs/01653] {1653} & 16:15:45.705 & -06:09:00.62 & 71.1 & VF & I1 & 0.170 & 49.11\\
ABELL 2204 $\dagger$ & \dataset [ADS/Sa.CXO\#obs/0499] {499} & 16:32:45.437 & +05:34:21.05 & 10.1 &  F & S3 & 0.152 & 20.77\\
ABELL 2204 & \dataset [ADS/Sa.CXO\#obs/06104] {6104} & 16:32:45.428 & +05:34:20.89 & 9.6 & VF & I3 & 0.152 & 22.03\\
ABELL 2218 & \dataset [ADS/Sa.CXO\#obs/01666] {1666} & 16:35:50.831 & +66:12:42.31 & 48.6 & VF & I0 & 0.171 &  8.39\\
ABELL 2219 $\dagger$ & \dataset [ADS/Sa.CXO\#obs/0896] {896} & 16:40:21.069 & +46:42:29.07 & 42.3 &  F & S3 & 0.226 & 33.15\\
ABELL 2255 & \dataset [ADS/Sa.CXO\#obs/00894] {894} & 17:12:40.385 & +64:03:50.63 & 39.4 &  F & I3 & 0.081 &  3.67\\
ABELL 2256 $\dagger$ & \dataset [ADS/Sa.CXO\#obs/01386] {1386} & 17:03:44.567 & +78:38:11.51 & 12.4 &  F & I3 & 0.058 &  4.65\\
ABELL 2259 & \dataset [ADS/Sa.CXO\#obs/03245] {3245} & 17:20:08.299 & +27:40:11.53 & 10.0 & VF & I3 & 0.164 &  5.37\\
ABELL 2261 & \dataset [ADS/Sa.CXO\#obs/05007] {5007} & 17:22:27.254 & +32:07:58.60 & 24.3 & VF & I3 & 0.224 & 17.49\\
ABELL 2294 & \dataset [ADS/Sa.CXO\#obs/03246] {3246} & 17:24:10.149 & +85:53:09.77 & 10.0 & VF & I3 & 0.178 & 10.35\\
ABELL 2384 & \dataset [ADS/Sa.CXO\#obs/04202] {4202} & 21:52:21.178 & -19:32:51.90 & 31.5 & VF & I3 & 0.095 &  1.95\\
ABELL 2390 $\dagger$ & \dataset [ADS/Sa.CXO\#obs/04193] {4193} & 21:53:36.825 & +17:41:44.38 & 95.1 & VF & S3 & 0.230 & 31.02\\
ABELL 2409 & \dataset [ADS/Sa.CXO\#obs/03247] {3247} & 22:00:52.567 & +20:58:34.11 & 10.2 & VF & I3 & 0.148 &  7.01\\
ABELL 2537 & \dataset [ADS/Sa.CXO\#obs/04962] {4962} & 23:08:22.313 & -02:11:29.88 & 36.2 & VF & S3 & 0.295 & 10.16\\
ABELL 2550 & \dataset [ADS/Sa.CXO\#obs/02225] {2225} & 23:11:35.806 & -21:44:46.70 & 59.0 & VF & S3 & 0.154 &  0.58\\
ABELL 2554 $\dagger$ & \dataset [ADS/Sa.CXO\#obs/01696] {1696} & 23:12:19.939 & -21:30:09.84 & 19.9 & VF & S3 & 0.110 &  1.57\\
ABELL 2556 $\dagger$ & \dataset [ADS/Sa.CXO\#obs/02226] {2226} & 23:13:01.413 & -21:38:04.47 & 19.9 & VF & S3 & 0.086 &  1.43\\
ABELL 2631 & \dataset [ADS/Sa.CXO\#obs/03248] {3248} & 23:37:38.560 & +00:16:28.64 & 9.2 & VF & I3 & 0.278 & 12.59\\
ABELL 2667 & \dataset [ADS/Sa.CXO\#obs/02214] {2214} & 23:51:39.395 & -26:05:02.75 & 9.6 & VF & S3 & 0.230 & 19.91\\
ABELL 2670 & \dataset [ADS/Sa.CXO\#obs/04959] {4959} & 23:54:13.687 & -10:25:08.85 & 39.6 & VF & I3 & 0.076 &  1.39\\
ABELL 2717 & \dataset [ADS/Sa.CXO\#obs/06974] {6974} & 00:03:11.996 & -35:56:08.01 & 19.8 & VF & I3 & 0.048 &  0.26\\
ABELL 2744 & \dataset [ADS/Sa.CXO\#obs/02212] {2212} & 00:14:14.396 & -30:22:40.04 & 24.8 & VF & S3 & 0.308 & 29.00\\
ABELL 3128 $\dagger$ & \dataset [ADS/Sa.CXO\#obs/00893] {893} & 03:29:50.918 & -52:34:51.04 & 19.6 &  F & I3 & 0.062 &  0.35\\
ABELL 3158 $\dagger$ & \dataset [ADS/Sa.CXO\#obs/03201] {3201} & 03:42:54.675 & -53:37:24.36 & 24.8 & VF & I3 & 0.059 &  3.01\\
ABELL 3158 $\dagger$ & \dataset [ADS/Sa.CXO\#obs/03712] {3712} & 03:42:54.683 & -53:37:24.37 & 30.9 & VF & I3 & 0.059 &  3.01\\
ABELL 3164 & \dataset [ADS/Sa.CXO\#obs/06955] {6955} & 03:46:16.839 & -57:02:11.38 & 13.5 & VF & I3 & 0.057 &  0.19\\
ABELL 3376 & \dataset [ADS/Sa.CXO\#obs/03202] {3202} & 06:02:05.122 & -39:57:42.82 & 44.3 & VF & I3 & 0.046 &  0.75\\
ABELL 3376 & \dataset [ADS/Sa.CXO\#obs/03450] {3450} & 06:02:05.162 & -39:57:42.87 & 19.8 & VF & I3 & 0.046 &  0.75\\
ABELL 3391 $\dagger$ & \dataset [ADS/Sa.CXO\#obs/04943] {4943} & 06:26:21.511 & -53:41:44.81 & 18.4 & VF & I3 & 0.056 &  1.44\\
ABELL 3921 & \dataset [ADS/Sa.CXO\#obs/04973] {4973} & 22:49:57.829 & -64:25:42.17 & 29.4 & VF & I3 & 0.093 &  3.37\\
AC 114 & \dataset [ADS/Sa.CXO\#obs/01562] {1562} & 22:58:48.196 & -34:47:56.89 & 72.5 &  F & S3 & 0.312 & 10.90\\
CL 0024+17 & \dataset [ADS/Sa.CXO\#obs/00929] {929} & 00:26:35.996 & +17:09:45.37 & 39.8 &  F & S3 & 0.394 &  2.88\\
CL 1221+4918 & \dataset [ADS/Sa.CXO\#obs/01662] {1662} & 12:21:26.709 & +49:18:21.60 & 79.1 & VF & I3 & 0.700 &  8.65\\
CL J0030+2618 & \dataset [ADS/Sa.CXO\#obs/05762] {5762} & 00:30:34.339 & +26:18:01.58 & 17.9 & VF & I3 & 0.500 &  3.41\\
CL J0152-1357 & \dataset [ADS/Sa.CXO\#obs/00913] {913} & 01:52:42.141 & -13:57:59.71 & 36.5 &  F & I3 & 0.831 & 13.30\\
CL J0542.8-4100 & \dataset [ADS/Sa.CXO\#obs/00914] {914} & 05:42:49.994 & -40:59:58.50 & 50.4 &  F & I3 & 0.630 &  6.18\\
CL J0848+4456 & \dataset [ADS/Sa.CXO\#obs/01708] {1708} & 08:48:48.235 & +44:56:17.11 & 61.4 & VF & I1 & 0.574 &  0.62\\
CL J0848+4456 & \dataset [ADS/Sa.CXO\#obs/00927] {927} & 08:48:48.252 & +44:56:17.13 & 125.1 & VF & I1 & 0.574 &  0.62\\
CL J1113.1-2615 & \dataset [ADS/Sa.CXO\#obs/00915] {915} & 11:13:05.167 & -26:15:40.43 & 104.6 &  F & I3 & 0.730 &  2.22\\
CL J1213+0253 & \dataset [ADS/Sa.CXO\#obs/04934] {4934} & 12:13:34.948 & +02:53:45.45 & 18.9 & VF & I3 & 0.409 &  0.00\\
CL J1226.9+3332 & \dataset [ADS/Sa.CXO\#obs/03180] {3180} & 12:26:58.373 & +33:32:47.36 & 31.7 & VF & I3 & 0.890 & 30.76\\
CL J1226.9+3332 & \dataset [ADS/Sa.CXO\#obs/05014] {5014} & 12:26:58.372 & +33:32:47.18 & 32.7 & VF & I3 & 0.890 & 30.76\\
CL J1641+4001 & \dataset [ADS/Sa.CXO\#obs/03575] {3575} & 16:41:53.704 & +40:01:44.40 & 46.5 & VF & I3 & 0.464 &  0.00\\
CL J2302.8+0844 & \dataset [ADS/Sa.CXO\#obs/00918] {918} & 23:02:48.156 & +08:43:52.74 & 108.6 &  F & I3 & 0.730 &  2.93\\
DLS J0514-4904 & \dataset [ADS/Sa.CXO\#obs/04980] {4980} & 05:14:40.037 & -49:03:15.07 & 19.9 & VF & I3 & 0.091 &  0.68\\
EXO 0422-086 $\dagger$ & \dataset [ADS/Sa.CXO\#obs/04183] {4183} & 04:25:51.271 & -08:33:36.42 & 10.0 & VF & I3 & 0.040 &  0.65\\
HERCULES A $\dagger$ & \dataset [ADS/Sa.CXO\#obs/01625] {1625} & 16:51:08.161 & +04:59:32.44 & 14.8 & VF & S3 & 0.154 &  3.27\\
IRAS 09104+4109 & \dataset [ADS/Sa.CXO\#obs/00509] {509} & 09:13:45.481 & +40:56:27.49 & 9.1 &  F & S3 & 0.442 &  0.00\\
LYNX E & \dataset [ADS/Sa.CXO\#obs/017081] {17081} & 08:48:58.851 & +44:51:51.44 & 61.4 & VF & I2 & 1.260 &  0.00\\
LYNX E & \dataset [ADS/Sa.CXO\#obs/09271] {9271} & 08:48:58.858 & +44:51:51.46 & 125.1 & VF & I2 & 1.260 &  0.00\\
MACS J0011.7-1523 & \dataset [ADS/Sa.CXO\#obs/03261] {3261} & 00:11:42.965 & -15:23:20.79 & 21.6 & VF & I3 & 0.360 & 10.75\\
MACS J0011.7-1523 & \dataset [ADS/Sa.CXO\#obs/06105] {6105} & 00:11:42.957 & -15:23:20.76 & 37.3 & VF & I3 & 0.360 & 10.75\\
MACS J0025.4-1222 & \dataset [ADS/Sa.CXO\#obs/03251] {3251} & 00:25:29.368 & -12:22:38.05 & 19.3 & VF & I3 & 0.584 & 13.00\\
MACS J0025.4-1222 & \dataset [ADS/Sa.CXO\#obs/05010] {5010} & 00:25:29.399 & -12:22:38.10 & 24.8 & VF & I3 & 0.584 & 13.00\\
MACS J0035.4-2015 & \dataset [ADS/Sa.CXO\#obs/03262] {3262} & 00:35:26.573 & -20:15:46.06 & 21.4 & VF & I3 & 0.364 & 19.79\\
MACS J0111.5+0855 & \dataset [ADS/Sa.CXO\#obs/03256] {3256} & 01:11:31.515 & +08:55:39.21 & 19.4 & VF & I3 & 0.263 &  0.64\\
MACS J0152.5-2852 & \dataset [ADS/Sa.CXO\#obs/03264] {3264} & 01:52:34.479 & -28:53:38.01 & 17.5 & VF & I3 & 0.341 &  6.33\\
MACS J0159.0-3412 & \dataset [ADS/Sa.CXO\#obs/05818] {5818} & 01:59:00.366 & -34:13:00.23 & 9.4 & VF & I3 & 0.458 & 18.92\\
MACS J0159.8-0849 & \dataset [ADS/Sa.CXO\#obs/03265] {3265} & 01:59:49.453 & -08:50:00.90 & 17.9 & VF & I3 & 0.405 & 26.31\\
MACS J0159.8-0849 & \dataset [ADS/Sa.CXO\#obs/06106] {6106} & 01:59:49.422 & -08:50:00.42 & 35.3 & VF & I3 & 0.405 & 26.31\\
MACS J0242.5-2132 & \dataset [ADS/Sa.CXO\#obs/03266] {3266} & 02:42:35.906 & -21:32:26.30 & 11.9 & VF & I3 & 0.314 & 12.74\\
MACS J0257.1-2325 & \dataset [ADS/Sa.CXO\#obs/01654] {1654} & 02:57:09.130 & -23:26:06.25 & 19.8 &  F & I3 & 0.505 & 21.72\\
MACS J0257.1-2325 & \dataset [ADS/Sa.CXO\#obs/03581] {3581} & 02:57:09.152 & -23:26:06.21 & 18.5 & VF & I3 & 0.505 & 21.72\\
MACS J0257.6-2209 & \dataset [ADS/Sa.CXO\#obs/03267] {3267} & 02:57:41.024 & -22:09:11.12 & 20.5 & VF & I3 & 0.322 & 10.77\\
MACS J0308.9+2645 & \dataset [ADS/Sa.CXO\#obs/03268] {3268} & 03:08:55.927 & +26:45:38.34 & 24.4 & VF & I3 & 0.324 & 20.42\\
MACS J0329.6-0211 & \dataset [ADS/Sa.CXO\#obs/03257] {3257} & 03:29:41.681 & -02:11:47.67 & 9.9 & VF & I3 & 0.450 & 12.82\\
MACS J0329.6-0211 & \dataset [ADS/Sa.CXO\#obs/03582] {3582} & 03:29:41.688 & -02:11:47.81 & 19.9 & VF & I3 & 0.450 & 12.82\\
MACS J0329.6-0211 & \dataset [ADS/Sa.CXO\#obs/06108] {6108} & 03:29:41.681 & -02:11:47.57 & 39.6 & VF & I3 & 0.450 & 12.82\\
MACS J0404.6+1109 & \dataset [ADS/Sa.CXO\#obs/03269] {3269} & 04:04:32.491 & +11:08:02.10 & 21.8 & VF & I3 & 0.355 &  3.90\\
MACS J0417.5-1154 & \dataset [ADS/Sa.CXO\#obs/03270] {3270} & 04:17:34.686 & -11:54:32.71 & 12.0 & VF & I3 & 0.440 & 37.99\\
MACS J0429.6-0253 & \dataset [ADS/Sa.CXO\#obs/03271] {3271} & 04:29:36.088 & -02:53:09.02 & 23.2 & VF & I3 & 0.399 & 11.58\\
MACS J0451.9+0006 & \dataset [ADS/Sa.CXO\#obs/05815] {5815} & 04:51:54.291 & +00:06:20.20 & 10.2 & VF & I3 & 0.430 &  8.20\\
MACS J0455.2+0657 & \dataset [ADS/Sa.CXO\#obs/05812] {5812} & 04:55:17.426 & +06:57:47.15 & 9.9 & VF & I3 & 0.425 &  9.77\\
MACS J0520.7-1328 & \dataset [ADS/Sa.CXO\#obs/03272] {3272} & 05:20:42.052 & -13:28:49.38 & 19.2 & VF & I3 & 0.340 &  9.63\\
MACS J0547.0-3904 & \dataset [ADS/Sa.CXO\#obs/03273] {3273} & 05:47:01.582 & -39:04:28.24 & 21.7 & VF & I3 & 0.210 &  1.59\\
MACS J0553.4-3342 & \dataset [ADS/Sa.CXO\#obs/05813] {5813} & 05:53:27.200 & -33:42:53.02 & 9.9 & VF & I3 & 0.407 & 32.68\\
MACS J0717.5+3745 & \dataset [ADS/Sa.CXO\#obs/01655] {1655} & 07:17:31.654 & +37:45:18.52 & 19.9 &  F & I3 & 0.548 & 46.58\\
MACS J0717.5+3745 & \dataset [ADS/Sa.CXO\#obs/04200] {4200} & 07:17:31.651 & +37:45:18.46 & 59.2 & VF & I3 & 0.548 & 46.58\\
MACS J0744.8+3927 & \dataset [ADS/Sa.CXO\#obs/03197] {3197} & 07:44:52.802 & +39:27:24.41 & 20.2 & VF & I3 & 0.686 & 24.67\\
MACS J0744.8+3927 & \dataset [ADS/Sa.CXO\#obs/03585] {3585} & 07:44:52.779 & +39:27:24.41 & 19.9 & VF & I3 & 0.686 & 24.67\\
MACS J0744.8+3927 & \dataset [ADS/Sa.CXO\#obs/06111] {6111} & 07:44:52.800 & +39:27:24.41 & 49.5 & VF & I3 & 0.686 & 24.67\\
MACS J0911.2+1746 & \dataset [ADS/Sa.CXO\#obs/03587] {3587} & 09:11:11.325 & +17:46:31.06 & 17.9 & VF & I3 & 0.541 & 10.52\\
MACS J0911.2+1746 & \dataset [ADS/Sa.CXO\#obs/05012] {5012} & 09:11:11.309 & +17:46:30.92 & 23.8 & VF & I3 & 0.541 & 10.52\\
MACS J0949+1708   & \dataset [ADS/Sa.CXO\#obs/03274] {3274} & 09:49:51.824 & +17:07:05.62 & 14.3 & VF & I3 & 0.382 & 19.19\\
MACS J1006.9+3200 & \dataset [ADS/Sa.CXO\#obs/05819] {5819} & 10:06:54.668 & +32:01:34.61 & 10.9 & VF & I3 & 0.359 &  6.06\\
MACS J1105.7-1014 & \dataset [ADS/Sa.CXO\#obs/05817] {5817} & 11:05:46.462 & -10:14:37.20 & 10.3 & VF & I3 & 0.466 & 11.29\\
MACS J1108.8+0906 & \dataset [ADS/Sa.CXO\#obs/03252] {3252} & 11:08:55.393 & +09:05:51.16 & 9.9 & VF & I3 & 0.449 &  8.96\\
MACS J1108.8+0906 & \dataset [ADS/Sa.CXO\#obs/05009] {5009} & 11:08:55.402 & +09:05:51.14 & 24.5 & VF & I3 & 0.449 &  8.96\\
MACS J1115.2+5320 & \dataset [ADS/Sa.CXO\#obs/03253] {3253} & 11:15:15.632 & +53:20:03.71 & 8.8 & VF & I3 & 0.439 & 14.29\\
MACS J1115.2+5320 & \dataset [ADS/Sa.CXO\#obs/05008] {5008} & 11:15:15.646 & +53:20:03.74 & 18.0 & VF & I3 & 0.439 & 14.29\\
MACS J1115.2+5320 & \dataset [ADS/Sa.CXO\#obs/05350] {5350} & 11:15:15.632 & +53:20:03.37 & 6.9 & VF & I3 & 0.439 & 14.29\\
MACS J1115.8+0129 & \dataset [ADS/Sa.CXO\#obs/03275] {3275} & 11:15:52.048 & +01:29:56.56 & 15.9 & VF & I3 & 0.120 &  1.47\\
MACS J1131.8-1955 & \dataset [ADS/Sa.CXO\#obs/03276] {3276} & 11:31:56.011 & -19:55:55.85 & 13.9 & VF & I3 & 0.307 & 17.45\\
MACS J1149.5+2223 & \dataset [ADS/Sa.CXO\#obs/01656] {1656} & 11:49:35.856 & +22:23:55.02 & 18.5 & VF & I3 & 0.544 & 21.60\\
MACS J1149.5+2223 & \dataset [ADS/Sa.CXO\#obs/03589] {3589} & 11:49:35.848 & +22:23:55.05 & 20.0 & VF & I3 & 0.544 & 21.60\\
MACS J1206.2-0847 & \dataset [ADS/Sa.CXO\#obs/03277] {3277} & 12:06:12.276 & -08:48:02.40 & 23.5 & VF & I3 & 0.440 & 37.02\\
MACS J1226.8+2153 & \dataset [ADS/Sa.CXO\#obs/03590] {3590} & 12:26:51.207 & +21:49:55.22 & 19.0 & VF & I3 & 0.370 &  2.63\\
MACS J1311.0-0310 & \dataset [ADS/Sa.CXO\#obs/03258] {3258} & 13:11:01.665 & -03:10:39.50 & 14.9 & VF & I3 & 0.494 & 10.03\\
MACS J1311.0-0310 & \dataset [ADS/Sa.CXO\#obs/06110] {6110} & 13:11:01.680 & -03:10:39.75 & 63.2 & VF & I3 & 0.494 & 10.03\\
MACS J1319+7003   & \dataset [ADS/Sa.CXO\#obs/03278] {3278} & 13:20:08.370 & +70:04:33.81 & 21.6 & VF & I3 & 0.328 &  7.03\\
MACS J1427.2+4407 & \dataset [ADS/Sa.CXO\#obs/06112] {6112} & 14:27:16.175 & +44:07:30.33 & 9.4 & VF & I3 & 0.477 & 14.18\\
MACS J1427.6-2521 & \dataset [ADS/Sa.CXO\#obs/03279] {3279} & 14:27:39.389 & -25:21:04.66 & 16.9 & VF & I3 & 0.220 &  1.55\\
MACS J1621.3+3810 & \dataset [ADS/Sa.CXO\#obs/03254] {3254} & 16:21:25.552 & +38:09:43.56 & 9.8 & VF & I3 & 0.461 & 11.49\\
MACS J1621.3+3810 & \dataset [ADS/Sa.CXO\#obs/03594] {3594} & 16:21:25.558 & +38:09:43.44 & 19.7 & VF & I3 & 0.461 & 11.49\\
MACS J1621.3+3810 & \dataset [ADS/Sa.CXO\#obs/06109] {6109} & 16:21:25.535 & +38:09:43.34 & 37.5 & VF & I3 & 0.461 & 11.49\\
MACS J1621.3+3810 & \dataset [ADS/Sa.CXO\#obs/06172] {6172} & 16:21:25.559 & +38:09:43.63 & 29.8 & VF & I3 & 0.461 & 11.49\\
MACS J1731.6+2252 & \dataset [ADS/Sa.CXO\#obs/03281] {3281} & 17:31:39.902 & +22:52:00.55 & 20.5 & VF & I3 & 0.366 &  9.32\\
MACS J1824.3+4309 & \dataset [ADS/Sa.CXO\#obs/03255] {3255} & 18:24:18.444 & +43:09:43.39 & 14.9 & VF & I3 & 0.487 &  0.00\\
MACS J1931.8-2634 & \dataset [ADS/Sa.CXO\#obs/03282] {3282} & 19:31:49.656 & -26:34:33.99 & 13.6 & VF & I3 & 0.352 & 23.14\\
MACS J2046.0-3430 & \dataset [ADS/Sa.CXO\#obs/05816] {5816} & 20:46:00.522 & -34:30:15.50 & 10.0 & VF & I3 & 0.413 &  5.79\\
MACS J2049.9-3217 & \dataset [ADS/Sa.CXO\#obs/03283] {3283} & 20:49:56.245 & -32:16:52.30 & 23.8 & VF & I3 & 0.325 &  8.71\\
MACS J2211.7-0349 & \dataset [ADS/Sa.CXO\#obs/03284] {3284} & 22:11:45.856 & -03:49:37.24 & 17.7 & VF & I3 & 0.270 & 22.11\\
MACS J2214.9-1359 & \dataset [ADS/Sa.CXO\#obs/03259] {3259} & 22:14:57.467 & -14:00:09.35 & 19.5 & VF & I3 & 0.503 & 24.05\\
MACS J2214.9-1359 & \dataset [ADS/Sa.CXO\#obs/05011] {5011} & 22:14:57.481 & -14:00:09.39 & 18.5 & VF & I3 & 0.503 & 24.05\\
MACS J2228+2036   & \dataset [ADS/Sa.CXO\#obs/03285] {3285} & 22:28:33.241 & +20:37:11.42 & 19.9 & VF & I3 & 0.412 & 17.92\\
MACS J2229.7-2755 & \dataset [ADS/Sa.CXO\#obs/03286] {3286} & 22:29:45.358 & -27:55:38.41 & 16.4 & VF & I3 & 0.324 &  9.49\\
MACS J2243.3-0935 & \dataset [ADS/Sa.CXO\#obs/03260] {3260} & 22:43:21.537 & -09:35:44.30 & 20.5 & VF & I3 & 0.101 &  0.78\\
MACS J2245.0+2637 & \dataset [ADS/Sa.CXO\#obs/03287] {3287} & 22:45:04.547 & +26:38:07.88 & 16.9 & VF & I3 & 0.304 &  9.36\\
MACS J2311+0338   & \dataset [ADS/Sa.CXO\#obs/03288] {3288} & 23:11:33.213 & +03:38:06.51 & 13.6 & VF & I3 & 0.300 & 10.98\\
MKW3S & \dataset [ADS/Sa.CXO\#obs/0900] {900} & 15:21:51.930 & +07:42:31.97 & 57.3 & VF & I3 & 0.045 &  1.14\\
MS 0016.9+1609 & \dataset [ADS/Sa.CXO\#obs/00520] {520} & 00:18:33.503 & +16:26:12.99 & 67.4 & VF & I3 & 0.541 & 32.94\\
MS 0302.7+1658 & \dataset [ADS/Sa.CXO\#obs/00525] {525} & 03:05:31.614 & +17:10:02.06 & 10.0 & VF & I3 & 0.424 &  0.00\\
MS 0440.5+0204 $\dagger$ & \dataset [ADS/Sa.CXO\#obs/04196] {4196} & 04:43:09.952 & +02:10:18.70 & 59.4 & VF & S3 & 0.190 &  2.17\\
MS 0451.6-0305 & \dataset [ADS/Sa.CXO\#obs/00902] {902} & 04:54:11.004 & -03:00:52.19 & 44.2 &  F & S3 & 0.539 & 33.32\\
MS 0735.6+7421 & \dataset [ADS/Sa.CXO\#obs/04197] {4197} & 07:41:44.245 & +74:14:38.23 & 45.5 & VF & S3 & 0.216 &  7.57\\
MS 0839.8+2938 & \dataset [ADS/Sa.CXO\#obs/02224] {2224} & 08:42:55.969 & +29:27:26.97 & 29.8 &  F & S3 & 0.194 &  3.10\\
MS 0906.5+1110 & \dataset [ADS/Sa.CXO\#obs/00924] {924} & 09:09:12.753 & +10:58:32.00 & 29.7 & VF & I3 & 0.163 &  4.64\\
MS 1006.0+1202 & \dataset [ADS/Sa.CXO\#obs/00925] {925} & 10:08:47.194 & +11:47:55.99 & 29.4 & VF & I3 & 0.221 &  4.75\\
MS 1008.1-1224 & \dataset [ADS/Sa.CXO\#obs/00926] {926} & 10:10:32.312 & -12:39:56.80 & 44.2 & VF & I3 & 0.301 &  6.44\\
MS 1054.5-0321 & \dataset [ADS/Sa.CXO\#obs/00512] {512} & 10:56:58.499 & -03:37:32.76 & 89.1 &  F & S3 & 0.830 & 27.22\\
MS 1455.0+2232 & \dataset [ADS/Sa.CXO\#obs/04192] {4192} & 14:57:15.088 & +22:20:32.49 & 91.9 & VF & I3 & 0.259 & 10.25\\
MS 1621.5+2640 & \dataset [ADS/Sa.CXO\#obs/00546] {546} & 16:23:35.522 & +26:34:25.67 & 30.1 &  F & I3 & 0.426 &  6.49\\
MS 2053.7-0449 & \dataset [ADS/Sa.CXO\#obs/01667] {1667} & 20:56:21.295 & -04:37:46.81 & 44.5 & VF & I3 & 0.583 &  2.96\\
MS 2053.7-0449 & \dataset [ADS/Sa.CXO\#obs/00551] {551} & 20:56:21.297 & -04:37:46.80 & 44.3 &  F & I3 & 0.583 &  2.96\\
MS 2137.3-2353 & \dataset [ADS/Sa.CXO\#obs/04974] {4974} & 21:40:15.178 & -23:39:40.71 & 57.4 & VF & S3 & 0.313 & 11.28\\
MS J1157.3+5531 $\dagger$ & \dataset [ADS/Sa.CXO\#obs/04964] {4964} & 11:59:52.295 & +55:32:05.61 & 75.1 & VF & S3 & 0.081 &  0.12\\
NGC 6338 $\dagger$ & \dataset [ADS/Sa.CXO\#obs/04194] {4194} & 17:15:23.036 & +57:24:40.29 & 47.3 & VF & I3 & 0.028 &  0.13\\
PKS 0745-191 & \dataset [ADS/Sa.CXO\#obs/06103] {6103} & 07:47:31.469 & -19:17:40.01 & 10.3 & VF & I3 & 0.103 & 18.41\\
RBS 0797 & \dataset [ADS/Sa.CXO\#obs/02202] {2202} & 09:47:12.971 & +76:23:13.90 & 11.7 & VF & I3 & 0.354 & 26.07\\
RDCS 1252-29    & \dataset [ADS/Sa.CXO\#obs/04198] {4198} & 12:52:54.221 & -29:27:21.01 & 163.4 & VF & I3 & 1.237 &  2.28\\
RX J0232.2-4420 & \dataset [ADS/Sa.CXO\#obs/04993] {4993} & 02:32:18.771 & -44:20:46.68 & 23.4 & VF & I3 & 0.284 & 18.17\\
RX J0340-4542   & \dataset [ADS/Sa.CXO\#obs/06954] {6954} & 03:40:44.765 & -45:41:18.41 & 17.9 & VF & I3 & 0.082 &  0.33\\
RX J0439+0520   & \dataset [ADS/Sa.CXO\#obs/00527] {527} & 04:39:02.218 & +05:20:43.11 & 9.6 & VF & I3 & 0.208 &  3.57\\
RX J0439.0+0715 & \dataset [ADS/Sa.CXO\#obs/01449] {1449} & 04:39:00.710 & +07:16:07.65 & 6.3 &  F & I3 & 0.230 &  9.44\\
RX J0439.0+0715 & \dataset [ADS/Sa.CXO\#obs/03583] {3583} & 04:39:00.710 & +07:16:07.63 & 19.2 & VF & I3 & 0.230 &  9.44\\
RX J0528.9-3927 & \dataset [ADS/Sa.CXO\#obs/04994] {4994} & 05:28:53.039 & -39:28:15.53 & 22.5 & VF & I3 & 0.263 & 12.99\\
RX J0647.7+7015 & \dataset [ADS/Sa.CXO\#obs/03196] {3196} & 06:47:50.029 & +70:14:49.66 & 19.3 & VF & I3 & 0.584 & 26.48\\
RX J0647.7+7015 & \dataset [ADS/Sa.CXO\#obs/03584] {3584} & 06:47:50.014 & +70:14:49.69 & 20.0 & VF & I3 & 0.584 & 26.48\\
RX J0819.6+6336 $\dagger$ & \dataset [ADS/Sa.CXO\#obs/02199] {2199} & 08:19:26.007 & +63:37:26.53 & 14.9 &  F & S3 & 0.119 &  0.98\\
RX J0910+5422   & \dataset [ADS/Sa.CXO\#obs/02452] {2452} & 09:10:44.478 & +54:22:03.77 & 65.3 & VF & I3 & 1.100 &  1.33\\
RX J1053+5735   & \dataset [ADS/Sa.CXO\#obs/04936] {4936} & 10:53:39.844 & +57:35:18.42 & 92.2 &  F & S3 & 1.140 &  0.00\\
RX J1347.5-1145 & \dataset [ADS/Sa.CXO\#obs/03592] {3592} & 13:47:30.593 & -11:45:10.05 & 57.7 & VF & I3 & 0.451 & 100.36\\
RX J1347.5-1145 & \dataset [ADS/Sa.CXO\#obs/00507] {507} & 13:47:30.598 & -11:45:10.27 & 10.0 &  F & S3 & 0.451 & 100.36\\
RX J1350+6007   & \dataset [ADS/Sa.CXO\#obs/02229] {2229} & 13:50:48.038 & +60:07:08.39 & 58.3 & VF & I3 & 0.804 &  2.19\\
RX J1423.8+2404 & \dataset [ADS/Sa.CXO\#obs/01657] {1657} & 14:23:47.759 & +24:04:40.45 & 18.5 & VF & I3 & 0.545 & 15.84\\
RX J1423.8+2404 & \dataset [ADS/Sa.CXO\#obs/04195] {4195} & 14:23:47.763 & +24:04:40.63 & 115.6 & VF & S3 & 0.545 & 15.84\\
RX J1504.1-0248 & \dataset [ADS/Sa.CXO\#obs/05793] {5793} & 15:04:07.415 & -02:48:15.70 & 39.2 & VF & I3 & 0.215 & 34.64\\
RX J1525+0958   & \dataset [ADS/Sa.CXO\#obs/01664] {1664} & 15:24:39.729 & +09:57:44.42 & 50.9 & VF & I3 & 0.516 &  3.29\\
RX J1532.9+3021 & \dataset [ADS/Sa.CXO\#obs/01649] {1649} & 15:32:55.642 & +30:18:57.69 & 9.4 & VF & S3 & 0.345 & 20.77\\
RX J1532.9+3021 & \dataset [ADS/Sa.CXO\#obs/01665] {1665} & 15:32:55.641 & +30:18:57.31 & 10.0 & VF & I3 & 0.345 & 20.77\\
RX J1716.9+6708 & \dataset [ADS/Sa.CXO\#obs/00548] {548} & 17:16:49.015 & +67:08:25.80 & 51.7 &  F & I3 & 0.810 &  8.04\\
RX J1720.1+2638 & \dataset [ADS/Sa.CXO\#obs/04361] {4361} & 17:20:09.941 & +26:37:29.11 & 25.7 & VF & I3 & 0.164 & 11.39\\
RX J1720.2+3536 & \dataset [ADS/Sa.CXO\#obs/03280] {3280} & 17:20:16.953 & +35:36:23.63 & 20.8 & VF & I3 & 0.391 & 13.02\\
RX J1720.2+3536 & \dataset [ADS/Sa.CXO\#obs/06107] {6107} & 17:20:16.949 & +35:36:23.68 & 33.9 & VF & I3 & 0.391 & 13.02\\
RX J1720.2+3536 & \dataset [ADS/Sa.CXO\#obs/07225] {7225} & 17:20:16.947 & +35:36:23.69 & 2.0 & VF & I3 & 0.391 & 13.02\\
RX J2011.3-5725 & \dataset [ADS/Sa.CXO\#obs/04995] {4995} & 20:11:26.889 & -57:25:09.08 & 24.0 & VF & I3 & 0.279 &  2.77\\
RX J2129.6+0005 & \dataset [ADS/Sa.CXO\#obs/00552] {552} & 21:29:39.944 & +00:05:18.83 & 10.0 & VF & I3 & 0.235 & 12.56\\
S0463 & \dataset [ADS/Sa.CXO\#obs/06956] {6956} & 04:29:07.040 & -53:49:38.02 & 29.3 & VF & I3 & 0.099 & 22.19\\
S0463 & \dataset [ADS/Sa.CXO\#obs/07250] {7250} & 04:29:07.063 & -53:49:38.11 & 29.1 & VF & I3 & 0.099 & 22.19\\
TRIANG AUSTR $\dagger$ & \dataset [ADS/Sa.CXO\#obs/01281] {1281} & 16:38:22.712 & -64:21:19.70 & 11.4 &  F & I3 & 0.051 &  9.41\\
V 1121.0+2327 & \dataset [ADS/Sa.CXO\#obs/01660] {1660} & 11:20:57.195 & +23:26:27.60 & 71.3 & VF & I3 & 0.560 &  3.28\\
ZWCL 1215 & \dataset [ADS/Sa.CXO\#obs/04184] {4184} & 12:17:40.787 & +03:39:39.42 & 12.1 & VF & I3 & 0.075 &  3.49\\
ZWCL 1358+6245 & \dataset [ADS/Sa.CXO\#obs/00516] {516} & 13:59:50.526 & +62:31:04.57 & 54.1 &  F & S3 & 0.328 & 12.42\\
ZWCL 1953 & \dataset [ADS/Sa.CXO\#obs/01659] {1659} & 08:50:06.677 & +36:04:16.16 & 24.9 &  F & I3 & 0.380 & 17.11\\
ZWCL 3146 & \dataset [ADS/Sa.CXO\#obs/00909] {909} & 10:23:39.735 & +04:11:08.05 & 46.0 &  F & I3 & 0.290 & 29.59\\
ZWCL 5247 & \dataset [ADS/Sa.CXO\#obs/00539] {539} & 12:34:21.928 & +09:47:02.83 & 9.3 & VF & I3 & 0.229 &  4.87\\
ZWCL 7160 & \dataset [ADS/Sa.CXO\#obs/00543] {543} & 14:57:15.158 & +22:20:33.85 & 9.9 &  F & I3 & 0.258 & 10.14\\
ZWICKY 2701 & \dataset [ADS/Sa.CXO\#obs/03195] {3195} & 09:52:49.183 & +51:53:05.27 & 26.9 & VF & S3 & 0.210 &  5.19\\
ZwCL 1332.8+5043 & \dataset [ADS/Sa.CXO\#obs/05772] {5772} & 13:34:20.698 & +50:31:04.64 & 19.5 & VF & I3 & 0.620 &  4.46\\
ZwCl 0848.5+3341 & \dataset [ADS/Sa.CXO\#obs/04205] {4205} & 08:51:38.873 & +33:31:08.00 & 11.4 & VF & S3 & 0.371 &  4.58
\enddata
\tablecomments{(1) Cluster name, (2) CDA observation identification number, (3) R.A. of cluster center, (4) Dec. of cluster center, (5) nominal exposure time, (6) observing mode, (7) CCD location of centroid, (8) redshift, (9) NRAO absorbing Galactic neutral hydrogen column density, (10) bolometric luminosity. $\dagger$ indicates clusters analyzed within R$_{5000}$ only.}
\end{deluxetable}

\clearpage
\begin{figure}[htp]
  \begin{center}
    \begin{minipage}[htp]{0.9\linewidth}
      \includegraphics*[width=\textwidth, trim=15mm 10mm 10mm 10mm, clip]{beta.eps}
      \caption{Surface brightness profiles for clusters requiring a
        $\beta$-model fit for deprojection (discussed in
        \S\ref{sec:beta}). The best-fit $\beta$-model for each cluster
        is overplotted as a dashed line. The discrepancy between the
        data and best-fit model for some clusters results from the
        presence of a compact X-ray source at the center of the
        cluster. These cases are discussed in Appendix
        \ref{app:beta}.}
      \label{fig:betamods}
    \end{minipage}
  \end{center}
\end{figure}
\clearpage
\begin{figure}[htp]
  \begin{center}
    \begin{minipage}[htp]{0.9\linewidth}
      \includegraphics*[width=\textwidth, trim=5mm 0mm 5mm 5mm, clip]{itplflat_rat.eps}
      \caption{Ratio of best-fit \kna\ for the two treatments of
        central temperature interpolation (see \S\ref{sec:temppr}):
        (1) temperature is free to decline across the central density
        bins ($\Delta T_{center} \ne 0$), and (2) the temperature
        across the central density bins is isothermal ($\Delta
        T_{center} = 0$). Filled black squares are clusters for which
        the \kna\ ratio is inconsistent with unity.}
      \label{fig:kcomp}
    \end{minipage}
  \end{center}
\end{figure}
\clearpage
\begin{figure}[htp]
  \begin{center}
    \begin{minipage}[htp]{0.9\linewidth}
      \includegraphics*[width=\textwidth, trim=5mm 0mm 5mm 5mm, clip]{k0res.eps}
      \caption{Best-fit \kna\ vs. redshift. Some clusters have
        \kna\ error bars smaller than the point. The clusters with
        upper-limits ({\it{black points with downward arrows}}) are:
        A2151, AS0405, MS 0116.3-0115, and RX J1347.5-1145. The
        numerically labeled clusters are: (1) M87, (2) Centaurus
        Cluster, (3) RBS 533, (4) HCG 42, (5) HCG 62, (6) SS2B153, (7)
        A1991, (8) MACS0744.8+3927, and (9) CL J1226.9+3332. For
        CLJ1226, \cite{2007ApJ...659.1125M} found best-fit $\kna = 132
        \pm 24 \ent$ which is not significantly different from our
        value of $\kna = 166 \pm 45 \ent$. The lack of $\kna < 10
        \ent$ clusters at $z > 0.1$ is most likely the result of
        insufficient angular resolution (see \S\ref{sec:angres}).}
      \label{fig:k0res}
    \end{minipage}
  \end{center}
\end{figure}
\clearpage
\begin{center}
  \begin{figure}[htp]
    \begin{minipage}[htp]{0.5\linewidth}
      \includegraphics*[width=\textwidth, trim=28mm 7mm 30mm 17mm, clip]{curvk0.eps}
    \end{minipage}
    \begin{minipage}[htp]{0.5\linewidth}
      \includegraphics*[width=\textwidth, trim=28mm 7mm 30mm 17mm, clip]{nbins_k0.eps}
    \end{minipage}
    \begin{minipage}[htp]{0.5\linewidth}
      \includegraphics*[width=\textwidth, trim=28mm 7mm 30mm 17mm, clip]{texpk0.eps}
    \end{minipage}
    \begin{minipage}[htp]{0.5\linewidth}
      \includegraphics*[width=\textwidth, trim=28mm 7mm 30mm 17mm, clip]{ntxbins_k0.eps}
    \end{minipage}
    \caption{Plots of possible systematics versus best-fit \kna.
      {\it{Top left:}} Best-fit \kna\ plotted versus average curvature
      of the corresponding entropy profile (see eq. \ref{eqn:avgcurv})
      There is no trend between these two quantities suggesting that
      \kna\ is not heavily influenced by the total shape of the
      entropy profile. {\it{Top right:}} Best-fit \kna\ plotted versus
      number of bins in the entropy profile which were used during
      fitting. Again, no trend is found. {\it{Bottom left:}} Best-fit
      \kna\ plotted versus the total used exposure time for each
      cluster. No trend is found. {\it{Bottom right:}} Best-fit
      \kna\ plotted versus the number of bins in the temperature
      profile for each cluster. As expected, fewer $\Tx(r)$ does not
      correlate with \kna.}
    \label{fig:sys}
  \end{figure}
\end{center}
\clearpage
\begin{center}
  \begin{figure}[htp]
    \begin{minipage}[htp]{0.5\linewidth}
      \includegraphics*[width=\textwidth, trim=28mm 7mm 30mm 17mm, clip]{splots_allt.eps}
    \end{minipage}
    \begin{minipage}[htp]{0.5\linewidth}
      \includegraphics*[width=\textwidth, trim=28mm 7mm 30mm 17mm, clip]{splots_tle4.eps}
    \end{minipage}
    \begin{minipage}[htp]{0.5\linewidth}
      \includegraphics*[width=\textwidth, trim=28mm 7mm 30mm 17mm, clip]{splots_gt4tle8.eps}
    \end{minipage}
    \begin{minipage}[htp]{0.5\linewidth}
      \includegraphics*[width=\textwidth, trim=28mm 7mm 30mm 17mm, clip]{splots_tgt8.eps}
    \end{minipage}
    \caption{Composite plots of entropy profiles for varying cluster
      temperature ranges. Profiles are color-coded based on average
      cluster temperature. Units of the color bars are keV. The solid
      line is the pure-cooling model of \cite{voitbryan}, the dashed
      line is the mean profile for clusters with $\kna \le 50 \ent$,
      and the dashed-dotted line is the mean profile for clusters with
      $\kna > 50 \ent$. {\it{Top left:}} This panel contains all the
      entropy profiles in our study. {\it{Top right:}} Clusters with
      $kT_X < 4$ keV. {\it{Bottom left:}} Clusters with $4\keV < kT_X
      < 8\keV$. {\it{Bottom right:}} Clusters with $kT_X > 8$
      keV. Note that while the dispersion of core entropy for each
      temperature range is large, as the $kT_X$ range increases so to
      does the mean core entropy.}
    \label{fig:splots}
  \end{figure}
\end{center}
\clearpage
\begin{figure}[htp]
  \begin{center}
    \begin{minipage}[htp]{0.9\linewidth}
      \includegraphics*[width=\textwidth, trim=20mm 10mm 10mm 10mm, clip]{k0hist.eps}
      \caption{{\it{Top panel:}} Histogram of best-fit \kna\ for all
        the clusters in \accept. Bin widths are 0.15 in log space.
        {\it{Bottom panel:}} Cumulative distribution of \kna\ values
        for the full sample. The distinct bimodality in \kna\ is
        present in both distributions, which would not be seen if it
        were an artifact of the histogram binning. A KMM test finds
        the \kna\ distribution cannot arise from a simple unimodal
        Gaussian.}
      \label{fig:k0hist}
    \end{minipage}
  \end{center}
\end{figure}
\clearpage
\begin{figure}[htp]
  \begin{center}
    \begin{minipage}[htp]{0.9\linewidth}
      \includegraphics*[width=\textwidth, trim=20mm 10mm 10mm 10mm, clip]{hifl_k0hist.eps}
      \caption{{\it{Top panel:}} Histogram of best-fit \kna\ values
        for the primary \hifl\ sample. Bin widths are 0.15 in log
        space.  {\it{Bottom panel:}} Cumulative distribution of
        best-fit \kna\ values. The distinct bimodality seen in the
        full \accept\ sample (Fig. \ref{fig:k0hist}) is also present
        in the \hifl\ subsample and shares the same gap between the
        low-entropy peak at 10-20 \ent\ and the high-entropy peak at
        100-200 \ent. That bimodality is present in both samples is
        strong evidence it is not a result of an unknown archival
        bias.}
      \label{fig:hiflk0}
    \end{minipage}
  \end{center}
\end{figure}
\clearpage
\begin{figure}[htp]
  \begin{center}
    \begin{minipage}[htp]{0.8\linewidth}
      \includegraphics*[width=\textwidth, trim=20mm 10mm 10mm 10mm, clip]{t0.eps}
    \end{minipage}
    \begin{minipage}[htp]{0.8\linewidth}
      \includegraphics*[width=\textwidth, trim=20mm 10mm 10mm 10mm, clip]{k0cool.eps}
    \end{minipage}
    \caption{{\it{Top panel:}} Log-binned histogram and cumulative
      distribution of best-fit core cooling times, $t_{c0}$
      (eqn. \ref{eqn:tc0}), for all the clusters in \accept. Histogram
      bin widths are 0.2 in log space. {\it{Bottom panel:}} Log-binned
      histogram and cumulative distribution of core cooling times
      calculated from best-fit \kna\ values, $t_{c0}(\kna)$
      (eqn. \ref{eqn:tck0}), for all the clusters in
      \accept. Histogram bin widths are 0.2 in log space. The
      bimodality we observe in the \kna\ distribution is also present
      in best-fit $t_{c0}$. However, the gaps between the two
      populations of $t_{c0}$ and $t_{c0}(\kna)$ differ by $\sim 0.3$
      Gyrs which may be an artifact of the binning.}
    \label{fig:t0}
  \end{center}
\end{figure}



%%%%%%%%%%%%%%%%%%%%
% End the document %
%%%%%%%%%%%%%%%%%%%%
\end{document}

% LocalWords:  PN
