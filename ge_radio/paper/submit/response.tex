\documentclass[11pt]{article}
\setlength{\topmargin}{-.3in}
\setlength{\oddsidemargin}{-0.1in}
\setlength{\evensidemargin}{-0.1in}
\setlength{\textwidth}{6.7in}
\setlength{\headheight}{0in}
\setlength{\headsep}{0in}
\setlength{\topskip}{0.5in}
\setlength{\textheight}{9.25in}
\setlength{\parindent}{0.0in}
\setlength{\parskip}{1em}
\usepackage{common,graphicx,hyperref,epsfig}

\pagestyle{empty}

\begin{document}

Dear Dr. Wrobel:

Below is our reply to the referee's report for ApJ MS\#ApJ80631. The
referee's comments are {\it{italicized}} and our replies are not. We
have found the comments to be very helpful in improving the focus and
discussion of the manuscript, and thank the referee for a prompt
review.

\hrulefill

{\it{``This is a good piece of work, building on some solid
    foundations, extending and improving them. There are a few places
    where the text could be expanded and clarified which I detail
    below. Sect 1, Para 1 - Perhaps give an example of the Chandra
    evidence for AGN interaction for completeness"}}

We have added two citations to examples of the earliest CXO work which
found cavities.

``...many massive galaxies (e.g. Fabian et al. 2000; McNamara et al.
2000).''

\hrulefill

{\it{``Sect 1, Para 2 - Perhaps mention that the coupling of the
    cavities to the ICM is currently uncertain, even if the energies
    balance."}}

We have included such a statement in this section with citations to
two relevant papers.

``However, the details of how AGN feedback is coupled to the
thermodynamics of the host cluster is still uncertain (De Young et
al. 2008; Mathur et al. 2009).''

\hrulefill

{\it{``Mention the large scatter of the B06/08 relations, especially as
    you have an explanation for it later in the paper."}}

We have added a note that the B08 relation has rather large scatter,
and point the reader to the section where this is discussed.

``... and the relations have rather large scatter (discussed in
Section 4.1).''

\hrulefill

{\it{``Sect 2, Para 1 - Some reordering of this paragraph might help it
    be clearer - I was wondering how the Jones et al sample was
    created until the second reading.  e.g. The second sentence should
    go at the end of the paragraph."}}

Agreed. The second sentence has been moved to the end of the
paragraph.

\hrulefill

{\it{``I am surprised that the selection was done only on depressions
    in the X-ray emission as the visibility function of these is not
    sure. A combination with radio emission would make the detection
    of depressions firmer. Perhaps add a sentence justifying the
    selection procedure."}}

Our use of the word ``solely'' is incorrect, and we have changed the
wording to reflect that radio emission was considered, though not
requisite, for a cavity to be suspected. This is, in part, the reason
each cavity system was assigned a figure of merit. We point out that
there is a history of cavities without radio emission being identified
from X-ray data - and subsequently being shown to have radio emission.
NGC 5813 is a good case in point
(e.g. http://adsabs.harvard.edu/abs/2010arXiv1006.4379R).

We find it appropriate to combine this referree comment with the one
below and address them both in Section 3.1. This section now reads:

``Of the 160 gEs, AGN activity was suspected in 21 objects (see
Section 3.1 for details).''

\hrulefill

{\it{``Sect 3.1, Para 1 - As both Jones et al and Nulsen et al are in
    preparation, perhaps add a sentence explaining how the gas
    properties and how the cavity locations were determined."}}

Gas properties are measured via similar methods to Birzan et al. 2004,
\ie\ gas temperature and density and for brevity we have omitted the
details which appear in B04 and Nulsen in prep.

``...by the method of B04. Cavities were identified based primarily on
the presence of surface brightness depressions in the X-ray emitting
gas and their association with radio emission, though the latter is
not requisite (\eg\ Randall et al. 2010). To reflect this fact, each
cavity system is given a figure of merit (see Section 4.1).''

\hrulefill

{\it{``Why is the buoyancy timescale used as opposed to the sound speed
    timescale? And are there appreciable differences between the two
    estimates?"}}

The B08 analysis uses \tbuoy, therefore consistency demanded that we
use the buoyant ages. Because most cavities are about one diameter
from the cluster center, the buoyancy and sonic ages generally differ
by less than a factor of 2.

\hrulefill

{\it{``Sect 3.2, Para 1 - please clarify the 200-400MHz statement - I
    initially read it as a radio power over 200-400MHz range, rather
    than as one at some frequency within this range."}}

Reference to the 200--400 MHz values has been moved completely to
paragraph 4, which itself has been reworded to more clearly reflect
the CATS query and what sources were retrieved.

\hrulefill

{\it{``Sect 3.2, Para 3 - `... detected in NVSS because of a the
    higher flux limit' - please clarify this statement."}}

The NVSS's flux limit does not go as low as the archival VLA
observations used to study IC 4296 and NGC 4782, and thus the power in
diffuse emission was undetected. We have clarified the text to reflect
this point.

``The radio lobes for IC 4296 and NGC 4782 contain significant power
in diffuse, extended emission which is not detected in NVSS because
the NVSS flux limit is higher than the archival observations used.''

\hrulefill

{\it{``The low nuclear contribution to the total radio fluxes, perhaps
    be a little more quantitative - eg `all were below 0.1, except
    NGC12345 which had 0.??.' The lesssim symbol can hide a lot."}}

We have been more specific and point out that all but NGC 6269 have a
nuclear/total flux ratio $< 0.2$.

\hrulefill

{\it{``Sect 4.1, Para 1 - The Figures of Merit discussion, perhaps give
    well known examples of the AGN which fall into each category.
    Also how are the cavities `associated' with the AGN activity?
    Does this requires clear correspondence with extended radio
    emission?  How can cavities which are associated with AGN radio
    activity lack clear boundaries - does this mean that they are
    defined from the radio emission only, and the FM-A ones from radio
    and X-ray morphology. A bit more precision with the wording would
    make everything clear.''}}

We have added examples for each FM using clusters in B04 and changed
the wording to more clearly indicate that the FM-A's are coincident
with radio emission that can be traced back to an AGN. A surface
brightness depression which is coincident with a radio lobe is most
likely a cavity, but the extent of that cavity can be ambiguous
because of projection effects, asymmetry in the large-scale gas
distribution, and lack of statistics resulting from an observation
being too short. Hence, it is possible to see two varieties of
cavities, which we've classified as A's and B's. We have reworded the
text to try and make this point more clear.

\end{document}
