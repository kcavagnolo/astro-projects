\begin{center}
  \begin{figure}[htp]
    \begin{minipage}[htp]{0.5\linewidth}
      \includegraphics*[width=\textwidth, trim=30mm 5mm 40mm 15mm, clip]{f1a.eps}
    \end{minipage}
    \begin{minipage}[htp]{0.5\linewidth}
      \includegraphics*[width=\textwidth, trim=30mm 5mm 40mm 15mm, clip]{f1b.eps}
    \end{minipage}
    \caption{Cavity power vs. radio power. Orange triangles represent
      the galaxy clusters and groups sample from B08. Filled circles
      represent our sample of gEs with colors representing the cavity
      system figure of merit (see Section \S\ref{sec:xray}): green =
      `A,' blue = `B,' and red = `C.' The dotted red lines represent
      the best-fit power-law relations presented in B08 using only the
      orange triangles. The dashed black lines represent our
      \bces\ best-fit power-law relations. {\it{Left:}} Cavity power
      vs. 1.4 GHz radio power. {\it{Right:}} Cavity power vs. 200-400
      MHz radio power.}
    \label{fig:pcav}
  \end{figure}
\end{center}

\begin{figure}[htp]
  \begin{center}
    \begin{minipage}[htp]{0.5\linewidth}
      \includegraphics*[width=\textwidth, trim=30mm 5mm 40mm 15mm, clip]{f2.eps}
      \caption{Comparison of scaling relations between jet power and
        radio luminosity. The solid red line represents the
        \citet[][W99]{w99} model with $k=1$. The dashed black line is
        our best-fit \pjet-\phigh\ relation (Equation
        \ref{eqn:high}). The dotted black lines denote the upper and
        lower limits of our best-fit relation after including
        intrinsic scatter of $\epsilon_{\mathrm{int}} = 1.3$ dex. The
        unfilled black circles denote the poorly confined sources
        discussed in Section \ref{sec:jet}, and the downfacing black
        triangles are FR-I sources taken from the sample in
        \citet[][C08]{2008MNRAS.386.1709C}.}
      \label{fig:radeff}
    \end{minipage}
  \end{center}
\end{figure}

\begin{figure}[htp]
  \begin{center}
    \begin{minipage}[htp]{\linewidth}
      \includegraphics*[width=\textwidth, trim=0mm 0mm 0mm 0mm, clip]{f3.eps}
      \caption{{\it{Left:}} \chandra\ X-ray image of the giant
        elliptical M84 (NGC 4374). Contours trace out 1.4 GHz radio
        emission as observed with VLA C-configuration (green) and
        AB-configuration (white) ranging from $\approx 0.5-50$ mJy in
        log spaced steps of 10 mJy. Note the displacement of the X-ray
        gas around the bipolar AGN jet outflows. M84 examplifies the
        typical interaction between an AGN outflow and a hot gaseous
        halo. {\it{Right:}} VLA B-configuration 1.4 GHz radio image of
        the AGN jets eminating from the giant elliptical NGC
        4261. White contours trace \chandra\ observed X-ray emission
        of the hot halo surrounding N4261. The contours cover the
        surface brightness range of $\approx 5-50$ cts arcsec$^{-2}$
        in linear spaced steps of 5 cts arcsec$^{-2}$. N4261
        demonstrates the characteristic traits of what we have termed
        poorly confined sources: a compact X-ray halo, small
        centralized cavities along the jets, FR-I-like radio
        morphology, and pluming beyond the ``edge'' of the X-ray
        halo.}
      \label{fig:pics}
    \end{minipage}
  \end{center}
\end{figure}
