%% Proposal number 12800362

\documentclass[letterpaper,11pt]{article}
\usepackage{graphics,graphicx,common}
\bibliographystyle{unsrt}
\usepackage[nonamebreak,numbers,sort&compress]{natbib}
\setlength{\textwidth}{6.5in} 
\setlength{\textheight}{9in}
\setlength{\topmargin}{-0.0625in} 
\setlength{\oddsidemargin}{0in}
\setlength{\evensidemargin}{0in} 
\setlength{\headheight}{0in}
\setlength{\headsep}{0in} 
\setlength{\hoffset}{0in}
\setlength{\voffset}{0in}
\makeatletter
\renewcommand{\section}{\@startsection%
  {section}{1}{0mm}{-\baselineskip}%
  {0.5\baselineskip}{\normalfont\Large\bfseries}}%
\makeatother
\begin{document}
\pagestyle{plain}
\pagenumbering{arabic}

\begin{center}
  {\bf\uppercase{X-ray follow-up of radio mini-halo candidates}}
\end{center}
\noindent{\bf{Introduction:}} In a search for the most powerful
steep\footnote{``Steep'' spectral index defined using the VLSS \& NVSS
  radio surveys as $\alpha \equiv \log [S(\nu_1)/S(\nu_2)]/\log
  (\nu_1/\nu_2) < -1.3$, \eg\ radio sources with substantially more
  power at decreasing frequency.} spectrum radio sources residing in
cool core clusters, we have identified two candidate radio mini-halos
(MHs), a rare class of object, in the galaxy clusters Abell 2675
(hereafter A2675; $z = 0.071$) \& Zwicky 808 (hereafter Z808; $z =
0.169$). The candidates are shown in Fig \ref{fig:img}. The A2675 VLSS
74 MHz source is extended, has diameter $\approx 200$ kpc,
$L_{74~\mathrm{MHz}} \approx 10^{40} ~\lum$, $\alpha = -1.54$, and
surface brightness $\mu \approx 50 ~\mu$Jy arcsec$^{-2}$. Likewise,
Z808's VLSS source is extended, has diameter $\approx 450$ kpc,
$L_{74~\mathrm{MHz}} \approx 10^{42} ~\lum$, $\alpha = -1.41$, and
$\mu \approx 700 ~\mu$Jy arcsec$^{-2}$. For both clusters, the 1.4 GHz
emission is significantly more compact than the coincident 74 MHz
emission, and there are indications of merger activity. These are
distinctly the characteristics of the MH population. Utilizing a
sample of 750 galaxy clusters selected from the BCS/eBCS/MACS/REFLEX
surveys, just 10 objects were identified with an emission line central
dominant (cD) galaxy and an associated steep spectrum radio source
with VLSS 74 MHz flux $> 1$ Jy : 2A 0335+096, A133, A496, A2009,
A2675, MKW3s, MKW8, MS 0735.6+7421, Z808, Z2701. Of the 10, A2675 and
Z808 are the only objects not observed with \chandra, and 2A 0335+096
is the only other suspected MH. The correlation between sub-Gyr ICM
core cooling time and cD line emission [1,2] suggests A2675 and Z808
are cool core clusters, strengthening the MH candidacy as all MHs
reside in cool core clusters. The \rosat\ detections reveal
intracluster mediums (ICMs) with $L_X > 3 \times 10^{44} ~\lum$ and
\tcl\ $\sim 4$ and $\sim 7$ keV for A2675 and Z808, respectively
      [3]. The combination of low redshift, high $L_X$, rare MH-like
      radio properties, and likely merger activity make A2675 \& Z808
      ideal \chandra\ targets for studying the connections between
      MHs, cool core magnetism, and gas dynamics. \vspace{5.5pt}

\noindent{\bf{Why Care About Radio Halos?}} The ICM of galaxy clusters
is composed of thermal and non-thermal components. The thermal
component dominates clusters, and is observed via X-ray bremsstrahlung
radiation. Strong evidence exists that halo cooling of clusters with
ICM cooling times $< \Hn^{-1}$ is regulated primarily by feedback from
the cD active galactic nucleus (AGN) [4-6]. At the same time an AGN is
influencing the evolution of the thermal component, it also
contributes to the non-thermal component by injecting high-energy
particles and magnetic fields (\bmag) into the host
environment. Evidence of this contribution comes from the relativistic
jets and synchrotron emission which accompany AGN activity.

Radio MHs, like the candidates in A2675 \& Z808, may be related to the
process of AGN feedback. Interestingly, MHs are unilaterally found in
clusters with core cooling times $< 1$ Gyr that host a powerful ($\ga
10^{40} ~\lum$) central radio galaxy [7,8]. Unlike high-surface
brightness, low-volume FR-I/FR-II sources, MHs are characterized by
low radio surface brightness ($\sim 1-500$ $\mu$Jy arcsec$^{-2}$ at $<
400$ MHz), large extents which fill the core ($d \la 500$ kpc), and a
steep radio spectrum ($\alpha \la -1$). MHs ostensibly show no
connection to an AGN, and occupy a volume sufficiently large that if
the radiating population had originated from a cD AGN, the particles
would radiate away their energy prior to reaching the MH
outskirts. This strict energetic constraint implies {\it{in situ}}
particle acceleration, possibly by compressed or shear \bmag-fields,
or from cosmic-ray ion collisions with the magnetized ICM. However,
these explanations are uncertain, and even if they were not, it is
unclear where the fields and particles powering MH emission originate
and if, like their much larger brethren ``giant'' and ``relic'' halos,
MHs are connected to mergers [9].
\begin{figure}[ht]
  \begin{center}
    \begin{minipage}{0.495\columnwidth}
      \includegraphics*[width=\columnwidth, trim=45mm 20mm 40mm 8mm, clip]{a2675/comp.ps}
    \end{minipage}
    \begin{minipage}{0.495\columnwidth}
      \includegraphics*[width=\columnwidth, trim=42mm 15mm 45mm 15mm, clip]{zw808/comp.ps}
    \end{minipage}
    \caption{1.4 GHz radio images of A2675 {\it{(left)}} and Z808
      {\it{(right)}}. Red contours are X-ray emission, green
      contours are optical emission, and white contours are 74 MHz
      radio emission.}
    \label{fig:img}
  \end{center}
  \vspace{-22pt}
\end{figure}

Given that $\approx50\%$ of clusters have cool cores, with $> 90\%$ of
them undergoing some level of AGN outburst or minor merger, the number
of known MHs is curiously small ($< 20$): 2A 0335+096, A1068, A1413,
A1835, A2029, A2052, A2142, A2390, A2626, MRC 0116+111, MS
1455.0+2232, Ophiuchus, Perseus, PKS 0745-191, RX J1347.5-1145, and RX
J1720.1+2638 [10-18]. We point out that none of these objects
simultaneously satisfies all three of the stringent criterion used for
selecting our two targets, highlighting that we have identified MH
candidates which have the strongest features of other MHs. Further,
systematic low-frequency searches have yielded few confirmed MHs
[19,20]. Four MH candidates ($z=0.3 \dash 0.5$) were recently
identified with GMRT using a similar NVSS-VLSS radio-selected sample
of steep spectrum sources [21]. However, while the GMRT sources are
associated with clusters, none of them are detected X-ray sources,
giving $L_X \la 4 \times 10^{44} ~\lum$. No cD emission lines are
detected in three, and there is no spectroscopy for one. So while
radio-selected samples exist, their utility in defining candidates for
X-ray follow-up is limited.

The paucity of known MHs gives credence to the idea that they are
highly transient or require very specific ICM conditions to
form. While a variety of models have emerged attempting to explain MHs
[\eg\ 22-24], the lack of well-studied MH systems inhibits refinement
of these models using observational constraints. Additionally, if
simulations incorporating AGN feedback are to reproduce the range of
non-thermal sources observed in clusters and yield insight to their
importance for structure formation \& evolution, then better
observational constraints must be achieved. To this end, detailed
X-ray analysis of MH candidate systems is vital as it enables
diagnosis of cluster dynamics (\eg\ with substructure like cold
fronts), AGN energetics (\eg\ via cavities and shocks), and how these
correlate with diffuse non-thermal emission, in ways optical and radio
observations are incapable [\eg\ 25]. Joint high-resolution X-ray and
radio study of the ICM in clusters hosting radio halos is required to
better understand the link between thermal \& non-thermal ICM
components and to reveal the physics responsible for particle
(re-)acceleration in diffuse radio halos.

ICM cold fronts (CFs) may be an especially useful tool for
understanding the connection between a cool core and MH. In some MH
models, ICM bulk motions \& turbulence are responsible for the
re-acceleration of fossil electrons which emit the diffuse synchrotron
emission of a MH [\eg\ 22,24,26]. Interestingly, turbulence and bulk
motions (\eg\ gas sloshing) induced by a subsonic merger event in a
dense cluster core are the same processes which excite CFs [27]. A CF
is detected as a constant pressure contact discontinuity where
downstream gas density and temperature are higher \& lower,
respectively, than the upstream counterparts. That CFs appear to be
long-lived, in spite of cool, high-density gas being co-spatial with
gas sometimes twice as hot, indicates suppression of conduction at the
CF face by \bmag-fields which are likely draped over the front during
its formation [28]. Thus, the properties of a CF (\ie\ size,
thickness, contrast with ambient medium) provide a means for studying
cool core \bmag-field strengths and configurations. If the
\bmag-fields which define a CF are the same fields that produce MHs,
then the study of one illuminates the other. Indeed, there are
examples of MH \& CFs being present in the same system with
indications they are physically related [10,18,29-31]. If more
examples like this could be found, a task which requires deep X-ray
observations, then the cool core-CF-MH relationship could be explored
more thoroughly.\vspace{5.5pt}

\noindent{\bf{Merger Induced MHs in A2675 \& Z808?}} In
Fig. \ref{fig:img} note the large companion galaxies within 15 kpc of
both cDs and that the radio emission is elongated along the axis
including the companions. In both clusters the optical, X-ray, and
radio emission are significantly off-set from each other, suggesting
the cD is moving relative to the ICM. Also note the peculiar Z808 1.4
GHz morphology: sharp edge to NE of core, hole NW of core, radio plume
trailing galaxy SW of cD, compressed ``sandwich'' appearance. All
these features indicate recent merger activity. If so, are there CFs,
shocks, or cavities in these systems which would illuminate the
connection between the cool core, the MH, and previous AGN feedback?
Is the MH power correlated with the cooling luminosity as is expected
for models of particle re-acceleration via magnetohydrodynamic (MHD)
turbulence? Is there evidence of ICM turbulence in the X-ray emission
(\eg\ eddies or twisted wake like in A520)? Might there be powerful
cavities in the X-ray halos of these clusters which can be used to
measure the energetics of an AGN outburst and constrain how
relativistic plasma is transported in the core? Spurred along by these
questions, we propose \chandra\ X-ray observations of 80 ks for A2675
and 100 ks for Z808. We seek to (1) determine if a merger has taken
place as evidenced by CFs or shocks, (2) probe for signs of AGN
activity like cavities, (3) evaluate the connection of the steep
spectrum radio source to the X-ray properties, and (4) model the A2675
\& Z808 MHs using existing theories to constrain the magnetic field
properties and particle content as they relate to the cool core and
AGN. \vspace{5.5pt}

\noindent{\bf{Deep X-ray Data for a Radio Halo?}} The physics of how
MHs are formed may be encoded in the X-ray emission of the ICM. The MH
model of Gitti et al. (2004) employs MHD turbulence frozen into the
gas of the cool core region to re-accelerate fossil electrons. To
evaluate the Gitti model requires measurement of core properties
derived from X-ray data, specifically the cool core radius ($r_c$),
scaled electron gas density ($n_c$), and temperature structure
(\tx). The turbulence model of Kunz et al. (2010) and shear model of
Keshet et al. (2009) also require these parameters be known. The
energy scale where MH synchrotron and inverse Compton (IC) losses are
balanced by re-acceleration, $\gamma_b$, are related to these
quantities by $\gamma_b \propto r_c^{0.8} n_c^{-1}$. If an AGN is
responsible for the MHs, then the time for radio plasma to reach the
edge of the MH should be less than the plasma synchrotron lifetime,
$t_{\mathrm{sync}}$. Buoyancy and sound crossing time arguments are
useful in constraining the time a plasma takes to move about the ICM
[2], and these calculations are based on X-ray measurements ($t_{c_s}
\propto \tx^{-1/2}$ and $p \propto n_c \tx$). Further, constraints on
the ICM turbulent energy density, turbulent lengthscale, and diffusion
coefficient are required, quantities which are $n_c$ and
\tx\ dependent. If the cooling luminosity is much larger than the
synchrotron power, then at a minimum, cooling flow powered turbulence,
$P_{\mathrm{CF}}$, could drive re-acceleration. Since $P_{\mathrm{CF}}
\propto \mdot$, this can only be constrained using high-SN X-ray
spectral analysis of the core. We cannot neglect the possibility that
merger shocks have powered the MH. In which case the energy released
in the shock, determined from shock morphology and \tx\ \& $n_c$
discontinuities, can be used to constrain the energy spectrum of Fermi
accelerated electrons. Because of IC and synchrotron losses, the
energy spectrum implies a $t_{\mathrm{sync}}$, which, when set in the
context of the MH morphology, determines if the halo could be powered
by a shock. We will also measure radial properties of the ICM: gas
mass, gravitating mass, entropy, pressure, cooling time, effective
conductivity (if a strong $T(r)$ gradient is discovered), and inferred
magnetic suppression (to prevent the rapid destruction of a strong
$T(r)$ gradient). \vspace{5.5pt}

\noindent{\bf{Request for Observations:}} Our time requests are aimed
at reaching temperature and density uncertainties necessary for
significant detections of a typical CF, weak shock, cavities, and to
collect sufficient counts for radial \& spatial mapping of ICM
structure. Use of ACIS-S versus ACIS-I results in 40-50\% more total
counts, and the ACIS-S3 FOV encloses a cluster-centric radius of
$R_{2000}$, which is far enough out to probe for large-scale shocks or
CFs. Using $\beta$-models fitted to the survey \rosat\ imaging data,
Cycle 12 ACIS-S count rates were determined from PIMMS. 5,000 mock
surface brightness (MSB) profiles extending to $R_{2000}$ were then
generated via a Monte Carlo. Shallow baseline temperature and
abundance profiles were also created, $T(r) \propto r^{0.2}$ \& $Z(r)
\propto r^{-0.2}$, with normalizations $T(R_{2000}) = \tcl$ \&
$Z(R_{2000}) = 0.3 ~\Zsol$. An isothermal core was also used,
$T(r<R_{7500}) = T(R_{7500})$. For each MSB profile, cumulative counts
profiles were created for exposure times ranging 40-140 ks in 20 ks
steps. Bins with 2,500 counts were defined, and mean \tx\ and
abundance were calculated for each bin from the gradient profiles. A
simulated spectrum was generated in \xspec\ for each bin using an
absorbed thermal model (\mekal), the corresponding exposure time, and
a normalization chosen so the spectral count rate matched the count
rate predicted by the MSB profile for that bin. Source images
consistent with these parameters were created in \idl, and mock
\chandra\ images were created using MARX. A fixed background composed
of the PIMMS and RASS R12 \& R45 emission was included in all
analysis.

Radial profiles were extracted from the simulated data to estimate the
mean uncertainties for each exposure time. We find exposure times of
80 ks for A2675 and 100 ks for Z808 return the best uncertainties per
unit time while maintaining the highest signal-to-noise such that
weighted Voronoi tessellation maps with a minimum binned spatial
resolution of $\sim 5 \dash 10\arcs$ can be created. For A2675, the
mean uncertainties will be $\Delta T_X \pm 0.3 \dash 0.6$ keV and
$\Delta n_c \pm 10\%$, and for Z808, $\Delta T_X \pm 0.5 \dash 0.8$
keV and $\Delta n_c \pm 12\%$. Our simulations indicate we will be
sensitive at $> 2\sigma$ to CFs with temperature jumps $> 1.5$ and
density decreases $> 1.3$, which are typical values for
CFs. Conversely, based on the Rankine-Hugoniot jump conditions, we
should detect a $M > 1.2$ shock at $\ga 2\sigma$. We will also be able
to measure \bmag-field strengths in the CF analysis to $\approx
3\mu$G, well below the level needed for discerning between MH
formation models.

We simulated the presence of cavities in the ICM of our targets by
carving out voids in the source images and creating new mock
observations with MARX. Cavity decrements, their geometries, and
distances from the cluster core span a wide-range of values, thus we
placed voids, obeying the radially dependent axial ratio and SB
dimming relations presented in [5], at a variety of radii. For
spherical plane-of-the-sky cavities, our observations should be
sensitive to $> 10\%$ decrements at $r < 100 ~\kpc$ and $> 30\%$
decrements at $100 < r < 200$ with no detections expected beyond. Our
mock observations loosely indicate we will be sensitive to cavity
powers of $10^{41 \dash 46} ~\lum$. \vspace{5.5pt}

\small
\noindent \input{short.bbl}
\normalsize

\clearpage
\noindent{\bf{Previous \chandra\ Programs:}}\vspace{5.5pt}

Co-I Edge, GO Cycle 2: ``The interaction between radio galaxies and
ICM in the cores of clusters.'' Observation of two potential cavity
systems both of which have been published in larger study of Cavagnolo
et al. 2009, ApJS, 182, 12.\\

Co-I Edge, GO Cycle 4: ``Probing the mass profile of clusters to 10
kpc.'' Observation of Abell 1201 which has been published in Owers et
al. 2009, ApJ, 692, 702.\\

Co-I McNamara, GO Cycle 8: ``AGN Feedback and Galaxy Formation in
Cluster Cores.'' A study of Abell 1664 by Kirkpatrick et al. 2009,
ApJ, 697, 867; a study of RBS 797 is presented in Cavagnolo et
al. 2010 (in preparation for ApJ).\\

Co-I McNamara, GO Cycle 10 LP: ``A Deep Image of the Most Powerful
Cluster AGN Outburst.'' A study of MS 0735.6+7421 utilizing 500 ks of
data is underway.\\

PI Cavagnolo, GO Cycle 10: ``The Hyperluminous Infrared Galaxy IRAS
09104+4109: An Extreme Brightest Cluster Galaxy.'' A detailed study of
IRAS 09104+4109 is presented in Cavagnolo et al. 2010 (in preparation
for MNRAS).\\

PI Cavagnolo maintains the Archive of Chandra Cluster Entropy Profile
Tables (ACCEPT) database and is adding 84 galaxy clusters (156
observations) to the 241 clusters currently in the database. As a
result of maintaining the database, PI Cavagnolo has reduced \&
analyzed 752 CXO observations ($\sim 16$ Msec of data) along with over
80,000 spectra.

\end{document}

The galaxy clusters A2675 and Z808 host candidate radio mini-halos
(MH), a rare class of object, and are ideal for deep Chandra
follow-up. The clusters were identified from 780 other clusters for
their steep spectrum radio source coincident with an optical emission
line cD in a X-ray bright cluster. The radio properties are consistent
with known MHs (r > 200 kpc, surface brightness < 1 mJy arcsec^2,
alpha < -1.1) and indicate in situ reacceleration of electrons. These
targets have no Chandra data, multiwavelength data indicates merger
activity, and the high-Lx's provide an opportunity to constrain the
dynamical state of each cluster in relation to the formation/evolution
of a MH. Our proposed observations will attain data sufficient to
detect cold fronts, weak shocks, and faint cavities.

%% A2675:
%%   no gmrt, only vla c at 1.4 ghz
%%   vlss = 1.09 +- 0.19 Jy
%%   nvss = 0.0112 +- 0.0019 Jy
%%   alpha = -1.54 +- 0.16
%%   halo size = 100 kpc
%%   sur bri = 50 muJy/arcs2
%%   $(\alpha,\delta)$ of (23:55:42.614, +11:20:35.83)
%%   z = 0.0726
%%   DL = 322 Mpc
%%   DA = 1.359 kpc/arcsec
%%   NH = 5.03e20 cm-2
%%   Lx = 1.9e44 e/s, Tx = 4.0 keV (est from Lx-Tx relation) \cite{1998MNRAS.301..881E}
%%   \lha\ not detected by \cite{crawford99}
  
%% Z808 (RXC J0301.6+0155):
%%   no gmrt, lots of vla, but only to 1.4 Ghz
%%   vlss = 24.47 +- 4.58 Jy
%%   nvss = 0.40 +- 0.08 Jy
%%   alpha = -1.41 +- 0.18
%%   halo size = 250 kpc
%%   sur bri = 700 muJy/arcs2
%%   $(\alpha,\delta)$ of (03:01:38.0, +01:55:14)
%%   z = 0.1695
%%   DL = 817 Mpc
%%   DA = 2.895 kpc/arcsec
%%   NH = 6.9e20 cm-2
%%   Lx = 4.2e44 e/s, Tx = 6.7 keV (est from Lx-Tx relation) \cite{1998MNRAS.301..881E}
%%   \lha\ = 4.6e40 e/s \cite{crawford99}
%%   no IR up-turn past 8 microns, prob no AGN \cite{quillen08}


%%%%%%%%%%%%%%%%%%%%%%%%%%%%%%%%%%%%%%%%%%%%%%%%%%%%%%%%%%%%%%%%%%%%
%%%%%%%%%%%%%%%%%%%%%%%%%%%%%%%%%%%%%%%%%%%%%%%%%%%%%%%%%%%%%%%%%%%%

%% Alastair's initial comments

%% 1) I am reluctant to get my name associated with the ``ROSAT All-Sky
%% GCS'' as lead author. Perhaps better to say that the combination of
%% BCS/eBCS/REFLEX gives 750 clusters of whch just 10.... This would
%% prevent the TAC saying this was more work from unpublished surveys
%% they don't understand!

%% 2) At the end of the first paragraph I got a bit confused when you
%% mentioned ``the 1.4GHz sources''. What 1.4GHz sources?  Why not say
%% that 1.4GHz maps of these sources show very little diffuse emission
%% indicating that the VLSS emission is steep spectrum.

%% 3) Not sure the mention of the ``swarm'' of galaxies offset from A2675
%% is very convincing compared to the level of the rest of the case. I'd
%% drop it.

%% 4) In the deep X-ray for a radio halo? section I found the sentence
%% about turbulent energy density a bit clunky as it talks about ``model
%% of [14]''. Better to say `` required in models [14].''...

%%%%%%%%%%%%%%%%%%%%%%%%%%%%%%%%%%%%%%%%%%%%%%%%%%%%%%%%%%%%%%%%%%%%
%%%%%%%%%%%%%%%%%%%%%%%%%%%%%%%%%%%%%%%%%%%%%%%%%%%%%%%%%%%%%%%%%%%%

%% Alastair's second round of comments

%% Add ``...there are just 10 objects... steep spectrum radio source with
%% a VLSS 74MHz flux greater than 1Jy.'' This should make the selection a
%% bit clearer still.

%% The rest looks very professional. As a Chandra TAC member for this
%% round I'd say that this is the sort of proposal that should be near
%% the top on the pile (but I'm biassed!)....

%%%%%%%%%%%%%%%%%%%%%%%%%%%%%%%%%%%%%%%%%%%%%%%%%%%%%%%%%%%%%%%%%%%%
%%%%%%%%%%%%%%%%%%%%%%%%%%%%%%%%%%%%%%%%%%%%%%%%%%%%%%%%%%%%%%%%%%%%
