% saved as prop # 12800017
% re-submitted as prop # 12800025

\documentclass[letterpaper,11pt,twocolumn]{article}
\usepackage{graphicx,common}
%\usepackage[nonamebreak,numbers,sort&compress]{natbib}
%\bibliographystyle{unsrt}
\setlength{\textwidth}{6.5in} 
\setlength{\textheight}{9in}
\setlength{\topmargin}{-0.0625in} 
\setlength{\oddsidemargin}{0in}
\setlength{\evensidemargin}{0in} 
\setlength{\headheight}{0in}
\setlength{\headsep}{0in} 
\setlength{\hoffset}{0in}
\setlength{\voffset}{0in}

\makeatletter
\renewcommand{\section}{\@startsection%
{section}{1}{0mm}{-\baselineskip}%
{0.5\baselineskip}{\normalfont\Large\bfseries}}%
\makeatother

\begin{document}
\pagestyle{plain}
\pagenumbering{arabic}

\begin{center}
\bfseries\uppercase{Abell 1983: An Exceptionally Rare Cool-Core
  Cluster with High Core Entropy}
\end{center}
\noindent{\bf{Introduction}}\\
We propose a 35 ksec observation of the peculiar cluster Abell 1983
(A1983; $z=0.0436$) which has not been targeted with \chandra. This
cluster has a long core cooling time ($\sim 3$ Gyr) and high ICM core
entropy ($\kna > 30~\ent$) suggesting $> 10^{61}~\erg$ of energy has
been injected into the gas, yet this cluster unambiguously has a cool
core, peaked central iron abundance, but no detected \halpha\ or radio
emission. A1983 shares characteristics with both the cool core and
non-cool core cluster populations, and a detailed study of the cluster
core ($r \la 100$ kpc) requires high-resolution data from
\chandra\ before a better understanding of this strange cluster's
dynamic state can be formed.

The central cooling time of the intracluster medium (ICM) in many
clusters of galaxies is $\ll H_0^{-1}$. An expected consequence of
short central cooling time was that massive cooling flows, $> 100
~\Msol~\pyr$, should form [1], but these massive flows have instead
turned out to be trickles [2,3] with most of the hot ICM never
reaching temperatures lower than $\sim T_{\mathrm{virial}}/3$. In
recent years, this ``cooling flow problem'' has been the focus of much
study as the solutions have broad impact in the areas of galaxy
formation, \eg\ explaining apparent suppression of the high-mass end
of the galaxy luminosity function. Researchers are therefore very
interested to know what, and how, heating mechanisms act to suppress
the formation of a continuous cooling gas phase in cluster cores, and
also why the cluster population divides into two types: cool cores
(CCs) and non-cool cores (NCCs).

One viable heating source comes in the form of feedback from active
galactic nuclei (AGN) [4]. But while several robust models for heating
the ICM via AGN feedback now exist, the details of the feedback loop
remain unresolved. ICM entropy has proven to be a very useful quantity
for understanding the process of AGN heating and its effects on
processes such as star formation the suppression of prodigious cooling
in CC clusters. ICM temperature, $T$, and density, $\rho$, primarily
reflect the depth and shape of the dark matter potential well, and
taken alone, they do not entirely reveal the thermal history of the
ICM. But, in the context of entropy, $K = T\rho^{-2/3}$, one finds a
more fundamental property of the ICM which is only altered by heating
and cooling. Measuring entropy from X-ray data thereby gives a direct
measure of a cluster's thermal history.

One method of parameterizing a cluster's radial ICM entropy profile is
by fitting the simple function $K(r) = \kna + \khun
(r/100~\kpc)^{\alpha}$ to the entropy profile and taking the best-fit
value of \kna\ to be a quantification of the cluster's core
entropy. This was the task undertaken in [5] for a
\chandra\ archive-limited sample of 240+ galaxy clusters ($\approx 13$
Msec of data). Shown in Fig. \ref{fig:hist} is the log-space
distribution of the best-fit \kna\ for the \chandra\ archival sample
[6]\footnote{See also
  http://www.pa.msu.edu/astro/MC2/accept/}). Utilizing the results
from that archival study, [7] showed that below a \kna\ of $\approx
30~\ent$, \halpha\ emission and powerful radio emission ($\lradio >
10^{41}~\ergps$) from the cluster BCG essentially turns-on, a similar
result regarding star formation was found by [8]. Clusters in the
low-\kna\ part of the bimodal population are generally characterized
by ``relaxed'' morphologies, bright compact cores, short central
cooling times, strong CCs, line emission from the BCG, and radio-loud
AGN. Clusters in the high-\kna\ part of the bimodal distribution are
generally hot, puffed-up, isothermal clusters with more merger systems
than the low-\kna\ population.

The characteristic entropy threshold ($\kna \la 30~\ent$, shown in
Fig. \ref{fig:hist} by the far-left vertical dashed line) has
subsequently yielded insight on the process of when, and possibly how,
a multiphase medium can form in cluster cores. [9] hypothesized that
the entropy threshold results from the influence of thermal electron
conduction. [10] have also shown in their theoretical work that
conduction is a possible explanation for the entropy threshold. [9]
further propose that the bimodal \kna\ distribution results from the
effects of conduction whereby cooling in the core of clusters with
$\kna \ga 30~\ent$ is dramatically slowed, and hence those clusters
can be slowly removed via mergers and/or very powerful AGN outbursts
from the $\kna=30-60~\ent$ region and moved to higher \kna.

The importance of the galaxy cluster A1983 to this discussion is that
it is an extraordinarily rare cluster which has a $\kna \approx
30-60~\ent$ (inferred from $K(r)$ profile shown in [11]) but has a CC
and peaked central metal abundance (again using the profiles in
[11]). Like all other clusters with \kna\ between $30-60~\ent$,
\eg\ the ``gap'' in the \kna\ distribution, the BCG in A1983 is not
detected in \halpha\ ($\lha < 0.2\times10^{40}~\ergps$) and has no
associated radio emission ($\lradio < 0.11\times10^{40}~\ergps$). But,
unlike any other cluster in the gap, A1983 has a cool core and a
highly peaked central ICM metal abundance [11]. It appears on the
surface as though A1983 is currently the only example of a cluster
which straddles the CC and NCC populations. Is that because the
cluster is undergoing a transition from one population to the other?
Does this cluster occupy a stage in the ICM entropy life-cycle which
is short-lived yet very important if we are to understand the CC-NCC
dichotomy?\\

\noindent{\bf{A1983: An Odd Galaxy Cluster}}\\
The discussion presented below utilizes results presented in [11] from
the analysis of \xmm\ data, in addition to results from performing our
own analysis of the same dataset. Does A1983 actually have a CC? The
temperature of the gas within a radius of 50 kpc of the cluster center
is $T_X(R < 50~\kpc) = 1.88 \pm 0.07$ keV, and the global cluster
temperature (with the central 70 kpc excised) is measured to be
$T_{cluster} \approx 3.2 \pm 0.4$ keV. Defining a CC cluster to have
$T_X(R < 50~\kpc) / T_{\mathrm{cluster}} < 1$ at $\ge 2\sigma$, A1983
solidly classifies as having a CC. In addition, the metal abundance
profile peaks in the core with a value of $\approx 0.6 \pm
0.08~\Zsol$. From the entropy profile presented in [11], A1983
unambiguously has a \kna\ between $30-60 ~\ent$, which lies squarely
within the poorly populated gap of the bimodal \kna\ population. Of
the more than 240 clusters in \accept, only 19 have a best-fit
\kna\ in this same range, and under the CC definition provided above,
none of those 19 clusters have a CC and none has a centrally peaked
abundance profile.

There is no detected \halpha\ emission from the BCG of A1983 [12], and
no detected radio source found in either the NVSS or VLA FIRST. The
BCG is however detected as an \oii\ emitter and
\hdelta-\hbeta\ absorber [13]. Additionally, there is a very bright
near-UV source detected by \galex\ ($M_{\mathrm{NUV}} \approx -16$
mag) which is associated with the BCG. So while there is no detection
of gas at $T \sim 10^4$ K (as indicated by the lack of \halpha), the
UV and optical spectral features suggest star formation is present in
the BCG (or nascent AGN activity). For a cluster with $\kna >
30~\ent$, the properties outlined above are very odd and it may be the
case that A1983's BCG has a very large corona [see [14] for a
  discussion of coronae], which acts like a ``mini-cooling core.''
A1983 has one last odd feature: the \xmm\ observation shows the X-ray
isophotes to the west of the cluster center are compressed, suggesting
the presence of ICM sub-structure, possibly in the form of a cold
front or weak shock.

In some ways, A1983 appears to be a typical non-cool core cluster: a
large surface brightness core, no \halpha\ emission from the BCG, no
radio emission in the cluster core, and an ICM feature which may be an
indicator of recent merger or very energetic AGN feedback
activity. But A1983 is an enigma in that it shares traits with cool
core clusters: the distance between the location of the BCG and X-ray
centroid is $< 5\arcs$, a core temperature $< 0.5
~T_{\mathrm{virial}}$, a centrally peaked iron abundance, and the BCG
may be forming stars or have a smoldering AGN. A1983 is already a rare
object because it has a \kna\ which places it in the gap of the
bimodal core entropy distribution, but A1983 is the {\bf{only}}
cluster we know of with $\kna = 30-60~\ent$ and a cool core, making
A1983 exceptionally rare.\\

\noindent{\bf{Scientific Questions}}\\
\begin{figure}[ht]
  \begin{center}
    \includegraphics*[width=\columnwidth, trim=33mm 8mm 42mm 18mm, clip]{k0hist}
    \caption{Log-space histogram of best-fit core entropy, \kna, for the
      current \accept\ database. The dashed vertical lines bound the
      region $\kna = 30-60~\ent$.}
    \vspace{-24pt}
    \label{fig:hist}
  \end{center}
\end{figure}
A1983 may be an example of a cluster transitioning being from CC to
NCC or vice versa. How this processes proceeds is poorly understood,
be it through mergers, very powerful AGN feedback (\eg\ MS
0735.6+7421), or the prolonged influence of conduction. Studying this
system in detail will yield insight into a long-standing question in
cluster science: how and why does the cluster population divide nearly
evenly between CC and NCC clusters?

The first question we'd like to address is: what is the temperature
structure of the cluster core, specifically the inner 50 kpc? Is there
multicomponent gas in the core? How is it possible that this
high-\kna\ cluster has a CC? Is there a distinct transition from hot
ICM to cool BCG corona as has been seen in many other
high-\kna\ clusters? If the BCG corona can be discerned from the ICM,
what are the corona's properties (temperature, abundance, density)?
Are there bubbles in the ICM? If so, what are the energetics of the
outburst which formed them? What do those energetics tell us about the
past and future of the cluster and the supermassive black hole at the
center of the BCG? 
\begin{figure}[ht]
  \begin{center}
    \includegraphics*[width=0.88\columnwidth, trim=69mm 34mm 75mm 29mm, clip]{a1983_fov}
    \caption{\xmm\ image of A1983. Green square bounds the
      \chandra\ ACIS-S3 field of view and the white circle has radius 50
      kpc.}
    \vspace{-22pt}
    \label{fig:a1983}
  \end{center}
\end{figure}

On the matter of the compressed western isophotes, we'd like to know
if there is a cold front or shock. The presence of a cold front would
serve as an excellent diagnostic of thermal conduction and diffusion
in A1983's ICM. If instead there is a shock, this would yield
interesting information regarding the energetics of a recent merger
event or AGN outburst and if the CC is being disrupted, or possibly
destroyed. Moreover, we'd like to know, can thermal conduction prevent
gas cooling all the way into the core of the cluster?  Is it possible
that the CC is presently being heated efficiently by conduction and
the core is effectively evaporating?\\

\noindent{\bf{Improvements on \xmm\ Analysis}}\\
We have analyzed the 32 ksec archival \xmm\ observation taken
2002-02-14 by Arnaud. Within an aperture of $R_{1000}$ we measure
$\sim 15000$ source counts for the flare-clean, point source clean
events file. We measure a 90\% confidence cluster $\tx = 3.21 \pm 0.4$
keV sans the central 70 kpc (0.65 cts s$^{-1}$), and $2.22 \pm 0.5$
keV with the central 70 kpc (0.78 cts s$^{-1}$). The \xmm\ resolution
is insufficient to address the scientific questions we have put
forward, to create radial profiles of the core region, and the
observation lacks the signal-to-noise for 2D map creation.\\

\noindent{\bf{Request for \chandra\ Observation}}\\
We request a 35 ksec ACIS-S observation of the galaxy cluster A1983
for the purpose of studying the cluster core with a specific focus on
analyzing the dynamics and energetics encoded in the ICM. The
\xmm\ image of A1983 is shown in Fig. \ref{fig:a1983} with regions
overlaid designating the \chandra\ field of view and the inner 50
kpc. \chandra's high spatial resolution is ideally, and necessarily,
suited for observing A1983. We are attempting to resolve features on
scales of 5-10 kpc, and at $z = 0.044$, $10 ~\kpc = 11.9\arcs$ or 24
pixels at the resolution of the ACIS detectors. Using the
background-subtracted \xmm\ 0.3-6.0 keV count rate for a region
extending to $R_{1000} \approx 620\arcs$, an \xmm\ determined global
temperature of 2.2 keV, a \chandra\ energy window of 0.5-8.0 keV, for
an extended source, and Galactic $\nhi = 1.79\times10^{20}$ cm$^{-2}$,
PIMMS predicts a source count rate of 1.237 cts s$^{-1}$ for the Cycle
12 ACIS-S detector responses. We have selected the ACIS-S detector
because the combination of the low cluster temperature range ($\tx =
1.8-2.5$ keV) and larger soft energy ($\tx < 2$ keV) effective area of
ACIS-S compared with ACIS-I results in an additional 15K counts during
a 35 ksec observation.

Under the assumption of no time lost to flares, the requested exposure
time is sufficient to yield 9 radial temperature bins containing
$\approx 5000$ counts each. Using the \xmm\ $\tx$ profile as a guide,
we simulated spectra in \xspec\ using the Cycle 12 responses. Our
requested observation enables us to measure temperatures within $\pm
0.2$ keV for $\tx < 4$ keV and $\pm 0.5$ keV for $\tx > 4$ keV. For
the inner 50 kpc, the signal-to-noise (SN) will be sufficient to
measure temperatures in 2D bins as small as $2.2.\arcs$. The high-SN
is vital for measuring properties of a BCG coronae, if one is found,
which are typically faint and compact.

To investigate the presence of cold front(s) or shock(s), we will
generate high-quality profiles for: temperature, abundance, density,
and pressure. If either is found, the cold front(s) will be modeled in
detail using the methods outlined in [15], while shock(s) will be
modeled using the code of [16]. In addition, we will use profiles of
gas mass, gravitating mass, and gas fraction to study the cluster
dynamical state. We will then use the difference between the hard-band
($2.0_{\mathrm{rest}}-7.0$ keV) \& broad-band (0.7-7.0 keV)
temperatures, which has been shown to be a good measure of the
dynamical state of the cluster [17], to determine if the cluster has
experienced a merger recently. We will also use a hardness ratio
profile \& map, which are equally good diagnostics of mergers [18], to
probe dynamical state. Profiles for entropy, cooling time, effective
conductivity, and inferred magnetic suppression factor will be
utilized to constrain the physical processes which may be responsible
for the cluster thermal state, \eg\ conduction, shocks, or AGN
feedback. Using weighted Voronoi tessellation [19] and contour binning
[20] methods we will produce 2D temperature, entropy, density,
pressure, and hardness ratio maps which will further illuminate the
cluster thermal state and dynamics.

We are encouraged by our extensive experience with similar analyses
that A1983, once imaged with \chandra, will yield interesting results
and information regarding the CC-NCC dichotomy. How this unique, and
ostensibly rare, object fits into the framework of cool core evolution
may tell us about a very short-lived but very important stage of
cluster formation.\\

\small
%\bibliography{cavagnolo}
\noindent(1) Fabian. ARA\&A, 32:277-318, 1994.\\
(2) Peterson, et al. A\&A, 365:L104-L109, 2001.\\
(3) Tamura, et al. A\&A, 365:L87-L92, 2001.\\
(4) McNamara \& Nulsen. ARA\&A, 45:117-175, 2007.\\
(5) Cavagnolo. PhD thesis, Michigan State, 2008.\\
(6) Cavagnolo, et al. ApJS, 182:12-32, 2009.\\
(7) Cavagnolo, et al. ApJ, 683:L107-L110, 2008.\\
(8) Rafferty, et al. ApJ, 687:899-918, 2008.\\
(9) Voit, et al. ApJ, 681:L5-L8, 2008.\\
(10) Guo, et al. ApJ, 688:859-874, 2008.\\
(11) Pratt \& Arnaud. A\&A, 408:1-16, 2003.\\
(12) Crawford, et al. MNRAS, 306:857-896, 1999.\\
(13) Edwards, et al. MNRAS, 379:100-110, 2007.\\
(14) Sun, et al. ApJ, 657:197-231, 2007.\\
(15) Markevitch \& Vikhlinin. PhRep., 443:1-53, '07.\\
(16) Nulsen, et al. ApJ, 625:L9-L12, 2005.\\
(17) Cavagnolo, et al. ApJ, 682:821-834, 2008.\\
(18) Henning, et al. ApJ, 697:1597-1620, 2009.\\
(19) Diehl \& Statler. MNRAS, 368:497-510, 2006.\\
(20) Sanders. MNRAS, 371:829-842, 2006.

\onecolumn
\normalsize
\noindent{\bf{Previous \chandra\ Programs}}\\

Chandra General Observer Project, Cycle 10: ``The Hyperluminous
Infrared Galaxy IRAS 09104+4109: An Extreme Brightest Cluster
Galaxy.'' A detailed study of IRAS 09104+4109 utilizing the 75 ks
observation from Cycle 10 is presented in Cavagnolo et al. 2010 which
has been submitted to ApJ for publication.\\

PI Cavagnolo maintains the Archive of Chandra Cluster Entropy Profile
Tables (ACCEPT) database and is presently adding 68 new galaxy
clusters (128 observations) to the existing 241 clusters currently in
the database. As a result of maintaining the database, PI Cavagnolo
has analyzed and reduced +684 CXO observations ($> 15$ Msec of data)
and +50,000 spectra.

\end{document}
