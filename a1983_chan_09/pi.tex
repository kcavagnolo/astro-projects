\documentclass[letterpaper,11pt,twocolumn]{article}
%\documentclass[letterpaper,11pt]{article}
\usepackage{graphicx,common}
%\usepackage[nonamebreak,numbers,sort&compress]{natbib}
%\bibliographystyle{plainnat}
\setlength{\textwidth}{6.5in} 
\setlength{\textheight}{9in}
\setlength{\topmargin}{-0.0625in} 
\setlength{\oddsidemargin}{0in}
\setlength{\evensidemargin}{0in} 
\setlength{\headheight}{0in}
\setlength{\headsep}{0in} 
\setlength{\hoffset}{0in}
\setlength{\voffset}{0in}

\makeatletter
\renewcommand{\section}{\@startsection%
{section}{1}{0mm}{-\baselineskip}%
{0.5\baselineskip}{\normalfont\Large\bfseries}}%
\makeatother

\begin{document}
\pagestyle{plain}
\pagenumbering{arabic}

\begin{center}
\bfseries\uppercase{Abell 1983: An Exceptionally Rare Cool-Core
  Cluster with High Core Entropy}
\end{center}

\noindent{\bf{Introduction}}\\
We propose a 35 ksec observation of the previously unobserved peculiar
cluster Abell 1983. This cluster has a long core cooling time ($\sim
3$ Gyr) and high ICM core entropy ($\kna > 30 \ent$) suggesting $>
10^{61} \erg$ of energy has been injected into the gas, yet this
cluster unambiguously has a cool core, peaked central iron abundance,
but no detected \halpha\ or radio emission. A1983 shares
characteristics with both the cool core and non-cool core cluster
populations, and a detailed study of the cluster core ($r \la 100$
kpc) first requires high-resolution data from \chandra\ before a
better understanding of this strange cluster's dynamic state can be
formed.

The central cooling time of the intracluster medium (ICM) in many
clusters of galaxies is $\ll H_0^{-1}$. An expected consequence of
short central cooling time was that massive cooling flows, $> 100
\Msol \pyr$, should form (see Fabian 1994 for review), but these
massive flows have instead turned out to be trickles (\eg\ Peterson et
al. 2001 and Tamura et al. 2001) with most of the hot ICM never
reaching temperatures lower than $\sim T_{\mathrm{virial}}/3$. In
recent years, this ``cooling flow problem'' has been the focus of much
study as the solutions have broad impact in the areas of galaxy
formation, \eg\ explaining apparent suppression of the high-mass end
of the galaxy luminosity function. Researchers are therefore very
interested to know what, and how, heating mechanisms act to suppress
the formation of a continuous cooling gas phase in cluster cores, and
also why the cluster population divides into two types: cool cores
(CCs) and non-cool cores (NCCs).

One viable heating source comes in the form of feedback from active
galactic nuclei (AGN) (McNamara et al. 2007). But while several robust
models for heating the ICM via AGN feedback now exist, the details of
the feedback loop remain unresolved. ICM entropy has proven to be a
very useful quantity for understanding the process of AGN heating and
its effects on processes such as star formation the suppression of
prodigious cooling in CC clusters. ICM temperature, $T$, and density,
$\rho$, primarily reflect the depth and shape of the dark matter
potential well, and taken alone, they do not entirely reveal the
thermal history of the ICM. But put in the context of entropy, $K =
T\rho^{-2/3}$, one finds a more fundamental property of the ICM which
is only altered by heating and cooling. Measuring entropy from X-ray
data thereby gives a direct measure of a cluster's thermal history.

One method of parameterizing a cluster's radial ICM entropy profile is
by fitting the simple function $K(r) = \kna + \khun
(r/100\kpc)^{\alpha}$ to the entropy profile and taking the best-fit
value of \kna\ to be a quantification of the cluster's core
entropy. This was the task undertaken in Cavagnolo's Ph.D. thesis for
a \chandra\ archive-limited sample of 240+ galaxy clusters ($\approx
13$ Msec of data). Shown in Fig. \ref{fig:hist} is the log-space
distribution of the best-fit \kna\ for the \chandra\ archival sample
(Cavagnolo et al. 2009\footnote{See also
  http://www.pa.msu.edu/astro/MC2/accept/}). Utilizing the results
from that archival study, Cavagnolo et al. 2008a showed that below a
\kna\ of $\approx 30 \ent$, \halpha\ emission and powerful radio
emission ($\lradio > 10^{41} \ergps$) from the cluster BCG essentially
turns-on, a similar result regarding star formation was found by
Rafferty et al. 2008.  Clusters in the low-\kna\ part of the bimodal
population are generally characterized by ``relaxed'' morphologies,
bright compact cores, short central cooling times, strong CCs, line
emission from the BCG, and radio-loud AGN. Clusters in the
high-\kna\ part of the bimodal distribution are generally hot,
puffed-up, isothermal clusters with more merger systems than the
low-\kna\ population.
\begin{figure}[htp]
\begin{center}
  \includegraphics*[width=0.8\columnwidth, trim=30mm 8mm 40mm 18mm, clip]{k0hist}
  \caption{Histogram of best-fit core entropy, \kna, for the current
    \accept\ database. The dashed vertical lines bound the region
    $\kna = 30-60 \ent$.}
  \label{fig:hist}
\end{center}
\end{figure}

The characteristic entropy threshold ($\kna \la 30 \ent$, shown in
Fig. \ref{fig:hist} by the far-left vertical dashed line) has
subsequently yielded insight on the process of when, and possibly how,
a multiphase medium can form in cluster cores.  Voit et al. 2008
hypothesized that the entropy threshold results from the influence of
thermal electron conduction. Guo et al. 2008 have also shown in their
theoretical work that conduction is a possible explanation for the
entropy threshold. Voit et al. 2008 further propose that the bimodal
\kna\ distribution results from the effects of conduction whereby
cooling in the core of clusters with $\kna \ga 30 \ent$ is
dramatically slowed, and hence those clusters can be slowly removed
via mergers and/or very powerful AGN outbursts from the $\kna=30-60
\ent$ region and moved to higher \kna.

The importance of the galaxy cluster Abell 1983 (A1983) to this
discussion is that it is an extraordinarily rare cluster which has a
$\kna \approx 30-60 \ent$ (infered from $K(r)$ profile shown in Pratt
\& Arnaud 2003) but has a CC and peaked central metal abundance (again
using the profiles in Pratt \& Arnaud 2003). Like all other clusters
with \kna\ between $30-60 \ent$, \eg\ the ``gap'' in the
\kna\ distribution, the BCG in A1983 is not detected in
\halpha\ ($\lha < 0.2\times10^{40} \ergps$) and has no associated
radio emission ($\lradio < 0.11\times10^{40} \ergps$). But, unlike any
other cluster in the gap, A1983 has a cool core and a highly peaked
central ICM metal abundance (Pratt \& Arnaud 2003). It appears on the
surface as though A1983 is currently the only example of a cluster
which straddles the CC and NCC populations. Is that because the
cluster is undergoing a transition from one population to the other?
Does this cluster occupy a stage in the ICM entropy life-cycle which
is short-lived yet very important if we are to understand the CC-NCC
dichotomy?\\

\noindent{\bf{Abell 1983: An Odd Galaxy Cluster}}\\
The discussion presented below utilizes results presented in Pratt \&
Arnaud 2003 from the analysis of \xmm\ data, in addition to results
from performing our own analysis of the same dataset. Does A1983
actually have a CC? The temperature of the gas within a radius of 50
kpc of the cluster center is $T_X(R < 50\kpc) = 1.88 \pm 0.07$ keV,
and the global cluster temperature (with the central 70 kpc excised)
is measured to be $T_{cluster} \approx 3.2 \pm 0.4$ keV. Defining a CC
cluster to have $T_X(R < 50\kpc) / T_{\mathrm{cluster}} < 1$ at $\ge
2\sigma$, A1983 solidly classifies as having a CC. In addition, the
metal abundance profile peaks in the core with a value of $\approx 0.6
\pm 0.08 \Zsol$. From the entropy profile presented in Pratt \& Arnaud
2003, A1983 unambiguously has a \kna\ between $30-60 \ent$, which lies
squarely within the poorly populated gap of the bimodal
\kna\ population. {\bf{Of the more than 240 clusters in \accept, only
    19 have a best-fit \kna\ in this same range, and under the CC
    definition provided above, none of those 19 clusters have a CC and
    none has a centrally peaked abundance profile.}}

There is no detected \halpha\ emission from the BCG of A1983 (Crawford
et al. 1999), and no detected radio source found in either the NVSS or
VLA FIRST. The BCG is however detected as an \oii\ emitter and
\hdelta-\hbeta\ absorber (Dressler et al. 2004, Edwards et
al. 2007). Additionally, there is a very bright near-UV source ($M
\sim -16$) detected by \galex\ which is associated with the BCG. So
while there is no detection of gas at $T \sim 10^4$ K (as indicated by
the lack of \halpha), the UV and optical spectral features suggest
star formation is present in the BCG. For a cluster with $\kna > 30
\ent$, the properties outlined above are very odd and it may be the
case that A1983's BCG has a very large associated corona (Sun et
al. 2007), which acts like a ``mini-cooling core.'' A1983 has one last
odd feature: the \xmm\ observation shows the X-ray isophotes to the
west of the cluster center are compressed, suggesting the presence of
ICM sub-structure, possibly in the form of a cold front or weak shock.

In some ways, A1983 appears to be a typical non-cool core cluster: a
large surface brightness core, no \halpha\ emission from the BCG, no
radio emission in the cluster core, and an ICM feature which may be an
indicator of recent merger or very energetic AGN feedback
activity. But A1983 is an enigma in that it shares traits with cool
core clusters: the distance between the location of the BCG and X-ray
centroid is $< 5\arcs$, a core temperature $< 0.5
T_{\mathrm{virial}}$, a centrally peaked iron abundance, and the BCG
appears to be forming stars. A1983 is already a rare object because it
has a \kna\ which places it in the gap of the bimodal core entropy
distribution, but A1983 is the {\bf{only}} cluster we know of with
$\kna = 30-60 \ent$ and a cool core, making A1983 exceptionally
rare.\\

\noindent{\bf{Scientific Questions}}\\
A1983 may be an example of a cluster transitioning being from CC to
NCC or vice versa. How this processes proceeds is a complete unknown,
be it through mergers or very powerful AGN feedback (\eg\ MS
0735.6+7421). Studying this system in detail will yield insight into a
long-standing question in cluster science: how and why does the
cluster population divide nearly evenly between CC and NCC clusters?

The first question we'd like to address is: what is the temperature
structure of the cluster core, specifically the inner 50 kpc? Is there
multicomponent gas in the core? How is it possible that this
high-\kna\ cluster has a CC? Is there a distinct transition from hot
ICM to cool BCG corona as has been seen in many other
high-\kna\ clusters? If the BCG corona can be discerned from the ICM,
what are the corona's properties (temperature, abundance, density)?
Are there bubbles in the ICM? If so, what are the energetics of the
outburst which formed them? What do those energetics tell us about the
past and future of the cluster and the supermassive black hole at the
center of the BCG? 

On the matter of the compressed western isophotes, we'd like to know
if there is a cold front or shock. The presence of a cold front would
serve as an excellent diagnostic of thermal conduction and diffusion
in A1983's ICM. If instead there is a shock, this would yield
interesting information regarding the energetics of a recent merger
event or AGN outburst and if the CC is being disrupted, or possibly
destroyed. Moreover, we'd like to know, can thermal conduction prevent
gas cooling all the way into the core of the cluster?  Is it possible
that the CC is presently being heated efficiently by conduction and
the core is effectively evaporating?\\

\noindent{\bf{Improvements on \xmm\ Analysis}}\\
Using SAS version 8.0.0, we analyzed the 32 ksec archival
\xmm\ observation taken 2002-02-14 by Arnaud. While the
\xmm\ observation served the intended purpose of analyzing the
large-scale cluster emission to derive accurate masses, the
\xmm\ observation is ill-suited for detailed study of the cluster core
and ICM sub-structure on scales $< 10$ kpc, owing to both the larger
PSF of \xmm\ compared to \chandra\ and the diminished observation time
from the flare. Within an aperture of $R_{1000}$ we measure $\sim
15000$ source counts for the flare-clean, point source clean events
file. We also measure a global temperature of $3.21 \pm 0.4$ keV
without the central 70 kpc (0.65 cts/s), and $2.22 \pm 0.5$ keV with
the central 70 kpc (0.78 cts/s), both at 90\% confidence. The
\xmm\ observation is insufficient to address the scientific questions
we have put forward, to create detailed radial profiles of the core
region, and the observation also lacks the signal-to-noise to create
2D maps which can be used to study cluster structure and dynamics.
\begin{figure}[htp]
  \includegraphics*[width=\columnwidth, trim=70mm 40mm 70mm 30mm, clip]{a1983_fov}
  \caption{\xmm\ image of A1983. Green square bounds the
    \chandra\ ACIS-S3 field of view and the white circle has radius 50
    kpc.}
  \label{fig:a1983}
\end{figure}\\

\noindent{\bf{Request for \chandra\ Observation}}\\
We request a 35 ksec ACIS-S observation of the galaxy cluster Abell
1983 for the purpose of studying the cluster core with a specific
focus on analyzing the dynamics and energetics encoded in the ICM. The
\xmm\ image of A1983 is shown in Fig. \ref{fig:a1983} with regions
overlaid designating the \chandra\ field of view and the inner 50
kpc. \chandra's high spatial resolution is ideally, and necessarily,
suited for observing Abell 1983. We are attempting to resolve features
on scales of 5-10 kpc, and at $z = 0.044$, $10\ kpc = 11.9\arcs$ or 24
pixels at the resolution of the ACIS detectors. Using the
background-subtracted \xmm\ 0.3-6.0 keV count rate for a core excised
region extending to $R_{1000}$, an \xmm\ determined global temperature
of 2.2 keV, a \chandra\ energy window of 0.7-7.0 keV, for an extended
source, and Galactic $N_{HI} = 1.79\times10^{20}$ cm$^{-2}$, PIMMS
predicts a source count rate of 1.084 cts/s for the Cycle 11 ACIS-S
detector responses. We have selected the ACIS-S detector because the
combination of the low cluster temperature range ($kT_X = 1.8-2.5$
keV) and larger soft energy ($kT_X < 2$ keV) effective area of ACIS-S
compared with ACIS-I results in an additional 10K counts during a 35
ksec observation.

Under the assumption of no time lost to flares, the requested exposure
time is sufficient to yield 11 radial temperature bins containing 2500
counts each, which will allow us to measure temperatures within $\pm
0.5$ keV for $kT_X < 4$ keV and $\pm 0.8$ keV for $kT_X > 4$ keV. We
will then use the difference between the hard-band ($2.0_{rest}-7.0$
keV) and broad-band (0.7-7.0 keV) temperatures, which has been shown
to be a good measure of the dynamical state of the cluster (Cavagnolo
et al. 2008b), to determine if the cluster has experienced a merger
recently. In addition to generating high-quality temperature,
abundance, and hardness ratio profiles, the large number of
temperature bins combined with a high-resolution surface brightness
profile will be used to construct high-resolution radial density,
pressure, gas mass, entropy, cooling time, effective conductivity, and
inferred magnetic suppression factor profiles. Using the weighted
Voronoi tessellation binning code of Diehl et al. 2006 and contour
binning code of Sanders 2006, we will produce 2D temperature, entropy,
density, pressure, and hardness ratio maps. For the inner 50 kpc, the
signal-to-noise will be sufficient to measure temperatures in 2D bins
as small as $2.2.\arcs$. We will also use measured densities,
temperatures, and pressures to investigate the presence of a possible
cold front or shock to the west of the cluster center.

We are encouraged by our extensive experience with similar analysis
that Abell 1983, once imaged with \chandra, will yield very
interesting results. How this unique and ostensibly rare object fits
into the framework of cool core evolution may tell us about a very
short-lived but very important stage of cluster formation.\\

\noindent{\bf{References}}\\
Cavagnolo et al. 2008a, ApJ, 683, 107\\
Cavagnolo et al. 2008b, ApJ, 682, 821\\
Cavagnolo et al. 2009, arXiv 0902.1802\\
Diehl et al. 2006, MNRAS, 368, 497\\
Edwards et al. 2007, MNRAS, 379, 100\\
Fabian et al. 1994, ARA\&A, 32, 277\\
Guo et al. 2008, ApJ, 688, 859\\
McNamara et al. 2007, ARA\&A, 45, 117\\
Peterson et al. 2001, A\&A, 365, 104\\
Pratt \& Arnaud 2003, A\&A, 408, 1\\
Rafferty et al. 2008, ApJ, 687, 899\\
Sanders 2006, MNRAS, 371, 829\\
Tamura et al. 2001, AAP, 365, 87\\
Voit et al. 2008, ApJ, 681, 5

\onecolumn
\noindent{\bf{History of Chandra Programs (PI Only) For Cavagnolo}}

Chandra General Observer Project, Cycle 10: ``The Hyperluminous
Infrared Galaxy IRAS 09104+4109: An Extreme Brightest Cluster
Galaxy.'' A study of IRAS 09104+4109 is underway and will be ready for
submission to ApJ for publication by the end of June 2009.

\end{document}
