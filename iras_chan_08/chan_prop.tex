\documentclass[letterpaper,11pt,twocolumn]{article}
%\documentclass[letterpaper,11pt]{article}
\usepackage{graphicx,common}
\usepackage[nonamebreak,numbers,sort&compress]{natbib}
\bibliographystyle{plainnat}
\setlength{\textwidth}{6.5in} 
\setlength{\textheight}{9in}
\setlength{\topmargin}{-0.0625in} 
\setlength{\oddsidemargin}{0in}
\setlength{\evensidemargin}{0in} 
\setlength{\headheight}{0in}
\setlength{\headsep}{0in} 
\setlength{\hoffset}{0in}
\setlength{\voffset}{0in}

\makeatletter
\renewcommand{\section}{\@startsection%
{section}{1}{0mm}{-\baselineskip}%
{0.5\baselineskip}{\normalfont\Large\bfseries}}%
\makeatother

\begin{document}
\pagestyle{plain}
\pagenumbering{arabic}

\begin{center}
\bfseries\uppercase{The Hyperluminous Infrared Galaxy IRAS 09104+4109: An Extreme Brightest Cluster Galaxy}
\end{center}

\noindent{\bf{1. Introduction}}\\
The central cooling time of the intracluster medium (ICM) in many
clusters of galaxies is $\ll H_0^{-1}$. An expected consequence of
this short cooling time was that massive cooling flows, $> 100 \Msol
\pyr$, should form [6], but these massive
flows have instead turned out to be trickles [10] with
most of the hot ICM never getting lower than $\sim T_{\mathrm{virial}}/3$. In
recent years, this ``cooling flow problem'' has been the focus of much
study as the solutions have broad impact in the areas of
galaxy formation, e.g. explaining apparent suppression of the
high-mass end of the galaxy luminosity function. The cluster
community therefore is very interested in mechanisms that act to
retard the formation of a continuous cooling gas phase in cluster
cores.

One viable heating source comes in the form of feedback from active
galactic nuclei (AGN) [9]. But while several
robust models for heating the ICM via AGN feedback now exist, the
details of the feedback loop remain unresolved. Most models of AGN
feedback quench cooling by heating the ICM with kinetic or
mechanical energy supplied by the central supermassive black hole
(SMBH) of the brightest cluster galaxy (BCG). Cool gas is channeled
onto the SMBH, initiating an AGN feedback cycle, and the ejection
of high energy particles that carve out cavities/bubbles in the ICM. The
work then done by these bubbles on the surrounding ICM goes into
displacing ICM gas around and into the bubble wake where the
gravitational potential energy is then released as enthalpy
[1].

ICM entropy has proven to be a very useful quantity for understanding
this process of AGN heating and its effects on processes
such as star formation. ICM temperature, $T$, and density, $\rho$,
primarily reflect the depth and shape of the dark matter potential
well, and taken alone, they do not entirely reveal the thermal history
of the ICM. But put in the context of entropy, $K=T\rho^{-2/3}$, we have
a more fundamental property of the ICM which is only affected by
heating and cooling. Measuring entropy from X-ray data thereby gives
us a direct measure of the cluster thermal history. For example,
departures from a radial power-law entropy distribution are indicative
of past heating and cooling of the ICM.

\begin{figure}
\begin{center}
\includegraphics*[width=\columnwidth, trim=28mm 10mm 30mm 17mm, clip]{splots}
\caption{
Plot of entropy, $K(r) = Tn_{elec}^{-2/3}$, versus physical radius for
the 208 clusters from [2]. The observed range of core
entropy is consistent with models of AGN feedback. Color coding is
for global cluster temperature in keV.
}
\label{fig:ent}
\end{center}
\end{figure}

Figure \ref{fig:ent} shows the radial entropy profiles for 208
clusters analyzed in the Ph.D. thesis of PI Cavagnolo. These
profiles were derived using 272 {\it{Chandra}} archival observations
with a nominal total exposure time of $\sim9$ Msec. From this extensive entropy project we have found
that indicators of feedback, such as radio-loud sources (assumed to be
AGN-related) and H$\alpha$ emission from star formation, are correlated
with gas with low ($< 30 \ent$) core entropy [3]. The
importance of IRAS 09104+4109 is that we suspect it is an object which
doesn't fit neatly into the AGN feedback models, and as such, a
detailed study of this peculiar system will be important for further
understanding how feedback couples to the processes of star formation,
BCG assembly, and heating of the ICM.\\

\noindent{\bf{2. IRAS 09104+4109: An Extreme BCG}}\\
IRAS 094104+4109 ($z=0.442$) is the most luminous infrared BCG at $z<1.0$ and also
hosts the most powerful radio source in the IRAS catalog. At z $< 0.5$
and $L > 10^{11} L_{\odot}$ the most common extragalactic objects are
infrared galaxies. Among this population are a subset of rare
hyperluminous infrared galaxies (HLIRGs) with $L_{IR} \geq 10^{13}
L_{\odot}$. IRAS 09104+4109 falls into the category of HLIRG. Most all
ULIRGs and HLIRGs are interacting/merging spirals or relics of recent
mergers. Unlike fellow HLIRGs, IRAS 09104+4109 is the BCG in the
flattened, Abell richness class 2 cluster MACS J0913.7+4056. Even more
peculiar is that unlike most BCGs found in rich clusters, 99\% of
IRAS 09104+4109's bolometric luminosity emerges longward of 1$\mu$m
and peaks between 5-60$\mu$m. This enormous IR luminosity is
attributed to an obscured Seyfert type-2 AGN with a large dust torus
lying between the broad-line and narrow-line regions
[7] and not to starbursts, as is the case for
many luminous infrared galaxies. In addition, only three other objects
in the IRAS catalog are comparable to 09104+4109 in luminosity and
they all lie at redshifts which would require unreasonably long
exposures to attain spectroscopic quality signal to noise: IRAS
15307+3252 ($z=0.93$), IRAS 16347+703 ($z=1.33$), and IRAS 10214+4724
($z=2.29$).

Of all objects in the IRAS catalog, 09104+4109 hosts the most powerful
radio source, a borderline FRII/FRI with $P_{1.4GHz} =
3.2\times10^{24}$ W Hz$^{-1}$. Yet because of the steep radio spectrum
and huge IR luminosity, the radio source would be classified as
``quiet'' at higher frequencies. Based on the different beaming
directions of the radio jet and UV ionization cone,
[7] conclude the AGN has undergone a recent ($<
70$ kyr) merger or cataclysmic event which has altered the beaming
direction of the central AGN. As a result, the radio lobes are no
longer receiving power from the AGN and a new jet axis has been
established. Adding to the mystery of this object is that the nuclear
absorber may have changed column density in the last decade from
Compton-thick to -thin, making IRAS09104 the only ``changing-look''
AGN ever observed [11].\\

\begin{figure}
\begin{center}
\includegraphics*[width=\columnwidth, trim=0mm 0mm 0mm 0mm, clip]{chan_vla_utrao}
\caption{
The left panel shows the existing {\textit{Chandra}} data with low
resolution 20 cm {\it{VLA}} contours (white), high resolution 20 cm
{\it{VLA FIRST}} contours (green), and the new AGN outflow axis as
determined from the bipolar ionization cone discussed in
[7] (cyan line). The right panel shows the same
{\textit{Chandra}} image with X-ray contours and green elliptical
regions highlighting an X-ray surface brightness decrement and edge.
}
\label{fig:chanrad}
\end{center}
\end{figure}

\noindent{\bf{3. Bubbles in IRAS 09104+410}}\\
The first {\it{Chandra}} observation of this object showed 
what is most likely an X-ray cavity $\approx
30$ kpc NW of the BCG in a cluster at $z > 0.4$ (\S{3.1} of
[8]). Analysis of the publicly-available,
low-resolution radio data from {\it{VLA First}} reveals a suggestive
alignment of the NW radio jet and the observed X-ray decrement (see
Fig. \ref{fig:chanrad}). X-ray emission in the opposite direction and
on the opposing side of the nuclear region is also suspiciously flat
and coincident with the other pole of the radio jet. We suggest the NW
decrement is the stem of a larger bubble and the SE plateau results
from interaction of the ICM with another large bubble. The highest
redshift bubbles observed to date are in RBS 797 at $z = 0.350$, a
cluster which was awarded 40 ks of additional time in Cycle 9.

Bubbles are a way of indirectly studying heating of the ICM by AGN,
and in the case of IRAS 09104+4109 we have the chance to see how this
rare object fits into the overall picture of AGN feedback. The
presence of dust supports the case for bubbles, as opposed to a chance
superposition of shocked regions, because the rims of bubbles
tend to be dense and cold, not hot as would be the case with
shocks. Coincidence of the radio jets and two prominent X-ray features
is unlikely to be a chance projection of overdense regions which have
no underlying physical connection. If confirmed, these would be the
highest redshift bubbles found to date.\\

\noindent{\bf{4. Scientific Questions}}\\
We are interested in obtaining a better {\it{Chandra}} image of
the central 100 kpc around IRAS 09104+4109 for the purpose of studying
the ICM structure and core activity. The radio
source in IRAS 09104+4109 is undergoing a transition from a powerful
FRII to a mostly radio-quiet/FRI. Thus we would like to know if the radial
entropy distribution slope is relatively flat and resembles other
radio-quiet clusters like Abell 1650 [5], or if IRAS09104 is
interacting with the cluster environment in a exotic manner. Thus we ask:\\
1. Is the cluster housing IRAS 01904+4109 more like a typical cool
core cluster with moderate AGN activity, or is something more
unusual happening? Can we definitively
classify this object as being in a short-lived and elusive
transitional phase of galaxy, cluster, and AGN evolution?\\
2. Has the change in beaming direction of the radio source created
multiple sets of bubble pairs?\\
3. Using the size and gas pressures of the suspected bubbles as
diagnostics, how much energy is being released into the ICM via the
AGN, and how much has the SMBH grown in mass as a result?\\
4. We are aware of a recent change in the dynamics of the radio
source, so can we detect this change in the 2-dimensional entropy and
pressure structures at $r < 70$ kpc and $r > 70$ kpc?\\
5. Is the power output of the AGN (as measured by bubbles) large
enough to quench cooling and eventually suppress star formation in
the BCG?\\
6. The AGN feedback cycle is believed to be $\sim 10^{7-8}$ yrs
[12]. Do we find signatures of previous feedback
cycles at large radii (e.g. more bubbles or entropy plateaus) to
constrain the feedback timescale and compare it with other clusters?\\
Answers to these questions require resolving the extended X-ray
emission at radii greater than 70 kpc and having more counts from the
core, both of which are not satisfied by the existing
{\textit{Chandra}} and {\it{XMM-Newton}} observations.\\

\noindent{\bf{5. Proposed Research: Why not use existing data?}}\\
Using CIAO 3.4 and CALDB 3.4.2, we analyzed the existing ACIS-S3
observation taken 1999-11-03 by Fabian
[8]. The nominal 9.1 ksec exposure shows
contamination by two strong flares which reduce the usable exposure
time to $\approx5$ ksec. However, there is an additional long-duration
soft flare contaminating the remaining exposure time. Only by
addition of a cut-off power law to the background during spectral
fitting are we able to constrain a temperature. This additional
background component also introduces an unwanted systematic into the
spectral analysis which results in larger uncertainty.

While this observation serves the purpose of analyzing the very bright
nuclear point source well enough, it is ill-suited for studies of
extended emission. Within an aperture of $r_{2500}$ we find 8,500
background-subtracted source counts. This total is insufficient to
create more than two radial temperature bins, and the signal-to-noise
is far too low to create 2D temperature, entropy, or pressure maps. We
are able to measure a global temperature of $8.06^{+3.25}_{-2.02}$ keV
without the central 50 kpc (0.697 cts/s), and $5.45^{+1.31}_{1.05}$
keV with the central 50 kpc (0.888 cts/s), both at 90\% confidence. We
are unable to resolve any extended emission or spatial features beyond
the central $\approx 70$ kpc.\\

\noindent{\bf{6. Request for new observation}}\\
We request a new 75 ksec ACIS-I observation of this object for
the purpose of resolving ICM features beyond the central 70 kpc, with
a specific focus on analyzing X-ray cavities associated with the AGN,
threshing out the energetics of the radio-ICM interaction, and
deriving the radial entropy distribution. {\it{Chandra}}'s high
spatial resolution is necessary and ideal for observing
IRAS 09104+4109. We are attempting to resolve features on scales of
5-10 kpc, and at $z=0.442$, 10 kpc$=1.75''$ or 3.5 pixels at the
resolution of the ACIS detector. The outer edge of the NW radio lobe,
which is likely the maximum outer edge of any bubble we may find, lies
at 100 kpc from the nuclear point source and 90 kpc for the SE
lobe. Using the count rate for the core excised region, a temperature
of 8.06 keV, an energy window of 0.7-7.0 keV, an extended emission
area of $\approx80\mathrm{K~arcsec}^2$, and $N_{HI} =
1.36\times10^{20}$ cm$^{-2}$, PIMMS predicts a count rate of 0.538
cts/s for Cycle 10 which is consistent with our present analysis.

Under the assumption of no flares, the requested exposure time is
sufficient to yield eight radial temperature bins containing $\approx
5000$ counts each. This will enable us to measure temperatures within
$\pm0.1$ keV for $kT < 4$ keV and $\pm0.2$ keV for $kT > 4$ keV. These
temperature bins combined with surface brightness profiles will then
be used to construct high-resolution radial density, pressure,
entropy, and mass profiles to answer the first of our
scientific questions: Is the cluster containing IRAS 09104+4109 more
like radio-quiet clusters or low core entropy clusters with cool cores
and active feedback?
Using the adaptive binning code of [4], we
will also construct 2D temperature, entropy, and pressure maps to
answer the questions regarding the change in dynamics of the AGN. For
the inner 70 kpc, the signal-to-noise will be sufficient to measure
temperatures in bins as small as $1.5''$. We will also use measured
densities and temperatures to calculate bubble pressures and
consequently the $pV$ work done inflating these bubbles. These
energetics calculations will then be used to analyze the AGN feedback
mechanism and ICM heating.

We are encouraged by our extensive experience with similar analyses
that IRAS 09104+4109, once adequately exposed, will yield interesting
answers to the questions of how an AGN interacts with the
cluster environment (e.g. bubbles), and how BCG formation is coupled
to the ICM and AGN activity. How this unique and extreme object fits
into the framework of AGN feedback may tell us about a very
short-lived but important stage of cluster and BCG formation. It
will also provide interesting constraints of radiative efficiency and
power output.\\

\noindent{\bf{References}}\\
1. Birzan et al. ApJ, 607:800-809, 2004\\
2. Cavagnolo et al. ApJS, submitted\\
3. Cavagnolo et al. ApJL, submitted\\
4. Diehl et al. MNRAS, 368:497-510, 2006\\
5. Donahue et al. ApJ, 630:L13-16, 2005\\
6. Fabian et al. MNRAS, 180:479-484, 1977\\
7. Hines et al. ApJ, 512:145-156, 1999\\
8. Iwasawa et al. MNRAS, 321:L15-L19, 2001\\
9. McNamara et al. ARA\&A, 45:117-175, 2007\\
10. Peterson et al. A\&A, 365:104-109, 2001\\
11. Piconcelli et al. A\&A, 473:85-89, 2007\\
12. Voit \& Donahue ApJ, 634:955-963, 2005\\

\end{document}
