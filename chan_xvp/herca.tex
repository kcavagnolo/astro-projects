\documentclass[12pt,letterpaper]{article}
\setlength\textwidth{6in}
\setlength\textheight{8in}
\setlength\oddsidemargin{0.25in} % LaTeX adds a default 1in to this!
\setlength\evensidemargin{0.25in}
\setlength\topmargin{-0.0in} % LaTeX adds a default 1in to this!
\setlength\headsep{0in}
\setlength\headheight{0in}
\setlength\footskip{1in}
\usepackage{graphicx}
\usepackage{mathptmx}
\begin{document}

\noindent$\bullet$ Interactions between radio jets and their hot atmospheres\\
\noindent$\bullet$ Constraints on possible heating mechanisms\\

In our sample, the archetype of extremely powerful AGN outbursts and
transitional radio galaxies is represented by Hercules A -- currently
the second most powerful AGN outburst known ($E_{\rm{tot}} > 10^{61}$
erg). The spectacular and complex interaction between the large-scale
X-ray environment of Hercules A and the AGN outflow are shown in
Figure \ref{fig:herca}. Using $\sim 100$ ks of {\it{Chandra}} data,
the studies of Nulsen et al. 2005, 2011 present analyses of the ICM
cavities and spherical, Mach $\sim 1.6$ shock surrounding the
cavities. The studies showed that the shock forms a continuous cocoon
around the cavities/radio lobes, and possesses 100$\times$ the power
radiated by gas within the cluster cooling radius. The energetic
demands of the outburst are such that the accretion of hot or cold gas
to fuel the AGN are strained to near maximal efficiencies. This
distinguishes Hercules A from not just the other objects in our
sample, but from radio galaxies in general, and gives us the
opportunity to better understand the extreme end of the AGN feedback
process. The Hercules A shock is more energetic than any of the shocks
in our other targets, and is younger than those in Cygnus A and Hydra
A, but older than the shock in Virgo A. Within the context of our
sample, Hercules A's unique location in the age-energy plane
highlights its value as a probe for investigating (blah...)\\

\noindent$\bullet$ Jet particle content and entrainment\\
\noindent$\bullet$ Nonthermal Xray emission from radio source\\
\noindent$\bullet$ Mixing and turbulence (jet decollimation, entrainment)\\
\noindent$\bullet$ Complete cavity census\\

The synchrotron power output of Hercules A between 10--10,000 MHz
($P_{\rm{radio}} \approx 4 \times 10^{44}$ erg s$^{-1}$) is on-par
with Cygnus A, and is distinctly that of the FR-II class of radio
galaxies. However, similar to the lower-power FR-I class, the radio
morphology is jet-dominated and there are no detected hotspots. The
western jet-lobe structure contains ring-like features, which it has
been argued is strong evidence of intermittent nuclear activity that
has culimnated in the full-source having an immense projected size of
$\sim 500$ kpc across and $\sim 300$ kpc at its widest. Further, the
eastern jet is X-ray bright, but the western jet is not, and there are
clear signs of both jets decollimating at $\approx 50$ kpc from the
nucleus. The radio properties of Hercules A are unique in general and
particularly among our sample. As such, an X-ray study of Hercules A
at unprecedented resolution and depth will give the astrophysics
community the opportunity to (blah...)\\

\noindent$\bullet$ AGN fueling energetics and SMBH growth\\
\noindent$\bullet$ Constrain the presence (or absence) of multiphase gas\\
\noindent$\bullet$ Metal enriched outflows\\

At the heart of Hercules A is a compact ($\sim 10$ kpc), resolved
thermal source which does not host a point source but does host the
fuel powering the AGN outburst. HST observations revealed that the
optical core is composed of the cD AGN host galaxy and a small merger
companion. At the very center of the cD, VIMOS IFU H$\alpha$+N[II]
observations indicate the presence of a structured, rotating gas disk
which is oriented orthogonal to the jet axis. The optical complexity
of the Hercules A nucleus is difficult to understand in relation to
the hot, X-ray component given the limitations of the existing
data. In order to piece together the link between the nuclear hot and
cold gas phases, understand how the AGN is fueled and powered, and
investigate the redistribution of metal-enriched gas on scales the
size of the cD halo, we must have high-SN data in bins the size of
individual ACIS pixels. Data of this quality will allow us to study
(blah...)

\end{document}

cool core galaxy cluster ($z = 0.154$; $L_X \approx 5 \times 10^{44}$
erg s$^{-1}$; $M_{\rm{tot}} \approx 2 \times 10^{14} ~\rm{M_{\odot}}$)
with central cooling time of $\sim 500$ Myr, core entropy of $\sim 10$
keV cm$^{2} $ that has X-ray surface brightness peaked on central,
active radio galaxy with what appears to be a merging companion.
