Galaxy Clusters: Relaxation and Feedback
Ken Cavagnolo, Michigan State University

The adiabatic model of hierarchical structure formation predicts
clusters of galaxies which are scaled versions of each other. This
model also predicts the most massive galaxies in the Universe should
be rife with young stellar populations. Observations have long shown
however that clusters adhere to steeper scaling relations with larger
intrinsic dispersion than theory predicts. In addition, the most
massive galaxies are composed of old stars with much less mass than
predicted. In this talk I will address both of these discrepancies as
they relate to cosmological and galaxy formation studies. By
focusing on the processes of cluster relaxation and feedback, I will
draw attention to how a better understanding of intracluster medium
temperature inhomogeneity and entropy can aide in resolving these
issues.
