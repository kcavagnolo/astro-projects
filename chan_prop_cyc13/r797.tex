\newcommand{\ms}{MS 0735.6+7421}

\documentclass[letterpaper,11pt]{article}
\usepackage{graphics,graphicx,common}
\bibliographystyle{unsrt}
\usepackage[nonamebreak,numbers,sort&compress]{natbib}
\setlength{\textwidth}{6.5in} 
\setlength{\textheight}{9in}
\setlength{\topmargin}{-0.0625in} 
\setlength{\oddsidemargin}{0in}
\setlength{\evensidemargin}{0in} 
\setlength{\headheight}{0in}
\setlength{\headsep}{0in} 
\setlength{\hoffset}{0in}
\setlength{\voffset}{0in}
\makeatletter
\renewcommand{\section}{\@startsection%
  {section}{1}{0mm}{-\baselineskip}%
  {0.5\baselineskip}{\normalfont\Large\bfseries}}%
\makeatother
\begin{document}
\pagestyle{plain}
\pagenumbering{arabic}

\begin{center}
  {\bf\uppercase{The Complex Cavities, Shocks, and Nucleus of the
      Extreme AGN Outburst in RBS 797}}
\end{center}

Over the last decade, X-ray observations have revealed numerous
examples of the complex interaction between the relativistic outflows
from active galactic nuclei (AGN) and the hot halo of the host galaxy
\cite[\eg][]{hydraa0}. Shocks and cavities resulting from the AGN
outflow-halo interaction provide a direct measure of the $pV$ work
performed on the halo and thus are useful for estimating the total
energy of an AGN outburst \cite[see][for a review]{mcnamrev}. Using
cavities as calorimeters, studies have shown that AGN release $\sim
10^{55-62}$ erg of energy at rates of $\sim 10^{41-46} ~\lum$,
sufficient to offset a significant amount of the radiative losses of
the host halo \cite[\eg][]{birzan04}. Further, in these systems, short
halo cooling times, the appearance of multiphase gas, host galaxy star
formation rates, and AGN activity are closely correlated
\cite{crawford99, edge01, haradent, rafferty06}, indicating the
presence of a fine-tuned cooling-heating feedback loop. These results
have shaped prevailing galaxy formation models such that they now
utilize AGN feedback as the primary means for regulating late-time
galaxy evolution \cite[\eg][]{croton06, bower06}. Though our
understanding of AGN feedback has improved greatly, the details of how
AGN are fueled, the importance of shocks in heating cluster cores, how
multiple gas phases are uplifted/mixed, and the role AGN have in
re-distributing metal enriched gas remain elusive.

Of particular interest are extreme AGN outbursts ($E \ga 10^{61}$ erg)
in gas-poor hosts, such as Cygnus A, Hercules A, Hydra A, \ms, and RBS
797 \cite{2006ApJ...644L...9W, herca, hydraa, ms0735, r797}. With the
exception of RBS 797, large-scale shocks, metal-enriched outflows, and
gas mixing have been detected in these systems and directly linked to
the AGN activity. The extreme nature of the interaction in these
systems provides a unique opportunity to better understand similar
processes in more common lower power systems where detecting shocks or
mixing requires prohibitively long exposure times. In addition, the
mass accretion processes which can supply fuel to power these extreme
outbursts are strained to the point of unrealistic efficiencies, which
has led to speculation that some galaxies may host an ultramassive, or
maximally spinning, supermassive black hole (SMBH) \cite{msspin,
  minaspin}. These rare systems are thus important targets for
studying the general process of AGN feedback, and in this proposal we
focus on RBS 797, one the most powerful AGN outbursts currently known
but which has the shallowest X-ray data that is insufficent to study
shocks, metal enrichment, and mixing. {\bf Thus, we propose to obtain
  a XX ks observation of RBS 797 to...}.\\

{\bf{Why RBS 797?}} Analysis of the cavities and large-scale shock in
\ms\ using the X-ray data alone provides good limits on the outburst
energetics, gives reasonable constraints on the age of radio lobes,
but says little about the jet/lobe compositions and radiative
efficiencies. The inclusion of broadband radio data addresses these
disparities by making it possible to take a complete census of the
radiating particle population. \ms\ is a steep-spectrum radio source,
$\alpha > -2$, so it is critical to our science objectives to obtain
radio observations at the lowest frequencies possible where the cavity
system stores a tremendous amount of energy in old radiating
populations distributed over large-scales. The 74 MHz data, in
particular, is essential to the study of \ms\ because it extends our
knowledge of the radio spectrum into the frequency regime where the
break frequency is reliably estimated.\\

-- two deep cavs with symmetric ridge around them... just like CygA,
HercA, MS07!
-- low freq emission beyond cavs along axis of small-scale, nuclear
jets... multiple black holes? projection effects? earlier epoch of
activity?
-- bright nuclear source good for studying accretion 
-- multiple resid structures may be an even richer cav sys

{\bf{What will be new?}} By combining the total energy estimates from
the X-ray analysis and the synchrotron energy estimates from the radio
analysis, the jet and lobe magnetic field strengths, compositions,
radiative efficiencies, environmental pressure balance, degree of
confinement, and deviation from equipartition can be calculated and
inferred \cite[\eg][]{2003MNRAS.342..399G, 2004AJ....127...48L,
  2005MNRAS.364.1343D, 2006MNRAS.372.1741D, 2006ApJ...648..200D,
  birzan08, pjet}. We can also investigate secondary processes such as
matter entrainment and gas shocking by respectively mapping radio
source composition and comparing cavity dynamical ages with
synchrotron age estimates \cite[\eg][]{2006ApJ...644L...9W,
  birzan08}. In addition, the low-frequency radio emission (and most
definitely at 74 MHz) may reveal larger cavity volumes where higher
frequency emission has faded, thus providing better cavity volume
estimates and improved constraints on the \ms\ outburst energetics. In
these ways, the synergy between the X-ray and radio observations
provide vital information on the AGN outburst and an rare outlier in
the general process of AGN feedback.\\

{\bf{Request for Observations:}}\\
{\bf{Previous \chandra\ Programs:}}

\bibliography{cavagnolo}

\end{document}

