\documentclass[letterpaper,11pt]{article}
\usepackage{graphics,graphicx,common}
\bibliographystyle{unsrt}
\usepackage[nonamebreak,numbers,sort&compress]{natbib}
\setlength{\textwidth}{6.5in} 
\setlength{\textheight}{9in}
\setlength{\topmargin}{-0.0625in} 
\setlength{\oddsidemargin}{0in}
\setlength{\evensidemargin}{0in} 
\setlength{\headheight}{0in}
\setlength{\headsep}{0in} 
\setlength{\hoffset}{0in}
\setlength{\voffset}{0in}
\makeatletter
\renewcommand{\section}{\@startsection%
  {section}{1}{0mm}{-\baselineskip}%
  {0.5\baselineskip}{\normalfont\Large\bfseries}}%
\makeatother
\begin{document}
\pagestyle{plain}
\pagenumbering{arabic}

\begin{center}
  {\bf\uppercase{Deep X-ray Observations of Two Radio Mini-Halos:
      Testing Models of Halo Formation}}
\end{center}
In a search for the most powerful steep\footnote{``Steep'' spectral
  index defined using the VLSS \& NVSS radio surveys as $\alpha \equiv
  \log [S(\nu_1)/S(\nu_2)]/\log (\nu_1/\nu_2) < -1.3$, \eg\ radio
  sources with substantially more power at decreasing frequency.}
spectrum radio sources residing in cool core clusters, we have
discovered a pair of rare radio mini-halos (see Figure \ref{fig:img})
in the galaxy clusters Abell 2675 ($z = 0.071$) and Zwicky 808 ($z =
0.169$). As a class, mini-halos are characterized by their diffuse,
low radio surface brightness, steep radio spectral indices, unilateral
association with cool core clusters, confinement of the radio emission
to the region of the cluster where ICM cooling times are $\ll
\Hn^{-1}$, and suspected association with gentle disturbance of the
ICM (\eg\ subsonic mergers). The properties of the extended, resolved
radio emission in both clusters are consistent with these criteria:
the halo of A2675 has a radius of $\approx 100$ kpc, integrated
spectral index of $\approx -1.5$, and surface brightness $\mu \approx
50 ~\mu$Jy arcsec$^{-2}$. Likewise, Z808's halo has a radius $\approx
225$ kpc, spectral index $\approx -1.4$, and $\mu \approx 700 ~\mu$Jy
arcsec$^{-2}$. Further, both clusters are bright X-ray sources
detected with \rosat\ ($L_X > 3 \times 10^{44} ~\lum$) and host
central dominant (cD) galaxies which are optical line-emitters. The
high-X-ray luminosities and close correlation between short central
ICM cooling times and line-emitting cDs suggests the clusters both
have cool cores, in line with the class of clusters which are known to
host mini-halos. A2675 and Z808 are also incredibly rare objects: of
the 750 objects in the combined BCS/eBCS/MACS/REFLEX surveys, A2675
and Z808 are two of only 10 clusters that host a line-emitting cD and
have an associated steep spectrum radio source with a 74 MHz flux $>
1$ Jy. Of these 10 objects (the other eight being 2A 0335+096, A133,
A496, A2009, A2675, MKW3s, MKW8, MS 0735.6+7421, Z808, Z2701), only
A2675 and Z808 have radio morphologies consistent with that of
mini-halos.\\

\noindent{\bf{Why are mini-halos important?}} The ICM of galaxy
clusters is composed of thermal and non-thermal components. The
thermal component dominates clusters, and is observed via X-ray
bremsstrahlung radiation. Strong evidence exists that halo cooling of
clusters with ICM cooling times $< \Hn^{-1}$ is regulated primarily by
feedback from the cD active galactic nucleus (AGN) [4-6]. At the same
time an AGN is influencing the evolution of the thermal component, it
also contributes to the non-thermal component by injecting high-energy
particles and magnetic fields (\bmag) into the host
environment. Evidence of this contribution comes from the relativistic
jets and synchrotron emission which accompany AGN activity.

Radio mini-halos, like the candidates in A2675 \& Z808, may be related to the
process of AGN feedback. Interestingly, mini-halos are unilaterally found in
clusters with core cooling times $< 1$ Gyr that host a powerful ($\ga
10^{40} ~\lum$) central radio galaxy [7,8]. Unlike high-surface
brightness, low-volume FR-I/FR-II sources, mini-halos are characterized by
low radio surface brightness ($\sim 1-500$ $\mu$Jy arcsec$^{-2}$ at $<
400$ MHz), large extents which fill the core ($d \la 500$ kpc), and a
steep radio spectrum ($\alpha \la -1$). mini-halos ostensibly show no
connection to an AGN, and occupy a volume sufficiently large that if
the radiating population had originated from a cD AGN, the particles
would radiate away their energy prior to reaching the mini-halo
outskirts. This strict energetic constraint implies {\it{in situ}}
particle acceleration, possibly by compressed or shear \bmag-fields,
or from cosmic-ray ion collisions with the magnetized ICM. However,
these explanations are uncertain, and even if they were not, it is
unclear where the fields and particles powering mini-halo emission originate
and if, like their much larger brethren ``giant'' and ``relic'' halos,
mini-halos are connected to mergers [9].
\begin{figure}[ht]
  \begin{center}
    \begin{minipage}{0.495\columnwidth}
      \includegraphics*[width=\columnwidth, trim=45mm 20mm 40mm 8mm, clip]{a2675_comp.ps}
    \end{minipage}
    \begin{minipage}{0.495\columnwidth}
      \includegraphics*[width=\columnwidth, trim=42mm 15mm 45mm 15mm, clip]{zw808_comp.ps}
    \end{minipage}
    \caption{1.4 GHz VLA radio images of the mini-halo candidates
      A2675 {\it{(left)}} and Z808 {\it{(right)}}. Red contours are
      cluster X-ray emission, green contours are optical emission, and
      white contours are 74 MHz VLA radio emission.}
    \label{fig:img}
  \end{center}
  \vspace{-22pt}
\end{figure}

Given that $\approx50\%$ of clusters have cool cores, with $> 90\%$ of
them undergoing some level of AGN outburst or minor merger, the number
of known mini-halos is curiously small ($< 20$): 2A 0335+096, A1068, A1413,
A1835, A2029, A2052, A2142, A2390, A2626, MRC 0116+111, MS
1455.0+2232, Ophiuchus, Perseus, PKS 0745-191, RX J1347.5-1145, and RX
J1720.1+2638 [10-18]. We point out that none of these objects
simultaneously satisfies all three of the stringent criterion used for
selecting our two targets, highlighting that we have identified mini-halo
candidates which have the strongest features of other mini-halos. Further,
systematic low-frequency searches have yielded few confirmed mini-halos
[19,20]. Four mini-halo candidates ($z=0.3 \dash 0.5$) were recently
identified with GMRT using a similar NVSS-VLSS radio-selected sample
of steep spectrum sources [21]. However, while the GMRT sources are
associated with clusters, none of them are detected X-ray sources,
giving $L_X \la 4 \times 10^{44} ~\lum$. No cD emission lines are
detected in three, and there is no spectroscopy for one. So while
radio-selected samples exist, their utility in defining candidates for
X-ray follow-up is limited.

The paucity of known mini-halos gives credence to the idea that they are
highly transient or require very specific ICM conditions to
form. While a variety of models have emerged attempting to explain mini-halos
[\eg\ 22-24], the lack of well-studied mini-halo systems inhibits refinement
of these models using observational constraints. Additionally, if
simulations incorporating AGN feedback are to reproduce the range of
non-thermal sources observed in clusters and yield insight to their
importance for structure formation \& evolution, then better
observational constraints must be achieved. To this end, detailed
X-ray analysis of mini-halo candidate systems is vital as it enables
diagnosis of cluster dynamics (\eg\ with substructure like cold
fronts), AGN energetics (\eg\ via cavities and shocks), and how these
correlate with diffuse non-thermal emission, in ways optical and radio
observations are incapable [\eg\ 25]. Joint high-resolution X-ray and
radio study of the ICM in clusters hosting radio halos is required to
better understand the link between thermal \& non-thermal ICM
components and to reveal the physics responsible for particle
(re-)acceleration in diffuse radio halos.

ICM cold fronts (CFs) may be an especially useful tool for
understanding the connection between a cool core and mini-halo. In some mini-halo
models, ICM bulk motions \& turbulence are responsible for the
re-acceleration of fossil electrons which emit the diffuse synchrotron
emission of a mini-halo [\eg\ 22,24,26]. Interestingly, turbulence and bulk
motions (\eg\ gas sloshing) induced by a subsonic merger event in a
dense cluster core are the same processes which excite CFs [27]. A CF
is detected as a constant pressure contact discontinuity where
downstream gas density and temperature are higher \& lower,
respectively, than the upstream counterparts. That CFs appear to be
long-lived, in spite of cool, high-density gas being co-spatial with
gas sometimes twice as hot, indicates suppression of conduction at the
CF face by \bmag-fields which are likely draped over the front during
its formation [28]. Thus, the properties of a CF (\ie\ size,
thickness, contrast with ambient medium) provide a means for studying
cool core \bmag-field strengths and configurations. If the
\bmag-fields which define a CF are the same fields that produce mini-halos,
then the study of one illuminates the other. Indeed, there are
examples of mini-halo \& CFs being present in the same system with
indications they are physically related [10,18,29-31]. If more
examples like this could be found, a task which requires deep X-ray
observations, then the cool core-CF-mini-halo relationship could be explored
more thoroughly.\\

\noindent{\bf{Merger Induced mini-halos in A2675 \& Z808?}} In
Fig. \ref{fig:img} note the large companion galaxies within 15 kpc of
both cDs and that the radio emission is elongated along the axis
including the companions. In both clusters the optical, X-ray, and
radio emission are significantly off-set from each other, suggesting
the cD is moving relative to the ICM. Also note the peculiar Z808 1.4
GHz morphology: sharp edge to NE of core, hole NW of core, radio plume
trailing galaxy SW of cD, compressed ``sandwich'' appearance. All
these features indicate recent merger activity. If so, are there CFs,
shocks, or cavities in these systems which would illuminate the
connection between the cool core, the mini-halo, and previous AGN feedback?
Is the mini-halo power correlated with the cooling luminosity as is expected
for models of particle re-acceleration via magnetohydrodynamic (MHD)
turbulence? Is there evidence of ICM turbulence in the X-ray emission
(\eg\ eddies or twisted wake like in A520)? Might there be powerful
cavities in the X-ray halos of these clusters which can be used to
measure the energetics of an AGN outburst and constrain how
relativistic plasma is transported in the core? Spurred along by these
questions, we propose \chandra\ X-ray observations of 80 ks for A2675
and 100 ks for Z808. We seek to (1) determine if a merger has taken
place as evidenced by CFs or shocks, (2) probe for signs of AGN
activity like cavities, (3) evaluate the connection of the steep
spectrum radio source to the X-ray properties, and (4) model the A2675
\& Z808 mini-halos using existing theories to constrain the magnetic field
properties and particle content as they relate to the cool core and
AGN.\\

\noindent{\bf{Deep X-ray Data for a Radio Halo?}} The physics of how
mini-halos are formed may be encoded in the X-ray emission of the ICM. The mini-halo
model of Gitti et al. (2004) employs MHD turbulence frozen into the
gas of the cool core region to re-accelerate fossil electrons. To
evaluate the Gitti model requires measurement of core properties
derived from X-ray data, specifically the cool core radius ($r_c$),
scaled electron gas density ($n_c$), and temperature structure
(\tx). The turbulence model of Kunz et al. (2010) and shear model of
Keshet et al. (2009) also require these parameters be known. The
energy scale where mini-halo synchrotron and inverse Compton (IC) losses are
balanced by re-acceleration, $\gamma_b$, are related to these
quantities by $\gamma_b \propto r_c^{0.8} n_c^{-1}$. If an AGN is
responsible for the mini-halos, then the time for radio plasma to reach the
edge of the mini-halo should be less than the plasma synchrotron lifetime,
$t_{\mathrm{sync}}$. Buoyancy and sound crossing time arguments are
useful in constraining the time a plasma takes to move about the ICM
[2], and these calculations are based on X-ray measurements ($t_{c_s}
\propto \tx^{-1/2}$ and $p \propto n_c \tx$). Further, constraints on
the ICM turbulent energy density, turbulent lengthscale, and diffusion
coefficient are required, quantities which are $n_c$ and
\tx\ dependent. If the cooling luminosity is much larger than the
synchrotron power, then at a minimum, cooling flow powered turbulence,
$P_{\mathrm{CF}}$, could drive re-acceleration. Since $P_{\mathrm{CF}}
\propto \mdot$, this can only be constrained using high-SN X-ray
spectral analysis of the core. We cannot neglect the possibility that
merger shocks have powered the mini-halo. In which case the energy released
in the shock, determined from shock morphology and \tx\ \& $n_c$
discontinuities, can be used to constrain the energy spectrum of Fermi
accelerated electrons. Because of IC and synchrotron losses, the
energy spectrum implies a $t_{\mathrm{sync}}$, which, when set in the
context of the mini-halo morphology, determines if the halo could be powered
by a shock. We will also measure radial properties of the ICM: gas
mass, gravitating mass, entropy, pressure, cooling time, effective
conductivity (if a strong $T(r)$ gradient is discovered), and inferred
magnetic suppression (to prevent the rapid destruction of a strong
$T(r)$ gradient).\\

\noindent{\bf{Request for Observations:}} Our time requests are aimed
at reaching temperature and density uncertainties necessary for
significant detections of a typical CF, weak shock, cavities, and to
collect sufficient counts for radial \& spatial mapping of ICM
structure. Use of ACIS-S versus ACIS-I results in 40-50\% more total
counts, and the ACIS-S3 FOV encloses a cluster-centric radius of
$R_{2000}$, which is far enough out to probe for large-scale shocks or
CFs. Using $\beta$-models fitted to the survey \rosat\ imaging data,
Cycle 12 ACIS-S count rates were determined from PIMMS. 5,000 mock
surface brightness (MSB) profiles extending to $R_{2000}$ were then
generated via a Monte Carlo. Shallow baseline temperature and
abundance profiles were also created, $T(r) \propto r^{0.2}$ \& $Z(r)
\propto r^{-0.2}$, with normalizations $T(R_{2000}) = \tcl$ \&
$Z(R_{2000}) = 0.3 ~\Zsol$. An isothermal core was also used,
$T(r<R_{7500}) = T(R_{7500})$. For each MSB profile, cumulative counts
profiles were created for exposure times ranging 40-140 ks in 20 ks
steps. Bins with 2,500 counts were defined, and mean \tx\ and
abundance were calculated for each bin from the gradient profiles. A
simulated spectrum was generated in \xspec\ for each bin using an
absorbed thermal model (\mekal), the corresponding exposure time, and
a normalization chosen so the spectral count rate matched the count
rate predicted by the MSB profile for that bin. Source images
consistent with these parameters were created in \idl, and mock
\chandra\ images were created using MARX. A fixed background composed
of the PIMMS and RASS R12 \& R45 emission was included in all
analysis.

Radial profiles were extracted from the simulated data to estimate the
mean uncertainties for each exposure time. We find exposure times of
80 ks for A2675 and 100 ks for Z808 return the best uncertainties per
unit time while maintaining the highest signal-to-noise such that
weighted Voronoi tessellation maps with a minimum binned spatial
resolution of $\sim 5 \dash 10\arcs$ can be created. For A2675, the
mean uncertainties will be $\Delta T_X \pm 0.3 \dash 0.6$ keV and
$\Delta n_c \pm 10\%$, and for Z808, $\Delta T_X \pm 0.5 \dash 0.8$
keV and $\Delta n_c \pm 12\%$. Our simulations indicate we will be
sensitive at $> 2\sigma$ to CFs with temperature jumps $> 1.5$ and
density decreases $> 1.3$, which are typical values for
CFs. Conversely, based on the Rankine-Hugoniot jump conditions, we
should detect a $M > 1.2$ shock at $\ga 2\sigma$. We will also be able
to measure \bmag-field strengths in the CF analysis to $\approx
3\mu$G, well below the level needed for discerning between mini-halo
formation models.

We simulated the presence of cavities in the ICM of our targets by
carving out voids in the source images and creating new mock
observations with MARX. Cavity decrements, their geometries, and
distances from the cluster core span a wide-range of values, thus we
placed voids, obeying the radially dependent axial ratio and SB
dimming relations presented in [5], at a variety of radii. For
spherical plane-of-the-sky cavities, our observations should be
sensitive to $> 10\%$ decrements at $r < 100 ~\kpc$ and $> 30\%$
decrements at $100 < r < 200$ with no detections expected beyond. Our
mock observations loosely indicate we will be sensitive to cavity
powers of $10^{41 \dash 46} ~\lum$.\\

\small
\noindent \input{minishort.bbl}
\normalsize

\clearpage
\noindent{\bf{Previous \chandra\ Programs:}}\\

Co-I Edge, GO Cycle 2: ``The interaction between radio galaxies and
ICM in the cores of clusters.'' Observation of two potential cavity
systems both of which have been published in larger study of Cavagnolo
et al. 2009, ApJS, 182, 12.\\

Co-I Edge, GO Cycle 4: ``Probing the mass profile of clusters to 10
kpc.'' Observation of Abell 1201 which has been published in Owers et
al. 2009, ApJ, 692, 702.\\

Co-I McNamara, GO Cycle 8: ``AGN Feedback and Galaxy Formation in
Cluster Cores.'' A study of Abell 1664 by Kirkpatrick et al. 2009,
ApJ, 697, 867; a study of RBS 797 is presented in Cavagnolo et
al. 2010 (in preparation for ApJ).\\

Co-I McNamara, GO Cycle 10 LP: ``A Deep Image of the Most Powerful
Cluster AGN Outburst.'' A study of MS 0735.6+7421 utilizing 500 ks of
data is underway.\\

PI Cavagnolo, GO Cycle 10: ``The Hyperluminous Infrared Galaxy IRAS
09104+4109: An Extreme Brightest Cluster Galaxy.'' A detailed study of
IRAS 09104+4109 is presented in Cavagnolo et al. 2010 (in preparation
for MNRAS).\\

PI Cavagnolo maintains the Archive of Chandra Cluster Entropy Profile
Tables (ACCEPT) database and is adding 84 galaxy clusters (156
observations) to the 241 clusters currently in the database. As a
result of maintaining the database, PI Cavagnolo has reduced \&
analyzed 752 CXO observations ($\sim 16$ Msec of data) along with over
80,000 spectra.

\end{document}

%% A2675:
%%   no gmrt, only vla c at 1.4 ghz
%%   vlss = 1.09 +- 0.19 Jy
%%   nvss = 0.0112 +- 0.0019 Jy
%%   alpha = -1.54 +- 0.16
%%   halo size = 100 kpc
%%   sur bri = 50 muJy/arcs2
%%   $(\alpha,\delta)$ of (23:55:42.614, +11:20:35.83)
%%   z = 0.0726
%%   DL = 322 Mpc
%%   DA = 1.359 kpc/arcsec
%%   NH = 5.03e20 cm-2
%%   Lx = 1.9e44 e/s, Tx = 4.0 keV (est from Lx-Tx relation) \cite{1998MNRAS.301..881E}
%%   \lha\ not detected by \cite{crawford99}
  
%% Z808 (RXC J0301.6+0155):
%%   no gmrt, lots of vla, but only to 1.4 Ghz
%%   vlss = 24.47 +- 4.58 Jy
%%   nvss = 0.40 +- 0.08 Jy
%%   alpha = -1.41 +- 0.18
%%   halo size = 250 kpc
%%   sur bri = 700 muJy/arcs2
%%   $(\alpha,\delta)$ of (03:01:38.0, +01:55:14)
%%   z = 0.1695
%%   DL = 817 Mpc
%%   DA = 2.895 kpc/arcsec
%%   NH = 6.9e20 cm-2
%%   Lx = 4.2e44 e/s, Tx = 6.7 keV (est from Lx-Tx relation) \cite{1998MNRAS.301..881E}
%%   \lha\ = 4.6e40 e/s \cite{crawford99}
%%   no IR up-turn past 8 microns, prob no AGN \cite{quillen08}
