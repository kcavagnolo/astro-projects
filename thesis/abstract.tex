Presented in this dissertation is an analysis of the X-ray emission
from the intracluster medium (ICM) in clusters of galaxies observed
with the \chandra\ X-ray Observatory. The cluster dynamic state is
investigated via ICM temperature inhomogeneity, and ICM entropy is
used to evaluate the thermodynamics of cluster cores.

If the hot ICM is nearly isothermal in the projected region of
interest, the X-ray temperature inferred from a broadband (0.7-7.0
keV) spectrum should be identical to the X-ray temperature inferred
from a hard-band (2.0-7.0 keV) spectrum. However, if unresolved cool
lumps of gas are contributing soft X-ray emission, the temperature of
a best-fit single-component thermal model will be cooler for the
broadband spectrum than for the hard-band spectrum. Using this
difference as a diagnostic, the ratio of best-fitting hard-band and
broadband temperatures may indicate the presence of cooler gas even
when the X-ray spectrum itself may not have sufficient signal-to-noise
ratio (S/N) to resolve multiple temperature components.

In Chapter \ref{ch:eband} we explore this band dependence of the
inferred X-ray temperature of the ICM for \ebandnuma\ well-observed
galaxy clusters selected from the {\it Chandra} X-ray Observatory's
Data Archive. We extract X-ray spectra from core-excised annular
regions for each cluster in the archival sample. We compare the X-ray
temperatures inferred from single-temperature fits when the energy
range of the fit is 0.7-7.0 keV (broad) and when the energy range is
2.0/(1+$z$)-7.0 keV (hard). We find that the hard-band temperature is
significantly higher, on average, than the broadband temperature. On
further exploration, we find this temperature ratio is enhanced
preferentially for clusters which are known merging systems. In
addition, cool-core clusters tend to have best-fit hard-band
temperatures that are in closer agreement with their best-fit
broadband temperatures.

ICM entropy is of great interest because it dictates ICM global
properties and records the thermal history of a cluster. Entropy is
therefore a useful quantity for studying the effects of feedback on
the cluster environment and investigating the breakdown of cluster
self-similarity. Radial entropy profiles of the ICM for a collection
of \entsuppnum\ clusters taken from the \chandra\ X-ray Observatory's
Data Archive are presented in Chapter \ref{ch:ent_supp}. We find that
most ICM entropy profiles are well-fit by a model which is a power-law
at large radii and approaches a constant value at small radii: $K(r) =
\kna + \khun (r/100 \kpc)^{\alpha}$, where \kna\ quantifies the
typical excess of core entropy above the best fitting power-law found
at larger radii. We also show that the \kna\ distributions of both the
full archival sample and the primary \hifl\ sample of
\citet{hiflugcs1} are bimodal with a distinct gap centered at $\kna
\approx 40 \ent$ and population peaks at $\kna \sim 15 \ent$ and $\kna
\sim 150 \ent$.

Utilizing the results of the the \chandra\ X-ray Observatory archival
study of intracluster entropy presented in Chapter \ref{ch:ent_supp},
we show in Chapter \ref{ch:harad} that \halpha\ and radio emission
from the brightest cluster galaxy are much more pronounced when the
cluster's core gas entropy is $\la 30 \ent$. The prevalence of
\halpha\ emission below this threshold indicates that it marks a
dichotomy between clusters that can harbor multiphase gas and star
formation in their cores and those that cannot. The fact that strong
central radio emission also appears below this boundary suggests that
feedback from an active galactic nucleus (AGN) turns on when the ICM
starts to condense, strengthening the case for AGN feedback as the
mechanism that limits star formation in the Universe's most luminous
galaxies.
