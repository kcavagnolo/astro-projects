%%%%%%%%%%%%%%%%%%%%%%%%%%%%%%%%%%%%%%%%%%%%%%%%%%%%%%
\section{Energy Band Dependence of X-ray Temperatures}
%%%%%%%%%%%%%%%%%%%%%%%%%%%%%%%%%%%%%%%%%%%%%%%%%%%%%%

Using a sample of \ebandnuma\ galaxy clusters we explored the band
dependence of inferred X-ray temperatures for the ICM. Utilizing
core-excised global cluster spectra extracted from the regions
encircled by $R_{2500}$ and $R_{5000}$, we inferred X-ray temperatures
for a single-component absorbed thermal model in a broadband (0.7-7.0
keV) and a hard-band (2.0-7.0 keV). On average, we found that the
hard-band temperatures were greater, with the ratio of the
temperatures, $T_{HBR}$, having a mean value $1.16 \pm 0.01$ (where
the error is of the mean) for the $R_{2500}$ apertures, and $1.14 \pm
0.01$ for the $R_{5000}$ apertures. No systematic trends were found,
in either the values or dispersions, with S/N, redshift, Galactic
absorption, metallicity, observation date, or broadband
temperature. Analysis of a simulated ensemble of 12,765
observation-specific two-component spectra revealed statistical
fluctuations could not account for the skewing measured in
$T_{HBR}$. The simulations also helped establish a lower limit on the
flux contribution from a cooler gas component ($T_X < 2.0$ keV)
necessary to generate $T_{HBR}$ as large as those found in the data. A
second cool gas phase must be contributing $\ga 10\%$ of the total
emission. The simulations also reveal the observational scatter is
larger than the statistical scatter, a result one should expect if an
underlying physical process is responsible for creating dispersion in
$T_{HBR}$.

The {\it{a priori}} motivation for studying temperature inhomogeneity
comes from the prediction of \citet{2001ApJ...546..100M} that
$T_{HBR}$ may be related to the process of hierarchical structure
formation. After assuming that the establishment and prominence of a
cool core in a cluster is an indicator of relaxation, we compared
$T_{HBR}$ values with the ``strength'' of the cool core based on the
temperature decrement between the core and the cluster atmosphere. The
result was a significant correlation between clusters having higher
values of $T_{HBR}$ being less likely to host a cool core. A search of
the scientific literature involving the clusters with the highest
significant values of $T_{HBR}$ ($T_{HBR} > 1.1$) revealed that most,
if not all, of these clusters are undergoing, or have recently
undergone, a merger event. With two strong connections between
$T_{HBR}$ and cluster dynamical state established, we conclude that
temperature inhomogeneity is most likely related to the process of
cluster relaxation, and that it may be useful as a metric to further
quantify the degree to which a cluster is relaxed, thus addressing
point (1) brought up in Section \S\ref{sec:ssbreak}.

%%%%%%%%%%%%%%%%%%%%%%%%%%%%%%%%%%%%%%%%%%%%%%%%%%%%%%%%%%%%%%%%%%
\section{Chandra Archival Sample of Intracluster Entropy Profiles}
%%%%%%%%%%%%%%%%%%%%%%%%%%%%%%%%%%%%%%%%%%%%%%%%%%%%%%%%%%%%%%%%%%

A library of ICM entropy profiles was created for \entsuppnum\ galaxy
clusters taken from the \chandra\ Data Archive to better understand
the role of feedback and cooling in shaping global cluster
properties. Radial gas density, $\nelec(r)$, and gas temperature,
$kT_x(r)$, profiles were measured for each cluster. Radial entropy was
calculated from the relation $K(r)=kT_X(r)\nelec(r)^{-2/3}$. The
uncertainties for each profile were calculated using 5000 Monte Carlo
realizations of the observed surface brightness profile. Each profile
was then fit with two models: one which is a power-law at all radii
(\ref{eqn:plaw}), and another (eqn. \ref{eqn:plaw}) which is a
power-law at large radii but approaches a constant \kna\ value at
small radii. The \kna\ term is defined as the core entropy.

Comparison of p-values, \chisq, and the significance of \kna\ above
zero for the best-fit models revealed that for 90\% of the
\entsuppnum\ sample, the model with a constant core entropy is a
better description of the data. Systematics such as PSF smearing,
angular resolution, profile curvature, and number of radial bins
proved not to be important in setting or changing best-fit \kna. The
slope of the power-law component was also found to be remarkably
similar among the profiles with a mean value of $1.21 \pm 0.39$ which
is not significantly different from the value of $\sim1.1$ expected
from hierarchical structure formation.

The distribution of \kna\ for both the \accept\ and \hifl\ samples was
found to be bimodal. The populations comprising the bimodality are
strikingly similar between the two samples with peaks at $\kna \sim 15
\ent$ and $\kna \sim 150 \ent$. The KMM test \citep{kmm1,kmm2} was
applied and it determined, for both \accept\ and \hifl, that the
populations were statistically distinct and that a unimodal
distribution was ruled out. The poorly populated region between the
populations for both samples occurred at $\kna \approx 30-60
\ent$. The measured entropy profile shapes and distribution of \kna\
were consistent with existing models of AGN feedback which predict
non-zero core entropy. All of the results and data for \accept\ were
made available to the public with the intent that theorists and
observers might find utility for
\accept\ in their own work. The work presented in Chapter
\ref{ch:ent_supp} directly addressed point (2) of Section
\S\ref{sec:cfprob}.

%%%%%%%%%%%%%%%%%%%%%%%%%%%%%%%%%%%%%%%%%%%%%%%%%%%%%%%%%%%%%%%%%%%%
\section{An Entropy Threshold for Strong \halpha\ and Radio Emission
in the Cores of Galaxy Clusters}
%%%%%%%%%%%%%%%%%%%%%%%%%%%%%%%%%%%%%%%%%%%%%%%%%%%%%%%%%%%%%%%%%%%%

To study a suspected connection between low entropy gas in cluster
cores and the by-products of cooling, namely AGN activity and
formation of thermal instabilities, the best-fit \kna\ values from
clusters in \accept\ were compared against radio power and \halpha\
luminosity, both strong indicators of run-away cooling. A search of
the research literature turned-up \halpha\ observations for 110
clusters in \accept. The NVSS and SUMSS all-sky radio surveys were
queried for each cluster to attain \radpow, either detections or
upper-limits. New luminosities were then calculated using the
preferred cosmology assumed in this dissertation to place all
observations on equal footing.

A comparison of \halpha\ luminosities and best-fit \kna\ values showed
a strong relation between when \halpha\ emission is detected and when
it is not. Below an entropy threshold of $\kna \la 30 \ent$ \halpha\
emission is predominantly on, while above this threshold it is always
off sans the exception of one cluster very near the $\kna = 30 \ent$
boundary. A very similar correlation was found between \radpow\ and
\kna\ for clusters at $z < 0.2$. A redshift cut was applied because of
the low resolution of NVSS and SUMSS. The entropy threshold for
\radpow\ also occurs at $\kna \approx 30 \ent$ but with a larger
fraction of low power radio sources above $\kna \approx 30 \ent$ than
the fraction which was found in \halpha. However, it was found that
powerful radio sources ($\radpow > 10^{40} \flux$) were only found in
clusters with $\kna \la 30 \ent$ adding strength to the argument that
entropy sets a scale for development of a multiphase medium in cluster
cores. While the discussion is not presented in this dissertation,
\cite{conduction} propose that it is electron thermal conduction which
gives rise to the entropy threshold observed in our study. The work
presented in Chapter \ref{ch:harad} is an extension of point (2) in
\S\ref{sec:cfprob}.

