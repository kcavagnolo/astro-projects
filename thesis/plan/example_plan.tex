\documentclass[12pt]{plan}
\usepackage{psfig}
\usepackage{macros_desai}

\setlength{\topmargin}{-0.25in}
\setlength{\oddsidemargin}{-0.1in}
\setlength{\evensidemargin}{0in}
\setlength{\textwidth}{6.7in}
\setlength{\headheight}{0in}
\setlength{\headsep}{0in}
\setlength{\topskip}{0.5in}
\setlength{\textheight}{9.25in}
%\renewcommand\baselinestretch{1.62}

\pagestyle{myheadings}

\markright{\hspace*{\fill} {\it Research Proposal, Vandana Desai} \hspace{10mm}}
\setcounter{page}{6}

\begin{document}

%%%%%%%%%%%%%%%%%%%%%%%%%%%%%%%%  TITLE  %%%%%%%%%%%%%%%%%%%%%%%%%%%%%%%%%%%%%%
%\vspace*{-0.4in}
\begin{center}
\LARGE
%\vspace{-1mm}
{\bf The Environmental Dependence of Galaxy Star Formation Rates}
\normalsize
\end{center}
%\vspace{-3mm}
%%%%%%%%%%%%%%%%%%%%%%%%%%%%%%%%%%%%%%%%%%%%%%%%%%%%%%%%%%%%%%%%%%%%%%%%%%%%%%

Within the widely-accepted cosmological framework, the mass density of
the universe is dominated by cold dark matter.  Protogalaxies are
formed when normal baryonic gas falls into dark matter potential
wells, cooling and forming stars at the centers of dark matter halos.
These halos, along with their gaseous and stellar content, merge to
form larger galaxies, which in turn merge to form groups, clusters,
and filaments.

The buildup of stellar mass within this framework can be tested
observationally through the evolution of the
ultraviolet %\cite{Lilly96,Madau96,Madau98,Cowie99,Steidel99,Wilson02}
and
near-infrared
%\cite{Cole01,Kochanek01,Cowie96,Cohen99,Cohen02,Pozzetti03}
galaxy luminosity functions.  The ultraviolet emission of galaxies is
dominated by young, massive stars, and therefore permits measurements
of the ongoing star formation rate (SFR).  Near-infrared emission is
dominated by old, low-mass stars, providing an indication of the
amount of stars already formed.  Deep \emph{HST} imaging has formed
the basis of much of the high redshift work in this
area\cite{Cohen99,Cohen02,Madau96,Madau98,Steidel99}, providing
important constraints on the processes driving star formation.  In
particular, although standard, ``quiescent'' star formation models
within a hierarchical framework can successfully reproduce local star
formation rates, they underpredict the number of massive starforming
galaxies at high redshift and the stellar mass already in place by a
redshift of two\cite{Fontana99,Fontana03,Pozzetti03,Cimatti02}. Star
formation models that include starbursts triggered by galaxy
mergers
%\cite{Barton00,Lambas03,Mihos94,Mihos96,Tissera00,Tissera02} 
have been more successful in matching
the ultraviolet luminosity function at $z=2-3$, but
fail to produce enough stars at high redshift to match the infrared
data at $z \sim 2$.\cite{Somerville01,Cimatti02} Recent models including starbursts
from both galaxy mergers \emph{and} encounters that do not lead to
bound mergers (fly-by events) successfully reproduce both the
ultraviolet and infrared data without altering the agreement with
observations at low redshifts.\cite{Menci03}

In addition to the redshift dependence of star formation, there is
growing observational evidence for an environmental dependence. In
particular, star formation in the densest environments, namely galaxy
clusters, appears to be suppressed in comparison to the field
\cite{Couch87,Abraham96,Poggianti99,Balogh99,Balogh00b,Couch01,Postman01,Balogh02a,Pimbblet02}.
The suppression is strongest in cluster cores, with star formation
increasing systematically with increasing distance from the cluster
center.  Recent studies involving large samples of galaxies at low
redshift have provided important clues as to which mechanisms drive
the SFR-density relation\cite{Lewis02,Gomez03}.  First,
these studies reveal that the suppression of SFR persists at
relatively low densities and large distances from clusters, where cluster-specific processes
are not important.  Second, the star formation properties of galaxies
appear to be more dependent upon the large-scale (5 Mpc) density than
they are upon the mass of the group or cluster in which they are
embedded\cite{Balogh03a}.  The former is closely linked to the
formation time of a galaxy, while the latter is
linked to current processes within the group or cluster.  The
observations imply that the star formation properties of galaxies are
more closely tied to \emph{when} a galaxy formed, rather than
\emph{current} environmental effects.  Third, these observations
reveal that suppressed regions contain a lower fraction of starforming
galaxies, rather than galaxies with systematically lower SFRs compared
to the field.  This fact is evidence against models in which the
environment induces a slow decrease in a galaxy's SFR.

%Adding support to this conjecture is the observation that the only subset of galaxies experiencing an enhancement of SFR over the field
%are close pairs \cite{Balogh03a}.

A proper treatment of starbursts associated with galaxy mergers and
fly-by events can reproduce the observed evolution of star formation
within galaxies.  Can it also reproduce the observed environmental
dependence of star formation?  Galaxies presently found in high
density environments may have undergone more interactions in the
distant past.  Galaxies have access to only a limited supply of gas
from which to form stars, and thus those that have undergone multiple
interactions may quickly exhaust their gas reservoir, leading to a
decline in SFR.  Although galaxy-galaxy interactions at high redshift
could plausibly result in the observed relationship between SFR and
density observed in the local universe, it is a scenario that has yet
to be tested in detail.  The first goal of this proposal is to perform such a test using existing dark
matter simulations in combination with simple models that treat
starbursts due to both mergers and fly-by events.  My approach is
described in greater detail in the next section.

In the scenario described above, the present-day SFR-density relation
is the result of interactions at an early epoch, before the galaxies are
incorporated into larger structures.  In this case, the threshold
local galaxy density at which SFRs begin to correlate with environment
should systematically increase with decreasing redshift.  In contrast,
if instead the SFR of a galaxy depends on its current local
environment, but not its interaction history, the threshold density
should remain constant with redshift.  Quality observations at high
redshift offer the best opportunity for measuring the evolution of the
SFR-density relation, thereby discriminating between these models.
The second goal of my proposed program is to make such observations, described
in detail in the final section.

%%%%%%%%%%%%%%%%%%%%%%%%%%%%%%%%  TITLE  %%%%%%%%%%%%%%%%%%%%%%%%%%%%%%%%%%%%%%
%\large
%\vspace{-4mm}
%\begin{center}{\bf \underline{Theoretical Program:  Treating Starbursts in Simulations}} \end{center}
%\normalsize
%\vspace{-4mm}
%%%%%%%%%%%%%%%%%%%%%%%%%%%%%%%%%%%%%%%%%%%%%%%%%%%%%%%%%%%%%%%%%%%%%%%%%%%%%%

\vspace{-3mm}
\subsection*{Theoretical Program:  Modelling Starbursts in Simulations}
\vspace{-2mm}
In the simulation I propose to utilize, the dark matter halos of galaxies within a 70
Mpc box are represented by 80 million particles\cite{Reed03}.  This simulation is
able to resolve galaxies smaller than the Magellanic Clouds down to
the center of galaxy clusters.  The final output contains tens of
thousands of halos, so I will be able to achieve a statistical
significance similar to the largest existing observational studies
\cite{Balogh03a}.

I will model quiescent star formation using procedures developed in
standard semi-analytic models of galaxy
formation\cite{Somerville99,Cole00,Menci02}.  In these models, star formation
is driven by both the cooling of hot gas and by the refueling of cold
gas during mergers.  In addition to this quiescent mode of star
formation, I will model the starbursts associated with galaxy mergers
and fly-by events following the derivation of the fraction of cold gas
destabilized by encounters put forth by Cavaliere and
Vittorini\cite{Cavaliere00,Menci03}.  The spectral energy distribution
(SED) of a galaxy can be modelled by convolving the SFR with the SED
of a single-aged stellar population\cite{Bruzual03}. I will compare the resulting
dependence of SFR with observations\cite{Lewis02,Gomez03,Balogh03a} of
local galaxies, ``observing'' the simulation with a suitable projection of positions and velocities in order to mimic the
procedure adopted by observers.

In addition to its redshift and environmental dependences, star
formation depends upon galaxy mass.  The average starforming
galaxy was more massive in the past, an
effect known as {\em cosmic downsizing}\cite{Cowie99,Kauffmann03}.  This effect has also been
observed in clusters, which show evidence of luminous post-starburst
galaxies at intermediate redshifts that are absent in the nearby Coma
cluster \cite{Poggianti03}.  The most straightforward explanation is
that the mass of actively starforming galaxies falling into clusters
decreases at low redshifts.  Can we understand downsizing in the
context of galaxy-galaxy interactions?  If high mass galaxies have on
average undergone more interactions, they may have depleted all of
their gas by the current epoch, while low mass galaxies have not.  The
simulation described above will allow me to explore the connection
between interaction history and galaxy mass and how such a connection
might explain cosmic downsizing.
 
%%%%%%%%%%%%%%%%%%%%%%%%%%%%%%%%  TITLE  %%%%%%%%%%%%%%%%%%%%%%%%%%%%%%%%%%%%%%
%\large
%\vspace{-4mm}
%\begin{center}{\bf \underline{Observational Program:  Star Formation Rates at High Redshift}} \end{center}
%\normalsize
%\vspace{-4mm}
%%%%%%%%%%%%%%%%%%%%%%%%%%%%%%%%%%%%%%%%%%%%%%%%%%%%%%%%%%%%%%%%%%%%%%%%%%%%%%
\label{sec:observations}

\vspace*{-3mm}
\subsection*{Observational Program:  Star Formation Rates at High Redshift}
\vspace{-2mm}

Past studies of the SFR in distant galaxies suffer from two
limitations which prevent their use in measuring the SFR-environment
correlation.  First, they have not included low-density regions.
Second, they commonly utilize [OII] emission lines to measure the SFR.
Although it is easily detectable at high redshift ($z \sim 1$), [OII]
is not an ideal diagnostic for SFR \cite{Jansen01,Charlot01}. The
strength of the [OII] 3727$\lambda$ line is not directly coupled to
the ionizing luminosity of young stars, and it is sensitive to
variations in the metallicity and the ionization state of the gas.
More importantly, it is strongly affected by dust
\cite{Blain99,Flores99,Smail99,Poggianti00,Fadda00,Duc02}.  Past
studies may therefore have a highly biased view of the SFRs in distant
galaxies.  For example, several spectroscopic surveys of distant
clusters \cite{Couch87,Fabricant91,Fisher98,Dressler99,Poggianti99}
have revealed a population of galaxies with spectra displaying strong
Balmer absorption but lacking emission lines.  These objects have been
interpreted as galaxies that are not currently forming stars, but have
recently undergone a starburst \cite{Couch87,Poggianti99}. If this
interpretation is correct, the starforming progenitors of these
post-starburst galaxies have yet to be identified \cite{Couch01}.  An
alternate interpretation is that they are galaxies in which star
formation is totally obscured at optical wavelengths \cite{Smail99}.

My observational plan is designed to remedy both of these problems.  I
will map the SFR out to the low density regions (6R$_{\rm virial}$)
surrounding CL1216, one of the clusters I have been studying as part
of an $HST/ACS$ survey of ten $z \sim 0.8$ clusters (see Figure
\ref{cl1216}).  I will use mid- to far-infrared data obtainable with
\emph{SIRTF} to calculate robust SFRs that are unaffected by dust.  My
proposed \emph{SIRTF} observations, including overheads, can be
completed in 9 hours, well within the time for a reasonable program. I
will simultaneously measure redshifts and [OII] line strengths using
the Deep Extragalactic Imaging Multi-Object Spectrograph ({\em
DEIMOS}), installed on Keck II.  Accounting for observational
overheads, in two nights I can spectroscopically confirm $\sim$600
galaxies that are both near the cluster redshift and within $\sim$6
virial radii of CL1216.

The proposed observations nicely complement other works in progress.
In particular, Treu et al.\cite{Treu03} are conducting an optical
spectroscopic survey of galaxies up to 5 Mpc from the center of
CL0024+16 at $z$=0.4, for which they already have complementary {\em
HST/WFPC2} imaging.  Their study provides an intermediate redshift
sample, in addition to the zero redshift baseline discussed in the
introduction, with which to compare my high redshift results.  In
particular, my proposed optical spectroscopy will yield [OII]-derived
SFRs, which can be compared directly with those measured at $z$=0.4.

The DEEP2 survey will also return a wealth of information on the star
formation properties of galaxies at $z \sim 1$ \cite{Davis03}.  One
of their survey fields, the Groth Strip, will be particularly
well-studied, with mid- to far-infrared data from {\em SIRTF}, {\em
HST} imaging, groundbased $BRI$ imaging, and X-ray data.  My proposed
data set will be of similar quality and wavelength coverage, but for a
region surrounding a rich cluster.  This new dataset will allow a fair
comparison between cluster and field environments at similar
redshifts.

\vspace*{-3mm}
\subsubsection*{Strategy for Proposed \emph{SIRTF} Observations}
\vspace{-2mm}

The Multiband Imaging Photometer for \emph{SIRTF} (\emph{MIPS})
simultaneously collects mid- to far-infrared data at three narrow
bandpasses centered at 24 $\micron$, 72 $\micron$, and 160 $\micron$.
Studying a sample of 156 local galaxies, Papovich \etal\
(2002)\cite{Papovich02} found that the total infrared fluxes (8--1000
$\micron$) of $z\sim$ 1 galaxies can be constrained to within a factor
of 2.5 using 24 $\micron$ \emph{MIPS} data.  We therefore optimize our
observations for the 24 $\micron$ band, while keeping in mind that the
addition of 70 $\micron$ data will help to improve this constraint
by a factor of $\la$6.

In a study of three clusters at $z$=0.76, $z$=0.9, and $z$=0.92,
Postman \etal\ (2001)\cite{Postman01} used [OII] to measure the mean
SFRs of cluster members as a function of redshift.  Interpolating
their results, I expect a mean [OII]-derived SFR of 1.5 ${\rm
M}_{\odot} {\rm yr}^{-1}$ for cluster members at $z$=0.8, and higher
values in the surrounding regions.  The ratio between IR-derived SFRs
and [OII]-derived SFRs has been found to range from 10--100
\cite{Duc02}.  Thus, I expect an average IR-derived SFR of more than
15 ${\rm M}_{\odot} {\rm yr}^{-1}$.  In order to ensure high
signal-to-noise photometry on galaxies with SFRs significantly below
this average, I have calculated the exposure time necessary to detect
a galaxy forming stars at a rate 15 times smaller than this average.

According to Bell (2003)\cite{Bell03}, a SFR of 1 ${\rm M}_{\odot}
{\rm yr}^{-1}$ corresponds to a total IR luminosity of $\log_{10}$
(L$_{{\rm TIR}}$/L$_{\odot}$) $\sim$ 9.8.  The flux in the 24
$\micron$ band is typically a factor of 0.16L$_{{\rm TIR}}$
\cite{Papovich02}.  Thus, in order to detect galaxies with SFRs as low
as 1 ${\rm M}_{\odot} {\rm yr}^{-1}$, our IR data must be sensitive to
fluxes as low as $\log_{10}(L_{24 \micron} / L_{\odot}) \sim 9$.
Assuming the concordance cosmology, an object at this luminosity at
$z=0.8$ would have a flux of F$_{24 \micron} \sim$ 50 $\mu {\rm Jy}$.
Galaxies smaller than 15'' (113 kpc at $z$=0.8) will be spatially
unresolved in the 24 $\micron$ {\em MIPS} band.  Using SPOT, we
estimate that the background in the field of CL1216 falls in the
``medium'' category.  The integration time required for a 1$\sigma$
detection of a 50 $\mu {\rm Jy}$ point source with a medium background
is $\sim$400 seconds\footnote{{\tt
http://sirtf.caltech.edu/SSC/mips/sensmedgifs/24scanmed.gif}}.
Because L$_{{\rm TIR}}$ is a steep function of SFR, the
signal-to-noise ratio (SNR) increases quickly with SFR.  A 400 second
exposure time will therefore provide adequate SNRs for galaxies
forming stars well below the average expected IR-derived rate of 15
${\rm M}_{\odot} {\rm yr}^{-1}$ (see Figure \ref{sfr}).  Although both
the expected flux and sensitivity are higher at 70 $\micron$,
local confusion will limit its utility\footnote{{\tt http://sirtf.caltech.edu/SSC/mips/sensmedgifs/70scanmed.gif}}.  I therefore expect to have
usable 70 $\micron$ data for a bright subset of the galaxies detected
at 24 $\micron$.

Since I wish to cover a relatively large area of the sky (0.5$^\circ
\times$ 0.5$^\circ$), I will use the \emph{MIPS} Scan Map AOT.  The
exposure time per pixel resulting from a single scan at a medium scan
rate is 40 seconds at both 24 $\micron$ and 70 $\micron$.  A single map
cycle consisting of 6 scan legs with a cross scan step of 302''
(minimal overlap) and scan length of 0.5 degrees should cover the desired area with an exposure time of 40 seconds per pixel (see Figure
\ref{visual}).  In order to attain a total exposure time of 400
seconds per pixel, I propose 3 AORs consisting of 3 map cycles each
and 1 AOR consisting of just 1 map cycle.  Using SPOT, I estimate that
this program, including overheads, can be executed in (3 $\times$ 161
min) + 56 min = 539 min $\sim$ 9 hours.

\vspace{-3mm}

\subsubsection*{Strategy for Proposed {\em DEIMOS} Observations}

\vspace{-2mm}

Because of its high throughput and its ability to take spectra of 140
objects simultaneously, {\em DEIMOS} is the ideal instrument with which to execute my spectroscopic program.  Based upon spectra the EDisCS collaboration has
already obtained for galaxies in the center of CL1216, I need 1 hour
to acquire spectra with signal to noise suitable for redshift and
[OII] linewidth determination.  {\em DEIMOS} has a $9' \times 9'$
field of view.  Thus, in 9 hours, I can tile the 0.5$^\circ \times$
0.5$^\circ$ area around CL1216.  Using photometric redshift
pre-selection, $\ga$50$\%$ of spectroscopic targets should lie within
the redshift range in which I am interested.  Thus, accounting for
observational overheads I could spectroscopically confirm $\sim$600
galaxies within both my target redshift range and $\sim$6 virial
radii of CL1216 in less than two nights.

%If {\em SIRTF} time proves difficult to obtain, I will fall back on
%[OII]-derived SFRs, which are standard for high redshift galaxies.  If
%I fail to procure Keck time, I can use photometric redshifts
%combined with the IR data to accomplish our science goals.
 
\clearpage

\begin{figure}
  \centerline{
    \begin{minipage}[l]{5.5in}
      %\epsscale{1}
      %\plotone{cl1216.center.ps}
      \psfig{figure=cl1216.center.ps,width=5.435294118truein,height=4.2truein}
    \end{minipage} \ \hfill \
    \begin{minipage}[l]{2in}
      \caption{\footnotesize\baselineskip0.1cm{ The image at left shows a $\sim$0.43 Mpc $\times$ 0.56 Mpc region
around the center of CL1216, imaged in \emph{F814W} with the $ACS/WFC$ on
board the $HST$.  CL1216 is one of the high redshift clusters from the
ESO Distant Cluster Survey (EDiSCS).  The EDiSCS collaboration, of
which I am a member, has collected an extensive data set for the inner
1.5 Mpc $\times$ 1.5 Mpc region of the cluster, including groundbased
$BVRIK$ imaging, {\it F814W HST/ACS} imaging, and optical
spectroscopy.  A larger 13.5 Mpc $\times$ 13.5 Mpc region around the
cluster has been imaged in $VRI$ (sufficient for photometric redshift
estimation) and in the X-ray with {\em XMM-Newton}.  Our proposed
observations would add \emph{SIRTF} mid- to far-infrared data as well
as \emph{DEIMOS} spectroscopy in this larger region. \label{cl1216}
      }}
    \end{minipage} 
  }

\end{figure}
%----------
\begin{figure}
  \centerline{
    \begin{minipage}[l]{3in}
      %\epsscale{1}
      %\plotone{sfr.ps}
      \psfig{figure=sfr.ps,width=3.5truein,height=3.5truein}
       \caption{\footnotesize\baselineskip0.1cm{Signal-to-noise ratio (SNR) expected in the 24 $\micron$ band, as a function of SFR.  The average IR-derived SFR for cluster members is expected to be 15 ${\rm M}_{\odot} {\rm yr}^{-1}$, where the plot cuts off. \label{sfr}}}
    \end{minipage} \hfill \
    \begin{minipage}[l]{3in}
      \psfig{figure=both.color.ps,width=3truein,height=3truein}
  \caption{\footnotesize\baselineskip0.1cm{SPOT-generated visualization of the proposed 24 $\micron$ (turquoise) and 70 $\micron$ (pink) \emph{MIPS} observations overlaid on a $2^{\circ} \times 2^{\circ}$ ISSA image of a field centered on CL1216.  This figure is available in color at {\tt http://www.astro.washington.edu/desai/SIRTF}.\label{visual}}}
    \end{minipage}
  }
\end{figure}

\clearpage

\bibliographystyle{plan}
\bibliography{plan}

\end{document}
