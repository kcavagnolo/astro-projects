% An outline of the dissertation with the specific status of each
% chapter and projected completion dates (1 to 2 pages).
\documentclass[12pt]{plan}
\usepackage{psfig}
\usepackage{macros_desai}
\setlength{\topmargin}{-0.25in}
\setlength{\oddsidemargin}{-0.1in}
\setlength{\evensidemargin}{0in}
\setlength{\textwidth}{6.7in}
\setlength{\headheight}{0in}
\setlength{\headsep}{0in}
\setlength{\topskip}{0.55in}
\setlength{\textheight}{9.25in}
\pagestyle{myheadings}
\renewcommand{\labelenumi}{\arabic{enumi}.}
\renewcommand{\labelenumii}{\arabic{enumi}.\arabic{enumii}}
\renewcommand{\labelenumiii}{\arabic{enumi}.\arabic{enumii}.\arabic{enumiii}}
\renewcommand{\labelenumiv}{\arabic{enumi}.\arabic{enumii}.\arabic{enumiii}.\arabic{enumiv}}
\markright{\hspace*{\fill}{\it Dissertation Outline, Kenneth W. Cavagnolo}\hspace{10mm}}

\begin{document}
\begin{center}
\LARGE
\vspace{1.5mm}
{\bf Feedback, Evolution, and Dynamics in Clusters of Galaxies}\\
\vspace{1.5mm}
\end{center}
\small

\label{sec:intro}
\subsubsection*{Introduction}
Clusters of galaxies are the most massive objects to have yet formed
in the Universe. They cover a mass scale of 10$^{14}$ to 10$^{15}$ Solar
masses, contain hundreds to thousands of galaxies, and are 8-15
million light years in size. Clusters of galaxies form through a hierarchical
process: gas condenses into protogalaxies in which star clusters form,
these protogalaxies organize into well-defined galaxies, and these
galaxies gravitate together and form clusters. During the merger of galaxies to
form a cluster, gas which has not been converted into stars (and there
is a tremendous amount of it because star formation is highly
inefficient) is ram pressure stripped and collisionally
heated. This superheated gas has temperatures in excess of 10$^7$ K
(2-10 keV) and emits profusely in X-rays. In fact, the most massive
baryonic component of a cluster is this intracluster X-ray medium
(ICM) and not the stars or galaxies.

Without secondary processes acting on or within the cluster
other than gravity, all clusters should be scaled versions of each
other, with the defining characteristic of a cluster being its mass. The
mass of a cluster determines the depth of the gravitational potential
well and the well depth in turn defines global cluster properties such
as luminosity and temperature of the X-ray gas. The temperature
depends on mass via the virial theorem, and the luminosity depends on
mass because mass dictates how many particles are available for
emitting radiation. 

But the Universe is not such a simple place and this simplistic
picture neglects the cooling and feedback which naturally occur in a
cluster. Some of the ICM has a cooling time shorter than the age of
the Universe. And as the ICM cools it radiates away much of the energy
acquired during merger. This radiation is the X-ray emission we see
with telescopes such as Chandra. As the ICM cools, portions of it condense
and flow to the bottom of the cluster potential well. But, the
hypothesized products of these ``cooling flows'', such as stars or
molecular clouds, are not observed in the cores of clusters. There is
clearly some feedback mechanism operating within clusters which retards
unabated cooling. The most likely candidate for the feedback is
currently active galactic nuclei (AGN), and this area of study is
under heavy focus by both observationalists and theoreticians. This is
where my dissertation work comes in, with the study of feedback,
evolution, and dynamics in clusters of galaxies.

\label{sec:ch12}
\subsubsection*{Chapters I and II: Introduction and Background}
These chapters of my dissertation are best described as ``Clusters of
Galaxies 101''. They cover the fundamental physical and ideological
principles of what clusters are, why we care about them, how they
form, what they can tell us about the past, present, and future of the
Universe, and what they can tell us about galaxy formation. These
chapters also explain the physical connection of general
characteristics of clusters such as mass, luminosity, 
temperature, and density. Later in the dissertation I rely heavily on the quantity
entropy to study clusters. These are the chapters where I explain what
entropy is, why it is of fundamental importance in studying clusters
(specifically feedback mechanisms), how I measure entropy, and the
physical connection of entropy to other cluster properties. My
dissertation has made extensive use of the Chandra X-ray telescope
launched and maintained by NASA. In these chapters I discuss the design
and use of Chandra which made it the primary observational tool of my
work. I have written, presented on, and discussed the topics
associated with these chapters on innumerable occassions making these the
easiest chapters to write. {\bf These chapters will be complete in
mid-April 2008 as the final pieces of my dissertation.} 

\label{sec:ch3}
\subsubsection*{Chapter III: Energy Band Dependance of X-ray
Temperatures}
One method of understanding the dynamical state of a cluster is by
quantifying the cluster's degree of relaxation. Theoretical work by
Mathiesen and Evrard\cite{2001ApJ...546..100M} found clusters which
had experienced a recent merger were much cooler than cluster
mass-observable scaling relations predict. The consequence being an
under-prediction of cluster masses of $15-30\%$. They also found hard
band (2.0$_{rest}$-9.0 keV) temperatures were $\sim 20\%$ hotter than
broad-band (0.5-9.0 keV) temperatures. I have completed an 
observational follow-up study of their theoretical prediction and will be
submitting the paper for publication in September 2007. This paper
constitutes the backbone of this chapter. In this chapter I
describe the sample selection from the Chandra Data Archive, data
reduction techniques, and data analysis methods which are universal for
all of my dissertation work. Additionally this chapter summarizes our
results -- we do measure a temperature skewing between bands -- and
also draws conclusions which are being tested further by another
Ph.D. candidate. As this chapter will be a modestly revised version of
the referreed journal publication, it is near completion. {\bf The 
completion date of this chapter is mid-September 2007.}

\label{sec:ch4}
\subsubsection*{Chapter IV: Library of Entropy Profiles}
As mentioned previously, I rely heavily on entropy as it is 
fundamental to understanding clusters. Entropy as we
have defined it combines the density and temperature of the ICM into one
physical quantity, $K=T/\rho^2$. Entropy has been shown to be of vital
importance in understanding the feedback mechanisms active within
clusters\cite{2001Natur.414..425V} and the role of the cluster
environment on galaxy formation. To understand these feedback
mechanisms is to understand the entropy and thermal history of the
ICM. This work is nearing completion and constitutes the bulk of my
dissertation. We already have one accepted referreed
publication\cite{2006ApJ...643..730D} which forms the foundation of
this chapter. My dissertation is a much larger version of the work
presented in this first paper. In this chapter I discuss additional
sample selection to the one presented in \S Chapter II. I also present
findings of the connection between central entropy-radio luminosity
and central entropy-H$\alpha$ emission. In addition, I discuss the
distribution of central entropy values in our full sample which has
ramifications on the timescale of feedback mechanisms. This is also
the chapter which will highlight error analysis and statistics of my
work, such as usage of Monte Carlo simulations in calculating errors
and reducing the scatter in scaling relations. {\bf This chapter is
20\% written with paper submission occuring in late-December 2007 and
chapter completion by mid-January 2008.}

\label{sec:ch5}
\subsubsection*{Chapter V: Summary; Bibliography, Appendicies, Figures,
and Tables}
The remaining chapter will be a summary of my dissertation. It will
briefly highlight the main results, important conclusions, and tie
together the concepts and principles of the preceeding three
chapters. This chapter is unwritten at the moment and can only
be completed after all other work has been completed. {\bf The
completion date of this chapter is late-March 2008.} Additionally, the
bibliography, appendicies, figures, and tables are being completed
in concurrence with their respective chapters and as such have no
timeline. {\bf Finalized formatting and compilation of my dissertation
will be complete in late-April 2008 in expectation of an early-May
2008 defense.}

\subsection*{Hardcopy Outline}
\begin{enumerate}
\item Introduction
\item Background
\begin{enumerate}
\item Clusters of Galaxies
\begin{enumerate}
\item What are Clusters of Galaxies?
\begin{enumerate}
\item Dark Matter
\item Intracluster Medium
\item Baryonic Matter
\item AGN and Radio Sources
\end{enumerate}
\item Why Do We Care About Clusters in Astrophysics?
\item How Do Clusters Form?
\item Usefulness in Studies of Galaxy Formation
\item Cosmology with Clusters
\begin{enumerate}
\item Observational
\item Theoretical
\end{enumerate}
\end{enumerate}
\item Observables
\begin{enumerate}
\item Surface Brightness
\item Gas Temperature
\item Deriving Gas Density
\item What is Entropy?
\item Calculating Entropy from Observables
\item Calculating Mass from Observables
\item An Entropy-Feedback Connection?
\end{enumerate}
\item Instrumentation
\end{enumerate}
\item Energy Band Dependence of X-ray Temperatures
\begin{enumerate}
\item Motivation from Mathiesen and Evrard 2001
\item Sample Selection
\item Data Reduction
\item Handling of X-ray Background
\begin{enumerate}
\item Hard Particle Background
\item Soft Local Background
\end{enumerate}
\item Spectral Extraction and Analysis
\item Discussion of Results
\item Conclusions
\end{enumerate}
\item Library of Entropy Profiles
\begin{enumerate}
\item Refining Sample Selection
\item Deprojection of Observables
\item Fitting Radial Models to Entropy Profiles
\item Error Determination
\item Central Entropy-Radio Luminosity Correlation
\item Central Entropy-H$\alpha$ Emission Correlation
\item Distribution of Central Entropy Values
\item Discussion of Results
\item Conclusions
\end{enumerate}
\item Summary
\end{enumerate}

\bibliographystyle{plan}
\bibliography{cavagnolo}
 
\end{document}
