%%%%%%%%%%%%%%%%%%%%%%
\section{Introduction}
\label{sec:entsuppintro}
%%%%%%%%%%%%%%%%%%%%%%

The general process of galaxy cluster formation through hierarchical
merging is well understood, but many details, such as the impact of
feedback sources on the cluster environment and radiative cooling in
the cluster core, are not. The nature of feedback operating within
clusters is of great interest because of the implications for better
understanding massive galaxy formation and using mass-observable
scaling relations in cluster cosmological studies. Early models of
structure formation which included only gravitation predicted
self-similarity among the galaxy cluster population. That is to say,
the physical properties of galaxy clusters, such as temperature and
luminosity, scaled with cluster redshift and mass \citep{kaiser86,
  1991ApJ...383...95E, nfw1, nfw2, 1996ApJ...469..494E,
  1997MNRAS.292..289E, 1997ApJ...480...36T, 1998ApJ...503..569E,
  1998ApJ...495...80B}. However, numerous observational studies have
shown clusters do not follow the predicted simple, low-scatter
mass-observable scaling relations \citep{edge91, 1998MNRAS.297L..57A,
  1998ApJ...504...27M, 1999MNRAS.305..631A, 1999ApJ...520...78H,
  2000ApJ...536...73N, 2001A&A...368..749F}. To reconcile observation
with theory, it was realized non-gravitational effects, such as
heating and radiative cooling in cluster cores, could not be neglected
if models were to accurately replicate the process of cluster
formation, \eg\ \citet{2000ApJ...532...17L} and
\citet{2002MNRAS.336..409B}.

As a consequence of radiative cooling, best-fit total cluster
temperature decreases while total cluster luminosity increases. In
addition, feedback sources such as active galactic nuclei (AGN) and
supernovae can drive cluster cores (where most of the cluster flux
originates) away from hydrostatic equilibrium. Thus, at a given mass
scale, radiative cooling and feedback conspire to create dispersion in
otherwise tight mass-observable correlations like mass-luminosity and
mass-temperature. While considerable progress has been made both
observationally and theoretically in the areas of understanding,
quantifying, and reducing scatter in cluster scaling-relations
\citep[eg][]{1996ApJ...458...27B, 2005ApJ...624..606J, kravtsov06,
  nagai07, VV08}, it is still important to understand how, taken as a
whole, non-gravitational processes affect cluster formation and
evolution.

An appurtenant issue to the departure of clusters from self-similarity
is that of cooling flows in cluster cores. The core cooling time in
50\%-66\% of clusters is much shorter than both the Hubble time and
cluster age \citep{1984ApJ...285....1S, 1992MNRAS.258..177E, white97,
  1998MNRAS.298..416P, 2005MNRAS.359.1481B}. For such clusters (and
without compensatory heating), radiative cooling will result in the
formation a cooling flow \citep[see][for a
  review]{fabiancfreview}. Early estimates put the mass deposition
rates from cooling flows in the range of $100-1000 \Msol \pyr$,
\citep[\eg][]{1984ApJ...276...38J}. However, cooling flow mass
deposition rates inferred from soft X-ray spectroscopy were found to
be significantly less than predicted, with the ICM never reaching
temperatures lower than $T_{virial}/3$ \citep{tamura01, peterson01,
  peterson03, 2004A&A...413..415K}. Irrespective of system mass, the
massive torrents of cool gas turned out to be more like cooling
trickles.

In addition to the lack of soft X-ray line emission from cooling
flows, prior methodical searches for the end products of cooling flows
(\ie\ molecular gas and emission line nebulae) revealed far less mass
is locked-up in by-products than expected \citep{heckman89,
  mcnamara90, odea94, voit95}. The disconnects between observation and
theory have been termed ``the cooling flow problem'' and raise the
question, ``Where has all the cool gas gone?'' The substantial amount
of observational evidence suggests some combination of energetic
feedback sources have heated the ICM to selectively remove gas with a
short cooling time and establish quasi-stable thermal balance of the
ICM.

Both the breakdown of self-similarity and the cooling flow problem
point toward the need for better understanding cluster feedback and
radiative cooling. Recent revisions to models of how clusters form and
evolve by including feedback sources has led to better agreement
between observation and theory \citep{bower06, croton06, saro06}. The
current paradigm regarding the cluster feedback process holds that AGN
are the primary heat delivery mechanism and that an AGN outburst
deposits the requisite energy into the ICM to retard, and in some
cases quench, cooling \citep[see][for a review]{mcnamrev}. How the
feedback loop functions is still the topic of much debate, but that
AGN are interacting with the hot atmospheres of clusters is no longer
in doubt as evidenced by the prevalence of bubbles in clusters
\citep{birzan04}. One robust observable which has proven useful in
studying the effect of non-gravitational processes is ICM entropy.

Taken individually, ICM temperature and density do not fully reveal a
cluster's thermal history because these quantities are most influenced
by the underlying dark matter potential \citep{voitbryan}. Gas
temperature reflects the depth of the potential well, while density
reflects the capacity of the well to compress the gas. However, at
constant pressure the density of a gas is determined by its specific
entropy. Rewriting the expression for the adiabat, $K=P\rho^{-5/3}$,
in terms of temperature and electron density, one can define a new
quantity, $K=kT_X n_e^{-2/3}$, where $T_X$ is temperature and $n_e$ is
electron gas density \citep{1999Natur.397..135P, davies00}. This new
quantity, $K$, captures the thermal history of the gas because only
heating and cooling can change $K$. This quantity is commonly referred
to as entropy, but in actuality the classic thermodynamic entropy is
$s = \ln K^{3/2} + \mathrm{constant}$.

One important property of gas entropy is that a gas cloud is
convectively stable when $dK/dr \geq 0$. Thus, gravitational potential
wells are giant entropy sorting devices: low entropy gas sinks to the
bottom of the potential well, while high entropy gas buoyantly rises
to a radius of equal entropy. If cluster evolution proceeded under the
influence of gravitation only, then the radial entropy distribution of
clusters would exhibit power-law behavior for $r > 0.1 r_{200}$ with a
constant, low entropy core at small radii \citep{voitbryan}. Thus,
large-scale departures of the radial entropy distribution from a
power-law can be used to measure the affect of processes such as AGN
heating and radiative cooling. Several studies have found that the
radial ICM entropy distribution in some clusters flattens outside $0.1
r_{500}$ \citep{1996ApJ...473..692D, 1999Natur.397..135P, davies00,
  2003MNRAS.343..331P, piffaretti05, radioquiet, d06,
  morandi07}. These previous studies utilized smaller, focused samples
and we have undertaken a much larger study utilizing the
\chandra\ Data Archive.

In this chapter we present the data and results from a
\chandra\ archival project in which we studied the ICM entropy
distribution for \entsuppnum\ galaxy clusters. We have named this
project the ``Archive of \chandra\ Cluster Entropy Profile Tables,''
or \accept\ for short. In contrast to the sample of nine classic
cooling flow clusters studied in \citet[][hereafter D06]{d06},
\accept\ covers a broader range of luminosities, temperatures, and
morphologies, focusing on more than just cooling flow clusters. One of
our primary objectives for this project was to provide the research
community with an additional resource to study cluster evolution and
confront current models with a broad range of entropy profiles.

We have found that the departure of entropy profiles from a
self-similar power-law is not limited to cooling flow clusters, but is
a feature of most clusters, and given high enough angular resolution,
possibly all clusters. We also find that the core entropy distribution
of both the full \accept\ collection and the Highest X-Ray Flux Galaxy
Cluster Sample (\hifl, \citealt{hiflugcs1, hiflugcs2}) are bimodal. In
Chapter \ref{ch:harad}, we present results that show indicators of
feedback - namely radio sources assumed to be associated with AGN and
\halpha\ emission assumed to be the result of thermal instability
formation - are strongly correlated with core entropy.

A key aspect of this project is the dissemination of all data and
results to the public. We have created a searchable, interactive web
site\footnote{\url{http://www.pa.msu.edu/astro/MC2/accept}} which
hosts all of our results. The \accept\ web site is being continually
updated as new \chandra\ cluster and group observations are archived
and analyzed. The web site provides all data tables, plots, spectra,
reduced \chandra\ data products (forthcoming), reduction scripts, and
more. Given the large number of clusters in our sample, figures, fits,
and tables showing/listing results for individual clusters have been
omitted and are available at the \accept\ web site.

The structure of this chapter is as follows: In
\S\ref{sec:entsuppsample} we outline initial sample selection criteria
and information about the \chandra\ observations selected under these
criteria. Data reduction is discussed in
\S\ref{sec:entsuppdata}. Spectral extraction and analysis are
discussed in \S\ref{sec:entsupptemppr}, while our method for deriving
deprojected electron density profiles is outlined in
\S\ref{sec:entsuppdene}. A few possible sources of systematics are
discussed in \S\ref{sec:entsuppsys}. Results and discussion are
presented in \S\ref{sec:entsuppr&d}. A brief summary is given in
\S\ref{sec:entsuppsummary}. For this work we have assumed a flat
\LCDM\ Universe with cosmology $\OM=0.3$, $\OL=0.7$, and
$\Hn=70\km\ps\pMpc$. All quoted uncertainties are 90\% confidence
($1.6\sigma$).

%%%%%%%%%%%%%%%%%%%%%%%%%
\section{Data Collection}
\label{sec:entsuppsample}
%%%%%%%%%%%%%%%%%%%%%%%%%

Our sample was initially collected from observations publicly
available in the \chandra\ Data Archive (CDA) as of June 2006. We
first assembled a list of targets from multiple flux-limited surveys:
the \rosat\ Brightest Cluster Sample \citep{1998MNRAS.301..881E}, RBCS
Extended Sample \citep{2000MNRAS.318..333E}, \rosat\ Brightest 55
Sample \citep{1990MNRAS.245..559E, 1998MNRAS.298..416P},
\einstein\ Extended Medium Sensitivity Survey
\citep{1990ApJS...72..567G}, North Ecliptic Pole Survey
\citep{2006ApJS..162..304H}, \rosat\ Deep Cluster Survey
\citep{1995ApJ...445L..11R}, \rosat\ Serendipitous Survey
\citep{1998ApJ...502..558V}, Massive Cluster Survey
\citep{2001ApJ...553..668E}, and {\it{REFLEX}} Survey
\citep{reflex}. After the first round of data analysis concluded, we
continued to expand our collection by adding new archival data listed
under the CDA Science Categories ``clusters of galaxies'' or ``active
galaxies.'' As of submission, we have inspected all CDA clusters of
galaxies observations and analyzed 510 of those observations (14.16
Msec). The Coma and Fornax clusters have been intentionally left-out
of our sample because they are very well studied nearby clusters which
require a more intensive analysis than we undertook in this project.

The available data for some clusters limited our ability to derive an
entropy profile. Calculation of entropy requires measurement of the
radial gas temperature and density structure (discussed further in
\S\ref{sec:entsuppdata}). To infer a temperature which is reasonably well
constrained ($\Delta kT_X \approx \pm 1.0 \keV$) we imposed a minimum
requirement of three temperature bins containing 2500 counts each. A
post-analysis check shows our $T_X$ minimum criterion resulted in a
mean $\Delta kT_X = 0.87$ keV for the final sample.

In section \ref{sec:entsupphifl} we cull the \hifl\ primary sample
\citep{hiflugcs1, hiflugcs2} from our full archival collection. The
groups M49, NGC 507, NGC 4636, NGC 5044, NGC 5813, and NGC 5846 are
part of the \hifl\ primary sample but were not members of our initial
archival sample. In order to take full advantage of the \hifl\ primary
sample, we analyzed observations of these 6 groups. Note, however,
that none of these 6 groups are included in the general discussion of
\accept.

We were unable to analyze some clusters for this study because of
complications other than not meeting our minimum requirements for
analysis. These clusters were: 2PIGG J0311.8-2655, 3C 129, A168, A514,
A753, A1367, A2634, A2670, A2877, A3074, A3128, A3627, AS0463, APMCC
0421, MACS J2243.3-0935, MS J1621.5+2640, SDSS J198.070267-00.984433,
RX J1109.7+2145, RX J1206.6+2811, RX J1423.8+2404, Triangulum
Australis, and Zw5247.

After applying the $T_X$ constraint, adding the 6 \hifl\ groups, and
removing troublesome observations the final sample presented here
contains \entsuppobs\ observations of \entsuppnum\ clusters with a
total exposure time of \expt. The sample covers the temperatures range
$kT_X \sim 1-20$ keV, a bolometric luminosity range of $L_{bol} \sim
10^{42-46} \ergps$, and redshifts of $z \sim 0.05-0.89$. Table
\ref{tab:entsuppsample} lists the general properties for each cluster
in \accept.

We also report \halpha\ observations taken by M. Donahue while a
Carnegie Fellow. These observations were not utilized in this chapter
but are used in Chapter \ref{ch:harad}. The new $[NII]/\halpha$ ratios
and \halpha\ fluxes are listed in Table \ref{tab:newha}. The
upper-limits listed in Table \ref{tab:newha} are $3\sigma$
significance. The observations were taken with either the 5 m Hale
Telescope at the Palomar Observatory, USA, or the DuPont 2.5 m
telescope at the Las Campanas Observatory, Chile. All observations
were made with a $2\arcs$ slit centered on the BCG using two position
angles: one along the semi-major axis and one along the semi-minor
axis of the galaxy. The overlap area was $10$ pixels$^2$. The red
light (555-798 nm) setup on the Hale Double Spectrograph used a 316
lines/mm grating with a dispersion of 0.31 nm/pixel and an effective
resolution of 0.7-0.8 nm. The DuPont Modular Spectrograph setup
included a 1200 lines/mm grating with a dispersion of 0.12 nm/pixel
and an effective resolution of 0.3 nm. The statistical and calibration
uncertainties for the observations are both $\sim 10\%$. The
statistical uncertainty arises primarily from variability of the
spectral continuum and hence imperfect background subtraction.

%%%%%%%%%%%%%%%%%%%%%%%
\section{Data Analysis}
\label{sec:entsuppdata}
%%%%%%%%%%%%%%%%%%%%%%%

Measuring radial ICM entropy first requires measurement of radial ICM
temperature and density. The radial temperature structure of each
cluster was measured by fitting a single-temperature thermal model to
spectra extracted from concentric annuli centered on the cluster X-ray
``center''. As discussed in \citet{xrayband}, the ICM X-ray peak of
the point-source cleaned, exposure-corrected cluster image was used as
the cluster center, unless the iteratively determined X-ray centroid
was more than 70 kpc away from the X-ray peak, in which case the
centroid was used as the radial analysis zero-point. To derive the gas
density profile, we first deprojected an exposure-corrected,
background-subtracted, point source clean surface brightness profile
extracted in the 0.7-2.0\keV\ energy range to attain a volume emission
density. This emission density, along with spectroscopic information
(count rate and normalization in each annulus), was then used to
calculate gas density. The resulting entropy profiles were then fit
with two models: a simple model which has only a radial power-law
component, and a model which is the sum of a constant core entropy
term, \kna\, and the radial power-law component.

In this chapter we cover the basics of deriving gas entropy from X-ray
observables, and direct interested readers to D06 for in-depth
discussion of our data reprocessing and reduction, and
\citet{xrayband} for details regarding determination of each cluster's
``center'' and how the X-ray background was handled. The only
difference between the analysis presented in this chapter and that of
D06 and \citet{xrayband}, is that we have used newer versions of the
CXC issued data reduction software (\ciao\ 3.4.1 and \caldb\ 3.4.0).

%%%%%%%%%%%%%%%%%%%%%%%%%%%%%%%%%
\subsection{Temperature Profiles}
\label{sec:entsupptemppr}
%%%%%%%%%%%%%%%%%%%%%%%%%%%%%%%%%

One of the two components needed to derive a gas entropy profile is
the temperature as a function of radius. We therefore constructed
radial temperature profiles for each cluster in our collection. To
reliably constrain a temperature, and allow for the detection of
temperature structure beyond isothermality, we required each
temperature profile to have a minimum of three annuli containing 2500
counts each. The annuli for each cluster were generated by first
extracting a background-subtracted cumulative counts profile using 1
pixel width annular bins originating from the cluster center and
extending to the detector edge. Temperature profiles, however, were
truncated at the radius bounded by the detector edge, or $0.5
R_{180}$, whichever was smaller. Truncation occurred at $0.5 R_{180}$
as we are most interested in the radial entropy behavior of cluster
core regions ($r \la 100$ kpc) and $0.5 R_{180}$ is the approximate
radius where temperature profiles begin to turnover
\citep{2005ApJ...628..655V}.  Additionally, analysis of diffuse gas
temperature structure at large radii, which spectroscopically is
dominated by background, requires a time consuming,
observation-specific analysis of the X-ray background \cite[see][for a
  detailed discussion on this point]{minggroups}.

Cumulative counts profiles were divided into annuli containing at
least 2500 counts. For well resolved clusters, the number of counts
per annulus was increased to reduce the resulting uncertainty of
$kT_X$ and, for simplicity, to keep the number of annuli less than
50. The method we use to derive entropy profiles is most sensitive to
the surface brightness radial bin size and not the resolution or
uncertainties of the temperature profile. Thus, the loss of resolution
in the temperature profile from increasing the number of counts per
bin, and thereby reducing the number of annuli, has an insignificant
effect on the final entropy profiles and best-fit entropy models.

Background analysis was performed using the blank-sky datasets
provided in the \caldb. Backgrounds were reprocessed and reprojected
to match each observation. Off-axis chips were used to normalize for
variations of the hard-particle background by comparing blank-sky and
observation 9.5-12\keV\ count rates. Soft residuals were also created
and fitted for each observation to account for the spatially-varying
soft Galactic background. This component was added as an additional,
fixed background component during spectral fitting. Errors associated
with the soft background are estimated and added in quadrature to the
final error.

For each radial annular region, source and background spectra were
extracted from the target cluster and corresponding normalized
blank-sky dataset. Following standard
\ciao\ techniques\footnote{\url{http://cxc.harvard.edu/ciao/guides/esa.html}}
we created weighted response files (WARF) and redistribution matrices
(WRMF) for each cluster using a flux-weighted map (WMAP) across the
entire extraction region. These files quantify the effective area,
quantum efficiency, and imperfect resolution of the
\chandra\ instrumentation as a function of chip position. Each
spectrum was binned to contain a minimum of 25 counts per energy bin.

Spectra were fitted with \xspec\ 11.3.2ag \citep{xspec} using an
absorbed, single-temperature \mekal\ model \citep{mekal1, mekal2} over
the energy range 0.7-7.0 \keV. Neutral hydrogen column densities,
\nhi, were taken from \citet{dickeylockman}. A comparison between the
\nhi\ values of \citet{dickeylockman} and the higher-resolution LAB
Survey \citep{lab} revealed that the two surveys agree to within $\pm
20\%$ for 80\% of the clusters in our sample. For the other 20\% of
the sample, using the LAB value, or allowing \nhi\ to be free, did not
result in best-fit temperatures or metallicities which differ
significantly from fits using the \citet{dickeylockman} values.

The potentially free parameters of the absorbed thermal model are
\nhi, X-ray temperature, metal abundance normalized to solar
\citep[elemental ratios taken from]{ag89}, and a normalization
proportional to the integrated emission measure within the extraction
region,
\begin{equation}
\label{eqn:norm}
\eta = \frac{10^{-14}}{4\pi D_A^2(1+z)^2}\int \nelec \np dV,
\end{equation}
where $D_A$ is the angular diameter distance, $z$ is cluster redshift,
\nelec\ and \np\ are the electron and proton densities respectively,
and $V$ is the volume of the emission region. In all fits the metal
abundance in each annulus was a free parameter and \nhi\ was fixed to
the Galactic value. No systematic error is added during fitting and
thus all quoted errors are statistical only. The statistic used during
fitting was $\chi^2$ (\xspec\ statistics package \textsc{chi}). All
uncertainties were calculated using 90\% confidence.

For some clusters, more than one observation was available in the
archive. We utilized the combined exposure time by first extracting
independent spectra, WARFs, WRMFs, normalized background spectra, and
soft residuals for each observation. These independent spectra were
then read into \xspec\ simultaneously and fit with the same spectral
model which had all parameters, except normalization, tied among the
spectra.

In D06 we studied a sample of nine ``classic'' cooling flow clusters,
all of which have steep temperature gradients. Deprojection of
temperature should result in slightly lower temperatures in the
central bins of only the clusters with the steepest temperature
gradients. For these clusters, the end result would be a lowering of
the entropy for the central-most bins. Our analysis in D06 showed that
spectral deprojection did not result in significant differences
between entropy profiles derived using projected or deprojected
quantities. Thus, for this work, we quote projected temperatures
only. We stress that spectral deprojection does not significantly
change the shape of the entropy profiles nor the best-fit
\kna\ values.

%%%%%%%%%%%%%%%%%%%%%%%%%%%%%%%%%%%%%%%%%%%%%%%%%%
\subsection{Deprojected Electron Density Profiles}
\label{sec:entsuppdene}
%%%%%%%%%%%%%%%%%%%%%%%%%%%%%%%%%%%%%%%%%%%%%%%%%%

For predominantly free-free emission, emissivity strongly depends on
density and only weakly on temperature, $\epsilon \propto \rho^2
T^{1/2}$. Therefore the flux measured in a narrow temperature range is
an good diagnostic of ICM density. To reconstruct the relevant gas
density as a function of physical radius we deprojected the cluster
emission from high-resolution surface brightness profiles and
converted to electron density using normalizations and count rates
taken from the spectral analysis.

We extracted surface brightness profiles from the 0.7-2.0 keV energy
range using concentric annular bins of size $5\arcs$ originating from the
cluster center. Each surface brightness profile was corrected with an
observation specific, normalized radial exposure profile to remove the
effects of vignetting and exposure time fluctuations. Following the
recommendation in the \ciao\ guide for analyzing extended sources,
exposure maps were created using the monoenergetic value associated
with the observed count rate peak. The more sophisticated method of
creating exposure maps using spectral weights calculated for an
incident spectrum with the temperature and metallicity of the observed
cluster was also tested. For the narrow energy band we consider, the
chip response is relatively flat and we find no significant
differences between the two methods. For all clusters the
monoenergetic value used in creating exposure maps was between
$0.8-1.7\keV$.

The 0.7-2.0 keV spectroscopic count rate and spectral normalization
were interpolated from the radial temperature profile grid to match
the surface brightness radial grid. Utilizing the deprojection
technique of \citet{kriss83}, the interpolated spectral parameters
were used to convert observed surface brightness to deprojected
electron density. Radial electron density written in terms of relevant
quantities is,
\begin{equation}
\nelec(r) = \sqrt{\frac{r_{ion}~4 \pi [D_A(1+z)]^2~C(r)~\eta(r)}{10^{-14}~f(r)}}
\end{equation}
where $r_{ion}$ is an ionization ratio ($\nelec=1.2\np$), $C(r)$ is
the radial emission density derived from eqn. A1 in \citet{kriss83},
$\eta$ is the interpolated spectral normalization from
eqn. \ref{eqn:norm}, $D_A$ is the angular diameter distance, $z$ is
cluster redshift, and $f(r)$ is the interpolated spectroscopic count
rate. Cosmic dimming of source surface brightness is accounted for by
the $D_A^2 (1+z)^2$ term. This method of deprojection takes into
account temperature and metallicity fluctuations which affect observed
gas emissivity. Errors for the gas density profile were estimated
using 5000 Monte Carlo simulations of the original surface brightness
profile. The \citet{kriss83} deprojection technique assumes spherical
symmetry, but it was shown in D06 such an assumption has little effect
on final entropy profiles.

%%%%%%%%%%%%%%%%%%%%%%%%%%%%%%%
\subsection{$\beta$-model Fits}
\label{sec:entsuppbeta}
%%%%%%%%%%%%%%%%%%%%%%%%%%%%%%%

Noisy surface brightness profiles, or profiles with irregularities
such as inversions or extended flat cores, result in unstable,
unphysical quantities when using the ``onion'' deprojection
technique. For cases where deprojection of the raw data was
problematic, we resorted to fitting the surface brightness profile
with a $\beta$-model \citep{1978A&A....70..677C}. It is well known
that the $\beta$-model is only an approximation for an isothermal gas
distribution and does not precisely represent all the features of the
ICM \citep{2000MNRAS.311..313E, 2002ApJ...579..571L,
  2007ApJ...665..911H}. However, for the profiles which required a
fit, the $\beta$-model was a suitable approximation, and the models
use was only a means for creating a smooth function which was easily
deprojected. The single ($N=1$) and double ($N=2$) $\beta$-models were
used in fitting,
\begin{eqnarray}
S_X &=& \displaystyle\sum S
\left[1+\left(\frac{r}{r_{c}}\right)^2\right]^{-3\beta+\onehalf}.
\end{eqnarray}
The models were fitted using Craig Markwardt's robust non-linear least
squares minimization IDL
routines\footnote{\url{http://cow.physics.wisc.edu/~craigm/idl/}}. The
data input to the fitting routines were weighted using the inverse
square of the observational errors. Using this weighting scheme
resulted in residuals which were near unity for, on average, the inner
80\% of the radial range considered. Accuracy of errors output from
the fitting routine were checked against a bootstrap Monte Carlo
analysis of 1000 surface brightness realizations. Both the single- and
double-$\beta$ models were fit to each profile and using the F-test
functionality of
\sherpa\footnote{\url{http://cxc.harvard.edu/ciao3.4/ahelp/ftest.html}}
we determined if the addition of extra model components was justified
given the degrees of freedom and \chisq\ values of each fit. If the
significance was less than 0.05, the extra components were justified
and the double-$\beta$ model was used.

A best-fit $\beta$-model was used in place of the data when deriving
electron density for the clusters listed in Table
\ref{tab:betafits}. These clusters are also flagged in Table
\ref{tab:entsuppsample} with the note letter `a'. In Section
\S\ref{sec:haradsupp}, notes discussing individual clusters are
provided, and a figure of the best-fit $\beta$-models and
background-subtracted, exposure-corrected surface brightness profiles
are provided in Figure \ref{fig:betamods}. The disagreement between
the best-fit $\beta$-model and the surface brightness in the central
regions for some clusters is also discussed in Section
\S\ref{sec:haradsupp}. In short, the discrepancy arises from the
presence of compact X-ray sources, a topic which is addressed in
\S\ref{sec:entsuppcentsrc}. All clusters requiring a $\beta$-model fit
have $\kna > 95 \ent$ and the mean best-fit parameters are listed in
Table \ref{tab:bfparams}.

%%%%%%%%%%%%%%%%%%%%%%%%%%%%%
\subsection{Entropy Profiles}
\label{sec:entsuppkpr}
%%%%%%%%%%%%%%%%%%%%%%%%%%%%%

Radial entropy profiles were calculated using the widely adopted
formulation $K(r) = kT_x(r)\nelec(r)^{-2/3}$. To create the radial
entropy profiles, the temperature and density profiles must be on the
same radial grid. This was accomplished by interpolating the
temperature profile across the higher-resolution radial grid of the
deprojected electron density profile. Because, in general, density
profiles have higher radial resolution, the central bin of the
temperature profile spans several of the innermost bins of the density
profile. Since we are most interested in the behavior of the entropy
profiles in the central regions, how the interpolation was performed
for the inner regions. Thus, temperature interpolation over the region
of the density profile where a single central temperature bin
encompasses several density profile bins was applied in two ways: (1)
as a linear gradient consistent with the slope of the temperature
profile at radii larger than the central $T_X$ bin ($\Delta T_{center}
\ne 0$; `extr' in Table \ref{tab:kfits}), and (2) as a constant
($\Delta T_{center}=0$; `flat' in Table \ref{tab:kfits}). Shown in
Figure \ref{fig:kcomp} is the ratio of best-fit core entropy, \kna,
using the above two methods. The five points lying below the line of
equality are clusters which are best-fit by a power-law or have
\kna\ statistically consistent with zero. It is worth noting that both
schemes yield statistically consistent values for \kna\ except for the
few clusters ({\it{red points}} in Fig. \ref{fig:kcomp}) which have a
ratio significantly different from unity.

\begin{figure}[htp]
  \begin{center}
    \begin{minipage}[htp]{\linewidth}
      \includegraphics*[width=\textwidth, trim=5mm 0mm 5mm 5mm, clip]{itplflat_rat}
      \caption[Ratio of best-fit \kna\ for temperature interpolation
        schemes.]{Ratio of best-fit \kna\ for the two treatments of
        central temperature interpolation (see
        \S\ref{sec:entsupptemppr}): (1) temperature is free to decline
        across the central density bins ($\Delta T_{center} \ne 0$),
        and (2) the temperature across the central density bins is
        isothermal ($\Delta T_{center} = 0$). Red points are clusters
        for which the \kna\ ratio is inconsistent with unity, however,
        all of these clusters have steep temperature gradients which
        result in unsubstantiated cool temperatures in their cores
        when $kT_X$ is extrapolated to small radii.}
      \label{fig:kcomp}
    \end{minipage}
  \end{center}
\end{figure}

The clusters which significantly differ from one all have steep
temperature gradients with the maximum and minimum radial temperatures
differing by a factor of 1.3-5.0. Extrapolation of a steep temperature
gradient as $r \rightarrow 0$ results in very low central temperatures
(typically $T_X \leq T_{virial}/3$) which are inconsistent with
observations, most notably \citet{peterson03}. Most important however,
is that the flattening of entropy we observe in the cores of our
sample (discussed in \S\ref{sec:entsuppnonzerok0}) is
{\bfseries\em{not}} a result of the method chosen for interpolating
the temperature profile. For this chapter we therefore focus on the
results derived assuming a constant temperature across the
central-most bins.

Uncertainty in $K(r)$ arising from using a single-component
temperature model for each annulus during spectral analysis
contributes negligibly to our final fits and is discussed in detail in
the Appendix of D06. Briefly summarizing D06: we have primarily
measured the entropy of the lowest entropy gas because it is the most
luminous gas. For the best-fit entropy values to be significantly
changed, the volume filling fraction of a higher-entropy component
must be non-trivial ($> 50\%$). As discussed in D06, our results are
robust to the presence of multiple, low luminosity gas phases and
mostly insensitive to X-ray surface brightness decrements, such as
X-ray cavities and bubbles, although in extreme cases their influence
on an entropy profile can be detected (for an example, see the cluster
A2052).

Each entropy profile was fit with two models: a simple model which is
a power-law at large radii and approaches a constant value at small
radii (eqn. \ref{eqn:k0}), and a model which is a power-law only
(eqn. \ref{eqn:plaw}). The models were fitted using Craig Markwardt's
IDL routines in the package MPFIT. The output best-fit parameters and
associated errors were checked against a bootstrap Monte Carlo
analysis of 5000 entropy profile realizations to independently confirm
their accuracy.
\begin{eqnarray}
K(r) &=& \kna + \khun\ \left(\frac{r}{100 \kpc}\right)^{\alpha}\label{eqn:k0}\\
K(r) &=& \khun\ \left(\frac{r}{100 \kpc}\right)^{\alpha}\label{eqn:plaw}.
\end{eqnarray}
In our entropy models, \kna\ is what we call core entropy, \khun\ is a
normalization for entropy at 100 kpc, and $\alpha$ is the power-law
index. Note, however, that \kna\ does not necessarily represent the
minimum core entropy or the entropy at $r=0$. Nor does \kna\ capture
the gas entropy which would be measured immediately around an AGN or
in a BCG X-ray corona. Instead, \kna\ represents the typical excess of
core entropy above the best fitting power-law at larger radii. Fits
were truncated at a maximum radius (determined by-eye) to avoid the
influence of noisy bins at large radii which result from instability
of our deprojection method. A listing of all the best-fit parameters
for each cluster are listed in Table \ref{tab:kfits}. The mean
best-fit parameters for the full \accept\ sample are given in Table
\ref{tab:bfparams}. Also given in Table \ref{tab:bfparams} are the
mean best-fit parameters for clusters below and above $\kna = 50
\ent$. We show in \S\ref{sec:entsuppbimod} that the cut at $\kna=50
\ent$ is not completely arbitrary as it approximately demarcates the
division between two distinct populations in the \kna\ distribution.

Some clusters have a surface brightness profile which is comparable to
a double $\beta$-model. Our models for the behavior of $K(r)$ are
intentionally simplistic and are not intended to fully describe all
the features of $K(r)$. Thus, for the small number of clusters with
discernible double-$\beta$ behavior, fitting of the entropy profiles
was restricted to the innermost of the two $\beta$-like
features. These clusters have been flagged in Table
\ref{tab:entsuppsample} with the note letter `b'. The best-fit
power-law index is typically much steeper for these clusters, but the
outer regions, which we do not discuss here, have power-law indices
which are typical of the rest of the sample, \ie\ $\alpha \sim 1.2$.


%%%%%%%%%%%%%%%%%%%%%%%%%%%%%%%%%%%%%%%%%
\subsection{Exclusion of Central Sources}
\label{sec:entsuppcentsrc}
%%%%%%%%%%%%%%%%%%%%%%%%%%%%%%%%%%%%%%%%%

For many clusters in our sample the ICM X-ray peak, ICM X-ray
centroid, BCG optical emission, and BCG infrared emission are
coincident or well within 70 kpc of one another. This made
identification of the cluster center robust and trivial. However, in
some clusters, there is an X-ray point source or compact X-ray source
($r \la 5$ kpc) found very near ($r < 10$ kpc) the cluster center and
always associated with a BCG. We identified \centsrcnum\ clusters with
central sources and have flagged them in Table \ref{tab:entsuppsample}
with the note letter `d' for AGN and `e' for compact but resolved
sources. The mean best-fit parameters for these clusters are given in
Table \ref{tab:bfparams} under the sample name `CSE' for ``central
source excluded.'' These clusters cover the redshift range $z =
0.0044-0.4641$ with mean $z = 0.1196 \pm 0.1234$, and temperature
range $kT_X = 1-12$ keV with mean $kT_X = 4.43 \pm 2.53$ keV. For some
clusters -- such as 3C 295, A2052, A426, Cygnus A, Hydra A, or M87 --
the source is an AGN and there was no question the source must be
removed.

However, determining how to handle the compact X-ray sources was not
so straightforward. These compact sources are larger than the point
spread function (PSF), fainter than an AGN, but typically have
significantly higher surface brightness than the surrounding ICM such
that the compact source's extent was distinguishable from the
ICM. These sources are most prominent, and thus the most troublesome,
in non-cool core clusters (\ie\ clusters which are approximately
isothermal). They are troublesome because the compact source is
typically much cooler and denser than the surrounding ICM and hence
has an entropy much lower than the ambient ICM. We consider most of
these compact sources are X-ray coronae associated with the BCG
\citep{coronae}.

Without removing the compact sources, we derived radial entropy
profiles and found, for all cases, that $K(r)$ abruptly changes at the
outer edge of the compact source. Including the compact sources
results in the central cluster region(s) appearing overdense, and at a
given temperature the region will have a much lower entropy than if
the source were excluded. Such a discontinuity in $K(r)$ results in
our simple models of $K(r)$ not being a good description of the
profiles. Aside from producing poor fits, a significantly lower
entropy influences the value of best-fit parameters because the shape
of $K(r)$ is drastically changed. Obviously, two solutions are
available: exclude or keep the compact sources during analysis.
Deciding what to do with these sources depends upon what cluster
properties we are specifically interested in quantifying.

The compact X-ray sources discussed in this section are not
representative of the cluster's core entropy; these sources are
representative of the entropy within and immediately surrounding
peculiar BCGs. Our focus for the \accept\ project was to quantify the
entropy structure of the cluster core region and surrounding
``pristine'' ICM, not to determine the minimum entropy of cluster
cores or to quantify the entropy of peculiar core objects such as BCG
coronae. Thus, we concluded to exclude these compact sources during
our analysis. For a few extraordinary sources, it was simpler to
ignore the central bin of the surface brightness profile during
analysis because of imperfect exclusion of a compact source's extended
emission. These clusters have been flagged in Table
\ref{tab:entsuppsample} with the note letter `f'.

It is worth noting that when any source is excluded from the data, the
empty pixels where the source once was were not included in the
calculation of the surface brightness (counts and pixels are both
excluded). Thus, the decrease in surface brightness of a bin where a
source has been removed is not a result of the count to area ratio
being artificially reduced.

%%%%%%%%%%%%%%%%%%%%%
\section{Systematics}
\label{sec:entsuppsys}
%%%%%%%%%%%%%%%%%%%%%

Our models for $K(r)$ were designed so that the best-fit \kna\ values
are a good measure of the entropy profile flattening at small
radii. This flattening could potentially be altered through the
effects of systematics such as PSF smearing and surface brightness
profile angular resolution. To quantify the extent to which our
\kna\ values are being affected by these systematics, we have analyzed
mock \chandra\ observations created using the ray-tracing program
MARX\footnote{\url{http://space.mit.edu/CXC/MARX/}}, and also by
analyzing degraded entropy profiles generated from artificially
redshifting well-resolved clusters. In the analysis below we show that
the lack of $\kna \la 10 \ent$ at $z \ga 0.1$ is attributable to
resolution effects, but that deviation of an entropy profile from a
power-law, even if only in the centralmost bin, cannot be accounted
for by PSF effects. We also discuss the number of profiles which are
reasonably well-represented by the power-law only profile, and
establish that no more than $\sim 10\%$ of the entropy profiles in
\accept\ are consistent with a power-law.

%%%%%%%%%%%%%%%%%%%%%%%%
\subsection{PSF Effects}
\label{sec:entsupppsf}
%%%%%%%%%%%%%%%%%%%%%%%%

To assess the effect of PSF smearing on our entropy profiles, we have
updated the analysis presented in \S4.1 of D06 to use MARX
simulations. In the D06 analysis, we assumed the density and
temperature structure of the cluster core obeyed power-laws with $n_e
\propto r^{-1}$ and $T_X \propto r^{1/3}$. This results in a power-law
entropy profile with $K \propto r$. Further assuming the main emission
mechanism is thermal bremsstrahlung, \ie\ $\epsilon_X \propto
T_X^{1/2}$, yields a surface brightness profile which has the form
$S_X \propto r^{-5/6}$. A source image consistent with these
parameters was created in \idl\ and then input to MARX to create the
mock \chandra\ observations.

The MARX simulations were performed using the spectrum of a 4.0 keV,
$0.3 \Zsol$ abundance \mekal\ model. We have tested using input
spectra with $kT_X = 2-10$ keV with varying abundances and find the
effect of temperature and metallicity on the distribution of photons
in MARX to be insignificant for our discussion here. We have neglected
the X-ray background in this analysis as it is overwhelmed by cluster
emission in the core and is only important at large
radii. Observations for both ACIS-S and ACIS-I instruments were
simulated using an exposure time of 40 ksec. A surface brightness
profile was then extracted from the mock observations using the same
$5\arcs$ bins used on the real data.

For $5\arcs$ bins, we find the difference between the central bins of the
input surface brightness and the output MARX observations to be less
than the statistical uncertainty. One should expect this result, as
the on-axis \chandra\ PSF is $\la 1\arcs$ and the surface brightness bins
we have used on the data are five times this size. What is most
interesting and important though, is that our analysis using MARX
suggests any deviation of the surface brightness -- and consequently
the entropy profile -- from a power-law, even if only in the central
bin, is real and cannot be attributed to PSF effects. Even for the
most poorly resolved clusters, the deviation away from a power-law we
observe in so many of our entropy profiles is not a result of our
deprojection technique or systematics.

%%%%%%%%%%%%%%%%%%%%%%%%%%%%%%%%%%%%%%%
\subsection{Angular Resolution Effects}
\label{sec:entsuppangres}
%%%%%%%%%%%%%%%%%%%%%%%%%%%%%%%%%%%%%%%

Another possible limitation on evaluating \kna\ is the effect of using
fixed angular size bins for extracting surface brightness
profiles. This choice may introduce a redshift-dependence into the
best-fit \kna\ values because as redshift increases, a fixed angular
size encompasses a larger physical volume and the value of \kna\ may
increase if the bin includes a broad range of gas entropy. Shown in
Figure \ref{fig:k0res} is a plot of the best-fit \kna\ values for our
entire sample versus redshift. At low redshift ($z < 0.02$), there are
a few objects with $\kna < 10 \ent$ and only one at higher redshift
(A1991 -- $\kna = 1.53 \pm 0.32$, $z = 0.0587$ -- which is a very
peculiar cluster \citep{2004ApJ...613..180S}). This raises the
question: can the lack of clusters with $\kna \la 10 \ent$ at $z >
0.02$ be completely explained by resolution effects?

\begin{figure}[htp]
  \begin{center}
    \begin{minipage}[htp]{\linewidth}
      \includegraphics*[width=\textwidth, trim=5mm 0mm 5mm 5mm, clip]{k0res}
      \caption[Best-fit \kna\ versus redshift.]{Best-fit \kna\ versus
        redshift. Some clusters have \kna\ error bars smaller than the
        point. The clusters with upper-limits (black points with
        downward arrows) are: A2151, AS0405, MS 0116.3-0115, and RX
        J1347.5-1145. The numerically labeled clusters are: (1) M87,
        (2) Centaurus Cluster, (3) RBS 533, (4) HCG 42, (5) HCG 62,
        (6) SS2B153, (7) A1991, (8) MACS0744.8+3927, and (9) CL
        J1226.9+3332. For CLJ1226, \cite{2007ApJ...659.1125M} found
        best-fit $\kna = 132 \pm 24 \ent$ which is not significantly
        different from our value of $\kna = 166 \pm 45 \ent$. The lack
        of $\kna < 10 \ent$ clusters at $z > 0.1$ is most likely the
        result of insufficient angular resolution (see
        \S\ref{sec:entsuppangres}).}
      \label{fig:k0res}
    \end{minipage}
  \end{center}
\end{figure}

To answer this question we tested the affect redshift has on measuring
\kna\ by selecting all clusters with $\kna \leq 10 \ent$ and $z \leq
0.1$ and degrading their surface brightness profiles to mimic the
effect of increasing the cluster redshift. Our test is best
illustrated using an example: consider a cluster at $z = 0.1$. For
this cluster, $5\arcs \approx 9$ kpc. Were the cluster at $z = 0.2$, $5\arcs$
would equal $\approx 16$ kpc. To mimic moving this example cluster
from $z = 0.1 \rightarrow 0.2$, we can extract a new surface
brightness profile using a bin size of 16 kpc instead of $5\arcs$. This
will result in a new surface brightness profile which has the angular
resolution for a cluster at a higher redshift.

We used the preceding procedure to degrade the profiles of our
subsample. New surface brightness bin sizes were calculated for each
cluster over an evenly distributed grid of redshifts in the range $z =
0.1-0.4$ using step sizes of 0.02. The temperature profiles for each
cluster were also degraded by starting at the innermost temperature
profile annulus and moving outward pairing-up neighboring annuli. New
spectra were extracted for these enlarged regions and analyzed
following the same procedure detailed in \S\ref{sec:entsupptemppr}.

The ensemble of artificially redshifted clusters were analyzed using
the procedure outlined in \S\ref{sec:entsuppkpr}. The notable effects
on the entropy profiles arising from lower angular resolution are: (1)
less information about profile shape, and (2) increased entropy of the
centralmost bins. Obviously, as redshift increases, the number of
radial bins decreases. Fewer radial bins translates into less detail
of an entropy profile's curvature, \eg\ the profiles become less
``curvy.'' On its own this effect should lead to lower best-fit
\kna\ values, but, while profile curvature is reduced, the entropy of
the central-most bins is increasing because the bins encompass a
broader range of entropy. From $z = 0.1-0.3$ this last effect
dominates, resulting in an increase of $\dkna = 2.72 \pm 1.84$ where
\kna\ is the original best-fit value and $\kna^{\prime}$ is the
best-fit value of the degraded profiles. However, at $z > 0.3$, the
loss of radial resolution dominates and the degraded profiles begin to
resemble power-laws except for the innermost bin which still lies
above the power-law (the uncertainty of the best-fit \kna\ also
increases). The result of a power-law profile with a discrepant
central bin is that the degraded \kna\ values are only slightly larger
than the fiducial best-fit \kna\ of the un-degraded data, $\dkna =
0.71 \pm 0.57$.

Our analysis of the degraded entropy profiles suggests that \kna\ is
more sensitive to the value of $K(r)$ in the central bins than it is
to the shape of the profile or the number of radial bins (systematics
we explore further in \S\ref{sec:entsuppcurve}). Most importantly
however, is that low-redshift clusters with $\kna \le 10 \ent$ look
like $\kna \approx 10-30 \ent$ clusters at $z > 0.1$. Thus we conclude
that the lack of $\kna < 10 \ent$ clusters at $z \ga 0.1$ can be
attributed to resolution effects.

%%%%%%%%%%%%%%%%%%%%%%%%%%%%%%%%%%%%%%%%%%%%%%%%%
\subsection{Profile Curvature and Number of Bins}
\label{sec:entsuppcurve}
%%%%%%%%%%%%%%%%%%%%%%%%%%%%%%%%%%%%%%%%%%%%%%%%%

From our analysis of the degraded entropy profiles in
\S\ref{sec:entsuppangres} we found: (1) that the best-fit \kna\ is
sensitive to the curvature of the entropy profile, and (2) that the
number of radial bins may also affect the best-fit \kna. This raises
the possibility of two troubling systematics in our analysis. To check
for a possible correlation between best-fit \kna\ and profile
curvature we first calculated average profile curvatures,
$\kappa_A$. For each profile, $\kappa_A$ was calculated using the
standard formulation for curvature of a function, $\kappa =
\|y^{''}\|/(1+y^{'2})^{3/2}$, where we set $y = K(r) =
\kna+\khun(r/100\kpc)^{\alpha}$. This derivation yields,
\begin{equation}
\kappa_A = \frac{\int\frac{\| 100^{-\alpha} (\alpha-1) \alpha \khun
    r^{\alpha-2}\|}{[1+(100^{-\alpha} \alpha \khun
      r^{\alpha-1})^2]^{3/2}} dr}{\int dr}
\end{equation}
where $\alpha$ and \khun\ are the best-fit parameters unique to each
entropy profile. The integral over all space ensures we evaluate the
curvature of each profile in the limit where the profiles have
asymptotically approached a constant at small radii and the power-law
at large radii. Shown in Figure \ref{fig:curve} is a plot of
$\kappa_A$ versus \kna. We find that at any value of \kna, a large
range of curvatures are covered and that there is no systematic trend
in \kna\ associated with $\kappa_A$.

In \S\ref{sec:entsuppangres} we also found that profiles with fewer
radial bins tend toward lower best-fit \kna\ values. Shown in Figure
\ref{fig:nbins} is a plot of \kna\ versus the number of bins fit in
each entropy profile. From Figure \ref{fig:nbins} it is evident that
that there is only scatter and no trend.

\begin{figure}[!t]
  \begin{minipage}[t]{0.5\linewidth}
    \includegraphics*[width=\textwidth, trim=28mm 7mm 40mm 17mm, clip]{curvk0}
    \caption[Best-fit \kna\ versus average curvature,
      $\kappa_A$.]{Best-fit \kna\ versus average curvature. Clusters
      with \kna\ values consistent with zero are plotted using
      $2\sigma$ upper-limits and downward arrows. The lack of a trend
      in average curvature with \kna\ suggests the \kna\ values we
      find are more sensitive to core entropy than to shape of a
      profile.}
    \label{fig:curve}
  \end{minipage}
  \hspace{0.1in}
  \begin{minipage}[t]{0.5\linewidth}
    \includegraphics*[width=\textwidth, trim=28mm 7mm 40mm 17mm, clip]{nbins_k0}
    \caption[Best-fit \kna\ versus number of fit bins.]{Best-fit
      \kna\ versus number of bins fit in the entropy profile. The lack
      of a trend between $N_{bins}$ and \kna\ again suggests that our
      best-fit \kna\ values properly represent the core entropy and
      not the radial resolution of the profile.}
    \label{fig:nbins}
  \end{minipage}
\end{figure}

We do not find any systematic trends with profile shape or number of
fit bins which would significantly affect our best-fit \kna\
values. Thus we conclude that the \kna\ values presented in the
following sections are an adequate measure of the core entropy and any
undetected dependence on profile shape or radial resolution affect our
results at significance levels much smaller than the measured
uncertainties.

%%%%%%%%%%%%%%%%%%%%%%%%%%%%%%%
\subsection{Power-law Profiles}
\label{sec:entsuppquality}
%%%%%%%%%%%%%%%%%%%%%%%%%%%%%%%

An important question regarding our entropy profiles is what fraction
of the full \accept\ and \hifl\ samples are well-represented by the
power-law only model and/or the power-law plus constant core entropy
model? The fitting routine we used to find the best-fit entropy models
to our data is a least-squares minimizer which outputs a chi-square
value. Assuming chi-square is the statistic describing the probability
distribution, the number of degrees of freedom and \chisq\ values can
be used to calculate a p-value. For the discussion presented below, we
have adopted the conventional significance criterion which says if
p-value $>$ 0.05, then the null hypothesis cannot be rejected, assuming
the null hypothesis is ``the'' true model. The null hypotheses in the
case of our models are that $K(r)$ is best modeled as a power-law only
(eqn. \ref{eqn:plaw}) or a power-law plus constant term
(eqn. \ref{eqn:k0}).

Note that p-values can only determine if the null hypothesis can be
significantly rejected. We stress that p-values do not represent the
probability that the null hypothesis is correct, nor do p-values
measure the significance of the best-fit model compared to the null
hypothesis. These are both incorrect intepretations. To judge the
quality of the best-fit models, specifically in relation to one
another, other quantities must be brought to bear such as the
significance of \kna\ away from zero, the actual values of \chisq, and
the typical uncertainty associated with the data.

The fractions provided in Table \ref{tab:bfparams} represent the
number of clusters in the sample which are well-represented by our
$K(r)$ models where ``well-represented'' is defined as any model which
has a p-value $>$ 0.05. The fractions are independent of each other,
hence they do not sum to unity. It may appear odd that for several
sub-groups there are a large fraction of the clusters for which the
power-law only model cannot be rejected. But in Table \ref{tab:kfits}
we show that most clusters have best-fit \kna\ values which are
several $\sigma_{\kna}$ greater than zero. The number and percentage
of clusters with \kna\ statistically consistent with zero at various
confidence levels are given in Table \ref{tab:bfparams}. Even at
$3\sigma$ significance only $\sim10\%$ of the full \accept\ sample has
a best-fit \kna\ value which is consistent with zero.

So while it is tempting to think the p-values are implying the
power-law model is sufficient to describe $K(r)$ for $\sim60\%$ of the
\accept\ sample, this is not a proper interpretation of the p-values
and conflicts with the fact that at least $\sim 90\%$ of the sample
have significant non-zero \kna. Equation \ref{eqn:k0} is a special
case of eqn. \ref{eqn:plaw} with $\kna = 0$, \eg\ the models we fit to
$K(r)$ are nested. In addition, the added parameter has an acceptable
best-fit value, $\kna = 0$, which lies on the boundary of the
parameter space. While under these conditions \chisq, associated
p-values, and F-tests are not useful in determining which model is the
``best'' description of $K(r)$, comparison of the \chisq\ values for
each fit imply, even if only qualitatively, which model shows more
agreement with the data. We have made a comparison of the models using
an F-test to determine if the addition of the \kna\ parameter made a
significant improvement in the best-fit. For all clusters, the
addition of a \kna\ term was found to be warranted, although it is not
obvious that an F-test yields any information given the models are
nested.  Moreover, that there is a systematic trend for a single
power-law to be a poor fit mainly at the smallest radii suggests
non-zero \kna\ is not random.

Of the \entsuppnum\ clusters in \accept, only four clusters have a
\kna\ value which is statistically consistent with zero (at
$1\sigma$), or are better fit by the power-law only model (based on
comparison of reduced \chisq): A2151, AS0405, MS 0116.3-0115, and NGC
507\footnote{NGC 507 is part of \hifl\ analysis only}. Two additional
clusters, A1991 and A4059, are better fit by the power-law model only
when interpolation of the temperature profile in the core is not
constant (see \S\ref{sec:entsupptemppr}). We find that the entropy
model which approaches a constant core entropy at small radii appears
to be a better descriptor of the radial entropy distribution for most
\accept\ clusters. However, we cannot rule out the power-law only
model, but do point out that $\sim90\%$ of clusters have best-fit
\kna\ values greater than zero at $> 3\sigma$ significance.

%%%%%%%%%%%%%%%%%%%%%%%%%%%%%%%%
\section{Results and Discussion}
\label{sec:entsuppr&d}
%%%%%%%%%%%%%%%%%%%%%%%%%%%%%%%%

Presented in Figure \ref{fig:splots} is a montage of \accept\ entropy
profiles for different temperature ranges. These figures highlight the
cornerstone result of \accept: a uniformly analyzed collection of
entropy profiles covering a broad range of core entropy. Each profile
is color-coded in representation of the global cluster
temperature. Plotted in each panel of Figure \ref{fig:splots} are the
mean profiles representing $\kna \le 50 \ent$ clusters (dashed-line)
and $\kna > 50 \ent$ clusters (dashed-dotted line), in addition to the
pure-cooling model of \citet{voitbryan} (solid black line). The
theoretical pure-cooling curve represents the entropy profile of a 5
keV cluster simulated with radiative cooling but no feedback. Thus,
the pure-cooling curve represents a lower limit of possible entropy
distributions and gives us a useful baseline with which to compare
\accept\ profiles.

In the following sections we discuss results gleaned from analysis of
our library of entropy profiles. Results such as the departure of most
entropy profiles from a simple radial power-law profile, the bimodal
distribution of core entropy, and the asymptotic convergence of the
entropy profiles to the self-similar $K(r) \propto r^{1.1}$ power-law
at $r \geq 100\kpc$.

\begin{center}
  \begin{figure}[htp]
    \begin{minipage}[htp]{0.5\linewidth}
      \includegraphics*[width=\textwidth, trim=28mm 7mm 30mm 17mm, clip]{splots_allt}
    \end{minipage}
    \begin{minipage}[htp]{0.5\linewidth}
      \includegraphics*[width=\textwidth, trim=28mm 7mm 30mm 17mm, clip]{splots_tle4}
    \end{minipage}
    \begin{minipage}[htp]{0.5\linewidth}
      \includegraphics*[width=\textwidth, trim=28mm 7mm 30mm 17mm, clip]{splots_gt4tle8}
    \end{minipage}
    \begin{minipage}[htp]{0.5\linewidth}
      \includegraphics*[width=\textwidth, trim=28mm 7mm 30mm 17mm, clip]{splots_tgt8}
    \end{minipage}
    \caption[Montage of entropy profiles for varying cuts in cluster
      temperature.]{Composite plots of entropy profiles for varying
      cluster temperature ranges. Profiles are color-coded based on
      average cluster temperature. Units of the color bars are
      keV. The solid-line is the pure-cooling model of
      \cite{voitbryan}, the dashed-line is the mean profile for
      clusters with $\kna \le 50 \ent$, and the dashed-dotted line is
      the mean profile for clusters with $\kna > 50 \ent$. {\it{Top
          left:}} This panel contains all the entropy profiles in our
      study. {\it{Top right:}} Clusters with $kT_X < 4$
      keV. {\it{Bottom left:}} Clusters with $4\keV < kT_X <
      8\keV$. {\it{Bottom right:}} Clusters with $kT_X > 8$ keV. Note
      that while the dispersion of core entropy for each temperature
      range is large, as the $kT_X$ range increases so to does the
      mean core entropy.}
    \label{fig:splots}
  \end{figure}
\end{center}

%%%%%%%%%%%%%%%%%%%%%%%%%%%%%%%%%%
\subsection{Non-Zero Core Entropy}
\label{sec:entsuppnonzerok0}
%%%%%%%%%%%%%%%%%%%%%%%%%%%%%%%%%%

Arguably the most striking feature of Figure \ref{fig:splots} is the
departure of most profiles from a simple power-law. Core flattening of
surface brightness profiles (and consequently density profiles) is a
well known feature of clusters (\eg\ \citealt{1999ApJ...517..627M} and
\citealt{2000MNRAS.318..715X}). What is notable in our work however is
that, based on comparison of reduced $\chi^2$ and significance of
\kna\, very few of the clusters in our sample have an entropy
distribution which is best-fit by the power-law only model
(eqn. \ref{eqn:plaw}), rather they are sufficiently well-described by
the model which flattens in the core (eqn. \ref{eqn:k0}).

For the six clusters discussed in \S\ref{sec:entsuppquality} which are more
consistent with a power-law, it may be the case that the ICM entropy
departs from a power-law at a radial scale smaller than the $5\arcs$
bins we used for extracting surface brightness profiles. After
extracting new surface brightness profiles for these six clusters
using $2.5\arcs$ bins and repeating the analysis, we find that the
profiles for A4059 and AS0405 do flatten. This leaves A1919, A2151, MS
0116.3-0115, and NGC 507 as the only clusters in \accept\ for which
the power-law model cannot be reasonably argued against.

For clusters with central cooling times shorter than the age of the
cluster, non-zero core entropy is an expected consequence of episodic
heating of the ICM \citep{agnframework}, with AGN as one possible
heating source \citep{1997MNRAS.288..355B, 2000ApJ...532...17L,
2001Natur.414..425V, 2002MNRAS.332..729C, 2002Natur.418..301B,
2002MNRAS.331..545B, 2002MNRAS.333..145N, 2002ApJ...581..223R,
2002MNRAS.335..610A, 2004MNRAS.348.1105O, 2004ApJ...613..811M,
2004ApJ...615..681R, 2004ApJ...617..896H, 2004MNRAS.355..995D,
2005ApJ...622..847S, pizzolato05, 2006ApJ...643..120B,
2006ApJ...638..659M}. Clusters with cooling times of order the age of
the Universe, however, require other mechanisms to generate their core
entropy, for example via mergers or extremely energetic AGN
outbursts. For the very highest \kna\ values, $\kna > 100 \ent$, the
mechanism by which the core entropy came to be so large is not well
understood as it is difficult to boost the entropy of a gas parcel to
$> 100 \ent$ via merger shocks \citep{2008MNRAS.386.1309M} and would
require AGN outburst energies which have never been observed. We are
providing the data and results of \accept\ to the public with the hope
that the research community finds it a useful new resource to further
understand the processes which result in non-zero cluster core
entropy.

%%%%%%%%%%%%%%%%%%%%%%%%%%%%%%%%%%%%%%%%%%%%%%%%%%%%
\subsection{Bimodality of Core Entropy Distribution}
\label{sec:entsuppbimod}
%%%%%%%%%%%%%%%%%%%%%%%%%%%%%%%%%%%%%%%%%%%%%%%%%%%%

The time required for a gas parcel to radiate away its thermal energy
is a function of the gas entropy. Low entropy gas radiates profusely
and is thus subject to rapid cooling and vice versa for high entropy
gas. Hence, the distribution of \kna\ is of particular interest
because it is an approximate indicator of the cooling timescale in the
cluster core. The \kna\ distribution is also interesting because it
may be useful in better understanding the physical processes operating
in cluster cores. For example, if processes such as thermal conduction
and AGN feedback are important in establishing the entropy state of
cluster cores, then models which incorporate these processes should
approximately reproduce the observed \kna\ distribution.

In the top panel of Figure \ref{fig:k0hist} is plotted the
logarithmically binned distribution of \kna. In the bottom panel of
Figure \ref{fig:k0hist} is plotted the cumulative distribution of
\kna. One can immediately see from these distributions that there are
at least two distinct populations separated by a small number of
clusters with $\kna \approx 30-60 \ent$. If the distinct bimodality of
the \kna\ distribution seen in the binned histogram were an artifact
of binning, then the cumulative distribution should be relatively
smooth. But there are clearly plateaus in the cumulative distribution,
with one of these plateaus coincident with the division between the
two populations at $\kna \approx 30-60 \ent$.

\begin{figure}[htp]
  \begin{center}
    \begin{minipage}[htp]{\linewidth}
      \includegraphics*[width=\textwidth, trim=20mm 10mm 10mm 10mm, clip]{k0hist}
      \caption[Histogram and cumulative distribution of best-fit
        \kna\ for full \accept\ sample.]{{\it{Top panel:}} Histogram
        of best-fit \kna\ for all the clusters in \accept. Bin widths
        are 0.15 in log space.  {\it{Bottom panel:}} Cumulative
        distribution of \kna\ values for the full sample. The distinct
        bimodality in \kna\ is present in both distributions, which
        would not be seen if it were an artifact of the histogram
        binning. A KMM test finds the \kna\ distribution cannot arise
        from a simple unimodal Gaussian.}
      \label{fig:k0hist}
    \end{minipage}
  \end{center}
\end{figure}

To further test for the presence of a bimodal population we utilized
the KMM test of \citet{kmm1}. The KMM test estimates the probability
that a set of data points is better described by the sum of multiple
Gaussians than by a single Gaussian. We tested the unimodal case
versus the bimodal case, with the assumption that the dispersion of
the two Gaussian components are not the same. We have used the updated
KMM code of \citet{kmm2} which incorporates bootstrap resampling to
determine uncertainties for all parameters. A post-analysis comparison
of fits assuming the populations have the same and different
dispersions confirms our initial guess that the dispersions are
different is a better model.

The KMM test, as with any statistical test, is very specific. At
zeroth order, the KMM test simply determines if a population is
unimodal or not, and finds the means of these populations. However,
the dispersions of these populations are subject to the quality of
sampling and the presence of outliers (\eg\ KMM must assign all data
points to a population). The outputs of the KMM test are the best-fit
populations to the data, not necessarily the best-fit populations of
the underlying distribution (hence no goodness of fit is
output). However, the KMM test does output a P-value, $p$, and with
the assumption that \chisq\ describes the distribution of the
likelihood ratio statistic, $1-p$ is the confidence interval for the
null hypothesis.

There are a small number of clusters with $\kna \le 4 \ent$ that when
included in the KMM test significantly change the results. Thus we
conducted tests including and excluding $\kna \le 4 \ent$ clusters and
provide two sets of best-fit parameters. The results of the KMM test
neglecting $\kna \le 4 \ent$ clusters were two statistically distinct
peaks at \kmma\ and \kmmb. \kmmc\ clusters were assigned to the first
distribution, while \kmmd\ were assigned to the second. Including
$\kna \le 4 \ent$ clusters, the KMM test found populations at \kmmf\
(\kmmh\ clusters) and \kmmg\ (\kmmi\ clusters). The KMM test
neglecting $\kna \le 4 \ent$ clusters returned \kmme, while the test
including all clusters returned \kmmj. These tiny $p$-values indicate
the unimodal distribution is significantly rejected as the parent
distribution of the observed \kna\ distribution.

One possible explanation for a bimodal core entropy distribution is
that it arises from the effects of episodic AGN feedback and electron
thermal conduction in the cluster core. \citet{agnframework} put forth
a model of AGN feedback whereby outbursts of $\sim 10^{45} \ergps$
occurring every $\sim 10^8 \yrs$ can maintain a quasi-steady core
entropy of $\approx 10-30 \ent$. In addition, very energetic and
infrequent AGN outbursts of $10^{61} \erg$ can increase the core
entropy into the $\approx 30-50 \ent$ range. This model satisfactorily
explains the distribution of $\kna \lesssim 50 \ent$, but depletion of
the $\kna = 30-60 \ent$ region and populating $\kna > 60 \ent$
requires more physics. \citet{conduction} have recently suggested that
the dramatic fall-off of clusters beginning at $\kna \approx 30 \ent$
may be the result of electron thermal conduction. After \kna\ has gone
beyond $\kna \approx 30 \ent$, conduction could severely slow, if not
halt, a cluster's core from appreciably cooling and returning to a
core entropy state with $\kna < 30 \ent$. This model is supported by
results presented in Chapter \ref{ch:harad}, \citet{2008arXiv0804.3823G},
and \citet{2008arXiv0802.1864R} which find that the formation of
thermal instabilities are extremely sensitive to the core entropy
state of a cluster.

We acknowledge that \accept\ is not a complete, uniformly selected
sample of clusters. This raises the possibility that our sample is
biased towards clusters that have historically drawn the attention of
observers, such as cooling flows or mergers. If that were the case,
then one reasonable explanation of the \kna\ bimodality is that $\kna
= 30-60 \ent$ clusters are ``boring'' and thus go unobserved. However,
as we show in \S\ref{sec:entsupphifl}, the unbiased flux-limited
\hifl\ sample is also bimodal. A sociological explanation of
bimodality for both \accept\ and \hifl\ is highly unlikely.

%%%%%%%%%%%%%%%%%%%%%%%%%%%%%%%%%%
\subsection{The \hifl\ Sub-Sample}
\label{sec:entsupphifl}
%%%%%%%%%%%%%%%%%%%%%%%%%%%%%%%%%%

\accept\ is not a flux-limited or volume-limited sample. To ensure our
results are not affected by an unknown selection bias, we culled the
\hifl\ sample from \accept\ for separate analysis. \hifl\ is a
flux-limited sample ($f_X \ge 2 \times 10^{-11} \flux$) selected by
flux only from the {\it{REFLEX}} sample \citep{reflex} with no
consideration of morphology. Thus, at any given luminosity in
\hifl\ there is a good sampling of different morphologies, \ie\ the
bias toward cool-cores or mergers has been removed. The sample also
covers most of the sky with holes near Virgo and the Large and Small
Magellanic Clouds, and has no known incompleteness
\citep{2007A&A...466..805C}. There are a total of 106 objects in
\hifl: 63 in the primary sample and 43 in the extended sample. Of
these 106 objects, no public \chandra\ observations were available for
16 objects (A548e, A548w, A1775, A1800, A3528n, A3530, A3532, A3560,
A3695, A3827, A3888, AS0636, HCG 94, IC 1365, NGC 499, RXCJ
2344.2-0422), 6 objects did not meet our minimum analysis requirements
and were thus insufficient for study (3C 129, A1367, A2634, A2877,
A3627, Triangulum Australis), and as discussed in \S\ref{sec:entsuppsample},
Coma and Fornax were intentionally ignored. This left a total of 82
\hifl\ objects which we analyzed, 59 from the primary sample ($\sim
94\%$ complete) and 23 from the extended sample ($\sim 50\%$
complete). The primary sample is the more complete of the two, thus we
focus our following discussion on the primary sample only.

The clusters missing from the primary \hifl\ sample are A1367, A2634,
Coma, and Fornax. The extent to which these 4 clusters can change our
analysis of the \kna\ distribution for \hifl\ is limited.  To alter or
wash-out bimodality, all 4 clusters would need to fall in the range
$\kna = 20-40 \ent$, which is certainly not the case for any of these
clusters. A1367 has been studied by \citet{1998ApJ...500..138D} and
\citet{2002ApJ...576..708S}, with both finding that two sub-clusters
are merging in the cluster. The merger process, and the potential for
associated shock formation, is known to create large increases of gas
entropy \citep{2007MNRAS.376..497M}. Given the combination of low
surface brightness, moderate temperatures ($kT_X = 3.5-5.0$ keV), lack
of a temperature gradient, ongoing merger, and presence of a shock, it
is unlikely A1367 has a core entropy $\la 40 \ent$. A2634 is a very
low surface brightness cluster with the bright radio source 3C 465 at
the center of an X-ray coronae \citep{coronae}. Clusters with
comparable properties to A2634 are not found to have $\kna \la 40
\ent$. Coma and Fornax are known to have core entropy $> 40 \ent$
\citep[][C. Scharf, private communication]{2008arXiv0802.1864R}.

Shown in Figure \ref{fig:hiflk0} are the log-binned (top panel) and
cumulative (bottom panel) \kna\ distributions of the \hifl\ primary
sample. The bimodality seen in the full \accept\ collection is also
present in the \hifl\ sub-sample. Mean best-fit parameters are given
in Table \ref{tab:bfparams}. We again performed two KMM tests: one
test with, and another test without, clusters having $\kna \le 4
\ent$. For the test including $\kna \le 4 \ent$ clusters we find
populations at \hiflkmma\ (\hiflkmmc\ clusters) and \hiflkmmb\
(\hiflkmmd\ clusters) with \hiflkmme. Excluding clusters with $\kna
\le 4 \ent$ we find peaks at \hiflkmmf\ and \hiflkmmg, each having
\hiflkmmh\ and \hiflkmmi\ clusters, respectively. The probability
these populations are best described by a unimodal distribution is
\hiflkmmj.

\begin{figure}[htp]
  \begin{center}
    \begin{minipage}[htp]{\linewidth}
      \includegraphics*[width=\textwidth, trim=20mm 10mm 10mm 10mm, clip]{hifl_k0hist}
      \caption[Histogram and cumulative distribution of best-fit
        \kna\ for primary \hifl\ sample.]{{\it{Top panel:}} Histogram
        of best-fit \kna\ values for the primary \hifl\ sample. Bin
        widths are 0.15 in log space.  {\it{Bottom panel:}} Cumulative
        distribution of best-fit \kna\ values. The distinct bimodality
        seen in the full \accept\ sample (Fig. \ref{fig:k0hist}) is
        also present in the \hifl\ subsample and shares the same gap
        starting at $\kna \approx 30 \ent$. That bimodality is present
        in both samples is strong evidence it is not a result of an
        unknown archival bias.}
      \label{fig:hiflk0}
    \end{minipage}
  \end{center}
\end{figure}

\citet{2007hvcg.conf...42H} note a similar core entropy bimodality to
the one we find here. \citet{2007hvcg.conf...42H} discuss two distinct
groupings of objects in a plot of average cluster temperature versus
core entropy, with the dividing point being $K \approx 40 \ent$. Shown
in the left panel of Figure \ref{fig:hifltx} is a reproduction of the
\citet{2007hvcg.conf...42H} figure except using results from our
analysis. Our results agree with the findings of
\citet{2007hvcg.conf...42H} with the gap in \kna\ occurring at $\kna
\approx 40 \ent$. While the gaps of \accept\ and \hifl\ do not cover
the same \kna\ range, it is interesting that both gaps are the deepest
around $\kna \approx 30 \ent$. That bimodality is present in both
\accept\ and the unbiased \hifl\ sub-sample suggests bimodality cannot
be the result of simple archival bias.

\begin{figure}[htp]
  \begin{center}
    \begin{minipage}[htp]{\linewidth}
      \includegraphics*[width=\textwidth, trim=0mm 0mm 0mm 0mm, clip]{hifl_txk0}
      \caption[Best-fit \kna\ versus average cluster temperature for
        the \hifl\ sample.]{Best-fit \kna\ versus average cluster
        temperature for all the objects in \hifl\ sample we have
        analyzed. This figure is a reproduction of the
        \cite{2007hvcg.conf...42H} figure except using our best-fit
        \kna\ values for the \hifl\ clusters which we analyzed. We
        have color-coded for the points based on the \kna\ value: blue
        points are $\kna \le 30 \ent$, green points have $30 \ent <
        \kna \le 60 \ent$, and red points have $\kna > 60 \ent$. The
        dashed line marks $\kna = 40 \ent$. In both panels the dashed
        lines mark $K = 40 \ent$. We do not know if the same clusters
        are shown in both plots, but the bimodality noted in
        \cite{2007hvcg.conf...42H} is also evident in our data and
        occurs at approximately the same location along the entropy
        axis.}
    \label{fig:hifltx}
    \end{minipage}
  \end{center}
\end{figure}

%%%%%%%%%%%%%%%%%%%%%%%%%%%%%%%%%%%%%%%%%%%%%%%
\subsection{Distribution of Core Cooling Times}
\label{sec:entsupphifl}
%%%%%%%%%%%%%%%%%%%%%%%%%%%%%%%%%%%%%%%%%%%%%%%

In the X-ray regime, cooling time and entropy are related in that
decreasing gas entropy also means shorter cooling time. Thus, if the
\kna\ distribution is bimodal, the distribution of cooling times
should also be bimodal. We have calculated cooling time profiles from
the spectral analysis using the relation
\begin{equation}
\tcool = \frac{3nkT_X}{2\nelec \nH \Lambda(T,Z)}
\label{eqn:tcool}
\end{equation}
where $n$ is the total ion number ($\approx 2.3\nH$ for a fully
ionized plasma), \nelec\ and \nH\ are the electron and proton
densities respectively, $\Lambda(T,Z)$ is the cooling function for a
given temperature and metal abundance, and $3/2$ is a constant
associated with isochoric cooling. The values of the cooling function
for each temperature profile bin were calculated in \xspec\ using the
flux of the best-fit spectral model. Following the procedure discussed
in \S\ref{sec:entsuppkpr}, $\Lambda$ and $kT_X$ were interpolated across the
radial grid of the electron density profile. The cooling time profiles
were then fit with a simple model analogous to that used for fitting
$K(r)$:
\begin{equation}
\tcool(r) = t_{c0} + t_{100} \left(\frac{r}{100 \kpc}\right)^{\alpha}
\label{eqn:tc0}
\end{equation}
where $t_{c0}$ is core cooling time and $t_{100}$ is a normalization
at 100 kpc.

The \kna\ distribution can also be used to explore the distribution of
core cooling times. Assuming free-free interactions are the dominant
gas cooling mechanism (\ie\ $\epsilon \propto T^{1/2}$),
\citet{radioquiet} show that entropy is related to cooling time via
the formulation:
\begin{equation}
t_{c0}(\kna) \approx 10^8 \yrs\ \left(\frac{\kna}{10 \keV \cmsq}\right)^{3/2} \left(\frac{kT_X}{5 \keV}\right)^{-1}.
\label{eqn:tck0}
\end{equation}

Shown in Figure \ref{fig:t0} is the logarithmically binned and
cumulative distributions of best-fit core cooling times from
eqn. \ref{eqn:tc0} (top panel) and core cooling times calculated using
eqn. \ref{eqn:tck0} (bottom panel). The bin widths in both histograms
are 0.20 in log-space. The pile-up of cluster core cooling times below
1 Gyr is well known, \eg\ \citet{hu85} and more recently
\citet{dunn08}, and the cooling times we calculate are consistent with
the results of other cooling time studies,
\eg\ \citet{1998MNRAS.298..416P} and \citet{2008arXiv0802.1864R}.

Most important about Figure \ref{fig:t0} is that the distinct
bimodality of the \kna\ distribution is also present in best-fit core
cooling time, $t_{c0}$. A KMM bimodality test of $t_{c0}$ found peaks
at \tckmma\ and \tckmmb\ with \tckmmc\ and \tckmmd\ in each respective
population. The probability that the unimodal distribution is a better
fit is once again exceedingly small, \tckmme.

\begin{figure}[htp]
  \begin{center}
    \begin{minipage}[htp]{0.8\linewidth}
      \includegraphics*[width=\textwidth, trim=20mm 10mm 10mm 10mm, clip]{t0}
    \end{minipage}
    \begin{minipage}[htp]{0.8\linewidth}
      \includegraphics*[width=\textwidth, trim=20mm 10mm 10mm 10mm, clip]{k0cool}
    \end{minipage}
    \caption[Histograms and cumulative distributions of cooling
      times.]({{\bfseries{Top panel:}}) Log-binned histogram and
      cumulative distribution of best-fit core cooling times, $t_{c0}$
      (eqn. \ref{eqn:tc0}), for all the clusters in \accept. Histogram
      bin widths are 0.2 in log space. ({\bfseries{Bottom panel:}})
      Log-binned histogram and cumulative distribution of core cooling
      times calculated from best-fit \kna\ values, $t_{c0}(\kna)$
      (eqn. \ref{eqn:tck0}), for all the clusters in
      \accept. Histogram bin widths are 0.2 in log space. The
      bimodality we observe in the \kna\ distribution is also present
      in best-fit $t_{c0}$. However, the gaps between the two
      populations of $t_{c0}$ and $t_{c0}(\kna)$ differ by $\sim 0.3$
      Gyrs with scaled-entropy having the more pronounced gap with a
      shorter cooling time.}
    \label{fig:t0}
  \end{center}
\end{figure}

But while $t_{c0}$ is bimodal, the gaps in the $t_{c0}$ and
$t_{c0}(\kna)$ are offset from each other. The gap in $t_{c0}$ occurs
in the range $\sim 1-2$ Gyrs, while the gap in $t_{c0}(\kna)$ occurs
in the range $\sim 0.7-1.0$. It is also interesting that the
bimodality in $t_{c0}(\kna)$ is more abrupt and deeper than it is in
$t_{c0}$. The offset gaps and differing sharpness of the two
distributions suggests that while bimodality occurs only below a
particular cooling time scale ($t_{c0} \la 1$ Gyr), a short core
cooling time may not be the fundamental property responsible for
bimodality. If entropy is more closely related to the physical
processes which cause bimodality than is cooling time, then that the
cooling time distribution does not present with the sharp, deep
bimodality seen in \kna\ suggests entropy is the fundamental quantity
related to bimodality.

But, since cooling time profiles are more sensitive to the resolution
of the temperature profiles than are the entropy profiles, it may be
that resolution effects are limiting the quantification of the true
cooling time of the core. For example, if our temperature
interpolation scheme is to coarse, or averaging over many small-scale
temperature fluctuations significantly increases $t_{c0}$, then
$t_{c0}$ would not be the best approximation of true core cooling
time. In which case, the core cooling times might be lower and the
sharpness and offsets of the distributions gaps may significantly
change.

%%%%%%%%%%%%%%%%%%%%%%%%%%%%%%%%%%%%%%%%%%%%%%%%%%%%%%%%%%%%
\subsection{Slope and Normalization of Power-law Components}
\label{sec:entsuppslopes}
%%%%%%%%%%%%%%%%%%%%%%%%%%%%%%%%%%%%%%%%%%%%%%%%%%%%%%%%%%%%

Beyond $r \approx 100 \kpc$ the entropy profiles show a striking
similarity in the slope of the power-law component which is
independent of \kna. For the full sample, the mean value of \alphafs.
For clusters with $\kna < 50 \ent$, the mean \alphaga, and for
clusters with $\kna \geq 50 \ent$, the mean \alphagb. Our mean slope of
$\alpha \approx 1.2$ is not statistically different from the
theoretical value of 1.1 found by \citet{tozzi01}. For the full
sample, the mean value of \khunfs. Again distinguishing between
clusters below and above $\kna\ = 50 \ent$, we find \khunga\ and
\khungb, respectively. Scaling each entropy profile by the cluster
virial temperature and virial radius considerably reduces the
dispersion in \khun, but we reserve detailed discussion of scaling
relations for future work.

%%%%%%%%%%%%%%%%%%%%%%%%%%%%%%%%%
\section{Summary and Conclusions}
\label{sec:entsuppsummary}
%%%%%%%%%%%%%%%%%%%%%%%%%%%%%%%%%

We have presented intracluster medium entropy profiles for a sample of
\entsuppnum\ galaxy clusters (\expt) taken from the \chandra\ Data
Archive. We have named this project \accept\ for ``Archive of Chandra
Cluster Entropy Profile Tables.'' Our analysis software, reduced data
products, data tables, figures, cluster images, and results of our
analysis for all clusters and observations are freely available at the
\accept\ web site: \url{http://www.pa.msu.edu/astro/MC2/accept}. We
encourage observers and theorists to utilize this library of entropy
profiles in their own work.

We created radial temperature profiles using spectra extracted from a
minimum of three concentric annuli containing 2500 counts each and
extending to either the chip edge or $0.5 R_{180}$, whichever was
smaller. We deprojected surface brightness profiles extracted from
$5\arcs$ bins over the energy range 0.7-2.0 keV to obtain the electron
gas density as a function of radius. Entropy profiles were calculated
from the density and temperature profiles as $K(r) =
T(r)n(r)^{-2/3}$. Two models for the entropy distribution were then
fit to each profile: a power-law only model (eqn. \ref{eqn:plaw}) and
a power-law which approaches a constant value at small radii
(eqn. \ref{eqn:k0}).

We have demonstrated that the entropy profiles for the majority of
\accept\ clusters are well-represented by the model which approaches a
constant entropy, \kna, in the core. The entropy profiles of
\accept\ are also remarkably similar at radii greater than 100 kpc,
and asymptotically approach the self-similar pure-cooling curve ($r
\propto 1.1$) with a slope of \alphafs\ (the dispersion here is in the
sample, not in the uncertainty of the measurement). We also find that
the distribution of \kna\ for the full archival sample is bimodal with
the two populations separated by a poorly populated region at $\kna
\approx 30-60 \ent$. After culling out the primary \hifl\ sub-sample
of \citet{hiflugcs1}, we find the \kna\ distribution of this complete
sub-sample to be bimodal, refuting the possibility of archival bias.

Two core cooling times were derived for each cluster: (1) cooling time
profiles were calculated using eqn. \ref{eqn:tcool} and each cooling
time profile was then fit with eqn. \ref{eqn:tc0} returning a best-fit
core cooling time, $t_{c0}$; (2) Using best-fit \kna\ values, entropy
was converted to a core cooling time, $t_{c0}(\kna)$ using
eqn. \ref{eqn:tck0}. We find the distributions of both core cooling
times to be bimodal. Comparison of the core cooling times from method
(1) and (2) reveals that the gap in the bimodal cooling time
distributions occur over different timescales, $\sim 2-3$ Gyrs for
$t_{c0}$, and $\sim 0.7-1$ for $t_{c0}(\kna)$, and that the bimodality
of $t_{c0}(\kna)$ is more abrupt. We speculate these two results
indicate ICM entropy, and not ICM cooling time, is the fundamental
quantity related to bimodality.

After analyzing an ensemble of artificially redshifted entropy
profiles, we find the lack of $\kna \la 10 \ent$ clusters at $z > 0.1$
is most likely a result of resolution effects. Investigation of
possible systematics affecting best-fit \kna\ values, such as profile
curvature and number of profile bins, revealed no trends which would
significantly affect our results. We come to the conclusion that
\kna\ is an acceptable measure of average core entropy and is not
overly influenced by profile shape or radial resolution. We also find
that $\sim90\%$ of the sample clusters have a best-fit \kna\ more than
$3\sigma$ away from zero.

Our results regarding non-zero core entropy and \kna\ bimodality fit
nicely into the sharpening picture of how feedback and radiative
cooling in clusters alter global cluster properties and affect massive
galaxy formation. Among the many models of AGN feedback,
\citet{agnframework} put forth a model which specifically addresses
how AGN outbursts generate and sustain non-zero core entropy in the
regime of $\kna \la 70 \ent$. In addition, if electron thermal
conduction is an important process in clusters, \citet{radioquiet},
\citet{agnframework}, and \citet{conduction} propose there exists a
critical entropy threshold below which conduction is no longer
efficient at wiping out thermal instabilities. The consequences of
which should be a bimodal core entropy distribution and a sensitivity
of cooling by-product formation (like star formation and AGN activity)
to this entropy threshold. We show in Chapter \ref{ch:harad} that
indicators of feedback like \halpha\ and radio emission are extremely
sensitive to the lower-bound of the bimodal gap at $\kna \approx 30
\ent$. If mergers and some other unknown mechanism are capable of
producing cluster cores with $\kna > 70 \ent$ and $> 100 \ent$, then
taking all of these processes in concert, a closed-loop picture of the
ICM's entropy life-cycle is starting to emerge.

However, the details are still missing and there are many open
questions regarding the evolution of the ICM and formation of thermal
instabilities in cluster cores: How are clusters with $\kna > 100
\ent$ produced, is fine-tuned pre-heating still the answer? What are
the role of MHD instabilities, \eg\ MTI \citep{2000ApJ...534..420B,
2008ApJ...673..758Q} and HBI \citep{2008ApJ...677L...9P}, in shaping
the ICM?  Are the compact X-ray sources we find at the cores of some
BCGs truly coronae? If so, how did they form and survive in the harsh
ICM? And can their properties be used to constrain the effects of
conduction? We hope \accept\ will be a useful resource in answering
these questions.

%%%%%%%%%%%%%%%%%%%%%%%%%%
\section{Acknowledgements}
%%%%%%%%%%%%%%%%%%%%%%%%%%

K. W. C. thanks Chris Waters for supplying and supporting his new KMM
bimodality code. K. W. C. was supported in this work through
\chandra\ X-ray Observatory Archive grants AR-6016X and
AR-4017A. M. D. acknowledges support from the NASA LTSA program
NNG-05GD82G. The \chandra\ X-ray Observatory Center is operated by the
Smithsonian Astrophysical Observatory for and on behalf of NASA under
contract NAS8-03060. This research has made use of software provided
by the Chandra X-ray Center in the application packages \ciao, \chips,
and \sherpa. This research has made use of the NASA/IPAC Extragalactic
Database which is operated by the Jet Propulsion Laboratory,
California Institute of Technology, under contract with NASA. This
research has also made use of NASA's Astrophysics Data System. Some
software was obtained from the High Energy Astrophysics Science
Archive Research Center, provided by NASA's Goddard Space Flight
Center.

%%%%%%%%%%%%%%%%%%%%%%%%%%%%%%%%%%%%
\section{Supplemental Cluster Notes}
\label{sec:haradsupp}
%%%%%%%%%%%%%%%%%%%%%%%%%%%%%%%%%%%%

\begin{figure}[htp]
  \begin{center}
    \begin{minipage}[htp]{0.9\linewidth}
      \includegraphics*[width=\textwidth, trim=15mm 10mm 10mm 10mm, clip]{beta.eps}
      \caption[Surface brightness profiles for clusters requiring a
        $\beta$-model fit for deprojection]{Surface brightness profiles for clusters requiring a
        $\beta$-model fit for deprojection (discussed in
        \S\ref{sec:entsuppbeta}). The best-fit $\beta$-model for each cluster
        is overplotted as a dashed line. The discrepancy between the
        data and best-fit model for some clusters results from the
        presence of a compact X-ray source at the center of the
        cluster. These cases are discussed belows.}
      \label{fig:betamods}
    \end{minipage}
  \end{center}
\end{figure}

\begin{description}
\item[Abell 119 ($z=0.0442$):] This is a highly diffuse cluster
  without a prominent cool core. The large core region and slowly
  varying surface brightness made deprojection highly unstable. We
  have excluded a small source at the very center of the BCG. The
  exclusion region for the source is $\approx 2.2\arcs$ in radius
  which at the redshift of the cluster is $\sim 2$ kpc. This cluster
  required a double $\beta$-model.

\item[Abell 160 ($z=0.0447$):] The highly asymmetric, low surface
  brightness of this cluster resulted in a noisy surface brightness
  profile that could not be deprojected. This cluster required a
  double $\beta$-model. The BCG hosts a compact X-ray source. The
  exclusion region for the compact source has a radius of $\sim
  5\arcs$ or $\sim 4.3$ kpc. The BCG for this cluster is not
  coincident with the X-ray centroid and hence is not at the
  zero-point of our radial analysis.

\item[Abell 193 ($z=0.0485$):] This cluster has an azimuthally
  symmetric and a very diffuse ICM centered on a BCG which is
  interacting with a companion galaxy. In Fig. \ref{fig:betamods} one
  can see that the central three bins of this cluster's surface
  brightness profile are highly discrepant from the best-fit
  $\beta$-model. This is a result of the BCG being coincident with a
  bright, compact X-ray source. As we have concluded in
  \ref{sec:entsuppcentsrc}, compact X-ray sources are excluded from our
  analysis as they are not the focus of our study here. Hence we have
  used the best-fit $\beta$-model in deriving $K(r)$ instead of the
  raw surface brightness.

\item[Abell 400 ($z=0.0240$):] The two ellipticals at the center of
  this cluster have compact X-ray sources which are excluded during
  analysis. The core entropy we derive for this cluster is in
  agreement with that found by \cite{2006A&A...453..433H} which
  supports the accuracy of the $\beta$-model we have used.

\item[Abell 1060 ($z=0.0125$):] There is a distinct compact source
  associated with the BCG in this cluster. The ICM is also very faint
  and uniform in surface brightness making the compact source that
  much more obvious. Deprojection was unstable because of imperfect
  exclusion of the source.

\item[Abell 1240 ($z=0.1590$):] The surface brightness of this cluster
  is well-modeled by a $\beta$-model. There is nothing peculiar worth
  noting about the BCG or the core of this cluster.

\item[Abell 1736 ($z=0.0338$):] Another ``boring'' cluster with a very
  diffuse low surface brightness ICM, no peaky core, and no signs of
  merger activity in the X-ray. The noisy surface brightness profile
  necessitated the use of a double $\beta$-model. The BCG is
  coincident with a very compact X-ray source, but the BCG is offset
  from the X-ray centroid and thus the central bins are not adversely
  affected. The radius of the exclusion region for the compact source
  is $\approx 2.3\arcs$ or $1.5$ kpc.

\item[Abell 2125 ($z=0.2465$):] Although the ICM of this cluster is
  very similar to the other clusters listed here (\ie\ diffuse, large
  cores), A2125 is one of the more compact clusters. The presence of
  several merging sub-clusters \citep{1997ApJ...487L..13W,
    2004ApJ...611..821W} to the NW of the main cluster form a diffuse
  mass which cannot rightly be excluded. This complication yields
  inversions of the deprojected surface brightness profile if a double
  $\beta$-model is not used.

\item[Abell 2255 ($z=0.0805$):] This is a very well studied merger
  cluster \citep{1995ApJ...446..583B, 1997A&A...317..432F}. The core
  of this cluster is very large ($r > 200$ kpc). Such large extended
  cores cannot be deprojected using our methods because if too many
  neighboring bins have approximately the same surface brightness,
  deprojection results in bins with negative or zero value. The
  surface brightness for this cluster is well modeled as a $\beta$
  function.

\item[Abell 2319 ($z=0.0562$):] A2319 is another well studied merger
  cluster \citep{1997NewA....2..501F, 1999ApJ...525L..73M} with a very
  large core region ($r > 100$ kpc) and a prominent cold front
  \citep{2004ApJ...604..604O}. Once again, the surface brightness
  profile is well-fit by a $\beta$-model.

\item[Abell 2462 ($z=0.0737$):] This cluster is very similar in
  appearance to A193: highly symmetric ICM with a bright, compact
  X-ray source embedded at the center of an extended diffuse ICM. The
  central compact source has been excluded from our analysis with a
  region of radius $\approx 1.5\arcs$ or $\sim 3$ kpc. The central
  bin of the surface brightness profile is most likely boosted above
  the best-fit double $\beta$-model because of faint extended emission
  from the compact source which cannot be discerned from the ambient
  ICM.

\item[Abell 2631 ($z=0.2779$):] The surface brightness profile for
  this cluster is rather regular, but because the cluster has a large
  core it suffers from the same unstable deprojection as A2255 and
  A2319. The ICM is symmetric about the BCG and is incredibly uniform
  in the core region. We did not detect or exclude a source at the
  center of this cluster, but under heavy binning the cluster image
  appears to have a source coincident with the BCG, and the slightly
  higher flux in central bin of the surface brightness profile may be
  a result of an unresolved source.

\item[Abell 3376 ($z=0.0456$):] The large core of this cluster ($r >
  120$ kpc) makes deprojection unstable and a $\beta$-model must be
  used.

\item[Abell 3391 ($z=0.0560$):] The BCG is coincident with a compact
  X-ray source. The source is excluded using a region with radius
  $\approx 2\arcs$ or $\sim 2$ kpc. The large uniform core region
  made deprojection unstable and thus required a $\beta$-model fit.

\item[Abell 3395 ($z=0.0510$):] The surface brightness profile for
  this cluster is noisy resulting in deprojection inversions and
  requiring a $\beta$-model fit. The BCG of this cluster has a compact
  X-ray source and this source was excluded using a region with radius
  $\approx 1.9\arcs$ or $\sim 2$ kpc.

\item[MKW 08 ($z=0.0270$):] MKW 08 is a nearby large group/poor
  cluster with a pair of interacting elliptical galaxies in the
  core. The BCG falls directly in the middle of the ACIS-I detector
  gap. However, despite the lack of proper exposure, CCD dithering
  reveals that a very bright X-ray source is associated with the
  BCG. A double $\beta$-model was necessary for this cluster because
  the low surface brightness of the ICM is noisy and deprojection is
  unstable.

\item[RBS 461 ($z=0.0290$):] This is another nearby large group/poor
  cluster with an extended, diffuse, axisymmetric, featureless ICM
  centered on the BCG. The BCG is coincident with a compact source
  with size $r \approx 1.7$ kpc. This source was excluded during
  reduction. The $\beta$-model is a good fit to the surface brightness
  profile.
\end{description}

%% %%%%%%%%%%%%%%%%%%%%%%%%%%%%%%%%%%%%%%%%%%%%%%%%%%%%%%%
%% \section{Notes on clusters with central source removed}
%% \label{app:centsrc}
%% %%%%%%%%%%%%%%%%%%%%%%%%%%%%%%%%%%%%%%%%%%%%%%%%%%%%%%%

%% 2PIGG_J0011.5-2850, 3C_388, 4C_55.16, ABELL_0223, ABELL_0426, ABELL_0539, ABELL_0562,
%% ABELL_0576, ABELL_0611, ABELL_0744, ABELL_2052, ABELL_2151,
%% ABELL_2717, ABELL_3112, ABELL_3558, ABELL_3581, ABELL_3822,
%% CYGNUS_A, HYDRA_A, M87, MACS_J0547.0-3904, MACS_J1931.8-2634,
%% RBS_0797, RX_J1320.2+3308, ZwCl_0857.9+2107, ZWICKY_1742

%% The clusters A119, A160, A193, A1736, A2462, A3391, A3395, and RBS461
%% also have a central source removed during analysis, but they are
%% discussed in Appendix \ref{app:beta}.

%% \begin{description}
%% \item[3C 295 ($z=0.4641$):] The core of this cluster has been
%% studied in detail by \cite{2001MNRAS.324..842A}. In the central 50 kpc
%% \cite{2001MNRAS.324..842A} found, as we do, that the temperature drops
%% from $\sim 5.0$ keV to $\sim 3.5$ keV. \cite{2001MNRAS.324..842A} also
%% derive a mass deposition rate of $\dot{M} = 280~\msolpy$ indicating
%% the core of this cluster has a strong cooling flow. As was done in
%% \cite{2001MNRAS.324..842A}, three sources are excluded from the core
%% during our analysis: the region surrounding the central AGN and two
%% nearby radio hot spots \citep{2000ApJ...530L..81H}.

%% \item[3C 388 ($z=0.0917$):]
%% From \cite{2006ApJ...639..753K}:
%% in process of CF quenching
%% The radio galaxy 3C 388 is classified as a Fanaroff-Riley type II (FR
%% II) radio galaxy, although its luminosity ( W Hz−1; Fanaroff \& Riley
%% 1974) lies near the FR I/II dividing line. The radio morphology of
%% this source is closer to a “fat double” (Owen \& Laing 1989) than the
%% canonical FR II “classical double” such as Cyg A and 3C 98. Optically,
%% the nucleus of 3C 388 is classified as a low-excitation radio galaxy
%% (Jackson \& Rawlings 1997). Multifrequency VLA observations of 3C 388
%% show significant structure in spectral index maps that has been
%% interpreted as evidence for multiple nuclear outbursts (Roettiger et
%% al. 1994). Previous X-ray observations of this radio galaxy have shown
%% that it is embedded in a cluster environment (Feigelson \& Berg 1983;
%% Hardcastle \& Worrall 1999; Leahy \& Gizani 2001). The local galaxy
%% environment is extremely dense (Prestage \& Peacock 1988), and the
%% central elliptical galaxy that hosts 3C 388 is one of the most
%% luminous (MB=-24.24) in the local universe (Owen \& Laing 1989; Martel et
%% al. 1999).

%% \item[4C 55.16 ($z=0.2420$):]
%% From \cite{2001MNRAS.328L...5I}:
%% 4C+55.16 is a compact powerful radio source residing in a large galaxy
%% at a redshift of 0.240 (Pearson \& Readhead 1981, 1988; Whyborn et
%% al. 1985; Hutchings, Johnson \& Pyke 1988). Recently, luminous cluster
%% emission (~1045 erg s−1) around the radio galaxy has been recognized
%% through ASCA and ROSAT High Resolution Imager (HRI) observations
%% (Iwasawa et al. 1999).  The point source at the nucleus shows a hard
%% X-ray spectrum, which can be attributed naturally to non-thermal
%% emission from the active nucleus.

%% \item[Abell 223 ($z=0.2070$):]
%% From \cite{2000A&A...355..443P}:
%% As already noticed by Sandage et al. (1976), these two neighboring
%% clusters have nearly the same redshift and probably constitute an
%% interacting system which is going to merge in the future. Both are
%% dominated by a particularly bright cD galaxy. They have a richness
%% class R=3 and are X-ray luminous with [FORMULA] and [FORMULA] for A
%% 222 and A 223, respectively (Lea \& Henry 1988). The BOW83 sample
%% covers only the central regions of these two clusters and, in order to
%% study the galaxy distribution in these systems, as well as to estimate
%% the projected density for the galaxies in our sample (see below), we
%% have built a more extensive, although shallower, galaxy catalog,
%% covering a region of [FORMULA] centered on the median position of the
%% two clusters. This catalogue, with 356 objects, was extracted from
%% Digital Sky Survey (DSS) images, using the software SExtractor (Bertin
%% \& Arnouts 1996). It is more than 90\% complete to BOW83 magnitudes
%% [FORMULA].

%% \item[Abell 426 ($z=0.0179$):]
%% Come on, it's Perseus, you don't know about this cluster? Well, it's
%% got an AGN, just ask \cite{perseus1, perseus2, perseus3}.

%% \item[Abell 539 ($z=0.0288$):]
%% From \cite{1988AJ.....96.1775O}:
%% Within 1 Mpc of the center, the physical parameters of A539 are found
%% to be typical of those of rich clusters. It is shown that early-type
%% galaxies are more concentrated toward the cluster center and that the
%% velocity distributions of early-type and late-type galaxies differ
%% marginally.

%% \item[Abell 562 ($z=0.1100$):]
%% From \cite{1997ApJ...474..580G}:
%% The X-ray emission from this cluster is elongated and shows the radio
%% source offset from the central X-ray peak by 30''. The substructure
%% test (Table A1) detects a significant X-ray excess east of the
%% WAT. The radio pressure is in rough agreement with the thermal
%% pressure. Optically, the cluster is dominated by the WAT host galaxy.

%% \item[Abell 576 ($z=0.0385$):]
%% From \cite{1996ApJ...470..724M}:
%% The central cluster region contains a nonemission galaxy population
%% and an intracluster medium which is significantly cooler (σ\_core\_ =
%% 387\_-105\_\^+250\^ km s\^-1\^ and T\_x\_ = 1.6\_-0.3\_\^+0.4\^ keV at 90\%
%% confidence) than the global populations (σ = 977\_-96\_\^+124\^ km s\^- 1\^
%% for the nonemission population and T\_X\_ > 4 keV at 90\%
%% confidence). Because (1) the low-dispersion galaxy population is no
%% more luminous than the global population and (2) the evidence for a
%% cooling flow is weak, we suggest that the core of A576 may contain the
%% remnants of a lower mass subcluster.
%% From \cite{2004ApJ...607..220K}:
%% We present data from a Chandra observation of the nearby cluster of
%% galaxies A576. The core of the cluster shows a significant departure
%% from dynamical equilibrium. We show that this core gas is most likely
%% the remnant of a merging subcluster, which has been stripped of much
%% of its gas, depositing a stream of gas behind it in the main
%% cluster. The unstripped remnant of the subcluster is characterized by
%% a different temperature, density, and metallicity than that of the
%% surrounding main cluster, suggesting its distinct origin. Continual
%% dissipation of the kinetic energy of this minor merger may be
%% sufficient to counteract most cooling in the main cluster over the
%% lifetime of the merger event.

%% \item[Abell 611 ($z=0.2880$):]
%% From \cite{2002MNRAS.337.1207G}:
%% Abell 611 is a cluster at z= 0.288 (Crawford et al. 1995) originally
%% identified by Abell (1957). It has a 0.1–2.4 keV luminosity of 8.63 ×
%% 1044 W (Böhringer et al. 2000), with a temperature of 7.95+0.56−0.52×
%% 107 K (White 2000). White derived this value from a 57-ks ASCA
%% exposure by considering both a single-phase and two-phase cooling
%% model. The temperature values found for the bulk of the gas are
%% statistically equivalent, and a mass deposit rate of 0+177−0 Mo yr−1
%% was found for the cooling model. The 17-ks ROSAT HRI observation from
%% 1996 April is shown with 8-arcsec binning in Fig. 7. The image
%% contains two bright pixels, which, on comparison with the POSS image,
%% are coincident with a large galaxy. These pixels are ignored whilst
%% fitting a model to this observation.

%% \item[Abell 744 ($z=0.0729$):]
%% From \cite{1985AJ.....90.1665K}:
%% The authors present X-ray and optical observations of the cluster of
%% galaxies Abell 744. The X-ray flux (assuming H0 = 100 km s-1Mpc-1) is
%% ≡9×1042erg s-1. The X-ray source is extended, but shows no other
%% structure. The authors present photographic photometry (in
%% Kron-Cousins R), calibrated by deep CCD frames, for all galaxies
%% brighter than 19th magnitude within 0.75 Mpc of the cluster
%% center. The luminosity function is normal, and the isopleths show
%% little evidence of substructure near the cluster center. The cluster
%% has a dominant central galaxy which the authors classify as a normal
%% brightest-cluster elliptical on the basis of its luminosity
%% profile. New redshifts were obtained for 26 galaxies in the vicinity
%% of the cluster center; 20 appear to be cluster members. The spatial
%% distribution of redshifts is peculiar; the dispersion within the 150
%% kpc core radius is much greater than outside. Abell 744 is similar to
%% the nearby cluster Abell 1060.

%% \item[Abell 2052 ($z=0.0353$):]
%% AGN \cite{2001ApJ...558L..15B, 2003ApJ...585..227B}.

%% \item[Abell 2151 ($z=0.0366$):]
%% From \cite{1995AJ....109..465M}:
%% it's a merger with three distinct pops in vel disp space.

%% \item[Abell 2717 ($z=0.0475$):]
%% From \cite{1997A&A...321...64L}:
%% We present an X-ray, radio and optical study of the cluster A
%% 2717. The central D galaxy is associated with a Wide-Angled-Tailed
%% (WAT) radio source. A Rosat PSPC observation of the cluster shows that
%% the cluster has a well constrained temperature of
%% 1.9\^+0.3\^\_-0.2\_x10\^7\^K. The pressure of the intracluster medium was
%% found to be comparable to the mininum pressure of the radio source
%% suggesting that the tails may in fact be in equipartition with the
%% surrounding hot gas.

%% \item[Abell 3112 ($z=0.0720$):]
%% It's one of Lieu's soft excess clusters \cite{2007ApJ...668..796B}
%% searching for cold gas in A3112 \cite{2004A&A...421..503L}
%% From \cite{2003ApJ...595..142T}:
%% We present the results of a Chandra observation of the central region
%% of A3112. This cluster has a powerful radio source in the center and
%% was believed to have a strong cooling flow. The X-ray image shows that
%% the intracluster medium (ICM) is distributed smoothly on large scales
%% but has significant deviations from a simple concentric elliptical
%% isophotal model near the center. Regions of excess emission appear to
%% surround two lobelike radio-emitting regions. This structure probably
%% indicates that hot X-ray gas and radio lobes are interacting. From an
%% analysis of the X-ray spectra in annuli, we found clear evidence for a
%% temperature decrease and abundance increase toward the center. The
%% X-ray spectrum of the central region is consistent with a
%% single-temperature thermal plasma model. The contribution of X-ray
%% emission from a multiphase cooling flow component with gas cooling to
%% very low temperatures locally is limited to less than 10\% of the
%% total emission. However, the whole cluster spectrum indicates that the
%% ICM is cooling significantly as a whole, but only in a limited
%% temperature range (>=2 keV). Inside the cooling radius the conduction
%% timescales based on the Spitzer conductivity are shorter than the
%% cooling timescales. We detect an X-ray point source in the cluster
%% center that is coincident with the optical nucleus of the central cD
%% galaxy and the core of the associated radio source. The X-ray spectrum
%% of the central point source can be fitted by a 1.3 keV thermal plasma
%% and a power-law component whose photon index is 1.9. The thermal
%% component is probably plasma associated with the cD galaxy. We
%% attribute the power-law component to the central active galactic
%% nucleus.

%% \item[Abell 3558 ($z=0.0480$):]
%% From \cite{2007A&A...463..839R}:
%% Combining XMM-Newton and Chandra data, we have performed a detailed
%% study of A3558. Our analysis shows that its dynamical history is more
%% complicated than previously thought. We have found some traits typical
%% of cool core clusters (surface brightness peaked at the center, peaked
%% metal abundance profile) and others that are more common in merging
%% clusters, like deviations from spherical symmetry in the thermodynamic
%% quantities of the ICM. This last result has been achieved with a new
%% technique for deriving temperature maps from images. We have also
%% detected a cold front and, with the combined use of XMM-Newton and
%% Chandra, we have characterized its properties, such as the speed and
%% the metal abundance profile across the edge. This cold front is
%% probably due to the sloshing of the core, induced by the perturbation
%% of the gravitational potential associated with a past merger. The
%% hydrodynamic processes related to this perturbation have presumably
%% produced a tail of lower entropy, higher pressure and metal rich ICM,
%% which extends behind the cold front for~500 kpc. The unique
%% characteristics of A3558 are probably due to the very peculiar
%% environment in which it is located: the core of the Shapley
%% supercluster.

%% \item[Abell 3581 ($z=0.0218$):]
%% From \cite{2007A&A...463..839R}:
%% We present results from an analysis of a Chandra observation of the
%% cluster of galaxies A3581. We discover the presence of a point-source
%% in the central dominant galaxy that is coincident with the core of the
%% radio source PKS 1404-267. The emission from the intracluster medium
%% is analysed, both as seen in projection on the sky, and after
%% correcting for projection effects, to determine the spatial
%% distribution of gas temperature, density and metallicity. We find that
%% the cluster, despite hosting a moderately powerful radio source, shows
%% a temperature decline to around 0.4 Tmax within the central 5 kpc. The
%% cluster is notable for the low entropy within its core. We test and
%% validate the XSPEC PROJCT model for determining the intrinsic cluster
%% gas properties.

%% \item[Abell 3822 ($z=0.0759$):]
%% zero literature, seriously, only mentioned in survey papers.

%% \item[Cygnus A ($z=0.0561$):]
%% AGN \cite{2002ApJ...565..195S}

%% \item[Hydra A ($z=0.0549$):]
%% AGN \cite{2000ApJ...534L.135M, 2001ApJ...557..546D,
%%   2002ApJ...568..163N}

%% \item[M87 ($z=0.0044$):]
%% AGN \cite{2002ApJ...564..683M, 2005ApJ...635..894F}

%% \item[MACS J0547.0-3904 ($z=0.2100$):]
%% no lit

%% \item[MACS J1931.8-2634 ($z=0.3520$):]
%% no lit

%% \item[RBS 797 ($z=0.3540$):]
%% From \cite{2001A&A...376L..27S}:
%% We present CHANDRA observations of the X-ray luminous, distant galaxy
%% cluster RBS797 at z=0.35. In the central region the X-ray emission
%% shows two pronounced X-ray minima, which are located opposite to each
%% other with respect to the cluster centre. These depressions suggest an
%% interaction between the central radio galaxy and the intra-cluster
%% medium, which would be the first detection in such a distant
%% cluster. The minima are symmetric relative to the cluster centre and
%% very deep compared to similar features found in a few other nearby
%% clusters. A spectral and morphological analysis of the overall cluster
%% emission shows that RBS797 is a hot cluster (T=7.7+1.2-1.0 keV) with a
%% total mass of Mtot(r500)= 6.5+1.6-1.2 *E14Msun.

%% \item[RX J1320.2+3308 ($z=0.0366$):]
%% no lit

%% \item[ZwCl 0857.9+2107 ($z=0.2350$):]
%% no lit

%% \item[Zwicky 1742 ($z=0.0757$):]
%% brand new obs

%% \end{description}

