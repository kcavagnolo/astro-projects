%%%%%%%%%%%%%%%%%%%%%%%%%%%%%%
\section{Clusters of Galaxies}
\label{sec:cofg}
%%%%%%%%%%%%%%%%%%%%%%%%%%%%%%

Of the luminous matter in the Universe, stars and galaxies are often
the most familiar to a sky gazer. Aside from the Moon and the
occasional bright planet, stars are the most abundantly obvious
patrons of the night sky. Viewed from a sufficiently dark location,
the stars form a band of light interspersed with dust and gaseous
clouds which define the Milky Way, our home galaxy. The Milky Way is
only one of more than 30 galaxies in a gravitationally bound group of
galaxies, named the Local Group, which includes the well-known, nearby
galaxy Andromeda. But in cosmological terms, the Local Group is very
small in comparison to immense structures containing thousands of
galaxies. In a turn of wit, these structures are appropriately named
clusters of galaxies, and are the focus of this dissertation.

\invisiblesymbol{\mathrm{Mpc}}{Megaparsec: A unit of length representing
one million parsecs. The parsec (pc) is a historical unit for
measuring parallax and equals $3.0857\times10^{13}$
km.}
\invisiblesymbol{z}{Dimensionless redshift: As is common in most of
astronomy, I adopt the definition of redshift using a dimensionless
ratio of wavelengths, $z = (\lambda_{\mathrm{observed}} /
\lambda_{\mathrm{rest}})-1$, where the wavelength shift occurs because
of cosmic expansion.}
\invisiblesymbol{H_0}{Hubble constant: The current ratio of recessional
velocity arising from expansion of the Universe to an object's
distance from the observer, $v = \Hn D$. \Hn\ is assumed here to be
$\sim70 \hub$. Inverted, the Hubble constant yields the present
age of the Universe, $\Hn^{-1} \approx 13.7$ billion years. $H(z)$
denotes the Hubble constant at a particular redshift, $z$.}
\invisiblesymbol{\rho_c}{Critical density: The density necessary for a
universe which has spatially flat geometry and in which the expansion
rate of spacetime balances gravitational attraction and prevents
recollapse. In terms of relevant quantities $\rho_c=3H(z)^2/8\pi G$,
with units \gpcc.}
\invisiblesymbol{\Omega_{\Lambda}}{Cosmological constant energy density
of the Universe: The ratio of energy density due to a cosmological
constant to the critical density. \OL\ is assumed here to be
$\sim0.7$.}
\invisiblesymbol{\Omega_M}{Matter density of the Universe: The ratio of
total matter density to the critical density. \OM\ is currently
measured to be $\sim0.3$.}

Galaxy clusters are the most massive gravitationally bound structures
to have yet formed in the Universe. As where galaxy groups have
roughly 10-50 galaxies, galaxy clusters have hundreds to thousands of
galaxies. When viewed through a telescope, a galaxy cluster appears as
a tight distribution of mostly elliptical and S0 spiral galaxies
within a radius of $\sim1-5$ Mpc\footnote{Throughout this
dissertation, a flat \LCDM\ cosmology of $\Hn = 70 \hub$, $\OL =
0.7$, and $\OM = 0.3$ is assumed. These values are taken from
\citet{wmap}.} of each other. Rich galaxy clusters are truly
spectacular objects, as can be seen in Figure \ref{fig:a1689} which
shows the \hubble\ Space Telescope's close-up of the strong lensing
cluster Abell 1689.

\begin{figure}[htp]
  \begin{center}
    \includegraphics*[width=\textwidth, trim=0mm 0mm 0mm 0mm, clip]{a1689.eps}
    \caption[\hubble\ image of Abell 1689]{Optical image
      of the galaxy cluster Abell 1689 as observed with the ACS instrument
      on-board the \hubble\ Space Telescope. The fuzzy yellowish spheres
      are giant elliptical (gE) galaxies in the cluster, with the gE
      nearest the center of the image being the brightest cluster galaxy
      -- ostensibly, the cluster ``center''. Image taken from NASA's
      Hubblesite.org. Image Credits: NASA, N. Benitez (JHU), T. Broadhurst
      (The Hebrew University), H. Ford (JHU), M. Clampin(STScI), G. Hartig
      (STScI), G. Illingworth (UCO/Lick Observatory), the ACS Science Team
      and ESA.}
    \label{fig:a1689}
  \end{center}
\end{figure}

Galaxy clusters are deceptively named. As with most objects in the
Universe, one of the most revealing characteristics of an object is
its mass, and the mass of clusters of galaxies is not dominated by
galaxies. A cluster of galaxies mass is dominated ($\ga 85\%$) by dark
matter with most ($\ga 80\%$) of the baryonic mass\footnote{Baryonic
is a convenient term used to describe ordinary matter like atoms or
molecules, while non-baryonic matter is more exotic like free
electrons or dark matter particles.} in the form of a hot ($kT \approx
2-15$ keV; 10-100 million degrees K), luminous ($10^{43-46} \ergps$),
diffuse ($10^{-1}-10^{-4} \cm^{-3}$) intracluster medium (ICM) which
is co-spatial with the galaxies but dwarfs them in mass
\citep{1984Natur.311..517B, 1990ApJ...356...32D}. For comparison, the
ICM in the core region of a galaxy cluster is, on average, $10^{20}$
times less dense than typical Earth air, $10^5$ times denser than the
mean cosmic density, more than 2000 times hotter than the surface of
the Sun, and shines as bright as $10^{35}$ 100 watt light bulbs.

Because of the ICM's extreme temperature, the gas is mostly ionized,
making it a plasma. For the temperature range of clusters, the ICM is
most luminous at X-ray wavelengths of the electromagnetic
spectrum. This makes observing galaxy clusters with X-ray telescopes,
like NASA's \chandra\ X-ray Observatory, a natural choice. Clusters
have masses ranging over $10^{14-15}$
\definesymbol{\mathrm{M}_{\odot}}{Mass of the Sun: One solar mass
equals $1.9891\times10^{30} \kg$} with velocity dispersions of
$500-1500 \kmps$. The ICM has also been enriched with
metals\footnote{It is common practice in astronomy to classify
``metals'' as any element with more than two protons.} to an average
value of $\sim 0.3$ solar abundance. Shown in Figure \ref{fig:bullet}
is an optical, X-ray, and gravitational lensing composite image of the
galaxy cluster 1E0657-56. This cluster is undergoing an especially
spectacular and rare merger in the plane of the sky which allows for
the separate dominant components of a cluster -- dark matter, the ICM,
and galaxies -- to be ``seen'' distinctly.

\begin{figure}[htp]
  \begin{center}
    \includegraphics*[width=\textwidth, trim=0mm 0mm 0mm 0mm, clip]{bullet}
    \caption[Composite image of the Bullet Cluster]{The
      galaxy cluster 1E0657-56, a.k.a. the Bullet Cluster. All of the
      primary components of a galaxy cluster can be seen in this image:
      the X-ray ICM (pink), dark matter (blue), and galaxies. The
      brilliant white object with diffraction spikes is a star. This
      cluster has become very famous as the merger dynamics provide direct
      evidence for the existence of dark matter
      \citep{2006ApJ...648L.109C}. Image taken from NASA Press Release
      06-297. Image credits: NASA/CXC/CfA/\citet{2002ApJ...567L..27M}
      (X-ray); NASA/STScI/Magellan/U.Arizona/\citet{2006ApJ...648L.109C}
      (Optical); NASA/STScI/ESO
      WFI/Magellan/U.Arizona/\citet{2006ApJ...648L.109C} (Lensing).}
    \label{fig:bullet}
  \end{center}
\end{figure}

As knowing the characteristics of galaxy clusters is a small part of
the discovery process, we must also wonder, why study clusters of
galaxies? Galaxy clusters have two very important roles in the current
research paradigm:
\begin{enumerate}
\item Galaxy clusters represent a unique source of information about
  the Universe's underlying cosmological parameters, including the
  nature of dark matter and the dark energy equation of
  state. Large-scale structure growth is exponentially sensitive to
  some of these parameters, and by counting the number of clusters
  found in a comoving volume of space, specifically above a given mass
  threshold, clusters may be very useful in cosmological studies
  \citep{voitreview}.
\item The cluster gravitational potential well is deep enough to
  retain all the matter which has fallen in over the age of the
  Universe. This slowly evolving ``sealed box'' therefore contains a
  comprehensive history of all the physical processes involved in
  galaxy formation and evolution, such as: stellar evolution,
  supernovae feedback, black hole activity in the form of active
  galactic nuclei, galaxy mergers, ram pressure stripping of
  in-falling galaxies and groups, {\it{et cetera}}. The time required
  for the ICM in the outskirts of a cluster to radiate away its
  thermal energy is longer than the age of the Universe, hence the ICM
  acts as a record-keeper of all the aforementioned activity.
  Therefore, by studying the ICM's physical properties, the thermal
  history of the cluster can be partially recovered and utilized in
  developing a better understanding of cluster formation and
  evolution.
\end{enumerate}
In this dissertation I touch upon both these points by studying the
emergent X-ray emission of the ICM as observed with the \chandra\
X-ray Observatory.

While clusters have their specific uses in particular areas of
astrophysics research, they also are interesting objects in their own
right. A rich suite of physics is brought to bear when studying galaxy
clusters. A full-blown, theoretical construction of a galaxy cluster
requires, to name just a few: gravitation, fluid mechanics,
thermodynamics, hydrodynamics, magnetohydrodynamics, and
high-energy/particle/nuclear physics. Multiwavelength observations of
galaxy clusters provide excellent datasets for testing the theoretical
predictions from other areas of physics, and clusters are also a
unique laboratory for empirically establishing how different areas of
physics interconnect. Just this aspect of clusters puts them in a
special place among the objects in our Universe worth intense,
time-consuming, (and sometimes expensive) scrutiny. At a minimum,
galaxy clusters are most definitely worthy of being the focus of a
humble dissertation from a fledgling astrophysicist.

As this is a dissertation focused around observational work, in
Section \S\ref{sec:icm} I provide a brief primer on the X-ray
observable properties of clusters which are important to understanding
this dissertation. Section \S\ref{sec:entintro} provides discussion of
gas entropy, a physical property of the ICM which may be unfamiliar to
many readers and is utilized heavily in Chapters \ref{ch:ent_supp} and
\ref{ch:harad}. In Section \S\ref{sec:incomplete}, I more thoroughly
discuss reasons for studying clusters of galaxies which are specific
to this dissertation. Presented in Section \S\ref{sec:ssbreak} is a
discussion of why clusters of different masses are not simply scaled
versions of one another, and in Section \S\ref{sec:cfprob} the
unresolved ``cooling flow problem'' is briefly summarized. The current
chapter concludes with a brief description of the \chandra\ X-ray
Observatory (CXO) and its instruments in Section
\S\ref{sec:chandra}. \chandra\ is the space-based telescope with which
all of the data presented in this dissertation was collected.

%%%%%%%%%%%%%%%%%%%%%%%%%%%%%%%%%
\section{The Intracluster Medium}
\label{sec:icm}
%%%%%%%%%%%%%%%%%%%%%%%%%%%%%%%%%

In Section \S\ref{sec:cofg}, the ICM was presented as a mostly
ionized, hot, diffuse plasma which dominates the baryonic mass content
of clusters. But where did it come from and what is the composition of
this pervasive ICM? What are the mechanisms that result in the ICM's
X-ray luminescence? How do observations of the ICM get converted into
physical properties of a cluster? In this section I briefly cover the
answers to these questions in order to give the reader a better
understanding of the ICM.

Galaxy clusters are built-up during the process of hierarchical merger
of dark matter halos and the baryons gravitationally coupled to those
halos \citep{white&rees}. Owing to the inefficiency of galaxy
formation and the processes of galactic mass ejection and ram pressure
stripping, many of the baryons in these dark matter halos are in the
form of diffuse gas and not locked up in galaxies. During the merger
of dark matter halos, gravitational potential energy is converted to
thermal energy and the diffuse gas is heated to the virial temperature
of the cluster potential through processes like adiabatic compression
and accretion shocks. The cluster virial temperature is calculated by
equating the average kinetic energy of a gas particle to its thermal
energy,
\begin{eqnarray}
\frac{1}{2} \mu m \langle\sigma^2\rangle &=& \frac{3}{2}k \tvir\\
\tvir &=& \frac{\mu m \langle\sigma^2\rangle}{3k}
\end{eqnarray}
where $\mu$ is the mean molecular weight, $k$ is the Boltzmann
constant, \tvir\ is the virial temperature, $m$ is the mass of a test
particle, and $\langle \sigma \rangle$ is the average velocity of the
test particle. In this equation, $\langle \sigma \rangle$ can be
replaced with the line-of-sight galaxy velocity dispersion (a cluster
observable) because all objects within the cluster potential (stars,
galaxies, protons, \etc) are subject to the same dynamics and hence
have comparable thermal and kinetic energies.

Galaxy clusters are the most massive objects presently in the
Universe. The enormous mass means deep gravitational potential wells
and hence very high virial temperatures. Most cluster virial
temperatures are in the range $k\tvir = 1-15 \keV$. At these energy
scales, gases are collisionally ionized plasmas and will emit X-rays
via \tb\ (discussed in Section \S\ref{sec:xray}). The ICM is not a
pure ionized hydrogen gas, as a result, atomic line emission from
heavy elements with bound electrons will also occur. The ICM is also
optically thin at X-ray wavelengths, \eg\ the ICM optical depth to
X-rays is much smaller than unity, $\tau_{\lambda} \ll 1$, and hence
the X-rays emitted from clusters stream freely into the Universe. In
the next section I briefly cover the processes which give rise to ICM
X-ray emission and the observables which result. For a magnificently
detailed treatise of this topic, see \citet{sarazinbook}\footnote{Also
available at
http://nedwww.ipac.caltech.edu/level5/March02/Sarazin/TOC.html} and
references therein.

%%%%%%%%%%%%%%%%%%%%%%%%%%%
\subsection{X-ray Emission}
\label{sec:xray}
%%%%%%%%%%%%%%%%%%%%%%%%%%%

Detailed study of clusters proceeds mainly through spatial and
spectral analysis of the ICM. By directly measuring the X-ray emission
of the ICM, quantities such as temperature, density, and luminosity
per unit volume can be inferred. Having this knowledge about the ICM
provides an observational tool for indirectly measuring ICM dynamics,
composition, and mass. In this way a complete picture of a cluster can
be built up and other processes, such as brightest cluster galaxy
(BCG) star formation, AGN feedback activity, or using ICM temperature
inhomogeneity to probe cluster dynamic state, can be investigated. In
this section, I briefly cover how X-ray emission is produced in the
ICM and how basic physical properties are then measured.

The main mode of interaction in a fully ionized plasma is the
scattering of free electrons off heavy ions. During this process,
charged particles are accelerated and thus emit radiation. The
mechanism is known as `free-free' emission (ff), or by the
tongue-twisting \tb\ (German for ``braking radiation''). It is also
called bremsstrahlung cooling since the X-ray emission carries away
large amounts of energy. The timescale for protons, ions, and
electrons to reach equipartition is typically shorter than the age of
a cluster \citep{2003PhPl...10.1992S}, thus the gas particles
populating the emitting plasma can be approximated as being at a
uniform temperature with a Maxwell-Boltzmann velocity distribution,
\begin{equation}
f(\vec{v}) = 4 \pi \left(\frac{m}{2 \pi k T}\right)^{3/2} \vec{v}^2
\exp \left[\frac{-m\vec{v}^2}{2k T}\right]
\end{equation}
where $m$ is mass, $T$ is temperature, $k$ is the Boltzmann constant,
and velocity, $\vec{v}$, is defined as $\vec{v} = \sqrt{v_x^2 + v_y^2
  + v_z^2}$. The power emitted per cubic centimeter per second (erg
$\ps \pcc$) from this plasma can be written in the compact form
\begin{equation}
\label{eqn:ff}
\epsilon^{ff} \equiv 1.4\times10^{-27} T^{1/2} n_{e} n_{i} Z^{2} \bar{g}_B
\end{equation}
where $1.4\times10^{-27}$ is in cgs and is the condensed form of the
physical constants and geometric factors associated with integrating
over the power per unit area per unit frequency, $n_e$ and $n_i$ are
the electron and ion densities, $Z$ is the number of protons of the
bending charge, $\bar{g}_B$ is the frequency averaged Gaunt factor (of
order unity), and $T$ is the global temperature determined from the
spectral cut-off frequency \citep{rybicki}. Above the cut-off
frequency, $\nu_c=kT/\hbar$, few photons are created because the
energy supplied by charge acceleration is less than the minimum energy
required for creation of a photon. Worth noting is that free-free
emission is a two-body process and hence the emission goes as the gas
density squared while having a weak dependence on the thermal energy,
$\epsilon \propto \rho^2 T^{1/2}$ for $T \ga 10^7$ K when the gas has
solar abundances.

Superimposed on the thermal emission of the plasma are emission lines
of heavy element contaminants such as C, Fe, Mg, N, Ne, O, S, and
Si. The widths and relative strengths of these spectral lines are used
to constrain the metallicity of the ICM, which is typically quantified
using units relative to solar abundance,
\definesymbol{Z_{\odot}}{Metal abundance of the Sun: Individual
elemental abundances can be found in \citet{ag89}.}. On average, the
ICM has a metallicity of $\sim 0.3~\Zsol$, which is mostly stellar
detritus \citep{icmmetal1, icmmetal2, icmmetal3}. In collisionally
ionized plasmas with temperatures and metallicities comparable to the
ICM, the dominant ion species is that of the `closed-shell'
helium-like ground state (K and L-shells) \citep{cfreview}. The K and
L shell transitions are extremely sensitive to temperature and
electron densities, therefore providing an excellent diagnostic for
constraining both of these quantities. The strongest K-shell
transition of the ICM can be seen from iron at $kT \sim 6.7$ keV. If
signal-to-noise of the spectrum is of high enough quality, measuring a
shift in the energy of this spectral line can be used to confirm or
deduce the approximate redshift, and hence distance, of a cluster. The
rich series of iron L-shell transitions occur between $0.2 \la T \la
2.0$ keV and are the best diagnostic for measuring metallicity. For
the present generation of X-ray instruments, the L-shell lines are
seen as a blend with a peak around $\sim 1$ keV.

Shown in Figs. \ref{fig:brem} and \ref{fig:brem2} are the unredshifted
synthetic spectral models generated with \xspec\ \citep{xspec} of a 2
keV and 8 keV gas. Both spectral models have a component added to
mimic absorption by gas in the Milky Way, which is seen as attenuation
of flux at $E \la 0.4$ keV. For both spectral models the metal
abundance is $0.3~\Zsol$. These two spectral models differ by only a
factor of four in temperature but note the extreme sensitivity of both
the \tb\ exponential cut-off and emission line strengths to
temperature.

\begin{figure}[htp]
  \begin{center}
    \begin{minipage}[htp]{0.8\linewidth}
      \includegraphics*[width=0.7\textwidth,trim=10mm 0mm 0mm 10mm,angle=270,clip]{brem}
      \caption[Synthetic spectral model of $kT_X =2.0$ keV gas.]{Synthetic
        absorbed thermal spectral model of a \definesymbol{N_H}{Neutral
          hydrogen column density: The Galaxy is rich with metals such as C,
          N, O, S, and Si which absorb incoming extragalactic soft X-ray
          radiation. The density of neutral hydrogen is assumed to be a
          surrogate for the density of metals. Photoelectric absorption models
          are used to quantify the attenuation of soft X-rays, and typically
          take as input the column density (\pcmsq) of neutral hydrogen in a
          particular direction. \nhi\ is related to the number density, $n_H$
          (\pcc), along the line of sight, $dl$, as $\nhi = \int n_H dl$.}$=
        10^{20} \pcmsq$, $kT_X=2.0$ keV, $Z/\Zsol=0.3$, and zero redshift
        gas. Notice that the strength of the iron L-shell emission lines is
        much greater than the iron K-shell lines for this model.}
      \label{fig:brem}
    \end{minipage}
    \begin{minipage}[htp]{0.8\linewidth}
      \includegraphics*[width=0.7\textwidth,trim=10mm 0mm 0mm 10mm,angle=270,clip]{brem2}
      \caption[Synthetic spectral model of $kT_X=8.0$ keV gas.]{Same as
        Fig. \ref{fig:brem} except for a $kT_X=8.0$ keV gas. Notice that for
        this spectral model the iron L-shell emission lines are much weaker
        and the iron K-shell lines are much stronger than in the $kT_X=2.0$
        keV model. Also note that the exponential cut-off of this model
        occurs at a higher energy ($E > 10$ keV) than in the model shown in
        Figure \ref{fig:brem}.}
      \label{fig:brem2}
    \end{minipage}
  \end{center}
\end{figure}

Equation \ref{eqn:ff} says that observations of ICM X-ray emission
will yield two quantities: temperature and density. The gas density
can be inferred from the {\it{emission integral}},
\begin{equation}
\label{eqn:ei}
EI = \int n_e n_p~dV
\end{equation}
where $n_e$ is the electron density, $n_p$ is the density of
hydrogen-like ions, and $dV$ is the gas volume within a differential
element. The emission integral is essentially the sum of the square of
gas density for all the gas parcels in a defined region. Thus, the gas
density within a projected volume can be obtained from the spectral
analysis, but it can also be obtained from spatial analysis of the
cluster emission, for example from cluster surface brightness.

The number of photons detected per unit area (projected on the plane
of the sky) per second is given the name {\it{surface brightness}}.
Assuming spherical symmetry, 2-dimensional surface brightness can be
converted to 3-dimensional emission density. By dividing a cluster
observation into concentric annuli originating from the cluster center
and subtracting off cluster emission at larger radii from emission at
smaller radii, the amount of emission from a spherical shell can be
reconstructed from the emission in an annular ring. For the spherical
shell defined by radii $r_i$ and $r_{i+1}$, \citet{kriss83} shows the
relation between the emission density, $C_{i,i+1}$, to the observed
surface brightness, $S_{m,m+1}$, of the ring with radii $r_m$ and
$r_{m+1}$, is
\begin{equation}
\label{eqn:depro}
S_{m,m+1} = \frac{b}{A_{m,m+1}}\sum_{i-1}^m C_{i,i+1}~[(V_{i,m+1}-V_{i+1,m+1})-(V_{i,m}-V_{i+1,m})].
\end{equation}
where $b$ is the solid angle subtended on the sky by the object,
$A_{m,m+1}$ is the area of the ring, and the $V$ terms are the volumes
of various shells. This method of reconstructing the cluster emission
is called {\it{deprojection}}. While assuming spherical symmetry is
clearly imperfect, it is not baseless. The purpose of such an
assumption is to attain angular averages of the volume density at
various radii from an azimuthally averaged surface density. Systematic
uncertainties associated with deprojection are discussed in Section
\S\ref{sec:entsuppdene}.

In this dissertation the spectral model \mekal\ \citep{mekal1, mekal2,
  mekal3} is used for all of the spectral analysis. The \mekal\ model
normalization, $\eta$, is defined as
\begin{equation}
\label{eqn:norm}
\eta = \frac{10^{-14}}{4\pi D_A^2 (1+z)^2}~EI
\end{equation}
where $z$ is cluster redshift, $D_A$ is the angular diameter distance,
and $EI$ is the emission integral from eqn. \ref{eqn:ei}. Recognizing
that the count rate, $f(r)$, per volume is equivalent to the emission
density, $C_{i,i+1} = f(r)/\int dV$, where $dV$ can be a shell
(deprojected) or the sheath of a round column seen edge-on
(projected), combining eqns. \ref{eqn:depro} and \ref{eqn:norm} yields
an expression for the electron gas density which is a function of
direct observables,
\begin{equation}
\label{eqn:dens}
\nelec(r) = \sqrt{\frac{1.2 C(r) \eta(r) 4 \pi [D_A(1+z)]^2}{f(r) 10^{-14}}}
\end{equation}
where the factor of 1.2 comes from the ionization ratio \nelec=1.2\np,
$C(r)$ is the radial emission density derived from
eqn. \ref{eqn:depro}, $\eta$ is the spectral normalization from
eqn. \ref{eqn:norm}, $D_A$ is the angular diameter distance, $z$ is
the cluster redshift, and $f(r)$ is the spectroscopic count rate.

Simply by measuring surface brightness and analyzing spectra, the
cluster temperature, metallicity, and density can be inferred. These
quantities can then be used to derive pressure, $P = nkT$, where $n
\approx 2\nelec$. The total gas mass can be inferred using gas density
as $M_{gas} = \int (4/3) \pi r^3 \nelec dr$. By further assuming the
ICM is in hydrostatic equilibrium, the total cluster mass within
radius $r$ is
\begin{equation}
M(r) = \frac{kT(r)r}{\mu m_H
G}\left[\frac{d(log~\nelec(r))}{d(log~r)}+\frac{d(log~T(r))}{d(log~r)}\right]
\end{equation}
where all variables have their typical definitions. The rate at which
the ICM is cooling can also be expressed in simple terms of density
and temperature. Given a cooling function,
\definesymbol{\Lambda}{Cooling function: A function describing plasma
  emissivity for a given temperature and metal composition, and
  typically given in units of $\erg \cc \ps$.}, which is sensitive to
temperature and metal abundance (for the ICM $\Lambda(T,Z) \sim
10^{-23} \erg \cc \ps$), the cooling rate is given by $r_{cool} = n^2
\Lambda(T,Z)$. For some volume, $V$, the cooling time is then simply
the time required for a gas parcel to radiate away its thermal energy,
\begin{eqnarray}
t_{cool} V r_{cool} &=& \gamma NkT\\
t_{cool} &=& \frac{\gamma nkT}{n^2\Lambda(T,Z)}
\label{eqn:tcool}
\end{eqnarray}
where $\gamma$ is a constant specific to the type of cooling process
(\eg\ $3/2$ for isochoric and $5/2$ for isobaric). The cooling time of
the ICM can be anywhere between $10^{7-10}$ yrs. Cooling time is a
very important descriptor of the ICM because processes such as the
formation of stars and line-emitting nebulae are sensitive to cooling
time.

By ``simply'' pointing a high-resolution X-ray telescope, like
\chandra, at a cluster and exposing long enough to attain good
signal-to-noise, it is possible to derive a roster of fundamental
cluster properties: temperature, density, pressure, mass, cooling
time, and even entropy. Entropy is a very interesting quantity which
can be calculated using gas temperature and density and is most likely
fundamentally connected to processes like AGN feedback and star
formation (discussed in Chapters \ref{ch:ent_supp} and
\ref{ch:harad}). In the following section I discuss how gas entropy is
derived, why it is a useful quantity for understanding clusters, and
how it will be utilized later in this dissertation.

%%%%%%%%%%%%%%%%%%%%
\subsection{Entropy}
\label{sec:entintro}
%%%%%%%%%%%%%%%%%%%%

Entropy has both a macroscopic definition (the measure of available
energy) and microscopic definition (the measure of randomness), with
each being useful in many areas of science. Study of the ICM is a
macro-scale endeavor, so the definition of entropy pertinent to
discussion of the ICM is as a measure of the thermodynamic processes
involving heat transfer. But the conventional macroscopic definition
of entropy, $dS=dQ/T$, is not the quantity which is most useful in the
context of studying astrophysical objects. Thus we must resort to a
simpler, measurable surrogate for entropy, like the adiabat. The
adiabatic equation of state for an ideal monatomic gas is
$P=K\rho^{\gamma}$ where $K$ is the adiabatic constant and $\gamma$ is
the ratio of specific heat capacities and has the value of $5/3$ for a
monatomic gas. Setting $P=\rho kT/\mu m_H$ and solving for $K$ one
finds
\begin{equation}
\label{eqn:adi}
K = \frac{kT}{\mu m_H \rho^{2/3}}.
\end{equation}
where $\mu$ is the mean molecular weight of the gas and $m_H$ is the
mass of the Hydrogen atom. The true thermodynamic specific entropy
using this formulation is $s = k \ln K^{3/2}+s_0$, so neglecting
constants and scaling $K$ shall be called entropy in this
dissertation. A further simplification can be made to recast
eqn. \ref{eqn:adi} using the observables electron density,
\nelec, and X-ray temperature, $T_X$ (in keV):
\begin{equation}
K = \frac{T_X}{\nelec^{2/3}}.
\label{eqn:k}
\end{equation}
Equation \ref{eqn:k} is the definition of entropy used throughout this
dissertation. With a simple functional form, ``entropy'' can be
derived directly from X-ray observations. But why study the ICM in
terms of entropy?

ICM temperature and density alone primarily reflect the shape and
depth of the cluster dark matter potential \citep{voitbryan}. But it
is the specific entropy of a gas parcel, $s = c_v \ln
(T/\rho^{\gamma-1})$, which governs the density at a given
pressure. In addition, the ICM is convectively stable when $ds/dr \ge
0$, thus, without perturbation, the ICM will convect until the lowest
entropy gas is near the core and high entropy gas has buoyantly risen
to large radii. ICM entropy can also only be changed by addition or
subtraction of heat, thus the entropy of the ICM reflects most of the
cluster thermal history. Therefore, properties of the ICM can be
viewed as a manifestation of the dark matter potential and cluster
thermal history - which is encoded in the entropy structure. It is for
these reasons that the study of ICM entropy has been the focus of both
theoretical and observational study \citep{1996ApJ...473..692D,
1997MNRAS.288..355B, 1999Natur.397..135P, davies00, tozzi01,
voitbryan, ponman03, piffaretti05, pratt06, radioquiet,
d06, morandi07, 2008MNRAS.386.1309M}.

Hierarchical accretion of the ICM should produce an entropy
distribution which is a power-law across most radii with the only
departure occurring at radii smaller than 10\% of the virial radius
\citep{vkb05}. Hence deviations away from a power-law entropy
profile are indicative of prior heating and cooling and can be
exploited to reveal the nature of, for example, AGN feedback. The
implication of the intimate connection between entropy and
non-gravitational processes being that {\em{both}} the breaking of
self-similarity and the cooling flow problem (both discussed in
Section \S\ref{sec:incomplete}) can be studied with ICM entropy.

In Chapter \ref{ch:ent_supp} and Chapter \ref{ch:harad} I present the
results of an exhaustive study of galaxy cluster entropy profiles for
a sample of over 230 galaxy clusters taken from the \chandra\ Data
Archive. Analysis of these profiles has yielded important results
which can be used to constrain models of cluster feedback, understand
truncation of the high-mass end of the galaxy luminosity function, and
what effect these processes have on the global properties of
clusters. The size and scope of the entropy profile library presented
in this dissertation is unprecedented in the current scientific
literature, and we hope our library, while having provided immediate
results, will have a long-lasting and broad utility for the research
community. To this end, we have made all data and results available to
the public via a project web
site\footnote{http://www.pa.msu.edu/astro/MC2/accept/}.

%%%%%%%%%%%%%%%%%%%%%%%%%%%%%%%%%%%%%%%%%%%%
\section{The Incomplete Picture of Clusters}
\label{sec:incomplete}
%%%%%%%%%%%%%%%%%%%%%%%%%%%%%%%%%%%%%%%%%%%%

The literature on galaxy clusters is extensive. There has been a great
deal already written about clusters (with much more eloquence), and I
strongly suggest reading \citet{1984PhST....7..157M, kaiser86,
1990ApJ...363..349E, kaiser91, sarazinbook, fabiancfreview,
voitreview, cfreview, 2007PhR...443....1M, mcnamrev}, and references
therein for a comprehensive review of the concepts and topics to be
covered in this dissertation. The discussion of Sections
\S\ref{sec:ssbreak} and \S\ref{sec:cfprob} focuses on a few unresolved
mysteries involving galaxy clusters: the breaking of self-similarity
in relation to using clusters in cosmological studies and the cooling
flow problem as it relates to galaxy formation.

%%%%%%%%%%%%%%%%%%%%%%%%%%%%%%%%%%%%%%%%
\subsection{Breaking of Self-Similarity}
\label{sec:ssbreak}
%%%%%%%%%%%%%%%%%%%%%%%%%%%%%%%%%%%%%%%%

We now know the evolution of, and structure within, the Universe are a
direct result of the influence from dark energy and dark matter. An
all pervading repulsive dark energy has been posited to be responsible
for the accelerating expansion of the Universe
\citep{1998AJ....116.1009R, 1999ApJ...517..565P,
2007ApJ...659...98R}. Dark matter is an unknown form of matter which
interacts with itself and ordinary matter (both baryonic and
non-baryonic) through gravitational forces. Up until the last $\sim 5$
billion years \citep{1998AJ....116.1009R, 1999ApJ...517..565P,
2007ApJ...659...98R}, the influence of dark matter on the Universe has
been greater than that of dark energy. The early dominance of dark
matter is evident from the existence of large-scale structure like
galaxy clusters.

An end result of the gravitational attraction between amalgamations of
dark matter particles, called dark matter halos, is the merger of
small halos into ever larger halos. Since dark matter far outweighs
baryonic matter in the Universe, the baryons are coupled to the dark
matter halos via gravity, and hence are dragged along during the halo
merger process. Like raindrops falling in a pond that drains into a
river which flows into the ocean, the process of smaller units merging
to create larger units is found {\it{ad infinitum}} in the Universe
and is given the name hierarchical structure formation. A useful
visualization of the hierarchical structure formation process is shown
in Fig. \ref{fig:bigbang}. Hierarchical formation begins with small
objects like the first stars, continues on to galaxies, and culminates
in the largest present objects, clusters of galaxies.

\begin{figure}[htp]
  \begin{center}
    \includegraphics*[height=0.7\textheight, trim=0mm 0mm 0mm 0mm, clip]{bigbang2}
    \caption[Figures illustrating of large scale structure
      formation.]{{\it{Top panel:}} Illustration of hierarchical structure
      formation. {\it{Bottom panel:}} Snapshots from the simulation of a
      galaxy cluster forming. Each pane is 10 Mpc on a side. Color coding
      represents gas density along the line of sight (deep red is highest,
      dark blue is lowest). Each snapshot is numbered on the illustration
      at the approximate epoch each stage of cluster collapse
      occurs. Notice that, at first (1-2), very small objects like the
      first stars and protogalaxies collapse and then these smaller
      objects slowly merge into much larger halos (3-5). The hierarchical
      merging process ultimately results in a massive galaxy cluster (6)
      which continues to grow as sub-clusters near the box edge creep
      toward the cluster main body. Illustration taken from NASA/WMAP
      Science Team and modified by author. Simulation snapshots taken from
      images distributed to the public by the Virgo Consortium on behalf
      of Dr. Craig Booth: http://www.virgo.dur.ac.uk}
    \label{fig:bigbang}
  \end{center}
\end{figure}

In an oversimplified summary, one can say dark energy is attempting to
push space apart while dark matter is attempting to pull matter
together within that space. Were the balance and evolution of dark
energy and dark matter weighted heavily toward one or the other it
becomes clear that the amount of structure and its distribution will
be different. Thus, the nature of dark matter and dark energy
ultimately influence the number of clusters found at any given
redshift \citep[\eg][]{1993MNRAS.262.1023W} and hence cluster number
counts are immensely powerful in determining cosmological parameters
\citep[\eg][]{2001ApJ...561...13B}.

\invisiblesymbol{D_C}{Comoving distance: The distance which would be
  measured between two objects today if those two points were moving
  away from each other with the expansion of the Universe.}
\invisiblesymbol{D_A}{Angular diameter distance: The ratio of an
  object's true transverse size to its angular size. For a nearly flat
  universe, $D_A$ is a good approximation of the comoving distance,
  $D_A \approx D_C/(1+z)$.}
\invisiblesymbol{\Omega_S}{Solid angle: For a sphere of a given
  radius, for example the distance to an object $D_C$, the area of
  that object, $A$, on the sphere subtends an angle equal to $\Omega_S
  = A/D_C^2$. This is the solid angle.}
\invisiblesymbol{\Omega_k}{Curvature of the Universe: For a spatially
  flat universe, such as our own, $\Omega_k \approx 0$.}
\invisiblesymbol{V_C}{Comoving volume: The volume in which the number
  density of slowly evolving objects locked into the local Hubble
  flow, like galaxy clusters, is constant with redshift. The comoving
  volume element for redshift element $dz$ and solid angle element
  $d\Omega_S$ element is $dV_C=D_C[D_A(1+z)]^2 [\OM(1+z)^3 + \Omega_k
  (1+z)^2 + \OL]^{-1/2}~d\Omega_S~dz$.}
Individual clusters do not yield the information necessary to study
the underlying cosmogony. However, the number density of clusters
above a given mass threshold within a comoving volume element, \ie\
the cluster mass function, is a useful quantity
\citep{voitreview}. But the cluster mass function is a powerful
cosmological tool only if cluster masses can be accurately
measured. With no direct method of measuring cluster mass, easily
observable properties of clusters must be used as proxies to infer
mass.

Reliable mass proxies, such as cluster temperature and luminosity,
arise naturally from the theory that clusters are scaled versions of
each other. This property is commonly referred to as self-similarity
of mass-observables. More precisely, self-similarity presumes that
when cluster-scale gravitational potential wells are scaled by the
cluster-specific virial radius, the full cluster population has
potential wells which are simply scaled versions of one another
\citep{nfw1, nfw2}. Self-similarity is also expected to yield
low-scatter scaling relations between cluster properties such as
luminosity and temperature \citep{kaiser86, kaiser91,
  1991ApJ...383...95E, nfw1, nfw2, 1998ApJ...503..569E,
  1999ApJ...525..554F}. Consequently, mass-observable relations, such
as mass-temperature and mass-luminosity, derive from the fact that
most clusters are virialized, meaning the cluster's energy is shared
such that the virial theorem, $-2 \langle T \rangle = \langle V
\rangle$ where $\langle T \rangle$ is average kinetic energy and
$\langle V \rangle$ is average potential energy, is a valid
approximation. Both theoretical \citep{1996ApJ...469..494E,
  1998ApJ...495...80B, 1999ApJ...517..627M, 2001ApJ...555..597B,
  2002MNRAS.336..409B} and observational \citep{1984PhST....7..157M,
  edge91, white97, 1998MNRAS.297L..57A, 1998ApJ...503...77M,
  1999MNRAS.305..631A, 2001A&A...368..749F} studies have shown cluster
mass correlates well with X-ray temperature and luminosity, but that
there is much larger ($\approx 20-30\%$) scatter and different slopes
for these relations than expected. The breaking of self-similarity is
attributed to non-gravitational processes such as ongoing mergers
\citep[eg][]{2002ApJ...577..579R}, heating via feedback
\citep[eg][]{1999MNRAS.308..599C, bower01}, or radiative cooling in
the cluster core \citep[eg][]{2001Natur.414..425V, voitbryan}.

To reduce the scatter in mass scaling-relations and to increase their
utility for weighing clusters, how secondary processes alter
temperature and luminosity must first be quantified. It was predicted
that clusters with a high degree of spatial uniformity and symmetry
(\eg\ clusters with the least substructure in their dark matter and
gas distributions) would be the most relaxed and have the smallest
deviations from mean mass-observable relations. The utility of
substructure in quantifying relaxation is prevalent in many natural
systems, such as a placid lake or spherical gas cloud of uniform
density and temperature. Structural analysis of cluster simulations,
take for example the recent work of \citet{2008ApJ...681..167J} or
\citet{VV08}, have shown measures of substructure correlate well with
cluster dynamical state. But spatial analysis is at the mercy of
perspective. If equally robust aspect-independent measures of
dynamical state could be found, then quantifying deviation from mean
mass-scaling relations would be improved and the uncertainty of
inferred cluster masses could be further reduced. Scatter reduction
ultimately would lead to a more accurate cluster mass function, and by
extension, the constraints on theories explaining dark matter and dark
energy could grow tighter.

In Chapter \ref{ch:eband}, I present work investigating ICM
temperature inhomogeneity, a feature of the ICM which has been
proposed as a method for better understanding the dynamical state of a
cluster \citep{2001ApJ...546..100M}. Temperature inhomogeneity has the
advantage of being a spectroscopic quantity and therefore falls into
the class of aspect-independent metrics which may be useful for
reducing scatter in mass-observable relations. In a much larger
context, this dissertation may contribute to the improvement of our
understanding of the Universe's make-up and evolution.

%%%%%%%%%%%%%%%%%%%%%%%%%%%%%%%%%%%%%
\subsection{The Cooling Flow Problem}
\label{sec:cfprob}
%%%%%%%%%%%%%%%%%%%%%%%%%%%%%%%%%%%%%

For $50\%-66\%$ of galaxy clusters, the densest and coolest ($kT_X \la
\tvir/2$) ICM gas is found in the central $\sim 10\%$ of the cluster
gravitational potential well \citep{1984ApJ...285....1S,
  1992MNRAS.258..177E, white97, 1998MNRAS.298..416P,
  2005MNRAS.359.1481B}. For the temperature regime of the ICM,
radiative cooling time, $t_{cool}$ (see eqn. \ref{eqn:tcool}), is more
strongly dependent on density than temperature, $t_{cool} \propto
T_g^{1/2}\rho^2$, where $T_g$ is gas temperature and $\rho$ is gas
density. The energy lost via radiative cooling is seen as diffuse
thermal X-ray emission from the ICM \citep{gursky71, mitchell76,
  serle77}. When thermal energy is radiated away from the ICM, the gas
density must increase while gas temperature and internal pressure
respond by decreasing. The cluster core gas densities ultimately
reached through the cooling process are large enough such that the
cooling time required for the gas to radiate away its thermal energy
is much shorter than both the age of the Universe, \eg\ $t_{cool} \ll
\Hn^{-1}$, and the age of the cluster \citep{cowie77,
  fabian77}. Without compensatory heating, it thus follows that the
ICM in some cluster cores should cool and condense.

Gas within the cooling radius, \rcool, (defined as the radius at which
$t_{cool} = \Hn^{-1}$) is underpressured and cannot provide sufficient
pressure support to prevent overlying gas layers from forming a
subsonic flow of gas toward the bottom of the cluster gravitational
potential. However, if when the flowing gas enters the central galaxy
it has cooled to the point where the gas temperature equals the
central galaxy virial temperature, then adiabatic
compression\footnote{As the name indicates, no heat is exchanged
during adiabatic compression; but gas temperature rises because the
internal gas energy increases due to external work being done on the
system.} from the galaxy's gravitational potential well can balance
heat losses from radiative cooling. But, if the central galaxy's
gravitational potential is flat, then the gas energy gained via
gravitational effects can also be radiated away and catastrophic
cooling can proceed.

The sequence of events described above was given the name ``cooling
flow'' \citep{fabian77, cowie77, mathews78} and is the most simplistic
explanation of what happens to the ICM when it is continuously
cooling, spherically symmetric, and homogeneous \citep[see][for
reviews of cool gas in cluster cores]{fabiancfreview, cfreview,
2004cgpc.symp..143D}. The theoretical existence of cooling flows comes
directly from X-ray observations, yet the strongest observational
evidence for the existence of cooling flows will be seen when the gas
cools below X-ray emitting temperatures and forms stars, molecular
clouds, and emission line nebulae. Unfortunately, cooling flow models
were first presented at a time when no direct, complementary
observational evidence for cooling flows existed, highlighting the
difficulty of confronting the models. Undeterred, the X-ray
astrophysics community began referring to all clusters which had cores
meeting the criterion $t_{cool} < \Hn^{-1}$ as ``cooling flow
clusters,'' a tragic twist of nomenclature fate which has plagued many
scientific talks.

A mass deposition rate, $\dot{M}$, can be inferred for cooling flows
based on X-ray observations: $\dot{M} \propto
L_{X}(r<\rcool)(kT_X)^{-1}$, where $L_{X}(r<\rcool)$ is the X-ray
luminosity within the cooling region, $kT_X$ is the X-ray gas
temperature, and $\dot{M}$ typically has units of $\msol\pyr$. The
quantity $\dot{M}$ is useful in getting a handle on how much gas mass
is expected to be flowing into a cluster core. Mass deposition rates
have been estimated for many clusters and found to be in the range
$100-1000 \msol \pyr$ \citep{1984Natur.310..733F, white97,
  1998MNRAS.298..416P}. Mass deposition rates can also be estimated
using emission from individual spectral lines: $\dot{M} \propto
L_{X}(r<\rcool) \epsilon_f(T)$, where $L_{X}(r<\rcool)$ is the X-ray
luminosity within the cooling region and $\epsilon_f(T)$ is the
emissivity fraction attributable to a particular emission line. The
ICM soft X-ray emission lines of Fe XVII, O VIII, and Ne X at $E <
1.5$ keV for example, are especially useful in evaluating the
properties of cooling flows. Early low-resolution spectroscopy found
mass deposition rates consistent with those from X-ray observations
\citep[\eg][]{1982ApJ...262...33C}.

Not surprisingly, the largest, brightest, and most massive galaxy in a
cluster, the BCG, typically resides at the bottom of the cluster
potential, right at the center of where a cooling flow would
terminate. Real cooling flows were not expected to be symmetric,
continuous, or in thermodynamic equilibrium with the ambient
medium. Under these conditions, gas parcels at lower temperatures and
pressures experience thermal instability and are expected to rapidly
develop and collapse to form gaseous molecular clouds and stars. The
stellar and gaseous components of some BCGs clearly indicate some
amount of cooling and mass deposition has occurred. But are the
properties of the BCG population consistent with cooling flow model
predictions? For example, BCGs should be supremely luminous and
continually replenished with young, blue stellar populations since the
epoch of a BCG's formation. One should then expect the cores of
clusters suspected of hosting a cooling flow to have very bright, blue
BCGs bathed in clouds of emission line nebulae. However, observations
of cooling flow clusters reveal the true nature of the core to not
match these expectations of extremely high star formation rates, at
least not at the rate of $> 100 \msol
\pyr$.

The optical properties of massive galaxies and BCGs are well known and
neither population are as blue or bright as would be expected from the
extended periods of growth via cooling flows
\citep{1996MNRAS.283.1388M, 1996Natur.384..439S, 1996AJ....112..839C,
crawford99}. While attempts were made in the past to selectively
channel the unobserved cool gas into optically dark objects, such as
in low-mass, distributed star formation
\citep[\eg][]{1991ApJ...369L...1P}, methodical searches in the
optical, infrared, UV, radio, and soft X-ray wavelengths ($kT_X \la
2.0$ keV) have revealed that the total mass of cooler gas associated
with cooling flows is much less than expected \citep{hu85, heckman89,
  mcnamara90, odea94, 1994ApJ...436..669O, 1994AJ....107..448A,
  1994A&A...281..673M, voit95, 1997MNRAS.284L...1J,
  1998ApJ...494L.155F, 2000ApJ...545..670D, 2003ApJ...594L..13E}.

Confirming the suspicion that cooling flows are not cooling as
advertised, high-resolution \xmm\ RGS X-ray spectroscopy of clusters
expected to host very massive cooling flows definitively proved that
the ICM was not cooling to temperatures less than $1/3$ of the cluster
virial temperature \citep{peterson01, tamura01, peterson03}. A cooling
X-ray medium which has emission discontinuities at soft energies is
not predicted by the simplest single-phase cooling flow models, and a
troubling amount of fine-tuning (\eg\ minimum temperatures, hidden
soft emission) must be added to agree with observations. Modifications
such as preferential absorption of soft X-rays in the core region
\citep[\eg][]{1993MNRAS.262..901A} or turbulent mixing of a
multi-phase cooling flow \citep[\eg][]{2002MNRAS.332L..50F} have been
successful in matching observations, but these models lack the
universality needed to explain why {\it{all}} cooling flows are not as
massive as expected.

All of the observational evidence has resulted in a two-component
``cooling flow problem'': (1) spectroscopy of soft X-ray emission from
cooling flow clusters is inconsistent with theoretical predictions,
and (2) multiwavelength observations reveal a lack of cooled gas mass
or stars to account for the enormous theoretical mass deposition rates
implied by simple cooling flow models. So why and how is the cooling
of gas below $\tvir/3$ suppressed? As is the case with most questions,
the best answer thus far is simple: the cooling flow rates were wrong,
with many researchers suggesting the ICM has been intermittently
heated. But what feedback mechanisms are responsible for hindering
cooling in cluster cores?  How do these mechanisms operate?  What
observational constraints can we find to determine which combination
of feedback mechanism hypotheses are correct? The answers to these
questions have implications for both cluster evolution and massive
galaxy formation.

The cores of clusters are active places, so finding heating mechanisms
is not too difficult. The prime suspect, and best proposed solution to
the cooling flow problem thus far, invokes some combination of
supernovae and outbursts from the active galactic nucleus (AGN) in the
BCG \citep{1995MNRAS.276..663B, 1997MNRAS.288..355B,
2000ApJ...532...17L, 2001Natur.414..425V, 2002MNRAS.332..729C,
2002Natur.418..301B, 2002MNRAS.331..545B, 2002MNRAS.333..145N,
2002ApJ...581..223R, 2002MNRAS.335..610A, 2004MNRAS.348.1105O,
2004ApJ...613..811M, 2004ApJ...615..681R, 2004ApJ...617..896H,
2004MNRAS.355..995D, 2005ApJ...622..847S, pizzolato05, agnframework,
2006ApJ...643..120B, 2006ApJ...638..659M}. However, there are some big
problems: (1) AGN tend to deposit their energy along preferred axes,
while cooling in clusters proceeds in a nearly spherically symmetric
distribution in the core; (2) depositing AGN outburst energy at radii
nearest the AGN is difficult and how this mechanism works is not
understood; (3) there is a serious scale mismatch in heating and
cooling processes which has hampered the development of a
self-regulating feedback loop involving AGN. Radiative cooling
proceeds as the square of gas density, whereas heating is proportional
to volume. Hence, modeling feedback with a small source object, $r
\sim 1 \pc$, that is capable of compensating for radiative cooling
losses over an $\approx 10^6 \kpc^3$ volume, where the radial density
can change by four orders of magnitude, is quite
difficult. Dr. Donahue once framed this problem as, ``trying to heat
the whole of Europe with something the size of a button.''

The basic model of how AGN feedback works is that first gas accretes
onto a supermassive black hole at the center of the BCG, resulting in
the acceleration and ejection of very high energy particles back into
the cluster environment. The energy released in an AGN outburst is of
order $10^{58-61} \erg$. Under the right conditions, and via poorly
understood mechanisms, energy output by the AGN is transferred to the
ICM and thermalized, thereby heating the gas. The details of how this
process operates is beyond the scope of this dissertation
\citep[see][for a recent review]{mcnamrev}. However, in this
dissertation I do investigate some observable properties of clusters
which are directly impacted by feedback mechanisms.

Utilizing the quantity ICM entropy, I present results in Chapter
\ref{ch:ent_supp} which show that radial ICM entropy distributions for
a large sample of clusters have been altered in ways which are
consistent with AGN feedback models. Entropy and its connection to AGN
feedback is discussed in Subsection \ref{sec:entintro} of this
chapter. In Chapter \ref{ch:harad} I also present observational
results which support the hypothesis of \citet{conduction} that
electron thermal conduction may be an important mechanism in
distributing AGN feedback energy. Hence, this dissertation, in small
part, seeks to add to the understanding of how feedback functions in
clusters, and thus how to resolve the cooling flow problem -- the
resolution of which will lead to better models of galaxy formation and
cluster evolution.

%%%%%%%%%%%%%%%%%%%%%%%%%%%%%%%%%%%
\section{Chandra X-Ray Observatory}
\label{sec:chandra}
%%%%%%%%%%%%%%%%%%%%%%%%%%%%%%%%%%%

In this section I briefly describe what makes the \chandra\ X-ray
Observatory (\chandra\ or CXO for short) a ground-breaking and unique
telescope ideally suited for the work carried out in this
dissertation. In depth details of the telescope, instruments, and
spacecraft can be found at the CXO web
sites\footnote{http://chandra.harvard.edu/}$^{,}$\footnote{http://cxc.harvard.edu/}
or in \citet{chandra}. Much of what is discussed in the following
sections can also be found with more detail in ``The Chandra
Proposers' Observatory
Guide.''\footnote{http://cxc.harvard.edu/proposer/POG/} All figures
cited in this section are presented at the end of the corresponding
subsection.

%%%%%%%%%%%%%%%%%%%%%%%%%%%%%%%%%%%%%%
\subsection{Telescope and Instruments}
\label{sec:tele}
%%%%%%%%%%%%%%%%%%%%%%%%%%%%%%%%%%%%%%

The mean free path of an X-ray photon in a gas with the density of the
Earth's atmosphere is very short. Oxygen and nitrogen in the
atmosphere photoelectrically absorb X-ray photons resulting in 100\%
attenuation and make X-ray astronomy impossible from the Earth's
surface. Many long-standing theories in astrophysics predict a wide
variety of astronomical objects as X-ray emitters. Therefore,
astronomers and engineers have been sending X-ray telescopes into the
upper atmosphere and space for over 30 years now.

The most recent American X-ray mission to fly is the \chandra\ X-ray
Observatory. It is one of NASA's Great Observatories along with
\compton\ ($\gamma$-rays), \hubble\ (primarily optical), and
\spitzer\ (infrared). \chandra\ was built by Northrop-Grumman and is
operated by the National Aeronautics and Space Agency. \chandra\ was
launched in July 1999 and resides in a highly elliptical orbit with an
apogee of $\sim 140,000$ km and a perigee of $\sim 16,000$ km. One
orbit takes $\approx 64$ hours to complete. The telescope has four
nested iridium-coated paraboloid-hyperboloid mirrors with a focal
length of $\sim 10$ m. An illustration of the \chandra\ spacecraft is
shown in Figure \ref{fig:chandra}.

All data presented in this dissertation was collected with the
Advanced CCD Imaging Spectrometer (ACIS)
instrument\footnote{http://acis.mit.edu/acis}. ACIS is quite an
amazing and unique instrument in that it is an imager and
medium-resolution spectrometer at the same time. When an observation
is taken with ACIS, the data collected contains spatial and spectral
information since the location and energy of incoming photons are
recorded. The dual nature of ACIS allows the data to be analyzed by
spatially dividing up a cluster image and then extracting spectra for
these subregions of the image, a technique which is used heavily in
this dissertation.

The observing elements of ACIS are 10 $1024\times1024$ CCDs: six
linearly arranged CCDs (ACIS-S array) and four CCDs arranged in a
$2\times2$ mosaic (ACIS-I array). The ACIS focal plane is currently
kept at a temperature of $\sim -120\C$. During an observation, the
ACIS instrument is dithered along a Lissajous curve so parts of the
sky which fall in the chip gaps are also observed. Dithering also
ensures pixel variations of the CCD response are removed.

The high spatial and energy resolution of \chandra\ and its
instruments are ideal for studying clusters of galaxies. The telescope
on-board \chandra\ achieves on-axis spatial resolutions of $\la
0.5''$/pixel but it is the pixel size of the ACIS instrument ($\sim
0.492''$) which sets the resolution limit for observations. ACIS also
has an extraordinary energy resolution of $\Delta E/E \sim 100$. Below
energies of $\sim 0.3$ keV and above energies of $\sim 10$ keV the
ACIS effective area is ostensibly zero. The ACIS effective area also
peaks in the energy range $E \sim 0.7-2.0$ keV. As shown in
Figs. \ref{fig:brem} and \ref{fig:brem2}, a sizeable portion of galaxy
cluster emission occurs in the same energy range where the ACIS
effective area peaks. The energy resolution of ACIS also allows
individual emission line blends to be resolved in cluster
spectra. These aspects make \chandra\ a perfect choice for studying
clusters and the ICM in detail. Shown in Fig. \ref{fig:obs} are raw
observations of Abell 1795 with the aim-points on ACIS-I (top panel)
and ACIS-S (bottom panel). In Fig. \ref{fig:acisspec} is a spectrum
for the entire cluster extracted from the ACIS-I observation.

\begin{figure}[!hb]
  \begin{center}
    \includegraphics*[width=\textwidth, trim=0mm 0mm 0mm 0mm, clip]{chandra}
    \caption[\chandra\ X-ray Observatory spacecraft.]{An artist's
      rendition of the \chandra\ spacecraft. \chandra\ is the largest
      ($\sim 17$ m long; $\sim 4$ m wide) and most massive ($\sim 23$K
      kg) payload ever taken into space by NASA's Space Shuttle
      Program. The planned lifetime of the mission was 5 years, and
      the 10 year anniversary party is already planned. Illustration
      taken from Chandra X-ray Center.}
    \label{fig:chandra}
  \end{center}
\end{figure}

\begin{figure}[!hp]
  \begin{center}
    \includegraphics*[height=0.8\textheight, trim=0mm 0mm 0mm 0mm, clip]{obs}
    \caption[ACIS focal plane during observation.]{ In both panels
      celestial North is indicated by the blue arrow. {\it{Top
          panel:}} ACIS-I aimed observation of Abell 1795. The image
      has been binned by a factor of four so the whole field could be
      shown. {\it{Bottom panel:}} ACIS-S aimed observation of Abell
      1795. Again, the image is binned by a factor of four to show the
      whole field. For reference, the green boxes mark the ACIS-I
      chips which were off during this observation.}
    \label{fig:obs}
  \end{center}
\end{figure}

\begin{figure}[!hp]
  \begin{center}
    \includegraphics*[width=0.75\textwidth, trim=0mm 0mm 0mm 0mm,angle=270,clip]{acisspec}
    \caption[Spectrum of Abell 1795.]{Global spectrum of the cluster
      Abell 1795 with the best-fit single-component absorbed thermal
      spectral model overplotted (solid line). Comparing this spectrum
      with those of Figs. \ref{fig:brem} and \ref{fig:brem2}, the
      effects of finite energy resolution and convolving the spectral
      model with instrument responses are apparent. Individual
      spectral lines are now blends, and the spectral shape for $E <
      1.0$ keV has changed because of diminishing effective area.}
    \label{fig:acisspec}
  \end{center}
\end{figure}
\clearpage

%%%%%%%%%%%%%%%%%%%%%%%%%%%%%%%%%%%%%%%%%%%%%
\subsection{X-ray Background and Calibration}
\label{sec:cali}
%%%%%%%%%%%%%%%%%%%%%%%%%%%%%%%%%%%%%%%%%%%%%

\chandra\ is a magnificent piece of engineering, but it is not
perfect: observations are contaminated by background, the instruments
do not operate at full capacity, and the observatory has a finite
lifetime. In this section I briefly discuss these areas and how they
might affect past, current, and future scientific study with \chandra.

\subsubsection{Cosmic X-ray Background (CXB)}

\chandra\ is in a very high Earth orbit and is constantly bathed in
high-energy, charged particles originating from the cosmos which
interact with the CCDs (the eyes) and the materials housing the
instruments (the skull). The CXB is composed of a soft ($E < 2$ keV)
component attributable to extragalactic emission, local discrete
sources, and spatially varying diffuse Galactic emission. There are
also small contributions from the the ``local bubble''
\citep{2004ASSL..309..103S} and charge exchange within the solar
system \citep{2004ApJ...607..596W}. The possibility of emission from
unresolved point sources and other unknown CXB components also
exists. In most parts of the sky the soft CXB is not a large
contributor to the total background and can be modeled using a
combination of power-law and thermal spectral models and then
subtracted out of the data.

The CXB also has a hard ($E > 2$ keV) component which arises from
mostly extragalactic sources such as quasars and is well modeled as a
power-law. The spectral shape of the hard particle background has been
quite stable (up until mid-2005) and thus subtracting off the emission
by normalizing between observed and expected count rates in a
carefully chosen energy band makes removal of the hard component
straightforward.

Occasionally there are also very strong X-ray flares. These flares are
quite easy to detect in observations because, for a judiciously chosen
energy band/time bin combination, the count rate as a function of
observation time exhibits a dramatic spike during flaring. The time
intervals containing flare episodes can be excluded from the analysis
rendering them harmless. Harmless that is provided the flare was not
too long and some of the observing time allotment is usable.

\subsubsection{Instrumental Effects and Sources of Uncertainty}

There are a number of instrumental effects which must be considered
when analyzing data taken with \chandra. The geometric area of the
telescope's mirrors does not represent the ``usable'' area of the
mirrors. The true {\it{effective area}} of \chandra\ has been defined
by the Chandra X-ray Center (CXC) as the product of mirror geometric
area, reflectivity, off-axis vignetting, quantum efficiency of the
detectors, energy resolution of the detectors, and grating efficiency
(gratings were not in use during any of the observations used in this
dissertation). To varying degrees, all of these components depend on
energy and a few of them also have a spatial dependence. Discussion of
the effective area is a lengthy and involved topic. A more concise
understanding of the effective area can be attained from
visualization, hence the effective area as a function of energy is
shown in Figure \ref{fig:effarea}.

The ACIS instrument is also subject to dead/bad pixels, damage done by
interaction with very high-energy cosmic rays, imperfect read-out as a
function of CCD location, and a hydrocarbon contaminate which has been
building up since launch \citep{aciscontaminant}.

Observations are also at the mercy of uncertainty sources. The data
reduction software provided by the CXC (\ciao) and our own reduction
pipeline (CORP, discussed in Appendix \ref{ch:corp}) takes into
consideration:
\begin{enumerate}
\item Instrumental effects and calibration
\item $\approx 3\%$ error in absolute ACIS flux calibration
\item Statistical errors in the sky and background count rates
\item Errors due to uncertainty in the background normalization
\item Error due to the $\approx 2\%$ systematic uncertainty in the
background spectral shape
\item Cosmic variance of X-ray background sources
\item Unresolved source intensity
\item Scattering of source flux
\end{enumerate}

The list provided above is not comprehensive, but highlights the
largest sources of uncertainty: counting statistics, instrument
calibration, and background. In each section of this dissertation
where data analysis is discussed, the uncertainty and error analysis
is discussed in the context of the science objectives.

The \chandra\ mission was scheduled for a minimum five year mission
with the expectation that it would go longer. Nearing the ten year
anniversary of launch, it is therefore useful to wonder how
\chandra\ might be operating in years to come and what the future
holds for collecting data with \chandra\ five and even ten years from
now. The ``life expectancy'' of \chandra\ can be broken down into the
categories: spacecraft health, orbit stability, instrument
performance, and observation constraint evolution. Given the continued
progress of understanding \chandra's calibration, the relative
stability of the X-ray background, and the overall health of the
telescope as of last review, it has been suggested that \chandra\ will
survive at least a 15 year mission, \eg\ a decommissioning $\sim2014$
\citep{2007CUC, 2008ChNew..15...21B}.

\begin{figure}[!hp]
  \begin{center}
    \includegraphics*[width=\textwidth, trim=0mm 0mm 0mm 0mm,clip]{effarea}
    \caption[\chandra\ effective area as a function of energy.]{
      \chandra\ effective area as a function of energy. The effective
      area results from the product of mirror geometric area,
      reflectivity, off-axis vignetting, quantum efficiency of the
      detectors, energy resolution of the detectors, and grating
      efficiency. Note the ACIS peak sensitivity is in the energy
      range where the majority of the ICM emission occurs, $E =
      0.1-2.0$ keV. Figure taken from the CXC.}
    \label{fig:effarea}
  \end{center}
\end{figure}
