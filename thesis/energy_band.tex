%%%%%%%%%%%%%%%%%%%%%%
\section{Introduction}
\label{sec:ebandintro}
%%%%%%%%%%%%%%%%%%%%%%

The normalization, shape, and evolution of the cluster mass function
are useful for measuring cosmological parameters
\citep[\eg][]{1989ApJ...341L..71E, 1998ApJ...508..483W,
  2001ApJ...553..545H, 2004PhRvD..70l3008W}. In particular, the
evolution of large scale structure formation provides a complementary
and distinct constraint on cosmological parameters to those tests
which constrain them geometrically, such as supernovae
\citep{1998AJ....116.1009R, 2007ApJ...659...98R} and baryon acoustic
oscillations \citep{2005ApJ...633..560E}. However, clusters are a
useful cosmological tool only if we can infer cluster masses from
observable properties such as X-ray luminosity, X-ray temperature,
lensing shear, optical luminosity, or galaxy velocity
dispersion. Empirically, the correlation of mass to these observable
properties is well-established \citep[see][for a
  review]{voitreview}. But, there is non-negligible scatter in
mass-observable scaling relations which must be accounted for if
clusters are to serve as high-precision mass proxies necessary for
using clusters to study cosmological parameters such as the dark
energy equation of state. However, if we could identify a ``second
parameter" -- possibly reflecting the degree of relaxation in the
cluster -- we could improve the utility of clusters as cosmological
probes by parameterizing and reducing the scatter in mass-observable
scaling relations.

Toward this end, we desire to quantify the dynamical state of a
cluster beyond simply identifying which clusters appear relaxed and
those which do not. Most clusters are likely to have a dynamical state
which is somewhere in between \citep{2006ApJ...639...64O, kravtsov06,
  VV08}. The degree to which a cluster is virialized must first be
quantified within simulations that correctly predict the observable
properties of the cluster. Then, predictions for quantifying cluster
virialization may be tested, and possibly calibrated, with
observations of an unbiased sample of clusters \citep[\eg REXCESS
  sample of][]{rexcess}.

One study that examined how relaxation might affect the observable
properties of clusters was conducted by \citep[][hereafter
  ME01]{2001ApJ...546..100M} using the ensemble of simulations by
\citet{1997ApJ...491...38M}. ME01 found that most clusters which had
experienced a recent merger were cooler than the cluster
mass-observable scaling relations predicted. They attributed this to
the presence of cool, spectroscopically unresolved accreting
subclusters which introduce energy into the ICM and have a long
timescale for dissipation. The consequence was an under-prediction of
cluster binding masses of $15-30\%$ \citep{2001ApJ...546..100M}. It is
important to note that the simulations of \citet{1997ApJ...491...38M}
included only gravitational processes. The intervening years have
proven that radiative cooling is tremendously important in shaping the
global properties of clusters \citep[\eg][]{2004ApJ...613..811M,
  2006MNRAS.373..881P, nagai07}. Therefore, the magnitude of the
effect seen by ME01 could be somewhat different if radiative processes
are included.

One empirical observational method of quantifying the degree of
cluster relaxation involves using ICM substructure and employs the
power in ratios of X-ray surface brightness moments
\citep{1995ApJ...452..522B, 1996ApJ...458...27B,
  2005ApJ...624..606J}. Although an excellent tool, power ratios
suffer from being aspect-dependent \citep{2008ApJ...681..167J,
  VV08}. The work of ME01 suggested a complementary measure of
substructure which does not depend on projected perspective. In their
analysis, they found hard-band (2.0-9.0 keV) temperatures were $\sim
20\%$ hotter than broadband (0.5-9.0 keV) temperatures. Their
interpretation was that the cooler broadband temperature is the result
of unresolved accreting cool subclusters which are contributing
significant amounts of line emission to the soft band ($E < 2$
keV). This effect has been studied and confirmed by
\citet{2004MNRAS.354...10M} and \citet{2006ApJ...640..710V} using
simulated {\it Chandra} and {\it{XMM-Newton}} spectra.

ME01 suggested that this temperature skewing, and consequently the
fingerprint of mergers, could be detected utilizing the energy
resolution and soft-band sensitivity of {\it Chandra}. They proposed
selecting a large sample of clusters covering a broad dynamical range,
fitting a single-component temperature to the hard-band and broadband,
and then checking for a net skew above unity in the hard-band to
broadband temperature ratio. In this chapter we present the findings of
just such a temperature-ratio test using {\it Chandra} archival
data. We find the hard-band temperature exceeds the broadband
temperature, on average, by $\sim16\%$ in multiple flux-limited
samples of X-ray clusters from the {\it Chandra} archive. This mean
excess is weaker than the $20\%$ predicted by ME01, but is significant
at the $12\sigma$ level nonetheless. Hereafter, we refer to the
hard-band to broadband temperature ratio as $T_{HBR}$. We also find
that non-cool core systems and mergers tend to have higher values of
$T_{HBR}$. Our findings suggest that $T_{HBR}$ is an indicator of a
cluster's temporal proximity to the most recent merger event.

This chapter proceeds in the following manner: In \S\ref{sec:ebandselection}
we outline sample-selection criteria and {\it Chandra} observations
selected under these criteria. Data reduction and handling of the
X-ray background is discussed in \S\ref{sec:ebanddata}. Spectral extraction
is discussed in \S\ref{sec:ebandextraction}, while fitting and simulated
spectra are discussed in \S\ref{sec:ebandspecan}. Results and discussion of
our analysis are presented in \S\ref{sec:ebandr&d}. A summary of our work
is presented in \S\ref{sec:ebandsummary}. For this work we have assumed a
flat $\Lambda$CDM Universe with cosmology $\Omega_{M} = 0.3$,
$\Omega_{\Lambda} = 0.7$, and $H_{0} = 70$ km s$^{-1}$ Mpc$^{-1}$. All
quoted uncertainties are at the 1.6$\sigma$ level (90\% confidence).

%%%%%%%%%%%%%%%%%%%%%%%%%%
\section{Sample Selection}
\label{sec:ebandselection}
%%%%%%%%%%%%%%%%%%%%%%%%%%

Our sample was selected from observations publicly available in the
{\it Chandra} X-ray Telescope's Data Archive (CDA). Our initial
selection pass came from the {\it{ROSAT}} Brightest Cluster Sample
\citep{1998MNRAS.301..881E}, RBC Extended Sample
\citep{2000MNRAS.318..333E}, and {\it{ROSAT}} Brightest 55 Sample
\citep{1990MNRAS.245..559E, 1998MNRAS.298..416P}. The portion of our
sample at $z \gtrsim 0.4$ can also be found in a combination of the
{\it{Einstein}} Extended Medium Sensitivity Survey
\citep{1990ApJS...72..567G}, North Ecliptic Pole Survey
\citep{2006ApJS..162..304H}, {\it{ROSAT}} Deep Cluster Survey
\citep{1995ApJ...445L..11R}, {\it{ROSAT}} Serendipitous Survey
\citep{1998ApJ...502..558V}, and Massive Cluster Survey
\citep{2001ApJ...553..668E}. We later extended our sample to include
clusters found in the REFLEX Survey \citep{reflex}. Once we had a
master list of possible targets, we cross-referenced this list with
the CDA and gathered observations where a minimum of $R_{5000}$
(defined below) is fully within the CCD field of view.

$R_{\Delta_c}$ is defined as the radius at which the average cluster
density is $\Delta_c$ times the critical density of the Universe,
$\rho_c=3H(z)^2/8\pi G$. For our calculations of $R_{\Delta_c}$ we
adopt the relation from \citet{2002A&A...389....1A}:
\begin{eqnarray}
R_{\Delta_c} &=& 2.71 \mathrm{~Mpc~}
\beta_T^{1/2}
\Delta_{\mathrm{z}}^{-1/2}
(1+z)^{-3/2}
\left(\frac{kT_X}{10 \mathrm{~keV}}\right)^{1/2}\\
\Delta_z &=& \frac{\Delta_c \Omega_M}{18\pi^2\Omega_z} \nonumber \\
\Omega_z &=& \frac{\Omega_M (1+z)^3}{[\Omega_M
(1+z)^3]+[(1-\Omega_M-\Omega_{\Lambda})(1+z)^2]+\Omega_{\Lambda}} \nonumber
\end{eqnarray}
where $R_{\Delta_c}$ is in units of $h_{70}^{-1}$, $\Delta_c$ is the
assumed density contrast of the cluster at $R_{\Delta_c}$, and
$\beta_T$ is a numerically determined, cosmology-independent
($\lesssim \pm 20\%$) normalization for the virial relation $GM/2R =
\beta_T kT_{virial}$. We use $\beta_T = 1.05$ taken from
\citet{1996ApJ...469..494E}.

The result of our CDA search was a total of 374 observations of which
we used 244 for 202 clusters. The clusters making up our sample cover
a redshift range of $z = 0.045-1.24$, a temperature range of $T_X =
2.6-19.2 \mathrm{~keV}$, and bolometric luminosities of $L_{bol} =
0.12-100.4\times10^{44} \mathrm{~ergs~s}^{-1}$. The bolometric ($E =
0.1-100$ keV) luminosities for our sample clusters plotted as a
function of redshift are shown in Figure \ref{fig:lx_z}. These
$L_{bol}$ values are calculated from our best-fit spectral models and
are limited to the region of the spectral extraction (from $R=70$ kpc
to $R=R_{2500}$, or $R_{5000}$ in the cases in which no $R_{2500}$ fit
was possible). Basic properties of our sample are listed in Table
\ref{tab:sample}.

\begin{figure}
\begin{center}
\includegraphics*[width=\textwidth, trim=0mm 0mm 0mm 0mm, clip]{eband_f1.eps}
\caption[Redshift distribution of bolometric luminosities for
  $T_{HBR}$ sample]{Bolometric luminosity ($E = 0.1-100$ keV) plotted
  as a function of redshift for the 202 clusters which make up the
  initial sample. $L_{bol}$ values are limited to the region of
  spectral extraction, $R=R_{2500-\mathrm{CORE}}$. For clusters
  without $R_{2500-\mathrm{CORE}}$ fits, $R=R_{5000-\mathrm{CORE}}$
  fits were used and are denoted in the figure by empty stars. Dotted
  lines represent constant fluxes of $3.0\times10^{-15}$, $10^{-14}$,
  $10^{-13}$, and $10^{-12} \flux$.}
\label{fig:lx_z}
\end{center}
\end{figure}

For the sole purpose of defining extraction regions based on fixed
overdensities as discussed in \S\ref{sec:ebandextraction}, fiducial
temperatures (measured with {\it ASCA}) and redshifts were taken from
\citet{hornerthesis} (all redshifts confirmed with
NED\footnote{http://nedwww.ipac.caltech.edu/}). We show below that the
{\it ASCA} temperatures are sufficiently close to the {\it Chandra}
temperatures such that $R_{\Delta_c}$ is reliably estimated to within
20\%. Note that $R_{\Delta_c}$ is proportional to $T^{1/2}$, so that a
20\% error in the temperature leads to only a 10\% error in
$R_{\Delta_c}$, which in turn has no detectable effect on our final
results. For clusters not listed in \citet{hornerthesis} , we used a
literature search to find previously measured temperatures. If no
published value could be located, we measured the global temperature
by recursively extracting a spectrum in the region $0.1<r<0.2 R_{500}$
fitting a temperature and recalculating $R_{500}$. This process was
repeated until three consecutive iterations produced $R_{500}$ values
which differed by $\leq 1\sigma$. This method of temperature
determination has been employed in other studies, see
\citet{2006MNRAS.372.1496S} and \citet{2006ApJS..162..304H} as examples.

%%%%%%%%%%%%%%%%%%%%%%%%%%%%%
\section{{\it Chandra} Data}
\label{sec:ebanddata}
%%%%%%%%%%%%%%%%%%%%%%%%%%%%%

%%%%%%%%%%%%%%%%%%%%%%%%%%%%%%%%%%%%%%%
\subsection{Reprocessing and Reduction}
\label{sec:ebandreprocessing}
%%%%%%%%%%%%%%%%%%%%%%%%%%%%%%%%%%%%%%%

All data sets were reduced using the \chandra\} Interactive Analysis
of Observations package (\ciao) and accompanying Calibration Database
(\caldb). Using \ciao\ 3.3.0.1 and \caldb\ 3.2.2, standard data
analysis was followed for each observation to apply the most
up-to-date time-dependent gain correction and when appropriate, charge
transfer inefficiency correction \citep{2000ApJ...534L.139T}.

Point sources were identified in an exposure-corrected events file
using the adaptive wavelet tool {\textsc{wavdetect}}
\citep{2002ApJS..138..185F}. A $2 \sigma$ region surrounding each
point source was automatically output by {\textsc{wavdetect}} to
define an exclusion mask.  All point sources were then visually
confirmed and we added regions for point sources which were missed by
{\textsc{wavdetect}} and deleted regions for spuriously detected
``sources.'' Spurious sources are typically faint CCD features (chip
gaps and chip edges) not fully removed after dividing by the exposure
map. This process resulted in an events file (at ``level 2'') that has
been cleaned of point sources.

To check for contamination from background flares or periods of
excessively high background, light curve analysis was performed using
Maxim Markevitch's contributed \ciao\ script
{\textsc{lc\_clean.sl}}\footnote{http://cxc.harvard.edu/contrib/maxim/acisbg/}.
Periods with count rates $\geq 3\sigma$ and/or a factor $\geq 1.2$ of
the mean background level of the observation were removed from the
good time interval file. As prescribed by Markevitch's
cookbook\footnote{http://cxc.harvard.edu/contrib/maxim/acisbg/COOKBOOK},
ACIS front-illuminated (FI) chips were analyzed in the $0.3-12.0$ keV
range, and the $2.5-7.0$ keV energy range for the ACIS
back-illuminated (BI) chips.

When a FI and BI chip were both active during an observation, we
compared light curves from both chips to detect long duration,
soft-flares which can go undetected on the FI chips but show up on the
BI chips. While rare, this class of flare must be filtered out of the
data, as it introduces a spectral component which artificially
increases the best-fit temperature via a high energy tail. We find
evidence for a long duration soft flare in the observations of Abell
1758 \citep{2004ApJ...613..831D}, CL J2302.8+0844, and IRAS
09104+4109. These flares were handled by removing the time period of
the flare from the GTI file.

Defining the cluster ``center'' is essential for the later purpose of
excluding cool cores from our spectral analysis (see
\S\ref{sec:ebandextraction}). To determine the cluster center, we
calculated the centroid of the flare cleaned, point-source free
level-2 events file filtered to include only photons in the $0.7-7.0$
keV range. Before centroiding, the events file was exposure-corrected
and ``holes'' created by excluding point sources were filled using
interpolated values taken from a narrow annular region just outside
the hole (holes are not filled during spectral extraction discussed in
\S\ref{sec:ebandextraction}). Prior to centroiding, we defined the emission
peak by heavily binning the image, finding the peak value within a
circular region extending from the peak to the chip edge (defined by
the radius $R_{max}$), reducing $R_{max}$ by 5\%, reducing the binning
by a factor of 2, and finding the peak again. This process was
repeated until the image was unbinned (binning factor of 1). We then
returned to an unbinned image with an aperture centered on the
emission peak with a radius $R_{max}$ and found the centroid using
\ciao's {\textsc{dmstat}}. The centroid, ($x_c$, $y_c$), for a
distribution of $N$ good pixels with coordinates ($x_i$, $y_j$) and
values f($x_i$,$y_j$) is defined as:
\begin{eqnarray}
Q &=& \sum_{i,j=1}^N f(x_i,y_i) \\
x_c &=& \frac{\sum_{i,j=1}^N x_i \cdot f(x_i,y_i)}{Q} \nonumber \\
y_c &=& \frac{\sum_{i,j=1}^N y_i \cdot f(x_i,y_i)}{Q}. \nonumber
\end{eqnarray}

If the centroid was within 70 kpc of the emission peak, the emission
peak was selected as the center, otherwise the centroid was used as
the center. This selection was made to ensure all ``peaky'' cool cores
coincided with the cluster center, thus maximizing their exclusion
later in our analysis. All cluster centers were additionally verified
by eye.

%%%%%%%%%%%%%%%%%%%%%%%%%%%%%
\subsection{X-ray Background}
\label{sec:ebandbackground}
%%%%%%%%%%%%%%%%%%%%%%%%%%%%%

Because we measured a global cluster temperature, specifically looking
for a temperature ratio shift in energy bands which can be
contaminated by the high-energy particle background or the soft local
background, it was important to carefully analyze the background and
subtract it from our source spectra. Below we outline three steps
taken in handling the background: customization of blank-sky
backgrounds, re-normalization of these backgrounds for variation of
hard-particle count rates, and fitting of soft background residuals.

We used the blank-sky observations of the X-ray background from
\citet{2001ApJ...562L.153M} and supplied within the CXC \caldb. First,
we compared the flux from the diffuse soft X-ray background of the
{\it{ROSAT}} All-Sky Survey ({\it RASS}) combined bands $R12$, $R45$,
and $R67$ to the 0.7-2.0 keV flux in each extraction aperture for each
observation. {\it RASS} combined bands give fluxes for energy ranges
of 0.12-0.28, 0.47-1.21, and 0.76-2.04 keV respectively corresponding
to $R12$, $R45$, and $R67$. For the purpose of simplifying subsequent
analysis, we discarded observations with an $R45$ flux $\geq 10\%$ of
the total cluster X-ray flux.

The appropriate blank-sky dataset for each observation was selected
from the \caldb, reprocessed exactly as the observation was, and then
reprojected using the aspect solutions provided with each
observation. For observations on the ACIS-I array, we reprojected
blank-sky backgrounds for chips I0-I3 plus chips S2 and/or S3. For
ACIS-S observations, we created blank-sky backgrounds for the target
chip, plus chips I2 and/or I3. The additional off aim-point chips were
included only if they were active during the observation and had
available blank-sky data sets for the observation time period. Off
aim-point chips were cleaned for point sources and diffuse sources
using the method outlined in \S\ref{sec:ebandreprocessing}.

The additional off aim-point chips were included in data reduction
since they contain data which is farther from the cluster center and
are therefore more useful in analyzing the observation background. For
observations which did not have a matching off aim-point blank-sky
background, a source-free region of the active chips is
located and used for background normalization. To normalize the hard
particle component we measured fluxes for identical regions in the
blank-sky field and target field in the 9.5-12.0 keV range. The
effective area of the ACIS arrays above 9.5 keV is approximately zero,
and thus the collected photons there are exclusively from the particle
background.

A histogram of the ratios of the 9.5-12.0 keV count rate from an
observation's off aim-point chip to that of the observation specific
blank-sky background are presented in Figure \ref{fig:bgd}. The
majority of the observations are in agreement to $\lesssim 20\%$ of
the blank-sky background rate, which is small enough to not affect our
analysis. Even so, we re-normalized all blank-sky backgrounds to match
the observed background.

\begin{figure}
\begin{center}
\includegraphics*[width=\textwidth, trim=5mm 0mm 0mm 0mm,clip]{eband_f2.eps}
\caption[Histogram of hard-particle count rate ratios for $T_{HBR}$
  sample] {Ratio of target field and blank-sky field count rates in
  the 9.5-12.0 keV band for all 244 observations in our initial
  sample. Vertical dashed lines represent $\pm 20\%$ of unity. Despite
  the good agreement between the blank-sky background and observation
  count rates for most observations, all backgrounds are normalized.}
\label{fig:bgd}
\end{center}
\end{figure}

Normalization brings the observation background and blank-sky
background into agreement for $E > 2$ keV, but even after
normalization, typically, there may exist a soft excess/deficit
associated with the spatially varying soft Galactic
background. Following the technique detailed in
\citet{2005ApJ...628..655V}, we constructed and fit soft residuals for
this component. For each observation we subtracted a spectrum of the
blank-sky field from a spectrum of the off aim-point field to create a
soft residual. The residual was fit with a solar abundance,
zero-redshift \mekal\ model \citep{mekal1, mekal2, mekal3} in which
the normalization was allowed to be negative. The resulting best-fit
temperatures for all of the soft residuals identified here were
between 0.2-1.0 keV, which is in agreement with results of
\citet{2005ApJ...628..655V}. The model normalization of this background
component was then scaled to the cluster sky area. The re-scaled
component was included as a fixed background component during fitting
of a cluster's spectra.

%%%%%%%%%%%%%%%%%%%%%%%%%%%%%
\section{Spectral Extraction}
\label{sec:ebandextraction}
%%%%%%%%%%%%%%%%%%%%%%%%%%%%%

The simulated spectra calculated by ME01 were analyzed in a broad
energy band of $0.5-9.0$ keV and a hard energy band of
$2.0_{\mathrm{rest}}-9.0$ keV, but to make a reliable comparison with
{\it{Chandra}} data we used narrower energy ranges of 0.7-7.0 keV for
the broad energy band and $2.0_{\mathrm{rest}}-7.0$ keV for the hard
energy band. We excluded data below $0.7$ keV to avoid the effective
area and quantum efficiency variations of the ACIS detectors, and
excluded energies above $7.0$ keV in which diffuse source emission is
dominated by the background and where {\it{Chandra}}'s effective area
is small. We also accounted for cosmic redshift by shifting the lower
energy boundary of the hard-band from 2.0 keV to $2.0/(1+z)$ keV
(henceforth, the 2.0 keV cut is in the rest frame).

ME01 calculated the relation between $T_{0.5-9.0}$ and $T_{2.0-9.0}$
using apertures of $R_{200}$ and $R_{500}$ in size. While it is
trivial to calculate a temperature out to $R_{200}$ or $R_{500}$ for a
simulation, such a measurement at these scales is extremely difficult
with {\it Chandra} observations (see \citet{2005ApJ...628..655V} for a
detailed example). Thus, we chose to extract spectra from regions with
radius $R_{5000}$, and $R_{2500}$ when possible. Clusters analyzed
only within $R_{5000}$ are denoted in Table \ref{tab:sample} by a
double dagger ($\ddagger$).

The cores of some clusters are dominated by gas at $\lesssim
T_{virial}/2$ which can greatly affect the global best-fit
temperature; therefore, we excised the central 70 kpc of each
aperture. These excised apertures are denoted by ``-CORE'' in the
text. Recent work by \citet{2007ApJ...668..772M} has shown excising
0.15 $R_{500}$ rather than a static 70 kpc reduces scatter in
mass-observable scaling relations. However, our smaller excised region
seems sufficient for this investigation because for cool core clusters
the average radial temperature at $r > 70$ kpc is approximately
isothermal \citep{2005ApJ...628..655V}. Indeed, we find that cool core
clusters have smaller than average $T_{HBR}$ when the 70 kpc region
has been excised (\S\ref{sec:ebandccncc}).

Although some clusters are not circular in projection, but rather are
elliptical or asymmetric, we found that assuming spherical symmetry
and extracting spectra from a circular annulus did not significantly
change the best-fit values. For another such example see
\citet{2005MNRAS.359.1481B}.

After defining annular apertures, we extracted source spectra from the
target cluster and background spectra from the corresponding
normalized blank-sky dataset. By standard \ciao\ means we created
weighted effective area functions (WARFs) and redistribution matrices
(WRMFs) for each cluster using a flux-weighted map (WMAP) across the
entire extraction region. The WMAP was calculated over the energy
range 0.3-2.0 keV to weight calibrations that vary as a function of
position on the chip. The CCD characteristics which affect the
analysis of extended sources, such as energy dependent vignetting, are
contained within these files. Each spectrum was then binned to contain
a minimum of 25 counts per channel.

%%%%%%%%%%%%%%%%%%%%%%%%%%%
\section{Spectral Analysis}
\label{sec:ebandspecan}
%%%%%%%%%%%%%%%%%%%%%%%%%%%

%%%%%%%%%%%%%%%%%%%%
\subsection{Fitting}
\label{sec:ebandfitting}
%%%%%%%%%%%%%%%%%%%%

Spectra were fit with \xspec\ 11.3.2ag \citep{xspec} using a
single-temperature \mekal\ model in combination with the photoelectric
absorption model {\textsc{WABS}} \citep{wabs} to account for Galactic
absorption. Galactic absorption values, $N_{H}$, are taken from
\citet{dickeylockman}. The potentially free parameters of the absorbed
thermal model are $N_{H}$, X-ray temperature ($T_{X}$), metal
abundance normalized to solar \citep[elemental ratios taken
  from][]{ag89}, and a normalization proportional to the integrated
emission measure of the cluster. Results from the fitting are
presented in Tables \ref{tab:r2500specfits} and
\ref{tab:r5000specfits}. No systematic error is added during fitting,
and thus all quoted errors are statistical only. The statistic used
during fitting was $\chi^2$ (\xspec\ statistics package
\textsc{chi}). Every cluster analyzed was found to have greater than
1500 background-subtracted source counts in the spectrum.

For some clusters, more than one observation was available in the
archive. We utilized the power of the combined exposure time by first
extracting independent spectra, WARFs, WRMFs, normalized background
spectra, and soft residuals for each observation. Then, these
independent spectra were read into \xspec\ simultaneously and fit with
one spectral model which had all parameters, except normalization,
tied among the spectra. The simultaneous fit is what is reported for
these clusters, denoted by a star ($\star$), in Tables
\ref{tab:r2500specfits} and \ref{tab:r5000specfits}.

Additional statistical error was introduced into the fits because of
uncertainty associated with the soft local background component
discussed in \S\ref{sec:ebandbackground}. To estimate the sensitivity of
our best-fit temperatures to this uncertainty, we used the differences
between $T_{X}$ for a model using the best-fit soft background
normalization and $T_{X}$ for models using $\pm1\sigma$ of the soft
background normalization. The statistical uncertainty of the original
fit and the additional uncertainty inferred from the range of
normalizations to the soft X-ray background component were then added
in quadrature to produce a final error. In all cases this additional
background error on the temperature was less than 10\% of the total
statistical error, and therefore represents a minor inflation of the
error budget.

When comparing fits with fixed Galactic column density with those
where it was a free parameter, we found that neither the goodness of
fit per free parameter nor the best-fit $T_{X}$ were significantly
different. Thus, $N_{H}$ was fixed at the Galactic value with the
exception of three cases: Abell 399 \citep{2004MNRAS.351.1439S}, Abell
520, and Hercules A. For these three clusters $N_{H}$ is a free
parameter. In all fits, the metal abundance was a free parameter.

After fitting we rejected several data sets as their best-fit
$T_{2.0-7.0}$ had no upper bound in the 90\% confidence interval and
thus were insufficient for our analysis. All fits for the clusters
Abell 781, Abell 1682, CL J1213+0253, CL J1641+4001, IRAS 09104+4109,
Lynx E, MACS J1824.3+4309, MS 0302.7+1658, and RX J1053+5735 were
rejected. We also removed Abell 2550 from our sample after finding it
to be an anomalously cool ($T_{X} \sim$ 2 keV) ``cluster''. In fact,
Abell 2550 is a line-of-sight set of groups, as discussed by
\citet{2004cgpc.sympE..31M}. After these rejections, we are left with a
final sample of \ebandnuma\ clusters which have $R_{2500-\mathrm{CORE}}$ fits
and \ebandnumb\ clusters which have $R_{5000-\mathrm{CORE}}$ fits.

%%%%%%%%%%%%%%%%%%%%%%%%%%%%%%
\subsection{Simulated Spectra}
\label{sec:ebandsimulated}
%%%%%%%%%%%%%%%%%%%%%%%%%%%%%%

To quantify the effect a second, cooler gas component would have on
the fit of a single-component spectral model, we created an ensemble
of simulated spectra for each real spectrum in our entire sample using
{\textsc{XSPEC}}. With these simulated spectra we sought to answer the
question: Given the count level in each observation of our sample, how
bright must a second temperature component be for it to affect the
observed temperature ratio? Put another way, we asked at what flux
ratio a second gas phase produces a temperature ratio, $T_{HBR}$, of
greater than unity with 90\% confidence.

We began by adding the observation-specific background to a convolved,
absorbed thermal model with two temperature components observed for a
time period equal to the actual observation's exposure time and adding
Poisson noise. For each realization of an observation's simulated
spectrum, we defined the primary component to have the best-fit
temperature and metallicity of the $R_{2500-\mathrm{CORE}}$ 0.7-7.0
keV fit, or $R_{5000-\mathrm{CORE}}$ if no $R_{2500-\mathrm{CORE}}$
fit was performed. We then incremented the secondary component
temperature over the values 0.5, 0.75, 1.0, 2.0, and 3.0 keV. The
metallicity of the secondary component was fixed and set equal to the
metallicity of the primary component.

We adjusted the normalization of the simulated two-component spectra
to achieve equivalent count rates to those in the real spectra. The
sum of normalizations can be expressed as $N = N_1 + \xi N_2$. We set
the secondary component normalization to $N_2 = \xi N_{bf}$, where
$N_{bf}$ is the best-fit normalization of the appropriate 0.7-7.0 keV
fit and $\xi$ is a preset factor taking the values 0.4, 0.3, 0.2,
0.15, 0.1, and 0.05. The primary component normalization, $N_1$, was
determined through an iterative process to make real and simulated
spectral count rates match. The parameter $\xi$ therefore represents
the fractional contribution of the cooler component to the overall
count rate.

There are many systematics at work in the full ensemble of observation
specific simulated spectra, such as redshift, column density, and
metal abundance. Thus as a further check of spectral sensitivity to
the presence of a second gas phase, we simulated additional spectra
for the case of an idealized observation. We followed a similar
procedure to that outlined above, but in this instance we used a finer
temperature and $\xi$ grid of $T_2 = 0.5 \rightarrow 3.0$ in steps of
0.25 keV, and $\xi = 0.02 \rightarrow 0.4$ in steps of 0.02. The input
spectral model was $N_{H} = 3.0\times10^{20}$ cm$^{-2}$, $T_1 = 5$
keV, $Z/Z_{\odot} = 0.3$ and $z = 0.1$. We also varied the exposure
times such that the total number of counts in the 0.7-7.0 keV band was
15K, 30K, 60K, or 120K. For these spectra we used the on-axis sample
response files provided to Cycle 10
proposers\footnote{http://cxc.harvard.edu/caldb/prop\_plan/imaging/index.html}.
Poisson noise is added, but no background is considered.

We also simulated a control sample of single-temperature models. The
control sample is simply a simulated version of the best-fit
model. This control provides us with a statistical test of how often
the actual hard-component temperature might differ from a broadband
temperature fit if calibration effects are under control. Fits for the
control sample are shown in the far right panels of Figure
\ref{fig:ftx}.

For each observation, we have 65 total simulated spectra: 35
single-temperature control spectra and 30 two-component simulated
spectra (5 secondary temperatures, each with six different $\xi$). Our
resulting ensemble of simulated spectra contains 12,765 spectra. After
generating all the spectra we followed the same fitting routine
detailed in \S\ref{sec:ebandfitting}.

With the ensemble of simulated spectra we then asked the question: for
each $T_2$ and $\Delta T_X$ (defined as the difference between the
primary and secondary temperature components) what is the minimum
value of $\xi$, called $\xi_{min}$, that produces $T_{HBR} \geq 1.1$
at 90\% confidence? From our analysis of these simulated spectra we
have found these important results:
\begin{enumerate}

\item In the control sample, a single-temperature model rarely ($\sim
  2\%$ of the time) gives a significantly different $T_{0.7-7.0}$ and
  $T_{2.0-7.0}$. The weighted average (Fig. \ref{fig:ftx}, right
  panels) for the control sample is $1.002 \pm 0.001$ and the standard
  deviation is $\pm0.044$. The $T_{HBR}$ distribution for the control
  sample appears to have an intrinsic width which is likely associated
  with statistical noise of fitting in {\textsc{XSPEC}} (Dupke,
  private communication). This result indicates that our remaining set
  of observations is statistically sound, \eg\ our finding that
  $T_{HBR}$ significantly differs from 1.0 cannot result from
  statistical fluctuations alone.

\item Shown in Table \ref{tab:simres} are the contributions a second
  cooler component must make in the case of the idealized spectra in
  order to produce $T_{HBR} \geq 1.1$ at 90\% confidence. In general,
  the contribution of cooler gas must be $> 10\%$ for $T_2 < 2$ keV to
  produce $T_{HBR}$ as large as 1.1. The increase in percentages at
  $T_2 < 1.0$ keV is owing to the energy band we consider (0.7-7.0
  keV) as gas cooler than 0.7 keV must be brighter than at 1.0 keV in
  order to make an equivalent contribution to the soft end of the
  spectrum at 0.7 keV.

\item In the full ensemble of observation-specific simulated spectra,
  we find a great deal of statistical scatter in $\xi_{min}$ at any
  given $\Delta T_X$. This was expected as the full ensemble is a
  superposition of spectra with a broad range of total counts,
  $N_{H}$, redshifts, abundance, and backgrounds. But using the
  idealized simulated spectra as a guide, we find for those spectra
  with $N_{\mathrm{counts}} \gtrsim 15000$, producing $T_{HBR} \geq
  1.1$ at 90\% confidence again requires the cooler gas to be
  contributing $> 10\%$ of the emission. These results are also
  summarized in Table \ref{tab:simres}. The good agreement between the
  idealized and observation-specific simulated spectra indicates that
  while many more factors are in play for the observation-specific
  spectra, they do not degrade our ability to reliably measure
  $T_{HBR} > 1.1$.  The trend here of a common soft component
  sufficient to change the temperature measurement in a
  single-temperature model is statistical, a result that comes from an
  aggregate view of the sample rather than any individual fit.

\item As redshift increases, gas cooler than 1.0 keV is slowly
  redshifted out of the observable X-ray band. As expected, we find
  from our simulated spectra that for $z \geq 0.6$, $T_{HBR}$ is no
  longer statistically distinguishable from unity. In addition, the
  $T_{2.0-7.0}$ lower boundary nears convergence with the
  $T_{0.7-7.0}$ lower boundary as $z$ increases, and for $z = 0.6$,
  the hard-band lower limit is 1.25 keV, while at the highest redshift
  considered, $z = 1.2$, the hard-band lower limit is only 0.91
  keV. For the 14 clusters with $z \geq 0.6$ in our real sample we are
  most likely underestimating the actual amount of temperature
  inhomogeneity. We have tested the effect of excluding these clusters
  on our results, and find a negligible change in the overall skew of
  $T_{HBR}$ to greater than unity.
\end{enumerate}

\singlespacing
\begin{thesistable}{cc|cc}
\thesistablehead{Summary of two-component simulations}{Summary of
  two-component simulations}{\multicolumn{2}{c|}{Idealized Spectra} &
  \multicolumn{2}{c}{Observation-Specific Spectra}\\ $T_2$ &
  $\xi_{min}$ & $T_2$ & $\xi_{min}$\\ keV & & keV &}{tab:simres}
0.50 & $\geq 12\% \pm 4\%$ & 0.50 & $\geq 14.5\% \pm 0.1\%$\\
0.75 & $\geq 12\% \pm 4\%$ & 0.75 & $\geq 11.7\% \pm 0.1\%$\\
1.00 & $\geq 8\% \pm 3\%$  & 1.00 & $\geq 11.6\% \pm 0.1\%$\\
1.25 & $\geq 17\% \pm 3\%$ & - & -\\
1.50 & $\geq 23\% \pm 5\%$ & - & -\\
1.75 & $\geq 28\% \pm 4\%$ & - & -\\
2.00 & none                & 2.00 & $\geq 25.5\% \pm 0.1\%$\\
3.00 & none                & 3.00 & $\geq 28.9\% \pm 0.1\%$
\end{thesistable}
\doublespacing

Table \ref{tab:simres} summarizes the results of the two temperature
component spectra simulations for the ideal and observation-specific
cases (see \S\ref{sec:ebandsimulated} for details). The parameter
$\xi_{min}$ represents the minimum fractional contribution of the
cooler component, $T_2$, to the overall count rate in order to produce
$T_{HBR} \geq 1.1$ at 90\% confidence. The results for the
observation-specific spectra are for spectra with $N_{\mathrm{counts}}
> 15,000$.

%%%%%%%%%%%%%%%%%%%%%%%%%%%%%%%%
\section{Results and Discussion}
\label{sec:ebandr&d}
%%%%%%%%%%%%%%%%%%%%%%%%%%%%%%%%

%%%%%%%%%%%%%%%%%%%%%%%%%%%%%%%
\subsection{Temperature Ratios}
\label{sec:ebandtfresults}
%%%%%%%%%%%%%%%%%%%%%%%%%%%%%%%

For each cluster we have measured a ratio of the hard-band to
broadband temperature defined as $T_{HBR}$ =
$T_{2.0-7.0}$/$T_{0.7-7.0}$. We find that the mean $T_{HBR}$ for our
entire sample is greater than unity at more than $12\sigma$
significance. The weighted mean values for our sample are shown in
Table \ref{tab:wavg}. Quoted errors in Table \ref{tab:wavg} are
standard deviation of the mean calculated using an unbiased estimator
for weighted samples. Simulated sample has been culled to include only
$T_2$=0.75 keV. Presented in Figure \ref{fig:ftx} are the binned
weighted means and raw $T_{HBR}$ values for $R_{2500-\mathrm{CORE}}$,
$R_{5000-\mathrm{CORE}}$, and the simulated control sample. The
peculiar points with $T_{HBR} <$ 1 are all statistically consistent
with unity. The presence of clusters with $T_{HBR}$ = 1 suggests that
systematic calibration uncertainties are not the sole reason for
deviations of $T_{HBR}$ from 1. We also find that the temperature
ratio does not depend on the best-fit broadband temperature, and that
the observed dispersion of $T_{HBR}$ is greater than the predicted
dispersion arising from systematic uncertainties.

\begin{figure}
\begin{center}
\includegraphics*[width=\textwidth, trim=0mm 0mm 0mm 0mm, clip]{eband_f3.eps}
\caption[$T_{HBR}$ vs. broadband temperature for $T_{HBR}$ sample]{
  Best-fit temperatures for the hard-band, $T_{2.0-7.0}$, divided by
  the broadband, $T_{0.7-7.0}$, and plotted against the broadband
  temperature. For binned data, each bin contains 25 clusters, with
  the exception of the highest temperature bins which contain 16 and
  17 for $R_{2500-\mathrm{CORE}}$ and $R_{5000-\mathrm{CORE}}$,
  respectively. The simulated data bins contain 1000 clusters with the
  last bin having 780 clusters. The line of equality is shown as a
  dashed line and the weighted mean for the full sample is shown as a
  dashed-dotted line. Error bars are omitted in the unbinned data for
  clarity. Note the net skewing of $T_{HBR}$ to greater than unity for
  both apertures with no such trend existing in the simulated
  data. The dispersion of $T_{HBR}$ for the real data is also much
  larger than the dispersion of the simulated data.  }
\label{fig:ftx}
\end{center}
\end{figure}

The uncertainty associated with each value of $T_{HBR}$ is dominated
by the larger error in $T_{2.0-7.0}$, and on average, $\Delta
T_{2.0-7.0} \approx 2.3\Delta T_{0.7-7.0}$. This error interval
discrepancy naturally results from excluding the bulk of a cluster's
emission which occurs below 2 keV. While choosing a
temperature-sensitive cut-off energy for the hard-band (other than 2.0
keV) might maintain a more consistent error budget across our sample,
we do not find any systematic trend in $T_{HBR}$ or the associated
errors with cluster temperature.

\singlespacing
\begin{thesistable}{ccccccc}
\thesistablehead{Weighted averages for various apertures}{Weighted
averages for various apertures}{\multicolumn{1}{c}{} & \multicolumn{3}{l}{\dotfill Without
Core\dotfill} & \multicolumn{3}{l}{\dotfill With Core\dotfill}\\  & [0.7-7.0] & [2.0-7.0] & $T_{HBR}$ &
[0.7-7.0] & [2.0-7.0] & $T_{HBR}$\\ Aperture & keV & keV &  & keV & keV &}{tab:wavg}
R$_{2500}$ & 4.93$\pm 0.03$   & 6.24$\pm 0.07$   & 1.16$\pm 0.01$   & 4.47$\pm 0.02$ & 5.45$\pm 0.05$ & 1.13$\pm 0.01$\\
R$_{5000}$ & 4.75$\pm 0.02$   & 5.97$\pm 0.07$   & 1.14$\pm 0.01$   & 4.27$\pm 0.02$ & 5.29$\pm 0.05$ & 1.14$\pm 0.01$\\
Simulated  & 3.853$\pm 0.004$ & 4.457$\pm 0.009$ & 1.131$\pm 0.002$ & -       & -       & -\\
Control    & 4.208$\pm 0.003$ & 4.468$\pm 0.006$ & 1.002$\pm 0.001$ & -       & -       & -
\end{thesistable}
\doublespacing

%%%%%%%%%%%%%%%%%%%%%%%%
\subsection{Systematics}
\label{sec:ebandsys}
%%%%%%%%%%%%%%%%%%%%%%%%

In this study we have found the average value of $T_{HBR}$ is
significantly greater than one and that $\sigma_{HBR} >
\sigma_{\mathrm{control}}$, with the latter result being robust
against systematic uncertainties. As predicted by ME01, both of these
results are expected to arise naturally from the hierarchical
formation of clusters. But systematic uncertainty related to {\it
  Chandra} instrumentation or other sources could shift the average
value of $T_{HBR}$ one would get from ``perfect'' data. In this
section we consider some additional sources of uncertainty.
5A
First, the disagreement between {\it XMM-Newton} and {\it Chandra}
cluster temperatures has been noted in several independent studies,
i.e. \citet{2005ApJ...628..655V} and \citet{chanxmmdis}. But the source
of this discrepancy is not well understood and efforts to perform
cross-calibration between {\it XMM-Newton} and {\it Chandra} have thus
far not been conclusive. One possible explanation is poor calibration
of {\it Chandra} at soft X-ray energies which may arise from a
hydrocarbon contaminant on the High Resolution Mirror Assembly (HRMA)
similar in nature to the contaminant on the ACIS detectors
\citep{aciscontaminant}. We have assessed this possibility by looking
for systematic trends in $T_{HBR}$ with time or temperature, as such a
contaminant would most likely have a temperature and/or time
dependence.

As noted in \S\ref{sec:ebandtfresults} and seen in Figure \ref{fig:ftx}, we
find no systematic trend with temperature either for the full sample
or for a sub-sample of single-observation clusters with $> 75\%$ of
the observed flux attributable to the source (higher S/N observations
will be more affected by calibration uncertainty). Plotted in the
lower-left pane of Figures \ref{fig:sysr25} and \ref{fig:sysr50} is
$T_{HBR}$ versus time for single observation clusters (clusters with
multiple observations are fit simultaneously and any time effect would
be washed out) where the spectral flux is $> 75\%$ from the source. We
find no significant systematic trend in $T_{HBR}$ with time, which
suggests that if $T_{HBR}$ is affected by any contamination of {\it
  Chandra}'s HRMA, then the contaminant is most likely not changing
with time. Our conclusion on this matter is that the soft calibration
uncertainty is not playing a dominant role in our results.

Aside from instrumental and calibration effects, some other possible
sources of systematic error are S/N, redshift selection, Galactic
absorption, and metallicity. Also presented in Figures
\ref{fig:sysr25} and \ref{fig:sysr50} are three of these parameters
versus $T_{HBR}$ for $R_{2500-\mathrm{CORE}}$ and
$R_{5000-\mathrm{CORE}}$, respectively. The trend in $T_{HBR}$ with
redshift is expected as the 2.0/(1+$z$) keV hard-band lower boundary
nears convergence with the 0.7 keV broadband lower boundary at $z
\approx 1.85$. We find no systematic trends of $T_{HBR}$ with S/N or
Galactic absorption, which might occur if the skew in $T_{HBR}$ were a
consequence of poor count statistics, inaccurate Galactic absorption,
or very poor calibration. In addition, the ratio of $T_{HBR}$ for
$R_{2500-\mathrm{CORE}}$ to $R_{5000-\mathrm{CORE}}$ for every cluster
in our sample does not significantly deviate from unity. Our results
are robust to changes in aperture size.

\begin{figure}
\begin{center}
\includegraphics*[width=\textwidth, trim=0mm 0mm 0mm 0mm, clip]{eband_f4.eps}
\caption[Plot of several possible systematics for
  $R_{2500-\mathrm{CORE}}$ apertures.]{A few possible sources of
  systematic uncertainty vs. $T_{HBR}$ calculated for the
  $R_{2500-\mathrm{CORE}}$ apertures (\ebandnuma\ clusters). Error
  bars have been omitted in several plots for clarity. The line of
  equality is shown as a dashed line in all panels. {\it{(Top left:)}}
  $T_{HBR}$ vs. redshift for the entire sample. The trend in $T_{HBR}$
  with redshift is expected as the $T_{2.0-7.0}$ lower boundary nears
  convergence with the $T_{0.7-7.0}$ lower boundary at $z \approx
  1.85$. Weighted values of $T_{HBR}$ are consistent with unity
  starting at $z \sim 0.6$.  {\it{(Top right:)}} $T_{HBR}$
  vs. percentage of spectrum flux which is attributed to the
  source. We find no trend with signal-to-noise which suggests
  calibration uncertainty not is playing a major role in our results.
  {\it{(Middle left:)}} $T_{HBR}$ vs. Galactic column density. We find
  no trend in absorption which would result if $N_{H}$ values are
  inaccurate or if we had improperly accounted for local soft
  contamination.  {\it{(Middle right:)}} $T_{HBR}$ vs. the deviation
  from unity in units of measurement uncertainty. Recall that we have
  used 90\% confidence ($1.6\sigma$) for our analysis.  {\it{(Bottom
      left:)}} $T_{HBR}$ plotted vs. observation start date. The
  plotted points are culled from the full sample and represent only
  clusters which have a single observation and where the spectral flux
  is $> 75\%$ from the source. We note no systematic trend with time.
  {\it{(Bottom right:)}} Ratio of {\it Chandra} temperatures derived
  in this work to {\it ASCA} temperatures taken from
  \citet{hornerthesis}. We note a trend of comparatively hotter {\it
    Chandra} temperatures for clusters $> 10$ keV, otherwise our
  derived temperatures are in good agreement with those of {\it ASCA}.
}
\label{fig:sysr25}
\end{center}
\end{figure}

\begin{figure}
\begin{center}
\includegraphics*[width=\textwidth, trim=0mm 0mm 0mm 0mm, clip]{eband_f5.eps}
\caption[Plot of several possible systematics for
  $R_{5000-\mathrm{CORE}}$ apertures.]{ Same as Fig. \ref{fig:sysr25}
  except using the $R_{5000-\mathrm{CORE}}$ apertures (\ebandnumb\ clusters).}
\label{fig:sysr50}
\end{center}
\end{figure}

Also shown in Figures \ref{fig:sysr25} and \ref{fig:sysr50} are the
ratios of {\it ASCA} temperatures taken from \citet{hornerthesis} to
{\it Chandra} temperatures derived in this work. The spurious point
below 0.5 with very large error bars is MS 2053.7-0449, which has a
poorly constrained {\it ASCA} temperature of
$10.03^{+8.73}_{-3.52}$. Our value of $\sim 3.5$ keV for this cluster
is in agreement with the recent work of
\citet{2008ApJS..174..117M}. Not all our sample clusters have an {\it
  ASCA} temperature, but a sufficient number (53) are available to
make this comparison reliable. Apertures used in the extraction of
{\it ASCA} spectra had no core region removed and were substantially
larger than $R_{2500}$. {\it ASCA} spectra were also fit over a
broader energy range (0.6-10 keV) than we use here. Nonetheless, our
temperatures are in good agreement with those from {\it ASCA}, but we
do note a trend of comparatively hotter {\it Chandra} temperatures for
$T_{Chandra} > 10$ keV. For both apertures, the clusters with
$T_{Chandra} > 10$ keV are Abell 1758, Abell 2163, Abell 2255, and RX
J1347.5-1145. Based on this trend, we test excluding the hottest
clusters ($T_{Chandra} > 10$ keV where {\it ASCA} and {\it Chandra}
disagree) from our sample. The mean temperature ratio for
$R_{2500-\mathrm{CORE}}$ remains $1.16$ and the error of the mean
increases from $\pm 0.014$ to $\pm 0.015$, while for
$R_{5000-\mathrm{CORE}}$ $T_{HBR}$ increases by a negligible $0.9\%$
to $1.15\pm 0.014$. Our results are not being influenced by the
inclusion of hot clusters.

\begin{figure}
\begin{center}
\includegraphics*[width=\textwidth, trim=0mm 0mm 0mm 0mm, clip]{eband_f6.eps}
\caption[$T_{HBR}$ vs. best-fit metallicity]{$T_{HBR}$ as a function
  of metal abundance for $R_{2500-\mathrm{CORE}}$,
  $R_{5000-\mathrm{CORE}}$, and the control sample (see discussion of
  control sample in \S\ref{sec:ebandsimulated}). Error bars are omitted for
  clarity. The dashed-line represents the linear best-fit using the
  bivariate correlated error and intrinsic scatter (BCES) method of
  \citet{1996ApJ...470..706A} which takes into consideration errors on
  both $T_{HBR}$ and abundance when performing the fit. We note no
  trend in $T_{HBR}$ with metallicity (the apparent trend in the top
  panel is not significant) and also note the low dispersion in the
  control sample relative to the observations. The striation of
  abundance arises from our use of two decimal places in recording the
  best-fit values from {\textsc{XSPEC}}.  }
\label{fig:metal}
\end{center}
\end{figure}

The temperature range of the clusters we have analyzed ($T_X \sim
3-20$ keV) is broad enough that the effect of metal abundance on the
inferred spectral temperature is clearly not negligible. In Figure
\ref{fig:metal} we have plotted $T_{HBR}$ versus abundance in solar
units. Despite covering a factor of seven in temperature and metal
abundances ranging from $Z/Z_{\odot} \approx 0$ to solar, we find no
trend in $T_{HBR}$ with metallicity. The slight trend in the
$R_{2500-\mathrm{CORE}}$ aperture (Fig.  \ref{fig:metal}, top) is
insignificant, while there is no trend at all in the control sample or
$R_{5000-\mathrm{CORE}}$ aperture.

%%%%%%%%%%%%%%%%%%%%%%%%%%%%%%%%%%%%%%%%%%%%%%%%%%%%
\subsection{Using $T_{HBR}$ as a Test of Relaxation}
\label{sec:ebandrelax}
%%%%%%%%%%%%%%%%%%%%%%%%%%%%%%%%%%%%%%%%%%%%%%%%%%%%

%%%%%%%%%%%%%%%%%%%%%%%%%%%%%%%%%%%%%%%%%%%%%%
\subsubsection{Cool Core Versus Non-Cool Core}
\label{sec:ebandccncc}
%%%%%%%%%%%%%%%%%%%%%%%%%%%%%%%%%%%%%%%%%%%%%%

As discussed in \ref{sec:ebandintro}, ME01 gives us reason to believe the
observed skewing of $T_{HBR}$ to greater than unity is related to the
dynamical state of a cluster. It has also been suggested that the
process of cluster formation and relaxation may robustly result in the
formation of a cool core \citep{2006ApJ...640..673O,
  2008ApJ...675.1125B}. Depending on classification criteria,
completeness, and possible selection biases, studies of flux-limited
surveys have placed the prevalence of cool cores at $34\%-60\%$
\citep{white97, 1998MNRAS.298..416P, 2005MNRAS.359.1481B,
  2007A&A...466..805C}. It has thus become rather common to divide up
the cluster population into two distinct classes, cool core (CC) and
non-cool core (NCC), for the purpose of discussing their different
formation or merger histories. We thus sought to identify which
clusters in our sample have cool cores, which do not, and if the
presence or absence of a cool core is correlated with $T_{HBR}$. It is
very important to recall that we excluded the core during spectral
extraction and analysis.

To classify the core of each cluster, we extracted a spectrum for the
50 kpc region surrounding the cluster center and then defined a
temperature decrement,
\begin{equation}
T_{\mathrm{dec}} = T_{50}/T_{\mathrm{cluster}}
\label{eqn:tdec}
\end{equation}
where $T_{50}$ is the temperature of the inner 50 kpc and
$T_{\mathrm{cluster}}$ is either the $R_{2500-\mathrm{CORE}}$ or
$R_{5000-\mathrm{CORE}}$ temperature. If $T_{\mathrm{dec}}$ was
2$\sigma$ less than unity, we defined the cluster as having a CC,
otherwise the cluster was defined as NCC. We find CCs in 35\% of our
sample and when we lessen the significance needed for CC
classification from 2$\sigma$ to 1$\sigma$, we find 46\% of our sample
clusters have CCs. It is important to note that the frequency of CCs
in our study is consistent with other more detailed studies of CC/NCC
populations.

When fitting for $T_{50}$, we altered the method outlined in
\S\ref{sec:ebandfitting} to use the {\textsc{XSPEC}} modified Cash
statistic \citep{1979ApJ...228..939C}, {\textsc{cstat}}, on ungrouped
spectra. This choice was made because the distribution of counts per
bin in low count spectra is not Gaussian but instead Poisson. As a
result, the best-fit temperature using $\chi^2$ is typically cooler
\citep{1989ApJ...342.1207N, 2007A&A...462..429B}. We have explored
this systematic in {\bfseries\em{all}} of our fits and found it to be
significant only in the lowest count spectra of the inner 50 kpc
apertures discussed here. But, for consistency, we fit all inner 50
kpc spectra using the modified Cash statistic.

With each cluster core classified, we then took cuts in $T_{HBR}$ 
and asked how many CC and NCC clusters were above these cuts. 
Figure \ref{fig:cc_ncc_bin} shows the normalized number of CC and NCC
clusters as a function of cuts in $T_{HBR}$. If $T_{HBR}$ were
insensitive to the state of the cluster core, we expect, for normally
distributed $T_{HBR}$ values, to see the number of CC and NCC clusters
decreasing in the same way. However, the number of CC clusters falls
off more rapidly than the number of NCC clusters. If the presence of a
CC is indicative of a cluster's advancement towards complete
virialization, then the significantly steeper decline in the percent
of CC clusters versus NCC as a function of increasing $T_{HBR}$
indicates higher values of $T_{HBR}$ are associated with a less
relaxed state. This result is insensitive to our choice of
significance level in the core classification, i.e. the result is the
same whether using $1\sigma$ or $2\sigma$ significance when
considering $T_{\mathrm{dec}}$.

\begin{figure}
\begin{center}
\includegraphics*[width=\textwidth, trim=15mm 10mm 0mm 0mm, clip]{eband_f7.eps}
\caption[Number of cool and non-cool clusters as a function of
  $T_{HBR}$]{Normalized number of CC and NCC clusters as a function of
  cuts in $T_{HBR}$. There are \ebandnuma\ clusters plotted in the top panel
  and \ebandnumb\ in the bottom panel. We have defined a cluster as having a
  CC when the temperature for the 50 kpc region around the cluster
  center divided by the temperature for $R_{2500-\mathrm{CORE}}$, or
  $R_{5000-\mathrm{CORE}}$, was less than one at the $2\sigma$
  level. We then take cuts in $T_{HBR}$ at the $1\sigma$ level and ask
  how many CC and NCC clusters are above these cuts. The number of CC
  clusters falls off more rapidly than NCC clusters in this
  classification scheme suggesting higher values of $T_{HBR}$ prefer
  less relaxed systems which do not have cool cores. This result is
  insensitive to our choice of significance level in both the core
  classification and $T_{HBR}$ cuts.  }
\label{fig:cc_ncc_bin}
\end{center}
\end{figure}

Because of the CC/NCC definition we selected, our identification of
CCs and NCCs was only as robust as the errors on $T_{50}$ allowed. One
can thus ask the question, did our definition bias us towards finding
more NCCs than CCs? To explore this question we simulated 20 spectra
for each observation following the method outlined in
\S\ref{sec:ebandsimulated} for the control sample but using the inner 50
kpc spectral best-fit values as input. For each simulated spectrum, we
calculated a temperature decrement (eq. \ref{eqn:tdec}) and
re-classified the cluster as having a CC or NCC. Using the new set of
mock classifications we assigned a reliability factor, $\psi$, to each
real classification, which is simply the fraction of mock
classifications which agree with the real classification. A value of
$\psi = 1.0$ indicates complete agreement, with $\psi = 0.0$
indicating no agreement. When we removed clusters with $\psi < 0.9$
and repeated the analysis above, we found no significant change in the
trend of a steeper decrease in the relative number of CC versus NCC
clusters as a function of $T_{HBR}$.

Recall that the coolest ICM gas is being redshifted out of the
observable band as $z$ increases and becomes a significant effect at
$z \geq 0.6$ (\S\ref{sec:ebandsimulated}). Thus, we are likely not detecting
``weak'' CCs in the highest redshift clusters of our sample and
consequently these cores are classified as NCCs and are artificially
increasing the NCC population. When we excluded the 14 clusters at $z
\geq 0.6$ from this portion of our analysis and repeated the
calculations, we found no significant change in the results.

%%%%%%%%%%%%%%%%%%%%%%%%%%%%%%%%%%%%%%%%%%
\subsubsection{Mergers Versus Nonmergers}
\label{sec:ebandmerge}
%%%%%%%%%%%%%%%%%%%%%%%%%%%%%%%%%%%%%%%%%%

Looking for a correlation between cluster relaxation and a skewing in
$T_{HBR}$ was the primary catalyst of this work. The result that
increasing values of $T_{HBR}$ are more likely to be associated with
clusters harboring non-cool cores gives weight to that
hypothesis. But, the simplest relation to investigate is if $T_{HBR}$
is preferentially higher in merger systems. Thus, we now discuss
clusters with the highest significant values of $T_{HBR}$ and attempt
to establish, via literature based results, the dynamic state of these
systems.

The subsample of clusters on which we focus have a $T_{HBR} > 1.1$ at
90\% confidence for both their $R_{2500-\mathrm{CORE}}$ and
$R_{5000-\mathrm{CORE}}$ apertures. These clusters are listed in Table
\ref{tab:tf11} and are sorted by the lower limit of $T_{HBR}$. Shown
in Figure \ref{fig:ftx_tx} is a plot of $T_{HBR}$ versus $T_{0.7-7.0}$
for all the clusters in our sample. The clusters discussed in this
section are shown as green triangles and black stars. The clusters
with only a $R_{5000-\mathrm{CORE}}$ analysis are listed separately at
the bottom of the table. All 33 clusters listed have a core
classification of $\psi > 0.9$ (see \S\ref{sec:ebandccncc}). The choice of
the $T_{HBR} > 1.1$ threshold was arbitrary and intended to limit the
number of clusters to which we pay individual attention, but which is
still representative of mid- to high-$T_{HBR}$ values. Only two
clusters -- Abell 697 and MACS J2049.9-3217 -- do not have a $T_{HBR}
> 1.1$ in one aperture and not the other. In both cases although, this
was the result of the lower boundary narrowly missing the cut, but
both clusters still have $T_{HBR}$ significantly greater than unity.

For those clusters which have been individually studied, they are
listed as mergers based on the conclusions of the literature authors
(cited in Table \ref{tab:tf11}). Many different techniques were used
to determine if a system is a merger: bimodal galaxy velocity
distributions, morphologies, highly asymmetric temperature
distributions, ICM substructure correlated with subclusters, or
disagreement of X-ray and lensing masses. From Table \ref{tab:tf11} we
can see clusters exhibiting the highest significant values of
$T_{HBR}$ tend to be ongoing or recent mergers. At the 2$\sigma$
level, we find increasing values of $T_{HBR}$ favor merger systems
with NCCs over relaxed, CC clusters. It appears mergers have left a
spectroscopic imprint on the ICM which was predicted by ME01 and which
we observe in our sample.

\begin{figure}
\begin{center}
\includegraphics*[width=\textwidth, trim=15mm 10mm 0mm 0mm, clip]{eband_f8.eps}
\caption[Plot of $T_{HBR}$ vs. broadband temperatures color-coded for
  different cluster types]{$T_{HBR}$ plotted against $T_{0.7-7.0}$ for
  the $R_{2500-\mathrm{CORE}}$ and $R_{5000-\mathrm{CORE}}$
  apertures. Note that the vertical scales for both panels are not the
  same. The top and bottom panels contain \ebandnuma\ and \ebandnumb\
  clusters, respectively. Only two clusters -- Abell 697 and MACS
  J2049.9-3217 -- do not have a $T_{HBR} > 1.1$ in one aperture and
  not the other. In both cases however, it was a result of narrowly
  missing the cut. The dashed lines are the lines of
  equivalence. Symbols and color coding are based on two criteria: (1)
  the presence of a CC and (2) the value of $T_{HBR}$. Black stars (6
  in the top panel; 7 in the bottom) are clusters with a CC and
  $T_{HBR}$ significantly greater than 1.1. Green upright-triangles
  (21 in the top; 27 in the bottom) are NCC clusters with $T_{HBR}$
  significantly greater than 1.1. Blue down-facing triangles (49 top;
  60 bottom) are CC clusters and red squares (90 top; 98 bottom) are
  NCC clusters. We have found most, if not all, of the clusters with
  $T_{HBR} \gtrsim 1.1$ are merger systems. Note that the cut at
  $T_{HBR} > 1.1$ is arbitrary and there are more merger systems in
  our sample then just those highlighted in this figure. However it is
  rather suggestive that clusters with the highest values of $T_{HBR}$
  appear to be merging systems.  }
\label{fig:ftx_tx}
\end{center}
\end{figure}

Of the 33 clusters with $T_{HBR}$ significantly $> 1.1$, only 7 have
CCs. Three of those -- MKW3S, 3C 28.0, and RX J1720.1+2638 -- have
their apertures centered on the bright, dense cores in confirmed
mergers. Two more clusters -- Abell 2384 and RX J1525+0958 -- while
not confirmed mergers, have morphologies which are consistent with
powerful ongoing mergers. Abell 2384 has a long gas tail extending
toward a gaseous clump which we assume has recently passed through the
cluster. RXJ1525 has a core shaped like a rounded arrowhead and is
reminiscent of the bow shock seen in 1E0657-56. Abell 907 has no signs
of being a merger system, but the highly compressed surface brightness
contours to the west of the core are indicative of a prominent cold
front, a tell-tale signature of a subcluster merger event
\citep{2007PhR...443....1M}. Abell 2029 presents a very interesting
and curious case because of its seemingly high state of relaxation and
prominent cool core. There are no complementary indications it has
experienced a merger event. Yet its core hosts a wide-angle tail radio
source. It has been suggested that such sources might be attributable
to cluster merger activity \citep{2000MNRAS.311..649S}. Moreover, the
X-ray isophotes to the west of the bright, peaked core are slightly
more compressed and may be an indication of past gas sloshing
resulting from the merger of a small subcluster. Both of these
features have been noted previously, specifically by
\citet{2004ApJ...616..178C, 2005xrrc.procE7.08C}. We suggest the
elevated $T_{HBR}$ value for this cluster lends more weight to the
argument that A2029 has indeed experienced a merger recently, but how
long ago we do not know.

The remaining systems we could not verify as mergers -- RX
J0439.0+0715, MACS J2243.3-0935, MACS J0547.0-3904, Zwicky 1215, MACS
J2311+0338, Abell 267, and NGC 6338 -- have NCCs and X-ray
morphologies consistent with an ongoing or post-merger scenario. Abell
1204 shows no signs of recent or ongoing merger activity; however, it
resides at the bottom of the arbitrary $T_{HBR}$ cut, and as evidenced
by Abell 401 and Abell 1689, exceptional spherical symmetry is no
guarantee of relaxation. Our analysis here is partially at the mercy
of morphological assessment, and only a more stringent study of a
carefully selected subsample or analysis of simulated clusters can
better determine how closely correlated $T_{HBR}$ is with the timeline
of merger events.

%%%%%%%%%%%%%%%%%%%%%%%%%%%%%%%%%
\section{Summary and Conclusions}
\label{sec:ebandsummary}
%%%%%%%%%%%%%%%%%%%%%%%%%%%%%%%%%

We have explored the band dependence of the inferred X-ray temperature
of the ICM for \ebandnumb\ well-observed ($N_{counts} > 1500$) clusters
of galaxies selected from the {\it Chandra} Data Archive.

We extracted spectra from the annulus between $R=70$ kpc and
$R=R_{2500}$, $R_{5000}$ for each cluster. We compared the X-ray
temperatures inferred for single-component fits to global spectra when
the energy range of the fit was 0.7-7.0 keV (broad) and when the
energy range was $2.0/(1+z)$-7.0 keV (hard). We found that, on
average, the hard-band temperature is significantly higher than the
broadband temperature. For the $R_{2500-\mathrm{CORE}}$ aperture we
measured a weighted average of $T_{HBR} = 1.16$ with $\sigma = \pm
0.10$ and $\sigma_{mean} = \pm 0.01$ for the $R_{5000-\mathrm{CORE}}$
aperture, and $T_{HBR} = 1.14$ with $\sigma = \pm 0.12$ and
$\sigma_{mean} = \pm 0.01$. We also found no systematic trends in the
value of $T_{HBR}$, or the dispersion of $T_{HBR}$, with S/N,
redshift, Galactic absorption, metallicity, observation date, or
broadband temperature.

In addition, we simulated an ensemble of 12,765 spectra which
contained observation-specific and idealized two-temperature component
models, plus a control sample of single-temperature models. From
analysis of these simulations we found the statistical fluctuations
for a single temperature model are inadequate to explain the
significantly different $T_{0.7-7.0}$ and $T_{2.0-7.0}$ we measure in
our sample. We also found that the observed scatter, $\sigma_{HBR}$,
is consistent with the presence of unresolved cool ($T_X < 2.0$ keV)
gas contributing a minimum of $>10\%$ of the total emission. The
simulations also show the measured observational scatter in $T_{HBR}$
is greater than the statistical scatter, $\sigma_{control}$. These
results are consistent with the process of hierarchical cluster
formation.

Upon further exploration, we found that $T_{HBR}$ is enhanced
preferentially for clusters which are known merger systems and for
clusters without cool cores. Clusters with temperature decrements in
their cores (known as cool-core clusters) tend to have best-fit
hard-band temperatures that are consistently closer to their best-fit
broadband temperatures. The correlation of $T_{HBR}$ with the type of
cluster core is insensitive to our choice of classification scheme and
is robust against redshift effects. Our results qualitatively support
the finding by ME01 that the temperature ratio, $T_{HBR}$, might
therefore be useful for statistically quantifying the degree of
cluster relaxation/virialization.

An additional robust test of the ME01 finding should be made with
simulations by tracking $T_{HBR}$ during hierarchical assembly of a
cluster. If $T_{HBR}$ is tightly correlated with a cluster's degree of
relaxation, then it, along with other methods of substructure measure,
may provide a powerful metric for predicting (and therefore reducing)
a cluster's deviation from mean mass-scaling relations. The task of
reducing scatter in scaling relations will be very important if we are
to reliably and accurately measure the mass of clusters.

%%%%%%%%%%%%%%%%%%%%%%%%%%
\section{Acknowledgments}
%%%%%%%%%%%%%%%%%%%%%%%%%%

K. W. C. was supported in this work by the National Aeronautics and
Space Administration through {\it Chandra} X-Ray Observatory Archive
grants AR-6016X and AR-4017A, with additional support from a start-up
grant for Megan Donahue from Michigan State University. M. D.  and
Michigan State University acknowledge support from the NASA LTSA
program NNG-05GD82G. G. M. V. thanks NASA for support through theory
grant NNG-04GI89G. The {\it Chandra} X-ray Observatory Center is
operated by the Smithsonian Astrophysical Observatory for and on
behalf of the National Aeronautics Space Administration under contract
NAS8-03060. This research has made use of software provided by the
{\it Chandra} X-ray Center (CXC) in the application packages \ciao,
{\textsc{ChIPS}}, and {\textsc{Sherpa}}. We thank Alexey Vikhlinin for
helpful insight and expert advice. K. W. C. also thanks attendees of
the ``Eight Years of Science with {\it Chandra} Calibration Workshop''
for stimulating discussion regarding {\it XMM}-{\it Chandra}
cross-calibration. K. W. C. especially thanks Keith Arnaud for
personally providing support and advice for mastering
{\textsc{XSPEC}}. This research has made use of the NASA/IPAC
Extragalactic Database (NED), which is operated by the Jet Propulsion
Laboratory, California Institute of Technology, under contract with
the National Aeronautics and Space Administration. This research has
also made use of NASA's Astrophysics Data System. {\it ROSAT} data and
software were obtained from the High Energy Astrophysics Science
Archive Research Center (HEASARC), provided by NASA's Goddard Space
Flight Center.
