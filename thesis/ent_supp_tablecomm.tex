%%%%%%%%%%%%%%%%%%%%%%%%%%%%%%%%%%%%%%%%%%%%%%%%
\chapter{Tables cited in Chapter \ref{ch:ent_supp}}
%%%%%%%%%%%%%%%%%%%%%%%%%%%%%%%%%%%%%%%%%%%%%%%%

\begin{center}
\noindent{\bf{Table \ref{tab:entsuppsample} Notes}}\\
\end{center}
Col. (1) Cluster name; col. (2) CXC CDA Observation Identification
Number; col. (3) R.A. of cluster center; col. (4) Decl. of cluster
center; col. (5) exposure time; col. (6) observing mode; col. (7) CCD
location of cluster center; col. (8) redshift; col. (9) average
cluster temperature; col. (10) core entropy measured in this work;
col. (11) cluster bolometric luminosity; and col. (12) notes are as
follows: (a) - cluster analyzed using the best-fit $\beta$-model for
the surface brightness profiles (discussed in
\S\ref{sec:entsuppdene}); (b) - clusters with complex surface brightness of
which only the central regions were used in fitting $K(r)$; (c) -
cluster only used during analysis of the \hifl\ sub-sample (discussed
in \S\ref{sec:entsupphifl}); (d) - cluster with central AGN removed during
analysis (discussed in \S\ref{sec:entsuppcentsrc}); (e) - cluster with
central compact source removed during analysis (discussed in
\S\ref{sec:entsuppcentsrc}); and (f) - cluster with central bin ignored
during fitting (discussed in \S\ref{sec:entsuppcentsrc}).

\begin{center}
\noindent{\bf{Table \ref{tab:betafits} Notes}}\\
\end{center}
Col. (1) Cluster name; col. (2) central surface brightness of first
component; col. (3) core radius of first component; col. (4) $\beta$
parameter of first component; col. (5) central surface brightness of
second component; col. (6) core radius of second component; col. (7)
$\beta$ parameter of second component; col. (8) model degrees of
freedom; and col. (9) reduced chi-squared statistic for best-fit
model.

\begin{center}
\noindent{\bf{Table \ref{tab:bfparams} Notes}}\\
\end{center}
Listed here are the mean best-fit parameters of the model $K(r) = \kna
+ \khun (r/100 \kpc)^{\alpha}$ for various sub-groups of the full
\accept\ sample. The 'CSE' sample are the clusters with a central
source excluded (discussed in \S\ref{sec:entsuppcentsrc}). The $K_{12}$
values represent the entropy at 12 kpc and are calculated from the
best-fit models. Col. (1) Sample being considered; col. (2) number of
objects in the sub-group; col. (3) fraction of objects with p $>$ 0.05
for power-law only model (eqn. \ref{eqn:plaw}); col. (4) fraction of
objects with p $>$ 0.05 for power-law with constant core entropy model
(eqn. \ref{eqn:k0}); col. (5) fraction of objects which do not meet p
$>$ 0.05 criterion for either model; col. (6) mean best-fit \kna;
col. (7) mean entropy at 12 kpc; col. (8) mean best-fit \khun; and
col. (9) mean best-fit power-law index; and cols. (10,11,12) number of
clusters consistent with $\kna = 0 \ent$ at $1\sigma$, $2\sigma$, and
$3\sigma$ significance, respectively. Percentage of the sub-group
represented by each is also listed.

\begin{center}
\noindent{\bf{Table \ref{tab:kfits} Notes}}\\
\end{center}
Col. (1) Cluster name; col. (2) CDA observation identification number;
col. (3) method of $T_X$ interpolation (discussed in \S\ref{sec:entsuppkpr});
col. (4) maximum radius for fit; col. (5) number of radial bins
included in fit; col. (6) best-fit core entropy; col. (7) number of
sigma \kna\ is away from zero; col. (9) best-fit entropy at 100 kpc;
col. (10) best-fit power-law index; col. (11) degrees of freedom in
fit; col. (12) \chisq\ statistic of best-fit model; and col. (13)
probability of worse fit given \chisq\ and degrees of freedom.
