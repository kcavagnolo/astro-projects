\begin{figure}[htp]
  \begin{center}
    \begin{minipage}[htp]{0.9\linewidth}
      \includegraphics*[width=\textwidth, trim=15mm 10mm 10mm 10mm, clip]{beta.eps}
      \caption[Surface brightness profiles for clusters requiring a
        $\beta$-model fit for deprojection]{Surface brightness profiles for clusters requiring a
        $\beta$-model fit for deprojection (discussed in
        \S\ref{sec:entsuppbeta}). The best-fit $\beta$-model for each cluster
        is overplotted as a dashed line. The discrepancy between the
        data and best-fit model for some clusters results from the
        presence of a compact X-ray source at the center of the
        cluster. These cases are discussed belows.}
      \label{fig:betamods}
    \end{minipage}
  \end{center}
\end{figure}

\begin{description}
\item[Abell 119 ($z=0.0442$):] This is a highly diffuse cluster
  without a prominent cool core. The large core region and slowly
  varying surface brightness made deprojection highly unstable. We
  have excluded a small source at the very center of the BCG. The
  exclusion region for the source is $\approx 2.2\arcs$ in radius
  which at the redshift of the cluster is $\sim 2$ kpc. This cluster
  required a double $\beta$-model.

\item[Abell 160 ($z=0.0447$):] The highly asymmetric, low surface
  brightness of this cluster resulted in a noisy surface brightness
  profile that could not be deprojected. This cluster required a
  double $\beta$-model. The BCG hosts a compact X-ray source. The
  exclusion region for the compact source has a radius of $\sim
  5\arcs$ or $\sim 4.3$ kpc. The BCG for this cluster is not
  coincident with the X-ray centroid and hence is not at the
  zero-point of our radial analysis.

\item[Abell 193 ($z=0.0485$):] This cluster has an azimuthally
  symmetric and a very diffuse ICM centered on a BCG which is
  interacting with a companion galaxy. In Fig. \ref{fig:betamods} one
  can see that the central three bins of this cluster's surface
  brightness profile are highly discrepant from the best-fit
  $\beta$-model. This is a result of the BCG being coincident with a
  bright, compact X-ray source. As we have concluded in
  \ref{sec:entsuppcentsrc}, compact X-ray sources are excluded from our
  analysis as they are not the focus of our study here. Hence we have
  used the best-fit $\beta$-model in deriving $K(r)$ instead of the
  raw surface brightness.

\item[Abell 400 ($z=0.0240$):] The two ellipticals at the center of
  this cluster have compact X-ray sources which are excluded during
  analysis. The core entropy we derive for this cluster is in
  agreement with that found by \cite{2006A&A...453..433H} which
  supports the accuracy of the $\beta$-model we have used.

\item[Abell 1060 ($z=0.0125$):] There is a distinct compact source
  associated with the BCG in this cluster. The ICM is also very faint
  and uniform in surface brightness making the compact source that
  much more obvious. Deprojection was unstable because of imperfect
  exclusion of the source.

\item[Abell 1240 ($z=0.1590$):] The surface brightness of this cluster
  is well-modeled by a $\beta$-model. There is nothing peculiar worth
  noting about the BCG or the core of this cluster.

\item[Abell 1736 ($z=0.0338$):] Another ``boring'' cluster with a very
  diffuse low surface brightness ICM, no peaky core, and no signs of
  merger activity in the X-ray. The noisy surface brightness profile
  necessitated the use of a double $\beta$-model. The BCG is
  coincident with a very compact X-ray source, but the BCG is offset
  from the X-ray centroid and thus the central bins are not adversely
  affected. The radius of the exclusion region for the compact source
  is $\approx 2.3\arcs$ or $1.5$ kpc.

\item[Abell 2125 ($z=0.2465$):] Although the ICM of this cluster is
  very similar to the other clusters listed here (\ie\ diffuse, large
  cores), A2125 is one of the more compact clusters. The presence of
  several merging sub-clusters \citep{1997ApJ...487L..13W,
    2004ApJ...611..821W} to the NW of the main cluster form a diffuse
  mass which cannot rightly be excluded. This complication yields
  inversions of the deprojected surface brightness profile if a double
  $\beta$-model is not used.

\item[Abell 2255 ($z=0.0805$):] This is a very well studied merger
  cluster \citep{1995ApJ...446..583B, 1997A&A...317..432F}. The core
  of this cluster is very large ($r > 200$ kpc). Such large extended
  cores cannot be deprojected using our methods because if too many
  neighboring bins have approximately the same surface brightness,
  deprojection results in bins with negative or zero value. The
  surface brightness for this cluster is well modeled as a $\beta$
  function.

\item[Abell 2319 ($z=0.0562$):] A2319 is another well studied merger
  cluster \citep{1997NewA....2..501F, 1999ApJ...525L..73M} with a very
  large core region ($r > 100$ kpc) and a prominent cold front
  \citep{2004ApJ...604..604O}. Once again, the surface brightness
  profile is well-fit by a $\beta$-model.

\item[Abell 2462 ($z=0.0737$):] This cluster is very similar in
  appearance to A193: highly symmetric ICM with a bright, compact
  X-ray source embedded at the center of an extended diffuse ICM. The
  central compact source has been excluded from our analysis with a
  region of radius $\approx 1.5\arcs$ or $\sim 3$ kpc. The central
  bin of the surface brightness profile is most likely boosted above
  the best-fit double $\beta$-model because of faint extended emission
  from the compact source which cannot be discerned from the ambient
  ICM.

\item[Abell 2631 ($z=0.2779$):] The surface brightness profile for
  this cluster is rather regular, but because the cluster has a large
  core it suffers from the same unstable deprojection as A2255 and
  A2319. The ICM is symmetric about the BCG and is incredibly uniform
  in the core region. We did not detect or exclude a source at the
  center of this cluster, but under heavy binning the cluster image
  appears to have a source coincident with the BCG, and the slightly
  higher flux in central bin of the surface brightness profile may be
  a result of an unresolved source.

\item[Abell 3376 ($z=0.0456$):] The large core of this cluster ($r >
  120$ kpc) makes deprojection unstable and a $\beta$-model must be
  used.

\item[Abell 3391 ($z=0.0560$):] The BCG is coincident with a compact
  X-ray source. The source is excluded using a region with radius
  $\approx 2\arcs$ or $\sim 2$ kpc. The large uniform core region
  made deprojection unstable and thus required a $\beta$-model fit.

\item[Abell 3395 ($z=0.0510$):] The surface brightness profile for
  this cluster is noisy resulting in deprojection inversions and
  requiring a $\beta$-model fit. The BCG of this cluster has a compact
  X-ray source and this source was excluded using a region with radius
  $\approx 1.9\arcs$ or $\sim 2$ kpc.

\item[MKW 08 ($z=0.0270$):] MKW 08 is a nearby large group/poor
  cluster with a pair of interacting elliptical galaxies in the
  core. The BCG falls directly in the middle of the ACIS-I detector
  gap. However, despite the lack of proper exposure, CCD dithering
  reveals that a very bright X-ray source is associated with the
  BCG. A double $\beta$-model was necessary for this cluster because
  the low surface brightness of the ICM is noisy and deprojection is
  unstable.

\item[RBS 461 ($z=0.0290$):] This is another nearby large group/poor
  cluster with an extended, diffuse, axisymmetric, featureless ICM
  centered on the BCG. The BCG is coincident with a compact source
  with size $r \approx 1.7$ kpc. This source was excluded during
  reduction. The $\beta$-model is a good fit to the surface brightness
  profile.
\end{description}

%% %%%%%%%%%%%%%%%%%%%%%%%%%%%%%%%%%%%%%%%%%%%%%%%%%%%%%%%
%% \section{Notes on clusters with central source removed}
%% \label{app:centsrc}
%% %%%%%%%%%%%%%%%%%%%%%%%%%%%%%%%%%%%%%%%%%%%%%%%%%%%%%%%

%% 2PIGG_J0011.5-2850, 3C_388, 4C_55.16, ABELL_0223, ABELL_0426, ABELL_0539, ABELL_0562,
%% ABELL_0576, ABELL_0611, ABELL_0744, ABELL_2052, ABELL_2151,
%% ABELL_2717, ABELL_3112, ABELL_3558, ABELL_3581, ABELL_3822,
%% CYGNUS_A, HYDRA_A, M87, MACS_J0547.0-3904, MACS_J1931.8-2634,
%% RBS_0797, RX_J1320.2+3308, ZwCl_0857.9+2107, ZWICKY_1742

%% The clusters A119, A160, A193, A1736, A2462, A3391, A3395, and RBS461
%% also have a central source removed during analysis, but they are
%% discussed in Appendix \ref{app:beta}.

%% \begin{description}
%% \item[3C 295 ($z=0.4641$):] The core of this cluster has been
%% studied in detail by \cite{2001MNRAS.324..842A}. In the central 50 kpc
%% \cite{2001MNRAS.324..842A} found, as we do, that the temperature drops
%% from $\sim 5.0$ keV to $\sim 3.5$ keV. \cite{2001MNRAS.324..842A} also
%% derive a mass deposition rate of $\dot{M} = 280~\msolpy$ indicating
%% the core of this cluster has a strong cooling flow. As was done in
%% \cite{2001MNRAS.324..842A}, three sources are excluded from the core
%% during our analysis: the region surrounding the central AGN and two
%% nearby radio hot spots \citep{2000ApJ...530L..81H}.

%% \item[3C 388 ($z=0.0917$):]
%% From \cite{2006ApJ...639..753K}:
%% in process of CF quenching
%% The radio galaxy 3C 388 is classified as a Fanaroff-Riley type II (FR
%% II) radio galaxy, although its luminosity ( W Hz−1; Fanaroff \& Riley
%% 1974) lies near the FR I/II dividing line. The radio morphology of
%% this source is closer to a “fat double” (Owen \& Laing 1989) than the
%% canonical FR II “classical double” such as Cyg A and 3C 98. Optically,
%% the nucleus of 3C 388 is classified as a low-excitation radio galaxy
%% (Jackson \& Rawlings 1997). Multifrequency VLA observations of 3C 388
%% show significant structure in spectral index maps that has been
%% interpreted as evidence for multiple nuclear outbursts (Roettiger et
%% al. 1994). Previous X-ray observations of this radio galaxy have shown
%% that it is embedded in a cluster environment (Feigelson \& Berg 1983;
%% Hardcastle \& Worrall 1999; Leahy \& Gizani 2001). The local galaxy
%% environment is extremely dense (Prestage \& Peacock 1988), and the
%% central elliptical galaxy that hosts 3C 388 is one of the most
%% luminous (MB=-24.24) in the local universe (Owen \& Laing 1989; Martel et
%% al. 1999).

%% \item[4C 55.16 ($z=0.2420$):]
%% From \cite{2001MNRAS.328L...5I}:
%% 4C+55.16 is a compact powerful radio source residing in a large galaxy
%% at a redshift of 0.240 (Pearson \& Readhead 1981, 1988; Whyborn et
%% al. 1985; Hutchings, Johnson \& Pyke 1988). Recently, luminous cluster
%% emission (~1045 erg s−1) around the radio galaxy has been recognized
%% through ASCA and ROSAT High Resolution Imager (HRI) observations
%% (Iwasawa et al. 1999).  The point source at the nucleus shows a hard
%% X-ray spectrum, which can be attributed naturally to non-thermal
%% emission from the active nucleus.

%% \item[Abell 223 ($z=0.2070$):]
%% From \cite{2000A&A...355..443P}:
%% As already noticed by Sandage et al. (1976), these two neighboring
%% clusters have nearly the same redshift and probably constitute an
%% interacting system which is going to merge in the future. Both are
%% dominated by a particularly bright cD galaxy. They have a richness
%% class R=3 and are X-ray luminous with [FORMULA] and [FORMULA] for A
%% 222 and A 223, respectively (Lea \& Henry 1988). The BOW83 sample
%% covers only the central regions of these two clusters and, in order to
%% study the galaxy distribution in these systems, as well as to estimate
%% the projected density for the galaxies in our sample (see below), we
%% have built a more extensive, although shallower, galaxy catalog,
%% covering a region of [FORMULA] centered on the median position of the
%% two clusters. This catalogue, with 356 objects, was extracted from
%% Digital Sky Survey (DSS) images, using the software SExtractor (Bertin
%% \& Arnouts 1996). It is more than 90\% complete to BOW83 magnitudes
%% [FORMULA].

%% \item[Abell 426 ($z=0.0179$):]
%% Come on, it's Perseus, you don't know about this cluster? Well, it's
%% got an AGN, just ask \cite{perseus1, perseus2, perseus3}.

%% \item[Abell 539 ($z=0.0288$):]
%% From \cite{1988AJ.....96.1775O}:
%% Within 1 Mpc of the center, the physical parameters of A539 are found
%% to be typical of those of rich clusters. It is shown that early-type
%% galaxies are more concentrated toward the cluster center and that the
%% velocity distributions of early-type and late-type galaxies differ
%% marginally.

%% \item[Abell 562 ($z=0.1100$):]
%% From \cite{1997ApJ...474..580G}:
%% The X-ray emission from this cluster is elongated and shows the radio
%% source offset from the central X-ray peak by 30''. The substructure
%% test (Table A1) detects a significant X-ray excess east of the
%% WAT. The radio pressure is in rough agreement with the thermal
%% pressure. Optically, the cluster is dominated by the WAT host galaxy.

%% \item[Abell 576 ($z=0.0385$):]
%% From \cite{1996ApJ...470..724M}:
%% The central cluster region contains a nonemission galaxy population
%% and an intracluster medium which is significantly cooler (σ\_core\_ =
%% 387\_-105\_\^+250\^ km s\^-1\^ and T\_x\_ = 1.6\_-0.3\_\^+0.4\^ keV at 90\%
%% confidence) than the global populations (σ = 977\_-96\_\^+124\^ km s\^- 1\^
%% for the nonemission population and T\_X\_ > 4 keV at 90\%
%% confidence). Because (1) the low-dispersion galaxy population is no
%% more luminous than the global population and (2) the evidence for a
%% cooling flow is weak, we suggest that the core of A576 may contain the
%% remnants of a lower mass subcluster.
%% From \cite{2004ApJ...607..220K}:
%% We present data from a Chandra observation of the nearby cluster of
%% galaxies A576. The core of the cluster shows a significant departure
%% from dynamical equilibrium. We show that this core gas is most likely
%% the remnant of a merging subcluster, which has been stripped of much
%% of its gas, depositing a stream of gas behind it in the main
%% cluster. The unstripped remnant of the subcluster is characterized by
%% a different temperature, density, and metallicity than that of the
%% surrounding main cluster, suggesting its distinct origin. Continual
%% dissipation of the kinetic energy of this minor merger may be
%% sufficient to counteract most cooling in the main cluster over the
%% lifetime of the merger event.

%% \item[Abell 611 ($z=0.2880$):]
%% From \cite{2002MNRAS.337.1207G}:
%% Abell 611 is a cluster at z= 0.288 (Crawford et al. 1995) originally
%% identified by Abell (1957). It has a 0.1–2.4 keV luminosity of 8.63 ×
%% 1044 W (Böhringer et al. 2000), with a temperature of 7.95+0.56−0.52×
%% 107 K (White 2000). White derived this value from a 57-ks ASCA
%% exposure by considering both a single-phase and two-phase cooling
%% model. The temperature values found for the bulk of the gas are
%% statistically equivalent, and a mass deposit rate of 0+177−0 Mo yr−1
%% was found for the cooling model. The 17-ks ROSAT HRI observation from
%% 1996 April is shown with 8-arcsec binning in Fig. 7. The image
%% contains two bright pixels, which, on comparison with the POSS image,
%% are coincident with a large galaxy. These pixels are ignored whilst
%% fitting a model to this observation.

%% \item[Abell 744 ($z=0.0729$):]
%% From \cite{1985AJ.....90.1665K}:
%% The authors present X-ray and optical observations of the cluster of
%% galaxies Abell 744. The X-ray flux (assuming H0 = 100 km s-1Mpc-1) is
%% ≡9×1042erg s-1. The X-ray source is extended, but shows no other
%% structure. The authors present photographic photometry (in
%% Kron-Cousins R), calibrated by deep CCD frames, for all galaxies
%% brighter than 19th magnitude within 0.75 Mpc of the cluster
%% center. The luminosity function is normal, and the isopleths show
%% little evidence of substructure near the cluster center. The cluster
%% has a dominant central galaxy which the authors classify as a normal
%% brightest-cluster elliptical on the basis of its luminosity
%% profile. New redshifts were obtained for 26 galaxies in the vicinity
%% of the cluster center; 20 appear to be cluster members. The spatial
%% distribution of redshifts is peculiar; the dispersion within the 150
%% kpc core radius is much greater than outside. Abell 744 is similar to
%% the nearby cluster Abell 1060.

%% \item[Abell 2052 ($z=0.0353$):]
%% AGN \cite{2001ApJ...558L..15B, 2003ApJ...585..227B}.

%% \item[Abell 2151 ($z=0.0366$):]
%% From \cite{1995AJ....109..465M}:
%% it's a merger with three distinct pops in vel disp space.

%% \item[Abell 2717 ($z=0.0475$):]
%% From \cite{1997A&A...321...64L}:
%% We present an X-ray, radio and optical study of the cluster A
%% 2717. The central D galaxy is associated with a Wide-Angled-Tailed
%% (WAT) radio source. A Rosat PSPC observation of the cluster shows that
%% the cluster has a well constrained temperature of
%% 1.9\^+0.3\^\_-0.2\_x10\^7\^K. The pressure of the intracluster medium was
%% found to be comparable to the mininum pressure of the radio source
%% suggesting that the tails may in fact be in equipartition with the
%% surrounding hot gas.

%% \item[Abell 3112 ($z=0.0720$):]
%% It's one of Lieu's soft excess clusters \cite{2007ApJ...668..796B}
%% searching for cold gas in A3112 \cite{2004A&A...421..503L}
%% From \cite{2003ApJ...595..142T}:
%% We present the results of a Chandra observation of the central region
%% of A3112. This cluster has a powerful radio source in the center and
%% was believed to have a strong cooling flow. The X-ray image shows that
%% the intracluster medium (ICM) is distributed smoothly on large scales
%% but has significant deviations from a simple concentric elliptical
%% isophotal model near the center. Regions of excess emission appear to
%% surround two lobelike radio-emitting regions. This structure probably
%% indicates that hot X-ray gas and radio lobes are interacting. From an
%% analysis of the X-ray spectra in annuli, we found clear evidence for a
%% temperature decrease and abundance increase toward the center. The
%% X-ray spectrum of the central region is consistent with a
%% single-temperature thermal plasma model. The contribution of X-ray
%% emission from a multiphase cooling flow component with gas cooling to
%% very low temperatures locally is limited to less than 10\% of the
%% total emission. However, the whole cluster spectrum indicates that the
%% ICM is cooling significantly as a whole, but only in a limited
%% temperature range (>=2 keV). Inside the cooling radius the conduction
%% timescales based on the Spitzer conductivity are shorter than the
%% cooling timescales. We detect an X-ray point source in the cluster
%% center that is coincident with the optical nucleus of the central cD
%% galaxy and the core of the associated radio source. The X-ray spectrum
%% of the central point source can be fitted by a 1.3 keV thermal plasma
%% and a power-law component whose photon index is 1.9. The thermal
%% component is probably plasma associated with the cD galaxy. We
%% attribute the power-law component to the central active galactic
%% nucleus.

%% \item[Abell 3558 ($z=0.0480$):]
%% From \cite{2007A&A...463..839R}:
%% Combining XMM-Newton and Chandra data, we have performed a detailed
%% study of A3558. Our analysis shows that its dynamical history is more
%% complicated than previously thought. We have found some traits typical
%% of cool core clusters (surface brightness peaked at the center, peaked
%% metal abundance profile) and others that are more common in merging
%% clusters, like deviations from spherical symmetry in the thermodynamic
%% quantities of the ICM. This last result has been achieved with a new
%% technique for deriving temperature maps from images. We have also
%% detected a cold front and, with the combined use of XMM-Newton and
%% Chandra, we have characterized its properties, such as the speed and
%% the metal abundance profile across the edge. This cold front is
%% probably due to the sloshing of the core, induced by the perturbation
%% of the gravitational potential associated with a past merger. The
%% hydrodynamic processes related to this perturbation have presumably
%% produced a tail of lower entropy, higher pressure and metal rich ICM,
%% which extends behind the cold front for~500 kpc. The unique
%% characteristics of A3558 are probably due to the very peculiar
%% environment in which it is located: the core of the Shapley
%% supercluster.

%% \item[Abell 3581 ($z=0.0218$):]
%% From \cite{2007A&A...463..839R}:
%% We present results from an analysis of a Chandra observation of the
%% cluster of galaxies A3581. We discover the presence of a point-source
%% in the central dominant galaxy that is coincident with the core of the
%% radio source PKS 1404-267. The emission from the intracluster medium
%% is analysed, both as seen in projection on the sky, and after
%% correcting for projection effects, to determine the spatial
%% distribution of gas temperature, density and metallicity. We find that
%% the cluster, despite hosting a moderately powerful radio source, shows
%% a temperature decline to around 0.4 Tmax within the central 5 kpc. The
%% cluster is notable for the low entropy within its core. We test and
%% validate the XSPEC PROJCT model for determining the intrinsic cluster
%% gas properties.

%% \item[Abell 3822 ($z=0.0759$):]
%% zero literature, seriously, only mentioned in survey papers.

%% \item[Cygnus A ($z=0.0561$):]
%% AGN \cite{2002ApJ...565..195S}

%% \item[Hydra A ($z=0.0549$):]
%% AGN \cite{2000ApJ...534L.135M, 2001ApJ...557..546D,
%%   2002ApJ...568..163N}

%% \item[M87 ($z=0.0044$):]
%% AGN \cite{2002ApJ...564..683M, 2005ApJ...635..894F}

%% \item[MACS J0547.0-3904 ($z=0.2100$):]
%% no lit

%% \item[MACS J1931.8-2634 ($z=0.3520$):]
%% no lit

%% \item[RBS 797 ($z=0.3540$):]
%% From \cite{2001A&A...376L..27S}:
%% We present CHANDRA observations of the X-ray luminous, distant galaxy
%% cluster RBS797 at z=0.35. In the central region the X-ray emission
%% shows two pronounced X-ray minima, which are located opposite to each
%% other with respect to the cluster centre. These depressions suggest an
%% interaction between the central radio galaxy and the intra-cluster
%% medium, which would be the first detection in such a distant
%% cluster. The minima are symmetric relative to the cluster centre and
%% very deep compared to similar features found in a few other nearby
%% clusters. A spectral and morphological analysis of the overall cluster
%% emission shows that RBS797 is a hot cluster (T=7.7+1.2-1.0 keV) with a
%% total mass of Mtot(r500)= 6.5+1.6-1.2 *E14Msun.

%% \item[RX J1320.2+3308 ($z=0.0366$):]
%% no lit

%% \item[ZwCl 0857.9+2107 ($z=0.2350$):]
%% no lit

%% \item[Zwicky 1742 ($z=0.0757$):]
%% brand new obs

%% \end{description}
