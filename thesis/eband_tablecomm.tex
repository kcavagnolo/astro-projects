%%%%%%%%%%%%%%%%%%%%%%%%%%%%%%%%%%%%%%%%%%%%%%%%
\chapter{Tables cited in Chapter \ref{ch:eband}}
%%%%%%%%%%%%%%%%%%%%%%%%%%%%%%%%%%%%%%%%%%%%%%%%

\begin{center}
{\bf{Table \ref{tab:sample} Notes}}\\
\end{center}
A ($\ddagger$) indicates a cluster analyzed within R$_{5000}$
only. Italicized cluster names indicate a cluster which was excluded
from our analysis (discussed in \S\ref{sec:ebandfitting}). For clusters
with multiple observations, the X-ray centers differ by $< 0.5$ kpc.
Col. (1) Cluster name; col. (2) CDA observation identification number;
col. (3) R.A. of cluster center; col. (4) Dec. of cluster center;
col. (5) nominal exposure time; col. (6) observing mode; col. (7) CCD
location of centroid; col. (8) redshift; col. (9) bolometric
luminosity.

\begin{center}
{\bf{Table \ref{tab:tf11} Notes}}\\
\end{center}
Clusters ordered by lower limit of $T_{HBR}$. Listed $T_{HBR}$ values
are for the $R_{2500-\mathrm{CORE}}$ aperture, with the exception of
the ``$R_{5000-\mathrm{CORE}}$ Only'' clusters listed at the end of
the table. Excluding the ``$R_{5000-\mathrm{CORE}}$ Only'' clusters,
all clusters listed here had $T_{HBR}$ significantly greater than 1.1
and the same core classification for both the $R_{2500-\mathrm{CORE}}$
and $R_{5000-\mathrm{CORE}}$ apertures. Numbered references given in
table: [1] \citet{1994ApJS...94..583G}, [2] \citet{2003ApJ...593..291K},
[3] \citet{2005ChJAA...5..126Y}, [4] \citet{1998ApJ...503...77M}, [5]
\citet{2006Sci...314..791B}, [6] \citet{1990ApJS...72..715T}, [7]
\citet{2004ApJ...607..190A}, [8] \citet{1995ApJ...446..583B}, [9]
\citet{1997A&A...317..432F}, [10] \citet{1997ApJ...490...56G}, [11]
\citet{2002ApJS..139..313D}, [12] \citet{2005MNRAS.359..417S}, [13]
\citet{1982ApJ...255L..17G}, [14] \citet{2004ApJ...610L..81H}, [15]
\citet{2004ApJ...614..692Y}, [16] \citet{2003A&A...408...57M}, [17]
\citet{2000ApJ...540..726G}, [18] \citet{1998ApJ...496L...5T}, [19]
\citet{1999AcA....49..403K}, [20] \citet{2001ApJ...555..205M}, [21]
\citet{2001A&A...379..807G}, [22] \citet{1998MNRAS.301..609B}, [23]
\citet{2005ApJ...619..161G}, [24] \citet{1996ApJ...472L..17M}, [25]
\citet{2001ASPC..251..474O}, [26] \citet{2000ApJ...534L..43M}, [27]
\citet{2004ApJ...616..178C}, [28] \citet{2005xrrc.procE7.08C}, [29] this
work.

\begin{center}
{\bf{Table \ref{tab:r2500specfits} Notes}}\\
\end{center}
Note: ``77'' refers to 0.7-7.0 keV band and ``27'' refers to 2.0-7.0
keV band. Col. (1) Cluster name; col. (2) size of excluded core region
in kpc, (3) $R_{2500}$ in kpc; col. (4) absorbing Galactic neutral
hydrogen column density; col. (5,6) best-fit {\textsc{MeKaL}}
temperatures; col. (7) $T_{0.7-7.0}$/$T_{2.0-7.0}$ also called
$T_{HBR}$; col. (8) best-fit 77 {\textsc{MeKaL}} abundance;
col. (9,10) respective reduced $\chisq$ statistics, and (11) percent
of emission attributable to source. A star ($\star$) indicates a
cluster which has multiple observations. Each observation has an
independent spectrum extracted along with an associated WARF, WRMF,
normalized background spectrum, and soft residual. Each independent
spectrum is then fit simultaneously with the same spectral model to
produce the final fit.

\begin{center}
{\bf{Table \ref{tab:r5000specfits} Notes}}\\
\end{center}
Note: ``77'' refers to 0.7-7.0 keV band and ``27'' refers to 2.0-7.0
keV band. Col. (1) Cluster name; col. (2) size of excluded core region
in kpc, (3) $R_{5000}$ in kpc; col. (4) absorbing Galactic neutral
hydrogen column density; col. (5,6) best-fit {\textsc{MeKaL}}
temperatures; col. (7) $T_{0.7-7.0}$/$T_{2.0-7.0}$ also called
$T_{HBR}$; col. (8) best-fit 77 {\textsc{MeKaL}} abundance;
col. (9,10) respective reduced $\chisq$ statistics, and (11) percent
of emission attributable to source. A star ($\star$) indicates a
cluster which has multiple observations. Each observation has an
independent spectrum extracted along with an associated WARF, WRMF,
normalized background spectrum, and soft residual. Each independent
spectrum is then fit simultaneously with the same spectral model to
produce the final fit.
