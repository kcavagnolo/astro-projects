%%%%%%%%%%%%%%%%
% New commands %
%%%%%%%%%%%%%%%%

\newcommand{\mytitle}{Investigating Feedback and Relaxation in
  Clusters of Galaxies with the Chandra X-ray Observatory}
\newcommand{\tb}{thermal bremsstrahlung}
\newcommand{\ebandnuma}{192}
\newcommand{\ebandnumb}{166}
\newcommand{\haclnum}{\ensuremath{222}}
\newcommand{\kthr}{\ensuremath{K_{\mathrm{thresh}}}}
\newcommand{\fha}{\ensuremath{\kna = 13.9 \pm 4.9 \ent}}
\newcommand{\nfha}{\ensuremath{\kna = 130 \pm 55 \ent}}
\newcommand{\frad}{\ensuremath{\kna = 18.3 \pm 7.7 \ent}}
\newcommand{\nfrad}{\ensuremath{\kna = 112 \pm 45 \ent}}
\newcommand{\hifl}{\textit{HIFLUGCS}}
\newcommand{\entsuppobs}{317}
\newcommand{\entsuppnum}{239}
\newcommand{\expt}{9.86 Msec}
\newcommand{\dkna}{\ensuremath{(\kna^{\prime}-\kna)/\kna}}
\newcommand{\alphafs}{\ensuremath{\alpha = 1.21 \pm 0.39}}
\newcommand{\knafs}{\ensuremath{\kna = 72.9 \pm 33.7 \ent}}
\newcommand{\khunfs}{\ensuremath{\khun = 126 \pm 45 \ent}}
\newcommand{\alphaga}{\ensuremath{\alpha = 1.20 \pm 0.38}}
\newcommand{\knaga}{\ensuremath{\kna = 16.1 \pm  5.7 \ent}}
\newcommand{\khunga}{\ensuremath{\khun = 150 \pm 50 \ent}}
\newcommand{\alphagb}{\ensuremath{\alpha = 1.23 \pm 0.40}}
\newcommand{\knagb}{\ensuremath{\kna = 156 \pm 54 \ent}}
\newcommand{\khungb}{\ensuremath{\khun = 107 \pm 39 \ent}}
\newcommand{\centsrcnum}{\ensuremath{37}}
\newcommand{\alphacs}{\ensuremath{\alpha = 1.19 \pm 0.39}}
\newcommand{\knacs}{\ensuremath{\kna = 61.9 \pm 27.4 \ent}}
\newcommand{\khuncs}{\ensuremath{\khun = 132 \pm 45 \ent}}
\newcommand{\alphacsa}{\ensuremath{\alpha = 1.16 \pm 0.38}}
\newcommand{\knacsa}{\ensuremath{\kna = 15.6 \pm 5.2 \ent}}
\newcommand{\khuncsa}{\ensuremath{\khun = 146 \pm 48 \ent}}
\newcommand{\alphacsb}{\ensuremath{\alpha = 1.23 \pm 0.40}}
\newcommand{\knacsb}{\ensuremath{\kna = 148 \pm 49 \ent}}
\newcommand{\khuncsb}{\ensuremath{\khun = 118 \pm 42 \ent}}
\newcommand{\kmma}{\ensuremath{K_1 = 17.8 \pm 6.6 \ent}}
\newcommand{\kmmb}{\ensuremath{K_2 = 154 \pm 52 \ent}}
\newcommand{\kmmc}{\ensuremath{124}}
\newcommand{\kmmd}{\ensuremath{109}} 
\newcommand{\kmme}{\ensuremath{p = 1.16\times10^{-7}}}
\newcommand{\kmmf}{\ensuremath{K_1 = 15.0\pm 5.0 \ent}}
\newcommand{\kmmg}{\ensuremath{K_2 = 129 \pm 45 \ent}}
\newcommand{\kmmh}{\ensuremath{89}}
\newcommand{\kmmi}{\ensuremath{136}}
\newcommand{\kmmj}{\ensuremath{p = 1.90\times10^{-13}}}
\newcommand{\hifla}{\ensuremath{\alpha = 1.17 \pm 0.37}}
\newcommand{\hiflb}{\ensuremath{\kna = 11.4 \pm 4.2 \ent}}
\newcommand{\hiflc}{\ensuremath{\khun = 235 \pm 89 \ent}}
\newcommand{\hifld}{\ensuremath{\alpha = 1.19 \pm 0.39 \ent}}
\newcommand{\hifle}{\ensuremath{\kna = 151 \pm 53 \ent}}
\newcommand{\hiflf}{\ensuremath{\khun = 113 \pm 43 \ent}}
\newcommand{\hiflkmma}{\ensuremath{K_1 = 9.7 \pm 3.5 \ent}}
\newcommand{\hiflkmmb}{\ensuremath{K_2 = 131 \pm 46 \ent}}
\newcommand{\hiflkmmc}{\ensuremath{28}}
\newcommand{\hiflkmmd}{\ensuremath{31}}
\newcommand{\hiflkmme}{\ensuremath{p = 3.34\times10^{-3}}}
\newcommand{\hiflkmmf}{\ensuremath{K_1 = 10.5 \pm 3.4 \ent}}
\newcommand{\hiflkmmg}{\ensuremath{K_2 = 116 \pm 42 \ent}}
\newcommand{\hiflkmmh}{\ensuremath{21}}
\newcommand{\hiflkmmi}{\ensuremath{34}}
\newcommand{\hiflkmmj}{\ensuremath{p = 1.55\times10^{-5}}}
\newcommand{\tckmma}{\ensuremath{t_{c1} = 0.60 \pm 0.24 \Gyr}}
\newcommand{\tckmmb}{\ensuremath{t_{c2} = 6.23 \pm 2.19 \Gyr}}
\newcommand{\tckmmc}{\ensuremath{132}}
\newcommand{\tckmmd}{\ensuremath{101}}
\newcommand{\tckmme}{\ensuremath{p = 8.77\times10^{-7}}}


%%%%%%%%%%
% Header %
%%%%%%%%%%

\documentclass[final]{msuthesis}
%\documentclass[twoside]{msuthesis}
%\documentclass[draft]{msuthesis}
\usepackage{mathrsfs,common,setspace,longtable}
\allowlos{}
\bibliographystyle{apj}
\copyrightyear{2008}
\title{\mytitle}
\author{Kenneth W. Cavagnolo}
\subject{Astronomy \& Astrophysics}
\degree{Doctor of Philosophy}
\department{Department of Astronomy and Astrophysics}
\advisor{Dr. Megan Donahue}
\colorfigures
%\multivolume

%%%%%%%%%%%
% Symbols %
%%%%%%%%%%%

%%%%%%%%%%%%
% Abstract %
%%%%%%%%%%%%

\abstract{
Presented in this dissertation is an analysis of the X-ray emission
from the intracluster medium (ICM) in clusters of galaxies observed
with the \chandra\ X-ray Observatory. The cluster dynamic state is
investigated via ICM temperature inhomogeneity, and ICM entropy is
used to evaluate the thermodynamics of cluster cores.

If the hot ICM is nearly isothermal in the projected region of
interest, the X-ray temperature inferred from a broadband (0.7-7.0
keV) spectrum should be identical to the X-ray temperature inferred
from a hard-band (2.0-7.0 keV) spectrum. However, if unresolved cool
lumps of gas are contributing soft X-ray emission, the temperature of
a best-fit single-component thermal model will be cooler for the
broadband spectrum than for the hard-band spectrum. Using this
difference as a diagnostic, the ratio of best-fitting hard-band and
broadband temperatures may indicate the presence of cooler gas even
when the X-ray spectrum itself may not have sufficient signal-to-noise
ratio (S/N) to resolve multiple temperature components.

In Chapter \ref{ch:eband} we explore this band dependence of the
inferred X-ray temperature of the ICM for \ebandnuma\ well-observed
galaxy clusters selected from the {\it Chandra} X-ray Observatory's
Data Archive. We extract X-ray spectra from core-excised annular
regions for each cluster in the archival sample. We compare the X-ray
temperatures inferred from single-temperature fits when the energy
range of the fit is 0.7-7.0 keV (broad) and when the energy range is
2.0/(1+$z$)-7.0 keV (hard). We find that the hard-band temperature is
significantly higher, on average, than the broadband temperature. On
further exploration, we find this temperature ratio is enhanced
preferentially for clusters which are known merging systems. In
addition, cool-core clusters tend to have best-fit hard-band
temperatures that are in closer agreement with their best-fit
broadband temperatures.

ICM entropy is of great interest because it dictates ICM global
properties and records the thermal history of a cluster. Entropy is
therefore a useful quantity for studying the effects of feedback on
the cluster environment and investigating the breakdown of cluster
self-similarity. Radial entropy profiles of the ICM for a collection
of \entsuppnum\ clusters taken from the \chandra\ X-ray Observatory's
Data Archive are presented in Chapter \ref{ch:ent_supp}. We find that
most ICM entropy profiles are well-fit by a model which is a power-law
at large radii and approaches a constant value at small radii: $K(r) =
\kna + \khun (r/100 \kpc)^{\alpha}$, where \kna\ quantifies the
typical excess of core entropy above the best fitting power-law found
at larger radii. We also show that the \kna\ distributions of both the
full archival sample and the primary \hifl\ sample of
\citet{hiflugcs1} are bimodal with a distinct gap centered at $\kna
\approx 40 \ent$ and population peaks at $\kna \sim 15 \ent$ and $\kna
\sim 150 \ent$.

Utilizing the results of the the \chandra\ X-ray Observatory archival
study of intracluster entropy presented in Chapter \ref{ch:ent_supp},
we show in Chapter \ref{ch:harad} that \halpha\ and radio emission
from the brightest cluster galaxy are much more pronounced when the
cluster's core gas entropy is $\la 30 \ent$. The prevalence of
\halpha\ emission below this threshold indicates that it marks a
dichotomy between clusters that can harbor multiphase gas and star
formation in their cores and those that cannot. The fact that strong
central radio emission also appears below this boundary suggests that
feedback from an active galactic nucleus (AGN) turns on when the ICM
starts to condense, strengthening the case for AGN feedback as the
mechanism that limits star formation in the Universe's most luminous
galaxies.

}

%%%%%%%%%%%%%%
% Dedication %
%%%%%%%%%%%%%%

\dedication{Dedicated to my mother: Miss Lorna Lorraine Cox.}

%%%%%%%%%%%%%%%%%%%
% Acknowledgments %
%%%%%%%%%%%%%%%%%%%
\acknowledgments{
\classack My dissertation would not have been possible without the
multitude of grants from NASA and the Chandra X-ray Center. I also
thank the MSU College of Natural Science for awarding me the
Dissertation Completion Fellowship which helped fund my final year at
MSU.

My deepest thanks to Megan Donahue and Mark Voit for their guidance,
wisdom, patience, and without whom I would be in quantum computing. I
can only say, ``Thank you, Megan,'' for allowing me the time and space
to find my bearings after my mother's passing, words are insufficient
to express my gratitude. Many thanks to my friend Ming Sun who always
listened, always had time for a question, and was never wrong. Thanks
also to Jack Baldwin who nutured my painfully slow development as a
research assistant -- a more soothing voice there has never been. On
behalf of everyone that has never said so, ``We love you, Shawna
Prater. MSU Astronomy and Astrophysics could not function without
you.''  And of course, Debbie Simmons, without whom I would have been
dropped from all courses and locked out of the building.

Every time I feel the warmth of the Sun on my skin, it is an
invigorating experience. Bathed in photons millions of years old from
an inconceivably large nuclear power plant over a hundred million
kilometers away, I feel connected to the Universe in a way that is
surreal. To feel purposely cared for by the feckless Sun, whose
existence and operation are arguably devoid of purpose, is quite
profound. For that I say, ``Thank you, Sun!''}

%%%%%%%%%%%
% Preface %
%%%%%%%%%%%

\preface{Our universe is predominantly an untold story. Within a
  larger, nested framework of complex mechanisms, humans evolved with
  minimal impact on the systems which support and nurture our
  existance. Yet, during the short epoch of global industrialization,
  we have compromised the effectiveness and function of the systems
  which formed the biodiversity which makes our planet such a
  wonderful place. As an acknowledgment of our species' appreciation
  for the Earth, and as a show of our understanding that humanity's
  presence on Earth is fleeting, let us strive to utilize the pursuit
  of knowledge, through application of reason and logic, such that our
  actions benefit ``all the children, of all species, for all of
  time'' \citep{2002c2c..book.....O}. Let us all exert effort such
  that the Earth and the Universe will be enriched by humanity, and
  that our actions -- local, global, and possibly interplanetary --
  will leave the places we inhabit and visit nourished from our
  presence.}

%%%%%%%%%%%%
% Initiate %
%%%%%%%%%%%%

\begin{document}
%\extracopies
\requirements

%%%%%%%%%%%%%%%%%%%%%%
\chapter{Introduction}
\label{ch:intro}
%%%%%%%%%%%%%%%%%%%%%%

%%%%%%%%%%%%%%%%%%%%%%%%%%%%%%
\section{Clusters of Galaxies}
\label{sec:cofg}
%%%%%%%%%%%%%%%%%%%%%%%%%%%%%%

Of the luminous matter in the Universe, stars and galaxies are often
the most familiar to a sky gazer. Aside from the Moon and the
occasional bright planet, stars are the most abundantly obvious
patrons of the night sky. Viewed from a sufficiently dark location,
the stars form a band of light interspersed with dust and gaseous
clouds which define the Milky Way, our home galaxy. The Milky Way is
only one of more than 30 galaxies in a gravitationally bound group of
galaxies, named the Local Group, which includes the well-known, nearby
galaxy Andromeda. But in cosmological terms, the Local Group is very
small in comparison to immense structures containing thousands of
galaxies. In a turn of wit, these structures are appropriately named
clusters of galaxies, and are the focus of this dissertation.

\invisiblesymbol{\mathrm{Mpc}}{Megaparsec: A unit of length representing
one million parsecs. The parsec (pc) is a historical unit for
measuring parallax and equals $3.0857\times10^{13}$
km.}
\invisiblesymbol{z}{Dimensionless redshift: As is common in most of
astronomy, I adopt the definition of redshift using a dimensionless
ratio of wavelengths, $z = (\lambda_{\mathrm{observed}} /
\lambda_{\mathrm{rest}})-1$, where the wavelength shift occurs because
of cosmic expansion.}
\invisiblesymbol{H_0}{Hubble constant: The current ratio of recessional
velocity arising from expansion of the Universe to an object's
distance from the observer, $v = \Hn D$. \Hn\ is assumed here to be
$\sim70 \hub$. Inverted, the Hubble constant yields the present
age of the Universe, $\Hn^{-1} \approx 13.7$ billion years. $H(z)$
denotes the Hubble constant at a particular redshift, $z$.}
\invisiblesymbol{\rho_c}{Critical density: The density necessary for a
universe which has spatially flat geometry and in which the expansion
rate of spacetime balances gravitational attraction and prevents
recollapse. In terms of relevant quantities $\rho_c=3H(z)^2/8\pi G$,
with units \gpcc.}
\invisiblesymbol{\Omega_{\Lambda}}{Cosmological constant energy density
of the Universe: The ratio of energy density due to a cosmological
constant to the critical density. \OL\ is assumed here to be
$\sim0.7$.}
\invisiblesymbol{\Omega_M}{Matter density of the Universe: The ratio of
total matter density to the critical density. \OM\ is currently
measured to be $\sim0.3$.}

Galaxy clusters are the most massive gravitationally bound structures
to have yet formed in the Universe. As where galaxy groups have
roughly 10-50 galaxies, galaxy clusters have hundreds to thousands of
galaxies. When viewed through a telescope, a galaxy cluster appears as
a tight distribution of mostly elliptical and S0 spiral galaxies
within a radius of $\sim1-5$ Mpc\footnote{Throughout this
dissertation, a flat \LCDM\ cosmology of $\Hn = 70 \hub$, $\OL =
0.7$, and $\OM = 0.3$ is assumed. These values are taken from
\citet{wmap}.} of each other. Rich galaxy clusters are truly
spectacular objects, as can be seen in Figure \ref{fig:a1689} which
shows the \hubble\ Space Telescope's close-up of the strong lensing
cluster Abell 1689.

\begin{figure}[htp]
  \begin{center}
    \includegraphics*[width=\textwidth, trim=0mm 0mm 0mm 0mm, clip]{a1689.eps}
    \caption[\hubble\ image of Abell 1689]{Optical image
      of the galaxy cluster Abell 1689 as observed with the ACS instrument
      on-board the \hubble\ Space Telescope. The fuzzy yellowish spheres
      are giant elliptical (gE) galaxies in the cluster, with the gE
      nearest the center of the image being the brightest cluster galaxy
      -- ostensibly, the cluster ``center''. Image taken from NASA's
      Hubblesite.org. Image Credits: NASA, N. Benitez (JHU), T. Broadhurst
      (The Hebrew University), H. Ford (JHU), M. Clampin(STScI), G. Hartig
      (STScI), G. Illingworth (UCO/Lick Observatory), the ACS Science Team
      and ESA.}
    \label{fig:a1689}
  \end{center}
\end{figure}

Galaxy clusters are deceptively named. As with most objects in the
Universe, one of the most revealing characteristics of an object is
its mass, and the mass of clusters of galaxies is not dominated by
galaxies. A cluster of galaxies mass is dominated ($\ga 85\%$) by dark
matter with most ($\ga 80\%$) of the baryonic mass\footnote{Baryonic
is a convenient term used to describe ordinary matter like atoms or
molecules, while non-baryonic matter is more exotic like free
electrons or dark matter particles.} in the form of a hot ($kT \approx
2-15$ keV; 10-100 million degrees K), luminous ($10^{43-46} \ergps$),
diffuse ($10^{-1}-10^{-4} \cm^{-3}$) intracluster medium (ICM) which
is co-spatial with the galaxies but dwarfs them in mass
\citep{1984Natur.311..517B, 1990ApJ...356...32D}. For comparison, the
ICM in the core region of a galaxy cluster is, on average, $10^{20}$
times less dense than typical Earth air, $10^5$ times denser than the
mean cosmic density, more than 2000 times hotter than the surface of
the Sun, and shines as bright as $10^{35}$ 100 watt light bulbs.

Because of the ICM's extreme temperature, the gas is mostly ionized,
making it a plasma. For the temperature range of clusters, the ICM is
most luminous at X-ray wavelengths of the electromagnetic
spectrum. This makes observing galaxy clusters with X-ray telescopes,
like NASA's \chandra\ X-ray Observatory, a natural choice. Clusters
have masses ranging over $10^{14-15}$
\definesymbol{\mathrm{M}_{\odot}}{Mass of the Sun: One solar mass
equals $1.9891\times10^{30} \kg$} with velocity dispersions of
$500-1500 \kmps$. The ICM has also been enriched with
metals\footnote{It is common practice in astronomy to classify
``metals'' as any element with more than two protons.} to an average
value of $\sim 0.3$ solar abundance. Shown in Figure \ref{fig:bullet}
is an optical, X-ray, and gravitational lensing composite image of the
galaxy cluster 1E0657-56. This cluster is undergoing an especially
spectacular and rare merger in the plane of the sky which allows for
the separate dominant components of a cluster -- dark matter, the ICM,
and galaxies -- to be ``seen'' distinctly.

\begin{figure}[htp]
  \begin{center}
    \includegraphics*[width=\textwidth, trim=0mm 0mm 0mm 0mm, clip]{bullet}
    \caption[Composite image of the Bullet Cluster]{The
      galaxy cluster 1E0657-56, a.k.a. the Bullet Cluster. All of the
      primary components of a galaxy cluster can be seen in this image:
      the X-ray ICM (pink), dark matter (blue), and galaxies. The
      brilliant white object with diffraction spikes is a star. This
      cluster has become very famous as the merger dynamics provide direct
      evidence for the existence of dark matter
      \citep{2006ApJ...648L.109C}. Image taken from NASA Press Release
      06-297. Image credits: NASA/CXC/CfA/\citet{2002ApJ...567L..27M}
      (X-ray); NASA/STScI/Magellan/U.Arizona/\citet{2006ApJ...648L.109C}
      (Optical); NASA/STScI/ESO
      WFI/Magellan/U.Arizona/\citet{2006ApJ...648L.109C} (Lensing).}
    \label{fig:bullet}
  \end{center}
\end{figure}

As knowing the characteristics of galaxy clusters is a small part of
the discovery process, we must also wonder, why study clusters of
galaxies? Galaxy clusters have two very important roles in the current
research paradigm:
\begin{enumerate}
\item Galaxy clusters represent a unique source of information about
  the Universe's underlying cosmological parameters, including the
  nature of dark matter and the dark energy equation of
  state. Large-scale structure growth is exponentially sensitive to
  some of these parameters, and by counting the number of clusters
  found in a comoving volume of space, specifically above a given mass
  threshold, clusters may be very useful in cosmological studies
  \citep{voitreview}.
\item The cluster gravitational potential well is deep enough to
  retain all the matter which has fallen in over the age of the
  Universe. This slowly evolving ``sealed box'' therefore contains a
  comprehensive history of all the physical processes involved in
  galaxy formation and evolution, such as: stellar evolution,
  supernovae feedback, black hole activity in the form of active
  galactic nuclei, galaxy mergers, ram pressure stripping of
  in-falling galaxies and groups, {\it{et cetera}}. The time required
  for the ICM in the outskirts of a cluster to radiate away its
  thermal energy is longer than the age of the Universe, hence the ICM
  acts as a record-keeper of all the aforementioned activity.
  Therefore, by studying the ICM's physical properties, the thermal
  history of the cluster can be partially recovered and utilized in
  developing a better understanding of cluster formation and
  evolution.
\end{enumerate}
In this dissertation I touch upon both these points by studying the
emergent X-ray emission of the ICM as observed with the \chandra\
X-ray Observatory.

While clusters have their specific uses in particular areas of
astrophysics research, they also are interesting objects in their own
right. A rich suite of physics is brought to bear when studying galaxy
clusters. A full-blown, theoretical construction of a galaxy cluster
requires, to name just a few: gravitation, fluid mechanics,
thermodynamics, hydrodynamics, magnetohydrodynamics, and
high-energy/particle/nuclear physics. Multiwavelength observations of
galaxy clusters provide excellent datasets for testing the theoretical
predictions from other areas of physics, and clusters are also a
unique laboratory for empirically establishing how different areas of
physics interconnect. Just this aspect of clusters puts them in a
special place among the objects in our Universe worth intense,
time-consuming, (and sometimes expensive) scrutiny. At a minimum,
galaxy clusters are most definitely worthy of being the focus of a
humble dissertation from a fledgling astrophysicist.

As this is a dissertation focused around observational work, in
Section \S\ref{sec:icm} I provide a brief primer on the X-ray
observable properties of clusters which are important to understanding
this dissertation. Section \S\ref{sec:entintro} provides discussion of
gas entropy, a physical property of the ICM which may be unfamiliar to
many readers and is utilized heavily in Chapters \ref{ch:ent_supp} and
\ref{ch:harad}. In Section \S\ref{sec:incomplete}, I more thoroughly
discuss reasons for studying clusters of galaxies which are specific
to this dissertation. Presented in Section \S\ref{sec:ssbreak} is a
discussion of why clusters of different masses are not simply scaled
versions of one another, and in Section \S\ref{sec:cfprob} the
unresolved ``cooling flow problem'' is briefly summarized. The current
chapter concludes with a brief description of the \chandra\ X-ray
Observatory (CXO) and its instruments in Section
\S\ref{sec:chandra}. \chandra\ is the space-based telescope with which
all of the data presented in this dissertation was collected.

%%%%%%%%%%%%%%%%%%%%%%%%%%%%%%%%%
\section{The Intracluster Medium}
\label{sec:icm}
%%%%%%%%%%%%%%%%%%%%%%%%%%%%%%%%%

In Section \S\ref{sec:cofg}, the ICM was presented as a mostly
ionized, hot, diffuse plasma which dominates the baryonic mass content
of clusters. But where did it come from and what is the composition of
this pervasive ICM? What are the mechanisms that result in the ICM's
X-ray luminescence? How do observations of the ICM get converted into
physical properties of a cluster? In this section I briefly cover the
answers to these questions in order to give the reader a better
understanding of the ICM.

Galaxy clusters are built-up during the process of hierarchical merger
of dark matter halos and the baryons gravitationally coupled to those
halos \citep{white&rees}. Owing to the inefficiency of galaxy
formation and the processes of galactic mass ejection and ram pressure
stripping, many of the baryons in these dark matter halos are in the
form of diffuse gas and not locked up in galaxies. During the merger
of dark matter halos, gravitational potential energy is converted to
thermal energy and the diffuse gas is heated to the virial temperature
of the cluster potential through processes like adiabatic compression
and accretion shocks. The cluster virial temperature is calculated by
equating the average kinetic energy of a gas particle to its thermal
energy,
\begin{eqnarray}
\frac{1}{2} \mu m \langle\sigma^2\rangle &=& \frac{3}{2}k \tvir\\
\tvir &=& \frac{\mu m \langle\sigma^2\rangle}{3k}
\end{eqnarray}
where $\mu$ is the mean molecular weight, $k$ is the Boltzmann
constant, \tvir\ is the virial temperature, $m$ is the mass of a test
particle, and $\langle \sigma \rangle$ is the average velocity of the
test particle. In this equation, $\langle \sigma \rangle$ can be
replaced with the line-of-sight galaxy velocity dispersion (a cluster
observable) because all objects within the cluster potential (stars,
galaxies, protons, \etc) are subject to the same dynamics and hence
have comparable thermal and kinetic energies.

Galaxy clusters are the most massive objects presently in the
Universe. The enormous mass means deep gravitational potential wells
and hence very high virial temperatures. Most cluster virial
temperatures are in the range $k\tvir = 1-15 \keV$. At these energy
scales, gases are collisionally ionized plasmas and will emit X-rays
via \tb\ (discussed in Section \S\ref{sec:xray}). The ICM is not a
pure ionized hydrogen gas, as a result, atomic line emission from
heavy elements with bound electrons will also occur. The ICM is also
optically thin at X-ray wavelengths, \eg\ the ICM optical depth to
X-rays is much smaller than unity, $\tau_{\lambda} \ll 1$, and hence
the X-rays emitted from clusters stream freely into the Universe. In
the next section I briefly cover the processes which give rise to ICM
X-ray emission and the observables which result. For a magnificently
detailed treatise of this topic, see \citet{sarazinbook}\footnote{Also
available at
http://nedwww.ipac.caltech.edu/level5/March02/Sarazin/TOC.html} and
references therein.

%%%%%%%%%%%%%%%%%%%%%%%%%%%
\subsection{X-ray Emission}
\label{sec:xray}
%%%%%%%%%%%%%%%%%%%%%%%%%%%

Detailed study of clusters proceeds mainly through spatial and
spectral analysis of the ICM. By directly measuring the X-ray emission
of the ICM, quantities such as temperature, density, and luminosity
per unit volume can be inferred. Having this knowledge about the ICM
provides an observational tool for indirectly measuring ICM dynamics,
composition, and mass. In this way a complete picture of a cluster can
be built up and other processes, such as brightest cluster galaxy
(BCG) star formation, AGN feedback activity, or using ICM temperature
inhomogeneity to probe cluster dynamic state, can be investigated. In
this section, I briefly cover how X-ray emission is produced in the
ICM and how basic physical properties are then measured.

The main mode of interaction in a fully ionized plasma is the
scattering of free electrons off heavy ions. During this process,
charged particles are accelerated and thus emit radiation. The
mechanism is known as `free-free' emission (ff), or by the
tongue-twisting \tb\ (German for ``braking radiation''). It is also
called bremsstrahlung cooling since the X-ray emission carries away
large amounts of energy. The timescale for protons, ions, and
electrons to reach equipartition is typically shorter than the age of
a cluster \citep{2003PhPl...10.1992S}, thus the gas particles
populating the emitting plasma can be approximated as being at a
uniform temperature with a Maxwell-Boltzmann velocity distribution,
\begin{equation}
f(\vec{v}) = 4 \pi \left(\frac{m}{2 \pi k T}\right)^{3/2} \vec{v}^2
\exp \left[\frac{-m\vec{v}^2}{2k T}\right]
\end{equation}
where $m$ is mass, $T$ is temperature, $k$ is the Boltzmann constant,
and velocity, $\vec{v}$, is defined as $\vec{v} = \sqrt{v_x^2 + v_y^2
  + v_z^2}$. The power emitted per cubic centimeter per second (erg
$\ps \pcc$) from this plasma can be written in the compact form
\begin{equation}
\label{eqn:ff}
\epsilon^{ff} \equiv 1.4\times10^{-27} T^{1/2} n_{e} n_{i} Z^{2} \bar{g}_B
\end{equation}
where $1.4\times10^{-27}$ is in cgs and is the condensed form of the
physical constants and geometric factors associated with integrating
over the power per unit area per unit frequency, $n_e$ and $n_i$ are
the electron and ion densities, $Z$ is the number of protons of the
bending charge, $\bar{g}_B$ is the frequency averaged Gaunt factor (of
order unity), and $T$ is the global temperature determined from the
spectral cut-off frequency \citep{rybicki}. Above the cut-off
frequency, $\nu_c=kT/\hbar$, few photons are created because the
energy supplied by charge acceleration is less than the minimum energy
required for creation of a photon. Worth noting is that free-free
emission is a two-body process and hence the emission goes as the gas
density squared while having a weak dependence on the thermal energy,
$\epsilon \propto \rho^2 T^{1/2}$ for $T \ga 10^7$ K when the gas has
solar abundances.

Superimposed on the thermal emission of the plasma are emission lines
of heavy element contaminants such as C, Fe, Mg, N, Ne, O, S, and
Si. The widths and relative strengths of these spectral lines are used
to constrain the metallicity of the ICM, which is typically quantified
using units relative to solar abundance,
\definesymbol{Z_{\odot}}{Metal abundance of the Sun: Individual
elemental abundances can be found in \citet{ag89}.}. On average, the
ICM has a metallicity of $\sim 0.3~\Zsol$, which is mostly stellar
detritus \citep{icmmetal1, icmmetal2, icmmetal3}. In collisionally
ionized plasmas with temperatures and metallicities comparable to the
ICM, the dominant ion species is that of the `closed-shell'
helium-like ground state (K and L-shells) \citep{cfreview}. The K and
L shell transitions are extremely sensitive to temperature and
electron densities, therefore providing an excellent diagnostic for
constraining both of these quantities. The strongest K-shell
transition of the ICM can be seen from iron at $kT \sim 6.7$ keV. If
signal-to-noise of the spectrum is of high enough quality, measuring a
shift in the energy of this spectral line can be used to confirm or
deduce the approximate redshift, and hence distance, of a cluster. The
rich series of iron L-shell transitions occur between $0.2 \la T \la
2.0$ keV and are the best diagnostic for measuring metallicity. For
the present generation of X-ray instruments, the L-shell lines are
seen as a blend with a peak around $\sim 1$ keV.

Shown in Figs. \ref{fig:brem} and \ref{fig:brem2} are the unredshifted
synthetic spectral models generated with \xspec\ \citep{xspec} of a 2
keV and 8 keV gas. Both spectral models have a component added to
mimic absorption by gas in the Milky Way, which is seen as attenuation
of flux at $E \la 0.4$ keV. For both spectral models the metal
abundance is $0.3~\Zsol$. These two spectral models differ by only a
factor of four in temperature but note the extreme sensitivity of both
the \tb\ exponential cut-off and emission line strengths to
temperature.

\begin{figure}[htp]
  \begin{center}
    \begin{minipage}[htp]{0.8\linewidth}
      \includegraphics*[width=0.7\textwidth,trim=10mm 0mm 0mm 10mm,angle=270,clip]{brem}
      \caption[Synthetic spectral model of $kT_X =2.0$ keV gas.]{Synthetic
        absorbed thermal spectral model of a \definesymbol{N_H}{Neutral
          hydrogen column density: The Galaxy is rich with metals such as C,
          N, O, S, and Si which absorb incoming extragalactic soft X-ray
          radiation. The density of neutral hydrogen is assumed to be a
          surrogate for the density of metals. Photoelectric absorption models
          are used to quantify the attenuation of soft X-rays, and typically
          take as input the column density (\pcmsq) of neutral hydrogen in a
          particular direction. \nhi\ is related to the number density, $n_H$
          (\pcc), along the line of sight, $dl$, as $\nhi = \int n_H dl$.}$=
        10^{20} \pcmsq$, $kT_X=2.0$ keV, $Z/\Zsol=0.3$, and zero redshift
        gas. Notice that the strength of the iron L-shell emission lines is
        much greater than the iron K-shell lines for this model.}
      \label{fig:brem}
    \end{minipage}
    \begin{minipage}[htp]{0.8\linewidth}
      \includegraphics*[width=0.7\textwidth,trim=10mm 0mm 0mm 10mm,angle=270,clip]{brem2}
      \caption[Synthetic spectral model of $kT_X=8.0$ keV gas.]{Same as
        Fig. \ref{fig:brem} except for a $kT_X=8.0$ keV gas. Notice that for
        this spectral model the iron L-shell emission lines are much weaker
        and the iron K-shell lines are much stronger than in the $kT_X=2.0$
        keV model. Also note that the exponential cut-off of this model
        occurs at a higher energy ($E > 10$ keV) than in the model shown in
        Figure \ref{fig:brem}.}
      \label{fig:brem2}
    \end{minipage}
  \end{center}
\end{figure}

Equation \ref{eqn:ff} says that observations of ICM X-ray emission
will yield two quantities: temperature and density. The gas density
can be inferred from the {\it{emission integral}},
\begin{equation}
\label{eqn:ei}
EI = \int n_e n_p~dV
\end{equation}
where $n_e$ is the electron density, $n_p$ is the density of
hydrogen-like ions, and $dV$ is the gas volume within a differential
element. The emission integral is essentially the sum of the square of
gas density for all the gas parcels in a defined region. Thus, the gas
density within a projected volume can be obtained from the spectral
analysis, but it can also be obtained from spatial analysis of the
cluster emission, for example from cluster surface brightness.

The number of photons detected per unit area (projected on the plane
of the sky) per second is given the name {\it{surface brightness}}.
Assuming spherical symmetry, 2-dimensional surface brightness can be
converted to 3-dimensional emission density. By dividing a cluster
observation into concentric annuli originating from the cluster center
and subtracting off cluster emission at larger radii from emission at
smaller radii, the amount of emission from a spherical shell can be
reconstructed from the emission in an annular ring. For the spherical
shell defined by radii $r_i$ and $r_{i+1}$, \citet{kriss83} shows the
relation between the emission density, $C_{i,i+1}$, to the observed
surface brightness, $S_{m,m+1}$, of the ring with radii $r_m$ and
$r_{m+1}$, is
\begin{equation}
\label{eqn:depro}
S_{m,m+1} = \frac{b}{A_{m,m+1}}\sum_{i-1}^m C_{i,i+1}~[(V_{i,m+1}-V_{i+1,m+1})-(V_{i,m}-V_{i+1,m})].
\end{equation}
where $b$ is the solid angle subtended on the sky by the object,
$A_{m,m+1}$ is the area of the ring, and the $V$ terms are the volumes
of various shells. This method of reconstructing the cluster emission
is called {\it{deprojection}}. While assuming spherical symmetry is
clearly imperfect, it is not baseless. The purpose of such an
assumption is to attain angular averages of the volume density at
various radii from an azimuthally averaged surface density. Systematic
uncertainties associated with deprojection are discussed in Section
\S\ref{sec:entsuppdene}.

In this dissertation the spectral model \mekal\ \citep{mekal1, mekal2,
  mekal3} is used for all of the spectral analysis. The \mekal\ model
normalization, $\eta$, is defined as
\begin{equation}
\label{eqn:norm}
\eta = \frac{10^{-14}}{4\pi D_A^2 (1+z)^2}~EI
\end{equation}
where $z$ is cluster redshift, $D_A$ is the angular diameter distance,
and $EI$ is the emission integral from eqn. \ref{eqn:ei}. Recognizing
that the count rate, $f(r)$, per volume is equivalent to the emission
density, $C_{i,i+1} = f(r)/\int dV$, where $dV$ can be a shell
(deprojected) or the sheath of a round column seen edge-on
(projected), combining eqns. \ref{eqn:depro} and \ref{eqn:norm} yields
an expression for the electron gas density which is a function of
direct observables,
\begin{equation}
\label{eqn:dens}
\nelec(r) = \sqrt{\frac{1.2 C(r) \eta(r) 4 \pi [D_A(1+z)]^2}{f(r) 10^{-14}}}
\end{equation}
where the factor of 1.2 comes from the ionization ratio \nelec=1.2\np,
$C(r)$ is the radial emission density derived from
eqn. \ref{eqn:depro}, $\eta$ is the spectral normalization from
eqn. \ref{eqn:norm}, $D_A$ is the angular diameter distance, $z$ is
the cluster redshift, and $f(r)$ is the spectroscopic count rate.

Simply by measuring surface brightness and analyzing spectra, the
cluster temperature, metallicity, and density can be inferred. These
quantities can then be used to derive pressure, $P = nkT$, where $n
\approx 2\nelec$. The total gas mass can be inferred using gas density
as $M_{gas} = \int (4/3) \pi r^3 \nelec dr$. By further assuming the
ICM is in hydrostatic equilibrium, the total cluster mass within
radius $r$ is
\begin{equation}
M(r) = \frac{kT(r)r}{\mu m_H
G}\left[\frac{d(log~\nelec(r))}{d(log~r)}+\frac{d(log~T(r))}{d(log~r)}\right]
\end{equation}
where all variables have their typical definitions. The rate at which
the ICM is cooling can also be expressed in simple terms of density
and temperature. Given a cooling function,
\definesymbol{\Lambda}{Cooling function: A function describing plasma
  emissivity for a given temperature and metal composition, and
  typically given in units of $\erg \cc \ps$.}, which is sensitive to
temperature and metal abundance (for the ICM $\Lambda(T,Z) \sim
10^{-23} \erg \cc \ps$), the cooling rate is given by $r_{cool} = n^2
\Lambda(T,Z)$. For some volume, $V$, the cooling time is then simply
the time required for a gas parcel to radiate away its thermal energy,
\begin{eqnarray}
t_{cool} V r_{cool} &=& \gamma NkT\\
t_{cool} &=& \frac{\gamma nkT}{n^2\Lambda(T,Z)}
\label{eqn:tcool}
\end{eqnarray}
where $\gamma$ is a constant specific to the type of cooling process
(\eg\ $3/2$ for isochoric and $5/2$ for isobaric). The cooling time of
the ICM can be anywhere between $10^{7-10}$ yrs. Cooling time is a
very important descriptor of the ICM because processes such as the
formation of stars and line-emitting nebulae are sensitive to cooling
time.

By ``simply'' pointing a high-resolution X-ray telescope, like
\chandra, at a cluster and exposing long enough to attain good
signal-to-noise, it is possible to derive a roster of fundamental
cluster properties: temperature, density, pressure, mass, cooling
time, and even entropy. Entropy is a very interesting quantity which
can be calculated using gas temperature and density and is most likely
fundamentally connected to processes like AGN feedback and star
formation (discussed in Chapters \ref{ch:ent_supp} and
\ref{ch:harad}). In the following section I discuss how gas entropy is
derived, why it is a useful quantity for understanding clusters, and
how it will be utilized later in this dissertation.

%%%%%%%%%%%%%%%%%%%%
\subsection{Entropy}
\label{sec:entintro}
%%%%%%%%%%%%%%%%%%%%

Entropy has both a macroscopic definition (the measure of available
energy) and microscopic definition (the measure of randomness), with
each being useful in many areas of science. Study of the ICM is a
macro-scale endeavor, so the definition of entropy pertinent to
discussion of the ICM is as a measure of the thermodynamic processes
involving heat transfer. But the conventional macroscopic definition
of entropy, $dS=dQ/T$, is not the quantity which is most useful in the
context of studying astrophysical objects. Thus we must resort to a
simpler, measurable surrogate for entropy, like the adiabat. The
adiabatic equation of state for an ideal monatomic gas is
$P=K\rho^{\gamma}$ where $K$ is the adiabatic constant and $\gamma$ is
the ratio of specific heat capacities and has the value of $5/3$ for a
monatomic gas. Setting $P=\rho kT/\mu m_H$ and solving for $K$ one
finds
\begin{equation}
\label{eqn:adi}
K = \frac{kT}{\mu m_H \rho^{2/3}}.
\end{equation}
where $\mu$ is the mean molecular weight of the gas and $m_H$ is the
mass of the Hydrogen atom. The true thermodynamic specific entropy
using this formulation is $s = k \ln K^{3/2}+s_0$, so neglecting
constants and scaling $K$ shall be called entropy in this
dissertation. A further simplification can be made to recast
eqn. \ref{eqn:adi} using the observables electron density,
\nelec, and X-ray temperature, $T_X$ (in keV):
\begin{equation}
K = \frac{T_X}{\nelec^{2/3}}.
\label{eqn:k}
\end{equation}
Equation \ref{eqn:k} is the definition of entropy used throughout this
dissertation. With a simple functional form, ``entropy'' can be
derived directly from X-ray observations. But why study the ICM in
terms of entropy?

ICM temperature and density alone primarily reflect the shape and
depth of the cluster dark matter potential \citep{voitbryan}. But it
is the specific entropy of a gas parcel, $s = c_v \ln
(T/\rho^{\gamma-1})$, which governs the density at a given
pressure. In addition, the ICM is convectively stable when $ds/dr \ge
0$, thus, without perturbation, the ICM will convect until the lowest
entropy gas is near the core and high entropy gas has buoyantly risen
to large radii. ICM entropy can also only be changed by addition or
subtraction of heat, thus the entropy of the ICM reflects most of the
cluster thermal history. Therefore, properties of the ICM can be
viewed as a manifestation of the dark matter potential and cluster
thermal history - which is encoded in the entropy structure. It is for
these reasons that the study of ICM entropy has been the focus of both
theoretical and observational study \citep{1996ApJ...473..692D,
1997MNRAS.288..355B, 1999Natur.397..135P, davies00, tozzi01,
voitbryan, ponman03, piffaretti05, pratt06, radioquiet,
d06, morandi07, 2008MNRAS.386.1309M}.

Hierarchical accretion of the ICM should produce an entropy
distribution which is a power-law across most radii with the only
departure occurring at radii smaller than 10\% of the virial radius
\citep{vkb05}. Hence deviations away from a power-law entropy
profile are indicative of prior heating and cooling and can be
exploited to reveal the nature of, for example, AGN feedback. The
implication of the intimate connection between entropy and
non-gravitational processes being that {\em{both}} the breaking of
self-similarity and the cooling flow problem (both discussed in
Section \S\ref{sec:incomplete}) can be studied with ICM entropy.

In Chapter \ref{ch:ent_supp} and Chapter \ref{ch:harad} I present the
results of an exhaustive study of galaxy cluster entropy profiles for
a sample of over 230 galaxy clusters taken from the \chandra\ Data
Archive. Analysis of these profiles has yielded important results
which can be used to constrain models of cluster feedback, understand
truncation of the high-mass end of the galaxy luminosity function, and
what effect these processes have on the global properties of
clusters. The size and scope of the entropy profile library presented
in this dissertation is unprecedented in the current scientific
literature, and we hope our library, while having provided immediate
results, will have a long-lasting and broad utility for the research
community. To this end, we have made all data and results available to
the public via a project web
site\footnote{http://www.pa.msu.edu/astro/MC2/accept/}.

%%%%%%%%%%%%%%%%%%%%%%%%%%%%%%%%%%%%%%%%%%%%
\section{The Incomplete Picture of Clusters}
\label{sec:incomplete}
%%%%%%%%%%%%%%%%%%%%%%%%%%%%%%%%%%%%%%%%%%%%

The literature on galaxy clusters is extensive. There has been a great
deal already written about clusters (with much more eloquence), and I
strongly suggest reading \citet{1984PhST....7..157M, kaiser86,
1990ApJ...363..349E, kaiser91, sarazinbook, fabiancfreview,
voitreview, cfreview, 2007PhR...443....1M, mcnamrev}, and references
therein for a comprehensive review of the concepts and topics to be
covered in this dissertation. The discussion of Sections
\S\ref{sec:ssbreak} and \S\ref{sec:cfprob} focuses on a few unresolved
mysteries involving galaxy clusters: the breaking of self-similarity
in relation to using clusters in cosmological studies and the cooling
flow problem as it relates to galaxy formation.

%%%%%%%%%%%%%%%%%%%%%%%%%%%%%%%%%%%%%%%%
\subsection{Breaking of Self-Similarity}
\label{sec:ssbreak}
%%%%%%%%%%%%%%%%%%%%%%%%%%%%%%%%%%%%%%%%

We now know the evolution of, and structure within, the Universe are a
direct result of the influence from dark energy and dark matter. An
all pervading repulsive dark energy has been posited to be responsible
for the accelerating expansion of the Universe
\citep{1998AJ....116.1009R, 1999ApJ...517..565P,
2007ApJ...659...98R}. Dark matter is an unknown form of matter which
interacts with itself and ordinary matter (both baryonic and
non-baryonic) through gravitational forces. Up until the last $\sim 5$
billion years \citep{1998AJ....116.1009R, 1999ApJ...517..565P,
2007ApJ...659...98R}, the influence of dark matter on the Universe has
been greater than that of dark energy. The early dominance of dark
matter is evident from the existence of large-scale structure like
galaxy clusters.

An end result of the gravitational attraction between amalgamations of
dark matter particles, called dark matter halos, is the merger of
small halos into ever larger halos. Since dark matter far outweighs
baryonic matter in the Universe, the baryons are coupled to the dark
matter halos via gravity, and hence are dragged along during the halo
merger process. Like raindrops falling in a pond that drains into a
river which flows into the ocean, the process of smaller units merging
to create larger units is found {\it{ad infinitum}} in the Universe
and is given the name hierarchical structure formation. A useful
visualization of the hierarchical structure formation process is shown
in Fig. \ref{fig:bigbang}. Hierarchical formation begins with small
objects like the first stars, continues on to galaxies, and culminates
in the largest present objects, clusters of galaxies.

\begin{figure}[htp]
  \begin{center}
    \includegraphics*[height=0.7\textheight, trim=0mm 0mm 0mm 0mm, clip]{bigbang2}
    \caption[Figures illustrating of large scale structure
      formation.]{{\it{Top panel:}} Illustration of hierarchical structure
      formation. {\it{Bottom panel:}} Snapshots from the simulation of a
      galaxy cluster forming. Each pane is 10 Mpc on a side. Color coding
      represents gas density along the line of sight (deep red is highest,
      dark blue is lowest). Each snapshot is numbered on the illustration
      at the approximate epoch each stage of cluster collapse
      occurs. Notice that, at first (1-2), very small objects like the
      first stars and protogalaxies collapse and then these smaller
      objects slowly merge into much larger halos (3-5). The hierarchical
      merging process ultimately results in a massive galaxy cluster (6)
      which continues to grow as sub-clusters near the box edge creep
      toward the cluster main body. Illustration taken from NASA/WMAP
      Science Team and modified by author. Simulation snapshots taken from
      images distributed to the public by the Virgo Consortium on behalf
      of Dr. Craig Booth: http://www.virgo.dur.ac.uk}
    \label{fig:bigbang}
  \end{center}
\end{figure}

In an oversimplified summary, one can say dark energy is attempting to
push space apart while dark matter is attempting to pull matter
together within that space. Were the balance and evolution of dark
energy and dark matter weighted heavily toward one or the other it
becomes clear that the amount of structure and its distribution will
be different. Thus, the nature of dark matter and dark energy
ultimately influence the number of clusters found at any given
redshift \citep[\eg][]{1993MNRAS.262.1023W} and hence cluster number
counts are immensely powerful in determining cosmological parameters
\citep[\eg][]{2001ApJ...561...13B}.

\invisiblesymbol{D_C}{Comoving distance: The distance which would be
  measured between two objects today if those two points were moving
  away from each other with the expansion of the Universe.}
\invisiblesymbol{D_A}{Angular diameter distance: The ratio of an
  object's true transverse size to its angular size. For a nearly flat
  universe, $D_A$ is a good approximation of the comoving distance,
  $D_A \approx D_C/(1+z)$.}
\invisiblesymbol{\Omega_S}{Solid angle: For a sphere of a given
  radius, for example the distance to an object $D_C$, the area of
  that object, $A$, on the sphere subtends an angle equal to $\Omega_S
  = A/D_C^2$. This is the solid angle.}
\invisiblesymbol{\Omega_k}{Curvature of the Universe: For a spatially
  flat universe, such as our own, $\Omega_k \approx 0$.}
\invisiblesymbol{V_C}{Comoving volume: The volume in which the number
  density of slowly evolving objects locked into the local Hubble
  flow, like galaxy clusters, is constant with redshift. The comoving
  volume element for redshift element $dz$ and solid angle element
  $d\Omega_S$ element is $dV_C=D_C[D_A(1+z)]^2 [\OM(1+z)^3 + \Omega_k
  (1+z)^2 + \OL]^{-1/2}~d\Omega_S~dz$.}
Individual clusters do not yield the information necessary to study
the underlying cosmogony. However, the number density of clusters
above a given mass threshold within a comoving volume element, \ie\
the cluster mass function, is a useful quantity
\citep{voitreview}. But the cluster mass function is a powerful
cosmological tool only if cluster masses can be accurately
measured. With no direct method of measuring cluster mass, easily
observable properties of clusters must be used as proxies to infer
mass.

Reliable mass proxies, such as cluster temperature and luminosity,
arise naturally from the theory that clusters are scaled versions of
each other. This property is commonly referred to as self-similarity
of mass-observables. More precisely, self-similarity presumes that
when cluster-scale gravitational potential wells are scaled by the
cluster-specific virial radius, the full cluster population has
potential wells which are simply scaled versions of one another
\citep{nfw1, nfw2}. Self-similarity is also expected to yield
low-scatter scaling relations between cluster properties such as
luminosity and temperature \citep{kaiser86, kaiser91,
  1991ApJ...383...95E, nfw1, nfw2, 1998ApJ...503..569E,
  1999ApJ...525..554F}. Consequently, mass-observable relations, such
as mass-temperature and mass-luminosity, derive from the fact that
most clusters are virialized, meaning the cluster's energy is shared
such that the virial theorem, $-2 \langle T \rangle = \langle V
\rangle$ where $\langle T \rangle$ is average kinetic energy and
$\langle V \rangle$ is average potential energy, is a valid
approximation. Both theoretical \citep{1996ApJ...469..494E,
  1998ApJ...495...80B, 1999ApJ...517..627M, 2001ApJ...555..597B,
  2002MNRAS.336..409B} and observational \citep{1984PhST....7..157M,
  edge91, white97, 1998MNRAS.297L..57A, 1998ApJ...503...77M,
  1999MNRAS.305..631A, 2001A&A...368..749F} studies have shown cluster
mass correlates well with X-ray temperature and luminosity, but that
there is much larger ($\approx 20-30\%$) scatter and different slopes
for these relations than expected. The breaking of self-similarity is
attributed to non-gravitational processes such as ongoing mergers
\citep[eg][]{2002ApJ...577..579R}, heating via feedback
\citep[eg][]{1999MNRAS.308..599C, bower01}, or radiative cooling in
the cluster core \citep[eg][]{2001Natur.414..425V, voitbryan}.

To reduce the scatter in mass scaling-relations and to increase their
utility for weighing clusters, how secondary processes alter
temperature and luminosity must first be quantified. It was predicted
that clusters with a high degree of spatial uniformity and symmetry
(\eg\ clusters with the least substructure in their dark matter and
gas distributions) would be the most relaxed and have the smallest
deviations from mean mass-observable relations. The utility of
substructure in quantifying relaxation is prevalent in many natural
systems, such as a placid lake or spherical gas cloud of uniform
density and temperature. Structural analysis of cluster simulations,
take for example the recent work of \citet{2008ApJ...681..167J} or
\citet{VV08}, have shown measures of substructure correlate well with
cluster dynamical state. But spatial analysis is at the mercy of
perspective. If equally robust aspect-independent measures of
dynamical state could be found, then quantifying deviation from mean
mass-scaling relations would be improved and the uncertainty of
inferred cluster masses could be further reduced. Scatter reduction
ultimately would lead to a more accurate cluster mass function, and by
extension, the constraints on theories explaining dark matter and dark
energy could grow tighter.

In Chapter \ref{ch:eband}, I present work investigating ICM
temperature inhomogeneity, a feature of the ICM which has been
proposed as a method for better understanding the dynamical state of a
cluster \citep{2001ApJ...546..100M}. Temperature inhomogeneity has the
advantage of being a spectroscopic quantity and therefore falls into
the class of aspect-independent metrics which may be useful for
reducing scatter in mass-observable relations. In a much larger
context, this dissertation may contribute to the improvement of our
understanding of the Universe's make-up and evolution.

%%%%%%%%%%%%%%%%%%%%%%%%%%%%%%%%%%%%%
\subsection{The Cooling Flow Problem}
\label{sec:cfprob}
%%%%%%%%%%%%%%%%%%%%%%%%%%%%%%%%%%%%%

For $50\%-66\%$ of galaxy clusters, the densest and coolest ($kT_X \la
\tvir/2$) ICM gas is found in the central $\sim 10\%$ of the cluster
gravitational potential well \citep{1984ApJ...285....1S,
  1992MNRAS.258..177E, white97, 1998MNRAS.298..416P,
  2005MNRAS.359.1481B}. For the temperature regime of the ICM,
radiative cooling time, $t_{cool}$ (see eqn. \ref{eqn:tcool}), is more
strongly dependent on density than temperature, $t_{cool} \propto
T_g^{1/2}\rho^2$, where $T_g$ is gas temperature and $\rho$ is gas
density. The energy lost via radiative cooling is seen as diffuse
thermal X-ray emission from the ICM \citep{gursky71, mitchell76,
  serle77}. When thermal energy is radiated away from the ICM, the gas
density must increase while gas temperature and internal pressure
respond by decreasing. The cluster core gas densities ultimately
reached through the cooling process are large enough such that the
cooling time required for the gas to radiate away its thermal energy
is much shorter than both the age of the Universe, \eg\ $t_{cool} \ll
\Hn^{-1}$, and the age of the cluster \citep{cowie77,
  fabian77}. Without compensatory heating, it thus follows that the
ICM in some cluster cores should cool and condense.

Gas within the cooling radius, \rcool, (defined as the radius at which
$t_{cool} = \Hn^{-1}$) is underpressured and cannot provide sufficient
pressure support to prevent overlying gas layers from forming a
subsonic flow of gas toward the bottom of the cluster gravitational
potential. However, if when the flowing gas enters the central galaxy
it has cooled to the point where the gas temperature equals the
central galaxy virial temperature, then adiabatic
compression\footnote{As the name indicates, no heat is exchanged
during adiabatic compression; but gas temperature rises because the
internal gas energy increases due to external work being done on the
system.} from the galaxy's gravitational potential well can balance
heat losses from radiative cooling. But, if the central galaxy's
gravitational potential is flat, then the gas energy gained via
gravitational effects can also be radiated away and catastrophic
cooling can proceed.

The sequence of events described above was given the name ``cooling
flow'' \citep{fabian77, cowie77, mathews78} and is the most simplistic
explanation of what happens to the ICM when it is continuously
cooling, spherically symmetric, and homogeneous \citep[see][for
reviews of cool gas in cluster cores]{fabiancfreview, cfreview,
2004cgpc.symp..143D}. The theoretical existence of cooling flows comes
directly from X-ray observations, yet the strongest observational
evidence for the existence of cooling flows will be seen when the gas
cools below X-ray emitting temperatures and forms stars, molecular
clouds, and emission line nebulae. Unfortunately, cooling flow models
were first presented at a time when no direct, complementary
observational evidence for cooling flows existed, highlighting the
difficulty of confronting the models. Undeterred, the X-ray
astrophysics community began referring to all clusters which had cores
meeting the criterion $t_{cool} < \Hn^{-1}$ as ``cooling flow
clusters,'' a tragic twist of nomenclature fate which has plagued many
scientific talks.

A mass deposition rate, $\dot{M}$, can be inferred for cooling flows
based on X-ray observations: $\dot{M} \propto
L_{X}(r<\rcool)(kT_X)^{-1}$, where $L_{X}(r<\rcool)$ is the X-ray
luminosity within the cooling region, $kT_X$ is the X-ray gas
temperature, and $\dot{M}$ typically has units of $\msol\pyr$. The
quantity $\dot{M}$ is useful in getting a handle on how much gas mass
is expected to be flowing into a cluster core. Mass deposition rates
have been estimated for many clusters and found to be in the range
$100-1000 \msol \pyr$ \citep{1984Natur.310..733F, white97,
  1998MNRAS.298..416P}. Mass deposition rates can also be estimated
using emission from individual spectral lines: $\dot{M} \propto
L_{X}(r<\rcool) \epsilon_f(T)$, where $L_{X}(r<\rcool)$ is the X-ray
luminosity within the cooling region and $\epsilon_f(T)$ is the
emissivity fraction attributable to a particular emission line. The
ICM soft X-ray emission lines of Fe XVII, O VIII, and Ne X at $E <
1.5$ keV for example, are especially useful in evaluating the
properties of cooling flows. Early low-resolution spectroscopy found
mass deposition rates consistent with those from X-ray observations
\citep[\eg][]{1982ApJ...262...33C}.

Not surprisingly, the largest, brightest, and most massive galaxy in a
cluster, the BCG, typically resides at the bottom of the cluster
potential, right at the center of where a cooling flow would
terminate. Real cooling flows were not expected to be symmetric,
continuous, or in thermodynamic equilibrium with the ambient
medium. Under these conditions, gas parcels at lower temperatures and
pressures experience thermal instability and are expected to rapidly
develop and collapse to form gaseous molecular clouds and stars. The
stellar and gaseous components of some BCGs clearly indicate some
amount of cooling and mass deposition has occurred. But are the
properties of the BCG population consistent with cooling flow model
predictions? For example, BCGs should be supremely luminous and
continually replenished with young, blue stellar populations since the
epoch of a BCG's formation. One should then expect the cores of
clusters suspected of hosting a cooling flow to have very bright, blue
BCGs bathed in clouds of emission line nebulae. However, observations
of cooling flow clusters reveal the true nature of the core to not
match these expectations of extremely high star formation rates, at
least not at the rate of $> 100 \msol
\pyr$.

The optical properties of massive galaxies and BCGs are well known and
neither population are as blue or bright as would be expected from the
extended periods of growth via cooling flows
\citep{1996MNRAS.283.1388M, 1996Natur.384..439S, 1996AJ....112..839C,
crawford99}. While attempts were made in the past to selectively
channel the unobserved cool gas into optically dark objects, such as
in low-mass, distributed star formation
\citep[\eg][]{1991ApJ...369L...1P}, methodical searches in the
optical, infrared, UV, radio, and soft X-ray wavelengths ($kT_X \la
2.0$ keV) have revealed that the total mass of cooler gas associated
with cooling flows is much less than expected \citep{hu85, heckman89,
  mcnamara90, odea94, 1994ApJ...436..669O, 1994AJ....107..448A,
  1994A&A...281..673M, voit95, 1997MNRAS.284L...1J,
  1998ApJ...494L.155F, 2000ApJ...545..670D, 2003ApJ...594L..13E}.

Confirming the suspicion that cooling flows are not cooling as
advertised, high-resolution \xmm\ RGS X-ray spectroscopy of clusters
expected to host very massive cooling flows definitively proved that
the ICM was not cooling to temperatures less than $1/3$ of the cluster
virial temperature \citep{peterson01, tamura01, peterson03}. A cooling
X-ray medium which has emission discontinuities at soft energies is
not predicted by the simplest single-phase cooling flow models, and a
troubling amount of fine-tuning (\eg\ minimum temperatures, hidden
soft emission) must be added to agree with observations. Modifications
such as preferential absorption of soft X-rays in the core region
\citep[\eg][]{1993MNRAS.262..901A} or turbulent mixing of a
multi-phase cooling flow \citep[\eg][]{2002MNRAS.332L..50F} have been
successful in matching observations, but these models lack the
universality needed to explain why {\it{all}} cooling flows are not as
massive as expected.

All of the observational evidence has resulted in a two-component
``cooling flow problem'': (1) spectroscopy of soft X-ray emission from
cooling flow clusters is inconsistent with theoretical predictions,
and (2) multiwavelength observations reveal a lack of cooled gas mass
or stars to account for the enormous theoretical mass deposition rates
implied by simple cooling flow models. So why and how is the cooling
of gas below $\tvir/3$ suppressed? As is the case with most questions,
the best answer thus far is simple: the cooling flow rates were wrong,
with many researchers suggesting the ICM has been intermittently
heated. But what feedback mechanisms are responsible for hindering
cooling in cluster cores?  How do these mechanisms operate?  What
observational constraints can we find to determine which combination
of feedback mechanism hypotheses are correct? The answers to these
questions have implications for both cluster evolution and massive
galaxy formation.

The cores of clusters are active places, so finding heating mechanisms
is not too difficult. The prime suspect, and best proposed solution to
the cooling flow problem thus far, invokes some combination of
supernovae and outbursts from the active galactic nucleus (AGN) in the
BCG \citep{1995MNRAS.276..663B, 1997MNRAS.288..355B,
2000ApJ...532...17L, 2001Natur.414..425V, 2002MNRAS.332..729C,
2002Natur.418..301B, 2002MNRAS.331..545B, 2002MNRAS.333..145N,
2002ApJ...581..223R, 2002MNRAS.335..610A, 2004MNRAS.348.1105O,
2004ApJ...613..811M, 2004ApJ...615..681R, 2004ApJ...617..896H,
2004MNRAS.355..995D, 2005ApJ...622..847S, pizzolato05, agnframework,
2006ApJ...643..120B, 2006ApJ...638..659M}. However, there are some big
problems: (1) AGN tend to deposit their energy along preferred axes,
while cooling in clusters proceeds in a nearly spherically symmetric
distribution in the core; (2) depositing AGN outburst energy at radii
nearest the AGN is difficult and how this mechanism works is not
understood; (3) there is a serious scale mismatch in heating and
cooling processes which has hampered the development of a
self-regulating feedback loop involving AGN. Radiative cooling
proceeds as the square of gas density, whereas heating is proportional
to volume. Hence, modeling feedback with a small source object, $r
\sim 1 \pc$, that is capable of compensating for radiative cooling
losses over an $\approx 10^6 \kpc^3$ volume, where the radial density
can change by four orders of magnitude, is quite
difficult. Dr. Donahue once framed this problem as, ``trying to heat
the whole of Europe with something the size of a button.''

The basic model of how AGN feedback works is that first gas accretes
onto a supermassive black hole at the center of the BCG, resulting in
the acceleration and ejection of very high energy particles back into
the cluster environment. The energy released in an AGN outburst is of
order $10^{58-61} \erg$. Under the right conditions, and via poorly
understood mechanisms, energy output by the AGN is transferred to the
ICM and thermalized, thereby heating the gas. The details of how this
process operates is beyond the scope of this dissertation
\citep[see][for a recent review]{mcnamrev}. However, in this
dissertation I do investigate some observable properties of clusters
which are directly impacted by feedback mechanisms.

Utilizing the quantity ICM entropy, I present results in Chapter
\ref{ch:ent_supp} which show that radial ICM entropy distributions for
a large sample of clusters have been altered in ways which are
consistent with AGN feedback models. Entropy and its connection to AGN
feedback is discussed in Subsection \ref{sec:entintro} of this
chapter. In Chapter \ref{ch:harad} I also present observational
results which support the hypothesis of \citet{conduction} that
electron thermal conduction may be an important mechanism in
distributing AGN feedback energy. Hence, this dissertation, in small
part, seeks to add to the understanding of how feedback functions in
clusters, and thus how to resolve the cooling flow problem -- the
resolution of which will lead to better models of galaxy formation and
cluster evolution.

%%%%%%%%%%%%%%%%%%%%%%%%%%%%%%%%%%%
\section{Chandra X-Ray Observatory}
\label{sec:chandra}
%%%%%%%%%%%%%%%%%%%%%%%%%%%%%%%%%%%

In this section I briefly describe what makes the \chandra\ X-ray
Observatory (\chandra\ or CXO for short) a ground-breaking and unique
telescope ideally suited for the work carried out in this
dissertation. In depth details of the telescope, instruments, and
spacecraft can be found at the CXO web
sites\footnote{http://chandra.harvard.edu/}$^{,}$\footnote{http://cxc.harvard.edu/}
or in \citet{chandra}. Much of what is discussed in the following
sections can also be found with more detail in ``The Chandra
Proposers' Observatory
Guide.''\footnote{http://cxc.harvard.edu/proposer/POG/} All figures
cited in this section are presented at the end of the corresponding
subsection.

%%%%%%%%%%%%%%%%%%%%%%%%%%%%%%%%%%%%%%
\subsection{Telescope and Instruments}
\label{sec:tele}
%%%%%%%%%%%%%%%%%%%%%%%%%%%%%%%%%%%%%%

The mean free path of an X-ray photon in a gas with the density of the
Earth's atmosphere is very short. Oxygen and nitrogen in the
atmosphere photoelectrically absorb X-ray photons resulting in 100\%
attenuation and make X-ray astronomy impossible from the Earth's
surface. Many long-standing theories in astrophysics predict a wide
variety of astronomical objects as X-ray emitters. Therefore,
astronomers and engineers have been sending X-ray telescopes into the
upper atmosphere and space for over 30 years now.

The most recent American X-ray mission to fly is the \chandra\ X-ray
Observatory. It is one of NASA's Great Observatories along with
\compton\ ($\gamma$-rays), \hubble\ (primarily optical), and
\spitzer\ (infrared). \chandra\ was built by Northrop-Grumman and is
operated by the National Aeronautics and Space Agency. \chandra\ was
launched in July 1999 and resides in a highly elliptical orbit with an
apogee of $\sim 140,000$ km and a perigee of $\sim 16,000$ km. One
orbit takes $\approx 64$ hours to complete. The telescope has four
nested iridium-coated paraboloid-hyperboloid mirrors with a focal
length of $\sim 10$ m. An illustration of the \chandra\ spacecraft is
shown in Figure \ref{fig:chandra}.

All data presented in this dissertation was collected with the
Advanced CCD Imaging Spectrometer (ACIS)
instrument\footnote{http://acis.mit.edu/acis}. ACIS is quite an
amazing and unique instrument in that it is an imager and
medium-resolution spectrometer at the same time. When an observation
is taken with ACIS, the data collected contains spatial and spectral
information since the location and energy of incoming photons are
recorded. The dual nature of ACIS allows the data to be analyzed by
spatially dividing up a cluster image and then extracting spectra for
these subregions of the image, a technique which is used heavily in
this dissertation.

The observing elements of ACIS are 10 $1024\times1024$ CCDs: six
linearly arranged CCDs (ACIS-S array) and four CCDs arranged in a
$2\times2$ mosaic (ACIS-I array). The ACIS focal plane is currently
kept at a temperature of $\sim -120\C$. During an observation, the
ACIS instrument is dithered along a Lissajous curve so parts of the
sky which fall in the chip gaps are also observed. Dithering also
ensures pixel variations of the CCD response are removed.

The high spatial and energy resolution of \chandra\ and its
instruments are ideal for studying clusters of galaxies. The telescope
on-board \chandra\ achieves on-axis spatial resolutions of $\la
0.5''$/pixel but it is the pixel size of the ACIS instrument ($\sim
0.492''$) which sets the resolution limit for observations. ACIS also
has an extraordinary energy resolution of $\Delta E/E \sim 100$. Below
energies of $\sim 0.3$ keV and above energies of $\sim 10$ keV the
ACIS effective area is ostensibly zero. The ACIS effective area also
peaks in the energy range $E \sim 0.7-2.0$ keV. As shown in
Figs. \ref{fig:brem} and \ref{fig:brem2}, a sizeable portion of galaxy
cluster emission occurs in the same energy range where the ACIS
effective area peaks. The energy resolution of ACIS also allows
individual emission line blends to be resolved in cluster
spectra. These aspects make \chandra\ a perfect choice for studying
clusters and the ICM in detail. Shown in Fig. \ref{fig:obs} are raw
observations of Abell 1795 with the aim-points on ACIS-I (top panel)
and ACIS-S (bottom panel). In Fig. \ref{fig:acisspec} is a spectrum
for the entire cluster extracted from the ACIS-I observation.

\begin{figure}[!hb]
  \begin{center}
    \includegraphics*[width=\textwidth, trim=0mm 0mm 0mm 0mm, clip]{chandra}
    \caption[\chandra\ X-ray Observatory spacecraft.]{An artist's
      rendition of the \chandra\ spacecraft. \chandra\ is the largest
      ($\sim 17$ m long; $\sim 4$ m wide) and most massive ($\sim 23$K
      kg) payload ever taken into space by NASA's Space Shuttle
      Program. The planned lifetime of the mission was 5 years, and
      the 10 year anniversary party is already planned. Illustration
      taken from Chandra X-ray Center.}
    \label{fig:chandra}
  \end{center}
\end{figure}

\begin{figure}[!hp]
  \begin{center}
    \includegraphics*[height=0.8\textheight, trim=0mm 0mm 0mm 0mm, clip]{obs}
    \caption[ACIS focal plane during observation.]{ In both panels
      celestial North is indicated by the blue arrow. {\it{Top
          panel:}} ACIS-I aimed observation of Abell 1795. The image
      has been binned by a factor of four so the whole field could be
      shown. {\it{Bottom panel:}} ACIS-S aimed observation of Abell
      1795. Again, the image is binned by a factor of four to show the
      whole field. For reference, the green boxes mark the ACIS-I
      chips which were off during this observation.}
    \label{fig:obs}
  \end{center}
\end{figure}

\begin{figure}[!hp]
  \begin{center}
    \includegraphics*[width=0.75\textwidth, trim=0mm 0mm 0mm 0mm,angle=270,clip]{acisspec}
    \caption[Spectrum of Abell 1795.]{Global spectrum of the cluster
      Abell 1795 with the best-fit single-component absorbed thermal
      spectral model overplotted (solid line). Comparing this spectrum
      with those of Figs. \ref{fig:brem} and \ref{fig:brem2}, the
      effects of finite energy resolution and convolving the spectral
      model with instrument responses are apparent. Individual
      spectral lines are now blends, and the spectral shape for $E <
      1.0$ keV has changed because of diminishing effective area.}
    \label{fig:acisspec}
  \end{center}
\end{figure}
\clearpage

%%%%%%%%%%%%%%%%%%%%%%%%%%%%%%%%%%%%%%%%%%%%%
\subsection{X-ray Background and Calibration}
\label{sec:cali}
%%%%%%%%%%%%%%%%%%%%%%%%%%%%%%%%%%%%%%%%%%%%%

\chandra\ is a magnificent piece of engineering, but it is not
perfect: observations are contaminated by background, the instruments
do not operate at full capacity, and the observatory has a finite
lifetime. In this section I briefly discuss these areas and how they
might affect past, current, and future scientific study with \chandra.

\subsubsection{Cosmic X-ray Background (CXB)}

\chandra\ is in a very high Earth orbit and is constantly bathed in
high-energy, charged particles originating from the cosmos which
interact with the CCDs (the eyes) and the materials housing the
instruments (the skull). The CXB is composed of a soft ($E < 2$ keV)
component attributable to extragalactic emission, local discrete
sources, and spatially varying diffuse Galactic emission. There are
also small contributions from the the ``local bubble''
\citep{2004ASSL..309..103S} and charge exchange within the solar
system \citep{2004ApJ...607..596W}. The possibility of emission from
unresolved point sources and other unknown CXB components also
exists. In most parts of the sky the soft CXB is not a large
contributor to the total background and can be modeled using a
combination of power-law and thermal spectral models and then
subtracted out of the data.

The CXB also has a hard ($E > 2$ keV) component which arises from
mostly extragalactic sources such as quasars and is well modeled as a
power-law. The spectral shape of the hard particle background has been
quite stable (up until mid-2005) and thus subtracting off the emission
by normalizing between observed and expected count rates in a
carefully chosen energy band makes removal of the hard component
straightforward.

Occasionally there are also very strong X-ray flares. These flares are
quite easy to detect in observations because, for a judiciously chosen
energy band/time bin combination, the count rate as a function of
observation time exhibits a dramatic spike during flaring. The time
intervals containing flare episodes can be excluded from the analysis
rendering them harmless. Harmless that is provided the flare was not
too long and some of the observing time allotment is usable.

\subsubsection{Instrumental Effects and Sources of Uncertainty}

There are a number of instrumental effects which must be considered
when analyzing data taken with \chandra. The geometric area of the
telescope's mirrors does not represent the ``usable'' area of the
mirrors. The true {\it{effective area}} of \chandra\ has been defined
by the Chandra X-ray Center (CXC) as the product of mirror geometric
area, reflectivity, off-axis vignetting, quantum efficiency of the
detectors, energy resolution of the detectors, and grating efficiency
(gratings were not in use during any of the observations used in this
dissertation). To varying degrees, all of these components depend on
energy and a few of them also have a spatial dependence. Discussion of
the effective area is a lengthy and involved topic. A more concise
understanding of the effective area can be attained from
visualization, hence the effective area as a function of energy is
shown in Figure \ref{fig:effarea}.

The ACIS instrument is also subject to dead/bad pixels, damage done by
interaction with very high-energy cosmic rays, imperfect read-out as a
function of CCD location, and a hydrocarbon contaminate which has been
building up since launch \citep{aciscontaminant}.

Observations are also at the mercy of uncertainty sources. The data
reduction software provided by the CXC (\ciao) and our own reduction
pipeline (CORP, discussed in Appendix \ref{ch:corp}) takes into
consideration:
\begin{enumerate}
\item Instrumental effects and calibration
\item $\approx 3\%$ error in absolute ACIS flux calibration
\item Statistical errors in the sky and background count rates
\item Errors due to uncertainty in the background normalization
\item Error due to the $\approx 2\%$ systematic uncertainty in the
background spectral shape
\item Cosmic variance of X-ray background sources
\item Unresolved source intensity
\item Scattering of source flux
\end{enumerate}

The list provided above is not comprehensive, but highlights the
largest sources of uncertainty: counting statistics, instrument
calibration, and background. In each section of this dissertation
where data analysis is discussed, the uncertainty and error analysis
is discussed in the context of the science objectives.

The \chandra\ mission was scheduled for a minimum five year mission
with the expectation that it would go longer. Nearing the ten year
anniversary of launch, it is therefore useful to wonder how
\chandra\ might be operating in years to come and what the future
holds for collecting data with \chandra\ five and even ten years from
now. The ``life expectancy'' of \chandra\ can be broken down into the
categories: spacecraft health, orbit stability, instrument
performance, and observation constraint evolution. Given the continued
progress of understanding \chandra's calibration, the relative
stability of the X-ray background, and the overall health of the
telescope as of last review, it has been suggested that \chandra\ will
survive at least a 15 year mission, \eg\ a decommissioning $\sim2014$
\citep{2007CUC, 2008ChNew..15...21B}.

\begin{figure}[!hp]
  \begin{center}
    \includegraphics*[width=\textwidth, trim=0mm 0mm 0mm 0mm,clip]{effarea}
    \caption[\chandra\ effective area as a function of energy.]{
      \chandra\ effective area as a function of energy. The effective
      area results from the product of mirror geometric area,
      reflectivity, off-axis vignetting, quantum efficiency of the
      detectors, energy resolution of the detectors, and grating
      efficiency. Note the ACIS peak sensitivity is in the energy
      range where the majority of the ICM emission occurs, $E =
      0.1-2.0$ keV. Figure taken from the CXC.}
    \label{fig:effarea}
  \end{center}
\end{figure}


%%%%%%%%%%%%%%%%%%%%%%
% Reprint cover page %
%%%%%%%%%%%%%%%%%%%%%%

\newpage
\parbox[c][0.9\textheight][c]{\linewidth}{
\begin{center}
Chapter Two
\end{center}
\begin{spacing}{1.1}
Cavagnolo, Kenneth W., Donahue, Megan, Voit, G. Mark, Sun, Ming
(2008). Bandpass Dependence of X-ray Temperatures in Galaxy
Clusters. {\underline{The Astrophysical Journal}}. 682:821-830.
\end{spacing}
}

%%%%%%%%%%%%%%%%%%%%%%%%%%%%%%%%%%%%%%%%%%%%%%%%%%%%%%%%%%%%%%%%%%%%%%
\chapter{Bandpass Dependence of X-ray Temperatures in Galaxy Clusters}
\label{ch:eband}
%%%%%%%%%%%%%%%%%%%%%%%%%%%%%%%%%%%%%%%%%%%%%%%%%%%%%%%%%%%%%%%%%%%%%%

%%%%%%%%%%%%%%%%%%%%%%
\section{Introduction}
\label{sec:ebandintro}
%%%%%%%%%%%%%%%%%%%%%%

The normalization, shape, and evolution of the cluster mass function
are useful for measuring cosmological parameters
\citep[\eg][]{1989ApJ...341L..71E, 1998ApJ...508..483W,
  2001ApJ...553..545H, 2004PhRvD..70l3008W}. In particular, the
evolution of large scale structure formation provides a complementary
and distinct constraint on cosmological parameters to those tests
which constrain them geometrically, such as supernovae
\citep{1998AJ....116.1009R, 2007ApJ...659...98R} and baryon acoustic
oscillations \citep{2005ApJ...633..560E}. However, clusters are a
useful cosmological tool only if we can infer cluster masses from
observable properties such as X-ray luminosity, X-ray temperature,
lensing shear, optical luminosity, or galaxy velocity
dispersion. Empirically, the correlation of mass to these observable
properties is well-established \citep[see][for a
  review]{voitreview}. But, there is non-negligible scatter in
mass-observable scaling relations which must be accounted for if
clusters are to serve as high-precision mass proxies necessary for
using clusters to study cosmological parameters such as the dark
energy equation of state. However, if we could identify a ``second
parameter" -- possibly reflecting the degree of relaxation in the
cluster -- we could improve the utility of clusters as cosmological
probes by parameterizing and reducing the scatter in mass-observable
scaling relations.

Toward this end, we desire to quantify the dynamical state of a
cluster beyond simply identifying which clusters appear relaxed and
those which do not. Most clusters are likely to have a dynamical state
which is somewhere in between \citep{2006ApJ...639...64O, kravtsov06,
  VV08}. The degree to which a cluster is virialized must first be
quantified within simulations that correctly predict the observable
properties of the cluster. Then, predictions for quantifying cluster
virialization may be tested, and possibly calibrated, with
observations of an unbiased sample of clusters \citep[\eg REXCESS
  sample of][]{rexcess}.

One study that examined how relaxation might affect the observable
properties of clusters was conducted by \citep[][hereafter
  ME01]{2001ApJ...546..100M} using the ensemble of simulations by
\citet{1997ApJ...491...38M}. ME01 found that most clusters which had
experienced a recent merger were cooler than the cluster
mass-observable scaling relations predicted. They attributed this to
the presence of cool, spectroscopically unresolved accreting
subclusters which introduce energy into the ICM and have a long
timescale for dissipation. The consequence was an under-prediction of
cluster binding masses of $15-30\%$ \citep{2001ApJ...546..100M}. It is
important to note that the simulations of \citet{1997ApJ...491...38M}
included only gravitational processes. The intervening years have
proven that radiative cooling is tremendously important in shaping the
global properties of clusters \citep[\eg][]{2004ApJ...613..811M,
  2006MNRAS.373..881P, nagai07}. Therefore, the magnitude of the
effect seen by ME01 could be somewhat different if radiative processes
are included.

One empirical observational method of quantifying the degree of
cluster relaxation involves using ICM substructure and employs the
power in ratios of X-ray surface brightness moments
\citep{1995ApJ...452..522B, 1996ApJ...458...27B,
  2005ApJ...624..606J}. Although an excellent tool, power ratios
suffer from being aspect-dependent \citep{2008ApJ...681..167J,
  VV08}. The work of ME01 suggested a complementary measure of
substructure which does not depend on projected perspective. In their
analysis, they found hard-band (2.0-9.0 keV) temperatures were $\sim
20\%$ hotter than broadband (0.5-9.0 keV) temperatures. Their
interpretation was that the cooler broadband temperature is the result
of unresolved accreting cool subclusters which are contributing
significant amounts of line emission to the soft band ($E < 2$
keV). This effect has been studied and confirmed by
\citet{2004MNRAS.354...10M} and \citet{2006ApJ...640..710V} using
simulated {\it Chandra} and {\it{XMM-Newton}} spectra.

ME01 suggested that this temperature skewing, and consequently the
fingerprint of mergers, could be detected utilizing the energy
resolution and soft-band sensitivity of {\it Chandra}. They proposed
selecting a large sample of clusters covering a broad dynamical range,
fitting a single-component temperature to the hard-band and broadband,
and then checking for a net skew above unity in the hard-band to
broadband temperature ratio. In this chapter we present the findings of
just such a temperature-ratio test using {\it Chandra} archival
data. We find the hard-band temperature exceeds the broadband
temperature, on average, by $\sim16\%$ in multiple flux-limited
samples of X-ray clusters from the {\it Chandra} archive. This mean
excess is weaker than the $20\%$ predicted by ME01, but is significant
at the $12\sigma$ level nonetheless. Hereafter, we refer to the
hard-band to broadband temperature ratio as $T_{HBR}$. We also find
that non-cool core systems and mergers tend to have higher values of
$T_{HBR}$. Our findings suggest that $T_{HBR}$ is an indicator of a
cluster's temporal proximity to the most recent merger event.

This chapter proceeds in the following manner: In \S\ref{sec:ebandselection}
we outline sample-selection criteria and {\it Chandra} observations
selected under these criteria. Data reduction and handling of the
X-ray background is discussed in \S\ref{sec:ebanddata}. Spectral extraction
is discussed in \S\ref{sec:ebandextraction}, while fitting and simulated
spectra are discussed in \S\ref{sec:ebandspecan}. Results and discussion of
our analysis are presented in \S\ref{sec:ebandr&d}. A summary of our work
is presented in \S\ref{sec:ebandsummary}. For this work we have assumed a
flat $\Lambda$CDM Universe with cosmology $\Omega_{M} = 0.3$,
$\Omega_{\Lambda} = 0.7$, and $H_{0} = 70$ km s$^{-1}$ Mpc$^{-1}$. All
quoted uncertainties are at the 1.6$\sigma$ level (90\% confidence).

%%%%%%%%%%%%%%%%%%%%%%%%%%
\section{Sample Selection}
\label{sec:ebandselection}
%%%%%%%%%%%%%%%%%%%%%%%%%%

Our sample was selected from observations publicly available in the
{\it Chandra} X-ray Telescope's Data Archive (CDA). Our initial
selection pass came from the {\it{ROSAT}} Brightest Cluster Sample
\citep{1998MNRAS.301..881E}, RBC Extended Sample
\citep{2000MNRAS.318..333E}, and {\it{ROSAT}} Brightest 55 Sample
\citep{1990MNRAS.245..559E, 1998MNRAS.298..416P}. The portion of our
sample at $z \gtrsim 0.4$ can also be found in a combination of the
{\it{Einstein}} Extended Medium Sensitivity Survey
\citep{1990ApJS...72..567G}, North Ecliptic Pole Survey
\citep{2006ApJS..162..304H}, {\it{ROSAT}} Deep Cluster Survey
\citep{1995ApJ...445L..11R}, {\it{ROSAT}} Serendipitous Survey
\citep{1998ApJ...502..558V}, and Massive Cluster Survey
\citep{2001ApJ...553..668E}. We later extended our sample to include
clusters found in the REFLEX Survey \citep{reflex}. Once we had a
master list of possible targets, we cross-referenced this list with
the CDA and gathered observations where a minimum of $R_{5000}$
(defined below) is fully within the CCD field of view.

$R_{\Delta_c}$ is defined as the radius at which the average cluster
density is $\Delta_c$ times the critical density of the Universe,
$\rho_c=3H(z)^2/8\pi G$. For our calculations of $R_{\Delta_c}$ we
adopt the relation from \citet{2002A&A...389....1A}:
\begin{eqnarray}
R_{\Delta_c} &=& 2.71 \mathrm{~Mpc~}
\beta_T^{1/2}
\Delta_{\mathrm{z}}^{-1/2}
(1+z)^{-3/2}
\left(\frac{kT_X}{10 \mathrm{~keV}}\right)^{1/2}\\
\Delta_z &=& \frac{\Delta_c \Omega_M}{18\pi^2\Omega_z} \nonumber \\
\Omega_z &=& \frac{\Omega_M (1+z)^3}{[\Omega_M
(1+z)^3]+[(1-\Omega_M-\Omega_{\Lambda})(1+z)^2]+\Omega_{\Lambda}} \nonumber
\end{eqnarray}
where $R_{\Delta_c}$ is in units of $h_{70}^{-1}$, $\Delta_c$ is the
assumed density contrast of the cluster at $R_{\Delta_c}$, and
$\beta_T$ is a numerically determined, cosmology-independent
($\lesssim \pm 20\%$) normalization for the virial relation $GM/2R =
\beta_T kT_{virial}$. We use $\beta_T = 1.05$ taken from
\citet{1996ApJ...469..494E}.

The result of our CDA search was a total of 374 observations of which
we used 244 for 202 clusters. The clusters making up our sample cover
a redshift range of $z = 0.045-1.24$, a temperature range of $T_X =
2.6-19.2 \mathrm{~keV}$, and bolometric luminosities of $L_{bol} =
0.12-100.4\times10^{44} \mathrm{~ergs~s}^{-1}$. The bolometric ($E =
0.1-100$ keV) luminosities for our sample clusters plotted as a
function of redshift are shown in Figure \ref{fig:lx_z}. These
$L_{bol}$ values are calculated from our best-fit spectral models and
are limited to the region of the spectral extraction (from $R=70$ kpc
to $R=R_{2500}$, or $R_{5000}$ in the cases in which no $R_{2500}$ fit
was possible). Basic properties of our sample are listed in Table
\ref{tab:sample}.

\begin{figure}
\begin{center}
\includegraphics*[width=\textwidth, trim=0mm 0mm 0mm 0mm, clip]{eband_f1.eps}
\caption[Redshift distribution of bolometric luminosities for
  $T_{HBR}$ sample]{Bolometric luminosity ($E = 0.1-100$ keV) plotted
  as a function of redshift for the 202 clusters which make up the
  initial sample. $L_{bol}$ values are limited to the region of
  spectral extraction, $R=R_{2500-\mathrm{CORE}}$. For clusters
  without $R_{2500-\mathrm{CORE}}$ fits, $R=R_{5000-\mathrm{CORE}}$
  fits were used and are denoted in the figure by empty stars. Dotted
  lines represent constant fluxes of $3.0\times10^{-15}$, $10^{-14}$,
  $10^{-13}$, and $10^{-12} \flux$.}
\label{fig:lx_z}
\end{center}
\end{figure}

For the sole purpose of defining extraction regions based on fixed
overdensities as discussed in \S\ref{sec:ebandextraction}, fiducial
temperatures (measured with {\it ASCA}) and redshifts were taken from
\citet{hornerthesis} (all redshifts confirmed with
NED\footnote{http://nedwww.ipac.caltech.edu/}). We show below that the
{\it ASCA} temperatures are sufficiently close to the {\it Chandra}
temperatures such that $R_{\Delta_c}$ is reliably estimated to within
20\%. Note that $R_{\Delta_c}$ is proportional to $T^{1/2}$, so that a
20\% error in the temperature leads to only a 10\% error in
$R_{\Delta_c}$, which in turn has no detectable effect on our final
results. For clusters not listed in \citet{hornerthesis} , we used a
literature search to find previously measured temperatures. If no
published value could be located, we measured the global temperature
by recursively extracting a spectrum in the region $0.1<r<0.2 R_{500}$
fitting a temperature and recalculating $R_{500}$. This process was
repeated until three consecutive iterations produced $R_{500}$ values
which differed by $\leq 1\sigma$. This method of temperature
determination has been employed in other studies, see
\citet{2006MNRAS.372.1496S} and \citet{2006ApJS..162..304H} as examples.

%%%%%%%%%%%%%%%%%%%%%%%%%%%%%
\section{{\it Chandra} Data}
\label{sec:ebanddata}
%%%%%%%%%%%%%%%%%%%%%%%%%%%%%

%%%%%%%%%%%%%%%%%%%%%%%%%%%%%%%%%%%%%%%
\subsection{Reprocessing and Reduction}
\label{sec:ebandreprocessing}
%%%%%%%%%%%%%%%%%%%%%%%%%%%%%%%%%%%%%%%

All data sets were reduced using the \chandra\} Interactive Analysis
of Observations package (\ciao) and accompanying Calibration Database
(\caldb). Using \ciao\ 3.3.0.1 and \caldb\ 3.2.2, standard data
analysis was followed for each observation to apply the most
up-to-date time-dependent gain correction and when appropriate, charge
transfer inefficiency correction \citep{2000ApJ...534L.139T}.

Point sources were identified in an exposure-corrected events file
using the adaptive wavelet tool {\textsc{wavdetect}}
\citep{2002ApJS..138..185F}. A $2 \sigma$ region surrounding each
point source was automatically output by {\textsc{wavdetect}} to
define an exclusion mask.  All point sources were then visually
confirmed and we added regions for point sources which were missed by
{\textsc{wavdetect}} and deleted regions for spuriously detected
``sources.'' Spurious sources are typically faint CCD features (chip
gaps and chip edges) not fully removed after dividing by the exposure
map. This process resulted in an events file (at ``level 2'') that has
been cleaned of point sources.

To check for contamination from background flares or periods of
excessively high background, light curve analysis was performed using
Maxim Markevitch's contributed \ciao\ script
{\textsc{lc\_clean.sl}}\footnote{http://cxc.harvard.edu/contrib/maxim/acisbg/}.
Periods with count rates $\geq 3\sigma$ and/or a factor $\geq 1.2$ of
the mean background level of the observation were removed from the
good time interval file. As prescribed by Markevitch's
cookbook\footnote{http://cxc.harvard.edu/contrib/maxim/acisbg/COOKBOOK},
ACIS front-illuminated (FI) chips were analyzed in the $0.3-12.0$ keV
range, and the $2.5-7.0$ keV energy range for the ACIS
back-illuminated (BI) chips.

When a FI and BI chip were both active during an observation, we
compared light curves from both chips to detect long duration,
soft-flares which can go undetected on the FI chips but show up on the
BI chips. While rare, this class of flare must be filtered out of the
data, as it introduces a spectral component which artificially
increases the best-fit temperature via a high energy tail. We find
evidence for a long duration soft flare in the observations of Abell
1758 \citep{2004ApJ...613..831D}, CL J2302.8+0844, and IRAS
09104+4109. These flares were handled by removing the time period of
the flare from the GTI file.

Defining the cluster ``center'' is essential for the later purpose of
excluding cool cores from our spectral analysis (see
\S\ref{sec:ebandextraction}). To determine the cluster center, we
calculated the centroid of the flare cleaned, point-source free
level-2 events file filtered to include only photons in the $0.7-7.0$
keV range. Before centroiding, the events file was exposure-corrected
and ``holes'' created by excluding point sources were filled using
interpolated values taken from a narrow annular region just outside
the hole (holes are not filled during spectral extraction discussed in
\S\ref{sec:ebandextraction}). Prior to centroiding, we defined the emission
peak by heavily binning the image, finding the peak value within a
circular region extending from the peak to the chip edge (defined by
the radius $R_{max}$), reducing $R_{max}$ by 5\%, reducing the binning
by a factor of 2, and finding the peak again. This process was
repeated until the image was unbinned (binning factor of 1). We then
returned to an unbinned image with an aperture centered on the
emission peak with a radius $R_{max}$ and found the centroid using
\ciao's {\textsc{dmstat}}. The centroid, ($x_c$, $y_c$), for a
distribution of $N$ good pixels with coordinates ($x_i$, $y_j$) and
values f($x_i$,$y_j$) is defined as:
\begin{eqnarray}
Q &=& \sum_{i,j=1}^N f(x_i,y_i) \\
x_c &=& \frac{\sum_{i,j=1}^N x_i \cdot f(x_i,y_i)}{Q} \nonumber \\
y_c &=& \frac{\sum_{i,j=1}^N y_i \cdot f(x_i,y_i)}{Q}. \nonumber
\end{eqnarray}

If the centroid was within 70 kpc of the emission peak, the emission
peak was selected as the center, otherwise the centroid was used as
the center. This selection was made to ensure all ``peaky'' cool cores
coincided with the cluster center, thus maximizing their exclusion
later in our analysis. All cluster centers were additionally verified
by eye.

%%%%%%%%%%%%%%%%%%%%%%%%%%%%%
\subsection{X-ray Background}
\label{sec:ebandbackground}
%%%%%%%%%%%%%%%%%%%%%%%%%%%%%

Because we measured a global cluster temperature, specifically looking
for a temperature ratio shift in energy bands which can be
contaminated by the high-energy particle background or the soft local
background, it was important to carefully analyze the background and
subtract it from our source spectra. Below we outline three steps
taken in handling the background: customization of blank-sky
backgrounds, re-normalization of these backgrounds for variation of
hard-particle count rates, and fitting of soft background residuals.

We used the blank-sky observations of the X-ray background from
\citet{2001ApJ...562L.153M} and supplied within the CXC \caldb. First,
we compared the flux from the diffuse soft X-ray background of the
{\it{ROSAT}} All-Sky Survey ({\it RASS}) combined bands $R12$, $R45$,
and $R67$ to the 0.7-2.0 keV flux in each extraction aperture for each
observation. {\it RASS} combined bands give fluxes for energy ranges
of 0.12-0.28, 0.47-1.21, and 0.76-2.04 keV respectively corresponding
to $R12$, $R45$, and $R67$. For the purpose of simplifying subsequent
analysis, we discarded observations with an $R45$ flux $\geq 10\%$ of
the total cluster X-ray flux.

The appropriate blank-sky dataset for each observation was selected
from the \caldb, reprocessed exactly as the observation was, and then
reprojected using the aspect solutions provided with each
observation. For observations on the ACIS-I array, we reprojected
blank-sky backgrounds for chips I0-I3 plus chips S2 and/or S3. For
ACIS-S observations, we created blank-sky backgrounds for the target
chip, plus chips I2 and/or I3. The additional off aim-point chips were
included only if they were active during the observation and had
available blank-sky data sets for the observation time period. Off
aim-point chips were cleaned for point sources and diffuse sources
using the method outlined in \S\ref{sec:ebandreprocessing}.

The additional off aim-point chips were included in data reduction
since they contain data which is farther from the cluster center and
are therefore more useful in analyzing the observation background. For
observations which did not have a matching off aim-point blank-sky
background, a source-free region of the active chips is
located and used for background normalization. To normalize the hard
particle component we measured fluxes for identical regions in the
blank-sky field and target field in the 9.5-12.0 keV range. The
effective area of the ACIS arrays above 9.5 keV is approximately zero,
and thus the collected photons there are exclusively from the particle
background.

A histogram of the ratios of the 9.5-12.0 keV count rate from an
observation's off aim-point chip to that of the observation specific
blank-sky background are presented in Figure \ref{fig:bgd}. The
majority of the observations are in agreement to $\lesssim 20\%$ of
the blank-sky background rate, which is small enough to not affect our
analysis. Even so, we re-normalized all blank-sky backgrounds to match
the observed background.

\begin{figure}
\begin{center}
\includegraphics*[width=\textwidth, trim=5mm 0mm 0mm 0mm,clip]{eband_f2.eps}
\caption[Histogram of hard-particle count rate ratios for $T_{HBR}$
  sample] {Ratio of target field and blank-sky field count rates in
  the 9.5-12.0 keV band for all 244 observations in our initial
  sample. Vertical dashed lines represent $\pm 20\%$ of unity. Despite
  the good agreement between the blank-sky background and observation
  count rates for most observations, all backgrounds are normalized.}
\label{fig:bgd}
\end{center}
\end{figure}

Normalization brings the observation background and blank-sky
background into agreement for $E > 2$ keV, but even after
normalization, typically, there may exist a soft excess/deficit
associated with the spatially varying soft Galactic
background. Following the technique detailed in
\citet{2005ApJ...628..655V}, we constructed and fit soft residuals for
this component. For each observation we subtracted a spectrum of the
blank-sky field from a spectrum of the off aim-point field to create a
soft residual. The residual was fit with a solar abundance,
zero-redshift \mekal\ model \citep{mekal1, mekal2, mekal3} in which
the normalization was allowed to be negative. The resulting best-fit
temperatures for all of the soft residuals identified here were
between 0.2-1.0 keV, which is in agreement with results of
\citet{2005ApJ...628..655V}. The model normalization of this background
component was then scaled to the cluster sky area. The re-scaled
component was included as a fixed background component during fitting
of a cluster's spectra.

%%%%%%%%%%%%%%%%%%%%%%%%%%%%%
\section{Spectral Extraction}
\label{sec:ebandextraction}
%%%%%%%%%%%%%%%%%%%%%%%%%%%%%

The simulated spectra calculated by ME01 were analyzed in a broad
energy band of $0.5-9.0$ keV and a hard energy band of
$2.0_{\mathrm{rest}}-9.0$ keV, but to make a reliable comparison with
{\it{Chandra}} data we used narrower energy ranges of 0.7-7.0 keV for
the broad energy band and $2.0_{\mathrm{rest}}-7.0$ keV for the hard
energy band. We excluded data below $0.7$ keV to avoid the effective
area and quantum efficiency variations of the ACIS detectors, and
excluded energies above $7.0$ keV in which diffuse source emission is
dominated by the background and where {\it{Chandra}}'s effective area
is small. We also accounted for cosmic redshift by shifting the lower
energy boundary of the hard-band from 2.0 keV to $2.0/(1+z)$ keV
(henceforth, the 2.0 keV cut is in the rest frame).

ME01 calculated the relation between $T_{0.5-9.0}$ and $T_{2.0-9.0}$
using apertures of $R_{200}$ and $R_{500}$ in size. While it is
trivial to calculate a temperature out to $R_{200}$ or $R_{500}$ for a
simulation, such a measurement at these scales is extremely difficult
with {\it Chandra} observations (see \citet{2005ApJ...628..655V} for a
detailed example). Thus, we chose to extract spectra from regions with
radius $R_{5000}$, and $R_{2500}$ when possible. Clusters analyzed
only within $R_{5000}$ are denoted in Table \ref{tab:sample} by a
double dagger ($\ddagger$).

The cores of some clusters are dominated by gas at $\lesssim
T_{virial}/2$ which can greatly affect the global best-fit
temperature; therefore, we excised the central 70 kpc of each
aperture. These excised apertures are denoted by ``-CORE'' in the
text. Recent work by \citet{2007ApJ...668..772M} has shown excising
0.15 $R_{500}$ rather than a static 70 kpc reduces scatter in
mass-observable scaling relations. However, our smaller excised region
seems sufficient for this investigation because for cool core clusters
the average radial temperature at $r > 70$ kpc is approximately
isothermal \citep{2005ApJ...628..655V}. Indeed, we find that cool core
clusters have smaller than average $T_{HBR}$ when the 70 kpc region
has been excised (\S\ref{sec:ebandccncc}).

Although some clusters are not circular in projection, but rather are
elliptical or asymmetric, we found that assuming spherical symmetry
and extracting spectra from a circular annulus did not significantly
change the best-fit values. For another such example see
\citet{2005MNRAS.359.1481B}.

After defining annular apertures, we extracted source spectra from the
target cluster and background spectra from the corresponding
normalized blank-sky dataset. By standard \ciao\ means we created
weighted effective area functions (WARFs) and redistribution matrices
(WRMFs) for each cluster using a flux-weighted map (WMAP) across the
entire extraction region. The WMAP was calculated over the energy
range 0.3-2.0 keV to weight calibrations that vary as a function of
position on the chip. The CCD characteristics which affect the
analysis of extended sources, such as energy dependent vignetting, are
contained within these files. Each spectrum was then binned to contain
a minimum of 25 counts per channel.

%%%%%%%%%%%%%%%%%%%%%%%%%%%
\section{Spectral Analysis}
\label{sec:ebandspecan}
%%%%%%%%%%%%%%%%%%%%%%%%%%%

%%%%%%%%%%%%%%%%%%%%
\subsection{Fitting}
\label{sec:ebandfitting}
%%%%%%%%%%%%%%%%%%%%

Spectra were fit with \xspec\ 11.3.2ag \citep{xspec} using a
single-temperature \mekal\ model in combination with the photoelectric
absorption model {\textsc{WABS}} \citep{wabs} to account for Galactic
absorption. Galactic absorption values, $N_{H}$, are taken from
\citet{dickeylockman}. The potentially free parameters of the absorbed
thermal model are $N_{H}$, X-ray temperature ($T_{X}$), metal
abundance normalized to solar \citep[elemental ratios taken
  from][]{ag89}, and a normalization proportional to the integrated
emission measure of the cluster. Results from the fitting are
presented in Tables \ref{tab:r2500specfits} and
\ref{tab:r5000specfits}. No systematic error is added during fitting,
and thus all quoted errors are statistical only. The statistic used
during fitting was $\chi^2$ (\xspec\ statistics package
\textsc{chi}). Every cluster analyzed was found to have greater than
1500 background-subtracted source counts in the spectrum.

For some clusters, more than one observation was available in the
archive. We utilized the power of the combined exposure time by first
extracting independent spectra, WARFs, WRMFs, normalized background
spectra, and soft residuals for each observation. Then, these
independent spectra were read into \xspec\ simultaneously and fit with
one spectral model which had all parameters, except normalization,
tied among the spectra. The simultaneous fit is what is reported for
these clusters, denoted by a star ($\star$), in Tables
\ref{tab:r2500specfits} and \ref{tab:r5000specfits}.

Additional statistical error was introduced into the fits because of
uncertainty associated with the soft local background component
discussed in \S\ref{sec:ebandbackground}. To estimate the sensitivity of
our best-fit temperatures to this uncertainty, we used the differences
between $T_{X}$ for a model using the best-fit soft background
normalization and $T_{X}$ for models using $\pm1\sigma$ of the soft
background normalization. The statistical uncertainty of the original
fit and the additional uncertainty inferred from the range of
normalizations to the soft X-ray background component were then added
in quadrature to produce a final error. In all cases this additional
background error on the temperature was less than 10\% of the total
statistical error, and therefore represents a minor inflation of the
error budget.

When comparing fits with fixed Galactic column density with those
where it was a free parameter, we found that neither the goodness of
fit per free parameter nor the best-fit $T_{X}$ were significantly
different. Thus, $N_{H}$ was fixed at the Galactic value with the
exception of three cases: Abell 399 \citep{2004MNRAS.351.1439S}, Abell
520, and Hercules A. For these three clusters $N_{H}$ is a free
parameter. In all fits, the metal abundance was a free parameter.

After fitting we rejected several data sets as their best-fit
$T_{2.0-7.0}$ had no upper bound in the 90\% confidence interval and
thus were insufficient for our analysis. All fits for the clusters
Abell 781, Abell 1682, CL J1213+0253, CL J1641+4001, IRAS 09104+4109,
Lynx E, MACS J1824.3+4309, MS 0302.7+1658, and RX J1053+5735 were
rejected. We also removed Abell 2550 from our sample after finding it
to be an anomalously cool ($T_{X} \sim$ 2 keV) ``cluster''. In fact,
Abell 2550 is a line-of-sight set of groups, as discussed by
\citet{2004cgpc.sympE..31M}. After these rejections, we are left with a
final sample of \ebandnuma\ clusters which have $R_{2500-\mathrm{CORE}}$ fits
and \ebandnumb\ clusters which have $R_{5000-\mathrm{CORE}}$ fits.

%%%%%%%%%%%%%%%%%%%%%%%%%%%%%%
\subsection{Simulated Spectra}
\label{sec:ebandsimulated}
%%%%%%%%%%%%%%%%%%%%%%%%%%%%%%

To quantify the effect a second, cooler gas component would have on
the fit of a single-component spectral model, we created an ensemble
of simulated spectra for each real spectrum in our entire sample using
{\textsc{XSPEC}}. With these simulated spectra we sought to answer the
question: Given the count level in each observation of our sample, how
bright must a second temperature component be for it to affect the
observed temperature ratio? Put another way, we asked at what flux
ratio a second gas phase produces a temperature ratio, $T_{HBR}$, of
greater than unity with 90\% confidence.

We began by adding the observation-specific background to a convolved,
absorbed thermal model with two temperature components observed for a
time period equal to the actual observation's exposure time and adding
Poisson noise. For each realization of an observation's simulated
spectrum, we defined the primary component to have the best-fit
temperature and metallicity of the $R_{2500-\mathrm{CORE}}$ 0.7-7.0
keV fit, or $R_{5000-\mathrm{CORE}}$ if no $R_{2500-\mathrm{CORE}}$
fit was performed. We then incremented the secondary component
temperature over the values 0.5, 0.75, 1.0, 2.0, and 3.0 keV. The
metallicity of the secondary component was fixed and set equal to the
metallicity of the primary component.

We adjusted the normalization of the simulated two-component spectra
to achieve equivalent count rates to those in the real spectra. The
sum of normalizations can be expressed as $N = N_1 + \xi N_2$. We set
the secondary component normalization to $N_2 = \xi N_{bf}$, where
$N_{bf}$ is the best-fit normalization of the appropriate 0.7-7.0 keV
fit and $\xi$ is a preset factor taking the values 0.4, 0.3, 0.2,
0.15, 0.1, and 0.05. The primary component normalization, $N_1$, was
determined through an iterative process to make real and simulated
spectral count rates match. The parameter $\xi$ therefore represents
the fractional contribution of the cooler component to the overall
count rate.

There are many systematics at work in the full ensemble of observation
specific simulated spectra, such as redshift, column density, and
metal abundance. Thus as a further check of spectral sensitivity to
the presence of a second gas phase, we simulated additional spectra
for the case of an idealized observation. We followed a similar
procedure to that outlined above, but in this instance we used a finer
temperature and $\xi$ grid of $T_2 = 0.5 \rightarrow 3.0$ in steps of
0.25 keV, and $\xi = 0.02 \rightarrow 0.4$ in steps of 0.02. The input
spectral model was $N_{H} = 3.0\times10^{20}$ cm$^{-2}$, $T_1 = 5$
keV, $Z/Z_{\odot} = 0.3$ and $z = 0.1$. We also varied the exposure
times such that the total number of counts in the 0.7-7.0 keV band was
15K, 30K, 60K, or 120K. For these spectra we used the on-axis sample
response files provided to Cycle 10
proposers\footnote{http://cxc.harvard.edu/caldb/prop\_plan/imaging/index.html}.
Poisson noise is added, but no background is considered.

We also simulated a control sample of single-temperature models. The
control sample is simply a simulated version of the best-fit
model. This control provides us with a statistical test of how often
the actual hard-component temperature might differ from a broadband
temperature fit if calibration effects are under control. Fits for the
control sample are shown in the far right panels of Figure
\ref{fig:ftx}.

For each observation, we have 65 total simulated spectra: 35
single-temperature control spectra and 30 two-component simulated
spectra (5 secondary temperatures, each with six different $\xi$). Our
resulting ensemble of simulated spectra contains 12,765 spectra. After
generating all the spectra we followed the same fitting routine
detailed in \S\ref{sec:ebandfitting}.

With the ensemble of simulated spectra we then asked the question: for
each $T_2$ and $\Delta T_X$ (defined as the difference between the
primary and secondary temperature components) what is the minimum
value of $\xi$, called $\xi_{min}$, that produces $T_{HBR} \geq 1.1$
at 90\% confidence? From our analysis of these simulated spectra we
have found these important results:
\begin{enumerate}

\item In the control sample, a single-temperature model rarely ($\sim
  2\%$ of the time) gives a significantly different $T_{0.7-7.0}$ and
  $T_{2.0-7.0}$. The weighted average (Fig. \ref{fig:ftx}, right
  panels) for the control sample is $1.002 \pm 0.001$ and the standard
  deviation is $\pm0.044$. The $T_{HBR}$ distribution for the control
  sample appears to have an intrinsic width which is likely associated
  with statistical noise of fitting in {\textsc{XSPEC}} (Dupke,
  private communication). This result indicates that our remaining set
  of observations is statistically sound, \eg\ our finding that
  $T_{HBR}$ significantly differs from 1.0 cannot result from
  statistical fluctuations alone.

\item Shown in Table \ref{tab:simres} are the contributions a second
  cooler component must make in the case of the idealized spectra in
  order to produce $T_{HBR} \geq 1.1$ at 90\% confidence. In general,
  the contribution of cooler gas must be $> 10\%$ for $T_2 < 2$ keV to
  produce $T_{HBR}$ as large as 1.1. The increase in percentages at
  $T_2 < 1.0$ keV is owing to the energy band we consider (0.7-7.0
  keV) as gas cooler than 0.7 keV must be brighter than at 1.0 keV in
  order to make an equivalent contribution to the soft end of the
  spectrum at 0.7 keV.

\item In the full ensemble of observation-specific simulated spectra,
  we find a great deal of statistical scatter in $\xi_{min}$ at any
  given $\Delta T_X$. This was expected as the full ensemble is a
  superposition of spectra with a broad range of total counts,
  $N_{H}$, redshifts, abundance, and backgrounds. But using the
  idealized simulated spectra as a guide, we find for those spectra
  with $N_{\mathrm{counts}} \gtrsim 15000$, producing $T_{HBR} \geq
  1.1$ at 90\% confidence again requires the cooler gas to be
  contributing $> 10\%$ of the emission. These results are also
  summarized in Table \ref{tab:simres}. The good agreement between the
  idealized and observation-specific simulated spectra indicates that
  while many more factors are in play for the observation-specific
  spectra, they do not degrade our ability to reliably measure
  $T_{HBR} > 1.1$.  The trend here of a common soft component
  sufficient to change the temperature measurement in a
  single-temperature model is statistical, a result that comes from an
  aggregate view of the sample rather than any individual fit.

\item As redshift increases, gas cooler than 1.0 keV is slowly
  redshifted out of the observable X-ray band. As expected, we find
  from our simulated spectra that for $z \geq 0.6$, $T_{HBR}$ is no
  longer statistically distinguishable from unity. In addition, the
  $T_{2.0-7.0}$ lower boundary nears convergence with the
  $T_{0.7-7.0}$ lower boundary as $z$ increases, and for $z = 0.6$,
  the hard-band lower limit is 1.25 keV, while at the highest redshift
  considered, $z = 1.2$, the hard-band lower limit is only 0.91
  keV. For the 14 clusters with $z \geq 0.6$ in our real sample we are
  most likely underestimating the actual amount of temperature
  inhomogeneity. We have tested the effect of excluding these clusters
  on our results, and find a negligible change in the overall skew of
  $T_{HBR}$ to greater than unity.
\end{enumerate}

\singlespacing
\begin{thesistable}{cc|cc}
\thesistablehead{Summary of two-component simulations}{Summary of
  two-component simulations}{\multicolumn{2}{c|}{Idealized Spectra} &
  \multicolumn{2}{c}{Observation-Specific Spectra}\\ $T_2$ &
  $\xi_{min}$ & $T_2$ & $\xi_{min}$\\ keV & & keV &}{tab:simres}
0.50 & $\geq 12\% \pm 4\%$ & 0.50 & $\geq 14.5\% \pm 0.1\%$\\
0.75 & $\geq 12\% \pm 4\%$ & 0.75 & $\geq 11.7\% \pm 0.1\%$\\
1.00 & $\geq 8\% \pm 3\%$  & 1.00 & $\geq 11.6\% \pm 0.1\%$\\
1.25 & $\geq 17\% \pm 3\%$ & - & -\\
1.50 & $\geq 23\% \pm 5\%$ & - & -\\
1.75 & $\geq 28\% \pm 4\%$ & - & -\\
2.00 & none                & 2.00 & $\geq 25.5\% \pm 0.1\%$\\
3.00 & none                & 3.00 & $\geq 28.9\% \pm 0.1\%$
\end{thesistable}
\doublespacing

Table \ref{tab:simres} summarizes the results of the two temperature
component spectra simulations for the ideal and observation-specific
cases (see \S\ref{sec:ebandsimulated} for details). The parameter
$\xi_{min}$ represents the minimum fractional contribution of the
cooler component, $T_2$, to the overall count rate in order to produce
$T_{HBR} \geq 1.1$ at 90\% confidence. The results for the
observation-specific spectra are for spectra with $N_{\mathrm{counts}}
> 15,000$.

%%%%%%%%%%%%%%%%%%%%%%%%%%%%%%%%
\section{Results and Discussion}
\label{sec:ebandr&d}
%%%%%%%%%%%%%%%%%%%%%%%%%%%%%%%%

%%%%%%%%%%%%%%%%%%%%%%%%%%%%%%%
\subsection{Temperature Ratios}
\label{sec:ebandtfresults}
%%%%%%%%%%%%%%%%%%%%%%%%%%%%%%%

For each cluster we have measured a ratio of the hard-band to
broadband temperature defined as $T_{HBR}$ =
$T_{2.0-7.0}$/$T_{0.7-7.0}$. We find that the mean $T_{HBR}$ for our
entire sample is greater than unity at more than $12\sigma$
significance. The weighted mean values for our sample are shown in
Table \ref{tab:wavg}. Quoted errors in Table \ref{tab:wavg} are
standard deviation of the mean calculated using an unbiased estimator
for weighted samples. Simulated sample has been culled to include only
$T_2$=0.75 keV. Presented in Figure \ref{fig:ftx} are the binned
weighted means and raw $T_{HBR}$ values for $R_{2500-\mathrm{CORE}}$,
$R_{5000-\mathrm{CORE}}$, and the simulated control sample. The
peculiar points with $T_{HBR} <$ 1 are all statistically consistent
with unity. The presence of clusters with $T_{HBR}$ = 1 suggests that
systematic calibration uncertainties are not the sole reason for
deviations of $T_{HBR}$ from 1. We also find that the temperature
ratio does not depend on the best-fit broadband temperature, and that
the observed dispersion of $T_{HBR}$ is greater than the predicted
dispersion arising from systematic uncertainties.

\begin{figure}
\begin{center}
\includegraphics*[width=\textwidth, trim=0mm 0mm 0mm 0mm, clip]{eband_f3.eps}
\caption[$T_{HBR}$ vs. broadband temperature for $T_{HBR}$ sample]{
  Best-fit temperatures for the hard-band, $T_{2.0-7.0}$, divided by
  the broadband, $T_{0.7-7.0}$, and plotted against the broadband
  temperature. For binned data, each bin contains 25 clusters, with
  the exception of the highest temperature bins which contain 16 and
  17 for $R_{2500-\mathrm{CORE}}$ and $R_{5000-\mathrm{CORE}}$,
  respectively. The simulated data bins contain 1000 clusters with the
  last bin having 780 clusters. The line of equality is shown as a
  dashed line and the weighted mean for the full sample is shown as a
  dashed-dotted line. Error bars are omitted in the unbinned data for
  clarity. Note the net skewing of $T_{HBR}$ to greater than unity for
  both apertures with no such trend existing in the simulated
  data. The dispersion of $T_{HBR}$ for the real data is also much
  larger than the dispersion of the simulated data.  }
\label{fig:ftx}
\end{center}
\end{figure}

The uncertainty associated with each value of $T_{HBR}$ is dominated
by the larger error in $T_{2.0-7.0}$, and on average, $\Delta
T_{2.0-7.0} \approx 2.3\Delta T_{0.7-7.0}$. This error interval
discrepancy naturally results from excluding the bulk of a cluster's
emission which occurs below 2 keV. While choosing a
temperature-sensitive cut-off energy for the hard-band (other than 2.0
keV) might maintain a more consistent error budget across our sample,
we do not find any systematic trend in $T_{HBR}$ or the associated
errors with cluster temperature.

\singlespacing
\begin{thesistable}{ccccccc}
\thesistablehead{Weighted averages for various apertures}{Weighted
averages for various apertures}{\multicolumn{1}{c}{} & \multicolumn{3}{l}{\dotfill Without
Core\dotfill} & \multicolumn{3}{l}{\dotfill With Core\dotfill}\\  & [0.7-7.0] & [2.0-7.0] & $T_{HBR}$ &
[0.7-7.0] & [2.0-7.0] & $T_{HBR}$\\ Aperture & keV & keV &  & keV & keV &}{tab:wavg}
R$_{2500}$ & 4.93$\pm 0.03$   & 6.24$\pm 0.07$   & 1.16$\pm 0.01$   & 4.47$\pm 0.02$ & 5.45$\pm 0.05$ & 1.13$\pm 0.01$\\
R$_{5000}$ & 4.75$\pm 0.02$   & 5.97$\pm 0.07$   & 1.14$\pm 0.01$   & 4.27$\pm 0.02$ & 5.29$\pm 0.05$ & 1.14$\pm 0.01$\\
Simulated  & 3.853$\pm 0.004$ & 4.457$\pm 0.009$ & 1.131$\pm 0.002$ & -       & -       & -\\
Control    & 4.208$\pm 0.003$ & 4.468$\pm 0.006$ & 1.002$\pm 0.001$ & -       & -       & -
\end{thesistable}
\doublespacing

%%%%%%%%%%%%%%%%%%%%%%%%
\subsection{Systematics}
\label{sec:ebandsys}
%%%%%%%%%%%%%%%%%%%%%%%%

In this study we have found the average value of $T_{HBR}$ is
significantly greater than one and that $\sigma_{HBR} >
\sigma_{\mathrm{control}}$, with the latter result being robust
against systematic uncertainties. As predicted by ME01, both of these
results are expected to arise naturally from the hierarchical
formation of clusters. But systematic uncertainty related to {\it
  Chandra} instrumentation or other sources could shift the average
value of $T_{HBR}$ one would get from ``perfect'' data. In this
section we consider some additional sources of uncertainty.
5A
First, the disagreement between {\it XMM-Newton} and {\it Chandra}
cluster temperatures has been noted in several independent studies,
i.e. \citet{2005ApJ...628..655V} and \citet{chanxmmdis}. But the source
of this discrepancy is not well understood and efforts to perform
cross-calibration between {\it XMM-Newton} and {\it Chandra} have thus
far not been conclusive. One possible explanation is poor calibration
of {\it Chandra} at soft X-ray energies which may arise from a
hydrocarbon contaminant on the High Resolution Mirror Assembly (HRMA)
similar in nature to the contaminant on the ACIS detectors
\citep{aciscontaminant}. We have assessed this possibility by looking
for systematic trends in $T_{HBR}$ with time or temperature, as such a
contaminant would most likely have a temperature and/or time
dependence.

As noted in \S\ref{sec:ebandtfresults} and seen in Figure \ref{fig:ftx}, we
find no systematic trend with temperature either for the full sample
or for a sub-sample of single-observation clusters with $> 75\%$ of
the observed flux attributable to the source (higher S/N observations
will be more affected by calibration uncertainty). Plotted in the
lower-left pane of Figures \ref{fig:sysr25} and \ref{fig:sysr50} is
$T_{HBR}$ versus time for single observation clusters (clusters with
multiple observations are fit simultaneously and any time effect would
be washed out) where the spectral flux is $> 75\%$ from the source. We
find no significant systematic trend in $T_{HBR}$ with time, which
suggests that if $T_{HBR}$ is affected by any contamination of {\it
  Chandra}'s HRMA, then the contaminant is most likely not changing
with time. Our conclusion on this matter is that the soft calibration
uncertainty is not playing a dominant role in our results.

Aside from instrumental and calibration effects, some other possible
sources of systematic error are S/N, redshift selection, Galactic
absorption, and metallicity. Also presented in Figures
\ref{fig:sysr25} and \ref{fig:sysr50} are three of these parameters
versus $T_{HBR}$ for $R_{2500-\mathrm{CORE}}$ and
$R_{5000-\mathrm{CORE}}$, respectively. The trend in $T_{HBR}$ with
redshift is expected as the 2.0/(1+$z$) keV hard-band lower boundary
nears convergence with the 0.7 keV broadband lower boundary at $z
\approx 1.85$. We find no systematic trends of $T_{HBR}$ with S/N or
Galactic absorption, which might occur if the skew in $T_{HBR}$ were a
consequence of poor count statistics, inaccurate Galactic absorption,
or very poor calibration. In addition, the ratio of $T_{HBR}$ for
$R_{2500-\mathrm{CORE}}$ to $R_{5000-\mathrm{CORE}}$ for every cluster
in our sample does not significantly deviate from unity. Our results
are robust to changes in aperture size.

\begin{figure}
\begin{center}
\includegraphics*[width=\textwidth, trim=0mm 0mm 0mm 0mm, clip]{eband_f4.eps}
\caption[Plot of several possible systematics for
  $R_{2500-\mathrm{CORE}}$ apertures.]{A few possible sources of
  systematic uncertainty vs. $T_{HBR}$ calculated for the
  $R_{2500-\mathrm{CORE}}$ apertures (\ebandnuma\ clusters). Error
  bars have been omitted in several plots for clarity. The line of
  equality is shown as a dashed line in all panels. {\it{(Top left:)}}
  $T_{HBR}$ vs. redshift for the entire sample. The trend in $T_{HBR}$
  with redshift is expected as the $T_{2.0-7.0}$ lower boundary nears
  convergence with the $T_{0.7-7.0}$ lower boundary at $z \approx
  1.85$. Weighted values of $T_{HBR}$ are consistent with unity
  starting at $z \sim 0.6$.  {\it{(Top right:)}} $T_{HBR}$
  vs. percentage of spectrum flux which is attributed to the
  source. We find no trend with signal-to-noise which suggests
  calibration uncertainty not is playing a major role in our results.
  {\it{(Middle left:)}} $T_{HBR}$ vs. Galactic column density. We find
  no trend in absorption which would result if $N_{H}$ values are
  inaccurate or if we had improperly accounted for local soft
  contamination.  {\it{(Middle right:)}} $T_{HBR}$ vs. the deviation
  from unity in units of measurement uncertainty. Recall that we have
  used 90\% confidence ($1.6\sigma$) for our analysis.  {\it{(Bottom
      left:)}} $T_{HBR}$ plotted vs. observation start date. The
  plotted points are culled from the full sample and represent only
  clusters which have a single observation and where the spectral flux
  is $> 75\%$ from the source. We note no systematic trend with time.
  {\it{(Bottom right:)}} Ratio of {\it Chandra} temperatures derived
  in this work to {\it ASCA} temperatures taken from
  \citet{hornerthesis}. We note a trend of comparatively hotter {\it
    Chandra} temperatures for clusters $> 10$ keV, otherwise our
  derived temperatures are in good agreement with those of {\it ASCA}.
}
\label{fig:sysr25}
\end{center}
\end{figure}

\begin{figure}
\begin{center}
\includegraphics*[width=\textwidth, trim=0mm 0mm 0mm 0mm, clip]{eband_f5.eps}
\caption[Plot of several possible systematics for
  $R_{5000-\mathrm{CORE}}$ apertures.]{ Same as Fig. \ref{fig:sysr25}
  except using the $R_{5000-\mathrm{CORE}}$ apertures (\ebandnumb\ clusters).}
\label{fig:sysr50}
\end{center}
\end{figure}

Also shown in Figures \ref{fig:sysr25} and \ref{fig:sysr50} are the
ratios of {\it ASCA} temperatures taken from \citet{hornerthesis} to
{\it Chandra} temperatures derived in this work. The spurious point
below 0.5 with very large error bars is MS 2053.7-0449, which has a
poorly constrained {\it ASCA} temperature of
$10.03^{+8.73}_{-3.52}$. Our value of $\sim 3.5$ keV for this cluster
is in agreement with the recent work of
\citet{2008ApJS..174..117M}. Not all our sample clusters have an {\it
  ASCA} temperature, but a sufficient number (53) are available to
make this comparison reliable. Apertures used in the extraction of
{\it ASCA} spectra had no core region removed and were substantially
larger than $R_{2500}$. {\it ASCA} spectra were also fit over a
broader energy range (0.6-10 keV) than we use here. Nonetheless, our
temperatures are in good agreement with those from {\it ASCA}, but we
do note a trend of comparatively hotter {\it Chandra} temperatures for
$T_{Chandra} > 10$ keV. For both apertures, the clusters with
$T_{Chandra} > 10$ keV are Abell 1758, Abell 2163, Abell 2255, and RX
J1347.5-1145. Based on this trend, we test excluding the hottest
clusters ($T_{Chandra} > 10$ keV where {\it ASCA} and {\it Chandra}
disagree) from our sample. The mean temperature ratio for
$R_{2500-\mathrm{CORE}}$ remains $1.16$ and the error of the mean
increases from $\pm 0.014$ to $\pm 0.015$, while for
$R_{5000-\mathrm{CORE}}$ $T_{HBR}$ increases by a negligible $0.9\%$
to $1.15\pm 0.014$. Our results are not being influenced by the
inclusion of hot clusters.

\begin{figure}
\begin{center}
\includegraphics*[width=\textwidth, trim=0mm 0mm 0mm 0mm, clip]{eband_f6.eps}
\caption[$T_{HBR}$ vs. best-fit metallicity]{$T_{HBR}$ as a function
  of metal abundance for $R_{2500-\mathrm{CORE}}$,
  $R_{5000-\mathrm{CORE}}$, and the control sample (see discussion of
  control sample in \S\ref{sec:ebandsimulated}). Error bars are omitted for
  clarity. The dashed-line represents the linear best-fit using the
  bivariate correlated error and intrinsic scatter (BCES) method of
  \citet{1996ApJ...470..706A} which takes into consideration errors on
  both $T_{HBR}$ and abundance when performing the fit. We note no
  trend in $T_{HBR}$ with metallicity (the apparent trend in the top
  panel is not significant) and also note the low dispersion in the
  control sample relative to the observations. The striation of
  abundance arises from our use of two decimal places in recording the
  best-fit values from {\textsc{XSPEC}}.  }
\label{fig:metal}
\end{center}
\end{figure}

The temperature range of the clusters we have analyzed ($T_X \sim
3-20$ keV) is broad enough that the effect of metal abundance on the
inferred spectral temperature is clearly not negligible. In Figure
\ref{fig:metal} we have plotted $T_{HBR}$ versus abundance in solar
units. Despite covering a factor of seven in temperature and metal
abundances ranging from $Z/Z_{\odot} \approx 0$ to solar, we find no
trend in $T_{HBR}$ with metallicity. The slight trend in the
$R_{2500-\mathrm{CORE}}$ aperture (Fig.  \ref{fig:metal}, top) is
insignificant, while there is no trend at all in the control sample or
$R_{5000-\mathrm{CORE}}$ aperture.

%%%%%%%%%%%%%%%%%%%%%%%%%%%%%%%%%%%%%%%%%%%%%%%%%%%%
\subsection{Using $T_{HBR}$ as a Test of Relaxation}
\label{sec:ebandrelax}
%%%%%%%%%%%%%%%%%%%%%%%%%%%%%%%%%%%%%%%%%%%%%%%%%%%%

%%%%%%%%%%%%%%%%%%%%%%%%%%%%%%%%%%%%%%%%%%%%%%
\subsubsection{Cool Core Versus Non-Cool Core}
\label{sec:ebandccncc}
%%%%%%%%%%%%%%%%%%%%%%%%%%%%%%%%%%%%%%%%%%%%%%

As discussed in \ref{sec:ebandintro}, ME01 gives us reason to believe the
observed skewing of $T_{HBR}$ to greater than unity is related to the
dynamical state of a cluster. It has also been suggested that the
process of cluster formation and relaxation may robustly result in the
formation of a cool core \citep{2006ApJ...640..673O,
  2008ApJ...675.1125B}. Depending on classification criteria,
completeness, and possible selection biases, studies of flux-limited
surveys have placed the prevalence of cool cores at $34\%-60\%$
\citep{white97, 1998MNRAS.298..416P, 2005MNRAS.359.1481B,
  2007A&A...466..805C}. It has thus become rather common to divide up
the cluster population into two distinct classes, cool core (CC) and
non-cool core (NCC), for the purpose of discussing their different
formation or merger histories. We thus sought to identify which
clusters in our sample have cool cores, which do not, and if the
presence or absence of a cool core is correlated with $T_{HBR}$. It is
very important to recall that we excluded the core during spectral
extraction and analysis.

To classify the core of each cluster, we extracted a spectrum for the
50 kpc region surrounding the cluster center and then defined a
temperature decrement,
\begin{equation}
T_{\mathrm{dec}} = T_{50}/T_{\mathrm{cluster}}
\label{eqn:tdec}
\end{equation}
where $T_{50}$ is the temperature of the inner 50 kpc and
$T_{\mathrm{cluster}}$ is either the $R_{2500-\mathrm{CORE}}$ or
$R_{5000-\mathrm{CORE}}$ temperature. If $T_{\mathrm{dec}}$ was
2$\sigma$ less than unity, we defined the cluster as having a CC,
otherwise the cluster was defined as NCC. We find CCs in 35\% of our
sample and when we lessen the significance needed for CC
classification from 2$\sigma$ to 1$\sigma$, we find 46\% of our sample
clusters have CCs. It is important to note that the frequency of CCs
in our study is consistent with other more detailed studies of CC/NCC
populations.

When fitting for $T_{50}$, we altered the method outlined in
\S\ref{sec:ebandfitting} to use the {\textsc{XSPEC}} modified Cash
statistic \citep{1979ApJ...228..939C}, {\textsc{cstat}}, on ungrouped
spectra. This choice was made because the distribution of counts per
bin in low count spectra is not Gaussian but instead Poisson. As a
result, the best-fit temperature using $\chi^2$ is typically cooler
\citep{1989ApJ...342.1207N, 2007A&A...462..429B}. We have explored
this systematic in {\bfseries\em{all}} of our fits and found it to be
significant only in the lowest count spectra of the inner 50 kpc
apertures discussed here. But, for consistency, we fit all inner 50
kpc spectra using the modified Cash statistic.

With each cluster core classified, we then took cuts in $T_{HBR}$ 
and asked how many CC and NCC clusters were above these cuts. 
Figure \ref{fig:cc_ncc_bin} shows the normalized number of CC and NCC
clusters as a function of cuts in $T_{HBR}$. If $T_{HBR}$ were
insensitive to the state of the cluster core, we expect, for normally
distributed $T_{HBR}$ values, to see the number of CC and NCC clusters
decreasing in the same way. However, the number of CC clusters falls
off more rapidly than the number of NCC clusters. If the presence of a
CC is indicative of a cluster's advancement towards complete
virialization, then the significantly steeper decline in the percent
of CC clusters versus NCC as a function of increasing $T_{HBR}$
indicates higher values of $T_{HBR}$ are associated with a less
relaxed state. This result is insensitive to our choice of
significance level in the core classification, i.e. the result is the
same whether using $1\sigma$ or $2\sigma$ significance when
considering $T_{\mathrm{dec}}$.

\begin{figure}
\begin{center}
\includegraphics*[width=\textwidth, trim=15mm 10mm 0mm 0mm, clip]{eband_f7.eps}
\caption[Number of cool and non-cool clusters as a function of
  $T_{HBR}$]{Normalized number of CC and NCC clusters as a function of
  cuts in $T_{HBR}$. There are \ebandnuma\ clusters plotted in the top panel
  and \ebandnumb\ in the bottom panel. We have defined a cluster as having a
  CC when the temperature for the 50 kpc region around the cluster
  center divided by the temperature for $R_{2500-\mathrm{CORE}}$, or
  $R_{5000-\mathrm{CORE}}$, was less than one at the $2\sigma$
  level. We then take cuts in $T_{HBR}$ at the $1\sigma$ level and ask
  how many CC and NCC clusters are above these cuts. The number of CC
  clusters falls off more rapidly than NCC clusters in this
  classification scheme suggesting higher values of $T_{HBR}$ prefer
  less relaxed systems which do not have cool cores. This result is
  insensitive to our choice of significance level in both the core
  classification and $T_{HBR}$ cuts.  }
\label{fig:cc_ncc_bin}
\end{center}
\end{figure}

Because of the CC/NCC definition we selected, our identification of
CCs and NCCs was only as robust as the errors on $T_{50}$ allowed. One
can thus ask the question, did our definition bias us towards finding
more NCCs than CCs? To explore this question we simulated 20 spectra
for each observation following the method outlined in
\S\ref{sec:ebandsimulated} for the control sample but using the inner 50
kpc spectral best-fit values as input. For each simulated spectrum, we
calculated a temperature decrement (eq. \ref{eqn:tdec}) and
re-classified the cluster as having a CC or NCC. Using the new set of
mock classifications we assigned a reliability factor, $\psi$, to each
real classification, which is simply the fraction of mock
classifications which agree with the real classification. A value of
$\psi = 1.0$ indicates complete agreement, with $\psi = 0.0$
indicating no agreement. When we removed clusters with $\psi < 0.9$
and repeated the analysis above, we found no significant change in the
trend of a steeper decrease in the relative number of CC versus NCC
clusters as a function of $T_{HBR}$.

Recall that the coolest ICM gas is being redshifted out of the
observable band as $z$ increases and becomes a significant effect at
$z \geq 0.6$ (\S\ref{sec:ebandsimulated}). Thus, we are likely not detecting
``weak'' CCs in the highest redshift clusters of our sample and
consequently these cores are classified as NCCs and are artificially
increasing the NCC population. When we excluded the 14 clusters at $z
\geq 0.6$ from this portion of our analysis and repeated the
calculations, we found no significant change in the results.

%%%%%%%%%%%%%%%%%%%%%%%%%%%%%%%%%%%%%%%%%%
\subsubsection{Mergers Versus Nonmergers}
\label{sec:ebandmerge}
%%%%%%%%%%%%%%%%%%%%%%%%%%%%%%%%%%%%%%%%%%

Looking for a correlation between cluster relaxation and a skewing in
$T_{HBR}$ was the primary catalyst of this work. The result that
increasing values of $T_{HBR}$ are more likely to be associated with
clusters harboring non-cool cores gives weight to that
hypothesis. But, the simplest relation to investigate is if $T_{HBR}$
is preferentially higher in merger systems. Thus, we now discuss
clusters with the highest significant values of $T_{HBR}$ and attempt
to establish, via literature based results, the dynamic state of these
systems.

The subsample of clusters on which we focus have a $T_{HBR} > 1.1$ at
90\% confidence for both their $R_{2500-\mathrm{CORE}}$ and
$R_{5000-\mathrm{CORE}}$ apertures. These clusters are listed in Table
\ref{tab:tf11} and are sorted by the lower limit of $T_{HBR}$. Shown
in Figure \ref{fig:ftx_tx} is a plot of $T_{HBR}$ versus $T_{0.7-7.0}$
for all the clusters in our sample. The clusters discussed in this
section are shown as green triangles and black stars. The clusters
with only a $R_{5000-\mathrm{CORE}}$ analysis are listed separately at
the bottom of the table. All 33 clusters listed have a core
classification of $\psi > 0.9$ (see \S\ref{sec:ebandccncc}). The choice of
the $T_{HBR} > 1.1$ threshold was arbitrary and intended to limit the
number of clusters to which we pay individual attention, but which is
still representative of mid- to high-$T_{HBR}$ values. Only two
clusters -- Abell 697 and MACS J2049.9-3217 -- do not have a $T_{HBR}
> 1.1$ in one aperture and not the other. In both cases although, this
was the result of the lower boundary narrowly missing the cut, but
both clusters still have $T_{HBR}$ significantly greater than unity.

For those clusters which have been individually studied, they are
listed as mergers based on the conclusions of the literature authors
(cited in Table \ref{tab:tf11}). Many different techniques were used
to determine if a system is a merger: bimodal galaxy velocity
distributions, morphologies, highly asymmetric temperature
distributions, ICM substructure correlated with subclusters, or
disagreement of X-ray and lensing masses. From Table \ref{tab:tf11} we
can see clusters exhibiting the highest significant values of
$T_{HBR}$ tend to be ongoing or recent mergers. At the 2$\sigma$
level, we find increasing values of $T_{HBR}$ favor merger systems
with NCCs over relaxed, CC clusters. It appears mergers have left a
spectroscopic imprint on the ICM which was predicted by ME01 and which
we observe in our sample.

\begin{figure}
\begin{center}
\includegraphics*[width=\textwidth, trim=15mm 10mm 0mm 0mm, clip]{eband_f8.eps}
\caption[Plot of $T_{HBR}$ vs. broadband temperatures color-coded for
  different cluster types]{$T_{HBR}$ plotted against $T_{0.7-7.0}$ for
  the $R_{2500-\mathrm{CORE}}$ and $R_{5000-\mathrm{CORE}}$
  apertures. Note that the vertical scales for both panels are not the
  same. The top and bottom panels contain \ebandnuma\ and \ebandnumb\
  clusters, respectively. Only two clusters -- Abell 697 and MACS
  J2049.9-3217 -- do not have a $T_{HBR} > 1.1$ in one aperture and
  not the other. In both cases however, it was a result of narrowly
  missing the cut. The dashed lines are the lines of
  equivalence. Symbols and color coding are based on two criteria: (1)
  the presence of a CC and (2) the value of $T_{HBR}$. Black stars (6
  in the top panel; 7 in the bottom) are clusters with a CC and
  $T_{HBR}$ significantly greater than 1.1. Green upright-triangles
  (21 in the top; 27 in the bottom) are NCC clusters with $T_{HBR}$
  significantly greater than 1.1. Blue down-facing triangles (49 top;
  60 bottom) are CC clusters and red squares (90 top; 98 bottom) are
  NCC clusters. We have found most, if not all, of the clusters with
  $T_{HBR} \gtrsim 1.1$ are merger systems. Note that the cut at
  $T_{HBR} > 1.1$ is arbitrary and there are more merger systems in
  our sample then just those highlighted in this figure. However it is
  rather suggestive that clusters with the highest values of $T_{HBR}$
  appear to be merging systems.  }
\label{fig:ftx_tx}
\end{center}
\end{figure}

Of the 33 clusters with $T_{HBR}$ significantly $> 1.1$, only 7 have
CCs. Three of those -- MKW3S, 3C 28.0, and RX J1720.1+2638 -- have
their apertures centered on the bright, dense cores in confirmed
mergers. Two more clusters -- Abell 2384 and RX J1525+0958 -- while
not confirmed mergers, have morphologies which are consistent with
powerful ongoing mergers. Abell 2384 has a long gas tail extending
toward a gaseous clump which we assume has recently passed through the
cluster. RXJ1525 has a core shaped like a rounded arrowhead and is
reminiscent of the bow shock seen in 1E0657-56. Abell 907 has no signs
of being a merger system, but the highly compressed surface brightness
contours to the west of the core are indicative of a prominent cold
front, a tell-tale signature of a subcluster merger event
\citep{2007PhR...443....1M}. Abell 2029 presents a very interesting
and curious case because of its seemingly high state of relaxation and
prominent cool core. There are no complementary indications it has
experienced a merger event. Yet its core hosts a wide-angle tail radio
source. It has been suggested that such sources might be attributable
to cluster merger activity \citep{2000MNRAS.311..649S}. Moreover, the
X-ray isophotes to the west of the bright, peaked core are slightly
more compressed and may be an indication of past gas sloshing
resulting from the merger of a small subcluster. Both of these
features have been noted previously, specifically by
\citet{2004ApJ...616..178C, 2005xrrc.procE7.08C}. We suggest the
elevated $T_{HBR}$ value for this cluster lends more weight to the
argument that A2029 has indeed experienced a merger recently, but how
long ago we do not know.

The remaining systems we could not verify as mergers -- RX
J0439.0+0715, MACS J2243.3-0935, MACS J0547.0-3904, Zwicky 1215, MACS
J2311+0338, Abell 267, and NGC 6338 -- have NCCs and X-ray
morphologies consistent with an ongoing or post-merger scenario. Abell
1204 shows no signs of recent or ongoing merger activity; however, it
resides at the bottom of the arbitrary $T_{HBR}$ cut, and as evidenced
by Abell 401 and Abell 1689, exceptional spherical symmetry is no
guarantee of relaxation. Our analysis here is partially at the mercy
of morphological assessment, and only a more stringent study of a
carefully selected subsample or analysis of simulated clusters can
better determine how closely correlated $T_{HBR}$ is with the timeline
of merger events.

%%%%%%%%%%%%%%%%%%%%%%%%%%%%%%%%%
\section{Summary and Conclusions}
\label{sec:ebandsummary}
%%%%%%%%%%%%%%%%%%%%%%%%%%%%%%%%%

We have explored the band dependence of the inferred X-ray temperature
of the ICM for \ebandnumb\ well-observed ($N_{counts} > 1500$) clusters
of galaxies selected from the {\it Chandra} Data Archive.

We extracted spectra from the annulus between $R=70$ kpc and
$R=R_{2500}$, $R_{5000}$ for each cluster. We compared the X-ray
temperatures inferred for single-component fits to global spectra when
the energy range of the fit was 0.7-7.0 keV (broad) and when the
energy range was $2.0/(1+z)$-7.0 keV (hard). We found that, on
average, the hard-band temperature is significantly higher than the
broadband temperature. For the $R_{2500-\mathrm{CORE}}$ aperture we
measured a weighted average of $T_{HBR} = 1.16$ with $\sigma = \pm
0.10$ and $\sigma_{mean} = \pm 0.01$ for the $R_{5000-\mathrm{CORE}}$
aperture, and $T_{HBR} = 1.14$ with $\sigma = \pm 0.12$ and
$\sigma_{mean} = \pm 0.01$. We also found no systematic trends in the
value of $T_{HBR}$, or the dispersion of $T_{HBR}$, with S/N,
redshift, Galactic absorption, metallicity, observation date, or
broadband temperature.

In addition, we simulated an ensemble of 12,765 spectra which
contained observation-specific and idealized two-temperature component
models, plus a control sample of single-temperature models. From
analysis of these simulations we found the statistical fluctuations
for a single temperature model are inadequate to explain the
significantly different $T_{0.7-7.0}$ and $T_{2.0-7.0}$ we measure in
our sample. We also found that the observed scatter, $\sigma_{HBR}$,
is consistent with the presence of unresolved cool ($T_X < 2.0$ keV)
gas contributing a minimum of $>10\%$ of the total emission. The
simulations also show the measured observational scatter in $T_{HBR}$
is greater than the statistical scatter, $\sigma_{control}$. These
results are consistent with the process of hierarchical cluster
formation.

Upon further exploration, we found that $T_{HBR}$ is enhanced
preferentially for clusters which are known merger systems and for
clusters without cool cores. Clusters with temperature decrements in
their cores (known as cool-core clusters) tend to have best-fit
hard-band temperatures that are consistently closer to their best-fit
broadband temperatures. The correlation of $T_{HBR}$ with the type of
cluster core is insensitive to our choice of classification scheme and
is robust against redshift effects. Our results qualitatively support
the finding by ME01 that the temperature ratio, $T_{HBR}$, might
therefore be useful for statistically quantifying the degree of
cluster relaxation/virialization.

An additional robust test of the ME01 finding should be made with
simulations by tracking $T_{HBR}$ during hierarchical assembly of a
cluster. If $T_{HBR}$ is tightly correlated with a cluster's degree of
relaxation, then it, along with other methods of substructure measure,
may provide a powerful metric for predicting (and therefore reducing)
a cluster's deviation from mean mass-scaling relations. The task of
reducing scatter in scaling relations will be very important if we are
to reliably and accurately measure the mass of clusters.

%%%%%%%%%%%%%%%%%%%%%%%%%%
\section{Acknowledgments}
%%%%%%%%%%%%%%%%%%%%%%%%%%

K. W. C. was supported in this work by the National Aeronautics and
Space Administration through {\it Chandra} X-Ray Observatory Archive
grants AR-6016X and AR-4017A, with additional support from a start-up
grant for Megan Donahue from Michigan State University. M. D.  and
Michigan State University acknowledge support from the NASA LTSA
program NNG-05GD82G. G. M. V. thanks NASA for support through theory
grant NNG-04GI89G. The {\it Chandra} X-ray Observatory Center is
operated by the Smithsonian Astrophysical Observatory for and on
behalf of the National Aeronautics Space Administration under contract
NAS8-03060. This research has made use of software provided by the
{\it Chandra} X-ray Center (CXC) in the application packages \ciao,
{\textsc{ChIPS}}, and {\textsc{Sherpa}}. We thank Alexey Vikhlinin for
helpful insight and expert advice. K. W. C. also thanks attendees of
the ``Eight Years of Science with {\it Chandra} Calibration Workshop''
for stimulating discussion regarding {\it XMM}-{\it Chandra}
cross-calibration. K. W. C. especially thanks Keith Arnaud for
personally providing support and advice for mastering
{\textsc{XSPEC}}. This research has made use of the NASA/IPAC
Extragalactic Database (NED), which is operated by the Jet Propulsion
Laboratory, California Institute of Technology, under contract with
the National Aeronautics and Space Administration. This research has
also made use of NASA's Astrophysics Data System. {\it ROSAT} data and
software were obtained from the High Energy Astrophysics Science
Archive Research Center (HEASARC), provided by NASA's Goddard Space
Flight Center.


%%%%%%%%%%%%%%%%%%%%%%
% Reprint cover page %
%%%%%%%%%%%%%%%%%%%%%%

\newpage
\parbox[c][0.9\textheight][c]{\linewidth}{
\begin{center}
Chapter Three
\end{center}
\begin{spacing}{1.1}
Cavagnolo, Kenneth W., Donahue, Megan, Voit, G. Mark, Sun, Ming
(2008). Intracluster Medium Entropy Profiles For A Chandra Archival
Sample of Galaxy Clusters. {\underline{The Astrophysical Journal
Supplement Series}}. arXiv eprint 0902.1802.\\
\end{spacing}
}

%%%%%%%%%%%%%%%%%%%%%%%%%%%%%%%%%%%%%%%%%%%%%%%%%%%%%%%%%%%%%%%%%%%%%%%%%%%%%%%%%%%%%%%%%%%%%%%
\chapter{Intracluster Medium Entropy Profiles For A Chandra Archival Sample of Galaxy Clusters}
\label{ch:ent_supp}
%%%%%%%%%%%%%%%%%%%%%%%%%%%%%%%%%%%%%%%%%%%%%%%%%%%%%%%%%%%%%%%%%%%%%%%%%%%%%%%%%%%%%%%%%%%%%%%

\input{ent_supp3}

%%%%%%%%%%%%%%%%%%%%%%
% Reprint cover page %
%%%%%%%%%%%%%%%%%%%%%%

\newpage
\parbox[c][0.9\textheight][c]{\linewidth}{
\begin{center}
Chapter Four
\end{center}
\begin{spacing}{1.1}
Cavagnolo, Kenneth W., Donahue, Megan, Voit, G. Mark, Sun, Ming
(2008). An Entropy Threshold for Strong \halpha\ and Radio Emission in
the Cores of Galaxy Clusters. {\underline{The Astrophysical
Journal Letters}}. 683:107-110.\\
\end{spacing}
}

%%%%%%%%%%%%%%%%%%%%%%%%%%%%%%%%%%%%%%%%%%%%%%%%%%%%%%%%%%%%%%%%%%%%
\chapter{An Entropy Threshold for Strong \halpha\ and Radio Emission
in the Cores of Galaxy Clusters}
\label{ch:harad}
%%%%%%%%%%%%%%%%%%%%%%%%%%%%%%%%%%%%%%%%%%%%%%%%%%%%%%%%%%%%%%%%%%%%

\input{haradent}

%%%%%%%%%%%%%%%%%%
\chapter{Summary}
\label{ch:summary}
%%%%%%%%%%%%%%%%%%

\documentclass[11pt]{article}
\usepackage[colorlinks=true,linkcolor=blue,urlcolor=blue]{hyperref}
\usepackage{subfig,epsfig,colortbl,graphics,graphicx,wrapfig,amssymb}
\usepackage{macros_cavag}
\pagestyle{myheadings}
\font\cap=cmcsc10
\setlength{\topmargin}{-0.2in}
\setlength{\oddsidemargin}{-0.1in}
\setlength{\evensidemargin}{0.in}
\setlength{\headheight}{0.1in}
\setlength{\headsep}{0.25in}
\setlength{\topskip}{0.1in}
\setlength{\textwidth}{6.5in}
\setlength{\textheight}{9.5in}

\markright{K.W. Cavagnolo Summary}

\begin{document}
\begin{center}
\textbf{Summary of Past Research and Future Interests}\\
\end{center}

The general process of galaxy cluster formation through hierarchical
merging is well understood, but many details, such as the impact of
feedback sources on the cluster environment and radiative cooling in
the cluster core, are not. My thesis research has focused on studying
these details via X-ray properties of the ICM in clusters of
galaxies. I have paid particular attention to ICM entropy
distribution, the process of virialization, and the role of AGN
feedback in shaping large scale cluster properties.

\subsection*{Mining the CDA}

My thesis makes use of a 350 observation sample (276 clusters; 11.6
Msec) taken from the {\it Chandra} archive. This massive
undertaking necessitated the creation of a robust reduction and
analysis pipeline which 1) interacts with mission specific software,
2) utilizes analysis tools (i.e. {\tt{XSPEC}}, {\tt{IDL}}), 3)
incorporates calibration and software updates, and 4) is highly
automated. Because my pipeline is written in a very general manner,
adding pre-packaged analysis tools from missions such as
{\textit{XMM}}, {\textit{Spitzer}}, and {\textit{VLA}} will be
straightforward. Most importantly, my pipeline deemphasizes data
reduction and accords me the freedom to move quickly into an analysis
phase and generating publishable results.

\subsection*{Quantifying Cluster Virialization}

The normalization, shape, and evolution of the cluster mass function
are useful for measuring cosmological parameters. Cluster evolution
tests the effect of dark matter and dark energy on the evolution of
dark matter halos, and therefore provides a complementary and
distinct constraint on cosmological parameters to those tests which
constrain them geometrically (e.g. supernovae and baryon acoustic
oscillations). If we could identify a parameter possibly reflecting
the degree of relaxation in the cluster we could improve the utility
of clusters as cosmological probes by parameterizing and reducing the
scatter in mass-observable scaling relations.

One study that examined how relaxation affects the observable
properties of clusters was conducted by Mathiesen and Evrard
2001. They found that most clusters which had experienced a recent
merger were cooler than the cluster mass-observable scaling relations
predicted. They attributed this to the presence of cool,
spectroscopically unresolved accreting subclusters.

I have followed up their work by studying the bandpass dependence in
determining X-ray temperatures and what this dependence tells us about
the virialization state of a cluster. The ultimate goal of this
project was to find an aspect-independent measure for a cluster's
dynamic state. I thus investigated the net temperature skew of the
hard-band (2.0-7.0 keV) and full-band (0.7-7.0 keV) temperature
ratio. I have found this temperature ratio is statistically connected
to mergers and the presence of cool cores. Having confirmed the
predicted effect, the next step is to make a comparison to the
predicted distribution of temperature ratios and their relationship to
putative cool lumps and/or non-thermal soft X-ray emission in cluster
simulations.

\subsection*{Cluster Feedback and ICM Entropy}

The picture of the ICM entropy-feedback connection emerging from my
work suggests cluster radio luminosity and H$\alpha$ emission are
anti-correlated with cluster central entropy. Following my analysis of 169
cluster radial entropy profiles (Fig. \ref{fig:splots}) I have found
an apparent bimodality in the distribution of central
entropy and central cooling times (Fig. \ref{fig:tcool}) which is
likely related to AGN feedback (and to a lesser extent, mergers). I
have also found that clusters with central entropy $\leq 20$ keV
cm$^2$ show signs of star formation (Fig. \ref{fig:ha}) and AGN
activity (Fig. \ref{fig:rad}) while clusters above this threshold
unilaterally do not have star formation and exhibit diminished AGN
radio feedback. This entropy level is auspicious as it coincides with
the Field length, $\lambda_F$, (assuming reasonable suppression from
magnetic fields) at which thermal conduction can stabilize a cluster
core against further cooling and gas condensation. It is possible my
work has opened a window to solving a long-standing problem in massive
galaxy formation (and truncation): how are ICM gas properties coupled
to feedback mechanisms such that the system becomes self-regulating?
But my thesis has also highlighted some unresolved and new
issues.

What is the origin of the bimodality in $K_0$ and $t_{cool}$?
What role is star formation playing in the feedback cycle of clusters?
How is energy generated on the parsec scale from a SMBH deposited
uniformly over volumes which are orders of magnitude larger? There are
also exciting theoretical cluster feedback model developments on the
horizon which will need observational investigation. Developments such
as: how exactly are AGN fueled -- through a combination of hot/cold
accretion, mergers, and consumption of low entropy gas via cooling; or
is there a universal mode underlying all these processes? Does
accretion of the hot ICM/ISM proceed via Bondi-eque flows or is it
more like Eddington accretion? What is the efficiency of accretion and
is energy return from a SMBH really the presumed $\sim 10\%$? Why do
we see steep metallicity gradients in the ICM/ISM when some amount of
turbulent mixing should take place? How is feedback energy distributed
symmetrically throughout the ICM?

\subsection*{Future Work}

Looking ahead, the natural extension of my thesis is to further study
questions regarding cluster environments and their impact on galaxy
formation. I'd also like to participate in analyzing X-ray follow-up
observations for clusters found using large SZE surveys. More
specifically, I'd like to use these samples to measure the evolution
of the cluster mass function as a direct means of breaking the
degeneracy between $\Omega_M$ and $\sigma_8$. Combined with
complementary surveys (specifically those using the SZE which will
yield tens of thousands of cluster candidates) X-ray surveys will help
further constrain the fundamental parameters defining the current
cosmological model.

But, the detailed analysis of the cluster population at redshifts
greater than z $\sim 1$ will be very difficult, and establishing the
self similar model as a reliable tool for calibrating the cluster mass
function will lead to better studies of hierarchical structure
formation and dark energy. In addition, if we are to use SZE as
effectively as desired SZE flux must be calibrated to accurately predict
cluster mass. But even calibration is not enough, we must also
understand the scatter in scaling relations. And to this end one needs
two components: verification of cluster candidates and methods for
quantifying deviation from mean mass-scaling relations. But the simple
application of existing metrics which have been calibrated to low-z
samples or high resolution simulations may begin to breakdown as
spatial and spectroscopic information is reduced at high
redshifts. There is the even worse possibility that scaling relations
evolve with redshift which will present a number of technical
difficulties all their own (e.g. covariance and time evolving
scatter). I look forward to being a part of generating new, novel
solutions to these problems.

With potentially enumerable, unbiased samples of clusters emerging
from SZE surveys and low flux, all-sky X-ray surveys, the entropy
distribution and signatures of feedback culled from these samples
could tell us a great deal about the evolution of clusters and galaxy
formation. Many questions remain unanswered in this area, such as:
What are the micro-physics of ICM heating, including the
thermalization of mechanical work done by bubbles and the effect of
non-thermal sources like cosmic rays. How prevalent are cold fronts
and can they be used as an indicator of merger activity and onset of
feedback? Also of interest are how accretion onto the cD SMBH is
regulated by large-scale ICM properties, what the AGN energy injection
function looks like, and how it correlates with cluster
environment. It will also be useful to have a low-scatter, universal
relation between jet power and radio power -- a tool which can then be
directly applied to understanding both cluster feedback and could
possibly be useful in SZE studies.

\clearpage
\begin{figure}[t]
    \begin{minipage}[t]{0.5\linewidth}
        \centering
        \includegraphics*[width=\textwidth, trim=28mm 8mm 30mm 10mm, clip]{splots}
        \caption{\small Radial entropy profiles of 169 clusters of
        galaxies in my thesis sample. The observed range of $K_0 \lesssim
        40$ keV cm$^2$ is consistent with models of episodic AGN
        heating. Color coding indicates global cluster temperature (in keV)
        derived from core excised apertures of size R$_{2500}$.}
        \label{fig:splots}
    \end{minipage}
    \hspace{0.1in}
    \begin{minipage}[t]{0.5\linewidth}
        \centering
        \includegraphics*[width=\textwidth, trim=28mm 8mm 30mm 10mm, clip]{tcool}
        \caption{\small Distribution of central cooling times for 169
        clusters in my thesis sample. The peak in the range of cooling
        times (several hundred Myrs) is consistent with inferred AGN
        duty cycles of both weak ($\sim 10^{40-50}$ ergs) and strong ($\sim
        10^{60}$ ergs) outbursts. However, note the distinct gap at $0.6-1$
        Gyr. An explanation for this bimodality does not currently exist.}
        \label{fig:tcool}
    \end{minipage}
    \hspace{0.1cm}
    \begin{minipage}[t]{0.5\linewidth}
        \centering
        \includegraphics*[width=\textwidth, trim=28mm 8mm 30mm 10mm, clip]{ha}
        \caption{\small Central entropy plotted against H$\alpha$
        luminosity. Orange dots are detections and black boxes with
        arrows are non-detection upper-limits. Notice the characteristic entropy threshold for star
        formation of $K_0 \lesssim 20$ keV cm$^2$. This is also the entropy scale at
        which conduction no longer balances radiative cooling and condensation
        of low entropy gas onto a cD can proceed.}
        \label{fig:ha}
    \end{minipage}
    \hspace{0.1in}
    \begin{minipage}[t]{0.5\linewidth}
        \centering
        \includegraphics*[width=\textwidth, trim=28mm 8mm 30mm 10mm, clip]{rad}
        \caption{\small Central entropy plotted against NVSS or PKS radio
        luminosity. Orange dots are detections and black boxes with
        arrows are non-detection upper-limits. There appears to be a dichotomy which might be related to AGN
        fueling mechanisms: AGN which are feed via low entropy gas, and the
        smattering of points at $K_0 > 50$ keV cm$^2$ which are likely
        fueled by mergers or have X-ray coronae which promote ICM cooling.}
        \label{fig:rad}
    \end{minipage}
\end{figure}
 
\end{document}


%%%%%%%%%%%%%%%%
%% APPENDICES %%
%%%%%%%%%%%%%%%%

\appendix
%%%%%%%%%%%%%%%%%%%%%%%%%%%%%%%%%%%%%%%%%%%%%%%%
\chapter{Tables cited in Chapter \ref{ch:eband}}
%%%%%%%%%%%%%%%%%%%%%%%%%%%%%%%%%%%%%%%%%%%%%%%%

\begin{center}
{\bf{Table \ref{tab:sample} Notes}}\\
\end{center}
A ($\ddagger$) indicates a cluster analyzed within R$_{5000}$
only. Italicized cluster names indicate a cluster which was excluded
from our analysis (discussed in \S\ref{sec:ebandfitting}). For clusters
with multiple observations, the X-ray centers differ by $< 0.5$ kpc.
Col. (1) Cluster name; col. (2) CDA observation identification number;
col. (3) R.A. of cluster center; col. (4) Dec. of cluster center;
col. (5) nominal exposure time; col. (6) observing mode; col. (7) CCD
location of centroid; col. (8) redshift; col. (9) bolometric
luminosity.

\begin{center}
{\bf{Table \ref{tab:tf11} Notes}}\\
\end{center}
Clusters ordered by lower limit of $T_{HBR}$. Listed $T_{HBR}$ values
are for the $R_{2500-\mathrm{CORE}}$ aperture, with the exception of
the ``$R_{5000-\mathrm{CORE}}$ Only'' clusters listed at the end of
the table. Excluding the ``$R_{5000-\mathrm{CORE}}$ Only'' clusters,
all clusters listed here had $T_{HBR}$ significantly greater than 1.1
and the same core classification for both the $R_{2500-\mathrm{CORE}}$
and $R_{5000-\mathrm{CORE}}$ apertures. Numbered references given in
table: [1] \citet{1994ApJS...94..583G}, [2] \citet{2003ApJ...593..291K},
[3] \citet{2005ChJAA...5..126Y}, [4] \citet{1998ApJ...503...77M}, [5]
\citet{2006Sci...314..791B}, [6] \citet{1990ApJS...72..715T}, [7]
\citet{2004ApJ...607..190A}, [8] \citet{1995ApJ...446..583B}, [9]
\citet{1997A&A...317..432F}, [10] \citet{1997ApJ...490...56G}, [11]
\citet{2002ApJS..139..313D}, [12] \citet{2005MNRAS.359..417S}, [13]
\citet{1982ApJ...255L..17G}, [14] \citet{2004ApJ...610L..81H}, [15]
\citet{2004ApJ...614..692Y}, [16] \citet{2003A&A...408...57M}, [17]
\citet{2000ApJ...540..726G}, [18] \citet{1998ApJ...496L...5T}, [19]
\citet{1999AcA....49..403K}, [20] \citet{2001ApJ...555..205M}, [21]
\citet{2001A&A...379..807G}, [22] \citet{1998MNRAS.301..609B}, [23]
\citet{2005ApJ...619..161G}, [24] \citet{1996ApJ...472L..17M}, [25]
\citet{2001ASPC..251..474O}, [26] \citet{2000ApJ...534L..43M}, [27]
\citet{2004ApJ...616..178C}, [28] \citet{2005xrrc.procE7.08C}, [29] this
work.

\begin{center}
{\bf{Table \ref{tab:r2500specfits} Notes}}\\
\end{center}
Note: ``77'' refers to 0.7-7.0 keV band and ``27'' refers to 2.0-7.0
keV band. Col. (1) Cluster name; col. (2) size of excluded core region
in kpc, (3) $R_{2500}$ in kpc; col. (4) absorbing Galactic neutral
hydrogen column density; col. (5,6) best-fit {\textsc{MeKaL}}
temperatures; col. (7) $T_{0.7-7.0}$/$T_{2.0-7.0}$ also called
$T_{HBR}$; col. (8) best-fit 77 {\textsc{MeKaL}} abundance;
col. (9,10) respective reduced $\chisq$ statistics, and (11) percent
of emission attributable to source. A star ($\star$) indicates a
cluster which has multiple observations. Each observation has an
independent spectrum extracted along with an associated WARF, WRMF,
normalized background spectrum, and soft residual. Each independent
spectrum is then fit simultaneously with the same spectral model to
produce the final fit.

\begin{center}
{\bf{Table \ref{tab:r5000specfits} Notes}}\\
\end{center}
Note: ``77'' refers to 0.7-7.0 keV band and ``27'' refers to 2.0-7.0
keV band. Col. (1) Cluster name; col. (2) size of excluded core region
in kpc, (3) $R_{5000}$ in kpc; col. (4) absorbing Galactic neutral
hydrogen column density; col. (5,6) best-fit {\textsc{MeKaL}}
temperatures; col. (7) $T_{0.7-7.0}$/$T_{2.0-7.0}$ also called
$T_{HBR}$; col. (8) best-fit 77 {\textsc{MeKaL}} abundance;
col. (9,10) respective reduced $\chisq$ statistics, and (11) percent
of emission attributable to source. A star ($\star$) indicates a
cluster which has multiple observations. Each observation has an
independent spectrum extracted along with an associated WARF, WRMF,
normalized background spectrum, and soft residual. Each independent
spectrum is then fit simultaneously with the same spectral model to
produce the final fit.

\clearpage
\singlespacing
\begin{rotthesistable}{lcccccccc}
\thesistablehead{Summary of sample for energy band dependance study}{Summary of sample for energy band dependance study}{Cluster & Obs.ID & R.A. & Dec. & ExpT & Mode & ACIS & $z$ & $L_{bol.}$\\ & & hr:min:sec & $^{\circ}:':''$ & ksec &   &   &   & $10^{44}$ ergs s$^{-1}$\\ (1) & (2) & (3) & (4) & (5) & (6) & (7) & (8) & (9)}{tab:sample}
1E0657 56 & 3184 & 06:58:29.622 & -55:56:39.79 & 87.5 & VF & I3 & 0.296 & 52.48\\
1E0657 56 & 5356 & 06:58:29.619 & -55:56:39.78 & 97.2 & VF & I2 & 0.296 & 52.48\\
1E0657 56 & 5361 & 06:58:29.620 & -55:56:39.80 & 82.6 & VF & I3 & 0.296 & 52.48\\
1RXS J2129.4-0741 & 3199 & 21:29:26.274 & -07:41:29.38 & 19.9 & VF & I3 & 0.570 & 20.58\\
1RXS J2129.4-0741 & 3595 & 21:29:26.281 & -07:41:29.36 & 19.9 & VF & I3 & 0.570 & 20.58\\
2PIGG J0011.5-2850 & 5797 & 00:11:21.623 & -28:51:14.44 & 19.9 & VF & I3 & 0.075 &  2.15\\
2PIGG J0311.8-2655 $\ddagger$ & 5799 & 03:11:33.904 & -26:54:16.48 & 39.6 & VF & I3 & 0.062 &  0.25\\
2PIGG J2227.0-3041 & 5798 & 22:27:54.560 & -30:34:34.84 & 22.3 & VF & I2 & 0.073 &  0.81\\
3C 220.1 & 839 & 09:32:40.218 & +79:06:29.46 & 18.9 &  F & S3 & 0.610 &  3.25\\
3C 28.0 & 3233 & 00:55:50.401 & +26:24:36.47 & 49.7 & VF & I3 & 0.195 &  4.78\\
3C 295 & 2254 & 14:11:20.280 & +52:12:10.55 & 90.9 & VF & I3 & 0.464 &  6.92\\
3C 388 & 5295 & 18:44:02.365 & +45:33:29.31 & 30.7 & VF & I3 & 0.092 &  0.52\\
4C 55.16 & 4940 & 08:34:54.923 & +55:34:21.15 & 96.0 & VF & S3 & 0.242 &  5.90\\
ABELL 0013 $\ddagger$ & 4945 & 00:13:37.883 & -19:30:09.10 & 55.3 & VF & S3 & 0.094 &  1.41\\
ABELL 0068 & 3250 & 00:37:06.309 & +09:09:32.28 & 10.0 & VF & I3 & 0.255 & 12.70\\
ABELL 0119 $\ddagger$ & 4180 & 00:56:15.150 & -01:14:59.70 & 11.9 & VF & I3 & 0.044 &  1.39\\
ABELL 0168 & 3203 & 01:14:57.909 & +00:24:42.55 & 40.6 & VF & I3 & 0.045 &  0.23\\
ABELL 0168 & 3204 & 01:14:57.925 & +00:24:42.73 & 37.6 & VF & I3 & 0.045 &  0.23\\
ABELL 0209 & 3579 & 01:31:52.585 & -13:36:39.29 & 10.0 & VF & I3 & 0.206 & 10.96\\
ABELL 0209 & 522 & 01:31:52.595 & -13:36:39.25 & 10.0 & VF & I3 & 0.206 & 10.96\\
ABELL 0267 & 1448 & 01:52:29.181 & +00:57:34.43 & 7.9 &  F & I3 & 0.230 &  8.62\\
ABELL 0267 & 3580 & 01:52:29.180 & +00:57:34.23 & 19.9 & VF & I3 & 0.230 &  8.62\\
ABELL 0370 & 515 & 02:39:53.169 & -01:34:36.96 & 88.0 &  F & S3 & 0.375 & 11.95\\
ABELL 0383 & 2321 & 02:48:03.364 & -03:31:44.69 & 19.5 &  F & S3 & 0.187 &  5.32\\
ABELL 0399 & 3230 & 02:57:54.931 & +13:01:58.41 & 48.6 & VF & I0 & 0.072 &  4.37\\
ABELL 0401 & 518 & 02:58:56.896 & +13:34:14.48 & 18.0 &  F & I3 & 0.074 &  8.39\\
ABELL 0478 & 6102 & 04:13:25.347 & +10:27:55.62 & 10.0 & VF & I3 & 0.088 & 16.39\\
ABELL 0514 & 3578 & 04:48:19.229 & -20:30:28.79 & 44.5 & VF & I3 & 0.072 &  0.66\\
ABELL 0520 & 4215 & 04:54:09.711 & +02:55:23.69 & 66.3 & VF & I3 & 0.202 & 12.97\\
ABELL 0521 & 430 & 04:54:07.004 & -10:13:26.72 & 39.1 & VF & S3 & 0.253 &  9.77\\
ABELL 0586 & 530 & 07:32:20.339 & +31:37:58.59 & 10.0 & VF & I3 & 0.171 &  8.54\\
ABELL 0611 & 3194 & 08:00:56.832 & +36:03:24.09 & 36.1 & VF & S3 & 0.288 & 10.78\\
ABELL 0644 $\ddagger$ & 2211 & 08:17:25.225 & -07:30:40.03 & 29.7 & VF & I3 & 0.070 &  6.95\\
ABELL 0665 & 3586 & 08:30:59.231 & +65:50:37.78 & 29.7 & VF & I3 & 0.181 & 13.37\\
ABELL 0697 & 4217 & 08:42:57.549 & +36:21:57.65 & 19.5 & VF & I3 & 0.282 & 26.10\\
ABELL 0773 & 5006 & 09:17:52.566 & +51:43:38.18 & 19.8 & VF & I3 & 0.217 & 12.87\\
\it{ABELL 0781} & 534 & 09:20:25.431 & +30:30:07.56 & 9.9 & VF & I3 & 0.298 &  8.24\\
ABELL 0907 & 3185 & 09:58:21.880 & -11:03:52.20 & 48.0 & VF & I3 & 0.153 &  6.19\\
ABELL 0963 & 903 & 10:17:03.744 & +39:02:49.17 & 36.3 &  F & S3 & 0.206 & 10.65\\
ABELL 1063S & 4966 & 22:48:44.294 & -44:31:48.37 & 26.7 & VF & I3 & 0.354 & 71.09\\
ABELL 1068 $\ddagger$ & 1652 & 10:40:44.520 & +39:57:10.28 & 26.8 &  F & S3 & 0.138 &  4.19\\
ABELL 1201 $\ddagger$ & 4216 & 11:12:54.489 & +13:26:08.76 & 39.7 & VF & S3 & 0.169 &  3.52\\
ABELL 1204 & 2205 & 11:13:20.419 & +17:35:38.45 & 23.6 & VF & I3 & 0.171 &  3.92\\
ABELL 1361 $\ddagger$ & 2200 & 11:43:39.827 & +46:21:21.40 & 16.7 &  F & S3 & 0.117 &  2.16\\
ABELL 1423 & 538 & 11:57:17.026 & +33:36:37.44 & 9.8 & VF & I3 & 0.213 &  7.01\\
ABELL 1651 & 4185 & 12:59:22.830 & -04:11:45.86 & 9.6 & VF & I3 & 0.084 &  6.66\\
ABELL 1664 $\ddagger$ & 1648 & 13:03:42.478 & -24:14:44.55 & 9.8 & VF & S3 & 0.128 &  2.59\\
\it{ABELL 1682} & 3244 & 13:06:50.764 & +46:33:19.86 & 9.8 & VF & I3 & 0.226 &  7.92\\
ABELL 1689 & 1663 & 13:11:29.612 & -01:20:28.69 & 10.7 &  F & I3 & 0.184 & 24.71\\
ABELL 1689 & 5004 & 13:11:29.606 & -01:20:28.61 & 19.9 & VF & I3 & 0.184 & 24.71\\
ABELL 1689 & 540 & 13:11:29.595 & -01:20:28.47 & 10.3 &  F & I3 & 0.184 & 24.71\\
ABELL 1758 & 2213 & 13:32:42.978 & +50:32:44.83 & 58.3 & VF & S3 & 0.279 & 21.01\\
ABELL 1763 & 3591 & 13:35:17.957 & +40:59:55.80 & 19.6 & VF & I3 & 0.187 &  9.26\\
ABELL 1795 $\ddagger$ & 5289 & 13:48:52.829 & +26:35:24.01 & 15.0 & VF & I3 & 0.062 &  7.59\\
ABELL 1835 & 495 & 14:01:01.951 & +02:52:43.18 & 19.5 &  F & S3 & 0.253 & 39.38\\
ABELL 1914 & 3593 & 14:26:01.399 & +37:49:27.83 & 18.9 & VF & I3 & 0.171 & 26.25\\
ABELL 1942 & 3290 & 14:38:21.878 & +03:40:12.97 & 57.6 & VF & I2 & 0.224 &  2.27\\
ABELL 1995 & 906 & 14:52:57.758 & +58:02:51.34 & 0.0 &  F & S3 & 0.319 & 10.19\\
ABELL 2029 $\ddagger$ & 6101 & 15:10:56.163 & +05:44:40.89 & 9.9 & VF & I3 & 0.076 & 13.90\\
ABELL 2034 & 2204 & 15:10:11.003 & +33:30:46.46 & 53.9 & VF & I3 & 0.113 &  6.45\\
ABELL 2065 $\ddagger$ & 31821 & 15:22:29.220 & +27:42:46.54 & 0.0 & VF & I3 & 0.073 &  2.92\\
ABELL 2069 & 4965 & 15:24:09.181 & +29:53:18.05 & 55.4 & VF & I2 & 0.116 &  3.82\\
ABELL 2111 & 544 & 15:39:41.432 & +34:25:12.26 & 10.3 &  F & I3 & 0.230 &  7.45\\
ABELL 2125 & 2207 & 15:41:14.154 & +66:15:57.20 & 81.5 & VF & I3 & 0.246 &  0.77\\
ABELL 2163 & 1653 & 16:15:45.705 & -06:09:00.62 & 71.1 & VF & I1 & 0.170 & 49.11\\
ABELL 2204 $\ddagger$ & 499 & 16:32:45.437 & +05:34:21.05 & 10.1 &  F & S3 & 0.152 & 20.77\\
ABELL 2204 & 6104 & 16:32:45.428 & +05:34:20.89 & 9.6 & VF & I3 & 0.152 & 22.03\\
ABELL 2218 & 1666 & 16:35:50.831 & +66:12:42.31 & 48.6 & VF & I0 & 0.171 &  8.39\\
ABELL 2219 $\ddagger$ & 896 & 16:40:21.069 & +46:42:29.07 & 42.3 &  F & S3 & 0.226 & 33.15\\
ABELL 2255 & 894 & 17:12:40.385 & +64:03:50.63 & 39.4 &  F & I3 & 0.081 &  3.67\\
ABELL 2256 $\ddagger$ & 1386 & 17:03:44.567 & +78:38:11.51 & 12.4 &  F & I3 & 0.058 &  4.65\\
ABELL 2259 & 3245 & 17:20:08.299 & +27:40:11.53 & 10.0 & VF & I3 & 0.164 &  5.37\\
ABELL 2261 & 5007 & 17:22:27.254 & +32:07:58.60 & 24.3 & VF & I3 & 0.224 & 17.49\\
ABELL 2294 & 3246 & 17:24:10.149 & +85:53:09.77 & 10.0 & VF & I3 & 0.178 & 10.35\\
ABELL 2384 & 4202 & 21:52:21.178 & -19:32:51.90 & 31.5 & VF & I3 & 0.095 &  1.95\\
ABELL 2390 $\ddagger$ & 4193 & 21:53:36.825 & +17:41:44.38 & 95.1 & VF & S3 & 0.230 & 31.02\\
ABELL 2409 & 3247 & 22:00:52.567 & +20:58:34.11 & 10.2 & VF & I3 & 0.148 &  7.01\\
ABELL 2537 & 4962 & 23:08:22.313 & -02:11:29.88 & 36.2 & VF & S3 & 0.295 & 10.16\\
\it{ABELL 2550} & 2225 & 23:11:35.806 & -21:44:46.70 & 59.0 & VF & S3 & 0.154 &  0.58\\
ABELL 2554 $\ddagger$ & 1696 & 23:12:19.939 & -21:30:09.84 & 19.9 & VF & S3 & 0.110 &  1.57\\
ABELL 2556 $\ddagger$ & 2226 & 23:13:01.413 & -21:38:04.47 & 19.9 & VF & S3 & 0.086 &  1.43\\
ABELL 2631 & 3248 & 23:37:38.560 & +00:16:28.64 & 9.2 & VF & I3 & 0.278 & 12.59\\
ABELL 2667 & 2214 & 23:51:39.395 & -26:05:02.75 & 9.6 & VF & S3 & 0.230 & 19.91\\
ABELL 2670 & 4959 & 23:54:13.687 & -10:25:08.85 & 39.6 & VF & I3 & 0.076 &  1.39\\
ABELL 2717 & 6974 & 00:03:11.996 & -35:56:08.01 & 19.8 & VF & I3 & 0.048 &  0.26\\
ABELL 2744 & 2212 & 00:14:14.396 & -30:22:40.04 & 24.8 & VF & S3 & 0.308 & 29.00\\
ABELL 3128 $\ddagger$ & 893 & 03:29:50.918 & -52:34:51.04 & 19.6 &  F & I3 & 0.062 &  0.35\\
ABELL 3158 $\ddagger$ & 3201 & 03:42:54.675 & -53:37:24.36 & 24.8 & VF & I3 & 0.059 &  3.01\\
ABELL 3158 $\ddagger$ & 3712 & 03:42:54.683 & -53:37:24.37 & 30.9 & VF & I3 & 0.059 &  3.01\\
ABELL 3164 & 6955 & 03:46:16.839 & -57:02:11.38 & 13.5 & VF & I3 & 0.057 &  0.19\\
ABELL 3376 & 3202 & 06:02:05.122 & -39:57:42.82 & 44.3 & VF & I3 & 0.046 &  0.75\\
ABELL 3376 & 3450 & 06:02:05.162 & -39:57:42.87 & 19.8 & VF & I3 & 0.046 &  0.75\\
ABELL 3391 $\ddagger$ & 4943 & 06:26:21.511 & -53:41:44.81 & 18.4 & VF & I3 & 0.056 &  1.44\\
ABELL 3921 & 4973 & 22:49:57.829 & -64:25:42.17 & 29.4 & VF & I3 & 0.093 &  3.37\\
AC 114 & 1562 & 22:58:48.196 & -34:47:56.89 & 72.5 &  F & S3 & 0.312 & 10.90\\
CL 0024+17 & 929 & 00:26:35.996 & +17:09:45.37 & 39.8 &  F & S3 & 0.394 &  2.88\\
CL 1221+4918 & 1662 & 12:21:26.709 & +49:18:21.60 & 79.1 & VF & I3 & 0.700 &  8.65\\
CL J0030+2618 & 5762 & 00:30:34.339 & +26:18:01.58 & 17.9 & VF & I3 & 0.500 &  3.41\\
CL J0152-1357 & 913 & 01:52:42.141 & -13:57:59.71 & 36.5 &  F & I3 & 0.831 & 13.30\\
CL J0542.8-4100 & 914 & 05:42:49.994 & -40:59:58.50 & 50.4 &  F & I3 & 0.630 &  6.18\\
CL J0848+4456 & 1708 & 08:48:48.255 & +44:56:17.11 & 61.4 & VF & I1 & 0.574 &  3.02\\
CL J0848+4456 & 927 & 08:48:48.252 & +44:56:17.13 & 125.1 & VF & I1 & 0.574 &  3.02\\
CL J1113.1-2615 & 915 & 11:13:05.167 & -26:15:40.43 & 104.6 &  F & I3 & 0.730 &  2.22\\
\it{CL J1213+0253} & 4934 & 12:13:34.948 & +02:53:45.45 & 18.9 & VF & I3 & 0.409 &  1.29\\
CL J1226.9+3332 & 3180 & 12:26:58.373 & +33:32:47.36 & 31.7 & VF & I3 & 0.890 & 30.76\\
CL J1226.9+3332 & 5014 & 12:26:58.372 & +33:32:47.38 & 32.7 & VF & I3 & 0.890 & 30.76\\
\it{CL J1641+4001} & 3575 & 16:41:53.704 & +40:01:44.40 & 46.5 & VF & I3 & 0.464 &  1.19\\
CL J2302.8+0844 & 918 & 23:02:48.156 & +08:43:52.74 & 108.6 &  F & I3 & 0.730 &  2.93\\
DLS J0514-4904 & 4980 & 05:14:40.037 & -49:03:15.07 & 19.9 & VF & I3 & 0.091 &  0.68\\
EXO 0422-086 $\ddagger$ & 4183 & 04:25:51.271 & -08:33:36.42 & 10.0 & VF & I3 & 0.040 &  0.65\\
HERCULES A $\ddagger$ & 1625 & 16:51:08.161 & +04:59:32.44 & 14.8 & VF & S3 & 0.154 &  3.27\\
\it{IRAS 09104+4109} & 509 & 09:13:45.481 & +40:56:27.49 & 9.1 &  F & S3 & 0.442 & 20.15\\
\it{LYNX E} & 17081 & 08:48:58.851 & +44:51:51.44 & 61.4 & VF & I2 & 1.260 &  2.10\\
\it{LYNX E} & 9271 & 08:48:58.858 & +44:51:51.46 & 125.1 & VF & I2 & 1.260 &  2.10\\
MACS J0011.7-1523 & 3261 & 00:11:42.965 & -15:23:20.79 & 21.6 & VF & I3 & 0.360 & 10.75\\
MACS J0011.7-1523 & 6105 & 00:11:42.957 & -15:23:20.76 & 37.3 & VF & I3 & 0.360 & 10.75\\
MACS J0025.4-1222 & 3251 & 00:25:29.398 & -12:22:38.15 & 19.3 & VF & I3 & 0.584 & 13.00\\
MACS J0025.4-1222 & 5010 & 00:25:29.399 & -12:22:38.10 & 24.8 & VF & I3 & 0.584 & 13.00\\
MACS J0035.4-2015 & 3262 & 00:35:26.573 & -20:15:46.06 & 21.4 & VF & I3 & 0.364 & 19.79\\
MACS J0111.5+0855 & 3256 & 01:11:31.515 & +08:55:39.21 & 19.4 & VF & I3 & 0.263 &  0.64\\
MACS J0152.5-2852 & 3264 & 01:52:34.479 & -28:53:38.01 & 17.5 & VF & I3 & 0.341 &  6.33\\
MACS J0159.0-3412 & 5818 & 01:59:00.366 & -34:13:00.23 & 9.4 & VF & I3 & 0.458 & 18.92\\
MACS J0159.8-0849 & 3265 & 01:59:49.453 & -08:50:00.90 & 17.9 & VF & I3 & 0.405 & 26.31\\
MACS J0159.8-0849 & 6106 & 01:59:49.452 & -08:50:00.92 & 35.3 & VF & I3 & 0.405 & 26.31\\
MACS J0242.5-2132 & 3266 & 02:42:35.906 & -21:32:26.30 & 11.9 & VF & I3 & 0.314 & 12.74\\
MACS J0257.1-2325 & 1654 & 02:57:09.150 & -23:26:06.25 & 19.8 &  F & I3 & 0.505 & 21.72\\
MACS J0257.1-2325 & 3581 & 02:57:09.152 & -23:26:06.21 & 18.5 & VF & I3 & 0.505 & 21.72\\
MACS J0257.6-2209 & 3267 & 02:57:41.024 & -22:09:11.12 & 20.5 & VF & I3 & 0.322 & 10.77\\
MACS J0308.9+2645 & 3268 & 03:08:55.927 & +26:45:38.34 & 24.4 & VF & I3 & 0.324 & 20.42\\
MACS J0329.6-0211 & 3257 & 03:29:41.681 & -02:11:47.67 & 9.9 & VF & I3 & 0.450 & 12.82\\
MACS J0329.6-0211 & 3582 & 03:29:41.688 & -02:11:47.81 & 19.9 & VF & I3 & 0.450 & 12.82\\
MACS J0329.6-0211 & 6108 & 03:29:41.681 & -02:11:47.57 & 39.6 & VF & I3 & 0.450 & 12.82\\
MACS J0404.6+1109 & 3269 & 04:04:32.491 & +11:08:02.10 & 21.8 & VF & I3 & 0.355 &  3.90\\
MACS J0417.5-1154 & 3270 & 04:17:34.686 & -11:54:32.71 & 12.0 & VF & I3 & 0.440 & 37.99\\
MACS J0429.6-0253 & 3271 & 04:29:36.088 & -02:53:09.02 & 23.2 & VF & I3 & 0.399 & 11.58\\
MACS J0451.9+0006 & 5815 & 04:51:54.291 & +00:06:20.20 & 10.2 & VF & I3 & 0.430 &  8.20\\
MACS J0455.2+0657 & 5812 & 04:55:17.426 & +06:57:47.15 & 9.9 & VF & I3 & 0.425 &  9.77\\
MACS J0520.7-1328 & 3272 & 05:20:42.052 & -13:28:49.38 & 19.2 & VF & I3 & 0.340 &  9.63\\
MACS J0547.0-3904 & 3273 & 05:47:01.582 & -39:04:28.24 & 21.7 & VF & I3 & 0.210 &  1.59\\
MACS J0553.4-3342 & 5813 & 05:53:27.200 & -33:42:53.02 & 9.9 & VF & I3 & 0.407 & 32.68\\
MACS J0717.5+3745 & 1655 & 07:17:31.654 & +37:45:18.52 & 19.9 &  F & I3 & 0.548 & 46.58\\
MACS J0717.5+3745 & 4200 & 07:17:31.651 & +37:45:18.46 & 59.2 & VF & I3 & 0.548 & 46.58\\
MACS J0744.8+3927 & 3197 & 07:44:52.802 & +39:27:24.43 & 20.2 & VF & I3 & 0.686 & 24.67\\
MACS J0744.8+3927 & 3585 & 07:44:52.809 & +39:27:24.41 & 19.9 & VF & I3 & 0.686 & 24.67\\
MACS J0744.8+3927 & 6111 & 07:44:52.800 & +39:27:24.42 & 49.5 & VF & I3 & 0.686 & 24.67\\
MACS J0911.2+1746 & 3587 & 09:11:11.325 & +17:46:31.02 & 17.9 & VF & I3 & 0.541 & 10.52\\
MACS J0911.2+1746 & 5012 & 09:11:11.329 & +17:46:30.99 & 23.8 & VF & I3 & 0.541 & 10.52\\
MACS J0949+1708   & 3274 & 09:49:51.824 & +17:07:05.62 & 14.3 & VF & I3 & 0.382 & 19.19\\
MACS J1006.9+3200 & 5819 & 10:06:54.668 & +32:01:34.61 & 10.9 & VF & I3 & 0.359 &  6.06\\
MACS J1105.7-1014 & 5817 & 11:05:46.462 & -10:14:37.20 & 10.3 & VF & I3 & 0.466 & 11.29\\
MACS J1108.8+0906 & 3252 & 11:08:55.393 & +09:05:51.16 & 9.9 & VF & I3 & 0.449 &  8.96\\
MACS J1108.8+0906 & 5009 & 11:08:55.402 & +09:05:51.14 & 24.5 & VF & I3 & 0.449 &  8.96\\
MACS J1115.2+5320 & 3253 & 11:15:15.632 & +53:20:03.71 & 8.8 & VF & I3 & 0.439 & 14.29\\
MACS J1115.2+5320 & 5008 & 11:15:15.636 & +53:20:03.74 & 18.0 & VF & I3 & 0.439 & 14.29\\
MACS J1115.2+5320 & 5350 & 11:15:15.632 & +53:20:03.77 & 6.9 & VF & I3 & 0.439 & 14.29\\
MACS J1115.8+0129 & 3275 & 11:15:52.048 & +01:29:56.56 & 15.9 & VF & I3 & 0.120 &  1.47\\
MACS J1131.8-1955 & 3276 & 11:31:56.011 & -19:55:55.85 & 13.9 & VF & I3 & 0.307 & 17.45\\
MACS J1149.5+2223 & 1656 & 11:49:35.856 & +22:23:55.02 & 18.5 & VF & I3 & 0.544 & 21.60\\
MACS J1149.5+2223 & 3589 & 11:49:35.858 & +22:23:55.05 & 20.0 & VF & I3 & 0.544 & 21.60\\
MACS J1206.2-0847 & 3277 & 12:06:12.276 & -08:48:02.40 & 23.5 & VF & I3 & 0.440 & 37.02\\
MACS J1226.8+2153 & 3590 & 12:26:51.207 & +21:49:55.22 & 19.0 & VF & I3 & 0.370 &  2.63\\
MACS J1311.0-0310 & 3258 & 13:11:01.685 & -03:10:39.70 & 14.9 & VF & I3 & 0.494 & 10.03\\
MACS J1311.0-0310 & 6110 & 13:11:01.680 & -03:10:39.75 & 63.2 & VF & I3 & 0.494 & 10.03\\
MACS J1319+7003   & 3278 & 13:20:08.370 & +70:04:33.81 & 21.6 & VF & I3 & 0.328 &  7.03\\
MACS J1427.2+4407 & 6112 & 14:27:16.175 & +44:07:30.33 & 9.4 & VF & I3 & 0.477 & 14.18\\
MACS J1427.6-2521 & 3279 & 14:27:39.389 & -25:21:04.66 & 16.9 & VF & I3 & 0.220 &  1.55\\
MACS J1621.3+3810 & 3254 & 16:21:25.552 & +38:09:43.56 & 9.8 & VF & I3 & 0.461 & 11.49\\
MACS J1621.3+3810 & 3594 & 16:21:25.558 & +38:09:43.54 & 19.7 & VF & I3 & 0.461 & 11.49\\
MACS J1621.3+3810 & 6109 & 16:21:25.555 & +38:09:43.54 & 37.5 & VF & I3 & 0.461 & 11.49\\
MACS J1621.3+3810 & 6172 & 16:21:25.559 & +38:09:43.53 & 29.8 & VF & I3 & 0.461 & 11.49\\
MACS J1731.6+2252 & 3281 & 17:31:39.902 & +22:52:00.55 & 20.5 & VF & I3 & 0.366 &  9.32\\
\it{MACS J1824.3+4309} & 3255 & 18:24:18.444 & +43:09:43.39 & 14.9 & VF & I3 & 0.487 &  2.48\\
MACS J1931.8-2634 & 3282 & 19:31:49.656 & -26:34:33.99 & 13.6 & VF & I3 & 0.352 & 23.14\\
MACS J2046.0-3430 & 5816 & 20:46:00.522 & -34:30:15.50 & 10.0 & VF & I3 & 0.413 &  5.79\\
MACS J2049.9-3217 & 3283 & 20:49:56.245 & -32:16:52.30 & 23.8 & VF & I3 & 0.325 &  8.71\\
MACS J2211.7-0349 & 3284 & 22:11:45.856 & -03:49:37.24 & 17.7 & VF & I3 & 0.270 & 22.11\\
MACS J2214.9-1359 & 3259 & 22:14:57.487 & -14:00:09.35 & 19.5 & VF & I3 & 0.503 & 24.05\\
MACS J2214.9-1359 & 5011 & 22:14:57.481 & -14:00:09.39 & 18.5 & VF & I3 & 0.503 & 24.05\\
MACS J2228+2036   & 3285 & 22:28:33.241 & +20:37:11.42 & 19.9 & VF & I3 & 0.412 & 17.92\\
MACS J2229.7-2755 & 3286 & 22:29:45.358 & -27:55:38.41 & 16.4 & VF & I3 & 0.324 &  9.49\\
MACS J2243.3-0935 & 3260 & 22:43:21.537 & -09:35:44.30 & 20.5 & VF & I3 & 0.101 &  0.78\\
MACS J2245.0+2637 & 3287 & 22:45:04.547 & +26:38:07.88 & 16.9 & VF & I3 & 0.304 &  9.36\\
MACS J2311+0338   & 3288 & 23:11:33.213 & +03:38:06.51 & 13.6 & VF & I3 & 0.300 & 10.98\\
MKW3S & 900 & 15:21:51.930 & +07:42:31.97 & 57.3 & VF & I3 & 0.045 &  1.14\\
MS 0016.9+1609 & 520 & 00:18:33.503 & +16:26:12.99 & 67.4 & VF & I3 & 0.541 & 32.94\\
\it{MS 0302.7+1658} & 525 & 03:05:31.614 & +17:10:02.06 & 10.0 & VF & I3 & 0.424 &  2.41\\
MS 0440.5+0204 $\ddagger$ & 4196 & 04:43:09.952 & +02:10:18.70 & 59.4 & VF & S3 & 0.190 &  2.17\\
MS 0451.6-0305 & 902 & 04:54:11.004 & -03:00:52.19 & 44.2 &  F & S3 & 0.539 & 33.32\\
MS 0735.6+7421 & 4197 & 07:41:44.245 & +74:14:38.23 & 45.5 & VF & S3 & 0.216 &  7.57\\
MS 0839.8+2938 & 2224 & 08:42:55.969 & +29:27:26.97 & 29.8 &  F & S3 & 0.194 &  3.10\\
MS 0906.5+1110 & 924 & 09:09:12.753 & +10:58:32.00 & 29.7 & VF & I3 & 0.163 &  4.64\\
MS 1006.0+1202 & 925 & 10:08:47.194 & +11:47:55.99 & 29.4 & VF & I3 & 0.221 &  4.75\\
MS 1008.1-1224 & 926 & 10:10:32.312 & -12:39:56.80 & 44.2 & VF & I3 & 0.301 &  6.44\\
MS 1054.5-0321 & 512 & 10:56:58.499 & -03:37:32.76 & 89.1 &  F & S3 & 0.830 & 27.22\\
MS 1455.0+2232 & 4192 & 14:57:15.088 & +22:20:32.49 & 91.9 & VF & I3 & 0.259 & 10.25\\
MS 1621.5+2640 & 546 & 16:23:35.522 & +26:34:25.67 & 30.1 &  F & I3 & 0.426 &  6.49\\
MS 2053.7-0449 & 1667 & 20:56:21.295 & -04:37:46.81 & 44.5 & VF & I3 & 0.583 &  2.96\\
MS 2053.7-0449 & 551 & 20:56:21.297 & -04:37:46.80 & 44.3 &  F & I3 & 0.583 &  2.96\\
MS 2137.3-2353 & 4974 & 21:40:15.178 & -23:39:40.71 & 57.4 & VF & S3 & 0.313 & 11.28\\
MS J1157.3+5531 $\ddagger$ & 4964 & 11:59:52.295 & +55:32:05.61 & 75.1 & VF & S3 & 0.081 &  0.12\\
NGC 6338 $\ddagger$ & 4194 & 17:15:23.036 & +57:24:40.29 & 47.3 & VF & I3 & 0.028 &  0.13\\
PKS 0745-191 & 6103 & 07:47:31.469 & -19:17:40.01 & 10.3 & VF & I3 & 0.103 & 18.41\\
RBS 0797 & 2202 & 09:47:12.971 & +76:23:13.90 & 11.7 & VF & I3 & 0.354 & 26.07\\
RDCS 1252-29    & 4198 & 12:52:54.221 & -29:27:21.01 & 163.4 & VF & I3 & 1.237 &  2.28\\
RX J0232.2-4420 & 4993 & 02:32:18.771 & -44:20:46.68 & 23.4 & VF & I3 & 0.284 & 18.17\\
RX J0340-4542   & 6954 & 03:40:44.765 & -45:41:18.41 & 17.9 & VF & I3 & 0.082 &  0.33\\
RX J0439+0520   & 527 & 04:39:02.218 & +05:20:43.11 & 9.6 & VF & I3 & 0.208 &  3.57\\
RX J0439.0+0715 & 1449 & 04:39:00.710 & +07:16:07.65 & 6.3 &  F & I3 & 0.230 &  9.44\\
RX J0439.0+0715 & 3583 & 04:39:00.710 & +07:16:07.63 & 19.2 & VF & I3 & 0.230 &  9.44\\
RX J0528.9-3927 & 4994 & 05:28:53.039 & -39:28:15.53 & 22.5 & VF & I3 & 0.263 & 12.99\\
RX J0647.7+7015 & 3196 & 06:47:50.029 & +70:14:49.66 & 19.3 & VF & I3 & 0.584 & 26.48\\
RX J0647.7+7015 & 3584 & 06:47:50.024 & +70:14:49.69 & 20.0 & VF & I3 & 0.584 & 26.48\\
RX J0819.6+6336 $\ddagger$ & 2199 & 08:19:26.007 & +63:37:26.53 & 14.9 &  F & S3 & 0.119 &  0.98\\
RX J0910+5422   & 2452 & 09:10:44.478 & +54:22:03.77 & 65.3 & VF & I3 & 1.100 &  1.33\\
\it{RX J1053+5735} & 4936 & 10:53:39.844 & +57:35:18.42 & 92.2 &  F & S3 & 1.140 &  1.59\\
RX J1347.5-1145 & 3592 & 13:47:30.593 & -11:45:10.25 & 57.7 & VF & I3 & 0.451 & 100.36\\
RX J1347.5-1145 & 507 & 13:47:30.598 & -11:45:10.27 & 10.0 &  F & S3 & 0.451 & 100.36\\
RX J1350+6007   & 2229 & 13:50:48.038 & +60:07:08.39 & 58.3 & VF & I3 & 0.804 &  2.19\\
RX J1423.8+2404 & 1657 & 14:23:47.759 & +24:04:40.65 & 18.5 & VF & I3 & 0.545 & 15.84\\
RX J1423.8+2404 & 4195 & 14:23:47.763 & +24:04:40.63 & 115.6 & VF & S3 & 0.545 & 15.84\\
RX J1504.1-0248 & 5793 & 15:04:07.415 & -02:48:15.70 & 39.2 & VF & I3 & 0.215 & 34.64\\
RX J1525+0958   & 1664 & 15:24:39.729 & +09:57:44.42 & 50.9 & VF & I3 & 0.516 &  3.29\\
RX J1532.9+3021 & 1649 & 15:32:55.642 & +30:18:57.69 & 9.4 & VF & S3 & 0.345 & 20.77\\
RX J1532.9+3021 & 1665 & 15:32:55.641 & +30:18:57.61 & 10.0 & VF & I3 & 0.345 & 20.77\\
RX J1716.9+6708 & 548 & 17:16:49.015 & +67:08:25.80 & 51.7 &  F & I3 & 0.810 &  8.04\\
RX J1720.1+2638 & 4361 & 17:20:09.941 & +26:37:29.11 & 25.7 & VF & I3 & 0.164 & 11.39\\
RX J1720.2+3536 & 3280 & 17:20:16.953 & +35:36:23.63 & 20.8 & VF & I3 & 0.391 & 13.02\\
RX J1720.2+3536 & 6107 & 17:20:16.949 & +35:36:23.68 & 33.9 & VF & I3 & 0.391 & 13.02\\
RX J1720.2+3536 & 7225 & 17:20:16.947 & +35:36:23.69 & 2.0 & VF & I3 & 0.391 & 13.02\\
RX J2011.3-5725 & 4995 & 20:11:26.889 & -57:25:09.08 & 24.0 & VF & I3 & 0.279 &  2.77\\
RX J2129.6+0005 & 552 & 21:29:39.944 & +00:05:18.83 & 10.0 & VF & I3 & 0.235 & 12.56\\
S0463 & 6956 & 04:29:07.040 & -53:49:38.02 & 29.3 & VF & I3 & 0.099 & 22.19\\
S0463 & 7250 & 04:29:07.063 & -53:49:38.11 & 29.1 & VF & I3 & 0.099 & 22.19\\
TRIANG AUSTR $\ddagger$ & 1281 & 16:38:22.712 & -64:21:19.70 & 11.4 &  F & I3 & 0.051 &  9.41\\
V 1121.0+2327 & 1660 & 11:20:57.195 & +23:26:27.60 & 71.3 & VF & I3 & 0.560 &  3.28\\
ZWCL 1215 & 4184 & 12:17:40.787 & +03:39:39.42 & 12.1 & VF & I3 & 0.075 &  3.49\\
ZWCL 1358+6245 & 516 & 13:59:50.526 & +62:31:04.57 & 54.1 &  F & S3 & 0.328 & 12.42\\
ZWCL 1953 & 1659 & 08:50:06.677 & +36:04:16.16 & 24.9 &  F & I3 & 0.380 & 17.11\\
ZWCL 3146 & 909 & 10:23:39.735 & +04:11:08.05 & 46.0 &  F & I3 & 0.290 & 29.59\\
ZWCL 5247 & 539 & 12:34:21.928 & +09:47:02.83 & 9.3 & VF & I3 & 0.229 &  4.87\\
ZWCL 7160 & 543 & 14:57:15.158 & +22:20:33.85 & 9.9 &  F & I3 & 0.258 & 10.14\\
ZWICKY 2701 & 3195 & 09:52:49.183 & +51:53:05.27 & 26.9 & VF & S3 & 0.210 &  5.19\\
ZwCL 1332.8+5043 & 5772 & 13:34:20.698 & +50:31:04.64 & 19.5 & VF & I3 & 0.620 &  4.46\\
ZwCl 0848.5+3341 & 4205 & 08:51:38.873 & +33:31:08.00 & 11.4 & VF & S3 & 0.371 &  4.58
\end{rotthesistable}
\doublespacing

\clearpage
\singlespacing
\begin{rotthesistable}{lcccclc}
\thesistablehead{Clusters with $T_{HBR}$ $> 1.1$ with 90\% confidence.}{Clusters with $T_{HBR}$ $> 1.1$ with 90\% confidence.}{Name & $T_{HBR}$ & Merger? & Core Class & $T_{dec}$ & X-ray Morphology & Ref.}{tab:tf11}
RX J1525+0958       \dotfill & 1.86$^{+0.83}_{-0.51}$ & Y       &  CC & 0.42$^{+0.14}_{-0.08}$ & Arrowhead shape \& no discernible core & [29]\\
MS 1008.1-1224      \dotfill & 1.59$^{+0.37}_{-0.27}$ & Y       & NCC & 0.93$^{+0.19}_{-0.14}$ & Wide gas tail extending $\approx$550 kpc north & [1]\\
ABELL 2034          \dotfill & 1.40$^{+0.14}_{-0.11}$ & Y       & NCC & 1.07$^{+0.11}_{-0.09}$ & Prominent cold front \& gas tail extending south & [2]\\
ABELL 401           \dotfill & 1.37$^{+0.12}_{-0.10}$ & Y       & NCC & 1.13$^{+0.12}_{-0.10}$ & Highly spherical \& possible cold front to north & [3]\\
ABELL 1689          \dotfill & 1.36$^{+0.14}_{-0.12}$ & Y       & NCC & 0.95$^{+0.09}_{-0.07}$ & Exceptionally spherical \& bright central core & [6],[7]\\
RX J0439.0+0715     \dotfill & 1.42$^{+0.24}_{-0.18}$ & Unknown & NCC & 0.98$^{+0.11}_{-0.09}$ & Bright core \& possible cold front to north & [29]\\
ABELL 3376          \dotfill & 1.33$^{+0.11}_{-0.10}$ & Y       & NCC & 0.97$^{+0.07}_{-0.07}$ & Highly disturbed \& broad gas tail to west & [4],[5]\\
ABELL 2255          \dotfill & 1.32$^{+0.12}_{-0.10}$ & Y       & NCC & 1.48$^{+0.32}_{-0.23}$ & Spherical \& compressed isophotes west of core & [8],[9]\\
ABELL 2218          \dotfill & 1.36$^{+0.19}_{-0.15}$ & Y       & NCC & 1.39$^{+0.23}_{-0.19}$ & Spherical, core of cluster elongated NW-SE & [10]\\
ABELL 1763          \dotfill & 1.48$^{+0.39}_{-0.26}$ & Y       & NCC & 0.83$^{+0.17}_{-0.13}$ & Elongated ENE-SSW \& cold front to west of core & [11],[12]\\
MACS J2243.3-0935   \dotfill & 1.76$^{+0.81}_{-0.55}$ & Unknown & NCC & 1.73$^{+0.44}_{-0.32}$ & No core \& highly flattened along WNW-ESE axis & [29]\\
ABELL 2069          \dotfill & 1.32$^{+0.17}_{-0.14}$ & Y       & NCC & 1.00$^{+0.18}_{-0.14}$ & No core \& highly elongated NNW-SSE & [13]\\
ABELL 2384          \dotfill & 1.31$^{+0.16}_{-0.14}$ & Unknown &  CC & 0.59$^{+0.03}_{-0.03}$ & Gas tail extending 1.1 Mpc from core & [29]\\
ABELL 168           \dotfill & 1.31$^{+0.16}_{-0.14}$ & Y       & NCC & 1.16$^{+0.14}_{-0.10}$ & Highly disrupted \& irregular & [14],[15]\\
ABELL 209           \dotfill & 1.38$^{+0.28}_{-0.22}$ & Y       & NCC & 1.08$^{+0.22}_{-0.17}$ & Asymmetric core structure \& possible cold front & [16]\\
ABELL 665           \dotfill & 1.29$^{+0.15}_{-0.13}$ & Y       & NCC & 1.14$^{+0.19}_{-0.15}$ & Wide, broad gas tail to north \& cold front & [17]\\
1E0657-56           \dotfill & 1.21$^{+0.06}_{-0.05}$ & Y       & NCC & 1.04$^{+0.10}_{-0.08}$ & The famous ``Bullet Cluster'' & [18]\\
MACS J0547.0-3904   \dotfill & 1.51$^{+0.50}_{-0.36}$ & Unknown & NCC & 0.77$^{+0.14}_{-0.18}$ & Bright core \& gas spur extending NW & [29]\\
ZWCL 1215           \dotfill & 1.31$^{+0.21}_{-0.18}$ & Unknown & NCC & 0.95$^{+0.15}_{-0.12}$ & No core, flattened along NE-SW axis & [29]\\
ABELL 1204          \dotfill & 1.26$^{+0.17}_{-0.14}$ & Unknown & NCC & 0.96$^{+0.05}_{-0.05}$ & Highly spherical \& bright centralized core & [29]\\
MKW3S               \dotfill & 1.17$^{+0.05}_{-0.05}$ & Y       &  CC & 0.87$^{+0.02}_{-0.02}$ & High mass group, egg shaped \& bright core & [19]\\
MACS J2311+0338     \dotfill & 1.53$^{+0.69}_{-0.42}$ & Unknown & NCC & 0.69$^{+0.20}_{-0.15}$ & Elongated N-S \& disc-like core & [29]\\
ABELL 267           \dotfill & 1.33$^{+0.27}_{-0.21}$ & Unknown & NCC & 1.09$^{+0.20}_{-0.16}$ & Elongated NNE-SSW \& cold front to north & [29]\\
RX J1720.1+2638     \dotfill & 1.22$^{+0.12}_{-0.11}$ & Y       &  CC & 0.73$^{+0.04}_{-0.04}$ & Very spherical, bright peaky core, \& cold front & [20]\\
ABELL 907           \dotfill & 1.21$^{+0.10}_{-0.08}$ & Unknown &  CC & 0.76$^{+0.03}_{-0.03}$ & NW-SW elongation \& western cold front & [29]\\
ABELL 514           \dotfill & 1.26$^{+0.19}_{-0.15}$ & Y       & NCC & 1.56$^{+1.07}_{-0.40}$ & Very diffuse \& disrupted & [21]\\
ABELL 1651          \dotfill & 1.24$^{+0.16}_{-0.13}$ & Y       & NCC & 1.07$^{+0.10}_{-0.08}$ & Spherical \& compressed isophotes to SW & [22]\\
3C 28.0             \dotfill & 1.23$^{+0.14}_{-0.12}$ & Y       &  CC & 0.54$^{+0.03}_{-0.03}$ & Obvious merger \& $\sim$1 Mpc gas tail & [23]\\
\hline
\multicolumn{7}{c}{$R_{5000-\mathrm{CORE}}$ Only}\\
\hline
TRIANG AUSTR        \dotfill & 1.42$^{+0.14}_{-0.14}$ &    Y &  NCC &  0.90$^{+0.06}_{-0.09}$ & Highly diffuse \& no bright core & [24]\\
ABELL 3158          \dotfill & 1.23$^{+0.05}_{-0.05}$ &    Y &  NCC &  1.15$^{+0.05}_{-0.05}$ & Large centroid variation & [25]\\
ABELL 2256          \dotfill & 1.29$^{+0.13}_{-0.12}$ &    Y &  NCC &  1.40$^{+0.15}_{-0.12}$ & Spiral shaped \& distinct NW edge & [26]\\
NGC 6338            \dotfill & 1.22$^{+0.12}_{-0.10}$ & Unknown & NCC &  0.96$^{+0.04}_{-0.03}$ & Disrupted group companion to north & [29]\\
ABELL 2029          \dotfill & 1.21$^{+0.12}_{-0.10}$ &    Y &   CC &  0.86$^{+0.04}_{-0.04}$ & Possible cold front to W \& WAT radio source & [27],[28]\\
\end{rotthesistable}
\doublespacing

\clearpage
\singlespacing
\begin{rotthesistable}{lcccccccccc}
\thesistablehead{Summary of Excised $R_{2500}$ Spectral Fits}{Summary of Excised $R_{2500}$ Spectral Fits}
{Cluster & $R_{\mathrm{CORE}}$ & $R_{2500}$ & $N_{H}$ & $T_{77}$ & $T_{27}$ & $T_{HBR}$ & $Z_{77}$ & $\chisq_{red,77}$ & $\chisq_{red,27}$ & \% Source\\ 
& kpc & kpc & $10^{20}$ cm$^{-2}$ & keV & keV & & $Z_{\odot}$ & & & \\ (1) & (2) & (3) & (4) & (5) & (6) & (7) & (8) & (9) & (10) & (11)}{tab:r2500specfits}
1E0657 56 $\star$ &    69 &   688 & 6.53  & 11.99  $^{+0.27   }_{-0.26   }$  & 14.54  $^{+0.67   }_{-0.53   }$  & 1.21   $^{+0.06   }_{-0.05   }$  & 0.29$^{+0.03   }_{-0.02   }$  & 1.24 & 1.11 &  92\\
1RXS J2129.4-0741 $\star$ &    71 &   526 & 4.36  & 8.22   $^{+1.18   }_{-0.95   }$  & 8.10   $^{+1.47   }_{-1.10   }$  & 0.99   $^{+0.23   }_{-0.18   }$  & 0.43$^{+0.18   }_{-0.17   }$  & 1.07 & 1.05 &  80\\
2PIGG J0011.5-2850 &    69 &   547 & 2.18  & 5.15   $^{+0.25   }_{-0.24   }$  & 6.20   $^{+0.79   }_{-0.65   }$  & 1.20   $^{+0.16   }_{-0.14   }$  & 0.26$^{+0.09   }_{-0.08   }$  & 1.09 & 1.00 &  70\\
2PIGG J2227.0-3041 &    69 &   378 & 1.11  & 2.80   $^{+0.15   }_{-0.14   }$  & 2.97   $^{+0.34   }_{-0.27   }$  & 1.06   $^{+0.13   }_{-0.11   }$  & 0.35$^{+0.09   }_{-0.08   }$  & 1.16 & 1.15 &  69\\
3C 220.1 &    71 &   456 & 1.91  & 9.26   $^{+14.71  }_{-3.98   }$  & 8.00   $^{+17.66  }_{-4.03   }$  & 0.86   $^{+2.35   }_{-0.57   }$  & 0.00$^{+0.59   }_{-0.00   }$  & 1.20 & 1.40 &  30\\
3C 28.0 &    70 &   420 & 5.71  & 5.53   $^{+0.29   }_{-0.27   }$  & 6.81   $^{+0.71   }_{-0.60   }$  & 1.23   $^{+0.14   }_{-0.12   }$  & 0.30$^{+0.08   }_{-0.07   }$  & 0.98 & 0.88 &  87\\
3C 295 &    69 &   465 & 1.35  & 5.16   $^{+0.42   }_{-0.38   }$  & 5.93   $^{+0.84   }_{-0.69   }$  & 1.15   $^{+0.19   }_{-0.16   }$  & 0.38$^{+0.12   }_{-0.11   }$  & 0.91 & 0.93 &  79\\
3C 388 &    69 &   420 & 6.11  & 3.23   $^{+0.23   }_{-0.21   }$  & 3.26   $^{+0.49   }_{-0.37   }$  & 1.01   $^{+0.17   }_{-0.13   }$  & 0.51$^{+0.16   }_{-0.14   }$  & 0.95 & 0.95 &  68\\
4C 55.16 &    69 &   426 & 4.00  & 4.98   $^{+0.17   }_{-0.17   }$  & 5.54   $^{+0.40   }_{-0.36   }$  & 1.11   $^{+0.09   }_{-0.08   }$  & 0.49$^{+0.07   }_{-0.07   }$  & 0.89 & 0.80 &  58\\
ABELL 0068 &    70 &   680 & 4.60  & 9.01   $^{+1.53   }_{-1.14   }$  & 9.13   $^{+2.60   }_{-1.71   }$  & 1.01   $^{+0.34   }_{-0.23   }$  & 0.46$^{+0.24   }_{-0.22   }$  & 1.15 & 1.13 &  79\\
ABELL 0168 $\star$ &    70 &   398 & 3.27  & 2.56   $^{+0.11   }_{-0.08   }$  & 3.36   $^{+0.37   }_{-0.35   }$  & 1.31   $^{+0.16   }_{-0.14   }$  & 0.29$^{+0.06   }_{-0.04   }$  & 1.07 & 1.03 &  40\\
ABELL 0209 $\star$ &    70 &   609 & 1.68  & 7.30   $^{+0.59   }_{-0.51   }$  & 10.07  $^{+1.91   }_{-1.41   }$  & 1.38   $^{+0.28   }_{-0.22   }$  & 0.23$^{+0.10   }_{-0.09   }$  & 1.12 & 1.11 &  82\\
ABELL 0267 $\star$ &    70 &   545 & 2.74  & 6.70   $^{+0.56   }_{-0.47   }$  & 8.88   $^{+1.68   }_{-1.27   }$  & 1.33   $^{+0.27   }_{-0.21   }$  & 0.32$^{+0.11   }_{-0.11   }$  & 1.18 & 1.15 &  82\\
ABELL 0370 &    69 &   516 & 3.37  & 7.35   $^{+0.72   }_{-0.84   }$  & 10.35  $^{+1.89   }_{-2.27   }$  & 1.41   $^{+0.29   }_{-0.35   }$  & 0.45$^{+0.06   }_{-0.23   }$  & 1.08 & 1.04 &  39\\
ABELL 0383 &    69 &   423 & 4.07  & 4.91   $^{+0.29   }_{-0.27   }$  & 5.42   $^{+0.74   }_{-0.59   }$  & 1.10   $^{+0.16   }_{-0.13   }$  & 0.44$^{+0.11   }_{-0.11   }$  & 0.97 & 0.90 &  64\\
ABELL 0399 &    69 &   546 & 7.57$^{+0.71   }_{-0.71   }$  & 7.95   $^{+0.35   }_{-0.31   }$  & 8.87   $^{+0.55   }_{-0.50   }$  & 1.12   $^{+0.08   }_{-0.08   }$  & 0.30$^{+0.05   }_{-0.05   }$  & 1.12 & 0.99 &  82\\
ABELL 0401 &    69 &   643 & 12.48 & 6.37   $^{+0.19   }_{-0.19   }$  & 8.71   $^{+0.72   }_{-0.61   }$  & 1.37   $^{+0.12   }_{-0.10   }$  & 0.26$^{+0.06   }_{-0.06   }$  & 1.44 & 1.05 &  78\\
ABELL 0478 &    69 &   598 & 30.90 & 7.30   $^{+0.26   }_{-0.24   }$  & 8.62   $^{+0.58   }_{-0.54   }$  & 1.18   $^{+0.09   }_{-0.08   }$  & 0.45$^{+0.06   }_{-0.05   }$  & 1.05 & 0.95 &  91\\
ABELL 0514 &    71 &   516 & 3.14  & 3.33   $^{+0.16   }_{-0.16   }$  & 4.02   $^{+0.54   }_{-0.46   }$  & 1.21   $^{+0.17   }_{-0.15   }$  & 0.25$^{+0.08   }_{-0.06   }$  & 1.07 & 0.97 &  53\\
ABELL 0520 &    70 &   576 & 1.06$^{+1.06   }_{-1.05   }$  & 9.29   $^{+0.67   }_{-0.60   }$  & 9.88   $^{+0.85   }_{-0.73   }$  & 1.06   $^{+0.12   }_{-0.10   }$  & 0.37$^{+0.07   }_{-0.07   }$  & 1.11 & 1.04 &  87\\
ABELL 0521 &    70 &   558 & 6.17  & 7.03   $^{+0.59   }_{-0.53   }$  & 8.39   $^{+1.62   }_{-1.22   }$  & 1.19   $^{+0.25   }_{-0.20   }$  & 0.39$^{+0.13   }_{-0.12   }$  & 1.10 & 1.15 &  49\\
ABELL 0586 &    70 &   635 & 4.71  & 6.47   $^{+0.55   }_{-0.47   }$  & 8.06   $^{+1.46   }_{-1.11   }$  & 1.25   $^{+0.25   }_{-0.19   }$  & 0.56$^{+0.17   }_{-0.16   }$  & 0.91 & 0.81 &  82\\
ABELL 0611 &    70 &   523 & 4.99  & 7.06   $^{+0.55   }_{-0.48   }$  & 7.97   $^{+1.09   }_{-0.91   }$  & 1.13   $^{+0.18   }_{-0.15   }$  & 0.35$^{+0.11   }_{-0.10   }$  & 0.97 & 0.98 &  54\\
ABELL 0665 &    69 &   617 & 4.24  & 7.45   $^{+0.38   }_{-0.34   }$  & 9.61   $^{+1.02   }_{-0.85   }$  & 1.29   $^{+0.15   }_{-0.13   }$  & 0.31$^{+0.06   }_{-0.07   }$  & 1.02 & 0.93 &  87\\
ABELL 0697 &    69 &   612 & 3.34  & 9.52   $^{+0.87   }_{-0.76   }$  & 12.24  $^{+2.05   }_{-1.63   }$  & 1.29   $^{+0.25   }_{-0.20   }$  & 0.37$^{+0.12   }_{-0.11   }$  & 1.08 & 1.02 &  89\\
ABELL 0773 &    69 &   615 & 1.46  & 7.83   $^{+0.66   }_{-0.57   }$  & 9.75   $^{+1.65   }_{-1.27   }$  & 1.25   $^{+0.24   }_{-0.19   }$  & 0.44$^{+0.12   }_{-0.12   }$  & 1.06 & 1.09 &  84\\
ABELL 0907 &    69 &   488 & 5.69  & 5.62   $^{+0.18   }_{-0.17   }$  & 6.78   $^{+0.49   }_{-0.43   }$  & 1.21   $^{+0.10   }_{-0.08   }$  & 0.42$^{+0.06   }_{-0.05   }$  & 1.13 & 1.00 &  88\\
ABELL 0963 &    69 &   543 & 1.39  & 6.73   $^{+0.32   }_{-0.30   }$  & 6.98   $^{+0.66   }_{-0.57   }$  & 1.04   $^{+0.11   }_{-0.10   }$  & 0.29$^{+0.07   }_{-0.08   }$  & 1.06 & 1.02 &  64\\
ABELL 1063S &    69 &   648 & 1.77  & 11.96  $^{+0.88   }_{-0.79   }$  & 13.70  $^{+1.68   }_{-1.38   }$  & 1.15   $^{+0.16   }_{-0.14   }$  & 0.38$^{+0.09   }_{-0.09   }$  & 1.02 & 0.98 &  90\\
ABELL 1204 &    70 &   419 & 1.44  & 3.63   $^{+0.18   }_{-0.16   }$  & 4.58   $^{+0.57   }_{-0.45   }$  & 1.26   $^{+0.17   }_{-0.14   }$  & 0.31$^{+0.09   }_{-0.09   }$  & 1.06 & 0.90 &  88\\
ABELL 1423 &    70 &   614 & 1.60  & 6.01   $^{+0.75   }_{-0.64   }$  & 7.53   $^{+2.35   }_{-1.55   }$  & 1.25   $^{+0.42   }_{-0.29   }$  & 0.30$^{+0.18   }_{-0.17   }$  & 0.87 & 0.65 &  78\\
ABELL 1651 &    70 &   596 & 2.02  & 6.26   $^{+0.30   }_{-0.27   }$  & 7.78   $^{+0.90   }_{-0.76   }$  & 1.24   $^{+0.16   }_{-0.13   }$  & 0.42$^{+0.09   }_{-0.09   }$  & 1.19 & 1.20 &  86\\
ABELL 1689 $\star$ &    70 &   679 & 1.87  & 9.48   $^{+0.38   }_{-0.35   }$  & 12.89  $^{+1.23   }_{-1.01   }$  & 1.36   $^{+0.14   }_{-0.12   }$  & 0.36$^{+0.06   }_{-0.05   }$  & 1.13 & 1.02 &  91\\
ABELL 1758 &    69 &   574 & 1.09  & 12.14  $^{+1.15   }_{-0.92   }$  & 11.16  $^{+3.08   }_{-2.14   }$  & 0.92   $^{+0.27   }_{-0.19   }$  & 0.56$^{+0.13   }_{-0.13   }$  & 1.21 & 1.09 &  58\\
ABELL 1763 &    69 &   561 & 0.82  & 7.78   $^{+0.67   }_{-0.60   }$  & 11.49  $^{+2.89   }_{-1.84   }$  & 1.48   $^{+0.39   }_{-0.26   }$  & 0.25$^{+0.11   }_{-0.10   }$  & 1.12 & 0.92 &  84\\
ABELL 1835 &    70 &   570 & 2.36  & 9.77   $^{+0.57   }_{-0.52   }$  & 11.00  $^{+1.23   }_{-1.03   }$  & 1.13   $^{+0.14   }_{-0.12   }$  & 0.31$^{+0.08   }_{-0.07   }$  & 0.98 & 1.02 &  86\\
ABELL 1914 &    70 &   698 & 0.97  & 9.62   $^{+0.55   }_{-0.49   }$  & 11.42  $^{+1.26   }_{-1.06   }$  & 1.19   $^{+0.15   }_{-0.13   }$  & 0.30$^{+0.08   }_{-0.07   }$  & 1.07 & 1.03 &  92\\
ABELL 1942 &    69 &   473 & 2.75  & 4.77   $^{+0.38   }_{-0.35   }$  & 5.49   $^{+0.98   }_{-0.74   }$  & 1.15   $^{+0.22   }_{-0.18   }$  & 0.33$^{+0.12   }_{-0.14   }$  & 1.06 & 1.04 &  70\\
ABELL 1995 &    71 &   381 & 1.44  & 8.37   $^{+0.70   }_{-0.61   }$  & 9.23   $^{+1.44   }_{-1.13   }$  & 1.10   $^{+0.20   }_{-0.16   }$  & 0.39$^{+0.12   }_{-0.11   }$  & 1.02 & 0.96 &  74\\
ABELL 2034 &    69 &   594 & 1.58  & 7.15   $^{+0.23   }_{-0.22   }$  & 10.02  $^{+0.92   }_{-0.75   }$  & 1.40   $^{+0.14   }_{-0.11   }$  & 0.32$^{+0.05   }_{-0.05   }$  & 1.22 & 1.00 &  84\\
ABELL 2069 &    70 &   623 & 1.97  & 6.50   $^{+0.33   }_{-0.29   }$  & 8.61   $^{+1.02   }_{-0.84   }$  & 1.32   $^{+0.17   }_{-0.14   }$  & 0.26$^{+0.08   }_{-0.07   }$  & 1.04 & 0.96 &  71\\
ABELL 2111 &    70 &   592 & 2.20  & 7.13   $^{+1.29   }_{-0.95   }$  & 11.10  $^{+4.67   }_{-3.05   }$  & 1.56   $^{+0.71   }_{-0.48   }$  & 0.13$^{+0.19   }_{-0.13   }$  & 1.06 & 0.88 &  76\\
ABELL 2125 &    70 &   371 & 2.75  & 2.88   $^{+0.30   }_{-0.27   }$  & 3.76   $^{+0.98   }_{-0.65   }$  & 1.31   $^{+0.37   }_{-0.26   }$  & 0.31$^{+0.18   }_{-0.16   }$  & 1.26 & 1.30 &  61\\
ABELL 2163 &    69 &   751 & 12.04 & 19.20  $^{+0.87   }_{-0.80   }$  & 21.30  $^{+1.77   }_{-1.47   }$  & 1.11   $^{+0.11   }_{-0.09   }$  & 0.10$^{+0.06   }_{-0.06   }$  & 1.37 & 1.26 &  90\\
ABELL 2204 &    70 &   575 & 5.84  & 8.65   $^{+0.58   }_{-0.52   }$  & 10.57  $^{+1.48   }_{-1.23   }$  & 1.22   $^{+0.19   }_{-0.16   }$  & 0.37$^{+0.10   }_{-0.09   }$  & 0.95 & 1.00 &  90\\
ABELL 2218 &    70 &   558 & 3.12  & 7.35   $^{+0.39   }_{-0.35   }$  & 10.03  $^{+1.26   }_{-0.98   }$  & 1.36   $^{+0.19   }_{-0.15   }$  & 0.22$^{+0.07   }_{-0.06   }$  & 1.01 & 0.90 &  87\\
ABELL 2255 &    71 &   596 & 2.53  & 6.12   $^{+0.20   }_{-0.19   }$  & 8.10   $^{+0.66   }_{-0.58   }$  & 1.32   $^{+0.12   }_{-0.10   }$  & 0.30$^{+0.06   }_{-0.06   }$  & 1.13 & 0.95 &  76\\
ABELL 2259 &    69 &   480 & 3.70  & 5.18   $^{+0.46   }_{-0.39   }$  & 6.40   $^{+1.33   }_{-0.95   }$  & 1.24   $^{+0.28   }_{-0.21   }$  & 0.41$^{+0.14   }_{-0.14   }$  & 1.05 & 1.01 &  85\\
ABELL 2261 &    69 &   576 & 3.31  & 7.63   $^{+0.47   }_{-0.43   }$  & 9.30   $^{+1.21   }_{-0.91   }$  & 1.22   $^{+0.18   }_{-0.14   }$  & 0.36$^{+0.08   }_{-0.08   }$  & 0.99 & 0.95 &  90\\
ABELL 2294 &    69 &   572 & 6.10  & 9.98   $^{+1.43   }_{-1.12   }$  & 11.07  $^{+3.19   }_{-2.11   }$  & 1.11   $^{+0.36   }_{-0.25   }$  & 0.53$^{+0.21   }_{-0.21   }$  & 1.07 & 0.95 &  82\\
ABELL 2384 &    70 &   436 & 2.99  & 4.75   $^{+0.22   }_{-0.20   }$  & 6.22   $^{+0.72   }_{-0.60   }$  & 1.31   $^{+0.16   }_{-0.14   }$  & 0.23$^{+0.07   }_{-0.07   }$  & 1.06 & 0.92 &  81\\
ABELL 2409 &    70 &   511 & 6.72  & 5.94   $^{+0.43   }_{-0.38   }$  & 6.77   $^{+0.99   }_{-0.82   }$  & 1.14   $^{+0.19   }_{-0.16   }$  & 0.37$^{+0.13   }_{-0.11   }$  & 1.13 & 0.96 &  88\\
ABELL 2537 &    69 &   497 & 4.26  & 8.40   $^{+0.76   }_{-0.68   }$  & 7.81   $^{+1.15   }_{-0.93   }$  & 0.93   $^{+0.16   }_{-0.13   }$  & 0.40$^{+0.13   }_{-0.13   }$  & 0.91 & 0.84 &  46\\
ABELL 2631 &    70 &   631 & 3.74  & 7.06   $^{+1.06   }_{-0.84   }$  & 7.83   $^{+2.18   }_{-1.45   }$  & 1.11   $^{+0.35   }_{-0.24   }$  & 0.34$^{+0.19   }_{-0.18   }$  & 0.97 & 0.88 &  83\\
ABELL 2667 &    70 &   525 & 1.64  & 6.75   $^{+0.48   }_{-0.43   }$  & 7.45   $^{+1.06   }_{-0.88   }$  & 1.10   $^{+0.18   }_{-0.15   }$  & 0.36$^{+0.11   }_{-0.11   }$  & 1.17 & 1.08 &  76\\
ABELL 2670 &    69 &   451 & 2.88  & 3.95   $^{+0.14   }_{-0.12   }$  & 4.65   $^{+0.42   }_{-0.36   }$  & 1.18   $^{+0.11   }_{-0.10   }$  & 0.42$^{+0.08   }_{-0.06   }$  & 1.13 & 1.07 &  70\\
ABELL 2717 &    70 &   298 & 1.12  & 2.63   $^{+0.17   }_{-0.16   }$  & 3.17   $^{+0.58   }_{-0.43   }$  & 1.21   $^{+0.23   }_{-0.18   }$  & 0.48$^{+0.13   }_{-0.10   }$  & 0.88 & 0.87 &  55\\
ABELL 2744 &    71 &   647 & 1.82  & 9.18   $^{+0.68   }_{-0.60   }$  & 10.20  $^{+1.38   }_{-1.10   }$  & 1.11   $^{+0.17   }_{-0.14   }$  & 0.24$^{+0.10   }_{-0.09   }$  & 0.99 & 0.90 &  67\\
ABELL 3164 &    70 &   451 & 2.55  & 2.83   $^{+0.53   }_{-0.26   }$  & 3.81   $^{+3.56   }_{-1.42   }$  & 1.35   $^{+1.28   }_{-0.52   }$  & 0.39$^{+0.33   }_{-0.21   }$  & 0.88 & 0.94 &  29\\
ABELL 3376 $\star$ &    70 &   463 & 5.21  & 4.48   $^{+0.11   }_{-0.12   }$  & 5.95   $^{+0.47   }_{-0.42   }$  & 1.33   $^{+0.11   }_{-0.10   }$  & 0.39$^{+0.05   }_{-0.08   }$  & 1.16 & 1.09 &  63\\
ABELL 3921 &    69 &   535 & 3.07  & 5.70   $^{+0.24   }_{-0.23   }$  & 6.65   $^{+0.65   }_{-0.54   }$  & 1.17   $^{+0.12   }_{-0.11   }$  & 0.31$^{+0.08   }_{-0.07   }$  & 1.02 & 0.96 &  77\\
AC 114 &    70 &   550 & 1.44  & 7.53   $^{+0.49   }_{-0.44   }$  & 8.30   $^{+1.03   }_{-0.85   }$  & 1.10   $^{+0.15   }_{-0.13   }$  & 0.26$^{+0.08   }_{-0.09   }$  & 1.07 & 1.06 &  55\\
CL 0024+17 &    71 &   435 & 4.36  & 6.03   $^{+1.66   }_{-1.10   }$  & 7.18   $^{+7.91   }_{-3.16   }$  & 1.19   $^{+1.35   }_{-0.57   }$  & 0.60$^{+0.37   }_{-0.33   }$  & 1.00 & 1.44 &  37\\
CL 1221+4918 &    71 &   445 & 1.44  & 6.62   $^{+1.24   }_{-0.99   }$  & 7.11   $^{+1.73   }_{-1.31   }$  & 1.07   $^{+0.33   }_{-0.25   }$  & 0.34$^{+0.20   }_{-0.18   }$  & 0.94 & 0.93 &  62\\
CL J0030+2618 &    70 &   786 & 4.10  & 4.63   $^{+2.72   }_{-1.32   }$  & 5.18   $^{+8.29   }_{-1.96   }$  & 1.12   $^{+1.91   }_{-0.53   }$  & 0.26$^{+0.75   }_{-0.26   }$  & 1.00 & 1.23 &  37\\
CL J0152-1357 &    70 &   391 & 1.45  & 7.33   $^{+2.78   }_{-1.77   }$  & 7.31   $^{+3.43   }_{-2.02   }$  & 1.00   $^{+0.60   }_{-0.37   }$  & 0.00$^{+0.24   }_{-0.00   }$  & 0.89 & 1.00 &  36\\
CL J0542.8-4100 &    71 &   446 & 3.59  & 6.07   $^{+1.47   }_{-1.05   }$  & 6.29   $^{+2.14   }_{-1.41   }$  & 1.04   $^{+0.43   }_{-0.29   }$  & 0.16$^{+0.23   }_{-0.16   }$  & 1.04 & 0.91 &  66\\
CL J0848+4456 $\star$ &    71 &   319 & 2.53  & 4.53   $^{+1.57   }_{-1.13   }$  & 5.52   $^{+3.28   }_{-1.74   }$  & 1.22   $^{+0.84   }_{-0.49   }$  & 0.00$^{+0.45   }_{-0.00   }$  & 0.92 & 0.93 &  58\\
CL J1113.1-2615 &    70 &   435 & 5.51  & 4.19   $^{+1.61   }_{-1.02   }$  & 4.10   $^{+2.47   }_{-1.44   }$  & 0.98   $^{+0.70   }_{-0.42   }$  & 0.46$^{+0.63   }_{-0.44   }$  & 1.01 & 1.08 &  23\\
CL J1226.9+3332 $\star$ &    69 &   450 & 1.37  & 11.81  $^{+2.25   }_{-1.70   }$  & 11.29  $^{+2.45   }_{-1.77   }$  & 0.96   $^{+0.28   }_{-0.20   }$  & 0.21$^{+0.21   }_{-0.21   }$  & 0.81 & 0.86 &  86\\
CL J2302.8+0844 &    70 &   514 & 5.05  & 4.25   $^{+1.17   }_{-1.32   }$  & 4.67   $^{+2.00   }_{-1.80   }$  & 1.10   $^{+0.56   }_{-0.54   }$  & 0.13$^{+0.33   }_{-0.13   }$  & 0.89 & 0.97 &  50\\
DLS J0514-4904 &    70 &   507 & 2.52  & 4.62   $^{+0.53   }_{-0.47   }$  & 6.14   $^{+2.08   }_{-1.34   }$  & 1.33   $^{+0.48   }_{-0.32   }$  & 0.37$^{+0.24   }_{-0.20   }$  & 1.04 & 1.12 &  54\\
MACS J0011.7-1523 $\star$ &    69 &   451 & 2.08  & 6.49   $^{+0.48   }_{-0.43   }$  & 6.76   $^{+0.81   }_{-0.66   }$  & 1.04   $^{+0.15   }_{-0.12   }$  & 0.30$^{+0.10   }_{-0.09   }$  & 0.86 & 0.90 &  87\\
MACS J0025.4-1222 $\star$ &    70 &   473 & 2.72  & 6.33   $^{+0.85   }_{-0.70   }$  & 6.01   $^{+1.05   }_{-0.85   }$  & 0.95   $^{+0.21   }_{-0.17   }$  & 0.37$^{+0.16   }_{-0.15   }$  & 0.90 & 0.92 &  80\\
MACS J0035.4-2015 &    70 &   527 & 1.55  & 7.46   $^{+0.79   }_{-0.66   }$  & 9.31   $^{+1.75   }_{-1.29   }$  & 1.25   $^{+0.27   }_{-0.21   }$  & 0.33$^{+0.12   }_{-0.12   }$  & 0.94 & 0.93 &  90\\
MACS J0111.5+0855 &    70 &   435 & 4.18  & 4.11   $^{+1.61   }_{-1.05   }$  & 3.72   $^{+3.08   }_{-1.29   }$  & 0.91   $^{+0.83   }_{-0.39   }$  & 0.11$^{+0.59   }_{-0.11   }$  & 0.68 & 0.65 &  49\\
MACS J0152.5-2852 &    70 &   459 & 1.46  & 5.64   $^{+0.89   }_{-0.70   }$  & 7.24   $^{+2.57   }_{-1.59   }$  & 1.28   $^{+0.50   }_{-0.32   }$  & 0.22$^{+0.17   }_{-0.17   }$  & 1.10 & 1.02 &  84\\
MACS J0159.0-3412 &    70 &   572 & 1.54  & 10.90  $^{+4.77   }_{-2.53   }$  & 14.65  $^{+12.31  }_{-5.39   }$  & 1.34   $^{+1.27   }_{-0.58   }$  & 0.26$^{+0.35   }_{-0.26   }$  & 0.87 & 0.92 &  81\\
MACS J0159.8-0849 $\star$ &    69 &   585 & 2.01  & 9.16   $^{+0.71   }_{-0.63   }$  & 9.83   $^{+1.13   }_{-0.96   }$  & 1.07   $^{+0.15   }_{-0.13   }$  & 0.30$^{+0.09   }_{-0.09   }$  & 1.08 & 1.09 &  90\\
MACS J0242.5-2132 &    70 &   498 & 2.71  & 5.58   $^{+0.63   }_{-0.52   }$  & 6.26   $^{+1.38   }_{-0.99   }$  & 1.12   $^{+0.28   }_{-0.21   }$  & 0.34$^{+0.16   }_{-0.15   }$  & 1.03 & 0.83 &  87\\
MACS J0257.1-2325 $\star$ &    70 &   579 & 2.09  & 9.25   $^{+1.28   }_{-1.01   }$  & 10.16  $^{+1.95   }_{-1.54   }$  & 1.10   $^{+0.26   }_{-0.21   }$  & 0.14$^{+0.12   }_{-0.12   }$  & 0.99 & 1.08 &  84\\
MACS J0257.6-2209 &    69 &   540 & 2.02  & 8.02   $^{+1.12   }_{-0.88   }$  & 8.17   $^{+1.92   }_{-1.30   }$  & 1.02   $^{+0.28   }_{-0.20   }$  & 0.30$^{+0.16   }_{-0.17   }$  & 1.12 & 1.26 &  84\\
MACS J0308.9+2645 &    69 &   539 & 11.88 & 10.54  $^{+1.28   }_{-1.07   }$  & 11.38  $^{+2.16   }_{-1.66   }$  & 1.08   $^{+0.24   }_{-0.19   }$  & 0.28$^{+0.13   }_{-0.14   }$  & 0.97 & 1.01 &  87\\
MACS J0329.6-0211 $\star$ &    70 &   420 & 6.21  & 6.30   $^{+0.47   }_{-0.41   }$  & 7.50   $^{+0.83   }_{-0.69   }$  & 1.19   $^{+0.16   }_{-0.13   }$  & 0.41$^{+0.10   }_{-0.09   }$  & 1.10 & 1.17 &  86\\
MACS J0404.6+1109 &    70 &   494 & 14.96 & 5.77   $^{+1.14   }_{-0.88   }$  & 6.15   $^{+2.00   }_{-1.30   }$  & 1.07   $^{+0.41   }_{-0.28   }$  & 0.24$^{+0.22   }_{-0.20   }$  & 0.85 & 0.78 &  73\\
MACS J0417.5-1154 &    70 &   429 & 4.00  & 11.07  $^{+1.98   }_{-1.49   }$  & 14.90  $^{+5.03   }_{-3.24   }$  & 1.35   $^{+0.51   }_{-0.34   }$  & 0.33$^{+0.19   }_{-0.19   }$  & 1.07 & 0.97 &  94\\
MACS J0429.6-0253 &    69 &   495 & 5.70  & 5.66   $^{+0.64   }_{-0.54   }$  & 6.71   $^{+1.26   }_{-0.98   }$  & 1.19   $^{+0.26   }_{-0.21   }$  & 0.35$^{+0.14   }_{-0.13   }$  & 1.21 & 1.12 &  82\\
MACS J0451.9+0006 &    70 &   459 & 7.65  & 5.80   $^{+1.46   }_{-1.03   }$  & 7.02   $^{+3.29   }_{-1.80   }$  & 1.21   $^{+0.64   }_{-0.38   }$  & 0.51$^{+0.33   }_{-0.29   }$  & 1.25 & 1.35 &  83\\
MACS J0455.2+0657 &    71 &   481 & 10.45 & 7.25   $^{+2.04   }_{-1.33   }$  & 8.25   $^{+3.98   }_{-2.10   }$  & 1.14   $^{+0.64   }_{-0.36   }$  & 0.56$^{+0.37   }_{-0.33   }$  & 0.83 & 0.94 &  82\\
MACS J0520.7-1328 &    69 &   492 & 8.88  & 6.35   $^{+0.81   }_{-0.67   }$  & 8.22   $^{+2.18   }_{-1.45   }$  & 1.29   $^{+0.38   }_{-0.27   }$  & 0.43$^{+0.17   }_{-0.16   }$  & 1.23 & 1.38 &  86\\
MACS J0547.0-3904 &    69 &   364 & 4.08  & 3.58   $^{+0.44   }_{-0.37   }$  & 5.41   $^{+1.67   }_{-1.18   }$  & 1.51   $^{+0.50   }_{-0.36   }$  & 0.09$^{+0.15   }_{-0.09   }$  & 1.16 & 1.42 &  75\\
MACS J0553.4-3342 &    70 &   692 & 2.88  & 13.14  $^{+3.82   }_{-2.50   }$  & 13.86  $^{+6.45   }_{-3.44   }$  & 1.05   $^{+0.58   }_{-0.33   }$  & 0.57$^{+0.35   }_{-0.33   }$  & 0.80 & 0.76 &  87\\
MACS J0717.5+3745 $\star$ &    70 &   563 & 6.75  & 12.77  $^{+1.16   }_{-1.00   }$  & 13.21  $^{+1.58   }_{-1.29   }$  & 1.03   $^{+0.16   }_{-0.13   }$  & 0.30$^{+0.10   }_{-0.11   }$  & 0.93 & 0.90 &  88\\
MACS J0744.8+3927 $\star$ &    70 &   537 & 4.66  & 8.09   $^{+0.77   }_{-0.66   }$  & 8.77   $^{+1.04   }_{-0.87   }$  & 1.08   $^{+0.16   }_{-0.14   }$  & 0.32$^{+0.10   }_{-0.10   }$  & 1.14 & 1.18 &  82\\
MACS J0911.2+1746 $\star$ &    70 &   541 & 3.55  & 7.51   $^{+1.27   }_{-0.99   }$  & 7.17   $^{+1.60   }_{-1.20   }$  & 0.95   $^{+0.27   }_{-0.20   }$  & 0.21$^{+0.17   }_{-0.16   }$  & 0.93 & 0.84 &  78\\
MACS J0949+1708 &    70 &   580 & 3.17  & 9.16   $^{+1.53   }_{-1.18   }$  & 9.11   $^{+2.27   }_{-1.55   }$  & 0.99   $^{+0.30   }_{-0.21   }$  & 0.37$^{+0.20   }_{-0.20   }$  & 0.89 & 0.84 &  89\\
MACS J1006.9+3200 &    70 &   512 & 1.83  & 7.89   $^{+2.78   }_{-1.74   }$  & 8.05   $^{+5.70   }_{-2.45   }$  & 1.02   $^{+0.81   }_{-0.38   }$  & 0.15$^{+0.35   }_{-0.15   }$  & 1.84 & 1.15 &  76\\
MACS J1105.7-1014 &    71 &   502 & 4.58  & 7.54   $^{+2.29   }_{-1.51   }$  & 7.78   $^{+3.93   }_{-1.97   }$  & 1.03   $^{+0.61   }_{-0.33   }$  & 0.22$^{+0.29   }_{-0.22   }$  & 1.17 & 1.27 &  81\\
MACS J1108.8+0906 $\star$ &    70 &   491 & 2.52  & 6.52   $^{+0.94   }_{-0.82   }$  & 7.31   $^{+1.89   }_{-1.29   }$  & 1.12   $^{+0.33   }_{-0.24   }$  & 0.29$^{+0.18   }_{-0.17   }$  & 0.95 & 0.80 &  80\\
MACS J1115.2+5320 $\star$ &    70 &   527 & 0.98  & 8.91   $^{+1.42   }_{-1.12   }$  & 9.58   $^{+2.36   }_{-1.62   }$  & 1.08   $^{+0.32   }_{-0.23   }$  & 0.37$^{+0.20   }_{-0.18   }$  & 0.93 & 0.88 &  75\\
MACS J1115.8+0129 &    70 &   448 & 4.36  & 6.78   $^{+1.17   }_{-0.91   }$  & 8.27   $^{+3.27   }_{-2.16   }$  & 1.22   $^{+0.53   }_{-0.36   }$  & 0.07$^{+0.21   }_{-0.07   }$  & 1.00 & 0.97 &  65\\
MACS J1131.8-1955 &    69 &   576 & 4.49  & 8.64   $^{+1.23   }_{-0.97   }$  & 11.01  $^{+3.61   }_{-2.10   }$  & 1.27   $^{+0.46   }_{-0.28   }$  & 0.42$^{+0.17   }_{-0.17   }$  & 1.00 & 1.00 &  87\\
MACS J1149.5+2223 $\star$ &    69 &   504 & 2.32  & 7.65   $^{+0.89   }_{-0.75   }$  & 8.13   $^{+1.36   }_{-1.04   }$  & 1.06   $^{+0.22   }_{-0.17   }$  & 0.20$^{+0.12   }_{-0.11   }$  & 1.00 & 1.09 &  87\\
MACS J1206.2-0847 &    70 &   522 & 4.15  & 10.21  $^{+1.19   }_{-0.97   }$  & 12.51  $^{+2.44   }_{-1.87   }$  & 1.23   $^{+0.28   }_{-0.22   }$  & 0.33$^{+0.13   }_{-0.13   }$  & 0.96 & 1.05 &  93\\
MACS J1226.8+2153 &    71 &   489 & 1.82  & 4.21   $^{+1.07   }_{-0.80   }$  & 5.02   $^{+3.29   }_{-1.52   }$  & 1.19   $^{+0.84   }_{-0.43   }$  & 0.23$^{+0.38   }_{-0.23   }$  & 1.02 & 0.81 &  67\\
MACS J1311.0-0310 $\star$ &    69 &   425 & 2.18  & 5.76   $^{+0.48   }_{-0.42   }$  & 5.91   $^{+0.73   }_{-0.62   }$  & 1.03   $^{+0.15   }_{-0.13   }$  & 0.39$^{+0.13   }_{-0.11   }$  & 0.96 & 0.98 &  72\\
MACS J1319+7003 &    70 &   496 & 1.53  & 7.99   $^{+2.08   }_{-1.43   }$  & 10.62  $^{+7.35   }_{-3.22   }$  & 1.33   $^{+0.98   }_{-0.47   }$  & 0.30$^{+0.29   }_{-0.28   }$  & 1.25 & 1.24 &  74\\
MACS J1427.2+4407 &    71 &   488 & 1.41  & 9.80   $^{+3.87   }_{-2.53   }$  & 10.35  $^{+6.30   }_{-3.26   }$  & 1.06   $^{+0.77   }_{-0.43   }$  & 0.00$^{+0.34   }_{-0.00   }$  & 0.67 & 0.50 &  84\\
MACS J1427.6-2521 &    71 &   426 & 6.11  & 4.65   $^{+0.92   }_{-0.72   }$  & 8.11   $^{+5.04   }_{-2.77   }$  & 1.74   $^{+1.14   }_{-0.65   }$  & 0.18$^{+0.26   }_{-0.18   }$  & 1.19 & 1.40 &  68\\
MACS J1621.3+3810 $\star$ &    69 &   504 & 1.07  & 7.12   $^{+0.66   }_{-0.55   }$  & 7.09   $^{+0.92   }_{-0.75   }$  & 1.00   $^{+0.16   }_{-0.13   }$  & 0.34$^{+0.11   }_{-0.11   }$  & 0.93 & 0.86 &  73\\
MACS J1731.6+2252 &    71 &   521 & 6.48  & 7.45   $^{+1.32   }_{-0.99   }$  & 10.99  $^{+4.67   }_{-2.46   }$  & 1.48   $^{+0.68   }_{-0.38   }$  & 0.35$^{+0.19   }_{-0.17   }$  & 1.20 & 1.07 &  84\\
MACS J1931.8-2634 &    70 &   535 & 9.13  & 6.97   $^{+0.72   }_{-0.61   }$  & 7.72   $^{+1.31   }_{-0.99   }$  & 1.11   $^{+0.22   }_{-0.17   }$  & 0.27$^{+0.11   }_{-0.12   }$  & 0.95 & 0.86 &  90\\
MACS J2046.0-3430 &    71 &   386 & 4.98  & 4.64   $^{+1.18   }_{-0.82   }$  & 5.49   $^{+2.29   }_{-1.47   }$  & 1.18   $^{+0.58   }_{-0.38   }$  & 0.20$^{+0.32   }_{-0.20   }$  & 0.89 & 1.11 &  82\\
MACS J2049.9-3217 &    69 &   524 & 5.99  & 6.83   $^{+0.84   }_{-0.69   }$  & 8.94   $^{+2.08   }_{-1.48   }$  & 1.31   $^{+0.34   }_{-0.25   }$  & 0.43$^{+0.17   }_{-0.15   }$  & 0.99 & 0.92 &  83\\
MACS J2211.7-0349 &    69 &   663 & 5.86  & 11.30  $^{+1.46   }_{-1.17   }$  & 13.82  $^{+3.54   }_{-2.41   }$  & 1.22   $^{+0.35   }_{-0.25   }$  & 0.15$^{+0.13   }_{-0.14   }$  & 1.24 & 1.26 &  88\\
MACS J2214.9-1359 $\star$ &    70 &   529 & 3.32  & 9.78   $^{+1.38   }_{-1.09   }$  & 10.45  $^{+2.19   }_{-1.56   }$  & 1.07   $^{+0.27   }_{-0.20   }$  & 0.23$^{+0.14   }_{-0.14   }$  & 0.99 & 1.06 &  87\\
MACS J2228+2036 &    70 &   545 & 4.52  & 7.86   $^{+1.08   }_{-0.85   }$  & 9.17   $^{+2.05   }_{-1.46   }$  & 1.17   $^{+0.31   }_{-0.22   }$  & 0.39$^{+0.16   }_{-0.15   }$  & 0.99 & 1.00 &  88\\
MACS J2229.7-2755 &    69 &   465 & 1.34  & 5.01   $^{+0.50   }_{-0.43   }$  & 5.79   $^{+1.11   }_{-0.86   }$  & 1.16   $^{+0.25   }_{-0.20   }$  & 0.55$^{+0.19   }_{-0.18   }$  & 1.05 & 1.08 &  85\\
MACS J2243.3-0935 &    71 &   574 & 4.31  & 4.09   $^{+0.51   }_{-0.45   }$  & 7.20   $^{+3.17   }_{-2.12   }$  & 1.76   $^{+0.81   }_{-0.55   }$  & 0.03$^{+0.15   }_{-0.03   }$  & 1.17 & 0.92 &  51\\
MACS J2245.0+2637 &    69 &   454 & 5.50  & 6.06   $^{+0.63   }_{-0.54   }$  & 6.76   $^{+1.24   }_{-0.93   }$  & 1.12   $^{+0.24   }_{-0.18   }$  & 0.60$^{+0.20   }_{-0.18   }$  & 0.94 & 1.09 &  88\\
MACS J2311+0338 &    70 &   363 & 5.23  & 8.12   $^{+1.44   }_{-1.16   }$  & 12.40  $^{+5.12   }_{-2.88   }$  & 1.53   $^{+0.69   }_{-0.42   }$  & 0.46$^{+0.22   }_{-0.20   }$  & 1.07 & 1.15 &  88\\
MKW3S &    70 &   339 & 3.05  & 3.91   $^{+0.06   }_{-0.06   }$  & 4.58   $^{+0.18   }_{-0.18   }$  & 1.17   $^{+0.05   }_{-0.05   }$  & 0.34$^{+0.03   }_{-0.04   }$  & 1.38 & 0.97 &  86\\
MS 0016.9+1609 &    69 &   550 & 4.06  & 8.94   $^{+0.71   }_{-0.62   }$  & 9.78   $^{+1.09   }_{-0.90   }$  & 1.09   $^{+0.15   }_{-0.13   }$  & 0.29$^{+0.09   }_{-0.08   }$  & 0.91 & 0.88 &  83\\
MS 0451.6-0305 &    70 &   536 & 5.68  & 8.90   $^{+0.85   }_{-0.72   }$  & 10.43  $^{+1.59   }_{-1.26   }$  & 1.17   $^{+0.21   }_{-0.17   }$  & 0.37$^{+0.11   }_{-0.11   }$  & 1.00 & 0.93 &  60\\
MS 0735.6+7421 &    69 &   491 & 3.40  & 5.55   $^{+0.24   }_{-0.22   }$  & 6.34   $^{+0.57   }_{-0.50   }$  & 1.14   $^{+0.11   }_{-0.10   }$  & 0.35$^{+0.07   }_{-0.06   }$  & 1.05 & 1.05 &  62\\
MS 0839.8+2938 &    70 &   415 & 3.92  & 4.68   $^{+0.32   }_{-0.29   }$  & 5.05   $^{+0.82   }_{-0.65   }$  & 1.08   $^{+0.19   }_{-0.15   }$  & 0.46$^{+0.13   }_{-0.12   }$  & 0.90 & 0.87 &  60\\
MS 0906.5+1110 &    70 &   616 & 3.60  & 5.38   $^{+0.33   }_{-0.29   }$  & 6.76   $^{+0.92   }_{-0.77   }$  & 1.26   $^{+0.19   }_{-0.16   }$  & 0.27$^{+0.09   }_{-0.09   }$  & 1.21 & 1.08 &  75\\
MS 1006.0+1202 &    70 &   556 & 3.63  & 5.61   $^{+0.51   }_{-0.43   }$  & 7.48   $^{+1.66   }_{-1.22   }$  & 1.33   $^{+0.32   }_{-0.24   }$  & 0.24$^{+0.11   }_{-0.12   }$  & 1.30 & 1.34 &  75\\
MS 1008.1-1224 &    70 &   548 & 6.71  & 5.65   $^{+0.49   }_{-0.43   }$  & 9.01   $^{+1.95   }_{-1.38   }$  & 1.59   $^{+0.37   }_{-0.27   }$  & 0.26$^{+0.11   }_{-0.10   }$  & 1.21 & 0.98 &  78\\
MS 1054.5-0321 &    70 &   558 & 3.69  & 9.38   $^{+1.72   }_{-1.34   }$  & 9.91   $^{+2.66   }_{-1.77   }$  & 1.06   $^{+0.34   }_{-0.24   }$  & 0.13$^{+0.17   }_{-0.13   }$  & 1.02 & 1.03 &  41\\
MS 1455.0+2232 &    69 &   436 & 3.35  & 4.77   $^{+0.13   }_{-0.13   }$  & 5.37   $^{+0.36   }_{-0.22   }$  & 1.13   $^{+0.08   }_{-0.06   }$  & 0.44$^{+0.05   }_{-0.05   }$  & 1.29 & 1.10 &  90\\
MS 1621.5+2640 &    70 &   537 & 3.59  & 6.11   $^{+0.95   }_{-0.76   }$  & 6.22   $^{+1.56   }_{-1.10   }$  & 1.02   $^{+0.30   }_{-0.22   }$  & 0.40$^{+0.23   }_{-0.21   }$  & 1.02 & 1.21 &  68\\
MS 2053.7-0449 $\star$ &    70 &   561 & 5.16  & 3.66   $^{+0.81   }_{-0.60   }$  & 4.07   $^{+1.23   }_{-0.83   }$  & 1.11   $^{+0.42   }_{-0.29   }$  & 0.39$^{+0.38   }_{-0.33   }$  & 0.97 & 1.07 &  58\\
MS 2137.3-2353 &    70 &   502 & 3.40  & 6.01   $^{+0.52   }_{-0.46   }$  & 7.48   $^{+1.68   }_{-1.09   }$  & 1.24   $^{+0.30   }_{-0.20   }$  & 0.45$^{+0.13   }_{-0.14   }$  & 1.12 & 1.25 &  55\\
PKS 0745-191 &    69 &   651 & 40.80 & 8.13   $^{+0.37   }_{-0.34   }$  & 9.68   $^{+0.83   }_{-0.72   }$  & 1.19   $^{+0.12   }_{-0.10   }$  & 0.38$^{+0.06   }_{-0.06   }$  & 1.02 & 0.98 &  89\\
RBS 0797 &    69 &   493 & 2.22  & 7.68   $^{+0.92   }_{-0.77   }$  & 9.05   $^{+1.80   }_{-1.33   }$  & 1.18   $^{+0.27   }_{-0.21   }$  & 0.32$^{+0.14   }_{-0.13   }$  & 1.07 & 1.06 &  89\\
RDCS 1252-29 &    71 &   276 & 6.06  & 4.25   $^{+1.82   }_{-1.14   }$  & 4.47   $^{+2.16   }_{-1.29   }$  & 1.05   $^{+0.68   }_{-0.41   }$  & 0.79$^{+1.01   }_{-0.62   }$  & 1.07 & 1.17 &  50\\
RX J0232.2-4420 &    69 &   568 & 2.53  & 7.83   $^{+0.77   }_{-0.68   }$  & 9.92   $^{+2.11   }_{-1.44   }$  & 1.27   $^{+0.30   }_{-0.21   }$  & 0.36$^{+0.12   }_{-0.13   }$  & 1.13 & 1.09 &  85\\
RX J0340-4542 &    70 &   412 & 1.63  & 3.16   $^{+0.38   }_{-0.35   }$  & 2.80   $^{+0.94   }_{-0.57   }$  & 0.89   $^{+0.32   }_{-0.21   }$  & 0.62$^{+0.31   }_{-0.25   }$  & 1.27 & 1.22 &  43\\
RX J0439+0520 &    70 &   474 & 10.02 & 4.60   $^{+0.64   }_{-0.59   }$  & 4.95   $^{+1.28   }_{-0.88   }$  & 1.08   $^{+0.32   }_{-0.24   }$  & 0.44$^{+0.29   }_{-0.24   }$  & 1.03 & 1.14 &  77\\
RX J0439.0+0715 $\star$ &    70 &   532 & 11.16 & 5.63   $^{+0.36   }_{-0.32   }$  & 8.02   $^{+1.25   }_{-0.93   }$  & 1.42   $^{+0.24   }_{-0.18   }$  & 0.32$^{+0.10   }_{-0.08   }$  & 1.28 & 1.16 &  82\\
RX J0528.9-3927 &    70 &   640 & 2.36  & 7.89   $^{+0.96   }_{-0.76   }$  & 8.91   $^{+2.30   }_{-1.42   }$  & 1.13   $^{+0.32   }_{-0.21   }$  & 0.27$^{+0.14   }_{-0.14   }$  & 0.92 & 0.93 &  83\\
RX J0647.7+7015 $\star$ &    69 &   512 & 5.18  & 11.28  $^{+1.85   }_{-1.45   }$  & 11.01  $^{+2.17   }_{-1.63   }$  & 0.98   $^{+0.25   }_{-0.19   }$  & 0.20$^{+0.17   }_{-0.17   }$  & 1.02 & 1.00 &  80\\
RX J0910+5422 $\star$ &    71 &   246 & 2.07  & 4.53   $^{+3.02   }_{-1.70   }$  & 5.98   $^{+5.30   }_{-2.49   }$  & 1.32   $^{+1.46   }_{-0.74   }$  & 0.00$^{+0.73   }_{-0.00   }$  & 0.90 & 0.71 &  31\\
RX J1347.5-1145 $\star$ &    70 &   607 & 4.89  & 14.62  $^{+0.97   }_{-0.79   }$  & 16.62  $^{+1.54   }_{-1.24   }$  & 1.14   $^{+0.13   }_{-0.10   }$  & 0.32$^{+0.08   }_{-0.07   }$  & 1.12 & 1.12 &  93\\
RX J1350+6007 &    71 &   334 & 1.77  & 4.48   $^{+2.32   }_{-1.49   }$  & 5.31   $^{+3.02   }_{-2.07   }$  & 1.19   $^{+0.91   }_{-0.61   }$  & 0.13$^{+1.23   }_{-0.13   }$  & 0.82 & 0.72 &  57\\
RX J1423.8+2404 $\star$ &    71 &   441 & 2.65  & 6.64   $^{+0.38   }_{-0.34   }$  & 7.01   $^{+0.59   }_{-0.51   }$  & 1.06   $^{+0.11   }_{-0.09   }$  & 0.37$^{+0.07   }_{-0.07   }$  & 1.02 & 0.98 &  86\\
RX J1504.1-0248 &    70 &   628 & 6.27  & 8.00   $^{+0.27   }_{-0.24   }$  & 8.92   $^{+0.52   }_{-0.46   }$  & 1.11   $^{+0.08   }_{-0.07   }$  & 0.40$^{+0.04   }_{-0.05   }$  & 1.29 & 1.25 &  91\\
RX J1525+0958 &    70 &   416 & 2.96  & 3.74   $^{+0.63   }_{-0.45   }$  & 6.96   $^{+2.88   }_{-1.73   }$  & 1.86   $^{+0.83   }_{-0.51   }$  & 0.67$^{+0.36   }_{-0.29   }$  & 1.29 & 0.93 &  79\\
RX J1532.9+3021 $\star$ &    70 &   458 & 2.21  & 6.03   $^{+0.42   }_{-0.38   }$  & 6.95   $^{+0.88   }_{-0.72   }$  & 1.15   $^{+0.17   }_{-0.14   }$  & 0.42$^{+0.11   }_{-0.10   }$  & 0.94 & 1.05 &  73\\
RX J1716.9+6708 &    71 &   486 & 3.71  & 5.71   $^{+1.47   }_{-1.06   }$  & 5.77   $^{+1.88   }_{-1.28   }$  & 1.01   $^{+0.42   }_{-0.29   }$  & 0.68$^{+0.42   }_{-0.35   }$  & 0.79 & 0.74 &  55\\
RX J1720.1+2638 &    69 &   510 & 4.02  & 6.37   $^{+0.28   }_{-0.26   }$  & 7.78   $^{+0.69   }_{-0.61   }$  & 1.22   $^{+0.12   }_{-0.11   }$  & 0.35$^{+0.07   }_{-0.06   }$  & 1.10 & 1.02 &  90\\
RX J1720.2+3536 $\star$ &    71 &   455 & 3.35  & 7.21   $^{+0.53   }_{-0.46   }$  & 6.97   $^{+0.76   }_{-0.59   }$  & 0.97   $^{+0.13   }_{-0.10   }$  & 0.41$^{+0.10   }_{-0.10   }$  & 1.12 & 1.09 &  85\\
RX J2011.3-5725 &    71 &   416 & 4.76  & 3.94   $^{+0.45   }_{-0.37   }$  & 4.40   $^{+1.20   }_{-0.81   }$  & 1.12   $^{+0.33   }_{-0.23   }$  & 0.34$^{+0.21   }_{-0.18   }$  & 0.94 & 1.09 &  76\\
RX J2129.6+0005 &    70 &   690 & 4.30  & 5.91   $^{+0.54   }_{-0.47   }$  & 7.02   $^{+1.30   }_{-0.99   }$  & 1.19   $^{+0.25   }_{-0.19   }$  & 0.45$^{+0.15   }_{-0.15   }$  & 1.21 & 1.07 &  80\\
S0463 $\star$ &    70 &   433 & 1.06  & 3.10   $^{+0.29   }_{-0.25   }$  & 3.10   $^{+0.66   }_{-0.53   }$  & 1.00   $^{+0.23   }_{-0.19   }$  & 0.24$^{+0.14   }_{-0.11   }$  & 1.10 & 1.07 &  47\\
V 1121.0+2327 &    70 &   444 & 1.30  & 3.60   $^{+0.62   }_{-0.46   }$  & 4.08   $^{+1.09   }_{-0.80   }$  & 1.13   $^{+0.36   }_{-0.27   }$  & 0.36$^{+0.29   }_{-0.24   }$  & 1.21 & 1.19 &  66\\
ZWCL 1215 &    70 &   392 & 1.76  & 6.64   $^{+0.40   }_{-0.35   }$  & 8.72   $^{+1.30   }_{-1.07   }$  & 1.31   $^{+0.21   }_{-0.18   }$  & 0.29$^{+0.09   }_{-0.09   }$  & 1.17 & 1.04 &  88\\
ZWCL 1358+6245 &    70 &   553 & 1.94  & 10.66  $^{+1.48   }_{-1.13   }$  & 10.19  $^{+4.83   }_{-2.24   }$  & 0.96   $^{+0.47   }_{-0.23   }$  & 0.47$^{+0.19   }_{-0.19   }$  & 1.08 & 1.04 &  55\\
ZWCL 1953 &    69 &   730 & 3.10  & 7.37   $^{+1.00   }_{-0.78   }$  & 10.44  $^{+3.25   }_{-2.20   }$  & 1.42   $^{+0.48   }_{-0.33   }$  & 0.19$^{+0.13   }_{-0.13   }$  & 0.84 & 0.78 &  74\\
ZWCL 3146 &    70 &   723 & 2.70  & 7.48   $^{+0.32   }_{-0.30   }$  & 8.61   $^{+0.66   }_{-0.58   }$  & 1.15   $^{+0.10   }_{-0.09   }$  & 0.31$^{+0.05   }_{-0.06   }$  & 1.03 & 0.98 &  86\\
ZWCL 5247 &    70 &   635 & 1.70  & 5.06   $^{+0.85   }_{-0.64   }$  & 5.91   $^{+2.09   }_{-1.30   }$  & 1.17   $^{+0.46   }_{-0.30   }$  & 0.22$^{+0.21   }_{-0.19   }$  & 0.83 & 0.72 &  74\\
ZWCL 7160 &    69 &   637 & 3.10  & 4.53   $^{+0.40   }_{-0.35   }$  & 5.16   $^{+1.01   }_{-0.77   }$  & 1.14   $^{+0.24   }_{-0.19   }$  & 0.40$^{+0.15   }_{-0.14   }$  & 0.94 & 0.92 &  80\\
ZWICKY 2701 &    69 &   445 & 0.83  & 5.21   $^{+0.34   }_{-0.30   }$  & 5.68   $^{+0.85   }_{-0.66   }$  & 1.09   $^{+0.18   }_{-0.14   }$  & 0.43$^{+0.13   }_{-0.11   }$  & 0.89 & 0.94 &  57\\
ZwCL 1332.8+5043 &    70 &   642 & 1.10  & 3.62   $^{+3.46   }_{-1.20   }$  & 3.84   $^{+5.93   }_{-1.48   }$  & 1.06   $^{+1.93   }_{-0.54   }$  & 0.76$^{+12.45  }_{-0.76   }$  & 0.24 & 0.29 &  48\\
ZwCl 0848.5+3341 &    71 &   518 & 1.12  & 6.83   $^{+2.18   }_{-1.33   }$  & 7.24   $^{+5.11   }_{-2.26   }$  & 1.06   $^{+0.82   }_{-0.39   }$  & 0.56$^{+0.54   }_{-0.45   }$  & 0.82 & 0.93 &  37\\
\end{rotthesistable}
\doublespacing

\clearpage
\singlespacing
\begin{rotthesistable}{lcccccccccc}
\thesistablehead{Summary of Excised $R_{5000}$ Spectral Fits}{Summary of Excised $R_{5000}$ Spectral Fits}{Cluster & $R_{\mathrm{CORE}}$ & $R_{5000}$  & $N_{H}$ & $T_{77}$ & $T_{27}$ & $T_{HBR}$ & $Z_{77}$ & $\chisq_{red,77}$ & $\chisq_{red,27}$ & \% Source\\  & kpc & kpc & $10^{20}$ cm$^{-2}$ & keV & keV &   & $Z_{\odot}$ &   &   &\\ (1) & (2) & (3) & (4) & (5) & (6) & (7) & (8) & (9) & (10) & (11)}{tab:r5000specfits}
1E0657 56 $\star$ &    69 &   487 & 6.53  & 11.81  $^{+0.29   }_{-0.27   }$  & 14.13  $^{+0.58   }_{-0.53   }$  & 1.20   $^{+0.06   }_{-0.05   }$  & 0.29$^{+0.03   }_{-0.03   }$  & 1.22 & 1.10 &  95\\
1RXS J2129.4-0741 $\star$ &    71 &   373 & 4.36  & 8.47   $^{+1.31   }_{-1.04   }$  & 8.57   $^{+1.73   }_{-1.27   }$  & 1.01   $^{+0.26   }_{-0.19   }$  & 0.51$^{+0.20   }_{-0.19   }$  & 1.16 & 1.27 &  87\\
2PIGG J0011.5-2850 &    69 &   387 & 2.18  & 5.25   $^{+0.29   }_{-0.27   }$  & 6.21   $^{+0.83   }_{-0.68   }$  & 1.18   $^{+0.17   }_{-0.14   }$  & 0.23$^{+0.09   }_{-0.08   }$  & 1.08 & 1.01 &  78\\
2PIGG J0311.8-2655 &    69 &   321 & 1.46  & 3.35   $^{+0.25   }_{-0.22   }$  & 3.67   $^{+0.71   }_{-0.54   }$  & 1.10   $^{+0.23   }_{-0.18   }$  & 0.33$^{+0.13   }_{-0.11   }$  & 1.03 & 1.10 &  51\\
2PIGG J2227.0-3041 &    69 &   267 & 1.11  & 2.81   $^{+0.16   }_{-0.15   }$  & 2.99   $^{+0.36   }_{-0.28   }$  & 1.06   $^{+0.14   }_{-0.11   }$  & 0.35$^{+0.11   }_{-0.08   }$  & 1.14 & 1.10 &  77\\
3C 220.1 &    71 &   322 & 1.91  & 7.81   $^{+7.50   }_{-2.99   }$  & 7.49   $^{+11.53  }_{-3.51   }$  & 0.96   $^{+1.74   }_{-0.58   }$  & 0.00$^{+0.55   }_{-0.00   }$  & 0.60 & 0.78 &  36\\
3C 28.0 &    70 &   297 & 5.71  & 5.18   $^{+0.28   }_{-0.27   }$  & 7.11   $^{+1.15   }_{-0.90   }$  & 1.37   $^{+0.23   }_{-0.19   }$  & 0.30$^{+0.09   }_{-0.07   }$  & 0.96 & 0.77 &  90\\
3C 295 &    69 &   329 & 1.35  & 5.47   $^{+0.49   }_{-0.42   }$  & 6.51   $^{+0.92   }_{-0.78   }$  & 1.19   $^{+0.20   }_{-0.17   }$  & 0.29$^{+0.11   }_{-0.11   }$  & 1.02 & 1.04 &  87\\
3C 388 &    69 &   297 & 6.11  & 3.27   $^{+0.24   }_{-0.21   }$  & 3.44   $^{+0.73   }_{-0.51   }$  & 1.05   $^{+0.24   }_{-0.17   }$  & 0.43$^{+0.16   }_{-0.13   }$  & 1.09 & 1.04 &  76\\
4C 55.16 &    69 &   302 & 4.00  & 4.88   $^{+0.16   }_{-0.16   }$  & 5.11   $^{+0.44   }_{-0.39   }$  & 1.05   $^{+0.10   }_{-0.09   }$  & 0.52$^{+0.07   }_{-0.07   }$  & 0.93 & 0.85 &  71\\
ABELL 0013 &    69 &   404 & 2.03  & 5.39   $^{+0.28   }_{-0.25   }$  & 6.41   $^{+0.84   }_{-0.72   }$  & 1.19   $^{+0.17   }_{-0.14   }$  & 0.37$^{+0.09   }_{-0.09   }$  & 0.96 & 0.95 &  44\\
ABELL 0068 &    70 &   480 & 4.60  & 9.72   $^{+1.82   }_{-1.36   }$  & 10.89  $^{+5.21   }_{-2.85   }$  & 1.12   $^{+0.58   }_{-0.33   }$  & 0.41$^{+0.24   }_{-0.23   }$  & 1.08 & 1.03 &  87\\
ABELL 0119 &    69 &   399 & 3.30  & 5.86   $^{+0.28   }_{-0.27   }$  & 6.20   $^{+0.74   }_{-0.59   }$  & 1.06   $^{+0.14   }_{-0.11   }$  & 0.44$^{+0.10   }_{-0.10   }$  & 0.98 & 0.89 &  75\\
ABELL 0168 $\star$ &    70 &   281 & 3.27  & 2.56   $^{+0.13   }_{-0.10   }$  & 3.37   $^{+0.48   }_{-0.41   }$  & 1.32   $^{+0.20   }_{-0.17   }$  & 0.32$^{+0.07   }_{-0.05   }$  & 1.03 & 0.97 &  44\\
ABELL 0209 $\star$ &    70 &   430 & 1.68  & 7.32   $^{+0.65   }_{-0.56   }$  & 10.05  $^{+2.33   }_{-1.58   }$  & 1.37   $^{+0.34   }_{-0.24   }$  & 0.21$^{+0.11   }_{-0.10   }$  & 1.07 & 1.15 &  88\\
ABELL 0267 $\star$ &    70 &   385 & 2.74  & 6.46   $^{+0.51   }_{-0.45   }$  & 8.46   $^{+0.52   }_{-0.91   }$  & 1.31   $^{+0.13   }_{-0.17   }$  & 0.37$^{+0.12   }_{-0.11   }$  & 1.18 & 1.29 &  88\\
ABELL 0370 &    69 &   365 & 3.37  & 8.74   $^{+0.98   }_{-0.83   }$  & 10.15  $^{+2.17   }_{-1.52   }$  & 1.16   $^{+0.28   }_{-0.21   }$  & 0.37$^{+0.14   }_{-0.13   }$  & 1.05 & 1.02 &  50\\
ABELL 0383 &    69 &   300 & 4.07  & 4.95   $^{+0.30   }_{-0.28   }$  & 5.92   $^{+1.05   }_{-0.85   }$  & 1.20   $^{+0.22   }_{-0.18   }$  & 0.43$^{+0.12   }_{-0.11   }$  & 1.12 & 1.10 &  75\\
ABELL 0399 &    69 &   386 & 8.33$^{+0.82   }_{-0.80   }$  & 7.93   $^{+0.38   }_{-0.35   }$  & 8.86   $^{+0.67   }_{-0.59   }$  & 1.12   $^{+0.10   }_{-0.09   }$  & 0.32$^{+0.06   }_{-0.05   }$  & 1.06 & 0.96 &  87\\
ABELL 0401 &    69 &   454 & 12.48 & 6.54   $^{+0.22   }_{-0.20   }$  & 9.37   $^{+0.91   }_{-0.74   }$  & 1.43   $^{+0.15   }_{-0.12   }$  & 0.29$^{+0.07   }_{-0.06   }$  & 1.53 & 1.10 &  85\\
ABELL 0478 &    69 &   423 & 30.90 & 7.27   $^{+0.26   }_{-0.25   }$  & 8.19   $^{+0.56   }_{-0.50   }$  & 1.13   $^{+0.09   }_{-0.08   }$  & 0.47$^{+0.06   }_{-0.06   }$  & 1.02 & 0.93 &  95\\
ABELL 0514 &    71 &   365 & 3.14  & 3.57   $^{+0.24   }_{-0.23   }$  & 4.30   $^{+0.84   }_{-0.66   }$  & 1.20   $^{+0.25   }_{-0.20   }$  & 0.25$^{+0.11   }_{-0.10   }$  & 0.99 & 1.01 &  55\\
ABELL 0520 &    70 &   407 & 1.14$^{+1.14  }_{-1.16  }$  & 9.15   $^{+0.73   }_{-0.63   }$  & 10.43  $^{+1.41   }_{-1.06   }$  & 1.14   $^{+0.18   }_{-0.14   }$  & 0.36$^{+0.07   }_{-0.07   }$  & 1.12 & 1.01 &  91\\
ABELL 0521 &    70 &   394 & 6.17  & 7.31   $^{+0.79   }_{-0.64   }$  & 9.01   $^{+3.73   }_{-1.87   }$  & 1.23   $^{+0.53   }_{-0.28   }$  & 0.48$^{+0.17   }_{-0.16   }$  & 1.11 & 0.95 &  55\\
ABELL 0586 &    70 &   450 & 4.71  & 6.43   $^{+0.55   }_{-0.49   }$  & 8.06   $^{+1.51   }_{-1.14   }$  & 1.25   $^{+0.26   }_{-0.20   }$  & 0.50$^{+0.15   }_{-0.15   }$  & 0.88 & 0.81 &  87\\
ABELL 0611 &    70 &   370 & 4.99  & 6.79   $^{+0.51   }_{-0.46   }$  & 6.88   $^{+1.23   }_{-0.95   }$  & 1.01   $^{+0.20   }_{-0.16   }$  & 0.32$^{+0.10   }_{-0.10   }$  & 1.04 & 1.07 &  67\\
ABELL 0644 &    70 &   412 & 6.31  & 7.81   $^{+0.20   }_{-0.19   }$  & 8.08   $^{+0.44   }_{-0.39   }$  & 1.03   $^{+0.06   }_{-0.06   }$  & 0.42$^{+0.05   }_{-0.04   }$  & 1.15 & 1.05 &  92\\
ABELL 0665 &    69 &   436 & 4.24  & 7.35   $^{+0.40   }_{-0.37   }$  & 10.43  $^{+1.76   }_{-1.31   }$  & 1.42   $^{+0.25   }_{-0.19   }$  & 0.29$^{+0.07   }_{-0.07   }$  & 1.07 & 0.94 &  91\\
ABELL 0697 &    69 &   432 & 3.34  & 9.80   $^{+0.99   }_{-0.86   }$  & 13.50  $^{+2.90   }_{-2.04   }$  & 1.38   $^{+0.33   }_{-0.24   }$  & 0.48$^{+0.13   }_{-0.13   }$  & 1.06 & 0.96 &  93\\
ABELL 0773 &    69 &   434 & 1.46  & 8.09   $^{+0.75   }_{-0.65   }$  & 10.52  $^{+1.92   }_{-1.53   }$  & 1.30   $^{+0.27   }_{-0.22   }$  & 0.37$^{+0.12   }_{-0.12   }$  & 1.03 & 1.04 &  89\\
ABELL 0907 &    69 &   345 & 5.69  & 5.62   $^{+0.19   }_{-0.18   }$  & 6.82   $^{+0.27   }_{-0.22   }$  & 1.21   $^{+0.06   }_{-0.06   }$  & 0.46$^{+0.06   }_{-0.06   }$  & 1.18 & 1.05 &  92\\
ABELL 0963 &    69 &   384 & 1.39  & 6.97   $^{+0.35   }_{-0.32   }$  & 7.65   $^{+1.00   }_{-0.82   }$  & 1.10   $^{+0.15   }_{-0.13   }$  & 0.29$^{+0.08   }_{-0.07   }$  & 1.13 & 1.12 &  74\\
ABELL 1063S &    69 &   458 & 1.77  & 11.94  $^{+0.91   }_{-0.80   }$  & 14.04  $^{+1.83   }_{-1.47   }$  & 1.18   $^{+0.18   }_{-0.15   }$  & 0.38$^{+0.10   }_{-0.09   }$  & 1.01 & 0.98 &  94\\
ABELL 1068 &    69 &   305 & 0.71  & 4.67   $^{+0.18   }_{-0.18   }$  & 5.49   $^{+0.71   }_{-0.58   }$  & 1.18   $^{+0.16   }_{-0.13   }$  & 0.37$^{+0.06   }_{-0.07   }$  & 0.92 & 0.91 &  77\\
ABELL 1201 &    69 &   401 & 1.85  & 5.74   $^{+0.44   }_{-0.40   }$  & 5.99   $^{+1.39   }_{-0.95   }$  & 1.04   $^{+0.26   }_{-0.18   }$  & 0.35$^{+0.13   }_{-0.11   }$  & 1.06 & 1.10 &  50\\
ABELL 1204 &    70 &   297 & 1.44  & 3.67   $^{+0.18   }_{-0.16   }$  & 4.72   $^{+0.75   }_{-0.57   }$  & 1.29   $^{+0.21   }_{-0.17   }$  & 0.32$^{+0.09   }_{-0.09   }$  & 1.11 & 0.92 &  92\\
ABELL 1361 &    71 &   330 & 2.18  & 5.14   $^{+1.00   }_{-0.74   }$  & 7.24   $^{+8.23   }_{-2.78   }$  & 1.41   $^{+1.62   }_{-0.58   }$  & 0.29$^{+0.31   }_{-0.27   }$  & 1.10 & 0.82 &  61\\
ABELL 1423 &    70 &   435 & 1.60  & 6.04   $^{+0.82   }_{-0.68   }$  & 7.93   $^{+4.09   }_{-2.20   }$  & 1.31   $^{+0.70   }_{-0.39   }$  & 0.33$^{+0.20   }_{-0.17   }$  & 0.95 & 0.91 &  84\\
ABELL 1651 &    70 &   421 & 2.02  & 6.30   $^{+0.32   }_{-0.28   }$  & 7.72   $^{+0.71   }_{-0.65   }$  & 1.23   $^{+0.13   }_{-0.12   }$  & 0.44$^{+0.09   }_{-0.09   }$  & 1.13 & 1.19 &  91\\
ABELL 1664 &    69 &   291 & 8.47  & 4.26   $^{+0.30   }_{-0.26   }$  & 4.91   $^{+1.05   }_{-0.80   }$  & 1.15   $^{+0.26   }_{-0.20   }$  & 0.31$^{+0.12   }_{-0.11   }$  & 1.07 & 1.08 &  70\\
ABELL 1689 $\star$ &    70 &   481 & 1.87  & 9.76   $^{+0.40   }_{-0.38   }$  & 12.97  $^{+1.25   }_{-1.05   }$  & 1.33   $^{+0.14   }_{-0.12   }$  & 0.35$^{+0.06   }_{-0.05   }$  & 1.14 & 1.04 &  94\\
ABELL 1758 &    69 &   404 & 1.09  & 9.66   $^{+0.75   }_{-0.64   }$  & 9.90   $^{+1.22   }_{-1.89   }$  & 1.02   $^{+0.15   }_{-0.21   }$  & 0.48$^{+0.11   }_{-0.11   }$  & 1.03 & 0.96 &  68\\
ABELL 1763 &    69 &   396 & 0.82  & 7.74   $^{+0.73   }_{-0.64   }$  & 12.56  $^{+6.70   }_{-3.12   }$  & 1.62   $^{+0.88   }_{-0.42   }$  & 0.22$^{+0.11   }_{-0.12   }$  & 1.16 & 1.02 &  89\\
ABELL 1795 &    69 &   449 & 1.22  & 6.05   $^{+0.15   }_{-0.15   }$  & 6.85   $^{+0.42   }_{-0.38   }$  & 1.13   $^{+0.07   }_{-0.07   }$  & 0.33$^{+0.04   }_{-0.05   }$  & 1.19 & 1.03 &  93\\
ABELL 1835 &    70 &   404 & 2.36  & 9.55   $^{+0.55   }_{-0.51   }$  & 11.99  $^{+1.96   }_{-1.44   }$  & 1.26   $^{+0.22   }_{-0.17   }$  & 0.35$^{+0.07   }_{-0.08   }$  & 0.91 & 0.88 &  91\\
ABELL 1914 &    70 &   493 & 0.97  & 9.73   $^{+0.58   }_{-0.51   }$  & 11.97  $^{+1.90   }_{-1.40   }$  & 1.23   $^{+0.21   }_{-0.16   }$  & 0.32$^{+0.08   }_{-0.07   }$  & 1.11 & 1.03 &  95\\
ABELL 1942 &    69 &   334 & 2.75  & 4.96   $^{+0.45   }_{-0.39   }$  & 5.94   $^{+2.24   }_{-0.99   }$  & 1.20   $^{+0.46   }_{-0.22   }$  & 0.37$^{+0.15   }_{-0.14   }$  & 1.04 & 0.87 &  77\\
ABELL 1995 &    71 &   271 & 1.44  & 8.50   $^{+0.83   }_{-0.71   }$  & 9.41   $^{+1.87   }_{-1.32   }$  & 1.11   $^{+0.25   }_{-0.18   }$  & 0.33$^{+0.12   }_{-0.12   }$  & 1.05 & 1.02 &  81\\
ABELL 2029 &    70 &   434 & 3.26  & 8.22   $^{+0.31   }_{-0.30   }$  & 9.92   $^{+0.91   }_{-0.73   }$  & 1.21   $^{+0.12   }_{-0.10   }$  & 0.40$^{+0.06   }_{-0.06   }$  & 1.08 & 1.04 &  94\\
ABELL 2034 &    69 &   420 & 1.58  & 7.35   $^{+0.26   }_{-0.24   }$  & 9.96   $^{+1.09   }_{-0.84   }$  & 1.36   $^{+0.16   }_{-0.12   }$  & 0.34$^{+0.05   }_{-0.05   }$  & 1.17 & 1.02 &  90\\
ABELL 2065 &    69 &   370 & 2.96  & 5.75   $^{+0.19   }_{-0.17   }$  & 6.39   $^{+0.46   }_{-0.41   }$  & 1.11   $^{+0.09   }_{-0.08   }$  & 0.28$^{+0.05   }_{-0.05   }$  & 1.11 & 1.01 &  89\\
ABELL 2069 &    70 &   440 & 1.97  & 6.33   $^{+0.36   }_{-0.32   }$  & 8.29   $^{+1.36   }_{-1.02   }$  & 1.31   $^{+0.23   }_{-0.17   }$  & 0.24$^{+0.08   }_{-0.08   }$  & 1.14 & 1.15 &  78\\
ABELL 2111 &    70 &   417 & 2.20  & 5.74   $^{+1.43   }_{-0.97   }$  & 7.18   $^{+6.73   }_{-2.52   }$  & 1.25   $^{+1.21   }_{-0.49   }$  & 0.16$^{+0.30   }_{-0.16   }$  & 1.06 & 0.97 &  74\\
ABELL 2125 &    70 &   262 & 2.75  & 3.09   $^{+0.37   }_{-0.31   }$  & 3.69   $^{+1.99   }_{-0.81   }$  & 1.19   $^{+0.66   }_{-0.29   }$  & 0.36$^{+0.25   }_{-0.20   }$  & 1.25 & 1.22 &  68\\
ABELL 2163 &    69 &   531 & 12.04 & 18.78  $^{+0.89   }_{-0.83   }$  & 19.49  $^{+2.03   }_{-1.86   }$  & 1.04   $^{+0.12   }_{-0.11   }$  & 0.09$^{+0.06   }_{-0.05   }$  & 1.33 & 1.25 &  93\\
ABELL 2204 $\star$ &    70 &   406 & 5.84  & 9.35   $^{+0.43   }_{-0.41   }$  & 10.18  $^{+0.95   }_{-0.77   }$  & 1.09   $^{+0.11   }_{-0.10   }$  & 0.37$^{+0.07   }_{-0.07   }$  & 0.95 & 0.97 &  86\\
ABELL 2218 &    70 &   394 & 3.12  & 7.37   $^{+0.40   }_{-0.37   }$  & 9.36   $^{+1.42   }_{-1.07   }$  & 1.27   $^{+0.20   }_{-0.16   }$  & 0.22$^{+0.07   }_{-0.06   }$  & 1.00 & 0.91 &  91\\
ABELL 2219 &    69 &   463 & 1.76  & 12.60  $^{+0.65   }_{-0.61   }$  & 12.54  $^{+1.52   }_{-1.21   }$  & 1.00   $^{+0.13   }_{-0.11   }$  & 0.31$^{+0.07   }_{-0.07   }$  & 1.02 & 0.98 &  81\\
ABELL 2255 &    71 &   422 & 2.53  & 6.37   $^{+0.24   }_{-0.23   }$  & 7.70   $^{+0.79   }_{-0.49   }$  & 1.21   $^{+0.13   }_{-0.09   }$  & 0.34$^{+0.06   }_{-0.07   }$  & 0.93 & 0.84 &  81\\
ABELL 2256 &    70 &   441 & 4.05  & 5.66   $^{+0.19   }_{-0.17   }$  & 7.30   $^{+0.69   }_{-0.63   }$  & 1.29   $^{+0.13   }_{-0.12   }$  & 0.31$^{+0.07   }_{-0.07   }$  & 1.61 & 1.44 &  79\\
ABELL 2259 &    69 &   340 & 3.70  & 5.07   $^{+0.46   }_{-0.40   }$  & 5.49   $^{+1.29   }_{-0.91   }$  & 1.08   $^{+0.27   }_{-0.20   }$  & 0.40$^{+0.16   }_{-0.14   }$  & 0.92 & 0.92 &  90\\
ABELL 2261 &    69 &   407 & 3.31  & 7.86   $^{+0.51   }_{-0.47   }$  & 9.84   $^{+1.94   }_{-1.30   }$  & 1.25   $^{+0.26   }_{-0.18   }$  & 0.40$^{+0.09   }_{-0.09   }$  & 0.98 & 0.95 &  94\\
ABELL 2294 &    69 &   405 & 6.10  & 10.49  $^{+1.75   }_{-1.30   }$  & 12.33  $^{+5.72   }_{-3.05   }$  & 1.18   $^{+0.58   }_{-0.33   }$  & 0.57$^{+0.25   }_{-0.24   }$  & 1.16 & 1.08 &  88\\
ABELL 2384 &    70 &   308 & 2.99  & 4.53   $^{+0.22   }_{-0.21   }$  & 6.78   $^{+1.13   }_{-0.89   }$  & 1.50   $^{+0.26   }_{-0.21   }$  & 0.15$^{+0.07   }_{-0.06   }$  & 0.99 & 0.88 &  86\\
ABELL 2390 &    70 &   447 & 6.71  & 10.85  $^{+0.34   }_{-0.31   }$  & 10.53  $^{+0.62   }_{-0.53   }$  & 0.97   $^{+0.06   }_{-0.06   }$  & 0.35$^{+0.05   }_{-0.04   }$  & 1.15 & 1.03 &  81\\
ABELL 2409 &    70 &   362 & 6.72  & 5.93   $^{+0.45   }_{-0.39   }$  & 5.87   $^{+0.95   }_{-0.76   }$  & 0.99   $^{+0.18   }_{-0.14   }$  & 0.35$^{+0.13   }_{-0.11   }$  & 1.05 & 0.76 &  92\\
ABELL 2537 &    69 &   351 & 4.26  & 8.83   $^{+0.87   }_{-0.74   }$  & 7.83   $^{+1.54   }_{-1.16   }$  & 0.89   $^{+0.20   }_{-0.15   }$  & 0.39$^{+0.14   }_{-0.14   }$  & 0.93 & 0.83 &  59\\
ABELL 2554 &    71 &   415 & 2.04  & 5.35   $^{+0.45   }_{-0.40   }$  & 6.46   $^{+1.93   }_{-1.24   }$  & 1.21   $^{+0.37   }_{-0.25   }$  & 0.35$^{+0.15   }_{-0.13   }$  & 0.93 & 0.79 &  40\\
ABELL 2556 &    70 &   323 & 2.02  & 3.57   $^{+0.16   }_{-0.15   }$  & 4.07   $^{+0.56   }_{-0.46   }$  & 1.14   $^{+0.16   }_{-0.14   }$  & 0.36$^{+0.07   }_{-0.07   }$  & 0.99 & 0.95 &  58\\
ABELL 2631 &    70 &   445 & 3.74  & 7.18   $^{+1.18   }_{-0.94   }$  & 9.18   $^{+3.17   }_{-1.96   }$  & 1.28   $^{+0.49   }_{-0.32   }$  & 0.34$^{+0.20   }_{-0.19   }$  & 1.03 & 0.99 &  89\\
ABELL 2667 &    70 &   370 & 1.64  & 6.68   $^{+0.48   }_{-0.43   }$  & 7.35   $^{+1.27   }_{-1.05   }$  & 1.10   $^{+0.21   }_{-0.17   }$  & 0.41$^{+0.12   }_{-0.12   }$  & 1.05 & 0.95 &  84\\
ABELL 2670 &    69 &   319 & 2.88  & 3.96   $^{+0.13   }_{-0.13   }$  & 4.75   $^{+0.50   }_{-0.41   }$  & 1.20   $^{+0.13   }_{-0.11   }$  & 0.45$^{+0.08   }_{-0.07   }$  & 1.16 & 1.09 &  80\\
ABELL 2717 &    70 &   211 & 1.12  & 2.59   $^{+0.17   }_{-0.16   }$  & 3.18   $^{+0.59   }_{-0.44   }$  & 1.23   $^{+0.24   }_{-0.19   }$  & 0.53$^{+0.14   }_{-0.12   }$  & 0.90 & 0.95 &  67\\
ABELL 2744 &    71 &   458 & 1.82  & 9.82   $^{+0.89   }_{-0.77   }$  & 11.21  $^{+2.76   }_{-1.81   }$  & 1.14   $^{+0.30   }_{-0.20   }$  & 0.30$^{+0.12   }_{-0.12   }$  & 0.88 & 0.73 &  74\\
ABELL 3128 &    70 &   318 & 1.59  & 3.04   $^{+0.23   }_{-0.21   }$  & 3.48   $^{+0.73   }_{-0.54   }$  & 1.14   $^{+0.26   }_{-0.19   }$  & 0.33$^{+0.13   }_{-0.10   }$  & 1.05 & 1.13 &  64\\
ABELL 3158 $\star$ &    70 &   382 & 1.60  & 5.08   $^{+0.08   }_{-0.08   }$  & 6.26   $^{+0.26   }_{-0.24   }$  & 1.23   $^{+0.05   }_{-0.05   }$  & 0.40$^{+0.03   }_{-0.03   }$  & 1.15 & 0.97 &  89\\
ABELL 3164 &    70 &   319 & 2.55  & 2.40   $^{+0.65   }_{-0.48   }$  & 3.19   $^{+5.68   }_{-1.41   }$  & 1.33   $^{+2.39   }_{-0.64   }$  & 0.23$^{+0.32   }_{-0.19   }$  & 1.29 & 1.59 &  30\\
ABELL 3376 $\star$ &    70 &   327 & 5.21  & 4.44   $^{+0.14   }_{-0.13   }$  & 5.94   $^{+0.55   }_{-0.47   }$  & 1.34   $^{+0.13   }_{-0.11   }$  & 0.36$^{+0.06   }_{-0.06   }$  & 1.18 & 1.13 &  65\\
ABELL 3391 &    70 &   397 & 5.46  & 5.72   $^{+0.31   }_{-0.28   }$  & 6.44   $^{+0.80   }_{-0.66   }$  & 1.13   $^{+0.15   }_{-0.13   }$  & 0.11$^{+0.08   }_{-0.07   }$  & 1.00 & 0.97 &  67\\
ABELL 3921 &    69 &   378 & 3.07  & 5.69   $^{+0.25   }_{-0.24   }$  & 6.74   $^{+0.71   }_{-0.58   }$  & 1.18   $^{+0.14   }_{-0.11   }$  & 0.34$^{+0.08   }_{-0.07   }$  & 0.93 & 0.85 &  84\\
AC 114 &    70 &   389 & 1.44  & 7.75   $^{+0.56   }_{-0.50   }$  & 9.76   $^{+2.28   }_{-1.55   }$  & 1.26   $^{+0.31   }_{-0.22   }$  & 0.36$^{+0.11   }_{-0.10   }$  & 1.01 & 0.95 &  63\\
CL 0024+17 &    71 &   309 & 4.36  & 4.75   $^{+1.07   }_{-0.76   }$  & 7.14   $^{+5.42   }_{-2.83   }$  & 1.50   $^{+1.19   }_{-0.64   }$  & 0.58$^{+0.35   }_{-0.30   }$  & 1.07 & 0.97 &  44\\
CL 1221+4918 &    71 &   313 & 1.44  & 6.73   $^{+1.29   }_{-1.02   }$  & 7.60   $^{+4.33   }_{-2.01   }$  & 1.13   $^{+0.68   }_{-0.34   }$  & 0.32$^{+0.20   }_{-0.19   }$  & 0.92 & 0.69 &  73\\
CL J0030+2618 &    70 &   555 & 4.10  & 4.48   $^{+2.43   }_{-1.40   }$  & 3.77   $^{+9.73   }_{-1.96   }$  & 0.84   $^{+2.22   }_{-0.51   }$  & 0.00$^{+0.37   }_{-0.00   }$  & 1.01 & 0.85 &  51\\
CL J0152-1357 &    70 &   277 & 1.45  & 7.20   $^{+7.14   }_{-2.48   }$  & 6.07   $^{+6.16   }_{-2.51   }$  & 0.84   $^{+1.20   }_{-0.45   }$  & 0.00$^{+0.63   }_{-0.00   }$  & 2.97 & 3.26 &  49\\
CL J0542.8-4100 &    71 &   313 & 3.59  & 5.65   $^{+1.21   }_{-0.90   }$  & 5.93   $^{+3.52   }_{-1.76   }$  & 1.05   $^{+0.66   }_{-0.35   }$  & 0.25$^{+0.24   }_{-0.22   }$  & 0.67 & 0.58 &  72\\
CL J0848+4456 $\star$ &    71 &   224 & 2.53  & 3.73   $^{+1.47   }_{-0.85   }$  & 4.96   $^{+2.82   }_{-1.81   }$  & 1.33   $^{+0.92   }_{-0.57   }$  & 0.17$^{+0.98   }_{-0.17   }$  & 0.87 & 0.82 &  64\\
CL J1113.1-2615 &    70 &   308 & 5.51  & 4.74   $^{+1.52   }_{-0.98   }$  & 4.79   $^{+1.15   }_{-1.26   }$  & 1.01   $^{+0.40   }_{-0.34   }$  & 0.53$^{+0.52   }_{-0.37   }$  & 1.02 & 1.01 &  32\\
CL J1226.9+3332 $\star$ &    69 &   318 & 1.37  & 13.02  $^{+2.69   }_{-2.00   }$  & 12.33  $^{+2.78   }_{-2.13   }$  & 0.95   $^{+0.29   }_{-0.22   }$  & 0.18$^{+0.23   }_{-0.18   }$  & 0.75 & 0.80 &  91\\
CL J2302.8+0844 &    70 &   362 & 5.05  & 5.94   $^{+1.73   }_{-1.86   }$  & 6.58   $^{+8.08   }_{-2.67   }$  & 1.11   $^{+1.40   }_{-0.57   }$  & 0.10$^{+0.29   }_{-0.10   }$  & 0.94 & 1.01 &  56\\
DLS J0514-4904 &    70 &   359 & 2.52  & 4.94   $^{+0.61   }_{-0.55   }$  & 6.26   $^{+2.33   }_{-1.30   }$  & 1.27   $^{+0.50   }_{-0.30   }$  & 0.35$^{+0.27   }_{-0.23   }$  & 0.86 & 1.03 &  63\\
EXO 0422-086 &    70 &   294 & 6.22  & 3.41   $^{+0.14   }_{-0.13   }$  & 3.44   $^{+0.37   }_{-0.31   }$  & 1.01   $^{+0.12   }_{-0.10   }$  & 0.37$^{+0.08   }_{-0.08   }$  & 0.96 & 0.93 &  80\\
HERCULES A &    69 &   312 & 1.49$^{+2.01   }_{-1.49   }$  & 5.28   $^{+0.60   }_{-0.50   }$  & 4.50   $^{+0.88   }_{-0.65   }$  & 0.85   $^{+0.19   }_{-0.15   }$  & 0.42$^{+0.15   }_{-0.14   }$  & 0.98 & 0.98 &  70\\
MACS J0011.7-1523 $\star$ &    69 &   319 & 2.08  & 6.73   $^{+0.55   }_{-0.47   }$  & 7.27   $^{+0.99   }_{-0.74   }$  & 1.08   $^{+0.17   }_{-0.13   }$  & 0.27$^{+0.10   }_{-0.09   }$  & 0.90 & 0.95 &  92\\
MACS J0025.4-1222 $\star$ &    70 &   335 & 2.72  & 6.65   $^{+1.07   }_{-0.85   }$  & 6.31   $^{+1.38   }_{-1.02   }$  & 0.95   $^{+0.26   }_{-0.20   }$  & 0.39$^{+0.22   }_{-0.19   }$  & 0.66 & 0.75 &  86\\
MACS J0035.4-2015 &    70 &   372 & 1.55  & 7.72   $^{+0.88   }_{-0.74   }$  & 9.39   $^{+1.91   }_{-1.35   }$  & 1.22   $^{+0.28   }_{-0.21   }$  & 0.39$^{+0.14   }_{-0.13   }$  & 1.02 & 1.05 &  94\\
MACS J0111.5+0855 &    70 &   306 & 4.18  & 4.12   $^{+1.60   }_{-1.04   }$  & 4.16   $^{+2.96   }_{-1.44   }$  & 1.01   $^{+0.82   }_{-0.43   }$  & 0.00$^{+0.43   }_{-0.00   }$  & 0.79 & 1.23 &  62\\
MACS J0152.5-2852 &    70 &   324 & 1.46  & 5.75   $^{+1.05   }_{-0.78   }$  & 7.70   $^{+3.21   }_{-1.89   }$  & 1.34   $^{+0.61   }_{-0.38   }$  & 0.28$^{+0.22   }_{-0.21   }$  & 0.84 & 0.58 &  90\\
MACS J0159.0-3412 &    70 &   404 & 1.54  & 10.99  $^{+5.87   }_{-2.95   }$  & 12.74  $^{+12.45  }_{-4.72   }$  & 1.16   $^{+1.29   }_{-0.53   }$  & 0.50$^{+0.52   }_{-0.50   }$  & 1.35 & 1.34 &  85\\
MACS J0159.8-0849 $\star$ &    69 &   413 & 2.01  & 9.36   $^{+0.77   }_{-0.67   }$  & 10.37  $^{+1.29   }_{-1.04   }$  & 1.11   $^{+0.17   }_{-0.14   }$  & 0.29$^{+0.09   }_{-0.09   }$  & 1.05 & 1.01 &  94\\
MACS J0242.5-2132 &    70 &   352 & 2.71  & 5.48   $^{+0.62   }_{-0.51   }$  & 5.99   $^{+2.04   }_{-1.19   }$  & 1.09   $^{+0.39   }_{-0.24   }$  & 0.32$^{+0.16   }_{-0.15   }$  & 1.08 & 1.06 &  92\\
MACS J0257.1-2325 $\star$ &    70 &   409 & 2.09  & 9.42   $^{+1.37   }_{-1.05   }$  & 10.76  $^{+2.05   }_{-1.69   }$  & 1.14   $^{+0.27   }_{-0.22   }$  & 0.14$^{+0.13   }_{-0.13   }$  & 1.03 & 1.13 &  90\\
MACS J0257.6-2209 &    69 &   382 & 2.02  & 8.09   $^{+1.10   }_{-0.88   }$  & 7.90   $^{+1.64   }_{-1.20   }$  & 0.98   $^{+0.24   }_{-0.18   }$  & 0.41$^{+0.19   }_{-0.18   }$  & 1.13 & 1.24 &  90\\
MACS J0308.9+2645 &    69 &   381 & 11.88 & 10.64  $^{+1.38   }_{-1.14   }$  & 11.12  $^{+2.23   }_{-1.68   }$  & 1.05   $^{+0.25   }_{-0.19   }$  & 0.37$^{+0.15   }_{-0.15   }$  & 0.96 & 0.97 &  92\\
MACS J0329.6-0211 $\star$ &    70 &   297 & 6.21  & 6.44   $^{+0.50   }_{-0.45   }$  & 7.55   $^{+0.88   }_{-0.73   }$  & 1.17   $^{+0.16   }_{-0.14   }$  & 0.40$^{+0.10   }_{-0.09   }$  & 1.12 & 1.16 &  91\\
MACS J0404.6+1109 &    70 &   348 & 14.96 & 6.90   $^{+2.01   }_{-1.29   }$  & 7.40   $^{+3.63   }_{-1.93   }$  & 1.07   $^{+0.61   }_{-0.34   }$  & 0.22$^{+0.27   }_{-0.22   }$  & 0.96 & 0.92 &  80\\
MACS J0417.5-1154 &    70 &   304 & 4.00  & 10.44  $^{+2.08   }_{-1.56   }$  & 14.46  $^{+5.92   }_{-3.41   }$  & 1.39   $^{+0.63   }_{-0.39   }$  & 0.41$^{+0.23   }_{-0.21   }$  & 1.10 & 1.17 &  96\\
MACS J0429.6-0253 &    69 &   348 & 5.70  & 5.96   $^{+0.72   }_{-0.60   }$  & 7.48   $^{+2.65   }_{-1.64   }$  & 1.26   $^{+0.47   }_{-0.30   }$  & 0.34$^{+0.15   }_{-0.14   }$  & 1.02 & 0.78 &  89\\
MACS J0451.9+0006 &    70 &   325 & 7.65  & 5.76   $^{+1.77   }_{-1.11   }$  & 6.68   $^{+4.50   }_{-1.94   }$  & 1.16   $^{+0.86   }_{-0.40   }$  & 0.47$^{+0.46   }_{-0.38   }$  & 1.03 & 1.33 &  89\\
MACS J0455.2+0657 &    71 &   340 & 10.45 & 6.99   $^{+2.27   }_{-1.44   }$  & 8.35   $^{+5.66   }_{-2.49   }$  & 1.19   $^{+0.90   }_{-0.43   }$  & 0.48$^{+0.35   }_{-0.31   }$  & 1.04 & 1.24 &  88\\
MACS J0520.7-1328 &    69 &   348 & 8.88  & 6.77   $^{+1.01   }_{-0.79   }$  & 9.41   $^{+3.38   }_{-1.91   }$  & 1.39   $^{+0.54   }_{-0.33   }$  & 0.33$^{+0.16   }_{-0.16   }$  & 1.22 & 1.33 &  91\\
MACS J0547.0-3904 &    69 &   257 & 4.08  & 3.70   $^{+0.44   }_{-0.37   }$  & 5.82   $^{+2.97   }_{-1.36   }$  & 1.57   $^{+0.82   }_{-0.40   }$  & 0.24$^{+0.21   }_{-0.17   }$  & 1.14 & 1.21 &  83\\
MACS J0553.4-3342 &    70 &   490 & 2.88  & 13.90  $^{+5.89   }_{-3.28   }$  & 14.59  $^{+11.16  }_{-4.72   }$  & 1.05   $^{+0.92   }_{-0.42   }$  & 0.38$^{+0.39   }_{-0.38   }$  & 1.22 & 1.10 &  91\\
MACS J0717.5+3745 $\star$ &    70 &   398 & 6.75  & 13.30  $^{+1.44   }_{-1.21   }$  & 12.82  $^{+1.70   }_{-1.39   }$  & 0.96   $^{+0.17   }_{-0.14   }$  & 0.32$^{+0.12   }_{-0.13   }$  & 0.91 & 0.87 &  91\\
MACS J0744.8+3927 $\star$ &    70 &   381 & 4.66  & 8.58   $^{+0.85   }_{-0.73   }$  & 9.32   $^{+1.20   }_{-0.96   }$  & 1.09   $^{+0.18   }_{-0.15   }$  & 0.30$^{+0.11   }_{-0.11   }$  & 1.14 & 1.19 &  89\\
MACS J0911.2+1746 $\star$ &    70 &   382 & 3.55  & 7.71   $^{+1.55   }_{-1.16   }$  & 7.88   $^{+2.11   }_{-1.44   }$  & 1.02   $^{+0.34   }_{-0.24   }$  & 0.22$^{+0.20   }_{-0.20   }$  & 0.77 & 0.77 &  85\\
MACS J0949+1708 &    70 &   411 & 3.17  & 8.94   $^{+1.57   }_{-1.20   }$  & 10.29  $^{+5.60   }_{-2.41   }$  & 1.15   $^{+0.66   }_{-0.31   }$  & 0.48$^{+0.23   }_{-0.22   }$  & 0.74 & 0.58 &  93\\
MACS J1006.9+3200 &    70 &   363 & 1.83  & 7.03   $^{+2.66   }_{-1.64   }$  & 6.53   $^{+4.61   }_{-2.11   }$  & 0.93   $^{+0.74   }_{-0.37   }$  & 0.18$^{+0.45   }_{-0.18   }$  & 1.64 & 1.53 &  81\\
MACS J1105.7-1014 &    71 &   356 & 4.58  & 7.73   $^{+2.85   }_{-1.73   }$  & 6.61   $^{+3.02   }_{-1.79   }$  & 0.86   $^{+0.50   }_{-0.30   }$  & 0.20$^{+0.32   }_{-0.20   }$  & 1.27 & 1.08 &  87\\
MACS J1108.8+0906 $\star$ &    70 &   345 & 2.52  & 6.80   $^{+1.21   }_{-0.93   }$  & 7.52   $^{+2.39   }_{-1.53   }$  & 1.11   $^{+0.40   }_{-0.27   }$  & 0.24$^{+0.20   }_{-0.19   }$  & 1.08 & 1.01 &  86\\
MACS J1115.2+5320 $\star$ &    70 &   372 & 0.98  & 9.58   $^{+1.85   }_{-1.37   }$  & 9.80   $^{+2.74   }_{-1.81   }$  & 1.02   $^{+0.35   }_{-0.24   }$  & 0.37$^{+0.22   }_{-0.21   }$  & 0.94 & 0.91 &  82\\
MACS J1115.8+0129 &    70 &   316 & 4.36  & 6.82   $^{+1.15   }_{-0.88   }$  & 9.39   $^{+4.77   }_{-2.84   }$  & 1.38   $^{+0.74   }_{-0.45   }$  & 0.07$^{+0.19   }_{-0.07   }$  & 0.94 & 0.85 &  77\\
MACS J1131.8-1955 &    69 &   407 & 4.49  & 8.64   $^{+1.32   }_{-1.03   }$  & 9.45   $^{+2.52   }_{-1.68   }$  & 1.09   $^{+0.34   }_{-0.23   }$  & 0.49$^{+0.19   }_{-0.19   }$  & 1.07 & 1.02 &  91\\
MACS J1149.5+2223 $\star$ &    69 &   358 & 2.32  & 7.72   $^{+0.94   }_{-0.79   }$  & 8.36   $^{+1.51   }_{-1.14   }$  & 1.08   $^{+0.24   }_{-0.18   }$  & 0.25$^{+0.12   }_{-0.13   }$  & 0.87 & 0.94 &  75\\
MACS J1206.2-0847 &    70 &   367 & 4.15  & 9.98   $^{+1.27   }_{-1.01   }$  & 11.93  $^{+2.56   }_{-1.88   }$  & 1.20   $^{+0.30   }_{-0.22   }$  & 0.32$^{+0.13   }_{-0.14   }$  & 1.02 & 1.15 &  95\\
MACS J1226.8+2153 &    71 &   347 & 1.82  & 4.86   $^{+1.58   }_{-1.08   }$  & 5.84   $^{+3.45   }_{-2.14   }$  & 1.20   $^{+0.81   }_{-0.51   }$  & 0.00$^{+0.28   }_{-0.00   }$  & 1.32 & 1.36 &  78\\
MACS J1311.0-0310 $\star$ &    69 &   301 & 2.18  & 5.73   $^{+0.46   }_{-0.40   }$  & 5.92   $^{+0.70   }_{-0.60   }$  & 1.03   $^{+0.15   }_{-0.13   }$  & 0.44$^{+0.12   }_{-0.12   }$  & 0.93 & 1.00 &  83\\
MACS J1319+7003 &    70 &   351 & 1.53  & 8.08   $^{+2.14   }_{-1.56   }$  & 10.12  $^{+5.50   }_{-2.78   }$  & 1.25   $^{+0.76   }_{-0.42   }$  & 0.10$^{+0.25   }_{-0.10   }$  & 1.00 & 1.07 &  82\\
MACS J1427.2+4407 &    71 &   346 & 1.41  & 8.61   $^{+4.04   }_{-2.23   }$  & 8.83   $^{+5.55   }_{-2.81   }$  & 1.03   $^{+0.80   }_{-0.42   }$  & 0.14$^{+0.36   }_{-0.14   }$  & 0.68 & 0.58 &  90\\
MACS J1427.6-2521 &    71 &   302 & 6.11  & 4.44   $^{+0.86   }_{-0.64   }$  & 6.17   $^{+3.18   }_{-1.71   }$  & 1.39   $^{+0.77   }_{-0.43   }$  & 0.21$^{+0.26   }_{-0.21   }$  & 1.07 & 1.39 &  79\\
MACS J1621.3+3810 $\star$ &    69 &   358 & 1.07  & 7.49   $^{+0.73   }_{-0.63   }$  & 7.75   $^{+1.12   }_{-0.89   }$  & 1.03   $^{+0.18   }_{-0.15   }$  & 0.35$^{+0.13   }_{-0.12   }$  & 0.98 & 0.92 &  82\\
MACS J1731.6+2252 &    71 &   368 & 6.48  & 8.19   $^{+1.88   }_{-1.31   }$  & 10.50  $^{+4.76   }_{-2.46   }$  & 1.28   $^{+0.65   }_{-0.36   }$  & 0.49$^{+0.27   }_{-0.25   }$  & 1.16 & 0.98 &  87\\
MACS J1931.8-2634 &    70 &   378 & 9.13  & 6.85   $^{+0.73   }_{-0.61   }$  & 6.86   $^{+1.58   }_{-1.15   }$  & 1.00   $^{+0.25   }_{-0.19   }$  & 0.23$^{+0.12   }_{-0.11   }$  & 1.02 & 1.07 &  94\\
MACS J2046.0-3430 &    71 &   274 & 4.98  & 5.02   $^{+1.95   }_{-1.04   }$  & 6.23   $^{+2.57   }_{-2.30   }$  & 1.24   $^{+0.70   }_{-0.53   }$  & 0.23$^{+0.55   }_{-0.23   }$  & 1.10 & 1.14 &  89\\
MACS J2049.9-3217 &    69 &   370 & 5.99  & 7.88   $^{+1.22   }_{-0.98   }$  & 11.48  $^{+4.02   }_{-2.42   }$  & 1.46   $^{+0.56   }_{-0.36   }$  & 0.37$^{+0.18   }_{-0.16   }$  & 0.94 & 0.90 &  89\\
MACS J2211.7-0349 &    69 &   468 & 5.86  & 11.13  $^{+1.45   }_{-1.15   }$  & 13.77  $^{+3.49   }_{-2.40   }$  & 1.24   $^{+0.35   }_{-0.25   }$  & 0.18$^{+0.14   }_{-0.14   }$  & 1.33 & 1.34 &  93\\
MACS J2214.9-1359 $\star$ &    70 &   374 & 3.32  & 9.87   $^{+1.54   }_{-1.17   }$  & 9.97   $^{+2.17   }_{-1.50   }$  & 1.01   $^{+0.27   }_{-0.19   }$  & 0.31$^{+0.17   }_{-0.17   }$  & 1.03 & 1.01 &  92\\
MACS J2228+2036 &    70 &   385 & 4.52  & 7.79   $^{+1.14   }_{-0.90   }$  & 10.04  $^{+3.96   }_{-2.25   }$  & 1.29   $^{+0.54   }_{-0.32   }$  & 0.41$^{+0.18   }_{-0.17   }$  & 0.84 & 0.96 &  92\\
MACS J2229.7-2755 &    69 &   327 & 1.34  & 5.25   $^{+0.54   }_{-0.46   }$  & 6.07   $^{+1.76   }_{-1.18   }$  & 1.16   $^{+0.36   }_{-0.25   }$  & 0.59$^{+0.20   }_{-0.19   }$  & 0.98 & 1.02 &  91\\
MACS J2243.3-0935 &    71 &   406 & 4.31  & 5.15   $^{+0.65   }_{-0.54   }$  & 8.81   $^{+4.31   }_{-2.67   }$  & 1.71   $^{+0.86   }_{-0.55   }$  & 0.05$^{+0.17   }_{-0.05   }$  & 1.38 & 1.27 &  66\\
MACS J2245.0+2637 &    69 &   320 & 5.50  & 6.05   $^{+0.66   }_{-0.56   }$  & 7.05   $^{+1.31   }_{-1.08   }$  & 1.17   $^{+0.25   }_{-0.21   }$  & 0.64$^{+0.21   }_{-0.20   }$  & 0.78 & 0.95 &  92\\
MACS J2311+0338 &    70 &   257 & 5.23  & 7.66   $^{+1.63   }_{-1.20   }$  & 12.19  $^{+6.04   }_{-3.14   }$  & 1.59   $^{+0.86   }_{-0.48   }$  & 0.44$^{+0.24   }_{-0.23   }$  & 1.22 & 1.10 &  92\\
MKW3S &    70 &   239 & 3.05  & 3.93   $^{+0.06   }_{-0.06   }$  & 4.58   $^{+0.19   }_{-0.17   }$  & 1.17   $^{+0.05   }_{-0.05   }$  & 0.35$^{+0.02   }_{-0.03   }$  & 1.28 & 0.93 &  88\\
MS 0016.9+1609 &    69 &   389 & 4.06  & 9.11   $^{+0.79   }_{-0.68   }$  & 11.73  $^{+2.98   }_{-1.84   }$  & 1.29   $^{+0.35   }_{-0.22   }$  & 0.32$^{+0.10   }_{-0.09   }$  & 0.91 & 0.92 &  88\\
MS 0440.5+0204 &    71 &   497 & 9.10  & 5.99   $^{+0.91   }_{-0.73   }$  & 4.45   $^{+1.61   }_{-1.37   }$  & 0.74   $^{+0.29   }_{-0.25   }$  & 0.66$^{+0.32   }_{-0.29   }$  & 0.89 & 0.74 &  28\\
MS 0451.6-0305 &    70 &   378 & 5.68  & 9.25   $^{+0.89   }_{-0.77   }$  & 11.55  $^{+2.88   }_{-1.91   }$  & 1.25   $^{+0.33   }_{-0.23   }$  & 0.42$^{+0.12   }_{-0.11   }$  & 0.95 & 0.94 &  71\\
MS 0735.6+7421 &    69 &   348 & 3.40  & 5.54   $^{+0.24   }_{-0.23   }$  & 6.47   $^{+0.75   }_{-0.65   }$  & 1.17   $^{+0.14   }_{-0.13   }$  & 0.35$^{+0.07   }_{-0.07   }$  & 1.09 & 1.08 &  74\\
MS 0839.8+2938 &    70 &   294 & 3.92  & 4.63   $^{+0.30   }_{-0.28   }$  & 4.64   $^{+0.94   }_{-0.71   }$  & 1.00   $^{+0.21   }_{-0.16   }$  & 0.49$^{+0.13   }_{-0.13   }$  & 0.97 & 0.91 &  69\\
MS 0906.5+1110 &    70 &   435 & 3.60  & 5.56   $^{+0.34   }_{-0.31   }$  & 6.94   $^{+1.23   }_{-0.92   }$  & 1.25   $^{+0.23   }_{-0.18   }$  & 0.34$^{+0.10   }_{-0.10   }$  & 1.20 & 0.97 &  82\\
MS 1006.0+1202 &    70 &   393 & 3.63  & 5.79   $^{+0.54   }_{-0.46   }$  & 7.76   $^{+2.25   }_{-1.56   }$  & 1.34   $^{+0.41   }_{-0.29   }$  & 0.28$^{+0.12   }_{-0.12   }$  & 1.22 & 1.24 &  82\\
MS 1008.1-1224 &    70 &   389 & 6.71  & 5.76   $^{+0.56   }_{-0.47   }$  & 9.88   $^{+2.54   }_{-1.70   }$  & 1.72   $^{+0.47   }_{-0.33   }$  & 0.24$^{+0.11   }_{-0.11   }$  & 1.29 & 1.08 &  83\\
MS 1054.5-0321 &    70 &   395 & 3.69  & 9.75   $^{+1.69   }_{-1.28   }$  & 14.17  $^{+12.06  }_{-4.93   }$  & 1.45   $^{+1.26   }_{-0.54   }$  & 0.16$^{+0.16   }_{-0.16   }$  & 1.05 & 0.85 &  51\\
MS 1455.0+2232 &    69 &   309 & 3.35  & 4.82   $^{+0.14   }_{-0.13   }$  & 5.47   $^{+0.29   }_{-0.27   }$  & 1.13   $^{+0.07   }_{-0.06   }$  & 0.46$^{+0.05   }_{-0.05   }$  & 1.34 & 1.17 &  94\\
MS 1621.5+2640 &    70 &   379 & 3.59  & 5.72   $^{+0.90   }_{-0.72   }$  & 5.10   $^{+2.04   }_{-1.27   }$  & 0.89   $^{+0.38   }_{-0.25   }$  & 0.37$^{+0.23   }_{-0.21   }$  & 1.00 & 0.98 &  74\\
MS 2053.7-0449 $\star$ &    70 &   397 & 5.16  & 4.68   $^{+1.04   }_{-0.75   }$  & 5.37   $^{+1.73   }_{-1.19   }$  & 1.15   $^{+0.45   }_{-0.31   }$  & 0.26$^{+0.26   }_{-0.24   }$  & 0.99 & 0.94 &  65\\
MS 2137.3-2353 &    70 &   354 & 3.40  & 6.00   $^{+0.55   }_{-0.47   }$  & 7.56   $^{+2.79   }_{-1.46   }$  & 1.26   $^{+0.48   }_{-0.26   }$  & 0.35$^{+0.13   }_{-0.12   }$  & 1.08 & 1.28 &  69\\
MS J1157.3+5531 &    69 &   272 & 1.22  & 3.28   $^{+0.36   }_{-0.32   }$  & 6.57   $^{+6.42   }_{-3.33   }$  & 2.00   $^{+1.97   }_{-1.03   }$  & 0.76$^{+0.30   }_{-0.19   }$  & 1.22 & 1.15 &  37\\
NGC 6338 &    71 &   265 & 2.60  & 2.20   $^{+0.07   }_{-0.06   }$  & 2.68   $^{+0.24   }_{-0.20   }$  & 1.22   $^{+0.12   }_{-0.10   }$  & 0.22$^{+0.03   }_{-0.04   }$  & 1.04 & 1.01 &  51\\
PKS 0745-191 &    69 &   460 & 40.80 & 8.30   $^{+0.39   }_{-0.36   }$  & 9.69   $^{+0.84   }_{-0.73   }$  & 1.17   $^{+0.12   }_{-0.10   }$  & 0.42$^{+0.06   }_{-0.07   }$  & 1.01 & 0.97 &  93\\
RBS 0797 &    69 &   350 & 2.22  & 7.63   $^{+0.94   }_{-0.77   }$  & 8.62   $^{+2.60   }_{-1.69   }$  & 1.13   $^{+0.37   }_{-0.25   }$  & 0.25$^{+0.13   }_{-0.13   }$  & 1.06 & 0.83 &  93\\
RDCS 1252-29 &    71 &   196 & 6.06  & 4.63   $^{+2.39   }_{-1.41   }$  & 4.94   $^{+9.84   }_{-2.82   }$  & 1.07   $^{+2.20   }_{-0.69   }$  & 1.14$^{+2.11   }_{-0.83   }$  & 1.36 & 0.28 &  60\\
RX J0232.2-4420 &    69 &   402 & 2.53  & 7.92   $^{+0.85   }_{-0.74   }$  & 10.54  $^{+2.53   }_{-1.74   }$  & 1.33   $^{+0.35   }_{-0.25   }$  & 0.38$^{+0.13   }_{-0.13   }$  & 1.05 & 0.98 &  91\\
RX J0340-4542 &    70 &   291 & 1.63  & 3.10   $^{+0.43   }_{-0.38   }$  & 2.75   $^{+1.15   }_{-0.67   }$  & 0.89   $^{+0.39   }_{-0.24   }$  & 0.63$^{+0.39   }_{-0.28   }$  & 1.22 & 1.30 &  48\\
RX J0439+0520 &    70 &   336 & 10.02 & 4.67   $^{+0.58   }_{-0.47   }$  & 5.37   $^{+2.03   }_{-1.24   }$  & 1.15   $^{+0.46   }_{-0.29   }$  & 0.36$^{+0.22   }_{-0.20   }$  & 0.91 & 0.81 &  85\\
RX J0439.0+0715 $\star$ &    70 &   376 & 11.16 & 5.65   $^{+0.38   }_{-0.34   }$  & 8.21   $^{+1.29   }_{-0.96   }$  & 1.45   $^{+0.25   }_{-0.19   }$  & 0.34$^{+0.09   }_{-0.09   }$  & 1.32 & 1.14 &  87\\
RX J0528.9-3927 &    70 &   454 & 2.36  & 7.96   $^{+1.01   }_{-0.81   }$  & 9.84   $^{+2.92   }_{-1.81   }$  & 1.24   $^{+0.40   }_{-0.26   }$  & 0.26$^{+0.14   }_{-0.15   }$  & 0.96 & 1.04 &  88\\
RX J0647.7+7015 $\star$ &    69 &   361 & 5.18  & 11.46  $^{+2.05   }_{-1.58   }$  & 11.18  $^{+2.46   }_{-1.77   }$  & 0.98   $^{+0.28   }_{-0.20   }$  & 0.24$^{+0.18   }_{-0.20   }$  & 1.00 & 0.92 &  88\\
RX J0819.6+6336 &    71 &   322 & 4.11  & 3.92   $^{+0.46   }_{-0.40   }$  & 3.24   $^{+1.26   }_{-0.66   }$  & 0.83   $^{+0.34   }_{-0.19   }$  & 0.16$^{+0.17   }_{-0.14   }$  & 1.00 & 1.00 &  50\\
RX J0910+5422 $\star$ &    71 &   172 & 2.07  & 4.08   $^{+3.11   }_{-1.34   }$  & 5.00   $^{+5.09   }_{-2.03   }$  & 1.23   $^{+1.56   }_{-0.64   }$  & 0.43$^{+1.89   }_{-0.43   }$  & 0.64 & 0.56 &  42\\
RX J1347.5-1145 $\star$ &    70 &   429 & 4.89  & 15.12  $^{+1.03   }_{-0.86   }$  & 17.32  $^{+1.73   }_{-1.40   }$  & 1.15   $^{+0.14   }_{-0.11   }$  & 0.33$^{+0.07   }_{-0.08   }$  & 1.12 & 1.11 &  96\\
RX J1350+6007 &    71 &   236 & 1.77  & 4.22   $^{+3.13   }_{-1.53   }$  & 3.29   $^{+10.52  }_{-1.93   }$  & 0.78   $^{+2.56   }_{-0.54   }$  & 0.63$^{+5.75   }_{-0.63   }$  & 1.00 & 0.14 &  66\\
RX J1423.8+2404 $\star$ &    71 &   314 & 2.65  & 6.90   $^{+0.39   }_{-0.37   }$  & 7.19   $^{+0.59   }_{-0.52   }$  & 1.04   $^{+0.10   }_{-0.09   }$  & 0.38$^{+0.07   }_{-0.08   }$  & 0.94 & 0.90 &  90\\
RX J1504.1-0248 &    70 &   445 & 6.27  & 8.02   $^{+0.26   }_{-0.25   }$  & 8.52   $^{+0.58   }_{-0.50   }$  & 1.06   $^{+0.08   }_{-0.07   }$  & 0.39$^{+0.04   }_{-0.05   }$  & 1.25 & 1.17 &  95\\
RX J1525+0958 &    70 &   296 & 2.96  & 3.83   $^{+0.84   }_{-0.53   }$  & 9.10   $^{+7.62   }_{-3.25   }$  & 2.38   $^{+2.06   }_{-0.91   }$  & 0.69$^{+0.47   }_{-0.36   }$  & 1.96 & 0.08 &  83\\
RX J1532.9+3021 $\star$ &    70 &   322 & 2.21  & 6.06   $^{+0.43   }_{-0.39   }$  & 7.20   $^{+0.94   }_{-0.77   }$  & 1.19   $^{+0.18   }_{-0.15   }$  & 0.46$^{+0.10   }_{-0.11   }$  & 0.92 & 1.02 &  83\\
RX J1716.9+6708 &    71 &   342 & 3.71  & 6.51   $^{+1.79   }_{-1.24   }$  & 6.21   $^{+4.03   }_{-2.26   }$  & 0.95   $^{+0.67   }_{-0.39   }$  & 0.56$^{+0.39   }_{-0.32   }$  & 0.84 & 0.92 &  63\\
RX J1720.1+2638 &    69 &   359 & 4.02  & 6.33   $^{+0.29   }_{-0.25   }$  & 7.71   $^{+0.84   }_{-0.65   }$  & 1.22   $^{+0.14   }_{-0.11   }$  & 0.37$^{+0.07   }_{-0.07   }$  & 1.04 & 0.96 &  94\\
RX J1720.2+3536 $\star$ &    71 &   320 & 3.35  & 7.34   $^{+0.59   }_{-0.50   }$  & 7.40   $^{+0.86   }_{-0.71   }$  & 1.01   $^{+0.14   }_{-0.12   }$  & 0.43$^{+0.11   }_{-0.11   }$  & 1.03 & 0.94 &  91\\
RX J2011.3-5725 &    71 &   295 & 4.76  & 4.10   $^{+0.47   }_{-0.39   }$  & 3.93   $^{+0.98   }_{-0.70   }$  & 0.96   $^{+0.26   }_{-0.19   }$  & 0.41$^{+0.24   }_{-0.20   }$  & 0.95 & 1.08 &  84\\
RX J2129.6+0005 &    70 &   489 & 4.30  & 6.01   $^{+0.55   }_{-0.46   }$  & 7.19   $^{+1.68   }_{-1.21   }$  & 1.20   $^{+0.30   }_{-0.22   }$  & 0.51$^{+0.16   }_{-0.15   }$  & 1.29 & 1.34 &  87\\
S0463 $\star$ &    70 &   307 & 1.06  & 3.26   $^{+0.33   }_{-0.38   }$  & 3.92   $^{+1.16   }_{-0.94   }$  & 1.20   $^{+0.38   }_{-0.32   }$  & 0.23$^{+0.18   }_{-0.15   }$  & 1.08 & 1.08 &  54\\
TRIANG AUSTR &    71 &   539 & 13.27 & 8.50   $^{+0.29   }_{-0.25   }$  & 12.08  $^{+1.13   }_{-1.13   }$  & 1.42   $^{+0.14   }_{-0.14   }$  & 0.03$^{+0.04   }_{-0.03   }$  & 0.01 & 1.93 &  83\\
V 1121.0+2327 &    70 &   315 & 1.30  & 4.17   $^{+0.78   }_{-0.60   }$  & 4.70   $^{+3.00   }_{-1.17   }$  & 1.13   $^{+0.75   }_{-0.32   }$  & 0.46$^{+0.36   }_{-0.28   }$  & 1.09 & 0.87 &  74\\
ZWCL 1215 &    70 &   277 & 1.76  & 6.64   $^{+0.46   }_{-0.38   }$  & 8.69   $^{+0.74   }_{-0.80   }$  & 1.31   $^{+0.14   }_{-0.14   }$  & 0.37$^{+0.11   }_{-0.11   }$  & 1.10 & 1.03 &  91\\
ZWCL 1358+6245 &    70 &   391 & 1.94  & 9.70   $^{+1.16   }_{-0.94   }$  & 9.04   $^{+2.09   }_{-1.46   }$  & 0.93   $^{+0.24   }_{-0.18   }$  & 0.57$^{+0.19   }_{-0.19   }$  & 1.03 & 0.90 &  65\\
ZWCL 1953 &    69 &   516 & 3.10  & 8.28   $^{+1.22   }_{-0.96   }$  & 11.83  $^{+4.01   }_{-2.55   }$  & 1.43   $^{+0.53   }_{-0.35   }$  & 0.21$^{+0.14   }_{-0.15   }$  & 0.87 & 0.77 &  82\\
ZWCL 3146 &    70 &   512 & 2.70  & 7.46   $^{+0.32   }_{-0.30   }$  & 8.99   $^{+0.94   }_{-0.78   }$  & 1.21   $^{+0.14   }_{-0.12   }$  & 0.31$^{+0.06   }_{-0.05   }$  & 1.06 & 0.97 &  91\\
ZWCL 5247 &    70 &   449 & 1.70  & 4.89   $^{+0.86   }_{-0.65   }$  & 4.39   $^{+2.30   }_{-1.21   }$  & 0.90   $^{+0.50   }_{-0.27   }$  & 0.37$^{+0.30   }_{-0.25   }$  & 1.09 & 0.93 &  78\\
ZWCL 7160 &    69 &   451 & 3.10  & 4.63   $^{+0.42   }_{-0.36   }$  & 5.41   $^{+1.06   }_{-0.80   }$  & 1.17   $^{+0.25   }_{-0.20   }$  & 0.36$^{+0.14   }_{-0.14   }$  & 0.94 & 0.95 &  87\\
ZWICKY 2701 &    69 &   315 & 0.83  & 5.08   $^{+0.32   }_{-0.30   }$  & 4.96   $^{+0.87   }_{-0.69   }$  & 0.98   $^{+0.18   }_{-0.15   }$  & 0.45$^{+0.13   }_{-0.11   }$  & 0.95 & 0.76 &  70\\
ZwCL 1332.8+5043 &    70 &   453 & 1.10  & 3.82   $^{+3.34   }_{-1.42   }$  & 2.86   $^{+3.96   }_{-1.21   }$  & 0.75   $^{+1.23   }_{-0.42   }$  & 0.16$^{+4.75   }_{-0.16   }$  & 0.71 & 0.95 &  60\\
ZwCl 0848.5+3341 &    71 &   365 & 1.12  & 6.54   $^{+2.04   }_{-1.27   }$  & 6.41   $^{+3.79   }_{-1.88   }$  & 0.98   $^{+0.66   }_{-0.34   }$  & 0.59$^{+0.59   }_{-0.48   }$  & 0.89 & 1.01 &  47\\
\end{rotthesistable}
\doublespacing

%%%%%%%%%%%%%%%%%%%%%%%%%%%%%%%%%%%%%%%%%%%%%%%%
\chapter{Tables cited in Chapter \ref{ch:ent_supp}}
%%%%%%%%%%%%%%%%%%%%%%%%%%%%%%%%%%%%%%%%%%%%%%%%

\begin{center}
\noindent{\bf{Table \ref{tab:entsuppsample} Notes}}\\
\end{center}
Col. (1) Cluster name; col. (2) CXC CDA Observation Identification
Number; col. (3) R.A. of cluster center; col. (4) Decl. of cluster
center; col. (5) exposure time; col. (6) observing mode; col. (7) CCD
location of cluster center; col. (8) redshift; col. (9) average
cluster temperature; col. (10) core entropy measured in this work;
col. (11) cluster bolometric luminosity; and col. (12) notes are as
follows: (a) - cluster analyzed using the best-fit $\beta$-model for
the surface brightness profiles (discussed in
\S\ref{sec:entsuppdene}); (b) - clusters with complex surface brightness of
which only the central regions were used in fitting $K(r)$; (c) -
cluster only used during analysis of the \hifl\ sub-sample (discussed
in \S\ref{sec:entsupphifl}); (d) - cluster with central AGN removed during
analysis (discussed in \S\ref{sec:entsuppcentsrc}); (e) - cluster with
central compact source removed during analysis (discussed in
\S\ref{sec:entsuppcentsrc}); and (f) - cluster with central bin ignored
during fitting (discussed in \S\ref{sec:entsuppcentsrc}).

\begin{center}
\noindent{\bf{Table \ref{tab:betafits} Notes}}\\
\end{center}
Col. (1) Cluster name; col. (2) central surface brightness of first
component; col. (3) core radius of first component; col. (4) $\beta$
parameter of first component; col. (5) central surface brightness of
second component; col. (6) core radius of second component; col. (7)
$\beta$ parameter of second component; col. (8) model degrees of
freedom; and col. (9) reduced chi-squared statistic for best-fit
model.

\begin{center}
\noindent{\bf{Table \ref{tab:bfparams} Notes}}\\
\end{center}
Listed here are the mean best-fit parameters of the model $K(r) = \kna
+ \khun (r/100 \kpc)^{\alpha}$ for various sub-groups of the full
\accept\ sample. The 'CSE' sample are the clusters with a central
source excluded (discussed in \S\ref{sec:entsuppcentsrc}). The $K_{12}$
values represent the entropy at 12 kpc and are calculated from the
best-fit models. Col. (1) Sample being considered; col. (2) number of
objects in the sub-group; col. (3) fraction of objects with p $>$ 0.05
for power-law only model (eqn. \ref{eqn:plaw}); col. (4) fraction of
objects with p $>$ 0.05 for power-law with constant core entropy model
(eqn. \ref{eqn:k0}); col. (5) fraction of objects which do not meet p
$>$ 0.05 criterion for either model; col. (6) mean best-fit \kna;
col. (7) mean entropy at 12 kpc; col. (8) mean best-fit \khun; and
col. (9) mean best-fit power-law index; and cols. (10,11,12) number of
clusters consistent with $\kna = 0 \ent$ at $1\sigma$, $2\sigma$, and
$3\sigma$ significance, respectively. Percentage of the sub-group
represented by each is also listed.

\begin{center}
\noindent{\bf{Table \ref{tab:kfits} Notes}}\\
\end{center}
Col. (1) Cluster name; col. (2) CDA observation identification number;
col. (3) method of $T_X$ interpolation (discussed in \S\ref{sec:entsuppkpr});
col. (4) maximum radius for fit; col. (5) number of radial bins
included in fit; col. (6) best-fit core entropy; col. (7) number of
sigma \kna\ is away from zero; col. (9) best-fit entropy at 100 kpc;
col. (10) best-fit power-law index; col. (11) degrees of freedom in
fit; col. (12) \chisq\ statistic of best-fit model; and col. (13)
probability of worse fit given \chisq\ and degrees of freedom.

\clearpage
\clearpage
\singlespacing
\begin{rotthesistable}{lcccccccc}
\thesistablehead{Summary of Sample for Entropy Study}{Summary of Sample for Entropy Study}{Cluster & Obs. ID & R.A. & Decl. & Exposure Time & ACIS & $z$ & $kT_{X}$ & Notes\\ &  & hr:min:sec & $^{\circ}:':''$ & ksec &  &  & keV & \\(1) & (2) & (3) & (4) & (5) & (6) & (7) & (8) & (9)}{tab:entsuppsample}
1E0657 56 & 3184 & 06:58:29.627 & -55:56:39.79 & 87.5 & I3 & 0.2960 & 11.64 & \nodata\\
 & 5356 & \nodata & \nodata & 97.2 & I2 & \nodata & \nodata & \nodata\\
 & 5361 & \nodata & \nodata & 82.6 & I3 & \nodata & \nodata & \nodata\\
2A 335+096 &  919 & 03:38:41.105 & +09:58:00.66 & 19.7 & S3 & 0.0347 & 2.88 & \nodata\\
2PIGG J0011.5-2850 & 5797 & 00:11:21.623 & -28:51:14.44 & 19.9 & I3 & 0.0753 & 5.15 &      f\\
2PIGG J2227.0-3041 & 5798 & 22:27:54.560 & -30:34:34.84 & 22.3 & I2 & 0.0729 & 2.79 & \nodata\\
3C 28.0 & 3233 & 00:55:50.401 & +26:24:36.47 & 49.7 & I3 & 0.1952 & 5.53 & \nodata\\
3C 295 & 2254 & 14:11:20.280 & +52:12:10.55 & 90.9 & I3 & 0.4641 & 5.16 &      d\\
3C 388 & 5295 & 18:44:02.365 & +45:33:29.31 & 30.7 & I3 & 0.0917 & 3.23 &      d\\
4C 55.16 & 4940 & 08:34:54.923 & +55:34:21.15 & 96.0 & S3 & 0.2420 & 4.98 &      d\\
Abell 13 & 4945 & 00:13:37.883 & -19:30:09.10 & 55.3 & S3 & 0.0940 & 6.84 & \nodata\\
Abell 68 & 3250 & 00:37:06.475 & +09:09:32.28 & 10.0 & I3 & 0.2546 & 9.01 & \nodata\\
Abell 85 &  904 & 00:41:50.406 & -09:18:10.79 & 38.4 & I0 & 0.0558 & 6.40 & \nodata\\
Abell 119 & 4180 & 00:56:15.150 & -01:14:59.70 & 11.9 & I3 & 0.0442 & 5.86 &    a,e\\
Abell 133 & 2203 & 01:02:41.756 & -21:52:49.79 & 35.5 & S3 & 0.0558 & 4.31 & \nodata\\
Abell 141 & 9410 & 01:05:34.385 & -24:37:58.78 & 19.9 & I3 & 0.2300 & 5.31 & \nodata\\
Abell 160 & 3219 & 01:13:00.692 & +15:29:15.08 & 58.5 & I3 & 0.0447 & 1.88 &    a,e\\
Abell 193 & 6931 & 01:25:07.660 & +08:41:57.08 & 17.9 & S3 & 0.0485 & 2.50 &    a,e\\
Abell 209 & 3579 & 01:31:52.565 & -13:36:38.79 & 10.0 & I3 & 0.2060 & 8.28 & \nodata\\
 &  522 & \nodata & \nodata & 10.0 & I3 & \nodata & \nodata & \nodata\\
Abell 222 & 4967 & 01:37:34.562 & -12:59:34.88 & 45.1 & I3 & 0.2130 & 4.60 & \nodata\\
Abell 223 & 49671 & 01:37:55.963 & -12:49:10.53 & 45.1 & I0 & 0.2070 & 5.28 &      e\\
Abell 262 & 2215 & 01:52:46.299 & +36:09:11.80 & 28.7 & S3 & 0.0164 & 2.18 & \nodata\\
 & 7921 & \nodata & \nodata & 110.7 & S3 & \nodata & \nodata & \nodata\\
Abell 267 & 1448 & 01:52:42.269 & +01:00:45.33 & 7.9 & I3 & 0.2300 & 6.79 & \nodata\\
 & 3580 & \nodata & \nodata & 19.9 & I3 & \nodata & \nodata & \nodata\\
Abell 368 & 9412 & 02:37:27.640 & -26:30:28.99 & 18.4 & I3 & 0.2200 & 6.23 & \nodata\\
Abell 370 &  515 & 02:39:53.169 & -01:34:36.96 & 88.0 & S3 & 0.3747 & 7.35 & \nodata\\
Abell 383 & 2321 & 02:48:03.364 & -03:31:44.69 & 19.5 & S3 & 0.1871 & 4.91 & \nodata\\
Abell 399 & 3230 & 02:57:53.382 & +13:01:30.86 & 48.6 & I0 & 0.0716 & 7.95 & \nodata\\
Abell 400 & 4181 & 02:57:41.603 & +06:01:27.61 & 21.5 & I3 & 0.0240 & 2.31 &    a,e\\
Abell 401 & 2309 & 02:58:56.920 & +13:34:14.51 & 11.6 & I2 & 0.0745 & 8.07 & \nodata\\
 &  518 & \nodata & \nodata & 18.0 & I3 & \nodata & \nodata & \nodata\\
Abell 426 & 3209 & 03:19:48.194 & +41:30:40.73 & 95.8 & S3 & 0.0179 & 3.55 &      d\\
 & 4289 & \nodata & \nodata & 95.4 & S3 & \nodata & \nodata & \nodata\\
Abell 478 & 1669 & 04:13:25.345 & +10:27:55.15 & 42.4 & S3 & 0.0883 & 7.07 & \nodata\\
 & 6102 & \nodata & \nodata & 10.0 & I3 & \nodata & \nodata & \nodata\\
Abell 496 & 3361 & 04:33:38.038 & -13:15:39.65 & 10.0 & S3 & 0.0328 & 5.03 & \nodata\\
Abell 520 & 4215 & 04:54:10.303 & +02:55:36.48 & 66.3 & I3 & 0.2020 & 9.29 & \nodata\\
Abell 521 &  430 & 04:54:06.337 & -10:13:16.88 & 39.1 & S3 & 0.2533 & 7.03 & \nodata\\
Abell 539 & 5808 & 05:16:37.335 & +06:26:25.18 & 24.3 & I3 & 0.0288 & 3.24 &    b,e\\
 & 7209 & \nodata & \nodata & 18.6 & I3 & \nodata & \nodata & \nodata\\
Abell 562 & 6936 & 06:53:21.524 & +69:19:51.19 & 51.5 & S3 & 0.1100 & 3.04 &      e\\
Abell 576 & 3289 & 07:21:30.394 & +55:45:41.95 & 38.6 & S3 & 0.0385 & 4.43 &      e\\
Abell 586 &  530 & 07:32:20.339 & +31:37:58.59 & 10.0 & I3 & 0.1710 & 6.47 & \nodata\\
Abell 611 & 3194 & 08:00:56.832 & +36:03:24.09 & 36.1 & S3 & 0.2880 & 7.06 &      e\\
Abell 644 & 2211 & 08:17:25.225 & -07:30:40.03 & 29.7 & I3 & 0.0698 & 7.73 & \nodata\\
Abell 665 & 3586 & 08:30:59.226 & +65:50:20.06 & 29.7 & I3 & 0.1810 & 7.45 & \nodata\\
Abell 697 & 4217 & 08:42:57.549 & +36:21:57.65 & 19.5 & I3 & 0.2820 & 9.52 & \nodata\\
Abell 744 & 6947 & 09:07:20.455 & +16:39:06.18 & 39.5 & I3 & 0.0729 & 2.50 &      e\\
Abell 754 &  577 & 09:09:18.188 & -09:41:09.56 & 44.2 & I3 & 0.0543 & 9.94 & \nodata\\
Abell 773 & 5006 & 09:17:52.566 & +51:43:38.18 & 19.8 & I3 & 0.2170 & 7.83 & \nodata\\
Abell 907 & 3185 & 09:58:21.946 & -11:03:50.73 & 48.0 & I3 & 0.1527 & 5.59 & \nodata\\
 & 3205 & \nodata & \nodata & 47.1 & I3 & \nodata & \nodata & \nodata\\
 &  535 & \nodata & \nodata & 11.0 & I3 & \nodata & \nodata & \nodata\\
Abell 963 &  903 & 10:17:03.744 & +39:02:49.17 & 36.3 & S3 & 0.2056 & 6.73 & \nodata\\
Abell 1060 & 2220 & 10:36:42.828 & -27:31:42.06 & 31.9 & I3 & 0.0125 & 3.29 &  a,e,f\\
Abell 1063S & 4966 & 22:48:44.294 & -44:31:48.37 & 26.7 & I3 & 0.3540 & 11.96 & \nodata\\
Abell 1068 & 1652 & 10:40:44.520 & +39:57:10.28 & 26.8 & S3 & 0.1375 & 4.62 & \nodata\\
Abell 1201 & 4216 & 11:12:54.489 & +13:26:08.76 & 39.7 & S3 & 0.1688 & 5.61 & \nodata\\
Abell 1204 & 2205 & 11:13:20.419 & +17:35:38.45 & 23.6 & I3 & 0.1706 & 3.63 & \nodata\\
Abell 1240 & 4961 & 11:23:38.357 & +43:05:48.33 & 51.3 & I3 & 0.1590 & 4.77 &      a\\
Abell 1361 & 2200 & 11:43:39.637 & +46:21:20.41 & 16.7 & S3 & 0.1171 & 5.32 & \nodata\\
Abell 1413 & 5003 & 11:55:17.893 & +23:24:21.84 & 75.1 & I2 & 0.1426 & 7.41 & \nodata\\
Abell 1423 &  538 & 11:57:17.263 & +33:36:37.44 & 9.8 & I3 & 0.2130 & 6.01 & \nodata\\
Abell 1446 & 4975 & 12:02:03.744 & +58:02:17.93 & 58.4 & S3 & 0.1035 & 3.96 & \nodata\\
Abell 1569 & 6100 & 12:36:26.015 & +16:32:17.81 & 41.2 & I3 & 0.0735 & 2.51 & \nodata\\
Abell 1576 & 7938 & 12:36:58.274 & +63:11:13.88 & 15.0 & I3 & 0.2790 & 10.10 & \nodata\\
Abell 1644 & 2206 & 12:57:11.665 & -17:24:32.86 & 18.7 & I3 & 0.0471 & 4.60 &      b\\
 & 7922 & \nodata & \nodata & 51.5 & I3 & \nodata & \nodata & \nodata\\
Abell 1650 & 4178 & 12:58:41.499 & -01:45:44.32 & 27.3 & S3 & 0.0843 & 6.17 & \nodata\\
Abell 1651 & 4185 & 12:59:22.830 & -04:11:45.86 & 9.6 & I3 & 0.0840 & 6.26 & \nodata\\
Abell 1664 & 1648 & 13:03:42.622 & -24:14:41.59 & 9.8 & S3 & 0.1276 & 4.39 & \nodata\\
 & 7901 & \nodata & \nodata & 36.6 & S3 & \nodata & \nodata & \nodata\\
Abell 1689 & 1663 & 13:11:29.612 & -01:20:28.69 & 10.7 & I3 & 0.1843 & 10.10 & \nodata\\
 & 5004 & \nodata & \nodata & 19.9 & I3 & \nodata & \nodata & \nodata\\
 &  540 & \nodata & \nodata & 10.3 & I3 & \nodata & \nodata & \nodata\\
Abell 1736 & 4186 & 13:26:49.453 & -27:09:48.13 & 14.9 & I1 & 0.0338 & 3.45 &    a,e\\
Abell 1758 & 2213 & 13:32:48.398 & +50:32:32.53 & 58.3 & S3 & 0.2792 & 12.14 & \nodata\\
Abell 1763 & 3591 & 13:35:17.957 & +40:59:55.80 & 19.6 & I3 & 0.1866 & 7.78 & \nodata\\
Abell 1795 &  493 & 13:48:52.802 & +26:35:23.55 & 19.6 & S3 & 0.0625 & 7.80 & \nodata\\
 & 5289 & \nodata & \nodata & 15.0 & I3 & \nodata & \nodata & \nodata\\
Abell 1835 &  495 & 14:01:01.951 & +02:52:43.18 & 19.5 & S3 & 0.2532 & 9.77 & \nodata\\
Abell 1914 & 3593 & 14:26:03.060 & +37:49:27.84 & 18.9 & I3 & 0.1712 & 9.62 & \nodata\\
Abell 1942 & 3290 & 14:38:21.878 & +03:40:12.97 & 57.6 & I2 & 0.2240 & 4.77 & \nodata\\
Abell 1991 & 3193 & 14:54:31.620 & +18:38:41.48 & 38.3 & S3 & 0.0587 & 2.67 & \nodata\\
Abell 1995 & 7021 & 14:52:57.410 & +58:02:56.84 & 48.5 & I3 & 0.3186 & 3.40 & \nodata\\
Abell 2029 & 4977 & 15:10:56.139 & +05:44:40.47 & 77.9 & S3 & 0.0765 & 7.38 & \nodata\\
 & 6101 & \nodata & \nodata & 9.9 & I3 & \nodata & \nodata & \nodata\\
 &  891 & \nodata & \nodata & 19.8 & S3 & \nodata & \nodata & \nodata\\
Abell 2034 & 2204 & 15:10:12.498 & +33:30:39.57 & 53.9 & I3 & 0.1130 & 7.15 &      f\\
Abell 2052 & 5807 & 15:16:44.514 & +07:01:17.02 & 127.0 & S3 & 0.0353 & 2.98 &      d\\
 &  890 & \nodata & \nodata & 36.8 & S3 & \nodata & \nodata & \nodata\\
Abell 2063 & 4187 & 15:23:04.851 & +08:36:20.16 & 8.8 & I3 & 0.0351 & 3.61 & \nodata\\
 & 6263 & \nodata & \nodata & 16.8 & S3 & \nodata & \nodata & \nodata\\
Abell 2065 & 31821 & 15:22:29.517 & +27:42:22.93 & 27.7 & I3 & 0.0730 & 5.75 & \nodata\\
Abell 2069 & 4965 & 15:24:11.376 & +29:52:19.02 & 55.4 & I2 & 0.1160 & 6.50 & \nodata\\
Abell 2104 &  895 & 15:40:08.131 & -03:18:15.02 & 49.2 & S3 & 0.1554 & 8.53 & \nodata\\
Abell 2107 & 4960 & 15:39:39.113 & +21:46:57.66 & 35.6 & I3 & 0.0411 & 3.82 &      b\\
Abell 2111 &  544 & 15:39:40.637 & +34:25:28.01 & 10.3 & I3 & 0.2300 & 7.13 & \nodata\\
Abell 2124 & 3238 & 15:44:59.131 & +36:06:34.11 & 19.4 & S3 & 0.0658 & 4.73 & \nodata\\
Abell 2125 & 2207 & 15:41:14.154 & +66:15:57.20 & 81.5 & I3 & 0.2465 & 2.88 &      a\\
Abell 2142 & 1196 & 15:58:20.880 & +27:13:44.21 & 11.4 & S3 & 0.0898 & 8.24 & \nodata\\
 & 1228 & \nodata & \nodata & 12.1 & S3 & \nodata & \nodata & \nodata\\
 & 5005 & \nodata & \nodata & 44.6 & I3 & \nodata & \nodata & \nodata\\
Abell 2147 & 3211 & 16:02:17.025 & +15:58:28.32 & 17.9 & I3 & 0.0356 & 4.09 & \nodata\\
Abell 2151 & 4996 & 16:04:35.887 & +17:43:17.36 & 21.8 & I3 & 0.0366 & 2.90 &      e\\
Abell 2163 & 1653 & 16:15:45.705 & -06:09:00.62 & 71.1 & I1 & 0.1695 & 19.20 & \nodata\\
Abell 2199 &  497 & 16:28:38.249 & +39:33:04.28 & 19.5 & S3 & 0.0300 & 4.55 &      b\\
Abell 2204 &  499 & 16:32:46.920 & +05:34:32.86 & 10.1 & S3 & 0.1524 & 6.97 & \nodata\\
 & 6104 & \nodata & \nodata & 9.6 & I3 & \nodata & \nodata & \nodata\\
 & 7940 & \nodata & \nodata & 77.1 & I3 & \nodata & \nodata & \nodata\\
Abell 2218 & 1666 & 16:35:50.831 & +66:12:42.31 & 48.6 & I0 & 0.1713 & 7.35 & \nodata\\
Abell 2219 &  896 & 16:40:20.112 & +46:42:42.84 & 42.3 & S3 & 0.2256 & 12.75 & \nodata\\
Abell 2244 & 4179 & 17:02:42.579 & +34:03:37.34 & 57.0 & S3 & 0.0967 & 5.68 & \nodata\\
Abell 2255 &  894 & 17:12:42.935 & +64:04:10.81 & 39.4 & I3 & 0.0805 & 6.12 &      a\\
Abell 2256 & 1386 & 17:03:44.567 & +78:38:11.51 & 12.4 & I3 & 0.0579 & 6.90 &      a\\
Abell 2259 & 3245 & 17:20:08.299 & +27:40:11.53 & 10.0 & I3 & 0.1640 & 5.18 & \nodata\\
Abell 2261 & 5007 & 17:22:27.254 & +32:07:58.60 & 24.3 & I3 & 0.2240 & 7.63 & \nodata\\
Abell 2294 & 3246 & 17:24:10.149 & +85:53:09.77 & 10.0 & I3 & 0.1780 & 9.98 & \nodata\\
Abell 2319 & 3231 & 19:21:09.638 & +43:57:21.53 & 14.4 & I1 & 0.0562 & 10.87 &      a\\
Abell 2384 & 4202 & 21:52:21.178 & -19:32:51.90 & 31.5 & I3 & 0.0945 & 4.75 & \nodata\\
Abell 2390 & 4193 & 21:53:36.825 & +17:41:44.38 & 95.1 & S3 & 0.2301 & 11.15 & \nodata\\
Abell 2409 & 3247 & 22:00:52.567 & +20:58:06.55 & 10.2 & I3 & 0.1479 & 5.94 & \nodata\\
Abell 2420 & 8271 & 22:10:18.792 & -12:10:13.35 & 8.1 & I3 & 0.0846 & 6.47 & \nodata\\
Abell 2462 & 4159 & 22:39:11.367 & -17:20:28.33 & 39.2 & S3 & 0.0737 & 2.42 &    a,e\\
Abell 2537 & 4962 & 23:08:22.313 & -02:11:29.88 & 36.2 & S3 & 0.2950 & 8.40 & \nodata\\
Abell 2554 & 1696 & 23:12:19.622 & -21:30:11.32 & 19.9 & S3 & 0.1103 & 5.29 & \nodata\\
Abell 2556 & 2226 & 23:13:01.413 & -21:38:04.47 & 19.9 & S3 & 0.0862 & 3.50 & \nodata\\
Abell 2589 & 3210 & 23:23:57.315 & +16:46:38.43 & 13.7 & S3 & 0.0415 & 3.65 & \nodata\\
Abell 2597 &  922 & 23:25:19.779 & -12:07:27.63 & 39.4 & S3 & 0.0854 & 4.02 & \nodata\\
Abell 2626 & 3192 & 23:36:30.452 & +21:08:47.36 & 24.8 & S3 & 0.0573 & 3.29 & \nodata\\
Abell 2631 & 3248 & 23:37:38.560 & +00:16:05.02 & 9.2 & I3 & 0.2779 & 7.06 &      a\\
Abell 2657 & 4941 & 23:44:57.253 & +09:11:30.74 & 16.1 & I3 & 0.0402 & 3.77 & \nodata\\
Abell 2667 & 2214 & 23:51:39.395 & -26:05:02.75 & 9.6 & S3 & 0.2300 & 6.75 & \nodata\\
Abell 2717 & 6974 & 00:03:12.968 & -35:56:00.13 & 19.8 & I3 & 0.0475 & 1.69 &      e\\
Abell 2744 & 2212 & 00:14:19.529 & -30:23:30.24 & 24.8 & S3 & 0.3080 & 9.18 & \nodata\\
 & 7915 & \nodata & \nodata & 18.6 & I3 & \nodata & \nodata & \nodata\\
 & 8477 & \nodata & \nodata & 45.9 & I3 & \nodata & \nodata & \nodata\\
 & 8557 & \nodata & \nodata & 27.8 & I3 & \nodata & \nodata & \nodata\\
Abell 2813 & 9409 & 00:43:24.881 & -20:37:25.08 & 19.9 & I3 & 0.2924 & 8.96 & \nodata\\
Abell 3084 & 9413 & 03:04:03.920 & -36:56:27.17 & 19.9 & I3 & 0.0977 & 5.30 & \nodata\\
Abell 3088 & 9414 & 03:07:01.734 & -28:39:55.47 & 18.9 & I3 & 0.2534 & 6.71 & \nodata\\
Abell 3112 & 2516 & 03:17:57.681 & -44:14:17.16 & 16.9 & S3 & 0.0720 & 5.17 &      d\\
Abell 3120 & 6951 & 03:21:56.464 & -51:19:35.40 & 26.8 & I3 & 0.0690 & 4.40 & \nodata\\
Abell 3158 & 3201 & 03:42:54.675 & -53:37:24.36 & 24.8 & I3 & 0.0580 & 4.94 & \nodata\\
 & 3712 & \nodata & \nodata & 30.9 & I3 & \nodata & \nodata & \nodata\\
Abell 3266 &  899 & 04:31:13.304 & -61:27:12.59 & 29.8 & I1 & 0.0590 & 9.07 & \nodata\\
Abell 3364 & 9419 & 05:47:37.698 & -31:52:23.61 & 19.8 & I3 & 0.1483 & 7.88 & \nodata\\
Abell 3376 & 3202 & 06:02:11.756 & -39:56:59.07 & 44.3 & I3 & 0.0456 & 4.08 &      a\\
 & 3450 & \nodata & \nodata & 19.8 & I3 & \nodata & \nodata & \nodata\\
Abell 3391 & 4943 & 06:26:21.511 & -53:41:44.81 & 18.4 & I3 & 0.0560 & 6.07 &    a,e\\
Abell 3395 & 4944 & 06:26:48.463 & -54:32:59.21 & 21.9 & I3 & 0.0510 & 5.13 &    a,e\\
Abell 3528S & 8268 & 12:54:40.897 & -29:13:38.10 & 8.1 & I3 & 0.0530 & 5.44 & \nodata\\
Abell 3558 & 1646 & 13:27:56.854 & -31:29:43.78 & 14.4 & S3 & 0.0480 & 6.60 &    e,f\\
Abell 3562 & 4167 & 13:33:37.800 & -31:40:12.04 & 19.3 & I2 & 0.0490 & 4.59 & \nodata\\
Abell 3571 & 4203 & 13:47:28.434 & -32:51:52.45 & 34.0 & S3 & 0.0391 & 7.77 & \nodata\\
Abell 3581 & 1650 & 14:07:29.777 & -27:01:05.88 & 7.2 & S3 & 0.0218 & 2.10 &      d\\
Abell 3667 & 5751 & 20:12:41.231 & -56:50:35.70 & 128.9 & I3 & 0.0556 & 6.51 & \nodata\\
 & 5752 & \nodata & \nodata & 60.4 & I3 & \nodata & \nodata & \nodata\\
 & 5753 & \nodata & \nodata & 103.6 & I3 & \nodata & \nodata & \nodata\\
 &  889 & \nodata & \nodata & 50.3 & I2 & \nodata & \nodata & \nodata\\
Abell 3822 & 8269 & 21:54:04.203 & -57:52:02.71 & 8.1 & I3 & 0.0759 & 4.89 &      e\\
Abell 3827 & 7920 & 22:01:53.200 & -59:56:43.04 & 45.6 & S3 & 0.0984 & 8.05 & \nodata\\
Abell 3921 & 4973 & 22:49:57.829 & -64:25:42.17 & 29.4 & I3 & 0.0927 & 5.69 & \nodata\\
Abell 4038 & 4992 & 23:47:43.180 & -28:08:34.81 & 33.5 & I2 & 0.0300 & 3.11 & \nodata\\
Abell 4059 & 5785 & 23:57:01.065 & -34:45:33.28 & 92.1 & S3 & 0.0475 & 4.69 & \nodata\\
Abell S0405 & 8272 & 03:51:32.815 & -82:13:10.19 & 7.9 & I3 & 0.0613 & 4.11 & \nodata\\
Abell S0592 & 9420 & 06:38:48.610 & -53:58:26.32 & 19.9 & I3 & 0.2216 & 9.08 & \nodata\\
AC 114 & 1562 & 22:58:48.316 & -34:48:08.20 & 72.5 & S3 & 0.3120 & 7.53 & \nodata\\
AWM7 &  908 & 02:54:27.631 & +41:34:47.07 & 47.9 & I3 & 0.0172 & 3.71 &      b\\
Centaurus & 4190 & 12:48:49.267 & -41:18:39.54 & 34.3 & S3 & 0.0109 & 3.96 &      b\\
 & 4191 & \nodata & \nodata & 34.0 & S3 & \nodata & \nodata & \nodata\\
 & 4954 & \nodata & \nodata & 89.1 & S3 & \nodata & \nodata & \nodata\\
 & 4955 & \nodata & \nodata & 44.7 & S3 & \nodata & \nodata & \nodata\\
 &  504 & \nodata & \nodata & 31.8 & S3 & \nodata & \nodata & \nodata\\
 &  505 & \nodata & \nodata & 10.0 & S3 & \nodata & \nodata & \nodata\\
 & 5310 & \nodata & \nodata & 49.3 & S3 & \nodata & \nodata & \nodata\\
CID 72 & 2018 & 17:33:03.247 & +43:45:37.28 & 30.7 & S3 & 0.0344 & 1.91 & \nodata\\
 & 6949 & \nodata & \nodata & 38.6 & I3 & \nodata & \nodata & \nodata\\
 & 7321 & \nodata & \nodata & 37.5 & I3 & \nodata & \nodata & \nodata\\
 & 7322 & \nodata & \nodata & 37.5 & I3 & \nodata & \nodata & \nodata\\
CL J1226.9+3332 & 3180 & 12:26:58.373 & +33:32:47.36 & 31.7 & I3 & 0.8900 & 10.00 & \nodata\\
 & 5014 & \nodata & \nodata & 32.7 & I3 & \nodata & \nodata & \nodata\\
 &  932 & \nodata & \nodata & 9.8 & S3 & \nodata & \nodata & \nodata\\
Cygnus A &  360 & 19:59:28.381 & +40:44:01.98 & 34.7 & S3 & 0.0561 & 7.68 &      d\\
ESO 3060170 & 3188 & 05:40:06.687 & -40:50:12.82 & 14.0 & I3 & 0.0358 & 2.79 &      b\\
 & 3189 & \nodata & \nodata & 14.1 & I0 & \nodata & \nodata & \nodata\\
ESO 5520200 & 3206 & 04:54:52.318 & -18:06:56.52 & 23.9 & I3 & 0.0314 & 2.37 & \nodata\\
EXO 422-086 & 4183 & 04:25:51.271 & -08:33:36.42 & 10.0 & I3 & 0.0397 & 3.40 & \nodata\\
HCG 62 &  921 & 12:53:05.741 & -09:12:15.64 & 48.5 & S3 & 0.0146 & 1.10 & \nodata\\
HCG 42 & 3215 & 10:00:14.234 & -19:38:10.77 & 31.7 & S3 & 0.0133 & 0.70 & \nodata\\
Hercules A & 1625 & 16:51:08.161 & +04:59:32.44 & 14.8 & S3 & 0.1541 & 5.21 & \nodata\\
 & 5796 & \nodata & \nodata & 47.5 & S3 & \nodata & \nodata & \nodata\\
 & 6257 & \nodata & \nodata & 49.5 & S3 & \nodata & \nodata & \nodata\\
Hydra A & 4970 & 09:18:05.985 & -12:05:43.94 & 98.8 & S3 & 0.0549 & 4.00 &      d\\
 &  576 & \nodata & \nodata & 19.5 & S3 & \nodata & \nodata & \nodata\\
M49 &  321 & 12:29:46.841 & +08:00:01.98 & 39.6 & S3 & 0.0033 & 1.33 &      c\\
M87 & 5826 & 12:30:49.383 & +12:23:28.67 & 126.8 & I3 & 0.0044 & 2.50 &      d\\
 & 5827 & \nodata & \nodata & 156.2 & I3 & \nodata & \nodata & \nodata\\
MACS J0011.7-1523 & 3261 & 00:11:42.965 & -15:23:20.79 & 21.6 & I3 & 0.3600 & 5.42 & \nodata\\
 & 6105 & \nodata & \nodata & 37.3 & I3 & \nodata & \nodata & \nodata\\
MACS J0035.4-2015 & 3262 & 00:35:26.573 & -20:15:46.06 & 21.4 & I3 & 0.3644 & 7.39 & \nodata\\
MACS J0159.8-0849 & 3265 & 01:59:49.453 & -08:50:00.90 & 17.9 & I3 & 0.4050 & 9.59 & \nodata\\
 & 6106 & \nodata & \nodata & 35.3 & I3 & \nodata & \nodata & \nodata\\
MACS J0242.5-2132 & 3266 & 02:42:35.906 & -21:32:26.30 & 11.9 & I3 & 0.3140 & 5.58 & \nodata\\
MACS J0257.1-2325 & 1654 & 02:57:09.130 & -23:26:05.85 & 19.8 & I3 & 0.5053 & 10.50 & \nodata\\
 & 3581 & \nodata & \nodata & 18.5 & I3 & \nodata & \nodata & \nodata\\
MACS J0257.6-2209 & 3267 & 02:57:41.024 & -22:09:11.12 & 20.5 & I3 & 0.3224 & 8.02 & \nodata\\
MACS J0308.9+2645 & 3268 & 03:08:55.927 & +26:45:38.34 & 24.4 & I3 & 0.3240 & 10.54 & \nodata\\
MACS J0329.6-0211 & 3257 & 03:29:41.681 & -02:11:47.67 & 9.9 & I3 & 0.4500 & 5.20 & \nodata\\
 & 3582 & \nodata & \nodata & 19.9 & I3 & \nodata & \nodata & \nodata\\
 & 6108 & \nodata & \nodata & 39.6 & I3 & \nodata & \nodata & \nodata\\
MACS J0417.5-1154 & 3270 & 04:17:34.686 & -11:54:32.71 & 12.0 & I3 & 0.4400 & 11.07 & \nodata\\
MACS J0429.6-0253 & 3271 & 04:29:36.088 & -02:53:09.02 & 23.2 & I3 & 0.3990 & 5.66 & \nodata\\
MACS J0520.7-1328 & 3272 & 05:20:42.052 & -13:28:49.38 & 19.2 & I3 & 0.3398 & 6.27 & \nodata\\
MACS J0547.0-3904 & 3273 & 05:47:01.582 & -39:04:28.24 & 21.7 & I3 & 0.2100 & 3.58 &      e\\
MACS J0717.5+3745 & 1655 & 07:17:31.654 & +37:45:18.52 & 19.9 & I3 & 0.5480 & 10.50 & \nodata\\
 & 4200 & \nodata & \nodata & 59.2 & I3 & \nodata & \nodata & \nodata\\
MACS J0744.8+3927 & 3197 & 07:44:52.802 & +39:27:24.41 & 20.2 & I3 & 0.6860 & 11.29 & \nodata\\
 & 3585 & \nodata & \nodata & 19.9 & I3 & \nodata & \nodata & \nodata\\
 & 6111 & \nodata & \nodata & 49.5 & I3 & \nodata & \nodata & \nodata\\
MACS J1115.2+5320 & 3253 & 11:15:15.632 & +53:20:03.31 & 8.8 & I3 & 0.4390 & 8.03 & \nodata\\
 & 5008 & \nodata & \nodata & 18.0 & I3 & \nodata & \nodata & \nodata\\
 & 5350 & \nodata & \nodata & 6.9 & I3 & \nodata & \nodata & \nodata\\
MACS J1115.8+0129 & 3275 & 11:15:52.048 & +01:29:56.56 & 15.9 & I3 & 0.1200 & 6.78 & \nodata\\
MACS J1131.8-1955 & 3276 & 11:31:54.580 & -19:55:44.54 & 13.9 & I3 & 0.3070 & 8.64 & \nodata\\
MACS J1149.5+2223 & 1656 & 11:49:35.856 & +22:23:55.02 & 18.5 & I3 & 0.5440 & 8.40 & \nodata\\
 & 3589 & \nodata & \nodata & 20.0 & I3 & \nodata & \nodata & \nodata\\
MACS J1206.2-0847 & 3277 & 12:06:12.276 & -08:48:02.40 & 23.5 & I3 & 0.4400 & 10.21 & \nodata\\
MACS J1311.0-0310 & 3258 & 13:11:01.665 & -03:10:39.50 & 14.9 & I3 & 0.4940 & 5.60 & \nodata\\
 & 6110 & \nodata & \nodata & 63.2 & I3 & \nodata & \nodata & \nodata\\
MACS J1621.3+3810 & 3254 & 16:21:24.801 & +38:10:08.65 & 9.8 & I3 & 0.4610 & 7.53 & \nodata\\
 & 3594 & \nodata & \nodata & 19.7 & I3 & \nodata & \nodata & \nodata\\
 & 6109 & \nodata & \nodata & 37.5 & I3 & \nodata & \nodata & \nodata\\
 & 6172 & \nodata & \nodata & 29.8 & I3 & \nodata & \nodata & \nodata\\
MACS J1931.8-2634 & 3282 & 19:31:49.656 & -26:34:33.99 & 13.6 & I3 & 0.3520 & 6.97 &      e\\
MACS J2049.9-3217 & 3283 & 20:49:56.245 & -32:16:52.30 & 23.8 & I3 & 0.3254 & 6.98 & \nodata\\
MACS J2211.7-0349 & 3284 & 22:11:45.856 & -03:49:37.24 & 17.7 & I3 & 0.2700 & 11.30 & \nodata\\
MACS J2214.9-1359 & 3259 & 22:14:57.467 & -14:00:09.35 & 19.5 & I3 & 0.5026 & 8.80 & \nodata\\
 & 5011 & \nodata & \nodata & 18.5 & I3 & \nodata & \nodata & \nodata\\
MACS J2228+2036 & 3285 & 22:28:33.872 & +20:37:18.31 & 19.9 & I3 & 0.4120 & 7.86 & \nodata\\
MACS J2229.7-2755 & 3286 & 22:29:45.358 & -27:55:38.41 & 16.4 & I3 & 0.3240 & 5.01 & \nodata\\
MACS J2245.0+2637 & 3287 & 22:45:04.657 & +26:38:03.46 & 16.9 & I3 & 0.3040 & 6.06 & \nodata\\
MKW3S &  900 & 15:21:51.930 & +07:42:31.97 & 57.3 & I3 & 0.0450 & 2.18 & \nodata\\
MKW 4 & 3234 & 12:04:27.218 & +01:53:42.79 & 30.0 & S3 & 0.0198 & 2.06 & \nodata\\
MKW 8 & 4942 & 14:40:39.633 & +03:28:13.61 & 23.1 & I3 & 0.0270 & 3.29 &    a,b\\
MS J0016.9+1609 &  520 & 00:18:33.503 & +16:26:12.99 & 67.4 & I3 & 0.5410 & 8.94 & \nodata\\
MS J0116.3-0115 & 4963 & 01:18:53.944 & -01:00:07.54 & 39.3 & S3 & 0.0452 & 1.84 & \nodata\\
MS J0440.5+0204 & 4196 & 04:43:09.952 & +02:10:18.70 & 59.4 & S3 & 0.1900 & 5.46 & \nodata\\
MS J0451.6-0305 &  902 & 04:54:11.004 & -03:00:52.19 & 44.2 & S3 & 0.5386 & 8.90 & \nodata\\
MS J0735.6+7421 & 4197 & 07:41:44.245 & +74:14:38.23 & 45.5 & S3 & 0.2160 & 5.55 & \nodata\\
MS J0839.8+2938 & 2224 & 08:42:55.969 & +29:27:26.97 & 29.8 & S3 & 0.1940 & 4.68 & \nodata\\
MS J0906.5+1110 &  924 & 09:09:12.753 & +10:58:32.00 & 29.7 & I3 & 0.1630 & 5.38 & \nodata\\
MS J1006.0+1202 &  925 & 10:08:47.462 & +11:47:36.31 & 29.4 & I3 & 0.2210 & 5.61 & \nodata\\
MS J1008.1-1224 &  926 & 10:10:32.312 & -12:39:56.80 & 44.2 & I3 & 0.3010 & 7.45 & \nodata\\
MS J1455.0+2232 & 4192 & 14:57:15.088 & +22:20:32.49 & 91.9 & I3 & 0.2590 & 4.77 & \nodata\\
MS J2137.3-2353 & 4974 & 21:40:15.178 & -23:39:40.71 & 57.4 & S3 & 0.3130 & 6.01 & \nodata\\
MS J1157.3+5531 & 4964 & 11:59:52.295 & +55:32:05.61 & 75.1 & S3 & 0.0810 & 3.28 &      b\\
NGC 507 & 2882 & 01:23:39.905 & +33:15:21.73 & 43.6 & I3 & 0.0164 & 1.40 &      c\\
NGC 4636 & 3926 & 12:42:49.856 & +02:41:15.86 & 74.7 & I3 & 0.0031 & 0.66 &      c\\
 & 4415 & \nodata & \nodata & 74.4 & I3 & \nodata & \nodata & \nodata\\
NGC 5044 & 3225 & 13:15:23.947 & -16:23:07.62 & 83.1 & S3 & 0.0090 & 1.22 &      c\\
 & 3664 & \nodata & \nodata & 61.3 & S3 & \nodata & \nodata & \nodata\\
NGC 5813 & 5907 & 15:01:11.260 & +01:42:07.23 & 48.4 & S3 & 0.0066 & 0.76 &      c\\
NGC 5846 &  788 & 15:06:29.289 & +01:36:20.13 & 29.9 & S3 & 0.0057 & 0.64 &      c\\
Ophiuchus & 3200 & 17:12:27.731 & -23:22:06.74 & 50.5 & S3 & 0.0280 & 11.12 & \nodata\\
PKS 0745-191 & 2427 & 07:47:31.436 & -19:17:39.78 & 17.9 & S3 & 0.1028 & 8.50 & \nodata\\
 &  508 & \nodata & \nodata & 28.0 & S3 & \nodata & \nodata & \nodata\\
 & 6103 & \nodata & \nodata & 10.3 & I3 & \nodata & \nodata & \nodata\\
RBS 461 & 4182 & 03:41:17.490 & +15:23:54.66 & 23.4 & I3 & 0.0290 & 2.60 &    a,e\\
RBS 533 & 3186 & 04:19:38.105 & +02:24:35.54 & 10.0 & I3 & 0.0123 & 1.29 & \nodata\\
 & 3187 & \nodata & \nodata & 9.6 & I3 & \nodata & \nodata & \nodata\\
 & 5800 & \nodata & \nodata & 44.5 & S3 & \nodata & \nodata & \nodata\\
 & 5801 & \nodata & \nodata & 44.4 & S3 & \nodata & \nodata & \nodata\\
RBS 797 & 2202 & 09:47:12.693 & +76:23:13.40 & 11.7 & I3 & 0.3540 & 7.68 &      d\\
 & 7902 & \nodata & \nodata & 38.3 & S3 & \nodata & \nodata & \nodata\\
RCS J2327-0204 & 7355 & 23:27:27.524 & -02:04:39.01 & 24.7 & S3 & 0.2000 & 7.06 & \nodata\\
RX J0220.9-3829 & 9411 & 02:20:56.582 & -38:28:51.21 & 19.9 & I3 & 0.2287 & 5.02 & \nodata\\
RX J0232.2-4420 & 4993 & 02:32:18.771 & -44:20:46.68 & 23.4 & I3 & 0.2836 & 7.83 & \nodata\\
RX J0439+0520 &  527 & 04:39:02.218 & +05:20:43.11 & 9.6 & I3 & 0.2080 & 4.60 & \nodata\\
RX J0439.0+0715 & 1449 & 04:39:00.710 & +07:16:07.65 & 6.3 & I3 & 0.2300 & 6.50 & \nodata\\
 & 3583 & \nodata & \nodata & 19.2 & I3 & \nodata & \nodata & \nodata\\
RX J0528.9-3927 & 4994 & 05:28:53.039 & -39:28:15.53 & 22.5 & I3 & 0.2632 & 7.89 & \nodata\\
RX J0647.7+7015 & 3196 & 06:47:50.029 & +70:14:49.66 & 19.3 & I3 & 0.5840 & 9.07 & \nodata\\
 & 3584 & \nodata & \nodata & 20.0 & I3 & \nodata & \nodata & \nodata\\
RX J0819.6+6336 & 2199 & 08:19:26.007 & +63:37:26.53 & 14.9 & S3 & 0.1190 & 3.87 & \nodata\\
RX J1000.4+4409 & 9421 & 10:00:32.024 & +44:08:39.69 & 18.5 & I3 & 0.1540 & 3.42 & \nodata\\
RX J1022.1+3830 & 6942 & 10:22:10.034 & +38:31:23.54 & 41.5 & S3 & 0.0491 & 3.04 &      f\\
RX J1130.0+3637 & 6945 & 11:30:02.789 & +36:38:08.26 & 49.4 & S3 & 0.0600 & 2.00 & \nodata\\
RX J1320.2+3308 & 6941 & 13:20:14.650 & +33:08:33.06 & 38.6 & S3 & 0.0366 & 1.01 &      e\\
RX J1347.5-1145 & 3592 & 13:47:30.593 & -11:45:10.05 & 57.7 & I3 & 0.4510 & 10.88 & \nodata\\
 &  507 & \nodata & \nodata & 10.0 & S3 & \nodata & \nodata & \nodata\\
RX J1423.8+2404 & 1657 & 14:23:47.759 & +24:04:40.45 & 18.5 & I3 & 0.5450 & 5.92 & \nodata\\
 & 4195 & \nodata & \nodata & 115.6 & S3 & \nodata & \nodata & \nodata\\
RX J1504.1-0248 & 5793 & 15:04:07.415 & -02:48:15.70 & 39.2 & I3 & 0.2150 & 8.00 & \nodata\\
RX J1532.9+3021 & 1649 & 15:32:53.781 & +30:20:58.72 & 9.4 & I3 & 0.3450 & 5.44 & \nodata\\
 & 1665 & \nodata & \nodata & 10.0 & S3 & \nodata & \nodata & \nodata\\
RX J1539.5-8335 & 8266 & 15:39:32.485 & -83:35:23.83 & 8.0 & I3 & 0.0728 & 4.29 & \nodata\\
RX J1720.1+2638 & 4361 & 17:20:09.941 & +26:37:29.11 & 25.7 & I3 & 0.1640 & 6.37 & \nodata\\
RX J1720.2+3536 & 3280 & 17:20:16.953 & +35:36:23.63 & 20.8 & I3 & 0.3913 & 5.65 & \nodata\\
 & 6107 & \nodata & \nodata & 33.9 & I3 & \nodata & \nodata & \nodata\\
 & 7225 & \nodata & \nodata & 2.0 & I3 & \nodata & \nodata & \nodata\\
RX J1852.1+5711 & 5749 & 18:52:08.815 & +57:11:42.63 & 29.8 & I3 & 0.1094 & 3.66 & \nodata\\
RX J2129.6+0005 &  552 & 21:29:39.944 & +00:05:18.83 & 10.0 & I3 & 0.2350 & 5.91 & \nodata\\
RXCJ0331.1-2100 & 10790 & 03:31:06.020 & -21:00:32.93 & 10.0 & I3 & 0.1880 & 4.61 & \nodata\\
 & 9415 & \nodata & \nodata & 9.9 & I3 & \nodata & \nodata & \nodata\\
SC 1327-312 & 4165 & 13:29:47.748 & -31:36:23.54 & 18.4 & I3 & 0.0531 & 3.53 &      f\\
Sersic 159-03 & 1668 & 23:13:58.764 & -42:43:34.70 & 9.9 & S3 & 0.0580 & 2.65 & \nodata\\
SS2B153 & 3243 & 10:50:26.125 & -12:50:41.76 & 29.5 & S3 & 0.0186 & 0.80 & \nodata\\
UGC 3957 & 8265 & 07:40:58.335 & +55:25:38.30 & 7.9 & I3 & 0.0341 & 2.85 & \nodata\\
UGC 12491 & 7896 & 23:18:38.311 & +42:57:29.06 & 32.7 & S3 & 0.0174 & 0.87 & \nodata\\
ZWCL 1215 & 4184 & 12:17:41.708 & +03:39:15.81 & 12.1 & I3 & 0.0750 & 6.62 & \nodata\\
ZWCL 1358+6245 &  516 & 13:59:50.526 & +62:31:04.57 & 54.1 & S3 & 0.3280 & 10.66 & \nodata\\
ZWCL 1742 & 8267 & 17:44:14.515 & +32:59:29.68 & 8.0 & I3 & 0.0757 & 4.40 & \nodata\\
ZWCL 1953 & 1659 & 08:50:06.677 & +36:04:16.16 & 24.9 & I3 & 0.3800 & 7.37 & \nodata\\
ZWCL 3146 &  909 & 10:23:39.735 & +04:11:08.05 & 46.0 & I3 & 0.2900 & 7.48 & \nodata\\
ZWCL 7160 &  543 & 14:57:15.158 & +22:20:33.85 & 9.9 & I3 & 0.2578 & 4.53 & \nodata\\
Zwicky 2701 & 3195 & 09:52:49.183 & +51:53:05.27 & 26.9 & S3 & 0.2100 & 5.21 & \nodata\\
ZwCl 0857.9+2107 & 7897 & 09:00:36.835 & +20:53:40.36 & 9.0 & I3 & 0.2350 & 4.29 &      e
\end{rotthesistable}
\doublespacing

\clearpage
\input{ent_supp_betafits.tex}
\clearpage
\input{ent_supp_newha.tex}
\clearpage
\clearpage
\singlespacing
\begin{rotthesistable}{lcccccccc}
\thesistablehead{Statistics of Best-Fit Parameters}{Statistics of Best-Fit Parameters}{Sample & $N_{obj}$ & \kna & $K_{12}$ & \khun & $\alpha$ & \multicolumn{3}{c}{$N_{\kna=0}$}\\ &  & \ent & \ent & \ent &  & $1\sigma$ & $2\sigma$ & $3\sigma$\\(1) & (2) & (3) & (4) & (5) & (6) & (7) & (8) & (9)}{tab:bfparams}
\multicolumn{9}{c}{All \kna}\\
\hline
\accept\       & 233 & $72.9 \pm 33.7$ & $91.6 \pm 35.7$ & $126 \pm 45$   & $1.21 \pm 0.39$ & 4 (2\%) & 12 (5\%)  & 24 (11\%)\\
\hifl\         & 59  & $62.3 \pm 32.7$ & $87.2 \pm 34.5$ & $166 \pm 65$   & $1.18 \pm 0.38$ & 1 (2\%) & 3  (5\%)  & 4  (7\%) \\
CSE            & 37  & $61.9 \pm 27.4$ & $81.6 \pm 31.3$ & $132 \pm 45$   & $1.19 \pm 0.39$ & 1 (3\%) & 2  (5\%)  & 6  (16\%)\\
$\beta$ Models & 17  & $220  \pm 74$   & $230  \pm 76.9$ & $67.4\pm 27.0$ & $1.45 \pm 0.47$ & \nodata & \nodata   & \nodata  \\[0.25cm]
\hline
\multicolumn{9}{c}{$4 \ent < \kna \le 50 \ent$}\\
\hline
\accept\       & 99  & $17.5 \pm 5.8$  & $31.2 \pm 10.3$ & $148 \pm 49$   & $1.21 \pm 0.39$ & 3 (3\%) & 5  (5\%)  & 10 (10\%)\\
\hifl\         & 25  & $13.6 \pm 4.6$  & $29.4 \pm 9.63$ & $174 \pm 57$   & $1.15 \pm 0.37$ & 0 (0\%) & 0  (0\%)  & 0  (0\%) \\
CSE            & 17  & $16.4 \pm 5.4$  & $30.9 \pm 10.2$ & $146 \pm 48$   & $1.19 \pm 0.38$ & 0 (0\%) & 0  (0\%)  & 2  (12\%)\\[0.25cm]
\hline
\multicolumn{9}{c}{$\kna \le 50 \ent$}\\
\hline
\accept\       & 107 & $16.1 \pm 5.7$  & $30.5 \pm 10.0$ & $150 \pm 50$   & $1.20 \pm 0.38$ & 4 (4\%) & 6  (6\%)  & 11 (10\%)\\
\hifl\         & 29  & $11.4 \pm 4.2$  & $31.2 \pm 10.5$ & $235 \pm 89$   & $1.17 \pm 0.37$ & 1 (4\%) & 1  (4\%)  & 1  (4\%) \\
CSE            & 19  & $15.6 \pm 5.2$  & $30.9 \pm 10.2$ & $146 \pm 48$   & $1.16 \pm 0.38$ & 1 (5\%) & 1  (5\%)  & 3  (16\%)\\[0.25cm]
\hline
\multicolumn{9}{c}{$\kna > 50 \ent$}\\
\hline
\accept\       & 126 & $156  \pm 54$   & $175  \pm 59$   & $107 \pm 39$   & $1.23 \pm 0.40$ & 0 (0\%) & 6  (5\%)  & 13 (11\%)\\
\hifl\         & 30  & $151  \pm 53$   & $172  \pm 58$   & $113 \pm 43$   & $1.19 \pm 0.39$ & 0 (0\%) & 2  (7\%)  & 3  (10\%)\\
CSE            & 18  & $148  \pm 49$   & $165  \pm 54$   & $118 \pm 42$   & $1.23 \pm 0.40$ & 0 (0\%) & 1  (6\%)  & 3  (17\%)
\end{rotthesistable}
\doublespacing

\clearpage
\input{ent_supp_kfit.tex}


%%%%%%%%%%%%%%%%%%%%%%%%%%%%%%%%%%%%%%%%%%%%%%%%%%%%%%%%
\chapter{Chandra Observations Reduction Pipeline (CORP)}
\label{ch:corp}
%%%%%%%%%%%%%%%%%%%%%%%%%%%%%%%%%%%%%%%%%%%%%%%%%%%%%%%%

This appendix has been written as a tutorial for the first-time
analyzer of \chandra\ data (\eg\ the ``you'' role in the text). ``When
your \ciao-Fu is good, only then will you utilize the stowed
backgrounds of the \caldb.''

%%%%%%%%%%%%%%%%%%%
\section{Copyright}
%%%%%%%%%%%%%%%%%%%

As a formality, I have blanketed CORP with the GNU General Public
License. Below is the copyright and license agreement for all scripts
in CORP:\\
Kenneth W. Cavagnolo's Chandra Observations Reduction Pipeline (CORP)\\
Copyright \copyright\ 2008 Kenneth W. Cavagnolo, {\tt{kencavagnolo@gmail.com}}\\
These programs are free software; you can redistribute them and/or
modify them under the terms of the GNU General Public License as
published by the Free Software Foundation; either version 2 of the
License, or (at your option) any later version. These programs are
distributed in the hope that they will be useful, but WITHOUT ANY
WARRANTY; without even the implied warranty of MERCHANTABILITY or
FITNESS FOR A PARTICULAR PURPOSE. See the GNU General Public License
for more details. You should have received a copy of the GNU General
Public License along with these programs; if not, write to the Free
Software Foundation, Inc., 51 Franklin Street, Fifth Floor, Boston, MA
02110-1301, USA.

%%%%%%%%%%%%%%%%%%%%%%%%%%%%%%
\section{Introduction to CORP}
%%%%%%%%%%%%%%%%%%%%%%%%%%%%%%

The reduction and analysis of \chandra\ data is given in exquisite
detail in the \ciao\ threads on the CXC's
web site\footnote{http://cxc.harvard.edu/ciao/threads/index.html}.
There is very little which is not discussed in the CIAO and HelpDesk
threads at the CXC web site. However, to streamline the lengthy
reduction and analysis process of extended X-ray sources, such as
galaxy clusters and groups, I have written several \perl\ and
\idl\ scripts which make-up my own \chandra\ Observations Reduction
Pipeline (CORP, pronounced ``core''). The purpose of this pipeline
software is to condense \ciao's tedious prompts and command line
intensive steps into an easily executable series of scripts which
require minimal interaction and produce science-ready data products. A
pipeline also ensures that a large sample of observations are reduced
the same way, and a pipeline also eases the pain of analyzing several
hundred observations.

There is a critical caveat to the use of CORP and analyzing
\chandra\ data in general: {\bf{no two \chandra\ observations are the
same!}} CORP has allowances for many different tool settings and
instrument setups, but these options are finite, and no amount of
automation can replace human interactivity. It is an absolute
{\bf{**necessity**}} that users of CORP view, scrutinize, and double
check the output of every reduction/analysis step. This can be time
consuming, but not nearly as time consuming as finding and correcting
errors embedded in on-going, or goodness forbid, {\bf{published}}
work.

As of writing this dissertation, all CORP scripts properly interact
with \ciao\ 3.4.1 and \caldb\ 3.4. The CXC software versions matter
because the CXC programmers (whom are great people!) have a tendency
to reinvent the wheel every so often. This may result in a change to
output filenames, extensions, header keywords, data types, et cetera
and can cause a script to err. Errors are not guaranteed however, so
it is important to be mindful of how \ciao\ or the \caldb\ have
changed as a result of an update (release notes are always provided
with an update: read them!). There are some pretty great updates to
\ciao\ included in the final 4.1 version, so I highly recommend
CORP users email me to request new scripts when \ciao\ 4.1 is fully
deployed.

Now for a few notes about me, the author and programmer:
\begin{enumerate}
\item I use plenty of analysis and X-ray ``jargon'' in this
  Appendix. If you come across nomenclature which is unfamiliar,
  consult the CXC web site, there is absolutely nothing you could want
  to know about \chandra\ that is not there. CORP is my method for
  automating most of the logic trees which are in the CXC threads, but
  this does not mean CORP's operation will always be transparent to a
  user.

\item I am not a computer programmer. In fact, I only knew pseudo-code
  when I entered graduate school. Hence, my style of programming is
  best described as inelegant and brutish. Computer resources are
  cheap and abundant, so I use lots of inefficient code, but the
  overhead and time consumption are low, so writing efficient code
  does nothing to accelerate my work. The scripts of CORP use no
  command line options besides input, and sometimes output,
  filenames. Everything the script is being commanded to do is
  controlled by opening the code and editing the ``Options'' section
  near the beginning of the program. Do not worry, it's as simple as
  editing a Word document.

\item As of now, the scripts are available via a tarball which I will
  email you. Someday distribution of the code will be handled through
  a public CVS server. The tarball contains all the scripts, a README
  file (which is a copy of this appendix), and a folder of example
  data. Descriptions of what each script does, how it is called, what
  it takes as input, and what is generated as output are all listed in
  the header of each program.

\item I am not a debugger. Each script is written to run a very
  specific set of tasks. Given the proper input, the scripts return
  the expected output. However, while I wish the scripts were magic,
  alas, they are not. If you plan on giving the programs non-standard
  input and there is some operation you are not sure the script will
  perform, then find out first by dissecting the code. I may have
  coded a script to handle your odd data, but I may not. Find out
  first!
\end{enumerate}

%%%%%%%%%%%%%%%%%%%%%%%%%%%%%%
\section{Initial Reprocessing}
%%%%%%%%%%%%%%%%%%%%%%%%%%%%%%

%%%%%%%%%%%%%%%%%%%%%%%%%%%%
\subsection{Retrieving Data}
%%%%%%%%%%%%%%%%%%%%%%%%%%%%

Getting \chandra\ data from the CDA is not complicated. There are
three methods to get data: (1) run the stand-alone \chaser\ program,
(2) use the web-based version of \chaser, named
\webchaser\footnote{http://cda.harvard.edu/chaser/}, or (3) run my
script {\tt{query\_cda.pl}}. The first thing to do is determine which
ObsIDs need to be downloaded. This is accomplished by searching the
CDA via \webchaser. The \webchaser\ form is self-explanatory: search
via object name, sky coordinates, or using any set of other listed
methods and options. After searching the CDA, a list of archived
observations will be returned. {\bf{VERY IMPORTANT:}} Now is the time
to make a one column file where each line lists one of the ObsIDs to
be downloaded:
\begin{verbatim}
#Obsid
2419
791
etc.
\end{verbatim}
This file is needed to download data and build the reference file used
by every script in CORP.

Open the script {\tt{query\_cda.pl}} with an editor (like emacs or
xemacs). Within the script are three vital options that need to be
set: {\tt{\$get\_nh, \$get\_z}}, and {\tt{\$get\_data}}. When set to
'yes', these options tell the script to: (1) find the Galactic
absorbing column density (\nhi) using the LAB survey \citep{lab}, (2)
acquire a redshift from NED\footnote{http://nedwww.ipac.caltech.edu/},
and (3) download data from the CDA. The {\tt{\$get\_nh}} is
trustworthy, so set it to 'yes'. However, the {\tt{\$get\_z}} option
is not robust. The NED query returns a list of galaxy clusters nearest
the aimpoint of the \chandra\ observation, but the proper redshift is
not always returned. I typically leave this option set to 'yes' and
confirm redshifts by manually querying NED (hey, it's less
typing).

The {\tt{\$get\_data}} option will cause the script to download data and
create a directory for each ObsID and place both into the directory
specified by the variable {\tt{\$dest}}. The CDA is large, so a
pre-compiled listing of where every ObsID is stored is provided in the
file {\tt{cdaftp.dat}} (this file is part of the CORP tarball). The
variable {\tt{\$ftpdat}} needs to point to where this file is
stored. The CDA is continually updated, so you will need to update
{\tt{cdaftp.dat}} from time to time. This is accomplished by running:
\begin{verbatim}
[linux]% perl build_cdaftp.pl <output_filename>
\end{verbatim}

Now, run the query script. {\bf{WARNING:}} If there is an existing
{\tt{newref.list}} in the working directory, it will be overwritten.
\begin{verbatim}
[linux]% perl query_cda.pl <my_list_of_obsids>
\end{verbatim}
The download time is completely dependent on download speed, so if
there are many ObsIDs to download, work on something else for a
while. The script tells the user what \nhi\ and redshift it has found
and for what object. The script also informs the user how many of the
input ObsIDs were successfully found in the CDA and the total exposure
time of observations in the query. You will now have new directories
which bear the names of the downloaded ObsIDs (for example,
{\tt{<hard\_drive>/<my\_root\_datadir>/<obsid>}}). There should also
be a new file named {\tt{newref.list}}. Each column within
{\tt{newref.list}} is described below:

{\bf{Name}}: This is the name of the {\tt{TARGET}} object listed for
the observation. This is not necessarily the name you'd like to give
the object, so feel free to change it.

{\bf{ObsID}}: Obviously this is the ObsID. Most of the file naming
convention in CORP involves the ObsID since it is a unique
identifier. This may seem clumsy and awkward (especially for the
clusters that have multiple ObsIDs) but in the battle of clarity and
brevity, clarity wins in my book.

{\bf{RA}}: The right ascension of the observation target object. The
default output format is decimal degrees, but this can be changed to
sexigesimal by changing the {\tt{query\_cda.pl}} variable
{\tt{\$outcrdunit}} from {\tt{'decimal'}} to {\tt{'sexigesimal'}}.

{\bf{Dec}}: The declination of the observation target object in
decimal degrees.

{\bf{Rmax}}: The maximum observation radius. This is the radius from
the cluster center to the nearest detector edge. Rmax needs to be
specified by the user for each observation. A script which
automatically finds Rmax is forthcoming.

{\bf{MinCts}}: The minimum number of counts per temperature
annulus. The default is 2500 but this number should be increased for
observations with sufficient counts or adjusted depending on
scientific goals.
 
{\bf{z}}: The redshift of the target object. Even if the script has
automatically queried NED for this value, it is best to
double-check. A batch query via NED is
simple\footnote{http://nedwww.ipac.caltech.edu/help/batch.html}.

{\bf{Nh20}}: The galactic absorbing neutral hydrogen column density,
\nhi, in $10^{20} \pcmsq$. These values are acquired from the
{\tt{nh}} tool which uses the L.A.B. Survey results \citep{lab}.

{\bf{Tx}}: The global/virial cluster temperature. There are a number
of ways to measure to a cluster temperature, it is best to keep a
detailed record of how you made this measurement or where you
looked-up the temperature (\eg\ KEEP CITATIONS you'll want them
later). If no temperature can be found in the literature, the script
{\tt{\$find\_tx.pl}} should be used to determine the cluster
temperature. This script iteratively determines temperature in
core-excised annuli using a user-specified fraction of the virial
radius as the outer radius. When to run the script after removing is
specified later.

{\bf{Fe}}: The global cluster metallicity. The value listed in the
reference file is used only as a starting point for most spectral
fits, so the value listed is not all that important. I'd go as far to
say this is a deprecated entry. Again, if this value if looked-up in
the literature, keep a citation record.

{\bf{Lbol}}: The cluster bolometric luminosity. \chandra\ has a small
field of view, so if this is a value best taken from the literature,
for example from \citet{hornerthesis}.

{\bf{Chip}}: The CCD on which the cluster center is located. The
default is to list the array on which the observation is taken (I or
S), but specifying which {\tt{chip\_id}} (\ie\ S3 or I0) is left to the
user.

{\bf{E\_obs, Diff, Robs}}: These are deprecated columns which have
been left-in because most scripts were written before they were no
longer needed. Forthcoming versions of CORP will not need these
columns.

{\bf{Location}}: Sometimes data is stored across multiple volumes,
hence this column specifies the path to each ObsID. For example if one
ObsID is stored on {\tt{/mnt/HD1}} and another is on
{\tt{/media/USB1}}. This allows a single instance of a script to be
run on distributed data.

Great! The data is now out of the archive and into your hands. What
are all these different files?  The answer to that question is lengthy
and of fundamental importance in honing your \ciao-Fu. Read the
documentation on data
products\footnote{http://cxc.harvard.edu/ciao/data/basics.html}$^{,}$\footnote{http://cxc.harvard.edu/ciao/threads/intro\_data/}.

Included in the CORP tarball is a script named
{\tt{ds9\_viewreg.pl}}. The script is very useful for viewing
observations quickly and has been invaluable throughout the years. The
program performs a multitude of tasks and relies on the functionality
of DS9\footnote{http://hea-www.harvard.edu/RD/ds9/}. The script header
completely explains all the variables and how to use the
program. Before starting the bulk of data reduction I recommend using
{\tt{ds9\_viewreg.pl}} to determine the {\tt{Chip}} and {\tt{Rmax}}
values for each ObsID and then entering them into the
{\tt{newref.list}} file.

%%%%%%%%%%%%%%%%%%%%%%%%%%%%%%%%%%%%%%%%%%%
\subsection{Create New Level-2 Events File}
%%%%%%%%%%%%%%%%%%%%%%%%%%%%%%%%%%%%%%%%%%%

Before completing any true analysis, such as finding the cluster
center or identifying point sources, the \ciao\ threads recommend
creating a new events file. Removing bad grades, time intervals with
flares, bad pixels, et cetera ensures that analysis further downstream
is more robust. The initial reprocessing step described here requires
running the script {\tt{reprocess.pl}} with some, but not all, of the
internal switches set to ``yes''. Descriptions for the internal
switches of this program are in the code header. The program
{\tt{reprocess.pl}} performs multiple tasks and will be used more than
once: In the first pass, {\tt{reprocess.pl}} will perform the tasks
outlined below, later it will be used to exclude point sources. For
more detail on any of the steps below, read the \ciao\
documentation\footnote{http://cxc.harvard.edu/ciao/guides/acis\_data.html}$^{,}$\footnote{http://cxc.harvard.edu/ciao/threads/createL2/}.

First, edit the script so all the options are ``yes'' and only
{\tt{\$exclude}} is equal to ``no''. To run the script, simply call it
with \perl\ giving the reference file as input (if you do not want an
ObsID analyzed, simply comment it out of the reference file by placing
a ``\#'' at the front of the line):
\begin{verbatim}
[linux]% perl reprocess.pl reference.list
\end{verbatim}
There is no specific version of \perl\ required to run CORP. For each
ObsID in the reference file, a new {\tt{reprocessed}} subdirectory has
been created and all new files have been placed in that directory. The
output files are listed at the end of the script header. The reduction
steps performed by {\tt{reprocess.pl}} are: remove the ACIS afterglow,
create a new bad pixel file, set {\tt{ardlib.par}}, update
time-dependent gain (TGAIN), apply charge transfer inefficiency (CTI)
correction (if appropriate), filter on event grade, filter on good
time intervals (GTIs), destreak, remove background flares and/or
periods of excessively high background, make blank-sky background
file, and make off-axis blank-sky background file.

Cleaning for flares is a detailed step and is best understood by
reading the \ciao\ documentation. The main steps in extracting and
filtering a light curve are removing bright sources, setting the time
bin size specific to front or back illuminated CCDs, setting the
energy window for the specific CCD type, analyzing the light curve
using the contributed routine {\tt{lc\_clean.sl}}, and filtering out
the time periods from the GTIs which contain high background. All of
these steps are handled automatically by the script, however
{\tt{lc\_clean.sl}} {\bf{DOES NOT}} always find the proper background
count rate mean. This results in the beginning or end of flare not
being excluded from the GTIs, while perfectly good intervals are
excluded (see Figure \ref{fig:flare} as an example). The solution to
this problem is very simple.

After reprocessing you should examine each light curve anyway, so
finding these cases will be easy. Run 
\begin{verbatim}
[linux]% perl view_prof.pl reference.list
\end{verbatim}
with the internal switches set to view lightcurves. If the lightcurve
cleaning failed then there will be no lightcurve to view, in which
case run the following script for only those clusters which
experienced a failure (logged in {\tt{errors.log}}) by commenting out
all the clusters in {\tt{reference.list}} (placing a {\tt{\#}} at the
beginning of the line in {\tt{reference.list}}) which did not fail,
then running:
\begin{verbatim}
[linux]% idl
IDL> load_ltcrv, 'reference.list'
\end{verbatim}

When you find a case where a flare has been improperly removed,
estimate the mean background count rate by finding the peak in the
count rate distribution in the bottom pane of the figure. Now we'll
create a new ASCII file which contains information about this OBSID
and it's flare. The file should have the format:
\begin{verbatim}
#Obsid  Mean Rate
2419    1.300
791     0.175
\end{verbatim}
Save this file, set the {\tt{\$flarefile}} keyword in
{\tt{reprocess.pl}} to point to this new file, and now re-run
{\tt{reprocess.pl}} with all other options set to ``no'' except the
{\tt{\$clean\_events}} option which should be ``yes''. Examine the new
lightcurve and repeat this process if the mean is not exactly where
you think it should be.

\begin{figure}[htp]
\begin{center}
\includegraphics*[width=\textwidth, trim=0mm 0mm 0mm 0mm, clip]{flare.ps}
\caption[Example of strong X-ray flare in \chandra\ data]{Shown here
is an example light curve output by {\tt{lc\_clean.sl}} which is
called when {\tt{reprocess.pl}} is run. {\it{Top panel:}} Plot of
background count rate versus time in the observation. Green points
mark those time intervals within $\pm 20\%$ of the mean rate. Zero
rate bins at the beginning and end of an observation are intentional
dead times. The flares can easily be identified in this example. Also
note that the automated mean detection did not remove the wings at
$\sim 5$ ksec and $\sim9$ ksec of the weaker flare. {\it{Bottom
panel:}} Histogram of the light curve shown in the top panel. The
green line is again for the intervals within $\pm 20\%$ of the mean
rate.}
\label{fig:flare}
\end{center}
\end{figure}

Making blank-sky backgrounds is easily the most involved step in
{\tt{reprocess.pl}} and to fully understand what is being done,
reading Maxim Markevitch's
Cookbook\footnote{http://cxc.harvard.edu/contrib/maxim/acisbg/} in
conjunction with the background
thread\footnote{http://cxc.harvard.edu/ciao/threads/acisbackground/}
are a must.

After {\tt{reprocess.pl}} has finished running, it is wise to spend
the time examining all of the output files. This entails steps such as
viewing the {\tt{evt1, evt2, bgevt,}} and {\tt{clean}} files; checking
the light curves for missed or improperly excluded flares; verifying
the background file have the proper exposure times in the headers. If
the data is trustworthy, then move along.

%%%%%%%%%%%%%%%%%%%%%%%%%%%%%%%%%%%%%%%%%%%%%%%%%%%%%%%%%%%%%
\subsection{Remove Point Sources and Identify Cluster Center}
%%%%%%%%%%%%%%%%%%%%%%%%%%%%%%%%%%%%%%%%%%%%%%%%%%%%%%%%%%%%%

Determining the cluster center and identifying points sources for
exclusion is a crucial step in extended source analysis as these steps
significantly affect the final results. Reliably determine the cluster
center and finding point sources first requires an exposure map. An
exposure map is a replication of the optical system's characteristics
(\eg\ CCD quantum efficiency, CCD non-uniformities, vignetting, bad
pixels, etc.) dithered and exposed exactly as the observation. The
exposure map is used to remove instrumental features, like chip gaps,
which will be erroneously detected as point sources or skew the
calculation of the cluster center.

There are two ways of calculating an exposure map: using a
monoenergetic incident spectrum or a more specific spectral model. The
script for making exposure maps can do both, and it is up to the user
to determine if one or the other is preferred for their analysis. For
all of the clusters I have analyzed, the monoenergetic assumption does
not give significantly different results from the more elegant
spectral model method.

There are many options for making an exposure map and they are
explained in the script header. To make an exposure map, run the
script {\tt{exp\_map.pl}}:
\begin{verbatim}
[linux]% perl exp_map.pl reference.list
\end{verbatim}
The output will be an instrument map, aspect histogram, exposure map,
and a normalized image of the cluster (in flux units). The normalized
image is what will be used to find point sources and find the cluster
center. By default, the script makes a full resolution (binning=1)
exposure map. This has the drawback of being time consuming, but the
advantage of more accuracy in spatial analysis. If you are simply
experimenting data, then increase the binning factor to four or
eight. I do not recommend using such highly binned data for analysis
unless the cluster center is especially obvious (\ie\ in highly peaked
clusters) or you plan on remaking point source exclusion regions by
hand.

Now that you have an exposure map, edit the options in the script
{\tt{cent\_emi.pl}} and run the script.
\begin{verbatim}
[linux]% perl cent_emi.pl reference.list
or
[linux]% perl dub_centroid.pl dub_reference.list
\end{verbatim}
This program will find the cluster centroid and peak using the \ciao\
tool {\tt{dmstat}}. If these two quantities differ by less than 70
kpc, the peak will be returned as the cluster center. The script also
outputs a new reference file with the cluster center in the RA and Dec
columns. Now is the moment to be a researcher: view the clusters with
the center marked, does this solution look right?

After determining if the cluster center finder worked properly, the
next step is to identify and exclude point sources. The script which
does this is {\tt{find\_pt\_src.pl}} which calls the \ciao\ tool
{\tt{wavdetect}}. The primary script output is an ASCII file listing
the exclusion regions.
\begin{verbatim}
[linux]% perl find_pt_src.pl reference.list
\end{verbatim}
It is very important at this stage to view the observation with the
exclusion regions overlaid. The {\tt{wavdetect}} algorithm is very
sophisticated, however it will miss sources (\eg\ sources very close
together), detect spurious sources (\eg\ chip gaps and edges), detect
sources which are bright, diffuse cluster emission (\eg\ the core of a
bright, peaky cluster), and miss sources in regions of high background
(\eg\ point sources in or near bright cluster core).

While viewing the observation and regions with {\tt{ds9\_viewreg.pl}},
it is straight forward to delete, add, and alter regions. After doing
so, go to the 'Regions' menu, 'File Format' tab, and click
'Ciao'. Then under the 'Regions' menu click 'Save regions...' and
simply overwrite the loaded {\tt{<obsid>\_exclude.reg}} region
file. That is all it takes to edit the exclusion regions in a quick,
by-eye batch session. Now edit the script {\tt{reprocess.pl}} so that
all options are ``no'' except for the {\tt{\$exclude}} option which
should be ``yes''. Running {\tt{reprocess.pl}} will now remove all the
regions you just saw/specified in DS9.
\begin{verbatim}
[linux]% perl reprocess.pl reference.list
\end{verbatim}
The output from this final step is the file
{\tt{<obsid>\_exclude.fits}} which is the crown jewel of CORP: an
up-to-date, flare-clean, point source-clean, events file at
level-2. As usual, you should view this file and make sure the point
source exclusion functioned as expected. The initial reprocessing
steps are now complete, congrats.

%%%%%%%%%%%%%%%%%%%%%%%%%%%%%%%
\section{Intermediate Analysis}
%%%%%%%%%%%%%%%%%%%%%%%%%%%%%%%

If at this point you still do not have a temperature for some number
of clusters, run the following script to find one:
\begin{verbatim}
[linux]% perl find_tx.pl reference.list
\end{verbatim}
This script can also be used to determine redshifts and
metallicities. Read the script header for more detailed instructions
on use.

The intermediate analysis steps involve extracting radial profiles and
spectra from the observations. These radial profiles form the basis
for the final analysis steps of the next section. One script,
{\tt{make\_profile.pl}}, extracts both a cumulative counts profile and
a surface brightness profile. In the options section of the script you
specify the size of the annular bins used to extract the profiles. For
the cumulative counts profile I recommend 2 pixel width bins, and
either 5 pixel or 10 pixel width bins for the surface brightness
profiles. For the surface brightness profiles you also need to specify
the energy range for the extraction. There are many other options for
this script which are detailed in the program header.
\begin{verbatim}
[linux]% perl make_profile.pl reference.list
or
[linux]% perl make_multiprof.pl dub_reference.list
\end{verbatim}
Depending on the number of counts in the observation, this step can
take a few minutes or a few hours. The script also outputs plots of
the two profiles which should be viewed to ensure everything ran
correctly.

Now run the script {\tt{exp\_corr.pl}} which extracts a radial profile
from the exposure map and will be used for exposure correcting radial
profiles later on. As usual, set the options in the header,
specifically the bin size of the profile to extract. The bin size
needs to match that of the surface brightness profile just
extracted. It is possible to extract multiple exposure profiles since
the bin size is amended to the output file name. This step is fast, so
extracting profile for bin sizes of 5, 10, or 20 pixels does not take
long.
\begin{verbatim}
[linux]% perl exp_corr.pl reference.list
or
[linux]% perl multiexp_corr.pl dub_reference.list
\end{verbatim}

With the cumulative profile, it is now possible to create annuli for
the temperature profile. The script {\tt{make\_annuli\_reg.pro}} is used
to make the annuli. The cumulative profile is divided up into annular
bins containing a minimum number of counts and then these bins are
output as region files later used to extract spectra. The entries in
the 'Mincts' column of the reference file are used to set the minimum
number of counts. I typically run the script with all options set to
``no'' (meaning only mock regions are produced) and view the output
plot to ensure the number and spacing of the bins is appropriate for
the cluster in question. What is appropriate? Well, too many closely
spaced annuli for a symmetric peaked cluster is redundant, and too few
widely spaced annuli for a complex cluster is insufficient, unless of
course there are not enough counts to produce more bins. After
ensuring the number of annuli produced is agreeable, run the script
with the options set to ``yes''. {\bf{WARNING:}} by default, the
script deletes all annuli and associated spectral files. If you run
this script after having extracted spectra, be sure to set
{\tt{mkbackup}} to ``yes'' AND provide the path to a valid, existing
back-up directory, {\tt{bkdir}}, this will save those existing
spectra.
\begin{verbatim}
[linux]% idl
IDL> make_annuli_reg, 'reference.list'
or
IDL> make_multiannuli_reg, 'reference.list'
\end{verbatim}
A whole ensemble of individual region files with the specified name
will be output by this script. With these region files extracting
spectra is straightforward, simply run the script
{\tt{extract\_spectra.pl}}. This script has a number of important
options which are detailed in the script header.
\begin{verbatim}
[linux]% perl extract_spectra.pl reference.list
\end{verbatim}

If there have been no errors, then you should have radial profiles and
spectra for each ObsID in the reference file, congratulations.

%%%%%%%%%%%%%%%%%%%%%%%%
\section{Final Analysis}
%%%%%%%%%%%%%%%%%%%%%%%%

The final steps in the analysis process are normalizing the background
spectra for differences between the blank-sky background and
observation background hard-particle count rates, extracting and
fitting residual spectra for the local soft background, fitting the
cluster spectra, and running the master IDL routine that produces
entropy profiles. An explanation of why and how background adjustments
are made is presented in Chapter \ref{ch:eband}.

%%%%%%%%%%%%%%%%%%%%%%%%%%%%%%%%%%%%%%%%%%%%%
\subsection{Spectral Adjustments and Fitting}
%%%%%%%%%%%%%%%%%%%%%%%%%%%%%%%%%%%%%%%%%%%%%

The hard-particle background is changing as a function of time. Thus,
the strength of this background component for the epoch in which the
blank-sky backgrounds were taken will be different from when the
observations were taken. The first step in accounting for this
difference involves running the script {\tt{bgd\_ratio.pl}} which will
output the ratio of observation to blank-sky 9.5-12.0 keV count rates.
\begin{verbatim}
[linux]% perl bgd_ratio.pl reference.list
\end{verbatim}
The script outputs a file containing these ratios. Keep this file
someplace which is easily accessible as a later script will be
querying this list to normalize the spectra.

To account for the spatially varying Galactic soft-component you must
extract soft-residuals. Soft-residuals are the leftovers of
subtracting a blank-sky spectrum from an observation spectrum both
extracted from the same part of the sky and far from the cluster
emission. The first step is to clean-up the off-axis observation chips
of all point and diffuse sources. This step can be performed blind
because if too many sources are removed it does not matter.
\begin{verbatim}
[linux]% perl addbg_rm_pt_src.pl reference.list
\end{verbatim}
The soft-residual spectra are output from this script.

Prior to fitting any spectra all background spectra need to be
normalized. This is done quickly by specifying the names of the
spectra to be normalized in the script {\tt{adj\_backscal.pl}} and then
running it.
\begin{verbatim}
[linux]% perl adj_backscal.pl reference.list
\end{verbatim}
Normalization is applied by adjusting the header keyword {\tt
BACKSCAL} in the spectrum. The {\tt BACKSCAL} keyword is related to
the final background subtracted spectrum of an observation by
\begin{equation}
SPEC_{ctr} = \frac{SRC_{cts}}{SRC_{exp}} -
\frac{BGD_{cts}}{BGD_{exp}} \cdot \frac{1}{BGD_{scal} \cdot
SRC_{scal}}
\end{equation}
where the abbreviations `SRC' $\rightarrow$ source, `BGD'
$\rightarrow$ background, `ctr' $\rightarrow$ count rate, `cts'
$\rightarrow$ counts, `exp' $\rightarrow$ exposure time, and `scal'
$\rightarrow$ {\tt BACKSCAL}. The {\tt BACKSCAL} keyword is defined as
the detector area over which the spectrum was extracted, divided by
the total detector area. Adjustment of the blank-sky background {\tt
BACKSCAL} value follows directly from the above equation by
multiplying the existing value of {\tt BACKSCAL} by a correction
factor, $\eta$, which is related to the ratio of the count rate in the
9.5-12.0 keV range of the observation to the blank-sky background by
\begin{equation}
\eta = (\frac{OBS_{ctr}}{BGD_{ctr}})^{-1}.
\end{equation}
Each background spectrum is copied into a new file before this
correction is applied so that reversal at a later date is possible.

Now that the spectra are all adjusted, the fitting can begin. There
are a large number of options in spectral fitting and these are
detailed in the individual script's header. In addition, interpreting
the results of the fitting requires more discussion than is useful
here. The order in which fitting is done is important since the master
spectral fitting routines need output from the fitting of the
soft-residuals.
\begin{verbatim}
[linux]% perl fit_sofex.pl reference.list <spectral model>
[linux]% perl fit_projected.pl reference.list <spectral model>
or
[linux]% perl fit_simulta.pl dub_reference.list <spectral model>
\end{verbatim}
The output of these scripts are data tables with the spectral fits for
each annulus associated with each ObsID. With radial profiles in-hand
and spectral analysis complete, it is now possible to calculate many
more physical properties of a cluster.

%%%%%%%%%%%%%%%%%%%%%%%%%%%%%%%%%%%%%%%%
\subsection{Generating Entropy Profiles}
%%%%%%%%%%%%%%%%%%%%%%%%%%%%%%%%%%%%%%%%

So here it is, the end of a long journey. Your \ciao-Fu is good,
congratulations. The master program which performs the deprojection,
derives electron gas density, pressure, entropy, mass (not to be
trusted), and cooling time is {\tt{kfitter.pro}}. This program has a
handler program {\tt{run\_kfit.pro}} which makes batch analysis
simpler. These programs have their own {\tt{README}} files which have
not been duplicated here. After running {\tt{kfitter.pro}} you will
have a suite of data tables and plots for each cluster which are
publication ready.

%%%%%%%%%%%%%%%%%%%%%%%%%%%%
\subsection{Additional Code}
%%%%%%%%%%%%%%%%%%%%%%%%%%%%

I have coded many other tools which are extremely useful in the
analysis of galaxy clusters. If you would like to use any of the
following tools, simply email me or check the public CVS repository.
\begin{enumerate}
\item Batch query NVSS, SUMSS, or VLA First for radio sources within a
region of interest.
\item Calculate the X-ray and Sunyaev-Zel'dovich signatures for a mock
cluster based on user input parameters.
\item Generate an image and calculate properties of the Chandra PSF
for any location or energy on the ACIS-S or ACIS-I arrays.
\item Generate 2D maps of cluster properties (temperature, density,
entropy, abundance, pressure, hardness ratio, \etc) and return nice
profiles and \LaTeX\ tables for each.
\item Create \mysql\ and/or FITS databases from ASCII tables of
cluster data.
\item Calculate gas mass and gravitating mass profiles for a cluster
with known temperature and density.
\item Many other tools for simulating data, extracting spectra, making
web pages, and on and on.
\end{enumerate}

``Pasta be upon you.''\\
-FSM

%%%%%%%%%%%%%%%%%%
%% BIBLIOGRAPHY %%
%%%%%%%%%%%%%%%%%%

\bibliography{cavagnolo}

%%%%%%%%%%%%%%
\end{document}
%%%%%%%%%%%%%%
