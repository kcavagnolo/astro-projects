%% LaTeX template for the science justification & technical
%% feasibility to be submitted as part of a Chandra X-ray Observatory
%% proposal.
%%
%% Chandra cycle 10


%%%%%%%%%%%%%%%%%%%%%%%%%%%
%%%%% DOCUMENT FORMAT %%%%%
%%%%%%%%%%%%%%%%%%%%%%%%%%%

%% The default font was chosen to be easily readable while allowing
%% sufficient material to be included.

%% The two-column, 11pt format fits the largest number of characters
%% per page while still being within allowed limits.

%% There are three documentclass commands provided below. Please
%% uncomment the version you would like to use and comment out the
%% others.

%% Please note that the proposal will be printed on US Letter size paper,
%% 8.5 in x 11 in, and that formatting the text for other sizes will
%% generally cause layout problems and may result in text being cut
%% off near the edges. PLEASE DO NOT CHANGE THE 'LETTERPAPER' OPTION
%% IN THE DOCUMENTCLASS COMMAND.

%%%%%%%%%%%%%%%%%%%%%%%%%%%%%%%%%%%%%%%%%%%%%%
%%%%% Converting this document to PDF %%%%%%%%
%%%%%%%%%%%%%%%%%%%%%%%%%%%%%%%%%%%%%%%%%%%%%%

%% We suggest the following method for this template because most
%% proposal reviewers will 
%% be reading your Scientific Justification on a computer screen, and
%% many reviewers only have PDF (not PostScript) viewing capability. By
%% default, a document produced with LaTeX uses so-called cm fonts, which
%% differ significantly from any of the standard Type1 fonts built-in to
%% PDF. PDF uses low-resolution bitmap images of these fonts to display
%% the document. It prints well, but shows degraded resolution on the
%% screen. 

%% Use ps2pdf (preferably version 1.4) to convert a postscript file to PDF:
%% 
%% latex file.tex
%% dvips -Ppdf -G0 -o file.ps file.dvi
%% ps2pdf14 file.ps file.pdf 

%%%%%%%%%%%%%%%%%%%%%%%%%%%%%%%%%%%%%%%%%%%%%%
%%%%% Displaying DS9 figures %%%%%%%%%%%%%%%%%
%%%%%%%%%%%%%%%%%%%%%%%%%%%%%%%%%%%%%%%%%%%%%%

%% If you experience pdf errors when displaying figures you produced with DS9:

%% Solution: Remake figure in DS9 and save as Postscript Level 1. In
%% the Print window choose Level 1 under the Postscript heading. At this
%% point you should re-LaTeX your SJ following the directions above. 




%%%%%%%%%%%%%%%%%%%%%%%%%%%%%%%%%%%%%%%%%%%%%%
%%%%% Default format: 11pt single column %%%%%
%%%%%%%%%%%%%%%%%%%%%%%%%%%%%%%%%%%%%%%%%%%%%%

\documentclass[letterpaper,11pt]{article}

%%%%%%%%%%%%%%%%%%%%%%%%%%%%%%%%%%%%
%%%%% Default font, two-column %%%%%
%%%%%%%%%%%%%%%%%%%%%%%%%%%%%%%%%%%%

%\documentclass[letterpaper,11pt,twocolumn]{article}

%%%%%%%%%%%%%%%%%%%%%%%%%%%%%%%%%%%%
%%%%% 'Maximum allowed' format %%%%%
%%%%%%%%%%%%%%%%%%%%%%%%%%%%%%%%%%%%

%\documentclass[letterpaper,11pt,twocolumn]{article}



%%%%%%%%%%%%%%%%%%%%%%%%%%%%%%%%%%
%%%%% HOW TO INCLUDE FIGURES %%%%%
%%%%%%%%%%%%%%%%%%%%%%%%%%%%%%%%%%

%% Please see the ``Included packages'' section below.



%%%%%%%%%%%%%%%%%%%%%%%%%%%%%
%%%%% Included packages %%%%%
%%%%%%%%%%%%%%%%%%%%%%%%%%%%%

\usepackage{graphics,graphicx}

%% Feel free to modify the included packages list to use your
%% favorite packages. 

%% In the graphics and graphicx packages, Postscript and eps figures
%% can be included using the \includegraphics command. The graphics
%% package is part of standard LaTeX2e and provides a basic way of including a
%% figure. The graphicx package is not standard, but extends the
%% \includegraphics command to make it more user-friendly. If graphicx
%% is not available on your system please remove it from the list of
%% included packages above.  

%% Syntax:
%% In the graphics package:
%%
%% \begin{figure}
%% \includegraphics[llx,lly][urx,ury]{file}
%% \end{figure}
%%
%% where ll denotes 'lower left' and ur 'upper right' and the x and y
%% values are the coordinates of the PostScript bounding box in
%% points. There are 72 points in an inch.
%%
%% In the graphicx package:
%% 
%% \begin{figure}
%% \includegraphics[key=val,key=val,...]{file}
%% \end{figure}
%%
%% where some of the useful keys are: angle, width, height,
%% keepaspectratio (='true' or 'false') and scale. Bounding box values
%% can be given as [bb=llx lly urx ury].
%%
%% In either case you have to use LaTeX figure placement commands to
%% position the figure on the page; \includegraphics will not do
%% that. Both these commands also have other options that are listed
%% in the LaTeX manual (for the graphics package) and in 'The LaTeX
%% Graphics Companion' (for the graphicx package).



%%%%%%%%%%%%%%%%%%%%%%%%%%%
%%%%% Page dimensions %%%%%
%%%%%  DO NOT CHANGE  %%%%%
%%%%%%%%%%%%%%%%%%%%%%%%%%%

\setlength{\textwidth}{6.5in} 
\setlength{\textheight}{9in}
\setlength{\topmargin}{-0.0625in} 
\setlength{\oddsidemargin}{0in}
\setlength{\evensidemargin}{0in} 
\setlength{\headheight}{0in}
\setlength{\headsep}{0in} 
\setlength{\hoffset}{0in}
\setlength{\voffset}{0in}



%%%%%%%%%%%%%%%%%%%%%%%%%%%%%%%%%%
%%%%% Section heading format %%%%%
%%%%%%%%%%%%%%%%%%%%%%%%%%%%%%%%%%

\makeatletter
\renewcommand{\section}{\@startsection%
{section}{1}{0mm}{-\baselineskip}%
{0.5\baselineskip}{\normalfont\Large\bfseries}}%
\makeatother



%%%%%%%%%%%%%%%%%%%%%%%%%%%%%
%%%%% Start of document %%%%% 
%%%%%%%%%%%%%%%%%%%%%%%%%%%%%

\begin{document}
\pagestyle{plain}
\pagenumbering{arabic}


 
%%%%%%%%%%%%%%%%%%%%%%%%%%%%%
%%%%% Title of proposal %%%%% 
%%%%%%%%%%%%%%%%%%%%%%%%%%%%%

\begin{center} 
\bfseries\uppercase{%
%%
%% ENTER TITLE OF PROPOSAL BELOW THIS LINE
Feeding the Feedback in Brightest Cluster Galaxies
%%
%%
}
\end{center}



%%%%%%%%%%%%%%%%%%%%%%%%%%%%%%%%%%%%%%%%%
%%%%% Body of science justification %%%%%
%%%%% and technical feasibility     %%%%%
%%%%%%%%%%%%%%%%%%%%%%%%%%%%%%%%%%%%%%%%%

%%
%% ENTER TEXT AND FIGURES BELOW
%%


The mass of a supermassive black hole tends to be about 0.2\% of the
mass of its host's galaxy bulge ( Gebhardt ; Ferrara ). The vast
majority of the stellar population of the most massive elliptical
galaxies is already present, billions of years ago, based on studies
of the color-magnitude relation and spectral energy distributions of
the ellipticals (REF). The mass correlation means that somehow,
formation and growth of the black hole is coupled to the formation and
growth of the galaxy, while at the same time the downsizing
phenomenon, wherein massive galaxies stop forming stars earlier than
low-mass galaxies, suggests that star formation somehow decouples from
the growth of the galaxy by hierarchical accretion. Structure
formation models where star formation is tracked predicts galaxies
which are far more blue and massive than real galaxies are. This
situation appears to be resolved in the models by the inclusion of
feedback driven by active galactic nuclei (AGN). 

Chandra's excellent spatial resolution reveals the presence of
cavities in the intracluster X-ray medium (ICM) surrounding radio
sources in the brightest central galaxy (BCG). Estimates of the PdV
work required to inflate such cavities has yielded for the the first
time the surprisingly high kinetic energy output of radio source
(REF). Central radio sources are quite common in BCGs residing in
clusters of galaxies with cool, low-entropy ICM cores, formerly known
as cooling-flow clusters. A low-level AGN operating at $\sim70-90\%$
duty cycle can plausibly stabilize such cores from a cooling
catastrophe; an occasional massive outburst could raise the level of
gas entropy well beyond the core. Cool cores, because they are so
common in luminous X-ray clusters are not short-lived features, and
therefore require some kind of feedback mechanism to stabilize
them. The most obvious source is the central AGN, which already
exhibits many of the features needed to counter cooling in the center
(sufficient energy, good location, visibly affecting the ICM.) While
this is not necessarily the feedback mode that quenches star
formation in elliptical galaxies at high redshift, this is the only
form of AGN feedback that we can study with this kind of detail.

Cavagnolo in his PhD thesis (completion date August 2008) has derived
and analyzed the entropy profiles of 200 clusters of galaxies from the
Chandra data archive. Entropy ($K$) here is the ratio of
$kT/n_{e}^{2/3}$, which is formally proportional to the natural log of
the thermodynamic quantity $S$. $K$ is the familiar adiabatic
constant $K$ in the equation of state $P=K\rho^{2/3}$. $K$ is the
quantity that will not change if the gas remains adiabatic. The ICM
can be rearranged, compressed, moved, but the quantity $K$ will remain
the same unless the gas radiates (cools) or shocked, and the gas will
re-arrange such that the lowest entropy gas sits at the bottom of the
potential well. $K$ is also directly related to the cooling time
(exactly so for a gas cooling only by bremsstrahlung). The entropy of
the gas in the outer regions of the cluster where the cooling time is
long retains the imprint of shocks that happened long ago. The
entropy of the gas in the cores with short cooling times is a direct
map of where radiative cooling is happening. The entropy profiles of
his thesis show that while the gas entropy of most clusters is quite
similar to each other at $r=100$ kpc (regardless of cluster
temperature, a coincidence of the fact that approximately, $K(r)
\propto r$), the core entropy varies quite a lot, from near 0 keV
cm$^2$ to several 100 keV cm$^2$ (Figure \ref{fig:ent}).

We noticed that if the core entropy were rather high, the BCG lacked
radio emission or H$\alpha$ emission (Figures \ref{fig:ha} and
\ref{fig:rad}). The presence of low-entropy ICM in clusters seemed to
be required to feed the central radio source and/or central star
formation. To pursue this trend, we looked through this sample for
examples of BCGs with radio emission but high core entropy ($>30$ keV
cm$^2$). We found only four clusters with $z<0.3$ that met this
criterion, and of those 4, 3 were at $z>0.2$, too high to follow up in
great detail. But the last cluster, Abell 193 ($z=0.048$, had only 10
kiloseconds of exposure, yet a very interesting core. Not only does it
have a radio source, but 10 ksec was sufficient to reveal a compact
($\leq 4$ kpc) corona inside the central galaxy. HST imaging
reveals three nuclei, one of which is coincident with the corona,
and another which appears to have its own X-ray counterpart (Figure
\ref{fig:hst}).

Coronae in cluster galaxies in hot clusters have been studied by Sun
et al. (2007) in his own PhD thesis. He found these coronae tend to be
smaller and less luminous than coronae in poorer environments; however
the existence of coronae in hot clusters is a testament to their
survival against evaporative conduction, rapid cooling, ram pressure
stripping, and even AGN feedback. Coronae, because they are typically
cool ($\sim0.5-1.1$ keV) and dense ($0.1-0.4$ cm$^{-3}$) are also
sources of low-entropy gas for any AGN inside the galaxy.

We are proposing deeper imaging of the center of Abell 193 in order to
assess a temperature for the corona, detect and characterize the X-ray
source around the second nucleus, and determine whether any X-ray
emission could be associated with the 3rd nucleus. This cluster is
close enough to place extremely interesting limits on corona
emission. The radio source is $2.51\times10^{39}$ ergs sec$^{-1}$ at
$21$ cm using the NVSS flux collected with VLA.

%%%%%%%%%%%%%
ADDED BY KWC:
%%%%%%%%%%%%%

Abell 193 was selected as a target because it stands out as a peculiar
object in a much larger sample. Of the 208 clusters in Cavagnolo's
archival thesis sample, 21 have a core entropy, $K_0$, greater than 30
keV cm$^2$ and a radio-loud source within 20 kpc of the X-ray surface
brightness peak. Of these clusters, 13 have radio emission which is
not coincident with the BCG or have a radio luminosity less than
$10^{40}$ ergs sec$^{-1}$. The remaining eight clusters have radio
emission from the BCG, but seven of these are at $z>0.1$ and resolving
or detecting X-ray corona would require unreasonably long
observations. This leaves only one cluster and its BCG, Abell 193 (IC
1695), as a viable target.

We are requesting an additional 65 ksec observation of Abell 193 to
determine the properties of the nuclear point sources and the
associated coronae. Using XSPEC 11.3.2ag we simulated a typical corona
spectrum ($T_X=0.8$, $Z/Z_{\odot}=0.8$) using cluster specific
parameters ($N_{HI}=4.57\times10^{20}$ cm$^2$, $z=0.0485$) and the
Cycle 10 RMF/ARF for ACIS-S. We find that an exposure time of 65 ksec
yields a count rate of $8.34\times10^{-3}$ cts sec$^{-1}$ for a total
of 542 counts. This count rate is consistent with results from PIMMS
using the same input spectrum and an emitting area equal to that of
the $\approx4$ kpc corona. Assuming good space weather, this
observation will enable us to constrain a coronal temperature to
$\pm0.3$ keV. We will be unable to constrain a metallicity, but this
is not unusual when analyzing corona spectra [REF Ming's paper].
Lauer et al 2007 ApJ, 664:226-256 calculate an effective radius for
the BCG of 4.74 kpc and a $M_V=-23.9$, both of which are consistent
with the size and luminosity of what we suspect is a corona.

Our focus is the brightest nuclear X-ray source in IC 1695, but the
two additional nuclei in the BCG will also be resolved and detected in
a new 65 ksec observation. Using HST data, Seigar et al 2003 MNRAS,
344:110-114 calculated the angular separations of the three nuclei in
IC 1695 to be 1.7$"$ (1-2), 3.5$"$ (1-3), and 2.1$"$
(2-3). And indeed, the current 10 ksec observation was able to resolve
nuclei 1 and 3. However, the count statistics are too poor for
spectral analysis of nucleus 3 and nucleus 2 is not detected in the
existing observation. For nucleus 3 we find a count rate of
$1.22\times10^{-3}$ cts sec$^{-1}$, and for 65 ksec this should result
in 80 counts. Again assuming a typical corona, we will be able to
constrain a temperature to $\pm 0.5$ keV. For nucleus 2, the
observation will be sufficient to detect X-ray emission beyond the
ambient background, but not to determine a temperature.

\begin{figure}[t]
\begin{minipage}[t]{0.5\linewidth}
	\centering
	\includegraphics*[width=\columnwidth, trim=28mm 10mm 30mm 17mm, clip]{splots}
	\caption{\small
	Plot of entropy versus physical radius for the 208 clusters from
	Cavagnolo et al. 2008a. The observed range of core entropy is
	consistent with models of AGN feedback. Color coding is for global
	cluster temperature in keV.
	}
	\label{fig:ent}
    \end{minipage}
    \hspace{0.25cm}
    \begin{minipage}[t]{0.5\linewidth}
        \centering
        \includegraphics*[width=\textwidth, trim=22mm 8mm 30mm 15mm, clip]{k0ha}
        \caption{\small
	Plot of H$\alpha$ and radio luminosity versus core entropy from
	Cavagnolo et al. 2008b. Above $K_0 \approx 30$ keV cm$^2$ signatures
	of feedback abate, which is most likely the result of heating the ICM
	via thermal electron conduction.}
        \label{fig:ha}
    \end{minipage}
    \hspace{0.25cm}
    \begin{minipage}[t]{0.5\linewidth}
        \centering
        \includegraphics*[width=\textwidth, trim=22mm 8mm 30mm 15mm, clip]{k0rad}
        \caption{\small
	Plot of radio luminosity versus core entropy. Radio
	luminosities are calculated from the NVSS 1.4 GHz flux.
	}
        \label{fig:rad}
    \end{minipage}
    \hspace{0.25cm}
    \begin{minipage}[t]{0.5\linewidth}
	\includegraphics*[width=\columnwidth, trim=0mm 0mm 0mm 0mm, clip]{xr_opt}
	\caption{\small
	Shown is the binned, smoothed {\it{Chandra}} X-ray image with an inset
	of the HST WFPC2 F814W I-band image. The green contours in the X-ray
	image are of the 1.4 GHz NVSS radio emission. Green contours in the
	optical image outline the X-ray emission from the inner
	$\sim10$ kpc. The three nuclei detected in the optical are labeled.
	}
	\label{fig:hst}
    \end{minipage}
\end{figure}

%%%%%%%%%%%%%
%%%%%%%%%%%%%

%% Technical Justification for Joint Facilities section
%% comment this section out on proposals not asking for
%% joint time

%\section{Technical Justification for Joint Facilities}

%% References section

\section{References}

Sun, M., Jones, C., Forman, W., Vikhlinin, A., Donahue, M., Voit, G. M. 2007, ApJ, 657, 197.

%%%%%%%%%%%%%%%%%%%%%%%%%%%%%%%%%%%%%%%%%%%%%%%%%%%%%%%%%%%%%%%% 
%%%%%  Previous Chandra Programs of the P.I. and Observer  %%%%% 
%%%%%%%%%%%%%%%%%%%%%%%%%%%%%%%%%%%%%%%%%%%%%%%%%%%%%%%%%%%%%%%% 
%% There is an easy way to change from twocolumn to onecolumn format and 
%% back in the same latex file
%%       \onecolumn
%% or
%%       \twocolumn
%% but note each will cause a clearpage.

\onecolumn
%\twocolumn

\section{Previous Chandra Programs}
 
%%%%%%%%%%%%%%%%%%%%%%%%%%%
%%%%% End of document %%%%%
%%%%%%%%%%%%%%%%%%%%%%%%%%%

\end{document}

