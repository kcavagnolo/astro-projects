\documentclass[letterpaper,11pt,twocolumn]{article}
\usepackage{graphicx,common}
\usepackage[nonamebreak,numbers,sort&compress]{natbib}
\bibliographystyle{plainnat}
\setlength{\textwidth}{6.5in} 
\setlength{\textheight}{9in}
\setlength{\topmargin}{-0.0625in} 
\setlength{\oddsidemargin}{0in}
\setlength{\evensidemargin}{0in} 
\setlength{\headheight}{0in}
\setlength{\headsep}{0in} 
\setlength{\hoffset}{0in}
\setlength{\voffset}{0in}

\makeatletter
\renewcommand{\section}{\@startsection%
{section}{1}{0mm}{-\baselineskip}%
{0.5\baselineskip}{\normalfont\Large\bfseries}}%
\makeatother

\begin{document}
\pagestyle{plain}
\pagenumbering{arabic}

\begin{center}
\bfseries\uppercase{The Hyperluminous Infrared Galaxy IRAS 09104+4109: An Extreme Brightest Cluster Galaxy}
\end{center}

\noindent{\bf{1. Introduction}}\\
The central cooling time of the intracluster medium (ICM) in many
clusters of galaxies is $\ll H_0^{-1}$. An expected consequence of
this short cooling time is that massive cooling flows, $> 100 \Msol
\pyr$, should form \citep{1977MNRAS.180..479F}, but these torrential
flows have instead turned out to be trickles \citep{peterson2001} with
the hot ICM never getting lower than $1/3 T_{\mathrm{virial}}$. In
recent years, this ``cooling flow problem'' has been the focus of much
study as the solutions should have broad impact in the areas of
galaxy formation, e.g. explaining truncation at the low- and
high-mass ends of the galaxy luminosity function. The cluster
community is converging on agreement regarding what mechanisms act to
retard the formation of a continuous cooling gas phase in cluster
cores.

The viable heating source comes in the form of feedback from active
galactic nuclei (AGN) \citep{hoticmrev}. But while several
robust models for heating the ICM via AGN feedback now exist, the
details of the feedback loop remain unresolved. Most models of AGN
feedback rely on quenched cooling from heat supplied by kinetic or
mechanical energy generated from a central supermassive black hole
(cSMBH) of the brightest cluster galaxy (BCG). Cool gas is channeled
into the cSMBH, which initiates an AGN feedback cycle, and is followed
by ejection of high energy particles into the ICM which carve out
cavities/bubbles. The work done by these bubbles on the surrounding
ICM, $W_{bubble} = pV$, goes into displacing ICM gas around and into
the bubble wake where the gravitational potential energy is then
released as enthalpy \citep{2004ApJ...607..800B}.

ICM entropy has proven to be a very useful quantity for understanding
the process of AGN heating and its effects on processes such as star
formation. ICM temperature, $T$, and density, $\rho$, primarily
reflect the depth and shape of the dark matter potential well, and
taken alone, they do not reveal the entire thermal history of a
cluster. But put in the context of entropy, $K=T\rho^{-2/3}$, we have
a more fundamental property of the ICM which is only affected by
heating and cooling. Measuring entropy from X-ray data thereby gives
us a direct measure of the cluster thermal history. Recalling that a
system is convectively stable only when $dK/dr \geq 0$, we can
observationally exploit gravitational potential wells as giant entropy
sorting devices. For example, departures from a radial power-law
entropy distribution are indicative of past heating and cooling of the
ICM.

\begin{figure}
\begin{center}
\includegraphics*[width=\columnwidth, trim=28mm 10mm 30mm 17mm, clip]{splots}
\caption{
Plot of entropy, $K(r) = Tn_{elec}^{-2/3}$, versus physical radius for
the 208 clusters from \cite{accept1}. The observed range of central
entropys is consistent with models of AGN feedback. Color coding is
for global cluster temperature in keV.
}
\label{fig:ent}
\end{center}
\end{figure}

Figure \ref{fig:ent} shows the radial entropy profiles for 208
clusters analyzed in the Ph.D. thesis of PI Cavagnolo. These
profiles were derived using 272 {\it{Chandra}} archival observations
with a nominal total exposure time of $\sim9$ Msec. The profiles show
a marked resemblance irrespective of global cluster temperatures
ranging from 1-15 keV. From this extensive entropy project we have found
that indicators of feedback, such as radio-loud sources assumed to be
AGN related and H$\alpha$ emission from star formation, are correlated
with low central entropy \citep{accept2}. We have also found that the
distribution of central entropy is at least bimodal, an expected
consequence of AGN feedback and conduction. The importance of
IRAS 09104+4109 to these results is that we suspect it is an object
which doesn't fit neatly into the AGN feedback models, and as such, a
detailed study of this peculiar system will be important for further
understanding how feedback couples to the processes of star formation,
BCG assembly, and heating of the ICM.\\

\noindent{\bf{2. IRAS 09104+4109: An Extreme BCG}}\\
At z $< 0.5$ and $L > 10^{11} L_{\odot}$ the most common extragalactic
objects are infrared galaxies. Among this population are a subset of
ultraluminous infrared galaxies (ULIRGs) with $L_{IR} \geq 10^{12}
L_{\odot}$, and an even more rare subset of hyperluminous infrared
galaxies (HLIRGs) with $L_{IR} \geq 10^{13} L_{\odot}$. IRAS
09104+4109 classifies as a HLIRG with a $L_{IR} \approx 10^{13}
L_{\odot}$. Most all ULIRGs and HLIRGs are interacting/merging spirals
or relics of recent mergers. Unlike fellow HLIRGs, IRAS 09104+4109 is
the BCG in the flattened, Abell richness class 2 cluster MACS
J0913.7+4056. Even more peculiar is that unlike most all BCGs found in
rich clusters, 99\% of IRAS 09104+4109's bolometric luminosity comes
longward of 1$\mu$m and peaks between 5-60$\mu$m. This enormous IR
luminosity is attributed to an obscured Seyfert type 2 AGN with a
large dust torus lying between the broad-line and narrow-line regions
\citep{1999ApJ...512..145H} and not to starbursts, as is the case for
many luminous infrared galaxies. In addition, only three other objects
in the IRAS catalog are comparable to 09104+4109 in luminosity and
they all lie at redshifts which would require unreasonably long
exposures to attain spectroscopic quality signal to noise: IRAS
15307+3252 ($z=0.93$), IRAS 16347+703 ($z=1.334$), and IRAS 10214+4724
($z=2.29$).

The fuel sources for the central AGN may be the optical nebular
filaments and companion galaxies within 30 kpc of the BCG which are
being stripped of their gas and cannibalized
\citep{1999Ap&SS.266..113A}. Dust embedded in hot gas has a short
sputtering time, thus the presence of dusty, substructure laden
filaments likely rules out the hot ICM as their origin
\citep{1993ApJ...414L..17D, 2000AJ....120..562T}.

Of all objects in the IRAS catalog, 09104+4109 hosts the most powerful
radio source, a borderline FRII/FRI with $P_{1.4GHz} =
3.2\times10^{24}$ W Hz$^{-1}$. Yet because of the steep radio spectrum
and huge IR luminosity, the radio source would be classified as
``quiet'' at higher frequencies. \cite{1999ApJ...512..145H} conclude
the radio source has undergone a recent ($< 70$ kyr) merger or
cataclysmic event which has altered the beaming direction of the
central AGN. As a result, the radio lobes are no longer receiving
power from the AGN and a new jet axis has been established. Adding to
the mystery of this object is that the nuclear absorber may have
flipped from Compton-thick to -thin in the last decade making
IRAS09104 the only ``changing-look'' AGN ever observed
\citep{2007A&A...473...85P}. Using the observation we propose here,
the Fe K$\alpha$ line can be used as a diagnostic for determining the
absorbing Compton thickness.\\

\begin{figure}
\begin{center}
\includegraphics*[width=\columnwidth, trim=0mm 0mm 0mm 0mm, clip]{chan_vla_utrao}
\caption{
The left panel shows the existing {\textit{Chandra}} data with low
resolution 20 cm {\it{VLA}} contours (white), high resolution 20 cm
{\it{VLA FIRST}} contours (green), and the new AGN outflow axis as
determined from the bipolar ionization cone discussed in
\cite{1999ApJ...512..145H} (cyan line). The right panel shows the same
{\textit{Chandra}} image with X-ray contours and green elliptical
regions highlighting an X-ray surface brightness decrement and edge.
}
\label{fig:chanrad}
\end{center}
\end{figure}

\noindent{\bf{3. Bubbles in IRAS 09104+410}}\\
We were surprised to see what is most likely an X-ray cavity $\approx
30$ kpc NW of the BCG in a cluster at $z > 0.43$. The X-ray decrement
in IRAS09104 was mentioned in passing by the PIs of the first
{\textit{Chandra}} observation (\S{3.1} of
\cite{2001MNRAS.321L..15I}). Analysis of the publicly-available,
low-resolution radio data from {\it{VLA First}} reveals a suggestive
alignment of the NW radio jet and the observed X-ray decrement (see
Fig. \ref{fig:chanrad}). X-ray emission in the opposite direction and
on the opposing side of the nuclear region is also suspiciously flat
and coincident with the other pole of the radio jet. We suggest the NW
decrement is the stem of a larger bubble and the SE plateau results
from interaction of the ICM with another large bubble. The highest
redshift bubbles observed to date are in RBS 797 at $z = 0.350$, a
cluster which was awarded 40 ks of additional time in Cycle 9.

Bubbles are a way of indirectly studying heating of the ICM by AGN,
and in the case of IRAS 09104+4109 we have the chance to see how a
rare, peculiar, and low mass object fits into the now widely used
model of quenching cooling in clusters by AGN feedback. The detection
of dust in the environment surrounding the two X-ray features
indicates shocking is not playing a major role in heating of the
gas. The presence of dust espouses the case for bubbles, as opposed to
a chance superposition of shocked regions, because the rims of bubbles
are dense and cold, not dense and hot as would be the case with
shocks. Coincidence of the radio jets and two prominent X-ray features
is unlikely to be a chance projection of overdense regions which have
no underlying physical connection. If confirmed, these would be the
highest redshift bubbles found to date.\\

\noindent{\bf{4. Scientific Questions}}\\
Calling the 100 kpc around IRAS 09104+4109 lively is an
understatement. The flurry of activity surrounding the AGN combined
with a peculiar amalgam of physical properties makes IRAS 09104+4109 a
rare, interesting, and extreme object. One should ask many questions
about the large scale structure of this object. The radio source in
IRAS 09104+4109 is undergoing a transition from a powerful FRII to a
mostly radio-quiet/FRI; thus we would expect the radial entropy
distribution slope to be relatively flat and resemble other
radio-quiet clusters, e.g. Abell 1650. But, IRAS09104 is no where near
as relaxed as other radio-quiet clusters and based on the X-ray
morphology we expect the entropy profile to be steeper, resembling
other ``dynamic'' clusters, such as 2A0335+096 or Abell 2029. Thus we
ask:\\
1. To which regime does the cluster housing IRAS 01904+4109 more
closely belong: radio-quiet or dynamically active? Can we definitively
classify this object as being in a short lived and elusive
transitional phase of galaxy, cluster, and AGN evolution?\\
2. Has the change in beaming direction of the radio source created
multiple bubbles?\\
3. What can the energetic properties of these bubbles
tell us about the connection between the present phase of AGN feedback
and epochs of feedback which may have occurred long prior?\\
4. Knowing of a recent change in the dynamics of the radio source, can
we detect this change by comparing the 2-dimensional entropy and
pressure structure at $r < 70$ kpc and $r > 70$ kpc?\\
5. Is the power output of the AGN (as measured by bubbles) large
enough to quench cooling and shutdown star formation?\\
6. The duty cycle of AGN is believed to be $\sim 10^8$ yrs, do we
find signatures of previous feedback cycles at large radii to
constrain the feedback timescale and compare it with other clusters?\\
Answers to these questions require resolving the extended X-ray
emission at radii greater than 70 kpc, which we cannot do with the
existing {\textit{Chandra}} and {\it{XMM-Newton}} observations.\\

\noindent{\bf{5. Proposed Research: Why not use existing data?}}\\
Using CIAO 3.4 and CALDB 3.4.2, we analyzed the existing ACIS-S3
observation taken 1999-11-03 by Fabian, which was originally used to
examine reflected X-ray emission of the AGN
\citep{2001MNRAS.321L..15I}. The nominal 9.1 ksec exposure shows
contamination by two strong flares which reduce the usable exposure
time to $\approx$5 ksec. However, there is an additional long duration
soft flare which is not associated with the local soft X-ray
background contaminating the remaining exposure time. Only by
addition of a cut-off power law to the background during spectral
fitting are we able to constrain a temperature. This additional
background component also introduces an unwanted systematic into the
spectral analysis which can create unreliable results.

While this observation serves the purpose of analyzing the bright
nuclear point source well enough, it is ill-suited for studies of
extended emission. Within an aperture of $r_{2500}$ we find 8,500
background-subtracted source counts. This total is insufficient to
create more than two radial temperature bins, and the signal-to-noise
is far too low to create 2D temperature, entropy, or pressure maps. We
are able to measure a global temperature of $8.06^{+3.25}_{-2.02}$ keV
without the central 50 kpc (0.697 cts/s), and $5.45^{+1.31}_{1.05}$
keV with the central 50 kpc (0.888 cts/s), both at 90\% confidence. We
are unable to resolve any extended emission or spatial features beyond
the central $\approx 70$ kpc.\\

\noindent{\bf{6. Request for new observation}}\\
We request a new 75 ksec observation of this object for
the purpose of resolving ICM features beyond the central 70 kpc, with
a specific focus on analyzing X-ray cavities associated with the AGN,
threshing out the energetics of the radio-ICM interaction, and
deriving the radial entropy distribution. {\it{Chandra}}'s high
spatial resolution is ideally, and necessarily, suited for observing
IRAS 09104+4109. We are attempting to resolve features on scales of
5-10 kpc, and at $z=0.442$, 10 kpc$=1.75''$ or 3.5 pixels at the
resolution of the ACIS detector. The outer edge of the NW radio lobe,
which is likely the maximum outer edge of any bubble we may find, lies
at 100 kpc from the nuclear point source and 90 kpc for the SE
lobe. Using the count rate for the core excised region, a temperature
of 8.06 keV, an energy window of 0.7-7.0 keV, an extended emission
area of $\approx80\mathrm{K~arcsec}^2$, and $N_{HI} = 1.36\times10^{20}$
cm$^{-2}$, PIMMS predicts a count rate of 0.538 cts/s for Cycle 10
which is consistent with our present analysis.

Under the assumption of no flares, the requested exposure time is
sufficient to yield eight radial temperature bins containing $\approx
5000$ counts each, which will allow us to measure temperatures within
$\pm0.2$ keV for $kT < 4$ keV and $\pm0.5$ keV for $kT > 4$ keV. These
temperature bins combined with surface brightness profiles will then
be used to construct high-resolution radial density, pressure,
entropy, and mass profiles to answer the first of our
scientific questions: is the cluster containing IRAS 09104+4109 more
like radio-quiet clusters or dynamically active clusters?
Using the adaptive binning code of \cite{2006MNRAS.368..497D}, we
will also construct 2D temperature, entropy, and pressure maps to
answer the questions regarding the change in dynamics of the AGN. For
the inner 70 kpc, the signal-to-noise will be sufficient to measure
temperatures in bins as small as $1.5''$. We will also use measured
densities and temperatures to calculate bubble pressures and
consequently the $pV$ work done inflating these bubbles. These
energetics calculations will then be used to analyze the AGN feedback
mechanism and ICM heating.

We are encouraged by our extensive experience with similar analyses
that IRAS 09104+4109, once adequately exposed, will yield interesting
results. How this unique and extreme object fits into the framework of
AGN feedback may tell us about a very short-lived but very important
stage of cluster and BCG formation. It will also provide interesting
constaints of radiative efficiency, power output, and quenching of
cooling in low mass systems, a study of which has not been done to
date.\\

\noindent{\bf{7. Budget Request and Justification}}\\
Show me the money?

\begin{thebibliography}{18}
\providecommand{\natexlab}[1]{#1}
\providecommand{\url}[1]{\texttt{#1}}
\expandafter\ifx\csname urlstyle\endcsname\relax
  \providecommand{\doi}[1]{doi: #1}\else
  \providecommand{\doi}{doi: \begingroup \urlstyle{rm}\Url}\fi

\bibitem[et~al.(1999{\natexlab{a}})]{1999Ap&SS.266..113A}
{Armus} et~al.
\newblock \apss, 266:\penalty0 113--118, 1999{\natexlab{a}}.

\bibitem[et~al.(2004)]{2004ApJ...607..800B}
{B{\^i}rzan} et~al.
\newblock \apj, 607:\penalty0 800--809, 2004.

\bibitem[et~al.(2008)]{accept1}
{Cavagnolo} et~al.
\newblock ApJS, submitted.

\bibitem[et~al.(2008)]{accept2}
{Cavagnolo} et~al.
\newblock ApJL, submitted.

\bibitem[et~al.(2006{\natexlab{a}})]{2006MNRAS.368..497D}
{Diehl} et~al.
\newblock \mnras, 368:\penalty0 497--510, 2006{\natexlab{a}}.

\bibitem[et~al.(1993{\natexlab{a}})]{1993ApJ...414L..17D}
{Donahue} et~al.
\newblock \apjl, 414:\penalty0 L17--L20, 1993{\natexlab{a}}.

\bibitem[et~al.(1977{\natexlab{b}})]{1977MNRAS.180..479F}
{Fabian} et~al.
\newblock \mnras, 180:\penalty0 479--484, 1977{\natexlab{b}}.

\bibitem[et~al.(1999{\natexlab{b}})]{1999ApJ...512..145H}
{Hines} et~al.
\newblock \apj, 512:\penalty0 145--156, 1999{\natexlab{b}}.

\bibitem[et~al.(2001{\natexlab{a}})]{2001MNRAS.321L..15I}
{Iwasawa} et~al.
\newblock \mnras, 321:\penalty0 L15--L19, 2001{\natexlab{a}}.

\bibitem[et~al.(2007)]{hoticmrev}
{McNamara} et~al.
\newblock \araa, 45:\penalty0 117--175, 2007.

\bibitem[et~al.(2001)]{peterson2001}
{Peterson} et~al.
\newblock \aap, 365:\penalty0 104--109, 2001.

\bibitem[et~al.(2007)]{2007A&A...473...85P}
{Piconcelli} et~al.
\newblock \aa, 473:\penalty0 85--89, 2007.

\bibitem[et~al.(2000)]{2000AJ....120..562T}
{Tran} et~al.
\newblock \aj, 120:\penalty0 562--574, 2000.

\end{thebibliography}

\noindent{\bf{Previous Chandra Programs}}
\noindent{\bf{Short CV of PI K. Cavagnolo}}

\end{document}
