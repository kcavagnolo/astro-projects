\documentclass[letterpaper,11pt]{article}
%\documentclass[letterpaper,11pt,twocolumn]{article}
\usepackage{graphics,graphicx}
\setlength{\textwidth}{6.5in} 
\setlength{\textheight}{9in}
\setlength{\topmargin}{-0.0625in} 
\setlength{\oddsidemargin}{0in}
\setlength{\evensidemargin}{0in} 
\setlength{\headheight}{0in}
\setlength{\headsep}{0in} 
\setlength{\hoffset}{0in}
\setlength{\voffset}{0in}
\makeatletter
\renewcommand{\section}{\@startsection%
{section}{1}{0mm}{-\baselineskip}%
{0.5\baselineskip}{\normalfont\Large\bfseries}}%
\makeatother
\begin{document}
\pagestyle{plain}
\pagenumbering{arabic}

\begin{center} 
\bfseries\uppercase{
Feeding the Feedback in Brightest Cluster Galaxies
}
\end{center}


The mass of a supermassive black hole tends to be about 0.2\% of the
mass of its host's galaxy bulge (Gebhardt et al. 2000 ; Ferrarese \&
Merritt 2000). The vast majority of the stellar population of the most
massive elliptical galaxies is already present, billions of years ago,
based on studies of the color-magnitude relation and spectral energy
distributions of the ellipticals (e.g. Kodama et al. 1998; Andreon et
al. 2008). The mass correlation means that somehow, formation and
growth of the black hole is coupled to the formation and growth of the
galaxy, while at the same time the downsizing phenomenon (Cowie et
al. 1999), wherein massive galaxies stop forming stars earlier than
low-mass galaxies, suggests that star formation somehow decouples from
the growth of the galaxy by hierarchical accretion (e.g. Benson et
al. 2002).  Structure formation models where star formation is tracked
predicts galaxies which are far more blue and massive than real
galaxies are (Kauffman \& Charlot 1998), even when supernovae feedback
was included (e.g. Dekel \& Silk 1986). This situation appears to be
resolved in the models by the inclusion of feedback driven by active
galactic nuclei (AGN) (e.g. Ciotti \& Ostriker 1997; Silk \& Rees
1998; Binney 2004).

{\it{Chandra}}'s excellent spatial resolution reveals the presence of
cavities in the intracluster X-ray medium (ICM) surrounding radio
sources in the brightest central galaxy (e.g. McNamara et
al. 2005). Estimates of the $PdV$ work required to inflate such
cavities has yielded for the the first time the surprisingly high
kinetic energy output of radio sources (cf. McNamara \& Nulsen
2007). Central radio sources are quite common in BCGs residing in
clusters of galaxies with cool, low-entropy ICM cores, formerly known
as cooling-flow clusters. A low-level AGN operating at $\sim70-90\%$
duty cycle can plausibly stabilize such cores from a cooling
catastrophe; an occasional massive outburst could raise the level of
gas entropy well beyond the core (e.g. Voit \& Donahue 2005). Cool
cores, because they are so common in luminous X-ray clusters cannot be
short-lived features, and therefore require some kind of feedback
mechanism to stabilize them. The most obvious source is the central
AGN, which already exhibits many of the features needed to counter
cooling in the center (sufficient energy, good location, visibly
affecting the ICM). While this is not necessarily the feedback mode
that quenches star formation in elliptical galaxies at high redshift,
this is the only form of AGN feedback that we can study with this kind
of detail.

Cavagnolo, in his Ph.D. thesis (completion date August 2008), has derived
and analyzed the entropy profiles of 200+ clusters of galaxies from the
{\it{Chandra}} data archive. Entropy ($K$) here is the ratio of
$kT/n_{e}^{2/3}$, which is formally proportional to the natural log of
the thermodynamic quantity $S$. $K$ is the familiar adiabatic
constant in the equation of state $P=K\rho^{2/3}$, and is the
quantity that will not change if the gas remains adiabatic. The ICM
can be rearranged, compressed, moved, but the quantity $K$ will remain
the same unless the gas radiates (cools) or shocks, and the gas will
re-arrange such that the lowest entropy gas sits at the bottom of the
potential well. $K$ is also directly related to the cooling time
(exactly so for as gas cooling only by Bremsstrahlung). The entropy of
the gas in the outer regions of the cluster where the cooling time is
long retains the imprint of shocks that happened long ago. The
entropy of the gas in the cores with short cooling times is a direct
map of where radiative cooling is happening. The entropy profiles of
Cavagnolo's thesis show that while the gas entropy of most clusters is quite
similar to each other at $r=100$ kpc (regardless of cluster
temperature, a coincidence of the fact that approximately, $K(r)
\propto r$), the core entropy varies quite a lot, from near 0 keV
cm$^2$ to several 100 keV cm$^2$ (Figure \ref{fig:ent}).

We noticed that if the core entropy were rather high, the BCG lacked
radio emission or H$\alpha$ emission (Figures \ref{fig:ha} and
\ref{fig:rad}). The presence of low-entropy ICM in clusters seemed to
be required to feed the central radio source and/or central star
formation. To pursue this trend, we looked through this sample for
examples of BCGs with radio emission but high core entropy ($>30$ keV
cm$^2$). We found only four clusters with $z<0.3$ that met this
criterion, and of those four, three were at $z>0.17$, too high to follow up in
great detail. But the last cluster, Abell 193 ($z=0.048$), had only 10
kiloseconds of exposure, yet a very interesting core. Not only does it
have a radio source, but 10 ksec was sufficient to reveal a compact
($\leq 4$ kpc) corona inside the central galaxy. HST imaging
reveals three nuclei, one of which is coincident with the corona,
and another which appears to have its own X-ray counterpart (Figure
\ref{fig:hst}).

Coronae in cluster galaxies embedded in hot clusters have been studied
by Sun et al. (2007). They found these coronae tend to be smaller and
less luminous than coronae in poorer environments; however the
existence of coronae in hot clusters is a testament to their survival
against evaporative conduction, rapid cooling, ram pressure stripping,
and even AGN feedback. Coronae, because they are typically cool
($\sim0.5-1.1$ keV) and dense ($0.1-0.4$ cm$^{-3}$), are also sources
of low-entropy gas for fueling an AGN inside the galaxy.

We are proposing deeper imaging of the center of Abell 193 in order to
assess a temperature for the corona, detect and characterize the X-ray
source around the second nucleus, and determine whether any X-ray
emission could be associated with the 3rd nucleus. This cluster is
close enough to place extremely interesting limits on corona
emission. The radio source power, $\nu L_\nu$, calculated
from the 21 cm flux collected with VLA, is $2.51\times10^{39}$ ergs
sec$^{-1}$, while the total radio luminosity is several $10^{40}$ erg
s$^{-1}$.

\section{Technical Feasibility}

Abell 193 was chosen for this study after an archival analysis of a
much larger sample. Of the 208 clusters in Cavagnolo's archival thesis
sample, 21 have a core entropy, $K_0$, greater than 30 keV cm$^2$ and
a radio-loud source within 20 kpc of the X-ray surface brightness
peak. Of these clusters, 13 have radio emission which is not
coincident with the BCG or have a total integrated radio luminosity
less than $10^{40}$ ergs sec$^{-1}$. The remaining eight clusters have
radio emission from the BCG, but seven of these are at $z>0.17$ and
resolving or detecting X-ray corona would require unreasonably long
observations. This left only one cluster and its BCG, Abell 193 (IC
1695), as a viable target. Abell 193 has been well-studied optically,
both by ground-based redshift studies (Girardi et al. 1998) and with HST
(Seppo et al. 2003; Seigar et al. 2003). 

We predicted that, based on the presence of a radio source, that our
standard radial entropy profile missed low-entropy gas in the core of
the cluster. Upon further investigation, this prediction was borne out
by our discovery of a compact, but extended, X-ray source in the
BCG. Surprisingly, the HST image reveals three nuclei in the core
(Seigar et al. 2003), one of which is coincident with the compact
corona source we found in the archival Chandra data. The presence of
multiple nuclei together with the significantly elevated core entropy
of $\sim 160$ keV cm$^2$ indicates a recent merger. This merger may
also represent the merger of individual BCG coronae, which in turn may
be the source of fuel for the radio source. It is this physical
situation which we would like to investigate in more detail. Lauer et
al (2007) calculate an effective radius for the optical starlight from
the BCG of 4.74 kpc and a $M_V=-23.9$, both of which are consistent
with the size and luminosity of what we suspect is a corona.

We are requesting a 65 ksec observation of Abell 193 (to add to 
the existing 10 ksec archive exposure) to determine the properties of
the nuclear point sources and the associated coronae. Using XSPEC
11.3.2ag we simulated a typical corona spectrum ($T_X=0.8$,
$Z/Z_{\odot}=0.8$) using cluster specific parameters
($N_{HI}=4.57\times10^{20}$ cm$^2$, $z=0.0485$) and the Cycle 10
RMF/ARF for ACIS-S. We find that an exposure time of 65 ksec yields a
count rate of $8.34\times10^{-3}$ cts sec$^{-1}$ for a total of 542
counts. This count rate is consistent with results from PIMMS using
the same input spectrum and an emitting area equal to that of the
$\approx4$ kpc corona. Assuming good space weather, the new
observation will enable us to constrain a coronal temperature to
$\pm0.3$ keV. We will be unable to constrain a metallicity, but this
is not unusual when analyzing corona or group spectra (Sun et
al. 2003; 2007).

The existing observation was aimed at ACIS-S3 and so we also simulated
the effect of taking a new observation on ACIS-I. ACIS-I offers the
advantage of a larger field of view, which would be useful for
studying the large-scale cluster environment, but we find that the
better soft energy ($< 1$ keV) sensitivity of ACIS-S is better suited
for this study. We predict from our simulations that an ACIS-I
observation will result in a count rate of $4.69\times10^{-3}$ cts
sec$^{-1}$ for a total of 305 counts.

Our main focus is the brightest nuclear X-ray source in IC 1695,
however the two additional nuclei in the BCG will also be resolved and
detected in a 65 ksec observation. Using HST data, Seigar et al (2003)
calculated the angular separations of the three nuclei in IC 1695 to
be $1.7"$ (1-2), $3.5"$ (1-3), and $2.1"$ (2-3). And indeed, the
archived 10 kilosecond observation was able to distinguish nucleus 1
from nucleus 3, and to resolve them both as extended sources. However,
the count statistics are too poor for any believable spectral analysis
of nucleus 3, and nucleus 2 is not detected at all. For nucleus 3 we
find a count rate of $1.22\times10^{-3}$ cts sec$^{-1}$, and for 65
ksec this should result in 80 counts. Again assuming a typical corona,
we will be able to constrain a temperature to $\pm 0.5$ keV and
distinguish the source as being point-like or extended. For nucleus 2,
the observation will be sufficient to detect X-ray emission beyond the
ambient background, but not to determine a temperature. These data
will allow us an unprecedented view into the activity occurring in the
belly of a massive central galaxy, of individual nuclei with coronae
about to merge and possibly feed a future outburst of the radio
source.

We will also have excellent counting statistics for the ICM in the
core of this nearby cluster. We will have 50,000 counts in the radial
range from $120-255"$, 123,000 counts minus the central source from
$0-255"$, which will allow us to detect holes, bubbles, decrements,
and fronts (whether shock fronts or cool fronts) in that
region. Current surface brightness data are fairly consistent with a
smooth beta model, but this observation will increase our sensitivity
to such features enormously. For example, we will be able to make 24
temperature annuli of 5,000 counts each for 0-255$"$. With the same
data, we'll be able to derive 12 metallicity bins. With 26,000 counts
in the central ``core'' region, we will be able to constrain the
contribution of a second thermal component in the ICM (ignoring the
coronae). The ACIS-S gives us far more sensitivity and improved
spectral resolution in the soft X-ray band, where the coronae and any
second thermal component are the brightest.

\begin{figure}[t]
\begin{minipage}[t]{0.5\linewidth}
	\centering
	\includegraphics*[width=\columnwidth, trim=28mm 10mm 30mm 17mm, clip]{splots}
	\caption{\small
	Plot of entropy versus physical radius for the 208 clusters from
	Cavagnolo et al. 2008a. The observed range of core entropy is
	consistent with models of AGN feedback. Color coding is for global
	cluster temperature in keV.
	}
	\label{fig:ent}
    \end{minipage}
    \hspace{0.25cm}
    \begin{minipage}[t]{0.5\linewidth}
        \centering
        \includegraphics*[width=\textwidth, trim=22mm 8mm 30mm 15mm, clip]{k0ha}
        \caption{\small
	Plot of H$\alpha$ versus core entropy from
	Cavagnolo et al. 2008b. Above $K_0 \approx 30$ keV cm$^2$ signatures
	of feedback abate, which is most likely the result of heating the ICM
	via thermal electron conduction.}
        \label{fig:ha}
    \end{minipage}
    \hspace{0.25cm}
    \begin{minipage}[t]{0.5\linewidth}
        \centering
        \includegraphics*[width=\textwidth, trim=22mm 8mm 30mm 15mm, clip]{k0rad}
        \caption{\small
	Plot of radio luminosity versus core entropy. Radio
	luminosities are calculated from the NVSS 1.4 GHz flux.
	}
        \label{fig:rad}
    \end{minipage}
    \hspace{0.25cm}
    \begin{minipage}[t]{0.5\linewidth}
	\includegraphics*[width=\columnwidth, trim=0mm 0mm 0mm 0mm, clip]{xr_opt}
	\caption{\small
	Shown is the binned, smoothed {\it{Chandra}} X-ray image with an inset
	of the HST WFPC2 F814W I-band image. The green contours in the X-ray
	image are of the 1.4 GHz NVSS radio emission. Green contours in the
	optical image outline the X-ray emission from the inner
	$\sim10$ kpc. The three nuclei detected in the optical are labeled.
	}
	\label{fig:hst}
    \end{minipage}
\end{figure}

%%%%%%%%%%%%%
%%%%%%%%%%%%%

%% Technical Justification for Joint Facilities section
%% comment this section out on proposals not asking for
%% joint time

%\section{Technical Justification for Joint Facilities}

%% References section

\section{References}
Andreon, S., Puddu, E., de Propris, Cuillandre, J.-C. 2008, MNRAS, 385, 979.\\
Benson, A. J, Ellis, R. S., Menanteau, F. 2002, MNRAS, 336, 564.\\
Binney, J. 2004, MNRAS, 347, 1093.\\
Ciotti, L. \& Ostriker, J. P. 1997, ApJ, 487, L105.\\
Cowie, L. L., Songaila, A., \& Barger, A. J. 1999, AJ, 118, 603.\\
Dekel, A.  \& Silk, J. 1986, ApJ, 303, 39.\\
Ferrarese, L. \& Merritt, D. 2000, ApJ, 539, L9.\\
Gebhardt, K. et al. 2000, ApJ, 539, L13.\\
Girardi, M. et al. 1998, ApJ, 505, 74.\\
Kauffman, G. \& Charlot, S. 1998, MNRAS, 294, 705.\\
Kodama, T., Arimoto, N., Barger, A., Aragon-Salamanca, A. 1998, A\&A, 334, 99.\\
Lauer, T. et al. 2007, ApJ, 664, 226.\\
McNamara, B. R. et al. 2005, Nature, 433, 45.\\
McNamara, B. R. \& Nulsen, P. E. J. 2007, ARAA, 45, 117.\\
Owen \& Ledlow 1997, ApJS, 108, 41.\\
Seigar, M. S. et al. 2003, MNRAS, 344, 110.\\
Seppo, L. et al. 2003, AJ, 125, 478.\\
Silk, J. \& Rees, M. J. 1998, A\&A, 331, L1.\\
Sun, M., Jones, C., Forman, W., Vikhlinin, A., Donahue, M., Voit, G. M. 2007, ApJ, 657, 197.\\
Sun, M., et al. 2003, ApJ, 598, 250.\\
Voit, G. M., \& Donahue, M. 2005, ApJ, 634, 955.

\onecolumn
%\twocolumn

\section{Previous Chandra Programs}
In reverse chronological order, by Cycle.
CYCLE 1:
Donahue, M., Gaskin, J. A., Patel, S. K, Joy, M., Clowe, D., Hughes, J. P. 2003, ApJ, 598, 190 (Prop ID 1800448)\\
Molnar, S. M., Hughes, J. P., Donahue, M., Joy, M. 2002, ApJ, 573, L91 (Prop ID 1800448).\\
Borys, C., Chapman, S., Donahue, M., Fahlman, G., Halpern, M., Kneib, J.-P., Newbury, P., Scott, D., Smith, G. P. 
2004, MNRAS, 352, 759. (Prop ID 1800448).\\

CYCLE 2:
Donahue, M., Daly, R. A., Horner, D. J. 2003, ApJ, 584, 643 (Prop ID2800376) .\\

CYCLE 4:
Donahue, M., Voit, G. M., O'Dea, C. P., Baum, S. a., Sparks, W. B. 2005, ApJ, 630, L13. (Prop ID 4800327).\\
Donahue, M., Horner, D. J., Cavagnolo, K. W., Voit, G. M., 2006, ApJ, 643, 730. (Archive proposal Prop ID 4800840, 
also was a pilot study for Cavagnolo's thesis.)\\

CYCLE 6:
Donahue, M.,Sun, M., Cavagnolo, K., Voit, G. M. 2006, AAS, 7711 (Abell 1650 Chandra proposal 6800721, paper in preparation).\\
Ventimiglia, D. A., Voit, G. M., Donahue, M., Ameglio, S. 2006 AAS 209, 7725 (Archive proposal, paper submitted in
revision to ApJ, 6800364).\\
Cavagnolo, K., Donahue, M., Voit, G. M., 2008, ApJ, paper to referee, 2nd round (Archive proposal ID 6800364)

On-going, started under Chandra observing and archive grants:

Cavagnolo, K. W. and Donahue, M. and Voit, G. M. and Sun, M.,
  ``Bandpass Dependence of X-ray Temperatures in Galaxy Clusters'',
  2008, ApJ submitted.

Cavagnolo, K. W. and Donahue, M. and Voit, G. M. and Sun, M., 2008,
  ``Athenaeum of Chandra Cluster Entropy Profile Tables -- I.
  Analysis and Disseminaton of Results'', in prep. for ApJS.

Cavagnolo, K. W. and Donahue, M. and Voit, G. M. and Sun, M., 2008, 
  ``Athenaeum of Chandra Cluster Entropy Profile Tables -- II.
  Star Formation and Radio-loud AGN'', in prep. for ApJL.

Cavagnolo, K. W. and Voit, G. M., and Donahue, M., 2008,
  ``Athenaeum of Chandra Cluster Entropy Profile Tables -- III.
  Entropy Scaling Relations in Galaxy Clusters'', in prep. for ApJ.
 
 
%%%%%%%%%%%%%%%%%%%%%%%%%%%
%%%%% End of document %%%%%
%%%%%%%%%%%%%%%%%%%%%%%%%%%

\enddocument

