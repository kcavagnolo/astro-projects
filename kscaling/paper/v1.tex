%%%%%%%%%%%%%%%%%%%
% Custom commands %
%%%%%%%%%%%%%%%%%%%

\newcommand{\clnum}{265}
\newcommand{\mykeywords}{galaxies: active -- galaxies: clusters:
  general -- galaxies: clusters: intracluster medium -- X-rays:
  galaxies: clusters}
\newcommand{\mystitle}{\accept\ Entropy Scaling Relations}
\newcommand{\mytitle}{Entropy Scaling Relations of \accept\ Galaxy
  Clusters}

%%%%%%%%%%
% Header %
%%%%%%%%%%

%\documentclass[11pt, preprint]{aastex}
%\usepackage{graphicx,common,longtable}
\documentclass[iop]{emulateapj-rtx4}
%\documentclass[iop]{emulateapj}
\usepackage{apjfonts,common,graphicx,longtable}
\usepackage[pagebackref,
  pdftitle={\mytitle},
  pdfauthor={K. W. Cavagnolo},
  pdfkeywords={\mykeywords},
  pdfsubject={Astrophysical Journal},
  pdfproducer={ps2pdf},
  pdfcreator={LaTeX with hyperref}
  pdfdisplaydoctitle=true,
  colorlinks=true,
  citecolor=blue,
  linkcolor=blue,
  urlcolor=blue]{hyperref}
\bibliographystyle{apj}
\begin{document}
\title{\mytitle}
\shorttitle{\mystitle}
\author{
  K. W. Cavagnolo\altaffilmark{1}
  G. M. Voit\altaffilmark{2},
  S. Bruch\altaffilmark{2},
  M. Donahue\altaffilmark{2},
  and M. Sun\altaffilmark{3}
}
\altaffiltext{1}{UNS, CNRS UMR 6202 Cassiop\'{e}e, Observatoire de la
  C\^{o}te d'Azur, Nice, France.}
\altaffiltext{2}{Department of Physics and Astronomy, Michigan State
University, East Lansing, MI, 48823, USA.}
\altaffiltext{3}{University of Virginia, Department of Astronomy,
  Charlottesville, VA, 22904}
\shortauthors{K. W. Cavagnolo et al.}

%%%%%%%%%%%%
% Abstract %
%%%%%%%%%%%%

\begin{abstract}
  Using archival data for a sample of \clnum\ galaxy clusters observed
  with the \chandra\ X-ray Observatory, we investigate the entropy
  scaling properties of the intracluster medium (ICM).
\end{abstract}

%%%%%%%%%%%%
% Keywords %
%%%%%%%%%%%%

\keywords{\mykeywords}

%%%%%%%%%%%%%%%%%%%%%%
\section{Introduction}
\label{sec:intro}
%%%%%%%%%%%%%%%%%%%%%%

- sources other than gravity modify cluster properties.\\
- understand mod to understand galaxy form/evo and use clusters in
cosmo\\
- explain entropy: dens temp alone, better together\\
- where does ent come from?\\
- ent conserves non-grav influence\\
- theoretical predictions\\
- observations\\
- \citet{accept}\\
- discuss possible sources of mod\\
- entropy scales w/ other props (T,M)\\
- $K \propto h(z)^{-4/3} T$ and $K \propto h(z)^{-2/3} M^{2/3}$\\
- what's presented in this paper\\

``The first measurements of entropy scaling were made at a fiducial
radius of $0.1r_{200}$ \citep{1999Natur.397..135P,davies00} in order
to sample the region between these two regimes. However,
\citet{pratt06} find consistent results for the scaling with
temperature of entropy when measured a range of radii at fractions of
r200 of 0.1, 0.2, 0.3 and 0.5 in the range $\beta = 0.5-0.6$ for $S
\propto T^{\beta}$.''

QQ: We've determined that $K(r)$ is well described by the equation
\begin{equation}
  K(r) = \kna + \khun \left(\frac{r}{100 ~\kpc}\right)^{\alpha}
\label{eqn:kr}
\end{equation}
with \kna\ representing the core entropy, \khun\ is the normalization
of $K(r)$ at $r=100\kpc$, and $\alpha$ is the slope of the entropy
profile power-law. 

QQ: How do these quantities scale, or relate, with cluster
temperature, $T_{cluster}$?

The work of \citet{minggroups} suggests that for $T_{X} \la 2\keV$,
the slope of the power-law component of $K(r)$ is shallower than is
found for cluster scale objects. This may indicate that above some
\tx\ the gravitationally established $K(r)$ scales with the dark
matter halo mass.

QQ: Is $\alpha$ independent of $T_{cluster}$?

QQ: Asymptotically, $\khun \propto T_{cluster}$, but is there a
coefficient in the log-space which minimizes scatter in \khun?

For a self-similar universe,
\begin{equation}
K(r) = \kna + f(T) K_{200} \left(\frac{r}{r_{200}}\right)^{\alpha}
\label{eqn:sskr}
\end{equation}
where $K_{200}$ is defined as the... from \citet{...}, $r_{200}$ is
the..., $f(T)$ is a function describing the temperature dependant
departure from self-similarity (\eg\ in a self-similar universe
$f(T)=1$), and \kna\ is the core entropy, which has been
observationally constrained, but is still poorly modeled in
simulations. This formalism also assumes that $\alpha$ is temperature
independent.

QQ: What is trend in $f(T)$ with cluster temperature?

Rewriting Eqn. \ref{eqn:sskr} as
\begin{eqnarray}
  K(r) &=& \kna + \left[f(T) K_{200}
    \left(\frac{100 ~\kpc}{r_{200}}\right)^{\alpha}\right]
    \left(\frac{r}{100 ~\kpc}\right)^{\alpha}\\
  \khun &=& f(T) K_{200} \left(\frac{100
    ~\kpc}{r_{200}}\right)^{\alpha}
  \label{eqn:obskr}
\end{eqnarray}
we recover Eqn. \ref{eqn:kr} in terms of the observational
parameters. Note that if the cluster mass-temperature relation ($M-T$)
has self-similar scaling behavior, then in the above equations,
$K_{200}$ and $r_{200}$ are functions of halo mass: $K_{200} \propto
T$ and $r_{200} \propto T^{1/2}$. But, irrespective of the $M-T$
behavior, $\khun \propto T^{\gamma}$ and hence \khun\ is mass model
assumption free. Thus, the general behavior of entropy in the
power-law dominated regime is written
\begin{equation}
\khun \propto f(T) T^{1-(\alpha/2)}
\label{eqn:genkr}
\end{equation}
and in the self-similar limit, $\gamma = 0.4$.

\LCDM. All errors are 68\% confidence level unless stated
otherwise.

%%%%%%%%%%%%%%%%%%%%%%%%%%%%%%%%%%%%%%%%
\section{Observations and Data Analysis}
\label{sec:obs}
%%%%%%%%%%%%%%%%%%%%%%%%%%%%%%%%%%%%%%%%

TBD

%%%%%%%%%%%%%%%%%
\section{Results}
\label{sec:r&d}
%%%%%%%%%%%%%%%%%

TBD

%%%%%%%%%%%%%%%%%%%%
\section{Discussion}
\label{sec:summary}
%%%%%%%%%%%%%%%%%%%%

TBD

%%%%%%%%%%%%%%%%%
\acknowledgements
%%%%%%%%%%%%%%%%%

TBD

%%%%%%%%%%%%%%%%
% Bibliography %
%%%%%%%%%%%%%%%%

\bibliography{cavagnolo}

%%%%%%%%%%%%%%%%%%%%%%
% Figures  and Tables%
%%%%%%%%%%%%%%%%%%%%%%

\clearpage
\clearpage
\begin{figure}[htp]
  \begin{center}
    \begin{minipage}[htp]{0.9\linewidth}
      \includegraphics*[width=\textwidth, trim=15mm 10mm 10mm 10mm, clip]{beta.eps}
      \caption{Surface brightness profiles for clusters requiring a
        $\beta$-model fit for deprojection (discussed in
        \S\ref{sec:beta}). The best-fit $\beta$-model for each cluster
        is overplotted as a dashed line. The discrepancy between the
        data and best-fit model for some clusters results from the
        presence of a compact X-ray source at the center of the
        cluster. These cases are discussed in Appendix
        \ref{app:beta}.}
      \label{fig:betamods}
    \end{minipage}
  \end{center}
\end{figure}
\clearpage
\begin{figure}[htp]
  \begin{center}
    \begin{minipage}[htp]{0.9\linewidth}
      \includegraphics*[width=\textwidth, trim=5mm 0mm 5mm 5mm, clip]{itplflat_rat.eps}
      \caption{Ratio of best-fit \kna\ for the two treatments of
        central temperature interpolation (see \S\ref{sec:temppr}):
        (1) temperature is free to decline across the central density
        bins ($\Delta T_{center} \ne 0$), and (2) the temperature
        across the central density bins is isothermal ($\Delta
        T_{center} = 0$). Filled black squares are clusters for which
        the \kna\ ratio is inconsistent with unity.}
      \label{fig:kcomp}
    \end{minipage}
  \end{center}
\end{figure}
\clearpage
\begin{figure}[htp]
  \begin{center}
    \begin{minipage}[htp]{0.9\linewidth}
      \includegraphics*[width=\textwidth, trim=5mm 0mm 5mm 5mm, clip]{k0res.eps}
      \caption{Best-fit \kna\ vs. redshift. Some clusters have
        \kna\ error bars smaller than the point. The clusters with
        upper-limits ({\it{black points with downward arrows}}) are:
        A2151, AS0405, MS 0116.3-0115, and RX J1347.5-1145. The
        numerically labeled clusters are: (1) M87, (2) Centaurus
        Cluster, (3) RBS 533, (4) HCG 42, (5) HCG 62, (6) SS2B153, (7)
        A1991, (8) MACS0744.8+3927, and (9) CL J1226.9+3332. For
        CLJ1226, \cite{2007ApJ...659.1125M} found best-fit $\kna = 132
        \pm 24 \ent$ which is not significantly different from our
        value of $\kna = 166 \pm 45 \ent$. The lack of $\kna < 10
        \ent$ clusters at $z > 0.1$ is most likely the result of
        insufficient angular resolution (see \S\ref{sec:angres}).}
      \label{fig:k0res}
    \end{minipage}
  \end{center}
\end{figure}
\clearpage
\begin{center}
  \begin{figure}[htp]
    \begin{minipage}[htp]{0.5\linewidth}
      \includegraphics*[width=\textwidth, trim=28mm 7mm 30mm 17mm, clip]{curvk0.eps}
    \end{minipage}
    \begin{minipage}[htp]{0.5\linewidth}
      \includegraphics*[width=\textwidth, trim=28mm 7mm 30mm 17mm, clip]{nbins_k0.eps}
    \end{minipage}
    \begin{minipage}[htp]{0.5\linewidth}
      \includegraphics*[width=\textwidth, trim=28mm 7mm 30mm 17mm, clip]{texpk0.eps}
    \end{minipage}
    \begin{minipage}[htp]{0.5\linewidth}
      \includegraphics*[width=\textwidth, trim=28mm 7mm 30mm 17mm, clip]{ntxbins_k0.eps}
    \end{minipage}
    \caption{Plots of possible systematics versus best-fit \kna.
      {\it{Top left:}} Best-fit \kna\ plotted versus average curvature
      of the corresponding entropy profile (see eq. \ref{eqn:avgcurv})
      There is no trend between these two quantities suggesting that
      \kna\ is not heavily influenced by the total shape of the
      entropy profile. {\it{Top right:}} Best-fit \kna\ plotted versus
      number of bins in the entropy profile which were used during
      fitting. Again, no trend is found. {\it{Bottom left:}} Best-fit
      \kna\ plotted versus the total used exposure time for each
      cluster. No trend is found. {\it{Bottom right:}} Best-fit
      \kna\ plotted versus the number of bins in the temperature
      profile for each cluster. As expected, fewer $\Tx(r)$ does not
      correlate with \kna.}
    \label{fig:sys}
  \end{figure}
\end{center}
\clearpage
\begin{center}
  \begin{figure}[htp]
    \begin{minipage}[htp]{0.5\linewidth}
      \includegraphics*[width=\textwidth, trim=28mm 7mm 30mm 17mm, clip]{splots_allt.eps}
    \end{minipage}
    \begin{minipage}[htp]{0.5\linewidth}
      \includegraphics*[width=\textwidth, trim=28mm 7mm 30mm 17mm, clip]{splots_tle4.eps}
    \end{minipage}
    \begin{minipage}[htp]{0.5\linewidth}
      \includegraphics*[width=\textwidth, trim=28mm 7mm 30mm 17mm, clip]{splots_gt4tle8.eps}
    \end{minipage}
    \begin{minipage}[htp]{0.5\linewidth}
      \includegraphics*[width=\textwidth, trim=28mm 7mm 30mm 17mm, clip]{splots_tgt8.eps}
    \end{minipage}
    \caption{Composite plots of entropy profiles for varying cluster
      temperature ranges. Profiles are color-coded based on average
      cluster temperature. Units of the color bars are keV. The solid
      line is the pure-cooling model of \cite{voitbryan}, the dashed
      line is the mean profile for clusters with $\kna \le 50 \ent$,
      and the dashed-dotted line is the mean profile for clusters with
      $\kna > 50 \ent$. {\it{Top left:}} This panel contains all the
      entropy profiles in our study. {\it{Top right:}} Clusters with
      $kT_X < 4$ keV. {\it{Bottom left:}} Clusters with $4\keV < kT_X
      < 8\keV$. {\it{Bottom right:}} Clusters with $kT_X > 8$
      keV. Note that while the dispersion of core entropy for each
      temperature range is large, as the $kT_X$ range increases so to
      does the mean core entropy.}
    \label{fig:splots}
  \end{figure}
\end{center}
\clearpage
\begin{figure}[htp]
  \begin{center}
    \begin{minipage}[htp]{0.9\linewidth}
      \includegraphics*[width=\textwidth, trim=20mm 10mm 10mm 10mm, clip]{k0hist.eps}
      \caption{{\it{Top panel:}} Histogram of best-fit \kna\ for all
        the clusters in \accept. Bin widths are 0.15 in log space.
        {\it{Bottom panel:}} Cumulative distribution of \kna\ values
        for the full sample. The distinct bimodality in \kna\ is
        present in both distributions, which would not be seen if it
        were an artifact of the histogram binning. A KMM test finds
        the \kna\ distribution cannot arise from a simple unimodal
        Gaussian.}
      \label{fig:k0hist}
    \end{minipage}
  \end{center}
\end{figure}
\clearpage
\begin{figure}[htp]
  \begin{center}
    \begin{minipage}[htp]{0.9\linewidth}
      \includegraphics*[width=\textwidth, trim=20mm 10mm 10mm 10mm, clip]{hifl_k0hist.eps}
      \caption{{\it{Top panel:}} Histogram of best-fit \kna\ values
        for the primary \hifl\ sample. Bin widths are 0.15 in log
        space.  {\it{Bottom panel:}} Cumulative distribution of
        best-fit \kna\ values. The distinct bimodality seen in the
        full \accept\ sample (Fig. \ref{fig:k0hist}) is also present
        in the \hifl\ subsample and shares the same gap between the
        low-entropy peak at 10-20 \ent\ and the high-entropy peak at
        100-200 \ent. That bimodality is present in both samples is
        strong evidence it is not a result of an unknown archival
        bias.}
      \label{fig:hiflk0}
    \end{minipage}
  \end{center}
\end{figure}
\clearpage
\begin{figure}[htp]
  \begin{center}
    \begin{minipage}[htp]{0.8\linewidth}
      \includegraphics*[width=\textwidth, trim=20mm 10mm 10mm 10mm, clip]{t0.eps}
    \end{minipage}
    \begin{minipage}[htp]{0.8\linewidth}
      \includegraphics*[width=\textwidth, trim=20mm 10mm 10mm 10mm, clip]{k0cool.eps}
    \end{minipage}
    \caption{{\it{Top panel:}} Log-binned histogram and cumulative
      distribution of best-fit core cooling times, $t_{c0}$
      (eqn. \ref{eqn:tc0}), for all the clusters in \accept. Histogram
      bin widths are 0.2 in log space. {\it{Bottom panel:}} Log-binned
      histogram and cumulative distribution of core cooling times
      calculated from best-fit \kna\ values, $t_{c0}(\kna)$
      (eqn. \ref{eqn:tck0}), for all the clusters in
      \accept. Histogram bin widths are 0.2 in log space. The
      bimodality we observe in the \kna\ distribution is also present
      in best-fit $t_{c0}$. However, the gaps between the two
      populations of $t_{c0}$ and $t_{c0}(\kna)$ differ by $\sim 0.3$
      Gyrs which may be an artifact of the binning.}
    \label{fig:t0}
  \end{center}
\end{figure}



%%%%%%%%%%%%%%%%%%%%
% End the document %
%%%%%%%%%%%%%%%%%%%%
\end{document}
