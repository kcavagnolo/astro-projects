%%%%%%%%%%%%%%%%%%%
% Custom commands %
%%%%%%%%%%%%%%%%%%%

\newcommand{\clnum}{265}
\newcommand{\mykeywords}{galaxies: active -- galaxies: clusters:
  general -- galaxies: clusters: intracluster medium -- X-rays:
  galaxies: clusters}
\newcommand{\mystitle}{\accept\ Entropy Scaling Relations}
\newcommand{\mytitle}{Entropy Scaling Relations of \accept\ Galaxy
  Clusters}

%%%%%%%%%%
% Header %
%%%%%%%%%%

%\documentclass[11pt, preprint]{aastex}
%\usepackage{graphicx,common,longtable}
\documentclass[iop]{emulateapj-rtx4}
%\documentclass[iop]{emulateapj}
\usepackage{apjfonts,common,graphicx,longtable}
\usepackage[pagebackref,
  pdftitle={\mytitle},
  pdfauthor={K. W. Cavagnolo},
  pdfkeywords={\mykeywords},
  pdfsubject={Astrophysical Journal},
  pdfproducer={ps2pdf},
  pdfcreator={LaTeX with hyperref}
  pdfdisplaydoctitle=true,
  colorlinks=true,
  citecolor=blue,
  linkcolor=blue,
  urlcolor=blue]{hyperref}
\bibliographystyle{apj}
\begin{document}
\title{\mytitle}
\shorttitle{\mystitle}
\author{
  K. W. Cavagnolo\altaffilmark{1}
  G. M. Voit\altaffilmark{2},
  S. Bruch\altaffilmark{2},
  M. Donahue\altaffilmark{2},
  and M. Sun\altaffilmark{3}
}
\altaffiltext{1}{UNS, CNRS UMR 6202 Cassiop\'{e}e, Observatoire de la
  C\^{o}te d'Azur, Nice, France.}
\altaffiltext{2}{Department of Physics and Astronomy, Michigan State
University, East Lansing, MI, 48823, USA.}
\altaffiltext{3}{University of Virginia, Department of Astronomy,
  Charlottesville, VA, 22904}
\shortauthors{K. W. Cavagnolo et al.}

%%%%%%%%%%%%
% Abstract %
%%%%%%%%%%%%

\begin{abstract}
  Using archival data for a sample of \clnum\ galaxy clusters observed
  with the \chandra\ X-ray Observatory, we investigate the entropy
  scaling properties of the intracluster medium (ICM).
\end{abstract}

%%%%%%%%%%%%
% Keywords %
%%%%%%%%%%%%

\keywords{\mykeywords}

%%%%%%%%%%%%%%%%%%%%%%
\section{Introduction}
\label{sec:intro}
%%%%%%%%%%%%%%%%%%%%%%

- sources other than gravity modify cluster properties.\\
- understand mod to understand galaxy form/evo and use clusters in
cosmo\\
- explain entropy: dens temp alone, better together\\
- where does ent come from?\\
- ent conserves non-grav influence\\
- theoretical predictions\\
- observations\\
- \citet{accept}\\
- discuss possible sources of mod\\
- entropy scales w/ other props (T,M)\\
- $K \propto h(z)^{-4/3} T$ and $K \propto h(z)^{-2/3} M^{2/3}$\\
- what's presented in this paper\\

``The first measurements of entropy scaling were made at a fiducial
radius of $0.1r_{200}$ \citep{1999Natur.397..135P,davies00} in order
to sample the region between these two regimes. However,
\citet{pratt06} find consistent results for the scaling with
temperature of entropy when measured a range of radii at fractions of
r200 of 0.1, 0.2, 0.3 and 0.5 in the range $\beta = 0.5-0.6$ for $S
\propto T^{\beta}$.''

QQ: We've determined that $K(r)$ is well described by the equation
\begin{equation}
  K(r) = \kna + \khun \left(\frac{r}{100 ~\kpc}\right)^{\alpha}
\label{eqn:kr}
\end{equation}
with \kna\ representing the core entropy, \khun\ is the normalization
of $K(r)$ at $r=100\kpc$, and $\alpha$ is the slope of the entropy
profile power-law. 

QQ: How do these quantities scale, or relate, with cluster
temperature, $T_{cluster}$?

The work of \citet{minggroups} suggests that for $T_{X} \la 2\keV$,
the slope of the power-law component of $K(r)$ is shallower than is
found for cluster scale objects. This may indicate that above some
\tx\ the gravitationally established $K(r)$ scales with the dark
matter halo mass.

QQ: Is $\alpha$ independent of $T_{cluster}$?

QQ: Asymptotically, $\khun \propto T_{cluster}$, but is there a
coefficient in the log-space which minimizes scatter in \khun?

For a self-similar universe,
\begin{equation}
K(r) = \kna + f(T) K_{200} \left(\frac{r}{r_{200}}\right)^{\alpha}
\label{eqn:sskr}
\end{equation}
where $K_{200}$ is defined as the... from \citet{...}, $r_{200}$ is
the..., $f(T)$ is a function describing the temperature dependant
departure from self-similarity (\eg\ in a self-similar universe
$f(T)=1$), and \kna\ is the core entropy, which has been
observationally constrained, but is still poorly modeled in
simulations. This formalism also assumes that $\alpha$ is temperature
independent.

QQ: What is trend in $f(T)$ with cluster temperature?

Rewriting Eqn. \ref{eqn:sskr} as
\begin{eqnarray}
  K(r) &=& \kna + \left[f(T) K_{200}
    \left(\frac{100 ~\kpc}{r_{200}}\right)^{\alpha}\right]
    \left(\frac{r}{100 ~\kpc}\right)^{\alpha}\\
  \khun &=& f(T) K_{200} \left(\frac{100
    ~\kpc}{r_{200}}\right)^{\alpha}
  \label{eqn:obskr}
\end{eqnarray}
we recover Eqn. \ref{eqn:kr} in terms of the observational
parameters. Note that if the cluster mass-temperature relation ($M-T$)
has self-similar scaling behavior, then in the above equations,
$K_{200}$ and $r_{200}$ are functions of halo mass: $K_{200} \propto
T$ and $r_{200} \propto T^{1/2}$. But, irrespective of the $M-T$
behavior, $\khun \propto T^{\gamma}$ and hence \khun\ is mass model
assumption free. Thus, the general behavior of entropy in the
power-law dominated regime is written
\begin{equation}
\khun \propto f(T) T^{1-(\alpha/2)}
\label{eqn:genkr}
\end{equation}
and in the self-similar limit, $\gamma = 0.4$.

\LCDM. All errors are 68\% confidence level unless stated
otherwise.

%%%%%%%%%%%%%%%%%%%%%%%%%%%%%%%%%%%%%%%%
\section{Observations and Data Analysis}
\label{sec:obs}
%%%%%%%%%%%%%%%%%%%%%%%%%%%%%%%%%%%%%%%%

TBD

%%%%%%%%%%%%%%%%%
\section{Results}
\label{sec:r&d}
%%%%%%%%%%%%%%%%%

TBD

%%%%%%%%%%%%%%%%%%%%
\section{Discussion}
\label{sec:summary}
%%%%%%%%%%%%%%%%%%%%

TBD

%%%%%%%%%%%%%%%%%
\acknowledgements
%%%%%%%%%%%%%%%%%

TBD

%%%%%%%%%%%%%%%%
% Bibliography %
%%%%%%%%%%%%%%%%

\bibliography{cavagnolo}

%%%%%%%%%%%%%%%%%%%%%%
% Figures  and Tables%
%%%%%%%%%%%%%%%%%%%%%%

\clearpage
\begin{figure}
  \begin{center}
    \begin{minipage}{\linewidth}
      \includegraphics*[width=\textwidth, trim=0mm 0mm 0mm 0mm, clip]{rbs797.ps}
    \end{minipage}
    \caption{Fluxed, unsmoothed 0.7--2.0 keV clean image of \rbs\ in
      units of ph \pcmsq\ \ps\ pix$^{-1}$. Image is $\approx 250$ kpc
      on a side and coordinates are J2000 epoch. Black contours in the
      nucleus are 2.5--9.0 keV X-ray emission of the nuclear point
      source; the outer contour approximately traces the 90\% enclosed
      energy fraction (EEF) of the \cxo\ point spread function. The
      dashed green ellipse is centered on the nuclear point source,
      encloses both cavities, and was drawn by-eye to pass through the
      X-ray ridge/rims.}
    \label{fig:img}
  \end{center}
\end{figure}

\begin{figure}
  \begin{center}
    \begin{minipage}{0.495\linewidth}
      \includegraphics*[width=\textwidth, trim=0mm 0mm 0mm 0mm, clip]{325.ps}
    \end{minipage}
   \begin{minipage}{0.495\linewidth}
      \includegraphics*[width=\textwidth, trim=0mm 0mm 0mm 0mm, clip]{8.4.ps}
   \end{minipage}
   \begin{minipage}{0.495\linewidth}
      \includegraphics*[width=\textwidth, trim=0mm 0mm 0mm 0mm, clip]{1.4.ps}
    \end{minipage}
    \begin{minipage}{0.495\linewidth}
      \includegraphics*[width=\textwidth, trim=0mm 0mm 0mm 0mm, clip]{4.8.ps}
    \end{minipage}
     \caption{Radio images of \rbs\ overlaid with black contours
       tracing ICM X-ray emission. Images are in mJy beam$^{-1}$ with
       intensity beginning at $3\sigma_{\rm{rms}}$ and ending at the
       peak flux, and are arranged by decreasing size of the
       significant, projected radio structure. X-ray contours are from
       $2.3 \times 10^{-6}$ to $1.3 \times 10^{-7}$ ph
       \pcmsq\ \ps\ pix$^{-1}$ in 12 square-root steps. {\it{Clockwise
           from top left}}: 325 MHz \vla\ A-array, 8.4 GHz
       \vla\ D-array, 4.8 GHz \vla\ A-array, and 1.4 GHz
       \vla\ A-array.}
    \label{fig:composite}
  \end{center}
\end{figure}

\begin{figure}
  \begin{center}
    \begin{minipage}{0.495\linewidth}
      \includegraphics*[width=\textwidth, trim=0mm 0mm 0mm 0mm, clip]{sub_inner.ps}
    \end{minipage}
    \begin{minipage}{0.495\linewidth}
      \includegraphics*[width=\textwidth, trim=0mm 0mm 0mm 0mm, clip]{sub_outer.ps}
    \end{minipage}
    \caption{Red text point-out regions of interest discussed in
      Section \ref{sec:cavities}. {\it{Left:}} Residual 0.3-10.0 keV
      X-ray image smoothed with $1\arcs$ Gaussian. Yellow contours are
      1.4 GHz emission (\vla\ A-array), orange contours are 4.8 GHz
      emission (\vla\ A-array), orange vector is 4.8 GHz jet axis, and
      red ellipses outline definite cavities. {\it{Bottom:}} Residual
      0.3-10.0 keV X-ray image smoothed with $3\arcs$ Gaussian. Green
      contours are 325 MHz emission (\vla\ A-array), blue contours are
      8.4 GHz emission (\vla\ D-array), and orange vector is 4.8 GHz
      jet axis.}
    \label{fig:subxray}
  \end{center}
\end{figure}

\begin{figure}
  \begin{center}
    \begin{minipage}{\linewidth}
      \includegraphics*[width=\textwidth]{r797_nhfro.eps}
      \caption{Gallery of radial ICM profiles. Vertical black dashed
        lines mark the approximate end-points of both
        cavities. Horizontal dashed line on cooling time profile marks
        age of the Universe at redshift of \rbs. For X-ray luminosity
        profile, dashed line marks \lcool, and dashed-dotted line
        marks \pcav.}
      \label{fig:gallery}
    \end{minipage}
  \end{center}
\end{figure}

\begin{figure}
  \begin{center}
    \begin{minipage}{\linewidth}
      \setlength\fboxsep{0pt}
      \setlength\fboxrule{0.5pt}
      \fbox{\includegraphics*[width=\textwidth]{cav_config.eps}}
    \end{minipage}
    \caption{Cartoon of possible cavity configurations. Arrows denote
      direction of AGN outflow, ellipses outline cavities, \rlos\ is
      line-of-sight cavity depth, and $z$ is the height of a cavity's
      center above the plane of the sky. {\it{Left:}} Cavities which
      are symmetric about the plane of the sky, have $z=0$, and are
      inflating perpendicular to the line-of-sight. {\it{Right:}}
      Cavities which are larger than left panel, have non-zero $z$,
      and are inflating along an axis close to our line-of-sight.}
    \label{fig:config}
  \end{center}
\end{figure}

\begin{figure}
  \begin{center}
    \begin{minipage}{0.495\linewidth}
      \includegraphics*[width=\textwidth, trim=25mm 0mm 40mm 10mm, clip]{edec.eps}
    \end{minipage}
    \begin{minipage}{0.495\linewidth}
      \includegraphics*[width=\textwidth, trim=25mm 0mm 40mm 10mm, clip]{wdec.eps}
    \end{minipage}
    \caption{Surface brightness decrement as a function of height
      above the plane of the sky for a variety of cavity radii. Each
      curve is labeled with the corresponding \rlos. The curves
      furthest to the left are for the minimum \rlos\ needed to
      reproduce $y_{\rm{min}}$, \ie\ the case of $z = 0$, and the
      horizontal dashed line denotes the minimum decrement for each
      cavity. {\it{Left}} Cavity E1; {\it{Right}} Cavity W1.}
    \label{fig:decs}
  \end{center}
\end{figure}


\begin{figure}
  \begin{center}
    \begin{minipage}{\linewidth}
      \includegraphics*[width=\textwidth, trim=15mm 5mm 5mm 10mm, clip]{pannorm.eps}
      \caption{Histograms of normalized surface brightness variation
        in wedges of a $2.5\arcs$ wide annulus centered on the X-ray
        peak and passing through the cavity midpoints. {\it{Left:}}
        $36\mydeg$ wedges; {\it{Middle:}} $14.4\mydeg$ wedges;
        {\it{Right:}} $7.2\mydeg$ wedges. The depth of the cavities
        and prominence of the rims can be clearly seen in this plot.}
      \label{fig:pannorm}
    \end{minipage}
  \end{center}
\end{figure}

\begin{figure}
  \begin{center}
    \begin{minipage}{0.5\linewidth}
      \includegraphics*[width=\textwidth, angle=-90]{nucspec.ps}
    \end{minipage}
    \caption{X-ray spectrum of nuclear point source. Black denotes
      year 2000 \cxo\ data (points) and best-fit model (line), and red
      denotes year 2007 \cxo\ data (points) and best-fit model (line).
      The residuals of the fit for both datasets are given below.}
    \label{fig:nucspec}
  \end{center}
\end{figure}

\begin{figure}
  \begin{center}
    \begin{minipage}{\linewidth}
      \includegraphics*[width=\textwidth, trim=10mm 5mm 10mm 10mm, clip]{radiofit.eps}
    \end{minipage}
    \caption{Best-fit continuous injection (CI) synchrotron model to
      the nuclear 1.4 GHz, 4.8 GHz, and 7.0 keV X-ray emission. The
      two triangles are \galex\ UV fluxes showing the emission is
      boosted above the power-law attributable to the nucleus.}
    \label{fig:sync}
    \end{center}
\end{figure}

\begin{figure}
  \begin{center}
    \begin{minipage}{\linewidth}
      \includegraphics*[width=\textwidth, trim=0mm 0mm 0mm 0mm, clip]{rbs797_opt.ps}
    \end{minipage}
    \caption{\hst\ \myi+\myv\ image of the \rbs\ BCG with units e$^-$
      s$^{-1}$. Green, dashed contour is the \cxo\ 90\% EEF. Emission
      features discussed in the text are labeled.}
    \label{fig:hst}
  \end{center}
\end{figure}

\begin{figure}
  \begin{center}
    \begin{minipage}{0.495\linewidth}
      \includegraphics*[width=\textwidth, trim=0mm 0mm 0mm 0mm, clip]{suboptcolor.ps}
    \end{minipage}
    \begin{minipage}{0.495\linewidth}
      \includegraphics*[width=\textwidth, trim=0mm 0mm 0mm 0mm, clip]{suboptrad.ps}
    \end{minipage}
    \caption{{\it{Left:}} Residual \hst\ \myv\ image. White regions
      (numbered 1--8) are areas with greatest color difference with
      \rbs\ halo. {\it{Right:}} Residual \hst\ \myi\ image. Green
      contours are 4.8 GHz radio emission down to
      $1\sigma_{\rm{rms}}$, white dashed circle has radius $2\arcs$,
      edge of ACS ghost is show in yellow, and southern whiskers are
      numbered 9--11 with corresponding white lines.}
    \label{fig:subopt}
  \end{center}
\end{figure}


%%%%%%%%%%%%%%%%%%%%
% End the document %
%%%%%%%%%%%%%%%%%%%%
\end{document}
