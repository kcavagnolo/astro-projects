%%%%%%%%%%%%%%%%%%%
% Custom commands %
%%%%%%%%%%%%%%%%%%%

\newcommand{\dmb}{\ensuremath{\dot{m_{\rm{B}}}}}
\newcommand{\dme}{\ensuremath{\dot{m_{\rm{E}}}}}
\newcommand{\ldisk}{\ensuremath{L_{\rm{disk}}}}
\newcommand{\lfir}{\ensuremath{L_{\rm{FIR}}}}
\newcommand{\lfuv}{\ensuremath{L_{\rm{FUV}}}}
\newcommand{\lnuv}{\ensuremath{L_{\rm{NUV}}}}
\newcommand{\ms}{MS 0735.6+7421}
\newcommand{\myi}{\ensuremath{I^{\prime}}}
\newcommand{\myv}{\ensuremath{V^{\prime}}}
\newcommand{\radec}{R.A.(J2000) $=09^h 47^m 12^s.8$, Dec.(J2000)
  $=+76\mydeg 23^{\arcm} 13^{\arcs}.66$}
\newcommand{\rbs}{R797}
\newcommand{\reff}{\ensuremath{r_{\rm{eff}}}}
\newcommand{\rlos}{\ensuremath{r_{\rm{los}}}}
\newcommand{\rxj}{RX J0947.2+7623}
\newcommand{\mykeywords}{}
\newcommand{\mystitle}{\rbs\ AGN Outburst}
\newcommand{\mytitle}{A Possibly Black Hole Spin Powered AGN Outburst in
  RBS 797}

%%%%%%%%%%
% Header %
%%%%%%%%%%

%\documentclass[11pt, preprint]{aastex}
%\usepackage{graphicx,common}
\documentclass{emulateapj}
\usepackage{apjfonts,graphicx,common}
%% \usepackage[pagebackref,
%%   pdftitle={\mytitle},
%%   pdfauthor={Dr. Kenneth W. Cavagnolo},
%%   pdfsubject={ApJ},
%%   pdfkeywords={},
%%   pdfproducer={LaTeX with hyperref},
%%   pdfcreator={LaTeX with hyperref},
%%   pdfdisplaydoctitle=true,
%%   colorlinks=true,
%%   citecolor=blue,
%%   linkcolor=blue,
%%   urlcolor=blue]{hyperref}
\bibliographystyle{apj}
\begin{document}
\title{\mytitle}
\shorttitle{\mystitle}
\author{
  K. W. Cavagnolo\altaffilmark{1,2,8},
  B. R. McNamara\altaffilmark{1,3,4},
  M. W. Wise\altaffilmark{5}, \\
  P. E. J. Nulsen\altaffilmark{4},
  M. Gitti\altaffilmark{4,6}, and
  M. Br\"uggen\altaffilmark{7},
}
\shortauthors{Cavagnolo et al.}

\altaffiltext{1}{University of Waterloo, 200 University Ave. W.,
  Waterloo, ON, N2L 3G1, Canada.}
\altaffiltext{2}{Observatoire de la C\^ote d'Azur, Boulevard de
  l'Observatoire, B.P. 4229, F-06304, Nice, France.}
\altaffiltext{3}{Perimeter Institute for Theoretical Physics, 31
  Caroline St. N., Waterloo, ON, N2L 2Y5, Canada.}
\altaffiltext{4}{Harvard-Smithsonian Center for Astrophysics, 60
  Garden St., Cambridge, MA, 02138-1516, United States.}
\altaffiltext{5}{Astronomical Institute Anton Pannekoek, P.O. Box
  94249, 1090 GE Amsterdam, The Netherlands.}
\altaffiltext{6}{INAF-Astronomical Observatory of Bologna, Via
  Ranzani 1, I-40127 Bologna, Italy.}
\altaffiltext{7}{Jacobs University Bremen, P.O. Box 750561, 28725
  Bremen, Germany.}
\altaffiltext{8}{kcavagno@uwaterloo.ca}

%%%%%%%%%%%%
% Abstract %
%%%%%%%%%%%%

\begin{abstract}
  Utilizing a $\sim 40$ ks \chandra\ X-ray observation, we present
  spatially resolved analysis of the intracluster medium (ICM),
  cavities, and the nuclear point source of \rxj\ and its brightest
  cluster galaxy (BCG), RBS 797. blahbity blah blah...
\end{abstract}

%%%%%%%%%%%%
% Keywords %
%%%%%%%%%%%%

\keywords{\mykeywords}

%%%%%%%%%%%%%%%%%%%%%%
\section{Introduction}
\label{sec:intro}
%%%%%%%%%%%%%%%%%%%%%%

Evidence has amassed over the last decade that supermassive black
holes (SMBHs), and specifically the energetic feedback from active
galactic nuclei (AGN), have a critical role in the growth and
evolution of galaxies \citep[\eg][]{1995ARA&A..33..581K, magorrian,
  1998A&A...331L...1S, 2000MNRAS.311..576K, 2000ApJ...539L...9F,
  2000ApJ...539L..13G, 2002ApJ...574..740T}. The discovery of AGN
induced cavities in the hot halos surrounding many massive galaxies
has strengthened this hypothesis \citep[see][for a review]{mcnamrev},
revealing that AGN mechanical heating is capable of regulating
late-time galaxy growth \citep[\eg][]{birzan04, dunn06, rafferty06,
  croton06, bower06, sijacki07}. Most models explaining AGN activity
rely on some amount of mass accretion onto a SMBH, but whether the
energy transported away by emergent jets is reflective of the
gravitational binding energy of the total accreted mass
\citep[see][for a review]{1984RvMP...56..255B} or the rotational
energy of the SMBH \citep[see][for a review]{2002NewAR..46..247M} is
unclear. Mass accretion alone can, in principle, meet the energetic
demands of an AGN outburst \citep[\eg][]{pizzolato05,
  2006MNRAS.372...21A}, and, on average, is not at odds with the gas
masses and stellar masses of a host galaxy
\citep[\eg][]{rafferty06}. However, for extremely powerful outbursts,
like those of \ms\ \citep{ms0735} and Hercules A \citep{herca} which
have total energies $> 10^{61}$ erg and powers $> 10^{45} ~\lum$, the
release of SMBH spin energy becomes a more appealing power source
\citep[\eg][]{msspin, minaspin}. In this paper, we present evidence
which suggests the AGN outburst associated with RBS 797 (\rbs,
hereafter) may have been powered by black hole spin.

\citet{schindler01} showed that \rbs\ ($z \approx 0.354$) is bracketed
by a pair of deep cavities in the ICM, and multifrequency radio
observations confirm that the cavities are co-spatial with extended
radio emission which is centered around a strong, jetted BCG radio
source \citep{2002astro.ph..1349D, gitti06, birzan08}. The data
implicates the central AGN as the cavities' progenitor. Analysis
presented in \citet[][hereafter B04]{birzan04} determined that the
cavity system contains $\sim 10^{60}$ erg of energy, deposited by jets
with powers of $\sim 10^{45} ~\lum$. These are relatively large values
for a cavity system. The B04 analysis assumed that the cavities are
distinct, symmetric about the plane of the sky (POS), and that their
centers lie in a plane passing through the central AGN and
perpendicular to the line of sight (LOS). However, the extreme depth
of the cavities, and the atypical relationship between the radio and
X-ray emission (see Section \ref{sec:cavities}), calls into question
these assumptions.

Using a longer follow-up \chandra\ observation, we have more deeply
investigated the cavity morphologies and suggest, with the aide of a
hydrodynamical cavity simulation, that the observed cavities may
result from the superposition of two larger, overlapping cavities
positioned nearly along the LOS with centers out of the POS. As a
result, the cavity volumes may be much larger than the B04 estimates,
and thus the cavity energies, \ecav, and power of the jets which
formed the cavities, \pjet, could be significantly higher. Our
analysis suggests \ecav\ and \pjet\ may be $> 10^{61}$ erg and $>
10^{45} ~\lum$, respectively, in the regime where SMBH spin is a more
plausible power source than mass accretion alone. The details of our
case supporting this conclusion are presented herein.

Observations and data reduction are discussed in Section
\ref{sec:obs}, results are given throughout Section \ref{sec:results},
and a brief summary concludes the paper in Section
\ref{sec:con}. \LCDM. At a redshift of $z = 0.354$, the look-back time
is 3.9 Gyr, $\da = 4.996$ kpc arcsec$^{-1}$, and $\dl = 1889$ Mpc. All
errors are 68\% confidence unless stated otherwise.

%%%%%%%%%%%%%%%%%%%%%%%%%%%%%%%%%%%%%%%%%
\section{Observations and Data Reduction}
\label{sec:obs}
%%%%%%%%%%%%%%%%%%%%%%%%%%%%%%%%%%%%%%%%%

%%%%%%%%%%%%%%%%%%
\subsection{X-ray}
%%%%%%%%%%%%%%%%%%

X-ray data was used to analyze the radial ICM properties (Section
\ref{sec:icm}), the cavity system (Section \ref{sec:cavities}), and
properties of the BCG point source (Section \ref{sec:accretion}).
\rxj\ was observed with the Chandra X-ray Observatory (\cxo) in
October 2000 for 11.4 ks using the ACIS-I array (\dataset
[ADS/Sa.CXO#Obs/02202] {ObsID 2202}; PI Schindler) and July 2007 for
38.3 ks using the ACIS-S array (\dataset [ADS/Sa.CXO#Obs/07902] {ObsID
  7902}; PI McNamara). Datasets were reduced using \ciao\ and
\caldb\ versions 4.2. Events were screened using \asca\ grades, and
cosmic rays were further rejected via {\textsc{vfaint}} filtering. The
level-1 events files were reprocessed to apply corrections for the
time-dependent gain change, charge transfer inefficiency, and degraded
quantum efficiency. The afterglow and dead area corrections were also
applied. Time intervals affected by background flares exceeding 20\%
of the mean background count rate were isolated and removed using
light curve filtering. The final, combined exposure time is 48.8
ks. Point sources were identified and removed via visual inspection
and use of the \ciao\ tool {\textsc{wavdetect}}. We refer to the
\cxo\ data free of point sources and flares as the ``clean'' data. A
mosaiced, fluxed image was generated by exposure correcting each clean
dataset and reprojecting the normalized images to a common tangent
point (see Figure \ref{fig:img}).

%%%%%%%%%%%%%%%%%%%%%%%%
\subsection{Ultraviolet}
%%%%%%%%%%%%%%%%%%%%%%%%

The UV measurements below are utilized in Section \ref{sec:bcg} to
constrain the \rbs\ star formation rate. \rbs\ was imaged in the
far-UV (FUV; 1344--1786 \AA; FWHM $\approx 4.5\arcs$) and near-UV
(NUV; 1771--2831 \AA; FWHM $\approx 6.0\arcs$) for 136 s by the Galaxy
Evolution Explorer (\galex) in December 2003. \rbs\ is detected in
both filters but is unresolved. The FUV, NUV, optical, and X-ray
sources are co-spatial, and the next nearest UV source is at a
distance of $19\arcs$, indicating the detection is unlikely to be a
spurious source. The source fluxes are $f_{\rm{FUV}} = 19.2 \pm 4.8
~\mu$Jy and $f_{\rm{NUV}} = 5.9 \pm 2.1 ~\mu$Jy.

The XMM-Newton Optical-Monitor (\xom) imaged \rbs\ in April 2008 for
4.5 ks (DataID 0502940301; PI M. Gitti). Exposures were taken with two
filters: UVW1 (2410--3565 \AA; FWHM $\approx 2.0\arcs$) and UVM2
(1970--2675 \AA; FWHM $\approx 1.8\arcs$). The data was processed
using \sas\ version 8.0.1 and the task {\textsc{omichain}}. Using the
tool {\textsc{omdetect}}, an unresolved source co-spatial with
\rbs\ was detected in the UVW1 image, but no corresponding source was
detected in the UVM2 image. A $3\sigma$ upper limit on the UVM2 flux
was calculated using the UVW1 source as a guide. For each filter, mean
count rates for the source region were measured using the tool
{\textsc{omsource}}, converted to instrumental magnitudes, and then to
fluxes, giving $f_{\rm{UVW1}} = 14.6 \pm 4.6$ $\mu$Jy and
$f_{\rm{UVM2}} < 117$ $\mu$Jy.

%%%%%%%%%%%%%%%%%%%%
\subsection{Optical}
%%%%%%%%%%%%%%%%%%%%

\hst\ optical images are used in Section \ref{sec:bcg} to dissect the
BCG structure and investigate interaction of the central AGN with the
\rbs\ halo. \rbs\ was imaged for 1.2 ks with the Hubble Space
Telescope (\hst) ACS/WFC instrument in March and December 2006 (prop
IDs 10491 and 10875; PI H. Ebeling). The F606W (4500--7500 \AA; \myv)
and F814W (6800--9800 \AA; \myi) filters were used during the
observations, and are denoted with primes to distinguish from the
standard Johnson $V$ and $I$ filters. The cosmic ray cleaned, drizzled
images from the Hubble Legacy Archive pipeline were used for
analysis.

The central $1.5\arcs$ of the BCG contains some substructure, with the
central $0.25\arcs$ having the appearance of being pinched into two
equal brightness sources. It is unclear which, if either, is at the
center of the galaxy or might be associated with an AGN. Two ACS
ghosts \citep{acsghost} are present in the \myi\ image: the northern
ghost overlaps with light from the BCG beginning at $\approx 2.7\arcs$
from the galaxy center, and the western ghost at $\approx
5.4\arcs$. Within a $2\arcs$ radius, circular aperture centered on
\rbs, the measured magnitudes are $m_{\myv} = 19.3 \pm 0.7$ mag and
$m_{\myi} = 18.2 \pm 0.6$ mag, consistent with the results of
\citet{rbs1} suggesting the photometry in this region is not impacted
by the ghosts.

Optical spectroscopy (see below) reveals strong emission lines of
\hbeta, [O \Rmnum{2}], [O \Rmnum{3}], and, presumably,
\halpha\ associated with the region around \rbs. These lines lie
within the \hst\ passbands, and their contribution to the \hst\ images
was determined using the ratio of line equivalent widths (EWs) to
passband widths. The \halpha\ contribution to the \myi\ image was
estimated to be $\approx 3\%$ (likely an overestimate since the
stellar continuum rises toward the red) by scaling \hbeta\ using a
Balmer decrement of EW$_{\hbeta}$/EW$_{\halpha}$ = 0.29
\citep{2006ApJ...642..775M}. The \hbeta, \oii, and \oiii\ contribution
to the \myv\ image is $\approx 7\%$. The low line contamination
indicates the stellar continuum is well-represented in the
observations.

A spectrum (3200-7600 \AA) of \rbs\ was obtained with the longslit
spectrograph on the Bolshoi Altazimuth Telescope (BTA) in February
1996 \citep{rbs1}. The slit setup is unknown, and flux calibration was
applied using the average of response functions for a set of standard
stars, resulting in wavelength-dependent photometric uncertainties
$\ga 50\%$. Because measurements made from the BTA spectrum are highly
uncertain, they are only discussed for completeness. Wavelength
calibration was applied using a He-Ar-Ne lamp, and sky subtraction was
performed pixel-wise along each CCD column of regions 25 CCD rows wide
below and above the trace of the spectrum. Spectral features were
located and fit with tools in the \iraf\ {\textsc{onedspec}} package.

%%%%%%%%%%%%%%%%%%
\subsection{Radio}
%%%%%%%%%%%%%%%%%%

Very Large Array (\vla) radio images at 325 MHz (A-array), 1.4 GHz (A-
and B-array), 4.8 GHz (A-array), and 8.4 GHz (D-array) were presented
in \citet{gitti06} and \citet{birzan08}. Re-analysis of the archival
\vla\ radio observations yielded no significant differences with the
previous studies. Using rms noise values for each frequency given in
\citet{gitti06} and \citet{birzan08}, emission contours between
$3\sigma_{\rm{rms}}$ and the peak image intensity were
generated. These are the contours referenced and shown in all
subsequent discussion and figures.

%%%%%%%%%%%%%%%%%%%
\section{Results}
\label{sec:results}
%%%%%%%%%%%%%%%%%%%

%%%%%%%%%%%%%%%%%%%%%%%%%%%%%%%%%%
\subsection{Radial ICM Properties}
\label{sec:icm}
%%%%%%%%%%%%%%%%%%%%%%%%%%%%%%%%%%

In this section we present radial analysis of the \rxj\ ICM, and
quantify properties of the cluster core and gas surrounding the
cavities. The radial profiles discussed below are shown in Figure
\ref{fig:gallery}.

The ICM temperature (\tx) structure was determined by fitting spectra
extracted from concentric circular annuli containing 2500 source
counts centered on the cluster X-ray peak. For each annulus, weighted
responses were created and background spectra were extracted from the
ObsID matched \caldb\ blank-sky dataset. The background data was
renormalized using the ratio of 9--12 keV count rates for identical
off-axis, source-free regions of the blank-sky and target
datasets. Using the methodology of \citet{2005ApJ...628..655V}, a
spectral model for the Galactic foreground was included as an
additional, fixed background component during spectral fitting. Source
spectra were binned to 25 counts per energy channel. Fitting was
performed with \xspec\ 12.4 \citep{xspec} using an absorbed,
single-component \mekal\ model \citep{mekal1} over the energy range
0.7-7.0 \keV. The absorbing Galactic column density was set to $\nhgal
= 2.28 \times 10^{20} ~\pcmsq$ \citep{lab} and assumed not to vary
over the cluster. Gas abundance was free to vary and normalized to the
solar ratios of \citet{ag89}. We discuss projected quantities only as
spectral deprojection using the {\textsc{proj}} \xspec\ model did not
produce significantly different results.

A 0.7-2.0 keV surface brightness (SB) profile was extracted from the
mosaiced, clean image using concentric $1\arcs$ wide elliptical annuli
centered on the BCG X-ray point source (central $\approx 1\arcs$ and
cavities excluded). The SB profile was fitted with a $\beta$-model
\citep{betamodel}, giving best-fit parameters $S_0 = 1.65 ~(\pm 0.15)
\times 10^{-3}$ \sbr, $\beta = 0.62 \pm 0.04$, and $r_{\rm{core}} =
7.98\arcs \pm 0.08$ for \chisq(DOF) = 79(97). A deprojected electron
density (\nelec) profile was derived from the SB profile using the
method of \citet{kriss83} which incorporates the 0.7-2.0 keV count
rates and best-fit normalizations from the $\tx(r)$ analysis. The
spectral quantities were transferred from the coarser $\tx(r)$ bins to
the finer SB$(r)$ bins via linear interpolation. Errors for
$\nelec(r)$ were estimated using 5000 Monte Carlo simulations of the
original SB profile. A total gas pressure ($P$) profile was calculated
as $P = n \tx$ where $n \approx 2.3 \nH$ and $\nH \approx \nelec/1.2$.

There is a significant \nelec\ discontinuity at $r \approx 60$ kpc
(insignificant $P$ and \tx\ discontinuities) which is coincident with
an elliptical ridge of X-ray emission enclosing the cavities. During
the formation of a cavity, bright cavity rims can result from the
displacement of lower entropy gas deep within in the core
\citep[\eg][]{2009ApJ...697L..95B} or gas shocking
\citep[\eg][]{ms0735}. After extracting spectra on either side of the
ridge using regions with more source counts than those of the
\tx\ profile, no significant \tx\ jump was detected to further signal
the presence of a shock. Assuming there is an unresolved shock in the
\tx\ profile, the Rankine-Hugoniot density jump relation predicts a
Mach number of $\approx 1.3$. In Section \ref{sec:cavities} we discuss
the impact of the bright rims on assessment of the cavity depths.

Profiles of entropy, $K = \tx \nelec^{-2/3}$, and cooling time,
$\tcool = 3 n \tx~[2 \nelec \nH \Lambda(T,Z)]^{-1}$ where
$\Lambda(T,Z)$ is the cooling function, were also generated. Errors
for each profile were determined by summing the individual parameter
uncertainties in quadrature. Fitting the $K$ profile with the function
$K = \kna +\khun (r/100 ~\kpc)^{\alpha}$ \citep{accept} yielded
best-fit parameters of $\kna = 17.9 \pm 2.2 ~\ent$, $\khun = 92.1 \pm
6.2 ~\ent$, and $\alpha = 1.65 \pm 0.11$. \rxj\ has a central cooling
time $< 0.5$ Gyr and $\kna < 30 ~\ent$, in general, clusters with
these core properties host line emitting BCGs, some of which have
nuclear star formation \citep{crawford99, haradent,
  rafferty08}. Partially consistent with this picture, we show in
Section \ref{sec:bcg} that wisps of infrared emission, which could be
\halpha\ filaments, radiate from \rbs, but that plausible evidence of
star formation is predominately found in areas of the BCG halo
directly interacting with the central AGN.

%%%%%%%%%%%%%%%%%%%%%%%%%%%%%%%%%%%%%%%%%%%%%%
\subsection{ICM Cavities and the AGN Outburst}
\label{sec:cavities}
%%%%%%%%%%%%%%%%%%%%%%%%%%%%%%%%%%%%%%%%%%%%%%

To provide guidance for this section, a montage of radio, optical, and
X-ray images is presented in Figure \ref{fig:composite}. Typically,
the connection between cavities, coincident radio emission, and an AGN
is unambiguous, but this is not the case for \rxj. As seen in
projection, the nuclear 4.8 GHz jets are almost orthogonal to the
cavity axis, and the 325 MHz, 1.4 GHz, and 8.4 GHz radio emission are
more that of a radio mini-halo than relativistic plasma confined to
the cavities (Doria et al., in preparation). To better reveal the
cavity morphologies and aide in disentangling the radio-X-ray
relationship, residual X-ray images were constructed.

The X-ray isophotes of two exposure-corrected images -- one smoothed
by a $1\arcs$ Gaussian and another by a $3\arcs$ Gaussian -- were
fitted with ellipses using the \iraf\ task \textsc{ellipse}. The
ellipse centers were fixed at the location of the BCG X-ray point
source, and the eccentricities and position angles were free to
vary. A 2D SB model was created from each fit using the \iraf\ task
\textsc{bmodel}, normalized to the parent X-ray image, and then
subtracted off. The residual images are shown in Figure
\ref{fig:subxray}.

In addition to the central east and west cavities (labeled E1 and W1),
depressions north and south (labeled N1 and S1) of the nucleus are
revealed. N1 and S1 lie along the 4.8 GHz jet axis and are coincident
with spurs of significant 1.4 GHz emission, indicating they may be
related to activity of the central AGN. A depression (labeled E2)
coincident with the southeastern concentration of 325 MHz emission is
also found, but no counterpart (labeled W2) on the opposing side of
the cluster is seen. There is an X-ray edge which extends southwest
from E2 and sits along a ridge of 325 MHz and 8.4 GHz emission. No
substructure associated with the western-most knot of 325 MHz emission
is found. However, there is a stellar object co-spatial with this
region, and the X-ray and radio properties -- for $D < 1$ kpc, $\lx
\la 10^{31} ~\lum$ and $L_{325} \sim 10^{27} ~\lum$ -- are consistent
with the presence of a galactic RS CVn star
\citep{1993RPPh...56.1145S}, suggesting the western 325 MHz emission
may not be associated with the cluster.

The cavity depths were quantified using decrements ($y$) defined as
the ratio of the cavity SB to the value of the best-fit ICM
$\beta$-model at the same radius. For a circular region with $r =
1\arcs$ centered on the deepest part of each cavity, the mean
decrements are $\bar{y}_{\rm{W1}} = 0.50 \pm 0.18$ and
$\bar{y}_{\rm{E1}} = 0.52 \pm 0.23$, with minima of
$y^{\rm{min}}_{\rm{W1}} = 0.44$ and $y^{\rm{min}}_{\rm{E1}} =
0.47$. Most other cavity systems have minima $> 0.6$ (B04),
highlighting the rarity and extreme nature of the pair in
\rxj. Because the $\beta$-model is symmetric about the POS, some part
of each cavity must cross this plane for $y$ to be $< 50\%$. To
determine if roughly spherical E1 and W1 geometries could produce the
measured decrements, the method described in \citet{hydraa} to study
the cavities of Hydra A was employed. In short, the best-fit ICM
$\beta$-model was integrated along the LOS over the center of each
cavity assuming that the cavities are symmetric about the POS. The LOS
cavity radius, \rlos, was assumed to equal the projected cavity
effective radius, $\reff = \sqrt{ab}$, where $a$ and $b$ are the
measured semi-major and semi-minor axes, respectively. With these
assumptions, we find decrements $< 0.67$ cannot be achieved for either
E1 or W1. This result demands some imperfection in the decrement
model.

One obvious fix is to assume $\rlos \ne \reff$. The measured
decrements {\it{can}} be reproduced when $\rlos = 21.1$ kpc for E1 and
$\rlos = 25.3$ kpc for W1, which suggests the cavities are two times
larger along the LOS than in the POS. If we were looking almost down
the axis of a highly elongated radio lobe cavity inflating along the
LOS, it would be unlikely to cross the POS, hence that case is a
little awkward. How highly flattened bubbles could be created so close
to the core is also unclear.

Another possibility is that the $\beta$-model is a poor representation
of the undisturbed ICM. In Section \ref{sec:icm} we highlighted the
presence of bright rims surrounding E1-W1. The rims -- which are
several arcseconds wide, azimuthally symmetric, and reside near
$r_{\rm{core}}$ -- may have influenced the $\beta$-model fit,
resulting in artificially low minima. Excluding the rims from the
$\beta$-model fitting did not provide insight because too much of the
SB profile was removed and the fit did not converge. Extrapolating the
SB profile at larger radii inward resulted in even lower decrements,
exacerbating the problem.

Alternatively, deep decrements can be achieved with large, overlapping
cavities, a scenario we now consider in more detail. Shown in Figure
\ref{fig:proj} are snapshots along three lines of sight of ICM X-ray
emission from a hydrodynamical simulation of cavity formation in a
cluster with a $\beta$-model atmosphere. The simulation was performed
with version 3.0 of the {\textsc{FLASH}} hydrodynamics code
\citep{flash}. The refinement criteria were the standard density and
pressure, with a minimum cell size of 0.66 kpc on a grid of $1024^3$
zones. In order to produce bubbles, the simulation was started by
injecting energy into two spheres of radius 4.5 kpc at distances of 13
kpc from the cluster center. The gas inside these spheres was heated
and expanded similar to a Sedov explosion to form a pair of bubbles in
a few Myr, a time much shorter than the rise time of the generated
bubbles. The parameters were chosen such that these regions reached a
radius of 12 kpc and a density contrast of approximately 0.05 as
compared to the surrounding ICM \citep[see][]{2009arXiv0909.1805S}. In
order to allow a direct comparison with the X-ray data, the gridded
simulation output was post-processed with the X-ray-imaging pipeline
{\textsc{XIM}}\footnote{http://www.astro.wisc.edu/~heinzs/XIM}
\citep{2009arXiv0903.0043H}.

The overlapping cavity configuration shown in the upper left panel of
Figure \ref{fig:proj} is appealing for several reasons: 1) overlapping
cavities close to the core can produce $y < 50\%$, 2) a bright,
azimuthally symmetric rim naturally occurs from the conjunction of the
cavities along the LOS, 3) there are faint depressions at the leading
edge of each cavity similar to E2, and 4) the superposition of radio
plasma within the cavities may explain the ambiguous \rxj\ radio/X-ray
relationship. There are difficulties with the overlapping cavity
scenario, such as the lack of a depression at W2, that the core does
not resemble a bullseye, and the POS cavity configuration satisfies
Occam's razor. But, the simulations are only suggestive of how
overlapping cavities may appear. Real cavities are known to entrain
denser material from the core, deform as they buoyantly rise, and the
region surrounding the jets is normally not so low-density. Also,
asymmetries of the large-scale and core gas distributions may explain
the lack of features such as the W2 depression and the `S'-shaped
X-ray emission near the nucleus. The simulations were convincing
enough that we divided the cavity analysis into two configurations: a
pair of small cavities in the POS (E1/2) and a pair of large cavities
along the LOS (O1/2; shown in Figure \ref{fig:proj}).

For the POS case, two different cavity semi-polar axes, $c$, are
considered: $c = \reff$ and $c = \rlos$ taken from the decrement
analysis. In both cases, the distance of each cavity from the central
AGN, $D$, was set to the projected distance from the ellipse centers
to the BCG X-ray point source. For the LOS case, the cavity distances
and volumes are poorly constrained, so we selected a simple
interpretation and set $c = \reff$ and $D = a$. We also give
calculations for $D = 2a$. For all cavities, a systematic error of
10\% is assigned to the geometries. Cavity ages were estimated using
the three time scales given in B04: ICM sound speed age (\tsonic),
buoyant rise time age (\tbuoy), and volume refilling age
(\trefill). The energy in each cavity, $\ecav = \gamma PV/(\gamma-1)$,
was calculated assuming the contents are a relativistic plasma
($\gamma = 4/3$), and that the mean internal cavity pressure is equal
to the ICM pressure at $r = D$. The time-averaged energy needed to
create each cavity was calculated as $\pcav = \ecav/\tbuoy$. The
cavity properties are listed in Table \ref{tab:cavities}.

Our POS E1 and W1 volumes are larger than those of B04, thus our
\ecav\ and \pcav\ are larger by a factor of $\approx 2$. Inserting the
B04 cavity volumes into our calculations produces no significant
differences, indicating our analysis is consistent with B04. The POS
and LOS configurations indicate aggregate cavity energy of $\ecav
\approx 3 \dash 73 \times 10^{60}$ erg and power output of $\pcav
\approx 3 \dash 38 \times 10^{45} ~\lum$. The energetics for the LOS
case are extreme, and possibly upper limits, but they are informative
because of the strain they place on how the outburst was powered (see
Section \ref{sec:accretion}). However, assuming the LOS upper limits
are accurate, they suggest the outburst may be one of the most
powerful found to date, on par with Hercules A and \ms. The mechanical
power output of the \rbs\ AGN is similar to the radiative power output
of typical quasar, and exceeds the $\approx 3 \times 10^{45} ~\lum$ of
energy radiated away by ICM gas with $\tcool < H_{z=0.35}^{-1}$.

%%%%%%%%%%%%%%%%%%%%%%%%%%%%%%%%%%%%%%
\subsection{Powering the AGN Outburst}
\label{sec:accretion}
%%%%%%%%%%%%%%%%%%%%%%%%%%%%%%%%%%%%%%

Assuming \ecav\ is representative of the gravitational binding energy
released by mass accreted onto the SMBH, then the total mass accreted
and its rate of consumption were $\macc = \ecav/(\epsilon c^2)$ and
$\dmacc = \macc/\tbuoy$, respectively. Here, $\epsilon$ is a poorly
understood mass-energy conversion factor assumed to be 0.1, which
gives $\macc \approx 4 \times 10^8 ~\msol$ and $\dmacc \approx 7
~\msolpy$. Below, we consider if the accretion of cold gas or hot gas
pervading the BCG could meet these requirements.

Substantial reservoirs of molecular gas and gaseous filaments are
found in many cool core clusters \citep{crawford99, edge01}, and via
the formation of thermal instabilities in these structures, it is
possible that they provide fuel for AGN activity and star formation
\citep[\eg][]{pizzolato05, 2010arXiv1003.4181P}. The molecular gas
mass (\mmol) of \rbs\ has not been measured, so the
\halpha-\mmol\ correlation presented by \citet{edge01} was used to
estimate \mmol, with the caveat that this correlation has substantial
scatter \citep{salome03}. Using the scaled \hbeta\ line as a surrogate
for \halpha, we estimate $\mmol \sim 10^{10} ~\msol$, sufficiently in
excess of \macc\ that we can infer an ample supply of cold gas exists
to have powered the outburst.

However, SMBHs do not grow independent of the host galaxy, and several
hundred times as much gas should go into stars as is accreted by a
SMBH \citep{1995ARA&A..33..581K, magorrian}. In which case, the
\rbs\ star formation rate (SFR) appears to be suspiciously low, $\psi
\ll 100 ~\msolpy$ (see Section \ref{sec:bcg}), when compared with
\dmacc. Assuming the present SFR is representative of the SFR
preceding and during the AGN outburst, the implication is that for
every mass unit accreted by the SMBH, less than 10 units went into
stars, a ratio well below what is expected from SMBH-host galaxy
co-evolution. Conversely, if the SFR and/or \dmacc\ estimates are
inaccurate, and the SFR during the outburst was commensurate with the
Magorrian relation, \rbs\ should have a strong blue color gradient
indicating newly formed stars \citep{rafferty08}, which it does not
(see Section \ref{sec:bcg}). Lacking better constraints on \mmol\ and
$\psi$, we can only conclude that if cold gas accretion did power the
outburst, it appears the SMBH gas consumption was highly efficient,
bordering on unphysical.

Direct accretion of the hot ICM via the Bondi mechanism provides
another possible AGN fuel source. The accretion flow arising from this
process is characterized via the Bondi equation (Equation
\ref{eqn:bon}) and is often compared with the Eddington limit
describing the maximal gas inflow rate (Equation \ref{eqn:edd}):
\begin{eqnarray}
  \dmbon &=& 0.013 ~\kbon^{-3/2} \left(\frac{\mbh}{10^9
    ~\msol}\right)^{2} ~\msolpy \label{eqn:bon}\\
  \dmedd &=& \frac{2.2}{\epsilon} \left(\frac{\mbh}{10^9~\msol}\right)
  ~\msolpy  \label{eqn:edd}
\end{eqnarray}
where $\epsilon$ is a mass-energy conversion factor, \kbon\ is the
mean entropy in \ent\ of gas within the Bondi radius, and \mbh\ is
black hole mass in \msol. We chose the relations of
\citet{2002ApJ...574..740T} and \citet{2007MNRAS.379..711G} to
estimate \mbh, and find a range of $0.7 \dash 7.0 \times 10^9 ~\msol$,
from which we adopted the weighted mean value $1.5 ~(\pm 0.2) \times
10^9 ~\msol$.

For $\epsilon = 0.1$ and $\kbon = \kna$, the relevant accretion rates
are $\dmbon \approx 4 \times 10^{-4} ~\msolpy$ and $\dmedd \approx 33
~\msolpy$. The Eddington and Bondi accretion ratios for the outburst
event were thus $\dme \equiv \dmacc/\dmedd \approx 0.25$ and $\dmb
\equiv \dmacc/\dmbon \approx 30000$. The large \dmb\ is disconcerting
as it implies all the gas which reached the Bondi radius ($\rbon
\approx 10$ pc) was accreted. But, \dmb\ may be overestimated if
$\kbon < \kna$ and \mbh\ is larger than our adopted value. If $\mbh =
7 \times 10^9 ~\msol$ and gas near \rbon\ has a mean temperature of
0.1 keV, for \dmb\ to approach unity, \kbon\ must be $< 0.2 ~\ent$,
corresponding to $\nelec \sim 0.4 ~\pcc$. This is four times the
current ICM central density, and a sphere $\approx 1$ kpc in radius is
required for $> 10^8 ~\msol$ to be available for accretion. But this
implies the inner-core of the cluster is fully consumed in $< 0.1$
Gyr, a time scale much shorter than \tcool\ of the overlying ICM,
short-circuiting the feedback loop. While Bondi accretion cannot be
ruled out because the cluster is observed post-outburst and the core
conditions may have changed dramatically since, a number of
observationally unsupported concessions must be made for Bondi
accretion to be viable.

The nuclear BCG X-ray source may also reveal valuable information
about on-going accretion. A spectrum for this source was extracted
from the region enclosing 90\% of the \cxo\ point-spread function, and
a background spectrum was taken from an annulus immediately outside
this region with 5 times the area. The background-subtracted spectrum
is inconsistent with thermal emission and was instead modeled using an
absorbed power law with two Gaussians to account for features which
may be blends of photoionized lines. The best-fit models are given in
Table \ref{tab:agn} and the spectrum is shown in Figure
\ref{fig:nucspec}. The model with a power-law distribution of $\nhobs
\sim 10^{22} ~\pcmsq$ absorbers is preferred, indicating a lack of
significant obscuration.

If mass accretion is powering the nuclear X-ray emission, the
accretion rate required is $\dmacc \approx \lbol/(0.1 c^2) \approx
0.04 ~\msolpy \approx 0.001 \dme$. The accretion disk model of
\citet{2002NewAR..46..247M} predicts an optical luminosity of $\ldisk
\approx 8 \times 10^{44} ~\lum$ for this \dmacc, but no $L > 10^{44}
~\lum$ sources are found in the \hst\ imaging. Rather, extrapolation
of the best-fit spectral model to radio frequencies showed good
agreement with the measured 1.4 GHz and 4.8 GHz nuclear radio fluxes,
and the continuous injection synchrotron model of
\citet{1987MNRAS.225..335H} produced an acceptable fit to the X-ray,
1.4 GHz, and 4.8 GHz nuclear fluxes. These latter results suggest the
nuclear emission may be from a synchrotron source, \eg\ unresolved
jets, and not the remnants of a very dense, hot gas phase which would
be associated with Bondi accretion.

Powering the \rbs\ outburst via mass accretion alone appears to be
laden with difficulties, but, like \ms\ \citep{msspin}, a rapidly
spinning SMBH potentially possesses enough energy and power to be an
interesting solution. The spin model of \citet[][GES
  hereafter]{gesspin} suggests jets are produced via a combination of
the Blandford-Znajek \citep[BZ;][]{bz} and Blandford-Payne
\citep[BP;][]{1982MNRAS.199..883B} mechanisms. Unlike pure mass
accretion, a modest mass accretion rate, \eg\ $\ll 5 ~\msolpy$, is
required to extract energy from the SMBH
\citep{1999ApJ...522..753M}. The GES model predicts jet power scales
with spin as
\begin{equation}
  P_{\rm{jet}} = 2 \times 10^{47} ~{\lum} ~\alpha(j) ~\beta^2(j) ~j^2
  B_{d,5}^2 \left(\frac{\mbh}{10^9 ~\msol}\right)^2
  \label{eqn:ges}
\end{equation}
where $j$ is a dimensionless spin parameter between -1 and 1,
$\alpha(j)$ and $\beta(j)$ are $j$-dependent functions capturing the
BZ and BP powers \citep[see][]{2009ApJ...699L..52G}, and $B_{d,5}$ is
the strength of the magnetic field threading the accretion disk with
units $10^5$ G. The GES model has the appealing feature that the most
powerful jets are produced for a SMBH which is spinning retrograde
relative to the direction of the accreting material, easing the
demands on the magnitude of $B_{d,5}$. In the model of
\citet{1999ApJ...522..753M}, the energy of a spinning SMBH is
\begin{equation}
  E_{\rm{spin}} = 1.6 \times 10^{62} ~{\erg} ~j^2
  \left(\frac{\mbh}{10^9 ~\msol}\right).
\end{equation}
Assuming all the spin energy of the \rbs\ SMBH was extracted in the
outburst, our upper limit of $\ecav = 7.3 \times 10^{61}$ erg gives $j
\approx 0.55$. Plugging $\pm j$ into Equation \ref{eqn:ges} requires
$B_{d,5}(-j) \approx 1400$ G and $B_{d,5}(+j) \approx 7400$ G,
reasonable values for a magnetic field connecting a spinning SMBH and
the surrounding accretion disk
\citep[\eg][]{2002Sci...295.1688K}. There are clearly uncertainties
associated with these estimates, and the spin power mechanism is
wrought with its own difficulties \citep[see][for discussion]{msspin}.
However, no highly efficient or optimistic scenarios need to play out
for SMBH spin to easily achieve the extreme energies and jet powers
demanded for \rbs, making it a credible alternative.

%%%%%%%%%%%%%%%%%%%%%%%%%%%%%%%%%%%%%%%%%%%%%%%%%%%%%%%%%%%%
\subsection{BCG Star Formation and Interaction with the AGN}
\label{sec:bcg}
%%%%%%%%%%%%%%%%%%%%%%%%%%%%%%%%%%%%%%%%%%%%%%%%%%%%%%%%%%%%

An estimate of the \rbs\ SFR is relevant given the general
relationship with the process of SMBH growth via mass accretion. The
SFRs below are likely upper limits since 1) the extremely blue
\galex\ and extremely red 2MASS colors signal the presence of an AGN,
and 2) sources of significant SFR uncertainties have been neglected
\citep[\eg][]{1992ApJ...388..310K, 2004AJ....127.2002K, hicksuv,
  2010MNRAS.tmp..626G}. Using the relations of \citet{kennicutt2},
\citet{2006ApJ...642..775M}, and \citet{salim2007}, the \rbs\ SFR may
lie in the range 1--10 ~\msolpy\ (see Table \ref{tab:sfr}). Though the
shallow 4000 \AA\ break and optical emission line ratios of the BTA
spectrum indicate it arises from a star forming source, the large
uncertainties and unknown slit position mask its strength and
location. Further, the radially-averaged \hst\ $\myv-\myi$ color
profile does not have a blue gradient, suggesting if there is star
formation, it may be confined to compact regions like the optical
substructure seen in the \hst\ images.

To better reveal the BCG substructure, residual galaxy images were
constructed by first fitting the \hst\ \myv\ and \myi\ isophotes with
ellipses using the \iraf\ tool {\textsc{ellipse}}. Stars and other
contaminating sources were rejected using a combination of $3\sigma$
clipping and by-eye masking. The ellipse centers were fixed at the
galaxy centroid, and ellipticity and position angle were fixed at
$0.25 \pm 0.02$ and $-64\mydeg \pm 2\mydeg$, respectively -- the mean
values when they were free parameters. Galaxy light models were
created using {\textsc{bmodel}} in \iraf\ and subtracted from the
corresponding parent image, leaving the residual images shown in
Figure \ref{fig:subopt}. A color map was also generated by subtracting
the fluxed \myi\ image from the fluxed \myv\ image.

The close alignment of the optical substructure with the nuclear AGN
outflow clearly indicates the jets are interacting with the BCG
halo. There also appears to be an in-falling galaxy to the NW of the
nucleus, possibly with a stripped tail
\citep[\eg][]{2007ApJ...671..190S}. The numbered regions overlaid on
the residual \myv\ image are the areas of the color map which have the
largest color difference with surrounding galaxy light. Regions 1--5
are relatively the bluest with $m_{\myv-\myi} = -0.40, -0.30, -0.25,
-0.22, ~\rm{and} -0.20$, respectively. Without spectroscopy, it is
unclear which, if any, of the regions host significant star formation
or strong line emission. Regions 6--8 are relatively the reddest with
$m_{\myv-\myi} =$ +0.10, +0.15, and +0.18, respectively. Again,
without spectroscopy, we can only speculate that these may be emission
line regions (\eg\ \halpha) or regions reddened by dust extinction.

In the residual \myi\ image, regions 9--11 denote what appear to be
8--10 kpc long ``whiskers'' of emission radiating out from the
BCG. The whiskers may be optical filaments similar to those seen
around other BCGs, \eg\ NGC 1275 \citep{2003MNRAS.344L..48F}, and the
low-entropy state of the cluster core is conducive to the formation of
such structures \citep{conduction}. It is interesting that regions 1
and 4 reside at the point where the southern jet appears to be
encountering whiskers 9 and 10. It may be that the jet has entrained
and compressed gas in the whiskers to the point of forming stars. If
the AGN is driving star formation, then the estimated SFRs may be
boosted above the putative rate, and thus are not representative of
prior episodes of gas fragmentation in the BCG. If this is the case,
then the discrepancy between the mass accreted by the SMBH and the
mass which went into stars only increases (discussed in Section
\ref{sec:accretion}), making cold gas accretion an even less likely
scenario for fueling the AGN.

%%%%%%%%%%%%%%%%%%%%%
\section{Conclusions}
\label{sec:con}
%%%%%%%%%%%%%%%%%%%%%

We have presented results from a study of the galaxy cluster
\rxj\ with specific focus on the cavity system surrounding the BCG
\rbs. \cxo\ observations have enabled us to constrain the energetics
of the AGN outburst and analyze different mechanisms which may have
powered it. We have shown the following:
\begin{enumerate}
\item \rxj\ is a cool core object with a central cooling time $< 0.5$
  Gyr, core entropy $< 30 ~\ent$, and bolometric cooling luminosity of
  $\sim 10^{45} ~\lum$. Bright rims surrounding the prominent cavity
  system $\approx 25$ kpc from the BCG nucleus indicate the AGN
  outburst has either shocked the ICM ($M \approx 1.3$) or uplifted
  cool, dense gas from the core.
\item The two central cavities have decrements consistent with voids
  that cross the plane of the sky, and residual X-ray images reveal
  additional SB depressions likely associated with AGN activity. We
  have placed constraints on the AGN outburst energetics by
  considering two limiting cases: small cavities with centers in the
  plane of the sky, and large overlapping cavities along the line of
  sight. The total energy in the cavities is $3-70 \times 10^{60}$
  erg, with powers $3-40 \times 10^{45} ~\lum$. If the upper limits
  are accurate, the AGN outburst has released enough energy to
  suppress cooling of the cluster halo.
\item The energetics demand that cold-mode and hot-mode gas accretion
  operate with unrealistic efficiency if mass accretion alone powered
  the outburst. We specifically show that Bondi accretion is an
  unattractive solution, and instead suggest that the outburst was
  powered by tapping the energy of SMBH spin. We show that the model
  of \citet{gesspin} can readily achieve our \ecav\ and \pcav\ limits.
\item The optical substructure of the BCG clearly indicates
  interaction of the AGN outflow with the galaxy halo. We are unable
  to determine if any regions of interest host star formation or line
  emission. But, the convergence of what appear to be optical
  filaments, bluish knots of emission, and the tip of one jet suggest
  there may be AGN driven star formation.
\end{enumerate}

Conservation of angular momentum suggests that gas condensing into the
very center of a BCG has a preferred direction of rotation, possibly
aligned with the semi-major axis of the cluster. The prevalence of
systems like \rbs, \ms, and Hercules A is unclear, but their numbers
thus far are small, possibly because their BCGs, for whatever reason,
formed SMBHs that at some point in the past spun retrograde relative
to the preferred rotation direction of the cluster. In which case, if
the GES model is assumed, they would undergo a very powerful AGN
outburst which releases $\ga 10^{61}$ erg of spin energy. If the
circumstances which result in retrograde SMBHs are rare, then this
naturally explains why not many such systems have been
found.

%%%%%%%%%%%%%%%%%
\acknowledgements
%%%%%%%%%%%%%%%%%

KWC and BRM were supported by CXO grant G07-8122X and a grant from the
Natural Science and Engineering Research Council of Canada. KWC thanks
David Gilbanks, Sabine Schindlesr, Axel Schwope, and Chris Z. Waters
for helpful input.

%%%%%%%%%%%%%%
% Facilities %
%%%%%%%%%%%%%%

{\it Facilities:} \facility{CXO (ACIS)} \facility{HST (WFPC2)}
\facility{GALEX} \facility{BTA (SP124)}

%%%%%%%%%%%%%%%%
% Bibliography %
%%%%%%%%%%%%%%%%

\bibliography{cavagnolo}

%%%%%%%%%%%%%%%%%%%%%%
% Figures and Tables %
%%%%%%%%%%%%%%%%%%%%%%

\clearpage
\begin{deluxetable}{ccccccccccc}
  \rotate
  \tablecolumns{11}
  \tablewidth{0pc}
  \tabletypesize{\small}
  \tablecaption{Cavity Properties.\label{tab:cavities}}
  \tablehead{
    \colhead{Config.} & \colhead{ID} & \colhead{$a$} & \colhead{$b$} & \colhead{$\rlos$} & \colhead{$D$} & \colhead{\tsonic} & \colhead{\tbuoy} & \colhead{\trefill} & \colhead{\ecav} & \colhead{\pcav}\\
    \colhead{-} & \colhead{-} & \colhead{kpc} & \colhead{kpc} & \colhead{kpc} & \colhead{kpc} & \colhead{Myr} & \colhead{Myr} & \colhead{Myr} & \colhead{$10^{60}$ erg} & \colhead{$10^{45}$ erg s$^{-1}$}\\
    \colhead{(1)} & \colhead{(2)} & \colhead{(3)} & \colhead{(4)} & \colhead{(5)} & \colhead{(6)} & \colhead{(7)} & \colhead{(8)} & \colhead{(9)} & \colhead{(10)} & \colhead{(11)}}
  \startdata
  1 & E1      & $17.3 \pm 1.7$ & $10.7 \pm 1.1$ & $13.6 \pm 1.4$ & $23.2 \pm 2.3$ & $20.1 \pm 3.1$ & $28.1 \pm 4.4$ & $70.3 \pm  9.9$ & $1.51 \pm 0.35$ & $1.70 \pm 0.48$\\
  1a & \nodata & \nodata        & \nodata        & $23.4 \pm 2.3$ & \nodata        & \nodata        & \nodata        & $76.9 \pm 10.9$ & $2.59 \pm 0.61$ & $2.92 \pm 0.83$\\
  1 & W1      & \nodata        & \nodata        & $13.6 \pm 1.4$ & $25.9 \pm 2.5$ & $20.3 \pm 4.1$ & $33.3 \pm 5.3$ & $74.4 \pm 10.5$ & $1.72 \pm 0.47$ & $1.64 \pm 0.52$\\
  1a & \nodata & \nodata        & \nodata        & $26.0 \pm 2.6$ & \nodata        & \nodata        & \nodata        & $82.8 \pm 11.7$ & $3.29 \pm 0.89$ & $3.13 \pm 0.99$
  \enddata
  \tablecomments{
    Col. (1) Cavity configuration;
    Col. (2) Cavity identification;
    Col. (3) Semi-major axis;
    Col. (4) Semi-minor axis;
    Col. (5) Line-of-sight cavity radius;
    Col. (6) Distance from central AGN;
    Col. (7) Sound speed age;
    Col. (8) Buoyancy age;
    Col. (9) Refill age;
    Col. (10) Cavity energy;
    Col. (11) Cavity power using \tbuoy.}
\end{deluxetable}

\begin{deluxetable}{lcccccccccccc}
  \tablecolumns{13}
  \tablewidth{0pc}
  \tabletypesize{\footnotesize}
  \tablecaption{Nuclear X-ray Point Source Spectral Models.\label{tab:agn}}
  \tablehead{
    \colhead{Absorber} & \colhead{\nhabs} & \colhead{$\Gamma_{\rm{pl}}$} & \colhead{$\eta_{\rm{pl}}$} & \colhead{$E_{\rm{ga}}$} & \colhead{$\sigma_{\rm{ga}}$} & \colhead{$\eta_{\rm{ga}}$} & \colhead{Param.} & \colhead{$L_{0.7-7.0}$} & \colhead{\lbol} & \colhead{\chisq} & \colhead{DOF} & \colhead{Goodness}\\
    \colhead{-} & \colhead{$10^{22}~\pcmsq$} & \colhead{-} & \colhead{$10^{-5} \dagger$} & \colhead{keV} & \colhead{eV} & \colhead{$10^{-6} \ddagger$} & \colhead{-} & \colhead{$10^{44}~\lum$} & \colhead{$10^{44}~\lum$} & \colhead{-} & \colhead{-} & \colhead{-}\\
    \colhead{(1)} & \colhead{(2)} & \colhead{(3)} & \colhead{(4)} & \colhead{(5)} & \colhead{(6)} & \colhead{(7)} & \colhead{(8)} & \colhead{(9)} & \colhead{(10)} & \colhead{(11)} & \colhead{(12)} & \colhead{(13)}}
  \startdata
  None          & \nodata             & $0.1^{+0.3}_{-0.3}$ & $0.3^{+0.1}_{-0.1}$  & [2.4, 3.4] & [63, 119] & [8.4, 14.9] & \nodata                & $0.69^{+0.11}_{-0.22}$ & $49.9^{+18.1}_{-19.0}$ & 1.88 & 61 & 56\%\\
  Neutral$^a$   & $4.2^{+1.9}_{-1.3}$ & $1.5^{+0.4}_{-0.3}$ & $6.6^{+8.1}_{-3.6}$  & [1.8, 3.0] & [31, 58] & [0.6, 1.8]  & 0.354                  & $0.68^{+0.12}_{-0.24}$ & $5.65^{+2.11}_{-2.50}$ & 1.17 & 60 & 29\%\\
  Warm$^b$      & $3.3^{+1.4}_{-1.6}$ & $1.9^{+0.2}_{-0.2}$ & $16.2^{+0.4}_{-0.5}$ & [1.8, 2.9] & [57, 44] & [0.9, 1.5]  & $0.97^{+0.03}_{-0.03}$ & $0.70^{+0.18}_{-0.26}$ & $3.46^{+1.10}_{-0.95}$ & 1.01 & 59 & 13\%\\
  Power-law$^c$ & 0.5--7.5            & $2.1^{+0.5}_{-0.3}$ & $22.1^{+2.7}_{-1.1}$ & [1.8, 3.0] & [54, 35] & [0.9, 1.4]  & $0.63^{+0.34}_{-0.31}$ & $0.71^{+1.48}_{-1.32}$ & $2.21^{+0.45}_{-0.30}$ & 1.00 & 58 & $< 1$\%
  \vspace{0.5mm}
  \enddata
  \tablecomments{
    For all models, $\nhgal = 2.28 \times 10^{20} ~\pcmsq$. 
    Col. (1) \xspec\ absorber models: ($a$) is \textsc{zwabs}, ($b$) is \textsc{pcfabs}, ($c$) is \textsc{pwab};
    Col. (2) Absorbing column density;
    Col. (3) Power-law index;
    Col. (4) Power-law normalization with units ($\dagger$) ph keV$^{-1}$ cm$^{-2}$ s$^{-1}$ at 1 keV;
    Col. (5) Gaussian central energies;
    Col. (6) Gaussian widths;
    Col. (7) Gaussian normalizations with units ($\ddagger$) ph cm$^{-2}$ s$^{-1}$;
    Col. (8) Model-dependent parameter: ($a$) absorber redshift, ($b$) absorber covering fraction, ($c$) absorber power law index of covering fraction;
    Col. (9) Model 0.7-7.0 keV luminosity;
    Col. (10) Unabsorbed model bolometric (0.01-100.0 keV) luminosity;
    Col. (11) Reduced \chisq\ of best-fit model;
    Col. (12) Model degrees of freedom;
    Col. (13) Percent of 10,000 Monte Carlo realizations with \chisq\ less than best-fit \chisq.
}
\end{deluxetable}

\begin{deluxetable}{ccccc}
  \tablecolumns{5}
  \tablewidth{0pc}
  \tablecaption{BCG Star Formation Rates.\label{tab:sfr}}
  \tablehead{
    \colhead{Source} & \colhead{ID} & \colhead{$\xi$ [Ref.]} & \colhead{$L$} & \colhead{$\psi$}\\
    \colhead{-} & \colhead{-} & \colhead{(\msolpy)/(\lum ~\phz)} & \colhead{\lum ~\phz} & \colhead{\msolpy}\\
    \colhead{(1)} & \colhead{(2)} & \colhead{(3)} & \colhead{(4)} & \colhead{(5)}}
  \startdata
  \galex\            & NUV             & $1.4 \times 10^{-28}$ [1] & $2.5 ~(\pm 0.9) \times 10^{28}$ & $3.5 \pm 1.3$\\
  \xom\              & UVW1            & $1.1 \times 10^{-28}$ [2] & $6.2 ~(\pm 1.9) \times 10^{28}$ & $6.9 \pm 2.2$\\
  \galex\            & FUV             & $1.1 \times 10^{-28}$ [2] & $8.2 ~(\pm 2.0) \times 10^{28}$ & $9.0 \pm 2.3$\\
  \galex\            & FUV             & $1.4 \times 10^{-28}$ [1] & $8.2 ~(\pm 2.0) \times 10^{28}$ & $11 \pm 3$\\
  \xom\              & UVM2            & $1.1 \times 10^{-28}$ [2] & $< 5.0 \times 10^{29}$          & $< 55$
  \enddata
  \tablecomments{A dagger ($\dagger$) indicates the removal of
    Hz$^{-1}$ from the units of $\xi$ \& $L$.  Col. (1) Source of
    measurement; Col. (2) Diagnostic identification; Col. (3)
    Conversion coefficient and references: [1] \citet{kennicutt2}, [2]
    \citet{salim2007}; Col. (4) Luminosity; Col. (5) Star formation
    rate.}
\end{deluxetable}

\begin{figure}
  \begin{center}
    \begin{minipage}{\linewidth}
      \includegraphics*[width=\textwidth, trim=0mm 0mm 0mm 0mm, clip]{rbs797.ps}
    \end{minipage}
    \caption{Fluxed, unsmoothed 0.7--2.0 keV clean image of \rbs\ in
      units of ph \pcmsq\ \ps\ pix$^{-1}$. Image is $\approx 250$ kpc
      on a side and coordinates are J2000 epoch. Black contours in the
      nucleus are 2.5--9.0 keV X-ray emission of the nuclear point
      source; the outer contour approximately traces the 90\% enclosed
      energy fraction (EEF) of the \cxo\ point spread function. The
      dashed green ellipse is centered on the nuclear point source,
      encloses both cavities, and was drawn by-eye to pass through the
      X-ray ridge/rims.}
    \label{fig:img}
  \end{center}
\end{figure}

\begin{figure}
  \begin{center}
    \begin{minipage}{0.495\linewidth}
      \includegraphics*[width=\textwidth, trim=0mm 0mm 0mm 0mm, clip]{325.ps}
    \end{minipage}
   \begin{minipage}{0.495\linewidth}
      \includegraphics*[width=\textwidth, trim=0mm 0mm 0mm 0mm, clip]{8.4.ps}
   \end{minipage}
   \begin{minipage}{0.495\linewidth}
      \includegraphics*[width=\textwidth, trim=0mm 0mm 0mm 0mm, clip]{1.4.ps}
    \end{minipage}
    \begin{minipage}{0.495\linewidth}
      \includegraphics*[width=\textwidth, trim=0mm 0mm 0mm 0mm, clip]{4.8.ps}
    \end{minipage}
     \caption{Radio images of \rbs\ overlaid with black contours
       tracing ICM X-ray emission. Images are in mJy beam$^{-1}$ with
       intensity beginning at $3\sigma_{\rm{rms}}$ and ending at the
       peak flux, and are arranged by decreasing size of the
       significant, projected radio structure. X-ray contours are from
       $2.3 \times 10^{-6}$ to $1.3 \times 10^{-7}$ ph
       \pcmsq\ \ps\ pix$^{-1}$ in 12 square-root steps. {\it{Clockwise
           from top left}}: 325 MHz \vla\ A-array, 8.4 GHz
       \vla\ D-array, 4.8 GHz \vla\ A-array, and 1.4 GHz
       \vla\ A-array.}
    \label{fig:composite}
  \end{center}
\end{figure}

\begin{figure}
  \begin{center}
    \begin{minipage}{0.495\linewidth}
      \includegraphics*[width=\textwidth, trim=0mm 0mm 0mm 0mm, clip]{sub_inner.ps}
    \end{minipage}
    \begin{minipage}{0.495\linewidth}
      \includegraphics*[width=\textwidth, trim=0mm 0mm 0mm 0mm, clip]{sub_outer.ps}
    \end{minipage}
    \caption{Red text point-out regions of interest discussed in
      Section \ref{sec:cavities}. {\it{Left:}} Residual 0.3-10.0 keV
      X-ray image smoothed with $1\arcs$ Gaussian. Yellow contours are
      1.4 GHz emission (\vla\ A-array), orange contours are 4.8 GHz
      emission (\vla\ A-array), orange vector is 4.8 GHz jet axis, and
      red ellipses outline definite cavities. {\it{Bottom:}} Residual
      0.3-10.0 keV X-ray image smoothed with $3\arcs$ Gaussian. Green
      contours are 325 MHz emission (\vla\ A-array), blue contours are
      8.4 GHz emission (\vla\ D-array), and orange vector is 4.8 GHz
      jet axis.}
    \label{fig:subxray}
  \end{center}
\end{figure}

\begin{figure}
  \begin{center}
    \begin{minipage}{\linewidth}
      \includegraphics*[width=\textwidth]{r797_nhfro.eps}
      \caption{Gallery of radial ICM profiles. Vertical black dashed
        lines mark the approximate end-points of both
        cavities. Horizontal dashed line on cooling time profile marks
        age of the Universe at redshift of \rbs. For X-ray luminosity
        profile, dashed line marks \lcool, and dashed-dotted line
        marks \pcav.}
      \label{fig:gallery}
    \end{minipage}
  \end{center}
\end{figure}

\begin{figure}
  \begin{center}
    \begin{minipage}{\linewidth}
      \setlength\fboxsep{0pt}
      \setlength\fboxrule{0.5pt}
      \fbox{\includegraphics*[width=\textwidth]{cav_config.eps}}
    \end{minipage}
    \caption{Cartoon of possible cavity configurations. Arrows denote
      direction of AGN outflow, ellipses outline cavities, \rlos\ is
      line-of-sight cavity depth, and $z$ is the height of a cavity's
      center above the plane of the sky. {\it{Left:}} Cavities which
      are symmetric about the plane of the sky, have $z=0$, and are
      inflating perpendicular to the line-of-sight. {\it{Right:}}
      Cavities which are larger than left panel, have non-zero $z$,
      and are inflating along an axis close to our line-of-sight.}
    \label{fig:config}
  \end{center}
\end{figure}

\begin{figure}
  \begin{center}
    \begin{minipage}{0.495\linewidth}
      \includegraphics*[width=\textwidth, trim=25mm 0mm 40mm 10mm, clip]{edec.eps}
    \end{minipage}
    \begin{minipage}{0.495\linewidth}
      \includegraphics*[width=\textwidth, trim=25mm 0mm 40mm 10mm, clip]{wdec.eps}
    \end{minipage}
    \caption{Surface brightness decrement as a function of height
      above the plane of the sky for a variety of cavity radii. Each
      curve is labeled with the corresponding \rlos. The curves
      furthest to the left are for the minimum \rlos\ needed to
      reproduce $y_{\rm{min}}$, \ie\ the case of $z = 0$, and the
      horizontal dashed line denotes the minimum decrement for each
      cavity. {\it{Left}} Cavity E1; {\it{Right}} Cavity W1.}
    \label{fig:decs}
  \end{center}
\end{figure}


\begin{figure}
  \begin{center}
    \begin{minipage}{\linewidth}
      \includegraphics*[width=\textwidth, trim=15mm 5mm 5mm 10mm, clip]{pannorm.eps}
      \caption{Histograms of normalized surface brightness variation
        in wedges of a $2.5\arcs$ wide annulus centered on the X-ray
        peak and passing through the cavity midpoints. {\it{Left:}}
        $36\mydeg$ wedges; {\it{Middle:}} $14.4\mydeg$ wedges;
        {\it{Right:}} $7.2\mydeg$ wedges. The depth of the cavities
        and prominence of the rims can be clearly seen in this plot.}
      \label{fig:pannorm}
    \end{minipage}
  \end{center}
\end{figure}

\begin{figure}
  \begin{center}
    \begin{minipage}{0.5\linewidth}
      \includegraphics*[width=\textwidth, angle=-90]{nucspec.ps}
    \end{minipage}
    \caption{X-ray spectrum of nuclear point source. Black denotes
      year 2000 \cxo\ data (points) and best-fit model (line), and red
      denotes year 2007 \cxo\ data (points) and best-fit model (line).
      The residuals of the fit for both datasets are given below.}
    \label{fig:nucspec}
  \end{center}
\end{figure}

\begin{figure}
  \begin{center}
    \begin{minipage}{\linewidth}
      \includegraphics*[width=\textwidth, trim=10mm 5mm 10mm 10mm, clip]{radiofit.eps}
    \end{minipage}
    \caption{Best-fit continuous injection (CI) synchrotron model to
      the nuclear 1.4 GHz, 4.8 GHz, and 7.0 keV X-ray emission. The
      two triangles are \galex\ UV fluxes showing the emission is
      boosted above the power-law attributable to the nucleus.}
    \label{fig:sync}
    \end{center}
\end{figure}

\begin{figure}
  \begin{center}
    \begin{minipage}{\linewidth}
      \includegraphics*[width=\textwidth, trim=0mm 0mm 0mm 0mm, clip]{rbs797_opt.ps}
    \end{minipage}
    \caption{\hst\ \myi+\myv\ image of the \rbs\ BCG with units e$^-$
      s$^{-1}$. Green, dashed contour is the \cxo\ 90\% EEF. Emission
      features discussed in the text are labeled.}
    \label{fig:hst}
  \end{center}
\end{figure}

\begin{figure}
  \begin{center}
    \begin{minipage}{0.495\linewidth}
      \includegraphics*[width=\textwidth, trim=0mm 0mm 0mm 0mm, clip]{suboptcolor.ps}
    \end{minipage}
    \begin{minipage}{0.495\linewidth}
      \includegraphics*[width=\textwidth, trim=0mm 0mm 0mm 0mm, clip]{suboptrad.ps}
    \end{minipage}
    \caption{{\it{Left:}} Residual \hst\ \myv\ image. White regions
      (numbered 1--8) are areas with greatest color difference with
      \rbs\ halo. {\it{Right:}} Residual \hst\ \myi\ image. Green
      contours are 4.8 GHz radio emission down to
      $1\sigma_{\rm{rms}}$, white dashed circle has radius $2\arcs$,
      edge of ACS ghost is show in yellow, and southern whiskers are
      numbered 9--11 with corresponding white lines.}
    \label{fig:subopt}
  \end{center}
\end{figure}


%%%%%%%%%%%%%%%%%%%%
% End the document %
%%%%%%%%%%%%%%%%%%%%

\end{document}

%% There is a significant density discontinuity at $r \approx 60$ kpc
%% (insignificant pressure and temperature discontinuities) which is
%% coincident with the elliptical ridge enclosing the cavities. During
%% the formation of a cavity, bright rims can result from the
%% displacement of low entropy gas in the core
%% \citep[\eg][]{2009ApJ...697L..95B} or gas shocking
%% \citep[\eg][]{ms0735}. After extracting spectra on either side of the
%% ridge using regions with more source counts than the temperature
%% profile bins, no significant temperature jump was detected to further
%% signal the presence of a shock. Assuming there is a shock, but its
%% temperature signal is unresolved, the Rankine-Hugoniot density jump
%% relation predicts a Mach number of $\approx 1.3$. In Section
%% \ref{sec:cavities} we argue that \rbs\ may be one of the most powerful
%% AGN outbursts ever observed, and the presence of a shock, though
%% unconfirmed with the present data, would be consistent with systems
%% hosting outbursts of similar power, \eg\ \ms.

%% Bright rims surrounding the prominent cavity system $\approx 25$ kpc
%% from the BCG nucleus indicate the AGN outburst may have shocked the
%% ICM or uplifted cool, dense gas from the core.
