%%%%%%%%%%%%%%%%%%%
% Custom commands %
%%%%%%%%%%%%%%%%%%%

\newcommand{\dmb}{\ensuremath{\dot{m_{\rm{B}}}}}
\newcommand{\dme}{\ensuremath{\dot{m_{\rm{E}}}}}
\newcommand{\ldisk}{\ensuremath{L_{\rm{disk}}}}
\newcommand{\ms}{MS 0735.6+7421}
%\newcommand{\myi}{\ensuremath{I^{\prime}}}
%\newcommand{\myv}{\ensuremath{V^{\prime}}}
\newcommand{\myi}{\ensuremath{I}}
\newcommand{\myv}{\ensuremath{V}}
\newcommand{\rbs}{RBS 797}
\newcommand{\reff}{\ensuremath{r_{\rm{eff}}}}
\newcommand{\rlos}{\ensuremath{r_{\rm{los}}}}
\newcommand{\mykeywords}{}
\newcommand{\mystitle}{\rbs\ AGN Outburst}
\newcommand{\mytitle}{A Powerful, Line-of-Sight AGN Outburst in RBS
  797}

%%%%%%%%%%
% Header %
%%%%%%%%%%

\documentclass[11pt, preprint]{aastex}
\usepackage{graphicx,common}
%\documentclass[iop]{emulateapj}
%\usepackage{apjfonts,graphicx,here,common,longtable,ifthen,amsmath,amssymb,natbib}
\usepackage[pagebackref,
  pdftitle={\mytitle},
  pdfauthor={Dr. Kenneth W. Cavagnolo},
  pdfsubject={Astrophysical Journal},
  pdfkeywords={galaxies: active -- galaxies: clusters: general --
    X-rays: galaxies -- radio continuum: galaxies},
  pdfproducer={LaTeX with hyperref},
  pdfcreator={LaTeX with hyperref}
  pdfdisplaydoctitle=true,
  colorlinks=true,
  citecolor=blue,
  linkcolor=blue,
  urlcolor=blue]{hyperref}
\bibliographystyle{apj}
\begin{document}
\title{\mytitle}
\shorttitle{\mystitle}
\author{
  K. W. Cavagnolo\altaffilmark{1,2,9},
  B. R. McNamara\altaffilmark{1,3,4},
  M. W. Wise\altaffilmark{5}, \\
  M. Br\"uggen\altaffilmark{6}, 
  P. E. J. Nulsen\altaffilmark{4},
  M. Gitti\altaffilmark{4,7}, and
  D. A. Rafferty\altaffilmark{8}
}
\shortauthors{Cavagnolo et al.}

\altaffiltext{1}{University of Waterloo, 200 University Ave. W.,
  Waterloo, ON, N2L 3G1, Canada.}
\altaffiltext{2}{Observatoire de la C\^ote d'Azur, Boulevard de
  l'Observatoire, B.P. 4229, F-06304, Nice, France.}
\altaffiltext{3}{Perimeter Institute for Theoretical Physics, 31
  Caroline St. N., Waterloo, ON, N2L 2Y5, Canada.}
\altaffiltext{4}{Harvard-Smithsonian Center for Astrophysics, 60
  Garden St., Cambridge, MA, 02138-1516, United States.}
\altaffiltext{5}{Astronomical Institute Anton Pannekoek, P.O. Box
  94249, 1090 GE Amsterdam, The Netherlands.}
\altaffiltext{6}{Jacobs University Bremen, P.O. Box 750561, 28725
  Bremen, Germany.}
\altaffiltext{7}{INAF-Astronomical Observatory of Bologna, Via
  Ranzani 1, I-40127 Bologna, Italy.}
\altaffiltext{8}{Leiden Observatory, University of Leiden, P.O. 9513,
  2300 RA Leiden, The Netherlands.}
\altaffiltext{9}{cavagnolo@oca.eu}

%%%%%%%%%%%%
% Abstract %
%%%%%%%%%%%%

\begin{abstract}
  Utilizing a $\sim 40$ ks observation from the Chandra X-ray
  Observatory, we present spatially resolved analysis of the
  intracluster medium, cavities, and the nuclear point source of RBS
  797.
\end{abstract}

%%%%%%%%%%%%
% Keywords %
%%%%%%%%%%%%

\keywords{\mykeywords}

%%%%%%%%%%%%%%%%%%%%%%
\section{Introduction}
\label{sec:intro}
%%%%%%%%%%%%%%%%%%%%%%

Evidence has amassed over the last decade that the growth of galaxies
and supermassive black holes (SMBHs) are coupled, and that energetic
feedback from active galactic nuclei (AGN) strongly influences galaxy
evolution \citep[\eg][]{1995ARA&A..33..581K, magorrian,
  1998A&A...331L...1S, 2000MNRAS.311..576K, 2000ApJ...539L...9F,
  2000ApJ...539L..13G, 2002ApJ...574..740T}. The discovery of AGN
induced cavities in the hot halos surrounding many massive galaxies
has strengthened this theory \citep[see][for a review]{mcnamrev},
revealing that AGN mechanical heating is capable of regulating the
radiative cooling responsible for late-time galaxy growth
\citep[\eg][]{birzan04, dunn06, rafferty06}. Current models of this
late-time ``feedback loop'' posit that cooling processes in a galaxy's
halo result in mass accretion onto a central SMBH, thereby driving AGN
activity and ultimately heating the cooling halo
\citep[\eg][]{croton06, bower06, sijacki07}. While there is direct
evidence that halo cooling and feedback are linked
\citep[\eg][]{haradent, rafferty08}, the observational constraints on
how AGN are fueled and powered remain loose. For example, what
fraction of the energy released in an AGN outburst is attributable to
the gravitational binding energy of accreting matter \citep[see][for a
  review]{1984RvMP...56..255B} and a SMBH's rotational energy
\citep[see][for a review]{2002NewAR..46..247M} is still unclear.

Mass accretion alone can, in principle, fuel most AGN
\citep[\eg][]{pizzolato05, 2006MNRAS.372...21A}, and is typically not
at odds with the gas masses and stellar masses of the host galaxy
\citep[\eg][]{rafferty06}. However, there are systems which are
gas-poor and host very powerful AGN where mass accretion alone does
not appear to be capable of sustaining the outburst. In the galaxy
clusters Hercules A, Hydra A, and \ms, for example, an AGN in each
cluster's brightest cluster galaxy (BCG) has deposited more than
$10^{61}$ erg of energy into the surrounding intracluster medium (ICM)
at rates of order $10^{45} ~\lum$ \citep{herca, hydraa, ms0735}. The
importance of these systems in understanding AGN feedback is that
either the AGN fueling has been astoundingly efficient, or the power
has come from an alternate source, such as the release of angular
momentum stored in a rapidly spinning SMBH \citep[\eg][]{msspin,
  minaspin}. SMBH spin is an intriguing energy source because
releasing the energy requires some mass accretion (maintaining the
feedback loop) and a typical BCG SMBH can store more than $10^{62}$
erg of energy if it is rapidly spinning. If more systems can be found
which are best explained as being powered by SMBH spin, then it may be
necessary to incorporate this feedback pathway into galaxy formation
models \citep[\eg][]{2003ApJ...585L.101H, 2007ApJ...658..815S,
  2009MNRAS.397.1302B, gesspin}.

In this paper, we present analysis of the AGN outburst in the galaxy
cluster \rbs\ and, because of the extreme energetic demands of the
outburst, suggest it may have been powered by black hole spin. The
discovery of prominent cavities in the ICM of \rbs\ was first reported
by \citet{schindler01} using data from the Chandra X-ray Observatory
(\cxo). Subsequent multifrequency radio observations showed that the
cavities are co-spatial with extended radio emission centered on a
strong, jetted radio source coincident with the \rbs\ BCG
\citep{2002astro.ph..1349D, gitti06, birzan08}. The observations
implicate an AGN in the BCG as the cavities' progenitor, and the
cavity analysis presented in \citet[][hereafter B04]{birzan04} shows
that the AGN deposited $\sim 10^{60}$ erg of energy into the ICM at a
rate of $\sim 10^{45} ~\lum$, relatively large values for an AGN
outburst. The analysis of B04 assumed that the observed cavities are
distinct, symmetric about the plane of the sky, and that their centers
lie in a plane passing through the central AGN and perpendicular to
our line-of-sight (configuration-1 in Figure
\ref{fig:config}). However, the depth of the cavities and the nebulous
correlation between the radio and X-ray morphologies calls into
question these assumptions.

In this paper, we utilize a longer, follow-up \cxo\ observation of
\rbs\ to suggest that the observed ``cavities'' result from the
superposition of much larger overlapping cavities positioned nearly
along the line-of-sight (configuration-2 in Figure
\ref{fig:config}). Thus, our estimates of the cavity energy, \ecav,
and the jet powers which formed the cavities, \pjet, are 10 times
larger than the B04 values -- of the order $10^{61}$ erg and $10^{46}
~\lum$, respectively. We find that mass accretion alone is an
implausible explanation for how the outburst was powered, and instead
suggest that the outburst resulted from the extraction of rotational
energy stored in a rapidly-spinning SMBH. \rbs\ may be further
observational evidence that some AGN are powered by the release of
SMBH spin energy. 

Reduction of X-ray and radio data is discussed in Section
\ref{sec:obs}. Interpretation of observational results are given
throughout Section \ref{sec:results}, and a brief summary concludes
the paper in Section \ref{sec:con}. \LCDM. At a redshift of $z =
0.354$, the look-back time is 3.9 Gyr, $\da = 4.996$ kpc
arcsec$^{-1}$, and $\dl = 1889$ Mpc. All errors are 68\% confidence
unless stated otherwise.

%%%%%%%%%%%%%%%%%%%%%%
\section{Observations}
\label{sec:obs}
%%%%%%%%%%%%%%%%%%%%%%

%%%%%%%%%%%%%%%%%%%%%%%
\subsection{X-ray Data}
\label{sec:xray}
%%%%%%%%%%%%%%%%%%%%%%%

\rbs\ was observed with \cxo\ in October 2000 for 11.4 ks using the
ACIS-I array (\dataset [ADS/Sa.CXO#Obs/02202] {ObsID 2202}; PI
Schindler) and in July 2007 for 38.3 ks using the ACIS-S array
(\dataset [ADS/Sa.CXO#Obs/07902] {ObsID 7902}). Datasets were reduced
using \ciao\ and \caldb\ versions 4.2. Events were screened of cosmic
rays using \asca\ grades and {\textsc{vfaint}} filtering. The level-1
events files were reprocessed to apply the most up-to-date corrections
for the time-dependent gain change, charge transfer inefficiency, and
degraded quantum efficiency of the ACIS detector. The afterglow and
dead area corrections were also applied. Time intervals affected by
background flares exceeding 20\% of the mean background count rate
were isolated and removed using light curve filtering. The final,
combined exposure time is 48.8 ks. Point sources were identified and
removed via visual inspection and use of the \ciao\ tool
{\textsc{wavdetect}}. We refer to the \cxo\ data free of point sources
and flares as the ``clean'' data. A mosaiced, fluxed image (see Figure
\ref{fig:img}) was generated by exposure correcting each clean dataset
and reprojecting the normalized images to a common tangent point.

%%%%%%%%%%%%%%%%%%%%%%%
\subsection{Radio Data}
\label{sec:radio}
%%%%%%%%%%%%%%%%%%%%%%%

Very Large Array (\vla) radio images at 325 MHz (A-array), 1.4 GHz (A-
and B-array), 4.8 GHz (A-array), and 8.4 GHz (D-array) are presented
in \citet{gitti06} and \citet{birzan08}. Our re-analysis of the
archival \vla\ radio observations yielded no significant differences
with these prior studies. Using the rms noise ($\sigma_{\rm{rms}}$)
values given in \citet{gitti06} and \citet{birzan08} for each
observation, emission contours between $3\sigma_{\rm{rms}}$ and the
peak image intensity were generated. These are the contours referenced
and shown in all subsequent discussion and figures.

%%%%%%%%%%%%%%%%%%%
\section{Results}
\label{sec:results}
%%%%%%%%%%%%%%%%%%%

%%%%%%%%%%%%%%%%%%%%%%%%%%%%%%%%%%%%%%%%%%%%%%%%%%%%%
\subsection{Cavity Morphologies and ICM Substructure}
\label{sec:morph}
%%%%%%%%%%%%%%%%%%%%%%%%%%%%%%%%%%%%%%%%%%%%%%%%%%%%%

Shown in Figure \ref{fig:img} is the 0.7--2.0 keV \cxo\ mosaiced clean
image and an \hst\ optical image of the \rbs\ BCG. Outside of $\approx
50$ kpc, the global ICM morphology is regular and elliptical in shape,
with the appearance of being skewed along the NW-SE direction. The
cavities discovered by \citet{schindler01} are clearly seen in the
cluster core east and west of the nuclear X-ray source and appear to
be enclosed by a bright, elliptical ridge of emission. The western
cavity has more internal structure, and its boundaries are less
well-defined, than the eastern cavity. The emission from the innermost
region of the core is elongated N-S and has a distinct `S'-shape
punctuated by a hard nuclear X-ray source. Comparison of the
\hst\ optical and \cxo\ X-ray images reveals the BCG coincides with
the nuclear X-ray source and has the appearance of being pinched into
two equal brightness halves. It is unclear which, if either, is at the
center of the galaxy or might be associated with an AGN. There is
substantial structure to the BCG optical halo, and its association
with the AGN is discussed in Section \ref{sec:bcg}.

Multifrequency radio images overlaid with ICM X-ray contours are shown
in Figure \ref{fig:composite}. \rbs\ radio properties are discussed in
\citet{gitti06}, and we summarize here. As seen in projection, the
nuclear 4.8 GHz jets are almost orthogonal to the axis connecting the
cavities. The 325 MHz, 1.4 GHz, and 8.4 GHz radio emission are diffuse
and extend well-beyond the cavities, more similar to the morphology of
a radio mini-halo than relativistic plasma confined to the cavities
(Doria et al., in preparation). Of all the radio frequencies, the 1.4
GHz emission most closely traces the cavity morphologies, yet it is
still very diffuse and uniform over the cavities with little structure
outside the radio core. Typically, the connection between a cavity
system, coincident radio emission, and the progenitor AGN is
unambiguous. This is not the case for \rbs, which suggests the cavity
morphologies may be complex or that projection effects are
important.

To better reveal the cavity morphologies, residual X-ray images of
\rbs\ were constructed by modeling the ICM emission and subtracting it
off. The X-ray isophotes of two exposure-corrected images -- one
smoothed by a $1\arcs$ Gaussian and another by a $3\arcs$ Gaussian --
were fitted with ellipses using the \iraf\ task \textsc{ellipse}. The
ellipse centers were fixed at the location of the BCG X-ray point
source, and the eccentricities and position angles were free to
vary. A 2D surface brightness model was created from each fit using
the \iraf\ task \textsc{bmodel}, normalized to the parent X-ray image,
and then subtracted off. The residual images are shown in Figure
\ref{fig:subxray}.

In addition to the central east and west cavities (labeled E1 and W1),
depressions north and south of the nucleus (labeled N1 and S1) are
revealed. N1 and S1 lie along the 4.8 GHz jet axis and are coincident
with spurs of significant 1.4 GHz emission, indicating they may be
related to activity of the central AGN. A depression coincident with
the southeastern concentration of 325 MHz emission is also found
(labeled E2), but no counterpart on the opposing side of the cluster
is seen (labeled W2). There is an X-ray edge which extends southeast
from E2 and sits along a ridge of 325 MHz and 8.4 GHz emission. No
substructure associated with the western-most knot of 325 MHz emission
is found, but there is a stellar object co-spatial with this region.
The X-ray and radio properties of the object are consistent with those
of a galactic RS CVn star \citep{1993RPPh...56.1145S} -- if the star
is less than 1 kpc away, $\lx \la 10^{31} ~\lum$ and $L_{325} \sim
10^{27} ~\lum$ -- suggesting the western 325 MHz emission may not be
associated with the cluster. In Section \ref{sec:cavities} we focus on
disentangling the radio-X-ray relationship to better understand the
AGN outburst energetics.

%%%%%%%%%%%%%%%%%%%%%%%%%%%%%%%%%%
\subsection{Radial ICM Properties}
\label{sec:icm}
%%%%%%%%%%%%%%%%%%%%%%%%%%%%%%%%%%

In order to analyze the \rbs\ cavity system and AGN outburst
energetics in-detail, the radial ICM density, temperature, and
pressure structure need to be quantified. All of the profiles
discussed below are shown in Figure \ref{fig:gallery}. A temperature
(\tx) profile was created by extracting spectra from concentric
circular annuli (2500 source counts per annulus) centered on the
cluster X-ray peak, binning the spectra to 25 counts per energy
channel, and then fitting each spectrum in \xspec\ 12.4 \citep{xspec}
with an absorbed, single-component \mekal\ model \citep{mekal1} over
the energy range 0.7--7.0 \keV. For each annulus, weighted responses
were created and a background spectrum was extracted from the ObsID
matched \caldb\ blank-sky dataset normalized using the ratio of 9--12
keV count rates for an identical off-axis, source-free region of the
blank-sky and target datasets. A spectral model for the Galactic
foreground was included as an additional, fixed background component
during spectral fitting \citep[see][for method]{2005ApJ...628..655V,
  xrayband}, and the absorbing Galactic column density was fixed to
$\nhgal = 2.28 \times 10^{20} ~\pcmsq$ \citep{lab}. Gas metal
abundance was free to vary and normalized to the \citet{ag89} solar
ratios. Spectral deprojection using the {\textsc{proj}} \xspec\ model
did not produce significantly different results, thus only projected
quantities are discussed.

A 0.7--2.0 keV surface brightness profile was extracted from the
\cxo\ mosaiced clean image using concentric $1\arcs$ wide elliptical
annuli centered on the BCG X-ray point source (central $\approx
1\arcs$ and cavities excluded). A deprojected electron density
(\nelec) profile was derived from the surface brightness profile using
the method of \citet{kriss83} which incorporates the 0.7--2.0 keV
count rates and best-fit normalizations from the spectral analysis
\citep[see][for details]{accept}. Errors for the density profile were
estimated using 5000 Monte Carlo simulations of the original surface
brightness profile. A total gas pressure profile was calculated as $P
= n \tx$ where $n \approx 2.3 \nH$ and $\nH \approx
\nelec/1.2$. Profiles of enclosed X-ray luminosity, entropy ($K = \tx
\nelec^{-2/3}$) and cooling time ($\tcool = 3 n \tx~(2 \nelec \nH
\Lambda)^{-1}$, where $\Lambda$ is the cooling function), were also
generated. Errors for each profile were determined by summing the
individual parameter uncertainties in quadrature.

%%%%%%%%%%%%%%%%%%%%%%%%%
\subsection{ICM Cavities}
\label{sec:cavities}
%%%%%%%%%%%%%%%%%%%%%%%%%

%%%%%%%%%%%%%%%%%%%%%%%%%%%%%%%%%%%%%%%%%
\subsubsection{Plane of the Sky Cavities}
\label{sec:ecav}
%%%%%%%%%%%%%%%%%%%%%%%%%%%%%%%%%%%%%%%%%

Of all the ICM substructure, the E1 and W1 cavities are unambiguous
detections, and their energetics were determined following standard
methods (see B04). Our fiducial cavity configuration assumes the
cavities are symmetric about the plane of the sky, and the cavity
centers lie in a plane which is perpendicular to our line-of-sight and
passes through the central AGN (configuration-1 in Figure
\ref{fig:config}). Hereafter we denote the line-of-sight distance of a
cavity's center from this plane as $z$ and the cavity radius along the
line-of-sight as \rlos. The volume of a cavity is $V = 4\pi a b
\rlos/3$ where $a$ and $b$ are the projected semi-major and -minor
axes, respectively, of the ellipses in Figure \ref{fig:subxray}. For
simplicity, the cavities were assumed to be roughly spherical and
\rlos\ was set equal to the projected effective radius, $\reff =
\sqrt{ab}$. For configuration-1, the distance of each cavity from the
central AGN, $D$, is simply the projected distance from the ellipse
centers to the BCG X-ray point source. A systematic error of 10\% is
assigned to the cavity volumes. The cavity properties are listed in
Table \ref{tab:cavities}.

Cavity ages were estimated using the three time scales discussed in
B04: ICM sound speed age (\tsonic), buoyant rise time age (\tbuoy),
and volume refilling age (\trefill). The energy in each cavity, $\ecav
= \gamma PV/(\gamma-1)$, was calculated assuming the contents are a
relativistic plasma ($\gamma = 4/3$), and that the mean internal
cavity pressure is equal to the ICM pressure at the cavity
mid-point. The time-averaged energy needed to create each cavity was
calculated as $\pcav = \ecav/\tbuoy$. For this configuration, we
measure an aggregate energy $\ecav = 2.85 ~(\pm 1.02) \times 10^{60}$
erg and total power $\pcav = 2.85 ~(\pm 1.21) \times 10^{45}
~\lum$. These values are larger than those of B04 ($\ecav = 1.52
\times 10^{60}$ erg, $\pcav = 1.13 \times 10^{44} ~\lum$) by a factor
of $\approx 2$ as a result of our larger E1 and W1 volumes. Using the
B04 cavity volumes in place of our own produces no significant
differences between our energetics calculations and those of B04. The
largest sources of uncertainty in the energetic calculations are the
cavity volumes and their location in the cluster, issues we consider
next.

%%%%%%%%%%%%%%%%%%%%%%%%%%%%%%%%%
\subsubsection{Cavity Decrements}
\label{sec:dec}
%%%%%%%%%%%%%%%%%%%%%%%%%%%%%%%%%

A cavity's morphology and location in the ICM dictates the X-ray
decrement, $y$, induced by the cavity. If the surface brightness of
the undisturbed ICM can be estimated, then $y$ is useful for
constraining \rlos\ and $z$ \citep[see][for details]{hydraa}. We
define $y$ as the ratio of the X-ray surface brightness inside the
cavities to the value of the best-fit ICM surface brightness
$\beta$-model at the same radius. Consistent with the analysis of
\citet{schindler01}, we find the best-fit $\beta$-model has parameters
$S_0 = 1.65 ~(\pm 0.15) \times 10^{-3}$ \sbr, $\beta = 0.62 \pm 0.04$,
and $r_{\rm{core}} = 7.98\arcs \pm 0.08$ for \chisq(DOF) =
79(97). Using a circular aperture with radius $1\arcs$ centered on the
deepest part of each cavity, we measure mean decrements of
$\bar{y}_{\rm{W1}} = 0.50 \pm 0.18$ and $\bar{y}_{\rm{E1}} = 0.52 \pm
0.23$, with minima of $y^{\rm{min}}_{\rm{W1}} = 0.44$ and
$y^{\rm{min}}_{\rm{E1}} = 0.47$.

Cavities this deep are uncommon, with most other cavity systems having
minima greater than 0.6 (B04). The extreme decrements raise the
concern that the $\beta$-model fit has been influenced by the rim-like
structures surrounding E1-W1 (see Section \ref{sec:icm}) and produced
artificially low decrements. In addition to excluding the cavities, we
tried excluding the rims during $\beta$-model fitting, but this did
not provide insight as too much of the surface brightness profile was
removed and the fit did not converge. Extrapolating the surface
brightness profile at larger radii inward resulted in even lower
decrements. With no suitable alternative, we assume the $\beta$-model
is the best representation of the undisturbed ICM.

To check if cavity configuration-1 can produce the measured
decrements, the best-fit $\beta$-model was integrated over each cavity
with a column of gas equal to $2\rlos$ excluded. The integration
revealed decrements $< 0.67$ cannot be achieved for configuration-1,
indicating our initial assumption that $\rlos = \reff$ is poor. The
measured decrements can be reproduced when $z = 0$ and $\rlos = 21.1$
kpc for E1 and $\rlos = 25.3$ kpc for W1. However, these radii suggest
that the cavities are two times larger along the line-of-sight than in
the plane of the sky. If the radio lobes are inflating perpendicular
to our line-of-sight, then how the cavities achieved a highly flattened
morphology so close to the launch point is unclear. Since the
$\beta$-model is symmetric about the plane of the sky, a decrement
less than 0.5 requires that gas on both sides of this plane be
displaced. Thus, cavity configuration-3 in Figure \ref{fig:config},
where an elongated radio lobe inflating parallel to the line-of-sight
is viewed along its axis, is unlikely.

Alternatively, if our sightlines cut through a pair of lobes inflating
along the line-of-sight, as shown in configuration-2 of Figure
\ref{fig:config}, then deep decrements could be achieved. Taking a cue
from planetary nebulae observations, the \rbs\ ICM morphology is
reminiscent of the overlapping bubbles found in some planetary nebulae
\citep{1999AJ....118..468S}, for example the Hourglass Nebula shown in
Figure \ref{fig:proj}. Line-of-sight cavities should also result in
the superposition of multiple radio emission features associated with
each cavity, possibly wiping-out otherwise clear correlations between
X-ray and radio features, and yielding a viable explanation for their
ambiguous relationship. If the cavities lie along the line-of-sight,
then the volumes may also be larger, and thus so will the estimated
AGN energy output. Cavity configuration-2 is considered in-detail
below.

%%%%%%%%%%%%%%%%%%%%%%%%%%%%%%%%%%%%%%
\subsubsection{Line-of-Sight Cavities}
%%%%%%%%%%%%%%%%%%%%%%%%%%%%%%%%%%%%%%

Shown in Figure \ref{fig:proj} is a proposed configuration for a pair
of large cavities (O1 and O2) lying along the line-of-sight whose
projected cross-sections overlap. The elliptical regions were chosen
such that they coincide with most of the X-ray substructure and
enclose most of the radio emission. For this cavity configuration, a
variety of \rlos\ and $z$ can reproduce the measured decrements. Thus,
we selected a simple morphology and again set $\rlos = \reff$. Taking
into consideration that each cavity will account for $\approx 50\%$ of
the total decrement, the $\beta$-model integration indicates
$z_{\rm{E1}} = 66^{+13}_{-9}$ and $z_{\rm{W1}} = 65^{+12}_{-9}$. The
centers of the cavities are then a distance $D \approx \sqrt{d^2+z^2}$
from the central AGN, where $d$ is the projected distance from the
ellipse centers to the BCG X-ray point source. The energetics
calculations of Section \ref{sec:ecav} were repeated for O1 and O2 and
are given in Table \ref{tab:cavities}.

The energetics for the line-of-sight cavity configuration are extreme
-- total cavity energy of $6.77 ~(\pm 2.98) \times 10^{61}$ erg and
cavity power of $2.69 ~(\pm 1.32) \times 10^{46} ~\lum$ -- and are
likely upper limits given the uncertainties of the cavity morphologies
and decrement estimates. But, they do suggest the \rbs\ outburst may
be one of the most powerful found to date. The mechanical power output
is similar to the radiative power output of a typical quasar, and
exceeds the $\approx 3 \times 10^{45} ~\lum$ of energy radiated away
by ICM gas with a cooling time less than the age of the Universe at
$z=0.35$. As we discuss in Section \ref{sec:accretion}, powering an
outburst of this magnitude by mass accretion alone is problematic,
allowing us to consider black hole spin energy as an alternate power
source.

%%%%%%%%%%%%%%%%%%%%%%%%%%%%%%%%%%%%%%
\subsection{Powering the AGN Outburst}
\label{sec:accretion}
%%%%%%%%%%%%%%%%%%%%%%%%%%%%%%%%%%%%%%

Assuming \ecav\ is representative of the gravitational binding energy
released by mass accreted onto the SMBH, then the total mass accreted
and its rate of consumption were $\macc = \ecav/(\epsilon c^2)$ and
$\dmacc = \macc/\tbuoy$, respectively. Here, $\epsilon$ is a poorly
understood mass-energy conversion factor assumed to be 0.1, which
gives $\macc = 3.79 ~(\pm 1.67) \times 10^8 ~\msol$ and $\dmacc = 4.74
\pm 2.34 ~\msolpy$. Below, we consider if the accretion of cold or hot
gas pervading the BCG could meet these requirements.

%%%%%%%%%%%%%%%%%%%%%%%%%%%%%%
\subsubsection{Cold Accretion}
\label{sec:cold}
%%%%%%%%%%%%%%%%%%%%%%%%%%%%%%

Substantial quantities of cold molecular gas and optical nebulae are
found in many cool core clusters \citep{crawford99, edge01}. If
rapidly cooling thermal instabilities form in these structures, it is
possible that they are the source of fuel for AGN activity and star
formation \citep[\eg][]{pizzolato05, 2010arXiv1003.4181P}. The
molecular gas mass (\mmol) of \rbs\ has not been measured, so it was
inferred using the \mmol-\halpha\ correlation in \citet{edge01}. An
optical spectrum (3200-7600 \AA) of the \rbs\ BCG reveals a strong
\hbeta\ emission line \citep{rbs1, rbs2}, which we used as a surrogate
for \halpha\ by assuming a Balmer decrement of
EW$_{\hbeta}$/EW$_{\halpha}$ = 0.29, where EW is line equivalent width
\citep{2006ApJ...642..775M}. Under these assumptions, we estimate
$\mmol \sim 10^{10} ~\msol$, sufficiently in excess of \macc\ that we
can infer there is a cold gas reservoir capable of providing fuel for
the outburst. It must be noted that the \mmol-\halpha\ relation has
substantial scatter \citep{salome03}, and that the \rbs\ emission line
measurements are highly uncertain (A. Schwope, private communication),
so this \mmol\ estimate is simply informative.

Having ample cold gas to power the outburst is not enough, mass
accretion also results in SMBH mass growth which is coupled to the
stellar mass of the host galaxy bulge \citep{1995ARA&A..33..581K}. It
has been demonstrated that for each mass unit going into SMBH mass
accretion, several hundred times as much goes into stars
\citep{magorrian}. The mass accretion rate needed to power the
outburst ($\dmacc \approx 5 ~\msolpy$) thus requires that at least
$500 ~\msolpy$ of star formation have accompanied the fueling event
which powered the outburst. Unfortunately, the current \rbs\ BCG star
formation rate (SFR) is $\sim 1$--10 \msolpy\ and the stellar halo's
color profile is inconsistent with recent star formation (see Section
\ref{sec:bcg}). Assuming the present SFR is representative of the SFR
preceding and during the AGN outburst, the implication is that for
every mass unit accreted by the SMBH, a nearly equal amount went into
stars, in conflict with expectations of SMBH-host galaxy
co-evolution. Given the limitations of the data available, we conclude
that if cold gas accretion did power the outburst, it appears to have
been astoundingly efficient.

%%%%%%%%%%%%%%%%%%%%%%%%%%%%%
\subsubsection{Hot Accretion}
%%%%%%%%%%%%%%%%%%%%%%%%%%%%%

Direct accretion of the hot ICM via the Bondi mechanism provides
another possible AGN fuel source. The accretion flow arising from this
process is characterized by the Bondi equation (Equation
\ref{eqn:bon}) and is often compared with the Eddington limit
describing the maximal accretion rate for a SMBH (Equation
\ref{eqn:edd}):
\begin{eqnarray}
  \dmbon &=& 0.013 ~\kbon^{-3/2} \left(\frac{\mbh}{10^9
    ~\msol}\right)^{2} ~\msolpy \label{eqn:bon}\\
  \dmedd &=& \frac{2.2}{\epsilon} \left(\frac{\mbh}{10^9~\msol}\right)
  ~\msolpy  \label{eqn:edd}
\end{eqnarray}
where $\epsilon$ is a mass-energy conversion factor, \kbon\ is the
mean entropy [\ent] of gas within the Bondi radius, and \mbh\ is black
hole mass [\msol]. We chose the relations of
\citet{2002ApJ...574..740T} and \citet{2007MNRAS.379..711G} to
estimate \mbh, and find a range of $0.6 \dash 7.8 \times 10^9 ~\msol$,
from which we adopted the weighted mean value $1.5^{+6.3}_{-0.9}
\times 10^9 ~\msol$. The errors on \mbh\ were chosen to span the
lowest and highest $1\sigma$ values of the individual
estimates. Fitting the ICM entropy profile with the function $K = \kna
+\khun (r/100 ~\kpc)^{\alpha}$, where \kna\ [\ent] is core entropy,
\khun\ [\ent] is a normalization at 100 kpc, and $\alpha$ is a
dimensionless index, reveals best-fit parameters of $\kna = 17.9 \pm
2.2 ~\ent$, $\khun = 92.1 \pm 6.2 ~\ent$, and $\alpha = 1.65 \pm
0.11$. For $\epsilon = 0.1$ and $\kbon = \kna$, the relevant accretion
rates are $\dmbon \approx 4 \times 10^{-4} ~\msolpy$ and $\dmedd
\approx 33 ~\msolpy$. The Eddington and Bondi accretion ratios for the
outburst event were thus $\dme \equiv \dmacc/\dmedd \approx 0.15$ and
$\dmb \equiv \dmacc/\dmbon \approx 17000$.

The large value of \dmb\ is troublesome since it implies all the gas
which reached the Bondi radius ($\rbon \approx 10$ pc) was
accreted. But, \dmb\ may be overestimated if $\kbon < \kna$ and
\mbh\ is larger than our adopted value. If $\mbh = 8 \times 10^9
~\msol$ and gas near \rbon\ has a mean temperature of 0.1 keV, for
\dmb\ to approach unity, \kbon\ must be less than $0.3 ~\ent$,
corresponding to $\nelec \sim 0.2 ~\pcc$. This density is twice the
measured ICM central density, and a sphere more than 2 kpc in radius
is required for up to $10^8 ~\msol$ to be available for accretion. But
this implies the inner-core of the cluster is fully consumed during
the outburst lifetime ($<$ 100 Myr), which is much shorter than the
central cooling time of the ICM (400 Myr), thereby short-circuiting
the feedback loop. Conversely, if the BCG harbors an ultramassive
black hole, $\sim 10^{10}$, the strain on the Bondi mechanism is eased
somewhat \citep[\eg][]{msspin}. While Bondi accretion cannot be ruled
out because the nucleus is unresolved and the cluster is observed
post-outburst when core conditions may have dramatically changed, a
number of observationally unsupported concessions must be made for
Bondi accretion to be viable.

%%%%%%%%%%%%%%%%%%%%%%%%%%%%%%%%
\subsubsection{Nucleus Emission}
%%%%%%%%%%%%%%%%%%%%%%%%%%%%%%%%

A bright, nuclear BCG X-ray source, which has no clear optical point
source counterpart, is apparent in the \cxo\ imaging. The X-ray
properties of the source were investigated to constrain possible
on-going accretion processes. A spectrum for the X-ray source was
extracted from a region enclosing 90\% of the normalized \cxo\ PSF
specific to the nuclear source median photon energy and off-axis
position. The source region had an effective radius of $1.16\arcs$,
and a background spectrum was taken from an enclosing annulus that had
5 times the area. The shape and features of the background-subtracted
spectrum, shown in Figure \ref{fig:nucspec}, are inconsistent with
thermal emission, and was instead modeled using an absorbed power law
with two Gaussians added to account for features which may be blends
of photoionized lines. A variety of models absorption models were fit
to the spectrum, and the best-fit values are given in Table
\ref{tab:agn}. The model with a power-law distribution of $\nhobs \sim
10^{22} ~\pcmsq$ absorbers yields the best statistical fit, and the
low column densities (well-below even moderately Compton-thick)
indicate the nucleus is not heavily obscured.

The nuclear bolometric luminosity from the best-fit model is $\lbol =
3.46 \times 10^{44} ~\lum$. If mass accretion is powering the nuclear
emission, the implied accretion rate is $\dmacc \approx \lbol/(0.1
c^2) \approx 0.06 ~\msolpy \approx 0.002 \dme$. For this \dmacc\ and
the upper-end of our black hole mass estimates, $8 \times 10^{9}
~\msol$, the accretion disk model of \citet{2002NewAR..46..247M}
predicts an optical luminosity of $\ldisk \approx 8 \times 10^{45}
~\lum$, but no $L_{\rm{opt}} > 10^{44} ~\lum$ sources are found in the
\hst\ imaging. The low column densities found in the X-ray modeling
could not conceal such a source, so either the disk model is
inappropriate for the nucleus, the optical emission is beamed away
from us, or it is absent.

Extrapolation of the best-fit X-ray spectral model to radio
frequencies reveals good agreement with the measured 1.4 GHz and 4.8
GHz nuclear radio fluxes. The continuous injection synchrotron model
of \citet{1987MNRAS.225..335H} also produced an acceptable fit to the
X-ray, 1.4 GHz, and 4.8 GHz nuclear fluxes (see Figure
\ref{fig:sync}). This result suggests the nuclear source may be
synchrotron emission from unresolved jets (which are obvious in the
high-resolution 4.8 GHz radio image), and not the remnants of a very
dense, hot gas phase which would be associated with Bondi accretion.

%%%%%%%%%%%%%%%%%%%%%%%%%%%%%%%
\subsubsection{Black Hole Spin}
%%%%%%%%%%%%%%%%%%%%%%%%%%%%%%%

Powering the outburst by mass accretion alone appears to be
inconsistent with the properties of the ICM and BCG. In an effort to
find lower accretion rates which still produce powerful jets, we
consider below if a rapidly-spinning SMBH could act as an alternate
power source. We consider the spin model here with the caveat that,
like cold- and hot-mode accretion, spin is wrought with its own
difficulties and hard to evaluate for any one system (see
\citealt{msspin} and \citealt{minaspin} for thorough discussion on
these points).

In the Blandford-Znajek \citep[BZ;][]{bz} and Blandford-Payne
\citep[BP;][]{bp} jet production mechanisms, strong magnetic fields
wound around a black hole extract energy from a rotating accretion
disk (BP), or rotation of the black hole itself (BZ), and funnel this
energy into a pair of opposing relativistic jets lying along the SMBH
spin axis. Hybrid spin models \citep[\eg][]{1999ApJ...522..753M,
  2001ApJ...548L...9M, 2006ApJ...651.1023R, 2007MNRAS.377.1652N,
  2009MNRAS.397.1302B, gesspin} pair the BP and BZ processes, and
typically use some variant of an advection-dominated accretion flow
\citep[ADAF;][]{adaf} to produce jets. In such models, the normalized
mass accretion rate (\dme) determines the magnetic field strength
($B_d$) which dictates the resulting jet power, \ie\ $B_d \propto
\gamma_1 \dme^{\Gamma_1}$ and $\pjet \propto \gamma_2 j^{\Gamma_2}
B_d^{\Gamma_3}$ where $j$ is a dimensionless spin parameter and model
dependencies like disk viscosity ($\alpha$), fluid equation of state,
and ratio of gas to magnetic pressure are folded into the $\gamma_i$
and $\Gamma_i$ factors. Of these parameters, only \pjet\ is
observationally constrained, so a range of $j$ and \dme\ combinations
can produce any particular \pjet. However, to avoid the need for
excessively large \dme, which is the shortcoming of the cold- and
hot-mode accretion processes, $j$ needs to be as near unity as
possible to keep the mass accretion rate as small as possible,
\ie\ the choice of \dme\ is not completely arbitrary.

In order to explain the \rbs\ jet power using the hybrid model of
\citet{2007MNRAS.377.1652N}, for $\mbh = 1.5 \times 10^9 ~\msol$,
$\alpha = 0.25$ \citep{1999ApJ...520..298Q}, and $j = 1$, the mass
accretion rate needs to be $\lmdot \ga 0.13$ or $\dmacc \approx 4
~\msolpy$. This accretion rate is subject to the same problems
discussed for cold- and hot-mode accretion, in which case spin appears
to be no better a solution. Alternatively, adopting the argument
presented in \citet{minaspin}, which suggests $\dme$ for similarly
powerful systems like \ms\ and Hercules A cannot be much more than
$\approx 0.02$, if we have underestimated \mbh\ and the true mass is
$\ge 3 \times 10^9 ~\msol$, then the \citet{2007MNRAS.377.1652N} model
can achieve our \pjet\ limit when $j \approx 1$, which implies
accretion rates $< 1.5 ~\msolpy$. Additional relief can be found in
the model of \citet{gesspin}, which has the feature that extremely
powerful jets are produced for a SMBH which is spinning retrograde
relative to the direction of the accreting material. For the
\citet{gesspin} model, when $j=1$ and $\mbh = 1.5 \times 10^9 ~\msol$,
the required mass accretion rate is $\dme \approx 0.01$ or $\approx
0.3 ~\msolpy$. So while we cannot constrain $j$ or $\dme$, there are
reasonable accretion rates for a maximally spinning SMBH which can
reproduce our \pjet\ limit.

%%%%%%%%%%%%%%%%%%%%%%%%%%%%%%%%%%%%%%%%%%%%%%%%%%%
\subsection{AGN-BCG Interaction and Star Formation}
\label{sec:bcg}
%%%%%%%%%%%%%%%%%%%%%%%%%%%%%%%%%%%%%%%%%%%%%%%%%%%

The \rbs\ BCG stellar halo properties were constrained using archival
Hubble Space Telescope (\hst) ACS/WFC F606W (4500--7500 \AA; \myv) and
F814W (6800--9800 \AA; \myi) images. Two ACS ghosts \citep{acsghost}
in the \myi\ image begin at $\approx 2.7\arcs$ and $\approx 5.4\arcs$
from the BCG center. Within a $2\arcs$ aperture centered on the BCG,
the measured magnitudes are $m_{\myv} = 19.3 \pm 0.7$ mag and
$m_{\myi} = 18.2 \pm 0.6$ mag, consistent with non-\hst\ measurements
of \citet{rbs1}, indicating the photometry in this region is
unaffected. A radial $(\myv-\myi)$ color profile for the central
$2\arcs$ ($\approx 10$ kpc) was fitted with the function
$\Delta(\myv-\myi) \log r + b$, where $\Delta(\myv-\myi)$ [mag
  dex$^{-1}$] is the color gradient, $r$ is radius, and $b$ [mag] is a
normalization, resulting in best-fit parameters $\Delta(\myv-\myi) =
-0.20 \pm 0.02$ and $b = 1.1 \pm 0.01$ for \chisq(DOF) =
0.009(21). The flat, red color profile is consistent with other BCGs
that do not have strong star formation \citep[\eg][]{rafferty06}. The
prominence of optical emission lines in the images was estimated using
the ratio of line equivalent widths taken from \citet{rbs1} to
\hst\ passband widths. The \halpha\ contribution to the \myi\ image is
estimated at $\approx 3\%$ using the scaled \hbeta\ line (see Section
\ref{sec:cold}), and the combined \hbeta, \oii, and
\oiii\ contribution to the \myv\ image is $\approx 7\%$. The low
percentages indicate the stellar continuum should be well-represented.

The BCG star formation rate was constrained using far-UV (FUV;
1344--1786 \AA) and near-UV (NUV; 1771--2831 \AA)
\galex\ observations, and XMM-Newton Optical-Monitor (\xom) UVW1
(2410--3565 \AA) and UVM2 (1970--2675 \AA) observations. The
\galex\ pipeline produced images were used for analysis, and the
\xom\ data was processed using \sas\ version 8.0.1. \rbs\ is detected
in all but the UVM2 observation as an unresolved source co-spatial
with the BCG optical and X-ray emission. The individual filter fluxes
are $f_{\rm{FUV}} = 19.2 \pm 4.8 ~\mu$Jy, $f_{\rm{NUV}} = 5.9 \pm 2.1
~\mu$Jy, $f_{\rm{UVW1}} = 14.6 \pm 4.6$ $\mu$Jy and $f_{\rm{UVM2}} <
117$ $\mu$Jy, all of which lie above the power-law emission of the
nucleus (see Figure \ref{fig:sync}). Star formation rates were
calculated from the UV fluxes using the relations of
\citet{kennicutt2}, \citet{2006ApJ...642..775M}, and
\citet{salim2007}, which yield rates in the range 1--10 ~\msolpy\ (see
Table \ref{tab:sfr}). These estimates are likely upper limits since
the blue \galex\ color signals some AGN contamination
\citep{2005AJ....130.1022A}, and sources of significant uncertainties
have been neglected \citep[\eg][]{1992ApJ...388..310K,
  2004AJ....127.2002K, hicksuv, 2010MNRAS.tmp..626G}. In light of the
flat $(\myv-\myi)$ color profile, it appears any star formation may
not be smoothly distributed in the halo but confined to compact
regions like the bright substructures seen in the \hst\ images.

To better reveal the BCG substructure, residual galaxy images were
constructed by first fitting the \hst\ \myv\ and \myi\ isophotes with
ellipses using the \iraf\ tool {\textsc{ellipse}}. Stars and other
contaminating sources were rejected using a combination of $3\sigma$
clipping and by-eye masking. The ellipse centers were fixed at the
galaxy centroid, and ellipticity and position angle were fixed at
$0.25 \pm 0.02$ and $-64\mydeg \pm 2\mydeg$, respectively -- the mean
values when they were free parameters. Galaxy light models were
created using {\textsc{bmodel}} in \iraf\ and subtracted from the
corresponding parent image, leaving the residual images shown in
Figure \ref{fig:subopt}. A color map was also generated by subtracting
the fluxed \myi\ image from the fluxed \myv\ image.

The close alignment of the optical substructure with the nuclear AGN
outflow clearly indicates the jets are interacting with the BCG
halo. There also appears to be an in-falling galaxy northwest of the
nucleus, possibly with a stripped tail \citep[see][for
  example]{2007ApJ...671..190S}. The numbered regions overlaid on the
residual \myv\ image are the areas of the color map which have the
largest color difference with surrounding galaxy light. Regions 1--5
are relatively the bluest with $m_{\myv-\myi} = -0.40, -0.30, -0.25,
-0.22, ~\rm{and} -0.20$, respectively. Regions 6--8 are relatively the
reddest with $m_{\myv-\myi} =$ +0.10, +0.15, and +0.18,
respectively. Without spectroscopy, we can only speculate that the
blue regions may be star formation sites, the red regions areas of
heavy reddened by dust extinction, or both may be strong emission line
regions.

Clusters with a core entropy less than $30 ~\ent$, like \rbs\ which
has $\kna \approx 20 ~\ent$, are known to, on average, host BCGs with
filamentary nebulae \citep[\eg][]{mcdonald10}. Like these systems, the
residual \myi\ image reveals what appear to be 8--10 kpc long
``whiskers'' surrounding the BCG (regions 9--11). It is interesting
that blue regions 1 and 4 reside at the point where the southern jet
appears to be encountering whiskers 9 and 10. It may be that the jet
is interacting with gas in the whiskers and possibly driving star
formation. If the AGN outflow is driving star formation, then the
estimated SFRs will be boosted above the putative rate, narrowing the
difference between the AGN mass accretion and star formation rates and
exacerbating the difficulty with cold-mode accretion (see Section
\ref{sec:accretion}).

%%%%%%%%%%%%%%%%%%%%%
\section{Conclusions}
\label{sec:con}
%%%%%%%%%%%%%%%%%%%%%

We have presented results from a study of the AGN outburst in the
galaxy cluster \rbs. \cxo\ observations have enabled us to constrain
the energetics of the AGN outburst and analyze different mechanisms
which may have powered it. We have shown the following:
\begin{enumerate}
\item \rbs\ is a cool core object with a central cooling time $< 0.5$
  Gyr, a core entropy $< 30 ~\ent$, and bolometric cooling luminosity
  of $\sim 10^{45} ~\lum$.
\item The two central cavities have decrements consistent with voids
  that cross the plane of the sky, and residual X-ray images reveal
  additional surface brightness depressions likely associated with AGN
  activity. We have placed constraints on the AGN outburst energetics
  by considering two limiting cases: small cavities with centers in
  the plane of the sky, and large overlapping cavities along the line
  of sight. The total energy in the cavities is $3-70 \times 10^{60}$
  erg, with powers $3-40 \times 10^{45} ~\lum$. If the upper limits
  are accurate, the AGN outburst has released enough energy to
  suppress cooling of the cluster halo.
\item The energetics demand that cold-mode and hot-mode gas accretion
  operate with unrealistic efficiency if mass accretion alone powered
  the outburst. We specifically show that Bondi accretion is an
  unattractive solution, and instead suggest that the outburst may
  have been powered by tapping the energy stored in a rapidly rotating
  SMBH. We show that accretion rates less than $2 ~\msolpy$ can
  achieve our \ecav\ and \pcav\ limits.
\item The optical substructure of the BCG clearly indicates
  interaction of the AGN outflow with the galaxy halo. We are unable
  to determine if any regions of interest host star formation or line
  emission. But, the convergence of what appear to be optical
  filaments, bluish knots of emission, and the tip of one jet suggest
  there may be AGN driven star formation.
\end{enumerate}

%%%%%%%%%%%%%%%%%
\acknowledgements
%%%%%%%%%%%%%%%%%

KWC and BRM were supported by CXO grant G07-8122X and a generous grant
from the Natural Science and Engineering Research Council of
Canada. KWC also acknowledges financial support from the Agence
Nationale de la Recherche through grant ANR-09-JCJC-0001-01. KWC
thanks David Gilbanks, Sabine Schindler, Axel Schwope, and Chris
Z. Waters for helpful input.

%%%%%%%%%%%%%%
% Facilities %
%%%%%%%%%%%%%%

{\it Facilities:} \facility{CXO (ACIS)} \facility{HST (WFPC2)}
\facility{GALEX} \facility{BTA (SP124)}

%%%%%%%%%%%%%%%%
% Bibliography %
%%%%%%%%%%%%%%%%

\bibliography{cavagnolo}

%%%%%%%%%%%%%%%%%%%%%%
% Figures and Tables %
%%%%%%%%%%%%%%%%%%%%%%

%% \clearpage
%% \begin{deluxetable*}{cccccccccc}[ht]
\tabletypesize{}
\tablewidth{\linewidth}
\tablecaption{Cavity Properties.\label{tab:cavities}}
\tablehead{
\colhead{ID} & \colhead{$a$} & \colhead{$b$} & \colhead{$c$} & \colhead{$D$} & \colhead{\tsonic} & \colhead{\tbuoy} & \colhead{\trefill} & \colhead{\ecav} & \colhead{\pcav}\\
\colhead{-} & \colhead{kpc} & \colhead{kpc} & \colhead{kpc} & \colhead{kpc} & \colhead{Myr} & \colhead{Myr} & \colhead{Myr} & \colhead{$10^{60}$ erg} & \colhead{$10^{45}$ erg s$^{-1}$}\\
\colhead{(1)} & \colhead{(2)} & \colhead{(3)} & \colhead{(4)} & \colhead{(5)} & \colhead{(6)} & \colhead{(7)} & \colhead{(8)} & \colhead{(9)} & \colhead{(10)}}
\startdata
O1 &  $60.5 \pm 6.1$ & $37.6 \pm 3.8$ & $47.7 \pm 4.8$ & $70.0 \pm 7.0$ & $50.3 \pm 10.4$ & $79.1 \pm 12.5$ & $228.9 \pm 28.1$ & $33.8 \pm 10.5$ & $13.6 \pm 4.7$\\
O2 &  $55.6 \pm 5.6$ & $41.0 \pm 4.1$ & $47.7 \pm 4.8$ & $69.0 \pm 6.9$ & $49.6 \pm 10.2$ & $80.7 \pm 12.8$ & $227.3 \pm 27.9$ & $33.9 \pm 10.5$ & $13.3 \pm 4.6$\\
\enddata
\tablecomments{
Col. (1) Cavity identification;
Col. (2) Semi-major axis;
Col. (3) Semi-minor axis;
Col. (4) Semi-polar axis;
Col. (5) Distance from AGN;
Col. (6) Sound speed age;
Col. (7) Buoyancy age;
Col. (8) Refill age;
Col. (9) Cavity energy assuming $\gamma = 4/3$;
Col. (10) Cavity power using \tbuoy.}
\end{deluxetable*}

%% \begin{deluxetable}{lcccccccccccc}
  \tablecolumns{13}
  \tablewidth{0pc}
  \tabletypesize{\footnotesize}
  \tablecaption{Nuclear X-ray Point Source Spectral Models.\label{tab:agn}}
  \tablehead{
    \colhead{Absorber} & \colhead{\nhabs} & \colhead{$\Gamma_{\rm{pl}}$} & \colhead{$\eta_{\rm{pl}}$} & \colhead{$E_{\rm{ga}}$} & \colhead{$\sigma_{\rm{ga}}$} & \colhead{$\eta_{\rm{ga}}$} & \colhead{Param.} & \colhead{$L_{0.7-7.0}$} & \colhead{\lbol} & \colhead{\chisq} & \colhead{DOF} & \colhead{Goodness}\\
    \colhead{-} & \colhead{$10^{22}~\pcmsq$} & \colhead{-} & \colhead{$10^{-5} \dagger$} & \colhead{keV} & \colhead{eV} & \colhead{$10^{-6} \ddagger$} & \colhead{-} & \colhead{$10^{44}~\lum$} & \colhead{$10^{44}~\lum$} & \colhead{-} & \colhead{-} & \colhead{-}\\
    \colhead{(1)} & \colhead{(2)} & \colhead{(3)} & \colhead{(4)} & \colhead{(5)} & \colhead{(6)} & \colhead{(7)} & \colhead{(8)} & \colhead{(9)} & \colhead{(10)} & \colhead{(11)} & \colhead{(12)} & \colhead{(13)}}
  \startdata
  None          & \nodata             & $0.1^{+0.3}_{-0.3}$ & $0.3^{+0.1}_{-0.1}$  & [2.4, 3.4] & [63, 119] & [8.4, 14.9] & \nodata                & $0.69^{+0.11}_{-0.22}$ & $49.9^{+18.1}_{-19.0}$ & 1.88 & 61 & 56\%\\
  Neutral$^a$   & $4.2^{+1.9}_{-1.3}$ & $1.5^{+0.4}_{-0.3}$ & $6.6^{+8.1}_{-3.6}$  & [1.8, 3.0] & [31, 58] & [0.6, 1.8]  & 0.354                  & $0.68^{+0.12}_{-0.24}$ & $5.65^{+2.11}_{-2.50}$ & 1.17 & 60 & 29\%\\
  Warm$^b$      & $3.3^{+1.4}_{-1.6}$ & $1.9^{+0.2}_{-0.2}$ & $16.2^{+0.4}_{-0.5}$ & [1.8, 2.9] & [57, 44] & [0.9, 1.5]  & $0.97^{+0.03}_{-0.03}$ & $0.70^{+0.18}_{-0.26}$ & $3.46^{+1.10}_{-0.95}$ & 1.01 & 59 & 13\%\\
  Power-law$^c$ & 0.5--7.5            & $2.1^{+0.5}_{-0.3}$ & $22.1^{+2.7}_{-1.1}$ & [1.8, 3.0] & [54, 35] & [0.9, 1.4]  & $0.63^{+0.34}_{-0.31}$ & $0.71^{+1.48}_{-1.32}$ & $2.21^{+0.45}_{-0.30}$ & 1.00 & 58 & $< 1$\%
  \vspace{0.5mm}
  \enddata
  \tablecomments{
    For all models, $\nhgal = 2.28 \times 10^{20} ~\pcmsq$. 
    Col. (1) \xspec\ absorber models: ($a$) is \textsc{zwabs}, ($b$) is \textsc{pcfabs}, ($c$) is \textsc{pwab};
    Col. (2) Absorbing column density;
    Col. (3) Power-law index;
    Col. (4) Power-law normalization with units ($\dagger$) ph keV$^{-1}$ cm$^{-2}$ s$^{-1}$ at 1 keV;
    Col. (5) Gaussian central energies;
    Col. (6) Gaussian widths;
    Col. (7) Gaussian normalizations with units ($\ddagger$) ph cm$^{-2}$ s$^{-1}$;
    Col. (8) Model-dependent parameter: ($a$) absorber redshift, ($b$) absorber covering fraction, ($c$) absorber power law index of covering fraction;
    Col. (9) Model 0.7-7.0 keV luminosity;
    Col. (10) Unabsorbed model bolometric (0.01-100.0 keV) luminosity;
    Col. (11) Reduced \chisq\ of best-fit model;
    Col. (12) Model degrees of freedom;
    Col. (13) Percent of 10,000 Monte Carlo realizations with \chisq\ less than best-fit \chisq.
}
\end{deluxetable}

%% \begin{deluxetable}{ccccc}
  \tablecolumns{5}
  \tablewidth{0pc}
  \tablecaption{BCG Star Formation Rates.\label{tab:sfr}}
  \tablehead{
    \colhead{Source} & \colhead{ID} & \colhead{$\xi$ [Ref.]} & \colhead{$L$} & \colhead{$\psi$}\\
    \colhead{-} & \colhead{-} & \colhead{(\msolpy)/(\lum ~\phz)} & \colhead{\lum ~\phz} & \colhead{\msolpy}\\
    \colhead{(1)} & \colhead{(2)} & \colhead{(3)} & \colhead{(4)} & \colhead{(5)}}
  \startdata
  \galex\            & NUV             & $1.4 \times 10^{-28}$ [1] & $2.5 ~(\pm 0.9) \times 10^{28}$ & $3.5 \pm 1.3$\\
  \xom\              & UVW1            & $1.1 \times 10^{-28}$ [2] & $6.2 ~(\pm 1.9) \times 10^{28}$ & $6.9 \pm 2.2$\\
  \galex\            & FUV             & $1.1 \times 10^{-28}$ [2] & $8.2 ~(\pm 2.0) \times 10^{28}$ & $9.0 \pm 2.3$\\
  \galex\            & FUV             & $1.4 \times 10^{-28}$ [1] & $8.2 ~(\pm 2.0) \times 10^{28}$ & $11 \pm 3$\\
  \xom\              & UVM2            & $1.1 \times 10^{-28}$ [2] & $< 5.0 \times 10^{29}$          & $< 55$
  \enddata
  \tablecomments{A dagger ($\dagger$) indicates the removal of
    Hz$^{-1}$ from the units of $\xi$ \& $L$.  Col. (1) Source of
    measurement; Col. (2) Diagnostic identification; Col. (3)
    Conversion coefficient and references: [1] \citet{kennicutt2}, [2]
    \citet{salim2007}; Col. (4) Luminosity; Col. (5) Star formation
    rate.}
\end{deluxetable}

%% \clearpage
\begin{figure}[htp]
  \begin{center}
    \begin{minipage}[htp]{0.9\linewidth}
      \includegraphics*[width=\textwidth, trim=15mm 10mm 10mm 10mm, clip]{beta.eps}
      \caption{Surface brightness profiles for clusters requiring a
        $\beta$-model fit for deprojection (discussed in
        \S\ref{sec:beta}). The best-fit $\beta$-model for each cluster
        is overplotted as a dashed line. The discrepancy between the
        data and best-fit model for some clusters results from the
        presence of a compact X-ray source at the center of the
        cluster. These cases are discussed in Appendix
        \ref{app:beta}.}
      \label{fig:betamods}
    \end{minipage}
  \end{center}
\end{figure}
\clearpage
\begin{figure}[htp]
  \begin{center}
    \begin{minipage}[htp]{0.9\linewidth}
      \includegraphics*[width=\textwidth, trim=5mm 0mm 5mm 5mm, clip]{itplflat_rat.eps}
      \caption{Ratio of best-fit \kna\ for the two treatments of
        central temperature interpolation (see \S\ref{sec:temppr}):
        (1) temperature is free to decline across the central density
        bins ($\Delta T_{center} \ne 0$), and (2) the temperature
        across the central density bins is isothermal ($\Delta
        T_{center} = 0$). Filled black squares are clusters for which
        the \kna\ ratio is inconsistent with unity.}
      \label{fig:kcomp}
    \end{minipage}
  \end{center}
\end{figure}
\clearpage
\begin{figure}[htp]
  \begin{center}
    \begin{minipage}[htp]{0.9\linewidth}
      \includegraphics*[width=\textwidth, trim=5mm 0mm 5mm 5mm, clip]{k0res.eps}
      \caption{Best-fit \kna\ vs. redshift. Some clusters have
        \kna\ error bars smaller than the point. The clusters with
        upper-limits ({\it{black points with downward arrows}}) are:
        A2151, AS0405, MS 0116.3-0115, and RX J1347.5-1145. The
        numerically labeled clusters are: (1) M87, (2) Centaurus
        Cluster, (3) RBS 533, (4) HCG 42, (5) HCG 62, (6) SS2B153, (7)
        A1991, (8) MACS0744.8+3927, and (9) CL J1226.9+3332. For
        CLJ1226, \cite{2007ApJ...659.1125M} found best-fit $\kna = 132
        \pm 24 \ent$ which is not significantly different from our
        value of $\kna = 166 \pm 45 \ent$. The lack of $\kna < 10
        \ent$ clusters at $z > 0.1$ is most likely the result of
        insufficient angular resolution (see \S\ref{sec:angres}).}
      \label{fig:k0res}
    \end{minipage}
  \end{center}
\end{figure}
\clearpage
\begin{center}
  \begin{figure}[htp]
    \begin{minipage}[htp]{0.5\linewidth}
      \includegraphics*[width=\textwidth, trim=28mm 7mm 30mm 17mm, clip]{curvk0.eps}
    \end{minipage}
    \begin{minipage}[htp]{0.5\linewidth}
      \includegraphics*[width=\textwidth, trim=28mm 7mm 30mm 17mm, clip]{nbins_k0.eps}
    \end{minipage}
    \begin{minipage}[htp]{0.5\linewidth}
      \includegraphics*[width=\textwidth, trim=28mm 7mm 30mm 17mm, clip]{texpk0.eps}
    \end{minipage}
    \begin{minipage}[htp]{0.5\linewidth}
      \includegraphics*[width=\textwidth, trim=28mm 7mm 30mm 17mm, clip]{ntxbins_k0.eps}
    \end{minipage}
    \caption{Plots of possible systematics versus best-fit \kna.
      {\it{Top left:}} Best-fit \kna\ plotted versus average curvature
      of the corresponding entropy profile (see eq. \ref{eqn:avgcurv})
      There is no trend between these two quantities suggesting that
      \kna\ is not heavily influenced by the total shape of the
      entropy profile. {\it{Top right:}} Best-fit \kna\ plotted versus
      number of bins in the entropy profile which were used during
      fitting. Again, no trend is found. {\it{Bottom left:}} Best-fit
      \kna\ plotted versus the total used exposure time for each
      cluster. No trend is found. {\it{Bottom right:}} Best-fit
      \kna\ plotted versus the number of bins in the temperature
      profile for each cluster. As expected, fewer $\Tx(r)$ does not
      correlate with \kna.}
    \label{fig:sys}
  \end{figure}
\end{center}
\clearpage
\begin{center}
  \begin{figure}[htp]
    \begin{minipage}[htp]{0.5\linewidth}
      \includegraphics*[width=\textwidth, trim=28mm 7mm 30mm 17mm, clip]{splots_allt.eps}
    \end{minipage}
    \begin{minipage}[htp]{0.5\linewidth}
      \includegraphics*[width=\textwidth, trim=28mm 7mm 30mm 17mm, clip]{splots_tle4.eps}
    \end{minipage}
    \begin{minipage}[htp]{0.5\linewidth}
      \includegraphics*[width=\textwidth, trim=28mm 7mm 30mm 17mm, clip]{splots_gt4tle8.eps}
    \end{minipage}
    \begin{minipage}[htp]{0.5\linewidth}
      \includegraphics*[width=\textwidth, trim=28mm 7mm 30mm 17mm, clip]{splots_tgt8.eps}
    \end{minipage}
    \caption{Composite plots of entropy profiles for varying cluster
      temperature ranges. Profiles are color-coded based on average
      cluster temperature. Units of the color bars are keV. The solid
      line is the pure-cooling model of \cite{voitbryan}, the dashed
      line is the mean profile for clusters with $\kna \le 50 \ent$,
      and the dashed-dotted line is the mean profile for clusters with
      $\kna > 50 \ent$. {\it{Top left:}} This panel contains all the
      entropy profiles in our study. {\it{Top right:}} Clusters with
      $kT_X < 4$ keV. {\it{Bottom left:}} Clusters with $4\keV < kT_X
      < 8\keV$. {\it{Bottom right:}} Clusters with $kT_X > 8$
      keV. Note that while the dispersion of core entropy for each
      temperature range is large, as the $kT_X$ range increases so to
      does the mean core entropy.}
    \label{fig:splots}
  \end{figure}
\end{center}
\clearpage
\begin{figure}[htp]
  \begin{center}
    \begin{minipage}[htp]{0.9\linewidth}
      \includegraphics*[width=\textwidth, trim=20mm 10mm 10mm 10mm, clip]{k0hist.eps}
      \caption{{\it{Top panel:}} Histogram of best-fit \kna\ for all
        the clusters in \accept. Bin widths are 0.15 in log space.
        {\it{Bottom panel:}} Cumulative distribution of \kna\ values
        for the full sample. The distinct bimodality in \kna\ is
        present in both distributions, which would not be seen if it
        were an artifact of the histogram binning. A KMM test finds
        the \kna\ distribution cannot arise from a simple unimodal
        Gaussian.}
      \label{fig:k0hist}
    \end{minipage}
  \end{center}
\end{figure}
\clearpage
\begin{figure}[htp]
  \begin{center}
    \begin{minipage}[htp]{0.9\linewidth}
      \includegraphics*[width=\textwidth, trim=20mm 10mm 10mm 10mm, clip]{hifl_k0hist.eps}
      \caption{{\it{Top panel:}} Histogram of best-fit \kna\ values
        for the primary \hifl\ sample. Bin widths are 0.15 in log
        space.  {\it{Bottom panel:}} Cumulative distribution of
        best-fit \kna\ values. The distinct bimodality seen in the
        full \accept\ sample (Fig. \ref{fig:k0hist}) is also present
        in the \hifl\ subsample and shares the same gap between the
        low-entropy peak at 10-20 \ent\ and the high-entropy peak at
        100-200 \ent. That bimodality is present in both samples is
        strong evidence it is not a result of an unknown archival
        bias.}
      \label{fig:hiflk0}
    \end{minipage}
  \end{center}
\end{figure}
\clearpage
\begin{figure}[htp]
  \begin{center}
    \begin{minipage}[htp]{0.8\linewidth}
      \includegraphics*[width=\textwidth, trim=20mm 10mm 10mm 10mm, clip]{t0.eps}
    \end{minipage}
    \begin{minipage}[htp]{0.8\linewidth}
      \includegraphics*[width=\textwidth, trim=20mm 10mm 10mm 10mm, clip]{k0cool.eps}
    \end{minipage}
    \caption{{\it{Top panel:}} Log-binned histogram and cumulative
      distribution of best-fit core cooling times, $t_{c0}$
      (eqn. \ref{eqn:tc0}), for all the clusters in \accept. Histogram
      bin widths are 0.2 in log space. {\it{Bottom panel:}} Log-binned
      histogram and cumulative distribution of core cooling times
      calculated from best-fit \kna\ values, $t_{c0}(\kna)$
      (eqn. \ref{eqn:tck0}), for all the clusters in
      \accept. Histogram bin widths are 0.2 in log space. The
      bimodality we observe in the \kna\ distribution is also present
      in best-fit $t_{c0}$. However, the gaps between the two
      populations of $t_{c0}$ and $t_{c0}(\kna)$ differ by $\sim 0.3$
      Gyrs which may be an artifact of the binning.}
    \label{fig:t0}
  \end{center}
\end{figure}



%%%%%%%%%%%%%%%%%%%%
% End the document %
%%%%%%%%%%%%%%%%%%%%

\end{document}

%% Conservation of angular momentum suggests that gas condensing into
%% the very center of a BCG has a preferred direction of rotation,
%% possibly aligned with the semi-major axis of the cluster. The
%% prevalence of systems like \rbs, \ms, and Hercules A is unclear,
%% but their numbers thus far are small, possibly because their BCGs,
%% for whatever reason, formed SMBHs that at some point in the past
%% spun retrograde relative to the preferred rotation direction of the
%% cluster. In which case, if the GES model is assumed, they would
%% undergo a very powerful AGN outburst which releases $\ga 10^{61}$
%% erg of spin energy. If the circumstances which result in retrograde
%% SMBHs are rare, then this naturally explains why not many such
%% systems have been found.

%% To help visualize the X-ray emission for a cluster with cavities
%% inflating along the line-of-sight, a hydrodynamical cavity
%% formation simulation was performed with {\textsc{FLASH}} version
%% 3.0 \citep{flash}. The simulation was tailored to \rbs\ by using
%% the best-fit $\beta$-model as the cluster atmosphere, and the the
%% O1-O2 parameters listed in Table \ref{tab:cavities}. The mesh
%% refinement criteria were the standard density and pressure, with a
%% minimum cell size of 0.66 kpc on a grid of $1024^3$ zones. The
%% simulation was started by injecting energy into two spheres of
%% radius XX kpc at distances of XX kpc from the cluster center. The
%% gas in the spheres was heated and expanded similar to a Sedov
%% explosion to form a pair of bubbles in a few Myr, a time much
%% shorter than the buoyant rise time of the bubbles. The simulation
%% ran until the bubbles reached a radius of XX kpc and a density
%% contrast with the ambient ICM of $\approx X.XX$
%% \citep[see][]{2009arXiv0909.1805S}. To aide direct comparison with
%% the \cxo\ data, the simulation output was processed with the
%% {\textsc{XIM}}\footnote{http://www.astro.wisc.edu/~heinzs/XIM}
%% X-ray imaging pipeline \citep{2009arXiv0903.0043H}.

%% Shown in Figure \ref{fig:sims} are images of the simulated,
%% projected ICM X-ray emission. The simulated image corresponding to
%% the cavities lying along the line-of-sight has much in common with
%% the real X-ray image of \rbs: 1) a bright, azimuthally symmetric
%% rim in the core region arises from the conjunction of the cavities
%% in projection, 2) there are faint depressions at the leading edge
%% of each cavity which may be analogs of E2, and 3) surface
%% brightness analysis of the cavities reveals decrements of $\la 0.5$
%% {\bf{(XX: True?)}}. There are also some inconsistencies with the
%% real X-ray images: the presence of a distinct depression at W2, the
%% core resembling a bullseye, and the much lower core
%% density. However, the simulation represents symmetric cavities
%% within an idealized cluster atmosphere. Real cavities are known to
%% entrain denser material and deform they as move away from the
%% AGN. Also, asymmetries in the large-scale and core ICM gas
%% distributions have not be considered, and may explain the
%% differences between the \cxo\ observations and the
%% simulation. Taking these caveats into consideration, we find a
%% line-of-sight cavity configuration plausible.

%% Because measurements made from the BTA spectrum are highly
%% uncertain, they are only discussed for completeness. Wavelength
%% calibration was applied using a He-Ar-Ne lamp, and sky subtraction
%% was performed pixel-wise along each CCD column of regions 25 CCD
%% rows wide below and above the trace of the spectrum. Spectral
%% features were located and fit with tools in the
%% \iraf\ {\textsc{onedspec}} package.

%% Using the tool {\textsc{omdetect}}, an unresolved source co-spatial
%% with \rbs\ was detected in the UVW1 image, but no corresponding
%% source was detected in the UVM2 image. A $3\sigma$ upper limit on
%% the UVM2 flux was calculated using the UVW1 source as a guide. For
%% each filter, mean count rates for the source region were measured
%% using the tool {\textsc{omsource}}, converted to instrumental
%% magnitudes, and then to fluxes, giving

%% \begin{eqnarray}
%%   B_d &=& 3.00 \times 10^8 \left(\frac{\mbh}{\msol}\right)^{-1/2} \frac{R_g}{r}\\
%%   &=& 3.03 \times 10^7 ~\lmdot^{2/5} \alpha^{1/20}
%%   \left(\frac{\mbh}{\msol}\right)^{-9/20}\\
%%   &=& 4.17 \times 10^9 ~\lmdot^{2/5} \left(\frac{\alpha \mbh}{\msol}\right)^{-9/20}
%%   \left(\frac{R_g}{r}\right)^{13/10}\\
%%   &=& 6.55 \times 10^8 ~\lmdot^{1/2} \left(\frac{\alpha
%%     \mbh}{\msol}\right)^{-1/2} \left(\frac{R_g}{r}\right)^{5/4}
%% \end{eqnarray}
%% where $R_g = 2G\mbh/c^2$ is the Schwarzschild radius, $r$ is the
%% radial location in the disk, \lmdot\ is mass accretion rate in units
%% of \dmedd, and \alpha\ is the disk viscosity.
