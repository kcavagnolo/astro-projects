Utilizing $\sim 50$ ks of Chandra X-ray Observatory imaging, we present an analysis of the intracluster medium (ICM) and cavity system in the galaxy cluster \rbs. In addition to the two previously known cavities in the cluster core, the new and deeper X-ray image has revealed additional structure associated with the active galactic nucleus (AGN). The surface brightness decrements of the two cavities are unusually large, and are consistent with elongated cavities lying close to our line-of-sight. We estimate a total AGN outburst energy and mean jet power of $\approx 3 \dash 6 \times 10^{60}$ erg and $\approx 3 \dash 6 \times 10^{45} ~\lum$, respectively, depending on the assumed geometrical configuration of the cavities. Thus, \rbs\ is apparently among the the most powerful AGN outbursts known in a cluster. The average mass accretion rate needed to power the AGN by accretion alone is $\sim 1 ~\msolpy$. We show that accretion of cold gas onto the AGN at this level is plausible, but that Bondi accretion of the hot atmosphere is probably not. The BCG harbors an unresolved, non-thermal nuclear X-ray source with a bolometric luminosity of $\approx 2 \times 10^{44} ~\lum$. The nuclear emission is probably associated with a rapidly-accreting, radiatively inefficient accretion flow. We present tentative evidence that star formation in the BCG is being triggered by the radio jets and suggest that the cavities may be driving weak shocks ($M \sim 1.5$) into the ICM, similar to the process in the galaxy cluster \ms. 
