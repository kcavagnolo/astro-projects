\documentclass[11pt]{article}
\setlength{\topmargin}{-.3in}
\setlength{\oddsidemargin}{-0.1in}
\setlength{\evensidemargin}{-0.1in}
\setlength{\textwidth}{6.7in}
\setlength{\headheight}{0in}
\setlength{\headsep}{0in}
\setlength{\topskip}{0.5in}
\setlength{\textheight}{9.25in}
\setlength{\parindent}{0.0in}
\setlength{\parskip}{1em}
\usepackage{common,graphicx,hyperref,epsfig}

\pagestyle{empty}

\begin{document}

Dear Ari Laor:

Below is our reply to the referee's report for manuscript ApJ82716 ``A
Powerful AGN Outburst in RBS 797.'' The referee's comments are
{\it{italicized}} and our replies are not. We have found the comments
helpful in improving the focus and discussion of the manuscript, and
thank the referee for their time.

\hrulefill

{\it{Page 4, 1st paragraph: Are some of these directions round the
    wrong way? Shouldn't ``northwest-southeast'' be
    ``northeast-southwest''? Is it not the eastern cavity that has
    more internal structure and has less well-defined cavities than
    the western cavity?}}

The coordinates are shown projected on the plane of the sky, hence
west is to the right of north (which is at the top of the image).

\hrulefill

{\it{Page 4, final paragraph: ``but no counterpart'' - maybe should be
    ``but no radio counterpart''?}}

The word ``radio'' has been added as it does make the description more
clear.

\hrulefill

{\it{Page 5, final paragraph: ``pronounced entropy core'' - should
    that be ``pronounced low entropy core''?}}

The word ``low'' has been added since it more clearly specifies the
nature of the core without consulting the figure.

\hrulefill

{\it{Fig 3: The caption has left and bottom panels
    mentioned. Presumably this should be left and right.}}

We thank the referee for catching this holdover from an earlier draft
version.

\hrulefill

{\it{Fig 4: are the blips around 200-300 kpc in the surface brightness
    profile real or is it some sort of point source subtraction
    issue?}}

All detectable or visible point sources were removed from the analysis
prior to extracting any radial profiles. The variations in the surface
brightness profile are interesting, and some of them do coincide with
possible X-ray features. However, for the sake of clarity and brevity,
we have reserved discussion of additional structure for later
investigations which will hopefully incorporate deeper Chandra
observations.

\hrulefill

{\it{Fig 10: Is it possible to get a better colour scale for this
    image? It's a bit muddy and you can't see the features very
    well. Maybe one like Fig 2 or 1 might be better.}}

We tried multicolor schemes, however, to make the features stand-out,
the contrast in values had to be very narrow, making the assignment of
colors to numerical values uninformative. The only color scheme where
this was not an issue were the ``monotone'' family (e.g. hot, cold,
blue, red, etc.). In order to preserve clarity of value assignment, we
thus stuck with the version from manuscript 1.

\end{document}
