%%%%%%%%%%%%%%%%%%%
% Custom commands %
%%%%%%%%%%%%%%%%%%%

\newcommand{\dmb}{\ensuremath{\dot{m_{\rm{B}}}}}
\newcommand{\dme}{\ensuremath{\dot{m_{\rm{E}}}}}
\newcommand{\ldisk}{\ensuremath{L_{\rm{disk}}}}
\newcommand{\lfir}{\ensuremath{L_{\rm{FIR}}}}
\newcommand{\lfuv}{\ensuremath{L_{\rm{FUV}}}}
\newcommand{\lnuv}{\ensuremath{L_{\rm{NUV}}}}
\newcommand{\ms}{MS 0735.6+7421}
\newcommand{\myi}{\ensuremath{I^{\prime}}}
\newcommand{\myv}{\ensuremath{V^{\prime}}}
\newcommand{\radec}{R.A.(J2000) $=09^h 47^m 12^s.8$, Dec.(J2000)
  $=+76\mydeg 23^{\arcm} 13^{\arcs}.66$}
\newcommand{\rbs}{R797}
\newcommand{\reff}{\ensuremath{r_{\rm{eff}}}}
\newcommand{\rlos}{\ensuremath{r_{\rm{los}}}}
\newcommand{\rxj}{RX J0947.2+7623}
\newcommand{\mykeywords}{cooling flows -- galaxies: clusters:
  individual (\rxj) -- galaxies: individual (RBS 797)}
\newcommand{\mystitle}{\rbs\ AGN Outburst}
\newcommand{\mytitle}{The \rbs\ AGN Outburst: Mass Accretion Alone
  Versus Supermassive Black Hole Spin}

%%%%%%%%%%
% Header %
%%%%%%%%%%

%% \documentclass[12pt, preprint]{aastex}
%% \usepackage{graphicx,common}
\documentclass{emulateapj}
\usepackage{apjfonts,graphicx,common}
\usepackage[pagebackref,
  pdftitle={\mytitle},
  pdfauthor={K. W. Cavagnolo},
  pdfsubject={Astrophysics},
  pdfkeywords={\mykeywords},
  pdfproducer={LaTeX with hyperref},
  pdfcreator={LaTeX}
  pdfdisplaydoctitle=true,
  colorlinks=true,
  citecolor=blue,
  linkcolor=blue,
  urlcolor=blue]{hyperref}
\bibliographystyle{apj}
\begin{document}
\title{\mytitle}
\shorttitle{\mystitle}
\author{
  K. W. Cavagnolo\altaffilmark{1,7},
  B. R. McNamara\altaffilmark{1,2,3},
  P. E. J. Nulsen\altaffilmark{3},\\
  M. Gitti\altaffilmark{3,4},
  M. Br\"uggen\altaffilmark{5},
  and M. W. Wise\altaffilmark{6}
}
\shortauthors{Cavagnolo et al.}

\altaffiltext{1}{Department of Physics \& Astronomy, University of
  Waterloo, 200 University Ave. W., Waterloo, Ontario, N2L 3G1,
  Canada.}
\altaffiltext{2}{Perimeter Institute for Theoretical Physics, 31
  Caroline St. N., Waterloo, Ontario, N2L 2Y5, Canada.}
\altaffiltext{3}{Harvard-Smithsonian Center for Astrophysics, 60
  Garden St., Cambridge, Massachusetts, 02138-1516, United States.}
\altaffiltext{4}{INAF - Astronomical Observatory of Bologna, Via
  Ranzani 1, I-40127 Bologna, Italy.}
\altaffiltext{5}{Jacobs University Bremen, P.O. Box 750561, 28725
  Bremen, Germany.}
\altaffiltext{6}{Astronomical Institute Anton Pannekoek, Science Park
  904, P.O. Box 94249, 1090 GE Amsterdam, The Netherlands.}
\altaffiltext{7}{kcavagno@uwaterloo.ca}

%%%%%%%%%%%%
% Abstract %
%%%%%%%%%%%%

\begin{abstract}
\end{abstract}

%%%%%%%%%%%%
% Keywords %
%%%%%%%%%%%%

\keywords{\mykeywords}

%%%%%%%%%%%%%%%%%%%%%%
\section{Introduction}
\label{sec:intro}
%%%%%%%%%%%%%%%%%%%%%%

%% The observational and theoretical study of the interaction between
%% active galactic nuceli (AGN), their host galaxy, and the environments
%% surrounding the host galaxy has increased rapidly in the last decade
%% with the launch of the \chandra\ X-ray Observatory (CXO). For example,
%% high-resolution X-ray observations of the interaction between AGN and
%% the hot, diffuse gas in galaxy clusters, the intracluster medium
%% (ICM), has helped further our understanding of how feedback, cooling,
%% and heating might couple in massive systems, and has shed light on how
%% these same processes may operate in lower mass systems. By measuring
%% the properties of AGN-excavated cavities in the ICM, the mechanical
%% energy output and mean jet power associated with an AGN can be
%% estimated. Under some basic assumptions, these quantities can then be
%% used to estimate the growth rate of the black hole engine of an AGN,
%% the gas accretion rate required to sustain an AGN outburst which
%% formed the cavities, and, most importantly, the total energy budget of
%% the cluster can be more accurately calculated. It has been shown that,
%% in some clusters, the energy putput by an AGN is comparable, and in
%% some cases in excess, of the energy needed to quench gas cooling out
%% of the ICM.

%% For one rare object, \ms, \citet{ms07} presented analaysis that showed
%% the energy released by the AGN outburst was so large that a cooling
%% flow would easily be quenched, and that the gas mass required to power
%% the outburst through accretion was implausibly high. For this reason,
%% \citet{ms07} have argued that the outburst in \ms\ was not powered
%% through gas accretion, but through the release of black hole spin
%% energy. The \citet{ms07} spin power hypothesis is a tantalizing one,
%% but the hypothesis is based mostly in theory with \ms\ and Hydra
%% Abeing the only objects where spin power is arguably the best
%% explanation. To further test the spin power hypothesis, which may be
%% the dominant or only powering mechanism for outbursts with an energy
%% $\ga 10^{61}$, very powerful AGN outbursts needed to be identified so
%% that detailed analysis can be used to confront the current models of
%% both accretion power and spin power. But a pair of troublesome problem
%% underly the hunt for very powerful AGN outbursts: (1) the very large
%% cavity systems associated with $\ga 10^{61}$ outbursts are difficult
%% to observe given the low contrast of such systems, and (2) projection
%% effects prevent us from fully grasping the size and distance from the
%% host galaxy of cavity systems.

%% In this paper we present an analysis of the peculiar cavity system in
%% \rbs. Using radio, infrared (IR), optical, ultraviolet (UV), and X-ray
%% data, we argue that the AGN outburst in \rbs\ is nearly along the line
%% of sight, and that the outburst was likely powered by the spin
%% mechanism. \rbs\ (also know as \rxj) is a poor(???) galaxy cluster
%% located at \radec\ and $z = 0.354$. The brightest cluster galaxy (BCG)
%% in \rbs\ was identified as a LINER by \citet{rbs1} using a sample of
%% targets taken from the \rosat\ Hard Survey. \rbs\ was subsequently
%% reclassified as a cluster in \citet{rbs2} with the use of an optical
%% spectrum which was interpreted as having emission lines associated
%% with a cooling flow. A 10 ks follow-up with CXO by \citet{schindler01}
%% revealed a pair of compact X-ray cavities in the ICM of \rbs.

%% \rbs\ is the BCG of the galaxy cluster \rxj. The most striking feature
%% of \rxj\ is the extremely deep cavity system which appears confined to
%% the central 50 kpc. X-ray cavities are thought to be inflated by jets
%% from the central AGN that inject energy into small regions at their
%% terminal points, which expand until they reach pressure equilibrium
%% with the surrounding ICM \citep{1974MNRAS.169..395B}. The result is a
%% pair of underdense, hot bubbles on opposite sides of the cluster
%% center.

Data reduction is presented in Section \ref{sec:obs}, results are
given in Section \ref{sec:results}, and a brief summary concludes the
paper in Section \ref{sec:con}. \LCDM\ throughout. At a redshift of
$z = 0.354$, the lookback time is 3.9 Gyr, $\da = 4.996$ kpc
arcsec$^{-1}$, and $\dl = 1889$ Mpc. All errors are 68\% confidence
unless stated otherwise.

%%%%%%%%%%%%%%%%%%%%%%%%%%%%%%%%%%%%%%%%%
\section{Observations and Data Reduction}
\label{sec:obs}
%%%%%%%%%%%%%%%%%%%%%%%%%%%%%%%%%%%%%%%%%

%%%%%%%%%%%%%%%%%%
\subsection{X-ray}
%%%%%%%%%%%%%%%%%%

\rxj\ was observed with the Chandra X-ray Observatory (\cxo) in
October 2000 for 11.4 ks using the ACIS-I array (\dataset
[ADS/Sa.CXO#Obs/02202] {ObsID 2202}; PI Schindler) and July 2007 for
38.3 ks using the ACIS-S array (\dataset [ADS/Sa.CXO#Obs/07902] {ObsID
  7902}; PI McNamara). Datasets were reduced using \ciao\ and
\caldb\ versions 4.2. Events were screened using \asca\ grades, and
cosmic rays were further rejected via {\textsc{vfaint}} filtering. The
level-1 events files were reprocessed to apply corrections for the
time-dependent gain change, charge transfer inefficiency, and degraded
quantum efficiency. The afterglow and dead area corrections were
applied. Time intervals affected by background flares exceeding 20\%
of the mean background count rate were isolated and removed using
light curve filtering. The final, combined exposure time was 48.8
ks. Point sources were identified and removed via visual inspection
and use of the \ciao\ tool {\textsc{wavdetect}}. We refer to the
\cxo\ data free of point sources and flares as the ``clean'' data. A
mosaiced, fluxed image was generated by exposure correcting each clean
dataset and reprojecting the normalized images to a common tangent
point (see Figure \ref{fig:img}). The X-ray data is in Sections
\ref{sec:icm}, \ref{sec:cavities}, and \ref{sec:accretion} to study
radial ICM properties, the cavity system, and properties of the BCG
point source.

%%%%%%%%%%%%%%%%%%%%%%%%
\subsection{Ultraviolet}
%%%%%%%%%%%%%%%%%%%%%%%%

The UV measurements presented below are utilized in Section
\ref{sec:bcg} to constrain the \rbs\ star formation rate. \rbs\ was
imaged in the far-UV (FUV; 1344--1786 \AA; FWHM $\approx 4.5\arcs$)
and near-UV (NUV; 1771--2831 \AA; FWHM $\approx 6.0\arcs$) for 136 s
by the Galaxy Evolution Explorer (\galex) in December 2003. \rbs\ is
detected in both filters, unresolved, and cataloged as GALEX
J094712.4+762313. The FUV, NUV, optical, and X-ray sources are
co-spatial, and the next nearest UV source is at a distance of
$19\arcs$, indicating the detection is unlikely to be a spurious
source. The FUV and NUV source fluxes are $19.2 \pm 4.8 ~\mu$Jy and
$5.9 \pm 2.1 ~\mu$Jy, respectively.

The XMM-Newton Optical-Monitor (\xom) imaged \rbs\ in April 2008 for
4.5 ks (DataID 0502940301; PI M. Gitti). Exposures were taken with two
filters: UVW1 (2410--3565 \AA; FWHM $\approx 2.0\arcs$) and UVM2
(1970--2675 \AA; FWHM $\approx 1.8\arcs$). The data was processed
using \sas\ version 8.0.1 and the task {\textsc{omichain}}. Using the
tool {\textsc{omdetect}}, an unresolved source co-spatial with
\rbs\ was detected in the UVW1 image, but no corresponding source was
detected in the UVM2 image. A $3\sigma$ upper limit on the UVM2 flux
was calculated using the UVW1 source as a guide. For each filter, mean
count rates for the source region were measured using the tool
{\textsc{omsource}}, converted to instrumental magnitudes, and then to
fluxes, giving $f_{\rm{UVW1}} = 14.6 \pm 4.6$ $\mu$Jy and
$f_{\rm{UVM2}} < 117$ $\mu$Jy.

%%%%%%%%%%%%%%%%%%%%
\subsection{Optical}
%%%%%%%%%%%%%%%%%%%%

\rbs\ was imaged for 1.2 ks with the Hubble Space Telescope (\hst)
ACS/WFC instrument in March and December 2006 (prop IDs 10491 \&
10875; PI H. Ebeling). The F606W (4500--7500 \AA) and F814W
(6800--9800 \AA) filters were used during the observations, and the
respective filters are denoted as \myv\ and \myi\ to distinguish from
the standard Johnson $V$ and $I$ filters. The cosmic ray cleaned,
drizzled images output by the Hubble Legacy Archive pipeline were used
for analysis. The central $1.5\arcs$ of the BCG contains some
substructure, with the central $0.25\arcs$ having the appearance of
being pinched into two equal brightness sources. It is unclear which,
if either, is at the center of the galaxy or might be associated with
an AGN. Two ACS ghosts \citep{acsghost} are present in the
\myi\ image. The northern ghost overlaps with light from the BCG
beginning at $\approx 2.7\arcs$ from the galaxy center, and the
western ghost at $\approx 5.4\arcs$. Within a circular aperture
centered on \rbs\ ($r = 2\arcs$), the measured magnitudes are
$m_{\myv} = 19.3 \pm 0.7$ mag and $m_{\myi} = 18.2 \pm 0.6$ mag,
consistent with the results of \citet{rbs1} suggesting the photometry
in this region is not impacted by the ghosts. The \hst\ images are
used in Section \ref{sec:bcg} to dissect the BCG structure and
investigate interaction of the central AGN with the \rbs\ halo.

Optical spectroscopy (see below) reveals strong emission lines of
\hbeta, [O \Rmnum{2}], [O \Rmnum{3}], and presumably
\halpha\ associated with the region around \rbs. These lines lie
within the \hst\ passbands, and their contribution to the \hst\ images
was determined using the ratio of line equivalent widths (EWs) to
passband widths. The \halpha\ contribution to the \myi\ image was
estimated by scaling \hbeta\ using a Balmer decrement of
EW$_{\hbeta}$/EW$_{\halpha}$ = 0.29 \citep{2006ApJ...642..775M}. The
\halpha\ line contribution was estimated at $\approx 3\%$ for the
\myi\ filter, which is likely an overestimate since the stellar
continuum rises toward the red. The \hbeta, \oii, and
\oiii\ contribution to the \myv\ image is $\approx 7\%$. The low line
contamination indicates the stellar continuum is well-represented in
the observations.

A spectrum (3200-7600 \AA) of \rbs\ was obtained with the longslit
spectrograph on the Bolshoi Altazimuth Telescope (BTA) in February
1996 \citep{rbs1}. The slit setup is unknown, and flux calibration was
applied using the average of response functions for a set of standard
stars, resulting in wavelength-dependent photometric uncertainties
$\ga 50\%$. For these reasons, measurements made from the BTA spectrum
are highly uncertain and are only discussed for
completeness. Wavelength calibration was applied using a He-Ar-Ne
lamp, and sky subtraction was performed pixelwise along each CCD
column of regions 25 CCD rows wide below and above the trace of the
spectrum. Spectral features were located and fit with tools in the
\iraf\ {\textsc{onedspec}} package.

%%%%%%%%%%%%%%%%%%
\subsection{Radio}
%%%%%%%%%%%%%%%%%%

Very Large Array (\vla) radio images at 325 MHz (A-array), 1.4 GHz (A-
and B-array), 4.8 GHz (A-array), and 8.4 GHz (D-array) were analyzed
and discussed in \citet{gitti06} and \citet{birzan08}. Our re-analysis
of the archival \vla\ radio observations yielded no significant
differences with the previous studies. Using rms noise values for each
frequency given in \citet{gitti06} and \citet{birzan08}, emission
contours between $3\sigma_{\rm{rms}}$ and the peak image intensity
were generated. These are the contours referenced and shown in all
subsequent discussion and figures.

%%%%%%%%%%%%%%%%%%%
\section{Results}
\label{sec:results}
%%%%%%%%%%%%%%%%%%%

%%%%%%%%%%%%%%%%%%%%%%%%%%%%%%%%%%
\subsection{Radial ICM Properties}
\label{sec:icm}
%%%%%%%%%%%%%%%%%%%%%%%%%%%%%%%%%%

In this section we present radial analysis of the \rxj\ ICM, and
quantify properties of the cluster core and gas surrounding the
cavities. The radial profiles discussed below are shown in Figure
\ref{fig:gallery}. The ICM temperature (\tx) structure was determined
by fitting spectra extracted from concentric circular annuli
containing 2500 source counts centered on the cluster X-ray peak. For
each annulus, weighted responses were created and background spectra
were extracted from the ObsID matched \caldb\ blank-sky dataset. The
background data was renormalized using the ratio of 9--12 keV count
rates for identical off-axis, source-free regions of the blank-sky and
target datasets. Using the methodology of \citet{2005ApJ...628..655V},
a spectral model for the Galactic background was included as an
additional, fixed background component during spectral fitting. Source
spectra were binned to 25 counts per energy channel. Fitting was
performed with \xspec\ 12.4 \citep{xspec} using an absorbed,
single-component \mekal\ model \citep{mekal1} over the energy range
0.7-7.0 \keV. The absorbing Galactic column density was set to $\nhi =
2.28 \times 10^{20} ~\pcmsq$ \citep{lab} and assumed not to vary over
the cluster. Gas abundance was free to vary and normalized to the
solar ratios of \citet{ag89}. We discuss projected quantities only as
spectral deprojection using the {\textsc{proj}} \xspec\ model did not
produce significantly different results.

A 0.7-2.0 keV surface brightness (SB) profile was extracted from the
mosaiced, clean image using concentric $1\arcs$ wide elliptical annuli
centered on the BCG X-ray point source (central $\approx 1\arcs$ and
cavities excluded). The SB profile was fitted with a $\beta$-model
\citep{betamodel} which had best-fit parameters $S_0 = 1.65 ~(\pm
0.15) \times 10^{-3}$ \sbr, $\beta = 0.62 \pm 0.04$, and
$r_{\rm{core}} = 7.98\arcs \pm 0.08$ for \chisq(DOF) = 79(97). A
deprojected electron density (\nelec) profile was derived from the SB
profile using the method of \citet{kriss83} which incorporates the
0.7-2.0 keV spectral count rates and best-fit spectral
normalizations. The spectral quantities were transfered from the
coarser \tx\ bins to the finer SB bins via linear
interpolation. Errors for \nelec\ were estimated using 5000 Monte
Carlo simulations of the original SB profile. A total gas pressure
($P$) profile was calculated as $P = n \tx$ where $n \approx 2.3 \nH$
and $\nH \approx \nelec/1.2$.

There is a significant \nelec\ discontinuity at $r \approx 60$ kpc
(insignificant $P$ and \tx\ discontinuities) which is coincident with
an elliptical ridge of X-ray emission enclosing the cavities. During
the formation of a cavity, lower entropy gas can be entrained or
displaced from deep within the core, resulting in bright cavity rims
\citep[\eg][]{2009ApJ...697L..95B}. However, gas shocking can also
create rim brightening \citep[\eg][]{ms0735}. After extracting spectra
on either side of the ridge using regions with more source counts than
those of the \tx\ profile, no significant \tx\ jump was detected to
signal the presence of a shock. Assuming there is a unresolved shock,
the Rankine-Hugoniot density jump relation predicts a Mach number of
$\approx 1.3$. In Section \ref{sec:cavities} we discuss the impact of
the bright rims on assessment of the cavity depths.

Profiles of entropy, $K = \tx \nelec^{-2/3}$, and cooling time,
$\tcool = 3 n \tx~[2 \nelec \nH \Lambda(T,Z)]^{-1}$ where
$\Lambda(T,Z)$ is the cooling function, were also generated. Errors
for each profile were determined by summing the individual parameter
uncertainties in quadrature. Fitting the $K$ profile with the function
$K = \kna +\khun (r/100 ~\kpc)^{\alpha}$ \citep{accept} yielded
best-fit parameters of $\kna = 17.9 \pm 2.2 ~\ent$, $\khun = 92.1 \pm
6.2 ~\ent$, and $\alpha = 1.65 \pm 0.11$. \rxj\ is a cool core cluster
with a central cooling time $< 0.5$ Gyr and $\kna < 30
~\ent$. Clusters with these core properties typically host line
emitting BCGs, some of which have nuclear star formation
\citep{crawford99, haradent, rafferty08}. Partially consistent with
this picture, we show in Section \ref{sec:bcg} that whisps of infrared
emission, which could be \halpha\ filaments, radiate from \rbs, but
that plausible evidence of star formation is predominately found in
areas of the BCG halo directly interacting with the central AGN.

%%%%%%%%%%%%%%%%%%%%%%%%%%%%%%%%%%%%%%%%%%%%%%
\subsection{ICM Cavities and the AGN Outburst}
\label{sec:cavities}
%%%%%%%%%%%%%%%%%%%%%%%%%%%%%%%%%%%%%%%%%%%%%%

A montage of radio, optical, and X-ray images is presented in Figure
\ref{fig:composite} to provide guidance for this section. Typically,
the connection between cavities, coincident radio emission, and an AGN
is unambiguous, but this is not the case for \rxj. As seen in
projection, the nuclear 4.8 GHz jets are almost orthogonal to the
cavity axis, and the 325 MHz, 1.4 GHz, \& 8.4 GHz radio emission are
more that of a radio mini-halo than relativistic plasma confined to
the cavities (Doria et al., in preparation). To better reveal the
cavity morphologies and aide in disentangling the radio-X-ray
relationship, residual X-ray images were constructed.

The X-ray isophotes of two exposure-corrected images -- one smoothed
by a $1\arcs$ Gaussian and another by a $3\arcs$ Gaussian -- were
fitted with ellipses using the \iraf\ task \textsc{ellipse}. The
ellipse centers were fixed at the location of the BCG X-ray point
source, and the eccentricities and position angles were free to
vary. A 2D SB model was created from each fit using the \iraf\ task
\textsc{bmodel}, normalized to the parent X-ray image, and then
subtracted off. The residual images are shown in Figure
\ref{fig:subxray}.

In addition to the central east and west cavities (labeled E1 \& W1),
depressions north and south (labeled N1 \& S1) of the nucleus are
revealed. N1 and S1 lie along the 4.8 GHz jet axis and are coincident
with spurs of significant 1.4 GHz emission, indicating they may be
related to activity of the central AGN. A depression (labeled E2)
coincident with the southeastern concentration of 325 MHz emission is
also found, but no counterpart (labeled W2) on the opposing side of
the cluster is seen. There is an X-ray edge which extends southwest
from E2 and sits along a ridge of 325 MHz and 8.4 GHz emission. No
substructure associated with the westernmost knot of 325 MHz emission
is found.  However, there is a stellar object co-spatial with this
region, and the X-ray and radio properties -- for $D < 1$ kpc, $\lx
\la 10^{31} ~\lum$ and $L_{325} \sim 10^{27} ~\lum$ -- are consistent
with the presence of a galactic RS CVn star
\citep{1993RPPh...56.1145S}.

The cavity decrements ($y$) are defined as the ratio of the cavity SB
to the value of the best-fit ICM $\beta$-model at the same radius. For
a circular region with $r = 1\arcs$ centered on the deepest part of
each cavity, the mean decrements are $\bar{y}_{\rm{W1}} = 0.50 \pm
0.18$ and $\bar{y}_{\rm{E1}} = 0.52 \pm 0.23$, with minima of
$y^{\rm{min}}_{\rm{W1}} = 0.44$ and $y^{\rm{min}}_{\rm{E1}} =
0.47$. Most other cavity systems have minima $> 0.6$ (B04),
highlighting the rarity and extreme nature of the pair in
\rxj. Because the $\beta$-model is symmetric about the plane of the
sky (POS), some part of each cavity must cross this plane for $y$ to
be $< 50\%$. To determine if the observed E1 and W1 geometries could
produce the measured decrements, the method described in
\citet{hydraa} to study the cavities of Hydra A was employed. In
short, the best-fit ICM $\beta$-model is integrated along the line of
sight (LOS) over the center of each cavity assuming that the cavities
are symmetric about the POS. The LOS cavity radius, \rlos, is assumed
to equal the projected cavity effective radius, $\reff = \sqrt{ab}$,
where $a$ and $b$ are the measured semi-major and semi-minor axes,
respectively. With these assumptions, we find decrements $< 0.67$
cannot be achieved for either E1 or W1.

This result demands some imperfection in the decrement model, for
example, that $\rlos \ne \reff$. The measured decrements can be
reproduced when $\rlos = 21.1$ kpc for E1 and $\rlos = 25.3$ kpc for
W1, suggesting the cavities are two times larger along the LOS than in
the POS. If we were looking almost down the axis of a highly elongated
radio lobe cavity inflating along the LOS, it would be unlikely to
cross the POS, hence this case is a little awkward. Another
possibility is that the $\beta$-model is a poor representation of the
undisturbed ICM. In Section \ref{sec:icm} we highlighted the presence
of bright rims surrounding E1-W1. The rims -- which are several
arcseconds wide, azimuthally symmetric, and reside near
$r_{\rm{core}}$ -- may have influenced the $\beta$-model fit,
resulting in artifically low minima. Excluding the rims from the
$\beta$-model fitting did not provide insight because too much of the
SB profile was removed and the fit did not converge. Extrapolating the
SB profile at larger radii inward resulted in even lower decrements,
exacerbating the problem.

Alternatively, deep decrements can be achieved with large, overlapping
cavities, a scenario we now consider in more detail. Shown in Figure
\ref{fig:proj} are snapshots along three lines of sight of ICM X-ray
emission from a hydrodynamical simulation of cavity formation in a
cluster with a $\beta$-model atmosphere. The simulation was performed
with version 3.0 of the {\textsc{FLASH}} hydrodynamics code
\citep{flash}. The refinement criteria were the standard density and
pressure, with a minimum cell size of 0.66 kpc on a grid of $1024^3$
zones. In order to produce bubbles, the simulation was started by
injecting energy into two spheres of radius 4.5 kpc at distances of 13
kpc from the cluster center. The gas inside these spheres was heated
and expanded similar to a Sedov explosion to form a pair of bubbles in
a few Myr, a time much shorter than the rise time of the generated
bubbles. The parameters were chosen such that these regions reached a
radius of 12 kpc and a density contrast of approximately 0.05 as
compared to the surrounding ICM \citep[see][]{2009arXiv0909.1805S}. In
order to allow a direct comparison with the X-ray data, the gridded
simulation output was post-processed with the X-ray-imaging pipeline
{\textsc{XIM}}\footnote{http://www.astro.wisc.edu/~heinzs/XIM}
\citep{2009arXiv0903.0043H}.

The overlapping cavity configuration shown in the upper left panel of
Figure \ref{fig:proj} is appealing for several reasons: 1) overlapping
cavities close to the core can readily produce deep decrements, 2) a
bright, azimuthally symmetric rim arises from the conjunction of the
cavities along the LOS, 3) there are faint depressions at the leading
edge of each cavity, similar to E2, and 4) the superposition of X-ray
structures, and radio plasma within those structures, may explain the
ambiguous \rxj\ radio/X-ray relationship. There are difficulties with
the overlapping cavity scenario, such as the lack of a depression at
W2, that the core does not resemble a bullseye, and the POS cavity
configuration satisfies Occam's razor. But, the simulations are only
suggestive of how overlapping cavities may appear. Real cavities are
known to entrain denser material from the core, deform as they rise
from the core (\eg\ elongate, twist, shred), and the region
surrounding the jets is normally not so low-density. Also, asymmetries
of the large-scale and core gas distributions may explain the lack of
a W2 depression and the `S'-shaped X-ray strucuture in the core. The
simulations are convincing enough that we have divided the cavity
analysis into two possible configurations: a pair of small cavities in
the POS (E1/2) and a pair of large cavities along the LOS (O1/2; shown
in Figure \ref{fig:proj}).

For the POS case, two different cavity semi-polar axes, $c$, are
considered: $c = \reff$ and $c = \rlos$ taken from the decrement
analysis. In both cases, the distance of each cavity from the central
AGN, $D$, was set to the projected distance from the ellipse centers
to the BCG X-ray point source. For the LOS case, the cavity distances
and volumes are poorly constrained, so we selected a simple
interpretation and set $c = \reff$ and $D = a$. We also give
calculations for $D = 2a$. For all cavities, a systematic error of
10\% is assigned to the geometries. Cavity ages were estimated using
the three time scales given in B04: ICM sound speed age (\tsonic),
buoyant rise time age (\tbuoy), and volume refilling age
(\trefill). The energy in each cavity, $\ecav = \gamma PV/(\gamma-1)$,
was calculated assuming the contents are a relativistic plasma
($\gamma = 4/3$), and that the mean internal cavity pressure is equal
to the ICM pressure at $r = D$. The time-averaged energy needed to
create each cavity was calculated as $\pcav = \ecav/\tbuoy$. The
cavity properties are listed in Table \ref{tab:cavities}.

Our POS E1 and W1 volumes are larger than those of B04, thus our
\ecav\ and \pcav\ are larger by a factor of $\approx 2$. Inserting the
B04 cavity volumes into our calculations produces no significant
differences, indicating our analysis is consistent with B04. The POS
and LOS configurations indicate aggregate cavity energy of $\ecav
\approx 3 \dash 73 \times 10^{60}$ erg and power output of $\pcav
\approx 3 \dash 38 \times 10^{45} ~\lum$. The energetics for the LOS
case are extreme, and likely upper limits, but they are informative
because of the strain they place on how the outburst was powered (see
Section \ref{sec:accretion}). However, assuming the LOS upper limits
are accurate, they suggest the outburst is one of the most powerful
found to date, on par with Hercules A \citep{herca} and
\ms\ \citep{ms0735}. The mechanical power of this AGN is similar to
the radiative power of an average quasar, exceeding the $\approx 3
\times 10^{45} ~\lum$ radiated away by ICM gas with $\tcool <
H_{z=0.35}^{-1}$.

%%%%%%%%%%%%%%%%%%%%%%%%%%%%%%%%%%%%%%
\subsection{Powering the AGN Outburst}
\label{sec:accretion}
%%%%%%%%%%%%%%%%%%%%%%%%%%%%%%%%%%%%%%

Assuming \ecav\ is representative of the gravitational binding energy
released by mass accreted onto the SMBH, then the total mass accreted
and its rate of consumption were $\macc = \ecav/(\epsilon c^2)$ and
$\dmacc = \macc/\tbuoy$, respectively. Here, $\epsilon$ is a poorly
understood mass-energy conversion factor assumed to be 0.1, which
gives $\macc \approx 4 \times 10^8 ~\msol$ and $\dmacc \approx 7
~\msolpy$. Below, we consider if the accretion of cold gas or hot gas
pervading the BCG could meet these requirements.

Substantial reservoirs of molecular gas and gaseous filaments are
found in many cool core clusters \citep{crawford99, edge01}, and via
the formation of thermal instabilities in these structures, it is
possible that they provide fuel for AGN activity and star formation
\citep[\eg][]{pizzolato05, 2010arXiv1003.4181P}. The molecular gas
mass (\mmol) of \rbs\ has not been measured, so the
\halpha-\mmol\ correlation presented by \citet{edge01} was used to
estimate \mmol. We caution however that this correlation has
substantial scatter \citep{salome03}. Using the scaled \hbeta\ line as
a surrogate for \halpha, we estimate $\mmol \sim 10^{10} ~\msol$,
sufficiently in excess of \macc\ that we can infer an ample supply of
cold gas exists to have powered the outburst.

However, SMBHs do not grow independent of the host galaxy, and several
hundred times as much gas should go into stars as was accreted by the
SMBH \citep{1995ARA&A..33..581K, magorrian}. The \rbs\ star formation
rate (SFR), $\psi \ll 100 ~\msolpy$ (see Section \ref{sec:bcg}), is
suspiciously low compared to \dmacc. Assuming the present SFR is
representative of the SFR during the AGN outburst, the implication is
that for every mass unit accreted by the SMBH, $< 10$ units went into
stars, a ratio well below what is expected from SMBH-host galaxy
co-evolution. Conversely, if our estimates are not accurate, and the
SFR during the outburst was comensurate with the Magorrian relation,
\rbs\ should have a strong blue color gradient indicating newly formed
stars \citep{rafferty08}, which it does not (see Section
\ref{sec:bcg}). If cold gas accretion did power the outburst, it
appears the gas consumption was highly efficient, but without better
constraints on \mmol\ and $\psi$, it cannot be ruled out.

Direct accretion of the hot ICM via the Bondi mechanism provides
another possible AGN fuel source. The accretion flow arising from this
process is charaterized via the Bondi equation (Equation
\ref{eqn:bon}) and is often compared with the Eddington limit
describing the maximal gas inflow rate (Equation \ref{eqn:edd}):
\begin{eqnarray}
  \dmbon &=& 0.013 ~\kbon^{-3/2} \left(\frac{\mbh}{10^9
    ~\msol}\right)^{2} ~\msolpy \label{eqn:bon}\\
  \dmedd &=& \frac{2.2}{\epsilon} \left(\frac{\mbh}{10^9~\msol}\right)
  ~\msolpy  \label{eqn:edd}
\end{eqnarray}
where $\epsilon$ is a mass-energy conversion factor, \kbon\ is the
mean entropy in \ent\ of gas within the Bondi radius, and \mbh\ is
black hole mass. We chose the relations of \citet{2002ApJ...574..740T}
and \citet{2007MNRAS.379..711G} to estimate \mbh, and find a range of
$0.7 \dash 7.0 \times 10^9 ~\msol$, from which we adopted the weighted
mean value $1.5 ~(\pm 0.2) \times 10^9 ~\msol$. For $\epsilon = 0.1$
and $\kbon = \kna$, the accretion rates are $\dmbon \approx 4 \times
10^{-4} ~\msolpy$ and $\dmedd \approx 33 ~\msolpy$. The Eddington and
Bondi accretion ratios for the outburst event were thus $\dme \equiv
\dmacc/\dmedd \approx 0.25$ and $\dmb \equiv \dmacc/\dmbon \approx
30000$. The large \dmb\ is disconcerting as it implies all the gas
which reached the Bondi radius ($\rbon \approx 10$ pc) was
accreted. But, \dmb\ may be overestimated if $\kbon < \kna$ and
\mbh\ is larger than our adopted value. If $\mbh = 7 \times 10^9
~\msol$ and gas near \rbon\ has a mean temperature of 0.1 keV, for
\dmb\ to approach unity, \kbon\ must be $< 0.2 ~\ent$, corresponding
to $\nelec \sim 0.4 ~\pcc$. This is four times the current ICM central
density, and a sphere $\approx 1$ kpc in radius is required for $>
10^8 ~\msol$ to be available for accretion. But this implies the
inner-core of the cluster is fully consumed in $< 0.1$ Gyr, a time
scale much shorter than \tcool\ of the overlying ICM, shortcircuiting
the feedback loop. While Bondi accretion cannot be ruled out because
the cluster is observed post-outburst and the core conditions may have
changed dramatically since, a number of observationally unsupported
concessions must be made for Bondi accretion to be viable.

The nuclear BCG X-ray source may also reveal valuable information
about on-going accretion. A source spectrum was extracted from the
region enclosing 90\% of the \cxo\ point-spread function, and a
background spectrum was taken from an annulus immediately outside this
region with 5 times the area. The background-subtracted spectrum is
inconsistent with thermal emission and was instead modeled using an
absorbed power law with two Gaussians to account for features which
may be blends of photoionized lines. The best-fit models are given in
Table \ref{tab:agn} and the spectrum is shown in Figure
\ref{fig:nucspec}. The model with a power-law distribution of $\nhobs
\sim 10^{22} ~\pcmsq$ absorbers is preferred, indicating a lack of
significant obscuration. If mass accretion is powering the emission,
the accretion rate is $\dmacc \approx \lbol/(0.1 c^2) \approx 0.04
~\msolpy \approx 0.001 \dme$. The accretion disk model of
\citet{2002NewAR..46..247M} predicts an optical luminosity of $\ldisk
\approx 8 \times 10^{44} ~\lum$, but no $L > 10^{44} ~\lum$ sources
are found in the \hst\ imaging. Extrapolating the best-fit spectral
model to radio frequencies, we find good agreement with the measured
1.4 GHz and 4.8 GHz nuclear radio fluxes. Further, fitting the
continuous injection synchrotron model of \citet{1987MNRAS.225..335H}
to the X-ray, 1.4 GHz, and 4.8 GHz nuclear fluxes produced an
acceptable fit. Analysis of the nuclear source seems to indicate no
remnants of a very dense, hot gas phase which would be associated with
Bondi accretion, but instead appears to be emission from a synchrotron
source, \eg\ unresolved jets.

Powering the \rbs\ outburst via mass accretion alone appears to be
laden with difficulties, but, like \ms\ \citep{ms0735}, a rapidly
spinning SMBH potentially possesses enough energy and power to be an
interesting solution. The spin model of \citet[][GES
  hereafter]{gesspin} suggests jets are produced via a combination of
the Blandford-Znajek \citep[BZ;][]{bz} and Blandford-Payne
\citep[BP;][]{1982MNRAS.199..883B} mechanisms, and a modest mass
accretion rate $\ll 5 ~\msolpy$ \citep{1999ApJ...522..753M} is
required to extract energy from the SMBH. The GES model predicts jet
power scales with spin as
\begin{equation}
  P_{\rm{jet}} = 2 \times 10^{47} ~{\lum} ~\alpha(j) ~\beta^2(j) ~j^2
  B_{d,5}^2 \left(\frac{\mbh}{10^9 ~\msol}\right)^2
  \label{eqn:ges}
\end{equation}
where $j$ is a dimensionless spin parameter between -1 and 1,
$\alpha(j)$ and $\beta(j)$ are $j$-dependent functions capturing the
BZ and BP powers \citep[see][]{2009ApJ...699L..52G}, and $B_{d,5}$ is
the strength of the magnetic field threading the accretion disk with
units $10^5$ G. The GES model has the appealing feature that the most
powerful jets are produced for a SMBH which is spinning retrograde
relative to the direction of the accreting material, easing the
demands on the magnitude of $B_{d,5}$. In the model of
\citet{1999ApJ...522..753M}, the energy of a spinning SMBH is
\begin{equation}
  E_{\rm{spin}} = 1.6 \times 10^{62} ~{\erg} ~j^2
  \left(\frac{\mbh}{10^9 ~\msol}\right).
\end{equation}
Assuming all the spin energy of the \rbs\ SMBH was extracted in the
outburst, our upper limit of $\ecav = 7.3 \times 10^{61}$ erg gives $j
\approx 0.55$. Plugging $\pm j$ into Equation \ref{eqn:ges} requires
$B_{d,5}(-j) \approx 1400$ G and $B_{d,5}(+j) \approx 7400$ G,
reasonable values for a magnetic field connecting a spinning SMBH and
the surrounding accretion disk
\citep[\eg][]{2002Sci...295.1688K}. There are clearly uncertainties
associated with these estimates, and the spin power mechanism is
wraught with its own difficulties \citep[\eg][]{ms0735}, but, for
\rbs, SMBH spin more easily achieves extreme jet powers than mass
accretion alone, and is a credible alternative.

%%%%%%%%%%%%%%%%%%%%%%%%%%%%%%%%%%%%%%%%%%%%%%%%%%%%%%%%%%%%
\subsection{BCG Star Formation and Interaction with the AGN}
\label{sec:bcg}
%%%%%%%%%%%%%%%%%%%%%%%%%%%%%%%%%%%%%%%%%%%%%%%%%%%%%%%%%%%%

An estimate of the \rbs\ SFR is relevant given the general
relationship with the process of SMBH growth via mass accretion. The
SFRs are likely upper limits since 1) the extremely blue \galex\ and
extremely red 2MASS colors signal the presence of an AGN, and 2)
sources of significant SFR uncertainties have been neglected
\citep[\eg][]{1992ApJ...388..310K, 2004AJ....127.2002K, hicksuv,
  2010MNRAS.tmp..626G}. Using the relations of \citet{kennicutt2},
\citet{2006ApJ...642..775M}, and \citet{salim2007}, the SFR may lie in
the range 1--10 ~\msolpy\ (see Table \ref{tab:sfr}). Though the
shallow 4000 \AA\ break and optical emission line ratios indicate the
BTA spectrum arises from a star forming source, the unknown BTA slit
position and unresolved UV emission mask its location. Further, the
\hst\ $\myv-\myi$ color profile does not have a blue gradient,
suggesting if there is star formation, it may be confined to compact
regions, \eg\ the optical substructure seen in the \hst\ images.

To better reveal the BCG substructure, residual galaxy images were
constructed by first fitting the \hst\ \myv\ and \myi\ isophotes with
ellipses using the \iraf\ tool {\textsc{ellipse}}. Stars and other
contaminating sources were rejected using a combination of $3\sigma$
clipping and by-eye masking. The ellipse centers were fixed at the
galaxy centroid. Ellipticity and position angle were fixed at $0.25
\pm 0.02$ and $-64\mydeg \pm 2\mydeg$, respectively -- the mean values
when they were free parameters. Galaxy light models were created using
{\textsc{bmodel}} in \iraf\ and subtracted from the corresponding
parent image, leaving the residual images shown in Figure
\ref{fig:subopt}. A color map was also generated by subtracting the
fluxed \myi\ image from the fluxed \myv\ image.

The close alignment of the optical substructure with the AGN outflow
clearly indicates the jets are interacting with the BCG halo. There
also appears to be an infalling galaxy to the NW of the nucleus,
possibly with a stripped tail \citep[\eg][]{2007ApJ...671..190S}. The
numbered regions overlaid on the residual \myv\ image are the areas of
the color map which have the largest color difference with the
surrounding galaxy light. Regions 1--5 are relatively the bluest with
$m_{\myv-\myi} = -0.40, -0.30, -0.25, -0.22, ~\rm{and} -0.20$,
respectively. Without spectroscopy, it is unclear which, if any, of
the regions host significant star formation or strong line
emission. Regions 6--8 are relatively the reddest with $m_{\myv-\myi}
=$ +0.10, +0.15, and +0.18, respectively. Again, without spectroscopy,
we can only speculate that these may be emission line regions
(\eg\ \halpha) or regions reddened by dust extinction. In the residual
\myi\ image the regions 9--11 denote what appear to be 8--10 kpc long
``whiskers'' of emission radiating out from the BCG. The whiskers may
be optical filaments similar to those seen around other BCGs, for
example in Perseus \citep{2003MNRAS.344L..48F}, and the low-entropy
state of the cluster core is conducive to the formation of such
structures \citep{conduction}. It is interesting that regions 1 and 4
reside at the point where the southern jet appears to be encountering
whiskers 9 and 10. It may be that the jets have entrained and
compressed gas which is now forming stars. If the AGN is driving star
formation, then our SFR estimates may be boosted above the putative
rate and are not representative of past gas condensation in the BCG,
increasing the discrepency with the Magorrian relation discussed in
Section \ref{sec:accretion}.

%%%%%%%%%%%%%%%%%%%%%
\section{Conclusions}
\label{sec:con}
%%%%%%%%%%%%%%%%%%%%%

We have presented results from a study of the galaxy cluster
\rxj\ with specific focus on the cavity system surrounding the BCG
\rbs. \cxo\ observations have enabled us to constrain the energetics
of the AGN outburst and analyze different mechanisms which may have
powered it. We have shown the following:
\begin{enumerate}
\item \rxj\ is a cool core object with a central cooling time $< 0.5$
  Gyr, core entropy $< 30 ~\ent$, and bolometric cooling luminosity of
  $\sim 10^{45} ~\lum$. Bright rims surrounding the prominent cavity
  system $\approx 25$ kpc from the BCG nucleus indicate the AGN
  outburst has either shocked the ICM ($M \approx 1.3$) or uplifted
  cool, dense gas from the core.
\item The two central cavities have decrements consistent with voids
  that cross the plane of the sky, and residual X-ray images reveal
  additional SB depressions likely associated with AGN activity. We
  have placed constraints on the AGN outburst energetics by
  considering two limiting cases: small cavities with centers in the
  plane of the sky, and large overlapping cavities along the line of
  sight. The total energy in the cavities is $3-70 \times 10^{60}$
  erg, with powers $3-40 \times 10^{45} ~\lum$. If the upper limits
  are accurate, the AGN outburst has released enough energy to
  suppress cooling of the cluster halo.
\item The energetics demand that cold-mode and hot-mode gas accretion
  operate with unrealisitc efficency if mass accretion alone powered
  the outburst. We specifically show that Bondi accretion is an
  unattractive solution, and instead suggest that the outburst was
  powered by tapping the energy of SMBH spin. We show that the model
  of \citet{gesspin} can readily achieve our \ecav\ and \pcav\ limits.
\item The optical substructure of the BCG clearly indicates
  interaction of the AGN outflow with the galaxy halo. We are unable
  to determine if any regions of interest host star formation or line
  emission. But, the convergence of what appear to be optical
  filaments, blueish knots of emission, and the tip of one jet suggest
  there may be AGN driven star formation.
\end{enumerate}

Conservation of angular momentum suggests that gas condensing into the
very center of a BCG has a preferred direction of rotation, possibly
aligned with the semi-major axis of the cluster. The prevelance of
systems like \rbs, \ms, and Hercules A is unclear, but their numbers
thus far are small, possibly because their BCGs, for whatever reason,
formed SMBHs that at some point in the past spun retrograde relative
to the preferred rotation direction of the cluster. In which case, if
the GES model is assumed, they would undergo a very powerful AGN
outburst which releases $\ga 10^{61}$ erg of spin energy. If the
circumstances which result in retrograde SMBHs are rare, then this
naturally explains why not many such systems have been
found.

%%%%%%%%%%%%%%%%%
\acknowledgements
%%%%%%%%%%%%%%%%%

KWC and BRM were supported by CXO grant G07-8122X and a grant from the
Natural Science and Engineering Research Council of Canada. KWC thanks
David Gilbanks, Sabine Schindler, Axel Schwope, and Chris Waters for
helpful input.

%%%%%%%%%%%%%%
% Facilities %
%%%%%%%%%%%%%%

{\it Facilities:} \facility{CXO (ACIS)} \facility{HST (WFPC2)}
\facility{GALEX} \facility{BTA (SP124)}

%%%%%%%%%%%%%%%%
% Bibliography %
%%%%%%%%%%%%%%%%

\bibliography{cavagnolo}

%%%%%%%%%%%%%%%%%%%%%%
% Figures and Tables %
%%%%%%%%%%%%%%%%%%%%%%

\clearpage
\begin{deluxetable}{ccccccccccc}
  \rotate
  \tablecolumns{11}
  \tablewidth{0pc}
  \tabletypesize{\small}
  \tablecaption{Cavity Properties.\label{tab:cavities}}
  \tablehead{
    \colhead{Config.} & \colhead{ID} & \colhead{$a$} & \colhead{$b$} & \colhead{$\rlos$} & \colhead{$D$} & \colhead{\tsonic} & \colhead{\tbuoy} & \colhead{\trefill} & \colhead{\ecav} & \colhead{\pcav}\\
    \colhead{-} & \colhead{-} & \colhead{kpc} & \colhead{kpc} & \colhead{kpc} & \colhead{kpc} & \colhead{Myr} & \colhead{Myr} & \colhead{Myr} & \colhead{$10^{60}$ erg} & \colhead{$10^{45}$ erg s$^{-1}$}\\
    \colhead{(1)} & \colhead{(2)} & \colhead{(3)} & \colhead{(4)} & \colhead{(5)} & \colhead{(6)} & \colhead{(7)} & \colhead{(8)} & \colhead{(9)} & \colhead{(10)} & \colhead{(11)}}
  \startdata
  1 & E1      & $17.3 \pm 1.7$ & $10.7 \pm 1.1$ & $13.6 \pm 1.4$ & $23.2 \pm 2.3$ & $20.1 \pm 3.1$ & $28.1 \pm 4.4$ & $70.3 \pm  9.9$ & $1.51 \pm 0.35$ & $1.70 \pm 0.48$\\
  1a & \nodata & \nodata        & \nodata        & $23.4 \pm 2.3$ & \nodata        & \nodata        & \nodata        & $76.9 \pm 10.9$ & $2.59 \pm 0.61$ & $2.92 \pm 0.83$\\
  1 & W1      & \nodata        & \nodata        & $13.6 \pm 1.4$ & $25.9 \pm 2.5$ & $20.3 \pm 4.1$ & $33.3 \pm 5.3$ & $74.4 \pm 10.5$ & $1.72 \pm 0.47$ & $1.64 \pm 0.52$\\
  1a & \nodata & \nodata        & \nodata        & $26.0 \pm 2.6$ & \nodata        & \nodata        & \nodata        & $82.8 \pm 11.7$ & $3.29 \pm 0.89$ & $3.13 \pm 0.99$
  \enddata
  \tablecomments{
    Col. (1) Cavity configuration;
    Col. (2) Cavity identification;
    Col. (3) Semi-major axis;
    Col. (4) Semi-minor axis;
    Col. (5) Line-of-sight cavity radius;
    Col. (6) Distance from central AGN;
    Col. (7) Sound speed age;
    Col. (8) Buoyancy age;
    Col. (9) Refill age;
    Col. (10) Cavity energy;
    Col. (11) Cavity power using \tbuoy.}
\end{deluxetable}

\begin{deluxetable}{lcccccccccccc}
  \tablecolumns{13}
  \tablewidth{0pc}
  \tabletypesize{\footnotesize}
  \tablecaption{Nuclear X-ray Point Source Spectral Models.\label{tab:agn}}
  \tablehead{
    \colhead{Absorber} & \colhead{\nhabs} & \colhead{$\Gamma_{\rm{pl}}$} & \colhead{$\eta_{\rm{pl}}$} & \colhead{$E_{\rm{ga}}$} & \colhead{$\sigma_{\rm{ga}}$} & \colhead{$\eta_{\rm{ga}}$} & \colhead{Param.} & \colhead{$L_{0.7-7.0}$} & \colhead{\lbol} & \colhead{\chisq} & \colhead{DOF} & \colhead{Goodness}\\
    \colhead{-} & \colhead{$10^{22}~\pcmsq$} & \colhead{-} & \colhead{$10^{-5} \dagger$} & \colhead{keV} & \colhead{eV} & \colhead{$10^{-6} \ddagger$} & \colhead{-} & \colhead{$10^{44}~\lum$} & \colhead{$10^{44}~\lum$} & \colhead{-} & \colhead{-} & \colhead{-}\\
    \colhead{(1)} & \colhead{(2)} & \colhead{(3)} & \colhead{(4)} & \colhead{(5)} & \colhead{(6)} & \colhead{(7)} & \colhead{(8)} & \colhead{(9)} & \colhead{(10)} & \colhead{(11)} & \colhead{(12)} & \colhead{(13)}}
  \startdata
  None          & \nodata             & $0.1^{+0.3}_{-0.3}$ & $0.3^{+0.1}_{-0.1}$  & [2.4, 3.4] & [63, 119] & [8.4, 14.9] & \nodata                & $0.69^{+0.11}_{-0.22}$ & $49.9^{+18.1}_{-19.0}$ & 1.88 & 61 & 56\%\\
  Neutral$^a$   & $4.2^{+1.9}_{-1.3}$ & $1.5^{+0.4}_{-0.3}$ & $6.6^{+8.1}_{-3.6}$  & [1.8, 3.0] & [31, 58] & [0.6, 1.8]  & 0.354                  & $0.68^{+0.12}_{-0.24}$ & $5.65^{+2.11}_{-2.50}$ & 1.17 & 60 & 29\%\\
  Warm$^b$      & $3.3^{+1.4}_{-1.6}$ & $1.9^{+0.2}_{-0.2}$ & $16.2^{+0.4}_{-0.5}$ & [1.8, 2.9] & [57, 44] & [0.9, 1.5]  & $0.97^{+0.03}_{-0.03}$ & $0.70^{+0.18}_{-0.26}$ & $3.46^{+1.10}_{-0.95}$ & 1.01 & 59 & 13\%\\
  Power-law$^c$ & 0.5--7.5            & $2.1^{+0.5}_{-0.3}$ & $22.1^{+2.7}_{-1.1}$ & [1.8, 3.0] & [54, 35] & [0.9, 1.4]  & $0.63^{+0.34}_{-0.31}$ & $0.71^{+1.48}_{-1.32}$ & $2.21^{+0.45}_{-0.30}$ & 1.00 & 58 & $< 1$\%
  \vspace{0.5mm}
  \enddata
  \tablecomments{
    For all models, $\nhgal = 2.28 \times 10^{20} ~\pcmsq$. 
    Col. (1) \xspec\ absorber models: ($a$) is \textsc{zwabs}, ($b$) is \textsc{pcfabs}, ($c$) is \textsc{pwab};
    Col. (2) Absorbing column density;
    Col. (3) Power-law index;
    Col. (4) Power-law normalization with units ($\dagger$) ph keV$^{-1}$ cm$^{-2}$ s$^{-1}$ at 1 keV;
    Col. (5) Gaussian central energies;
    Col. (6) Gaussian widths;
    Col. (7) Gaussian normalizations with units ($\ddagger$) ph cm$^{-2}$ s$^{-1}$;
    Col. (8) Model-dependent parameter: ($a$) absorber redshift, ($b$) absorber covering fraction, ($c$) absorber power law index of covering fraction;
    Col. (9) Model 0.7-7.0 keV luminosity;
    Col. (10) Unabsorbed model bolometric (0.01-100.0 keV) luminosity;
    Col. (11) Reduced \chisq\ of best-fit model;
    Col. (12) Model degrees of freedom;
    Col. (13) Percent of 10,000 Monte Carlo realizations with \chisq\ less than best-fit \chisq.
}
\end{deluxetable}

\begin{deluxetable}{ccccc}
  \tablecolumns{5}
  \tablewidth{0pc}
  \tablecaption{BCG Star Formation Rates.\label{tab:sfr}}
  \tablehead{
    \colhead{Source} & \colhead{ID} & \colhead{$\xi$ [Ref.]} & \colhead{$L$} & \colhead{$\psi$}\\
    \colhead{-} & \colhead{-} & \colhead{(\msolpy)/(\lum ~\phz)} & \colhead{\lum ~\phz} & \colhead{\msolpy}\\
    \colhead{(1)} & \colhead{(2)} & \colhead{(3)} & \colhead{(4)} & \colhead{(5)}}
  \startdata
  \galex\            & NUV             & $1.4 \times 10^{-28}$ [1] & $2.5 ~(\pm 0.9) \times 10^{28}$ & $3.5 \pm 1.3$\\
  \xom\              & UVW1            & $1.1 \times 10^{-28}$ [2] & $6.2 ~(\pm 1.9) \times 10^{28}$ & $6.9 \pm 2.2$\\
  \galex\            & FUV             & $1.1 \times 10^{-28}$ [2] & $8.2 ~(\pm 2.0) \times 10^{28}$ & $9.0 \pm 2.3$\\
  \galex\            & FUV             & $1.4 \times 10^{-28}$ [1] & $8.2 ~(\pm 2.0) \times 10^{28}$ & $11 \pm 3$\\
  \xom\              & UVM2            & $1.1 \times 10^{-28}$ [2] & $< 5.0 \times 10^{29}$          & $< 55$
  \enddata
  \tablecomments{A dagger ($\dagger$) indicates the removal of
    Hz$^{-1}$ from the units of $\xi$ \& $L$.  Col. (1) Source of
    measurement; Col. (2) Diagnostic identification; Col. (3)
    Conversion coefficient and references: [1] \citet{kennicutt2}, [2]
    \citet{salim2007}; Col. (4) Luminosity; Col. (5) Star formation
    rate.}
\end{deluxetable}

\begin{figure}
  \begin{center}
    \begin{minipage}{\linewidth}
      \includegraphics*[width=\textwidth, trim=0mm 0mm 0mm 0mm, clip]{rbs797.ps}
    \end{minipage}
    \caption{Fluxed, unsmoothed 0.7--2.0 keV clean image of \rbs\ in
      units of ph \pcmsq\ \ps\ pix$^{-1}$. Image is $\approx 250$ kpc
      on a side and coordinates are J2000 epoch. Black contours in the
      nucleus are 2.5--9.0 keV X-ray emission of the nuclear point
      source; the outer contour approximately traces the 90\% enclosed
      energy fraction (EEF) of the \cxo\ point spread function. The
      dashed green ellipse is centered on the nuclear point source,
      encloses both cavities, and was drawn by-eye to pass through the
      X-ray ridge/rims.}
    \label{fig:img}
  \end{center}
\end{figure}

\begin{figure}
  \begin{center}
    \begin{minipage}{0.495\linewidth}
      \includegraphics*[width=\textwidth, trim=0mm 0mm 0mm 0mm, clip]{325.ps}
    \end{minipage}
   \begin{minipage}{0.495\linewidth}
      \includegraphics*[width=\textwidth, trim=0mm 0mm 0mm 0mm, clip]{8.4.ps}
   \end{minipage}
   \begin{minipage}{0.495\linewidth}
      \includegraphics*[width=\textwidth, trim=0mm 0mm 0mm 0mm, clip]{1.4.ps}
    \end{minipage}
    \begin{minipage}{0.495\linewidth}
      \includegraphics*[width=\textwidth, trim=0mm 0mm 0mm 0mm, clip]{4.8.ps}
    \end{minipage}
     \caption{Radio images of \rbs\ overlaid with black contours
       tracing ICM X-ray emission. Images are in mJy beam$^{-1}$ with
       intensity beginning at $3\sigma_{\rm{rms}}$ and ending at the
       peak flux, and are arranged by decreasing size of the
       significant, projected radio structure. X-ray contours are from
       $2.3 \times 10^{-6}$ to $1.3 \times 10^{-7}$ ph
       \pcmsq\ \ps\ pix$^{-1}$ in 12 square-root steps. {\it{Clockwise
           from top left}}: 325 MHz \vla\ A-array, 8.4 GHz
       \vla\ D-array, 4.8 GHz \vla\ A-array, and 1.4 GHz
       \vla\ A-array.}
    \label{fig:composite}
  \end{center}
\end{figure}

\begin{figure}
  \begin{center}
    \begin{minipage}{0.495\linewidth}
      \includegraphics*[width=\textwidth, trim=0mm 0mm 0mm 0mm, clip]{sub_inner.ps}
    \end{minipage}
    \begin{minipage}{0.495\linewidth}
      \includegraphics*[width=\textwidth, trim=0mm 0mm 0mm 0mm, clip]{sub_outer.ps}
    \end{minipage}
    \caption{Red text point-out regions of interest discussed in
      Section \ref{sec:cavities}. {\it{Left:}} Residual 0.3-10.0 keV
      X-ray image smoothed with $1\arcs$ Gaussian. Yellow contours are
      1.4 GHz emission (\vla\ A-array), orange contours are 4.8 GHz
      emission (\vla\ A-array), orange vector is 4.8 GHz jet axis, and
      red ellipses outline definite cavities. {\it{Bottom:}} Residual
      0.3-10.0 keV X-ray image smoothed with $3\arcs$ Gaussian. Green
      contours are 325 MHz emission (\vla\ A-array), blue contours are
      8.4 GHz emission (\vla\ D-array), and orange vector is 4.8 GHz
      jet axis.}
    \label{fig:subxray}
  \end{center}
\end{figure}

\begin{figure}
  \begin{center}
    \begin{minipage}{\linewidth}
      \includegraphics*[width=\textwidth]{r797_nhfro.eps}
      \caption{Gallery of radial ICM profiles. Vertical black dashed
        lines mark the approximate end-points of both
        cavities. Horizontal dashed line on cooling time profile marks
        age of the Universe at redshift of \rbs. For X-ray luminosity
        profile, dashed line marks \lcool, and dashed-dotted line
        marks \pcav.}
      \label{fig:gallery}
    \end{minipage}
  \end{center}
\end{figure}

\begin{figure}
  \begin{center}
    \begin{minipage}{\linewidth}
      \setlength\fboxsep{0pt}
      \setlength\fboxrule{0.5pt}
      \fbox{\includegraphics*[width=\textwidth]{cav_config.eps}}
    \end{minipage}
    \caption{Cartoon of possible cavity configurations. Arrows denote
      direction of AGN outflow, ellipses outline cavities, \rlos\ is
      line-of-sight cavity depth, and $z$ is the height of a cavity's
      center above the plane of the sky. {\it{Left:}} Cavities which
      are symmetric about the plane of the sky, have $z=0$, and are
      inflating perpendicular to the line-of-sight. {\it{Right:}}
      Cavities which are larger than left panel, have non-zero $z$,
      and are inflating along an axis close to our line-of-sight.}
    \label{fig:config}
  \end{center}
\end{figure}

\begin{figure}
  \begin{center}
    \begin{minipage}{0.495\linewidth}
      \includegraphics*[width=\textwidth, trim=25mm 0mm 40mm 10mm, clip]{edec.eps}
    \end{minipage}
    \begin{minipage}{0.495\linewidth}
      \includegraphics*[width=\textwidth, trim=25mm 0mm 40mm 10mm, clip]{wdec.eps}
    \end{minipage}
    \caption{Surface brightness decrement as a function of height
      above the plane of the sky for a variety of cavity radii. Each
      curve is labeled with the corresponding \rlos. The curves
      furthest to the left are for the minimum \rlos\ needed to
      reproduce $y_{\rm{min}}$, \ie\ the case of $z = 0$, and the
      horizontal dashed line denotes the minimum decrement for each
      cavity. {\it{Left}} Cavity E1; {\it{Right}} Cavity W1.}
    \label{fig:decs}
  \end{center}
\end{figure}


\begin{figure}
  \begin{center}
    \begin{minipage}{\linewidth}
      \includegraphics*[width=\textwidth, trim=15mm 5mm 5mm 10mm, clip]{pannorm.eps}
      \caption{Histograms of normalized surface brightness variation
        in wedges of a $2.5\arcs$ wide annulus centered on the X-ray
        peak and passing through the cavity midpoints. {\it{Left:}}
        $36\mydeg$ wedges; {\it{Middle:}} $14.4\mydeg$ wedges;
        {\it{Right:}} $7.2\mydeg$ wedges. The depth of the cavities
        and prominence of the rims can be clearly seen in this plot.}
      \label{fig:pannorm}
    \end{minipage}
  \end{center}
\end{figure}

\begin{figure}
  \begin{center}
    \begin{minipage}{0.5\linewidth}
      \includegraphics*[width=\textwidth, angle=-90]{nucspec.ps}
    \end{minipage}
    \caption{X-ray spectrum of nuclear point source. Black denotes
      year 2000 \cxo\ data (points) and best-fit model (line), and red
      denotes year 2007 \cxo\ data (points) and best-fit model (line).
      The residuals of the fit for both datasets are given below.}
    \label{fig:nucspec}
  \end{center}
\end{figure}

\begin{figure}
  \begin{center}
    \begin{minipage}{\linewidth}
      \includegraphics*[width=\textwidth, trim=10mm 5mm 10mm 10mm, clip]{radiofit.eps}
    \end{minipage}
    \caption{Best-fit continuous injection (CI) synchrotron model to
      the nuclear 1.4 GHz, 4.8 GHz, and 7.0 keV X-ray emission. The
      two triangles are \galex\ UV fluxes showing the emission is
      boosted above the power-law attributable to the nucleus.}
    \label{fig:sync}
    \end{center}
\end{figure}

\begin{figure}
  \begin{center}
    \begin{minipage}{\linewidth}
      \includegraphics*[width=\textwidth, trim=0mm 0mm 0mm 0mm, clip]{rbs797_opt.ps}
    \end{minipage}
    \caption{\hst\ \myi+\myv\ image of the \rbs\ BCG with units e$^-$
      s$^{-1}$. Green, dashed contour is the \cxo\ 90\% EEF. Emission
      features discussed in the text are labeled.}
    \label{fig:hst}
  \end{center}
\end{figure}

\begin{figure}
  \begin{center}
    \begin{minipage}{0.495\linewidth}
      \includegraphics*[width=\textwidth, trim=0mm 0mm 0mm 0mm, clip]{suboptcolor.ps}
    \end{minipage}
    \begin{minipage}{0.495\linewidth}
      \includegraphics*[width=\textwidth, trim=0mm 0mm 0mm 0mm, clip]{suboptrad.ps}
    \end{minipage}
    \caption{{\it{Left:}} Residual \hst\ \myv\ image. White regions
      (numbered 1--8) are areas with greatest color difference with
      \rbs\ halo. {\it{Right:}} Residual \hst\ \myi\ image. Green
      contours are 4.8 GHz radio emission down to
      $1\sigma_{\rm{rms}}$, white dashed circle has radius $2\arcs$,
      edge of ACS ghost is show in yellow, and southern whiskers are
      numbered 9--11 with corresponding white lines.}
    \label{fig:subopt}
  \end{center}
\end{figure}


%%%%%%%%%%%%%%%%%%%%
% End the document %
%%%%%%%%%%%%%%%%%%%%

\end{document}

%%%%%%%%%%%%
% Endnotes %
%%%%%%%%%%%%

%% ``For quantitative surface brightness profile analysis, we ran the
%% images through 20 iterations of the Richardson-Lucy deconvolution
%% routine \citep{1972JOSA...62...55R, 1974AJ.....79..745L}. The
%% number of iterations was decided after considerable experimentation
%% with varying numbers of iterations. We used 20 iterations here
%% instead of the 40 used by Lauer et al. (1998) since our data have
%% lower signal-to-noise ratios (S/Ns). We used a point-spread
%% function (PSF) generated by the TinyTim software \citep{tinytim}
%% for the center of the PC chip of WFPC2 and a K-type stellar
%% spectrum. The diameter of the synthetic PSF was 3'', and we tapered
%% the PSF at the edges with an 8 pixel Gaussian.''

%% %%%%%%%%%%%%%%%%%%%%%%%%%%%%%%%%%%%%%%
%% \subsection{Global Cluster Properties}
%% \label{sec:global}
%% %%%%%%%%%%%%%%%%%%%%%%%%%%%%%%%%%%%%%%

%% \begin{deluxetable}{lccccccccc}
\tablewidth{0pt}
\tabletypesize{\scriptsize}
\tablecaption{Summary of Global Spectral Properties\label{tab:specfits}}
\tablehead{\colhead{Region} & \colhead{$R_{in}$} & \colhead{$R_{out}$ } & \colhead{$N_{HI}$} & \colhead{$T_{X}$} & \colhead{$Z$} & \colhead{redshift} & \colhead{$\chi^2_{red.}$} & \colhead{D.O.F.} & \colhead{\% Source}\\
\colhead{ } & \colhead{kpc} & \colhead{kpc} & \colhead{$10^{20}$ cm$^{-2}$} & \colhead{keV} & \colhead{$Z_{\sun}$} & \colhead{ } & \colhead{ } & \colhead{ } & \colhead{ }\\
\colhead{{(1)}} & \colhead{{(2)}} & \colhead{{(3)}} & \colhead{{(4)}} & \colhead{{(5)}} & \colhead{{(6)}} & \colhead{{(7)}} & \colhead{{(8)}} & \colhead{{(9)}} & \colhead{{(10)}}
}
\startdata
$R_{500-core}$ & 251 & 1675 & 2.86$^{+2.46}_{-2.75}$  & 13.26$^{+6.21}_{-2.95}$  & 0.54$^{+0.28}_{-0.26}$  & 0.3605$^{+0.0235}_{-0.0162}$  & 1.10 & 368 &  19\\
$R_{1000-core}$ & 251 & 1184 & 3.13$^{+2.31}_{-2.33}$  & 11.20$^{+3.11}_{-1.97}$  & 0.51$^{+0.21}_{-0.21}$  & 0.3639$^{+0.0201}_{-0.0156}$  & 1.08 & 292 &  25\\
$R_{2500-core}$ & 251 & 749 & 1.90$^{+2.30}_{-1.90}$ & 10.66$^{+2.20}_{-1.65}$  & 0.60$^{+0.22}_{-0.19}$  & 0.3611$^{+0.0143}_{-0.0127}$  & 1.01 & 219 &  37\\
$R_{5000-core}$ & 251 & 529 & 3.22$^{+2.75}_{-2.58}$  & 8.80$^{+1.87}_{-1.31}$  & 0.46$^{+0.19}_{-0.18}$  & 0.3556$^{+0.0134}_{-0.0119}$  & 1.11 & 170 &  50\\
$R_{7500-core}$ & 251 & 432 & 2.70$^{+3.02}_{-2.70}$ & 9.85$^{+2.80}_{-1.90}$  & 0.35$^{+0.22}_{-0.21}$  & 0.3632$^{+0.0240}_{-0.0231}$  & 1.06 & 138 &  57\\
$R_{500}$ & \nodata & 1675 & 3.22$^{+1.09}_{-1.02}$  & 7.28$^{+0.50}_{-0.45}$  & 0.41$^{+0.06}_{-0.06}$  & 0.3563$^{+0.0053}_{-0.0044}$  & 0.95 & 535 &  39\\
$R_{1000}$ & \nodata & 1184 & 3.25$^{+1.00}_{-0.93}$  & 7.05$^{+0.40}_{-0.39}$  & 0.40$^{+0.05}_{-0.05}$  & 0.3573$^{+0.0043}_{-0.0040}$  & 0.91 & 488 &  51\\
$R_{2500}$ & \nodata & 749 & 2.97$^{+0.87}_{-0.97}$  & 6.88$^{+0.38}_{-0.33}$  & 0.41$^{+0.05}_{-0.05}$  & 0.3558$^{+0.0026}_{-0.0046}$  & 0.88 & 442 &  70\\
$R_{5000}$ & \nodata & 529 & 3.10$^{+0.90}_{-0.96}$  & 6.66$^{+0.34}_{-0.31}$  & 0.40$^{+0.05}_{-0.05}$  & 0.3560$^{+0.0028}_{-0.0047}$  & 0.85 & 418 &  81\\
$R_{7500}$ & \nodata & 432 & 3.17$^{+0.97}_{-1.04}$  & 6.61$^{+0.38}_{-0.31}$  & 0.39$^{+0.05}_{-0.05}$  & 0.3550$^{+0.0037}_{-0.0047}$  & 0.86 & 410 &  85\\
\hline
$R_{500-core}$ & 251 & 1675 & 3.18$^{+2.46}_{-2.61}$  & 12.63$^{+5.19}_{-2.54}$  & 0.53$^{+0.27}_{-0.26}$  & 0.3540 & 1.10 & 369 &  19\\
$R_{1000-core}$ & 251 & 1184 & 3.40$^{+2.19}_{-2.22}$  & 10.79$^{+2.69}_{-1.70}$  & 0.50$^{+0.20}_{-0.21}$  & 0.3540 & 1.08 & 293 &  25\\
$R_{2500-core}$ & 251 & 749 & 2.19$^{+2.20}_{-2.15}$  & 10.33$^{+2.08}_{-1.50}$  & 0.60$^{+0.21}_{-0.20}$  & 0.3540 & 1.01 & 220 &  37\\
$R_{5000-core}$ & 251 & 529 & 3.25$^{+2.63}_{-2.49}$  & 8.76$^{+1.73}_{-1.30}$  & 0.46$^{+0.18}_{-0.17}$  & 0.3540 & 1.10 & 171 &  50\\
$R_{7500-core}$ & 251 & 432 & 2.99$^{+2.94}_{-2.79}$  & 9.56$^{+2.67}_{-1.74}$  & 0.34$^{+0.21}_{-0.21}$  & 0.3540 & 1.05 & 139 &  57\\
$R_{500}$ & \nodata & 1675 & 3.25$^{+1.01}_{-1.01}$  & 7.25$^{+0.46}_{-0.42}$  & 0.41$^{+0.06}_{-0.06}$  & 0.3540 & 0.95 & 536 &  39\\
$R_{1000}$ & \nodata & 1184 & 3.31$^{+0.99}_{-0.98}$  & 7.00$^{+0.40}_{-0.37}$  & 0.40$^{+0.05}_{-0.05}$  & 0.3540 & 0.91 & 489 &  51\\
$R_{2500}$ & \nodata & 749 & 2.95$^{+0.97}_{-0.95}$  & 6.87$^{+0.37}_{-0.33}$  & 0.41$^{+0.06}_{-0.05}$  & 0.3540 & 0.88 & 443 &  70\\
$R_{5000}$ & \nodata & 529 & 3.10$^{+0.98}_{-0.95}$  & 6.65$^{+0.34}_{-0.32}$  & 0.40$^{+0.05}_{-0.05}$  & 0.3540 & 0.85 & 419 &  81\\
$R_{7500}$ & \nodata & 432 & 3.17$^{+0.99}_{-0.97}$  & 6.60$^{+0.34}_{-0.31}$  & 0.39$^{+0.05}_{-0.05}$  & 0.3540 & 0.86 & 411 &  85\\
\hline
$R_{500-core}$ & 251 & 1675 & 2.22 & 14.16$^{+3.68}_{-2.43}$  & 0.54$^{+0.30}_{-0.28}$  & 0.3626$^{+0.0215}_{-0.0219}$  & 1.09 & 369 &  19\\
$R_{1000-core}$ & 251 & 1184 & 2.22 & 11.91$^{+2.15}_{-1.52}$  & 0.52$^{+0.22}_{-0.22}$  & 0.3658$^{+0.0226}_{-0.0160}$  & 1.08 & 293 &  25\\
$R_{2500-core}$ & 251 & 749 & 2.22 & 10.46$^{+1.49}_{-1.15}$  & 0.60$^{+0.20}_{-0.19}$  & 0.3613$^{+0.0144}_{-0.0132}$  & 1.01 & 220 &  37\\
$R_{5000-core}$ & 251 & 529 & 2.22 & 9.26$^{+1.27}_{-1.02}$  & 0.46$^{+0.19}_{-0.18}$  & 0.3560$^{+0.0140}_{-0.0061}$  & 1.11 & 171 &  50\\
$R_{7500-core}$ & 251 & 432 & 2.22 & 10.15$^{+1.81}_{-1.36}$  & 0.35$^{+0.22}_{-0.20}$  & 0.3647$^{+0.0123}_{-0.0231}$  & 1.05 & 139 &  57\\
$R_{500}$ & \nodata & 1675 & 2.22 & 7.63$^{+0.33}_{-0.30}$  & 0.41$^{+0.06}_{-0.06}$  & 0.3567$^{+0.0048}_{-0.0034}$  & 0.96 & 536 &  39\\
$R_{1000}$ & \nodata & 1184 & 2.22 & 7.38$^{+0.29}_{-0.29}$  & 0.40$^{+0.06}_{-0.05}$  & 0.3586$^{+0.0027}_{-0.0067}$  & 0.91 & 489 &  51\\
$R_{2500}$ & \nodata & 749 & 2.22 & 7.09$^{+0.26}_{-0.23}$  & 0.41$^{+0.06}_{-0.05}$  & 0.3556$^{+0.0031}_{-0.0045}$  & 0.88 & 443 &  70\\
$R_{5000}$ & \nodata & 529 & 2.22 & 6.90$^{+0.23}_{-0.23}$  & 0.40$^{+0.05}_{-0.05}$  & 0.3560$^{+0.0027}_{-0.0048}$  & 0.85 & 419 &  81\\
$R_{7500}$ & \nodata & 432 & 2.22 & 6.86$^{+0.24}_{-0.22}$  & 0.39$^{+0.05}_{-0.05}$  & 0.3558$^{+0.0020}_{-0.0046}$  & 0.87 & 411 &  85\\
\hline
$R_{500-core}$ & 251 & 1675 & 2.22 & 13.80$^{+3.08}_{-2.21}$  & 0.53$^{+0.28}_{-0.28}$  & 0.3540 & 1.09 & 370 &  19\\
$R_{1000-core}$ & 251 & 1184 & 2.22 & 11.64$^{+1.09}_{-1.44}$  & 0.50$^{+0.22}_{-0.22}$  & 0.3540 & 1.08 & 294 &  25\\
$R_{2500-core}$ & 251 & 749 & 2.22 & 10.31$^{+1.38}_{-1.09}$  & 0.60$^{+0.20}_{-0.20}$  & 0.3540 & 1.01 & 221 &  37\\
$R_{5000-core}$ & 251 & 529 & 2.22 & 9.22$^{+1.21}_{-0.97}$  & 0.47$^{+0.19}_{-0.18}$  & 0.3540 & 1.10 & 172 &  50\\
$R_{7500-core}$ & 251 & 432 & 2.22 & 10.01$^{+1.77}_{-1.33}$  & 0.34$^{+0.21}_{-0.22}$  & 0.3540 & 1.05 & 140 &  57\\
$R_{500}$ & \nodata & 1675 & 2.22 & 7.60$^{+0.32}_{-0.30}$  & 0.41$^{+0.06}_{-0.06}$  & 0.3540 & 0.96 & 537 &  39\\
$R_{1000}$ & \nodata & 1184 & 2.22 & 7.34$^{+0.28}_{-0.26}$  & 0.40$^{+0.06}_{-0.05}$  & 0.3540 & 0.92 & 490 &  51\\
$R_{2500}$ & \nodata & 749 & 2.22 & 7.08$^{+0.25}_{-0.23}$  & 0.42$^{+0.05}_{-0.06}$  & 0.3540 & 0.88 & 444 &  70\\
$R_{5000}$ & \nodata & 529 & 2.22 & 6.88$^{+0.23}_{-0.22}$  & 0.40$^{+0.05}_{-0.05}$  & 0.3540 & 0.85 & 420 &  81\\
$R_{7500}$ & \nodata & 432 & 2.22 & 6.85$^{+0.24}_{-0.22}$  & 0.39$^{+0.05}_{-0.05}$  & 0.3540 & 0.87 & 412 &  85\\
\hline
NE-Arm & \nodata & \nodata & 2.22 & 5.00$^{+1.08}_{-0.78}$  & 1.23$^{+1.28}_{-0.79}$  & 0.3540 & 0.19 & 300 &  98\\
NW-Arm & \nodata & \nodata & 2.22 & 5.73$^{+1.28}_{-0.98}$  & 0.54$^{+0.55}_{-0.48}$  & 0.3540 & 0.26 & 184 &  99\\
SE-Arm & \nodata & \nodata & 2.22 & 5.49$^{+1.24}_{-0.80}$  & 0.36$^{+0.41}_{-0.36}$  & 0.3540 & 0.16 & 268 &  99\\
SW-Arm & \nodata & \nodata & 2.22 & 5.99$^{+1.41}_{-0.98}$  & 1.04$^{+0.73}_{-0.49}$  & 0.3540 & 0.14 & 338 &  99\\
\hline
\enddata
\tablecomments{Col. (1) Name of region used for spectral extraction; col. (2) inner radius of extraction region; col. (3) outer radius of extraction region; col. (4) absorbing, Galactic neutral hydrogen column density; col. (5) best-fit temperature; col. (6) best-fit metallicity; col. (7) best-fit redshift; col. (8) reduced \chisq\ for best-fit model; col. (9) degrees of freedom for best-fit model; col. (10) percentage of emission attributable to source.}
\end{deluxetable}


%% Below we present global properties for \rbs\ determined from
%% characteristics of the ICM using the \chandra\ X-ray data. The mean
%% cluster temperature was assumed to be the ICM temperature within a
%% core-excised aperture of $R_{500}$ (defined below). We use the
%% convention of \citet{2007ApJ...668..772M} and set the core radius
%% to $0.15 R_{500}$. $R_{\Delta_c}$ is defined as the radius at which
%% the average cluster density is $\Delta_c$ times the critical
%% density for a spatially flat Universe, $\rho_c=3H(z)^2/8\pi G$. For
%% our calculations of $R_{\Delta_c}$ we adopt the relation from
%% \cite{2002A&A...389....1A}:
%% \begin{eqnarray}
%%   R_{\Delta_c} &=& 2.71 \rm{~Mpc~}
%%   \beta_T^{1/2}
%%   \Delta_{\rm{z}}^{-1/2}
%%   (1+z)^{-3/2}
%%   \left(\frac{kT_{cluster}}{10 \rm{~keV}}\right)^{1/2}\\
%%   \Delta_z &=& \frac{\Delta_c \Omega_M}{18\pi^2\Omega_z} \nonumber \\
%%   \Omega_z &=& \frac{\Omega_M (1+z)^3}{[\Omega_M
%%       (1+z)^3]+[(1-\Omega_M-\Omega_{\Lambda})(1+z)^2]+\Omega_{\Lambda}} \nonumber
%% \end{eqnarray}
%% where $kT_{cluster}$ is the mean cluster temperature in keV,
%% $R_{\Delta_c}$ is in units of $h_{70}^{-1}$, $\Delta_c$ is the
%% assumed density contrast of the cluster at $R_{\Delta_c}$, and
%% $\beta_T$ is a numerically determined, cosmology-independent
%% ($\lesssim \pm 20\%$) normalization for the virial relation $GM/2R
%% = \beta_TkT_{vir}$. We use $\beta_T = 1.05$ taken from
%% \cite{1996ApJ...469..494E}. Since both $R_{500}$ and $kT_{cluster}$
%% appear in the above relations and we wanted to measure both, we
%% used an iterative process of measuring the temperature within a
%% core-excised region with radius $R_{500}$ (based on an initial
%% guess of $T_{cluster}$), computing a new $R_{500}$, perturbing that
%% new $R_{500}$ by a random percentage between 0-10\%, and repeating
%% the process until three consecutive iterations return a value of
%% $R_{500}$ which did not significantly differ. Via this method we
%% measure a core-excised cluster temperature of $13.8^{+3.1}_{-2.2}$
%% keV, which corresponds to $R_{500} = 1.68^{+0.14}_{-0.17} \Mpc$,
%% and a cluster bolometric luminosity of $1.58^{+0.02}_{-0.01} \times
%% 10^{45} \lum$. The bolometric luminosity was determined using a
%% diagonalized spectral response over the energy range 0.01-100.0 keV
%% with 3000 linearly spaced energy channels.

%% Additional spectral fits for a variety of apertures are summaraized
%% in Table \ref{tab:specfits}.  Using the best-fit values of \tx\ in
%% Table \ref{tab:specfits} for the apertures including the core, we
%% measure a weighted mean global temperature of $7.02 \pm 0.07$ keV,
%% and with the core excluded we measure $10.5 \pm 0.36$ keV. The
%% global weighted mean metallicity including the cluster core is
%% $0.40 \pm 0.00~\Zsol$, and without the core $0.49 \pm
%% 0.02~\Zsol$. We also allowed the redshift to be free in several run
%% of fits and measure a weighted mean redshift of $0.3562 \pm 0.0003$
%% with the core, and $0.3609 \pm 0.0010$ without the core. Both these
%% values are reasonably close to the value of $z = 0.354$ determined
%% using the emission lines of the optical spectrum \citep[][and this
%% work]{rbs1}.

%% The correction during deprojection needed to account for
%% ellipticity has also been shown to be small relative to the
%% absolute uncertainty \cite[\eg][]{2003ApJ...598..190D,
%% 2005MNRAS.359.1481B}.

%% A $\beta$-model \citep{betamodel} was fitted to the profile to
%% ensure smooth log-space derivatives of the gas density when
%% deriving cluster mass profiles. The best-fit parameters for the
%% $\beta$-model were $S_0 = 1.63 \pm 0.28 \times 10^{-3} \sbr$,
%% $\beta = 0.59 \pm 0.003$, $R_{\rm{core}} = 34.4 \pm 0.39$ kpc,
%% and a constant background of $S_{\rm{bgd}} = 1.35 \pm 0.25
%% \times 10^{-8} \sbr$. The \chisq(DOF) statistic was 224.76(78),
%% with a goodness of fit of $99.9\%$.

%% This source has a signal-to-noise ratio (SNR) of 4.71. The UVM2
%% $3\sigma$ luminosity upper limit is $L_{UVM2} = 8.14 \times 10^{44}
%% \lum$. For comparison, the \galex\ FUV luminosity is $L_{FUV} =
%% 3.02 \pm 0.55 \times 10^{44} \lum$, and the NUV luminosity is
%% $L_{NUV} = 1.05 \pm 0.29 \times 10^{44} \lum$. In Section
%% \ref{sec:bcg} we discuss the UV emission as it relates to star
%% formation and AGN activity in the BCG.

%% The two cavities extend across four annular regions of our
%% deprojected gas pressure profile. Across these four annuli, the
%% pressure decreases by a factor of 50\%. To check if the isobaric
%% assumption, that the pressure at the location of the bubble center
%% is representative of the pressure surrounding the entire bubble,
%% significantly changes the energetics, we tested the approach of
%% \citet{2008ApJ...686..911F} and used a fit to the pressure profile
%% to calculate the pressure for each bubble volume element. Combining
%% the 3D integral over each bubble's surface, an analytic fit to the
%% pressure profile, and and anlytic fit to the gas density profile we
%% found no significant difference between the two methods. This is a
%% result of the higher and lower pressures at the ends of the bubbles
%% having comparable volumes and effectively avergaing out. Only for
%% complex bubbles geometries (\eg\ large asymmetries or extremely
%% prolate/oblate bubbles) or very steep pressure profiles will the
%% difference of the two methods be significant. Thus, we adopted the
%% isobaric assumption for simplicity. Variation of the temperature
%% profile in the region containing the cavities is small,
%% $\Delta\tx(R \la 50 \kpc) \la 1.2 \keV$, and for simplicity we also
%% assumed that the gas surrounding the bubbles is isothermal.

%% \begin{deluxetable}{lcccc}
  \tablewidth{0pt}
  \tabletypesize{}
  \tablecaption{Summary of Optical Line Measurements\label{tab:optspec}}
  \tablehead{\colhead{Line} & \colhead{$\lambda_0^{\rm{rest}}$} & \colhead{EW} & \colhead{$f_{\lambda}$} & \colhead{$f_{\lambda}^{\rm{cont}}$}\\
    \colhead{-} & \colhead{\AA} & \colhead{\AA}& \colhead{$10^{-16} ~\flux$} & \colhead{$10^{-16} ~\flux$}\\
    \colhead{(1)} & \colhead{(2)} & \colhead{(3)} & \colhead{(4)} & \colhead{(5)}}
  \startdata
  $[$O\Rmnum{2}$]$  & 3728 & -166  & 86.3  & 0.48\\
  Ca H              & 3939 & 9.38  & -6.21 & 0.67\\
  Ca K+H$\epsilon$  & 3959 & 1.89  & -1.31 & 0.69\\
  G Band            & 4302 & 4.10  & -3.53 & 0.85\\
  H$\gamma$         & 4340 & -20.6 & 18.7  & 0.91\\
  Fe \Rmnum{1}      & 4372 & -2.18 & 2.03  & 0.93\\
  H$\beta$          & 4860 & -15.4 & 19.0  & 1.08\\
  $[$O\Rmnum{3}$]$  & 4961 & -5.14 & 6.22  & 1.18\\
  $[$O\Rmnum{3}$]$  & 5009 & -16.8 & 20.2  & 1.11
  \enddata
  \tablecomments{
    Corrections for stellar absorption, dust attenuation, and cosmic
    reddening have not been applied.
    Col. (1) Line identification;
    Col. (2) Central rest-frame wavelength of best-fit Gaussian;
    Col. (3) Equivalent width;
    Col. (4) Line flux;
    Col. (5) Continuum flux.}
\end{deluxetable}

%% $[$O\Rmnum{2}$]$  & 5048 & -166  & 25.3 & 86.3  & 0.48\\
%% Ca H              & 5333 & 9.38  & 15.7 & -6.21 & 0.67\\
%% Ca K+H$\epsilon$  & 5361 & 1.89  & 9.28 & -1.31 & 0.69\\
%% G Band            & 5825 & 4.10  & 27.6 & -3.53 & 0.85\\
%% H$\gamma$         & 5876 & -20.6 & 36.2 & 18.7  & 0.91\\
%% Fe \Rmnum{1}      & 5920 & -2.18 & 11.9 & 2.03  & 0.93\\
%% H$\beta$          & 6580 & -15.4 & 28.0 & 19.0  & 1.08\\
%% $[$O\Rmnum{3}$]$  & 6717 & -5.14 & 30.2 & 6.22  & 1.18\\
%% $[$O\Rmnum{3}$]$  & 6782 & -16.8 & 29.9 & 20.2  & 1.11

%% \begin{deluxetable}{lcccccc}
\tablewidth{0pt}
\tabletypesize{\scriptsize}
\tablecaption{Summary of Temperature Map Spectral Fits\label{tab:tmapfits}}
\tablehead{\colhead{Bin} & \colhead{N$_{HI}$} & \colhead{T$_{X}$} & \colhead{Z} & \colhead{$\chi^2_{red.}$} & \colhead{DOF} & \colhead{\% Source}\\
\colhead{ } & \colhead{$10^{20}$ cm$^{-2}$} & \colhead{keV} & \colhead{Z$_{\sun}$} & \colhead{ } & \colhead{ } & \colhead{ }\\
\colhead{{(1)}} & \colhead{{(2)}} & \colhead{{(3)}} & \colhead{{(4)}} & \colhead{{(5)}} & \colhead{{(6)}} & \colhead{{(7)}}
}
\startdata
  0 & 2.22 & 65.64$^{+\infty}_{-30.74}$  & 0.51$^{+1.57}_{-0.51}$  & 1.19 &  78 &  99\\
  1 & ... & 4.83$^{+0.54}_{-0.45}$  & 0.51$^{+0.22}_{-0.20}$  & 1.23 &  75 &  99\\
  2 & ... & 4.99$^{+0.54}_{-0.47}$  & 0.45$^{+0.22}_{-0.20}$  & 0.73 &  80 &  99\\
  3 & ... & 5.96$^{+0.75}_{-0.61}$  & 0.81$^{+0.30}_{-0.26}$  & 1.13 &  75 &  99\\
  4 & ... & 6.82$^{+1.18}_{-0.86}$  & 0.36$^{+0.22}_{-0.21}$  & 1.05 &  77 &  99\\
  5 & ... & 5.92$^{+0.73}_{-0.60}$  & 0.58$^{+0.24}_{-0.23}$  & 0.82 &  81 &  98\\
  6 & ... & 79.90$^{+\infty}_{-22.89}$  & 2.18$^{+3.19}_{-2.15}$  & 1.20 & 142 &  46\\
  7 & ... & 6.35$^{+0.93}_{-0.73}$  & 0.50$^{+0.24}_{-0.22}$  & 1.04 &  74 &  98\\
  8 & ... & 11.22$^{+3.17}_{-2.07}$  & 1.47$^{+0.75}_{-0.58}$  & 1.31 &  79 &  57\\
  9 & ... & 8.23$^{+1.64}_{-1.16}$  & 0.42$^{+0.26}_{-0.24}$  & 0.77 &  77 &  97\\
 10 & ... & 7.78$^{+1.31}_{-0.98}$  & 0.53$^{+0.29}_{-0.27}$  & 0.97 &  75 &  96\\
 11 & ... & 23.89$^{+\infty}_{-7.62}$  & 0.30$^{+0.74}_{-0.30}$  & 0.99 &  83 &  69\\
 12 & ... & 7.40$^{+1.33}_{-0.95}$  & 0.30$^{+0.26}_{-0.26}$  & 1.27 &  78 &  94\\
 13 & ... & 8.45$^{+1.92}_{-1.34}$  & 0.29$^{+0.24}_{-0.24}$  & 1.01 &  73 &  91\\
 14 & ... & 9.69$^{+2.16}_{-1.47}$  & 0.53$^{+0.32}_{-0.29}$  & 0.98 &  80 &  86\\
 15 & ... & 9.89$^{+2.84}_{-1.81}$  & 0.25$^{+0.29}_{-0.25}$  & 1.16 &  75 &  79
\enddata
\tablecomments{Col. (1) Bin number (see Fig. \ref{fig:tmap} for reference); col. (2) absorbing, Galactic neutral hydrogen column density; col. (3) best-fit temperature; col. (4) best-fit metallicity; col. (5) reduced \chisq\ for best-fit model; col. (6) degrees of freedom for best-fit model; col. (7) percentage of emission attributable to source.}
\end{deluxetable}


%% In the region $45 ~\kpc < r < 500$ kpc, the ICM has mean
%% ellipticity $e = 1.35 \pm 0.03$ and orientation $\phi = 13.0 \pm
%% 1.4$. The central 45 kpc is excluded to avoid the ICM cavities
%% which significanlty affect the ellipsoid fitting.

%% It has been shown that low-frequency radio emission is an
%% acceptable tracer of cavity extent \citep[\eg][]{hydraa, herca,
%% birzan08}. Thus, the extent of the $3\sigma$ significance 327 MHz
%% emission was utilized to calculate cavity volumes. The same method
%% used for calculating volumes for C1 were used. Distances from the
%% AGN were assumed to be equal to the cavity semi-major axis. Cavity
%% ages, energies, and powers were calculated the same as for C1.  We
%% have extracted spectra for each arm of the cavity system and list
%% the best-fit values at the bottom of Table \ref{tab:specfits}. The
%% arms of the cavities are encompassed by the inner three radial
%% annuli of our temperature profile, which cover a temperature range
%% of 3.8-5.8 keV. As was also noted in \citet{schindler01}, we find
%% no evidence the rims of the cavities are shock heated.

%% For all X-ray spectral analysis we use a low-energy cut of 0.7 keV
%% to avoid the effective area and quantum efficiency variations of
%% the ACIS detectors, and a high-energy cut of 7.0 keV above which
%% diffuse source emission is dominated by the background and where
%% \chandra's effective area is small.

%% For each region, 5000 random, equal probability draws between the
%% maximum and minimum values of the semi-major, $a$, and semi-minor,
%% $b$, axes were used to calculate mean values and errors. For each
%% random draw, the polar axis, $c$, varied between the minimum and
%% maximum values of $a$ \& $b$ with the constraint that $a/c$ and
%% $b/c$ were $< 2.5$, which is the maximum eccentricity measured for
%% the B04 sample.

%% For the 50\% EEF region $f_X = 7.99 \times 10^{-14} ~\flux$ and $L_X
%% = 3.39 \times 10^{43} \lum$. Using only the $\approx 13$ ks ACIS-I
%% observation (ObsID 2202), \citet{rbs2} found a flux of $2.2 \times
%% 10^{-13} ~\flux$ for the central source ($r \la 2\arcs$) in the 2-10
%% keV energy range. In our cosmology this equates to a luminosity of
%% $8.73 \times 10^{43} \lum$. For the same energy range, same region
%% area, and using the ACIS-S data only (ObsID 7902), we find a source
%% flux of $4.01 \times 10^{-13} ~\flux$ giving a luminosity of $1.59
%% \times 10^{44} \lum$. A factor of 2 difference in the flux may seem
%% large, but the central point source is very hard, hardness ratio
%% $\approx 6.7$, and for such a source the small differences in the
%% effective area of ACIS-I and ACIS-S between 2-5 keV should result
%% in a higher observed flux for the ACIS-S observation. Regardless,
%% in the X-ray band, the AGN in \rbs\ has a luminosity of $\approx
%% 10^{44} \lum$. The luminosities derived from the spectral analysis
%% are comparable to those we derive from the imaging data.

%% The BCG in \rbs\ is a very bright infrared source. It was
%% catalogued in 2MASS as source J09471278+7623136 with apparent AB
%% magnitudes $m_J = 16.2 \pm 0.1$, $m_H = 15.3 \pm 0.1$, $m_{K_s} =
%% 14.6 \pm 0.1$. After applying $K$ and evolution corrections, the
%% absolute AB magnitudes are $-24.8 \pm 0.2$, $-25.7 \pm 0.2$, and
%% $-26.2 \pm 0.2$, respectively. This corresponds to luminosities in
%% the range $\sim 1-2 \times 10^{45} \lum$. The red $H-K$ and $J-K$
%% colors suggest that the infrared emission results from an obscured
%% AGN.

%% From Wilkes ApJ 2002 564 65:
%% ``Assuming that dust extinction is responsible for the red colors
%% of 2MASS AGNs and that absorp\ tion by associated gas is
%% responsible for the hardness of the X-ray spectra, we can compare
%% t\ he equivalent column densities on a case-by-case basis. We
%% assume that the median, rest-frame\ J-KS for an AGN is 2.04 (Elvis
%% et al. 1994) and determine E(J-KS) from the observed J-KS
%% col\ or. Comparison between this and the X-ray-derived NH shows
%% that the ratio E(J-Ks)/NH is reduc\ ed by a factor of a few to ~100
%% compared with the Galactic value of 0.98 �� 10-22 cm2 mag,
%% the\ latter being computed from the ROSAT dust extinction versus
%% gas absorption relation (Seward \ 1999) and the extinction curve
%% from Mathis (1999). A similar result was reported for optical
%% \ [E(B-V)] versus X-ray extinction in AGNs by Maiolino et
%% al. (2001), who suggest an explanatio\ n in terms of anomalous
%% dust, dominated by large grains, that absorbs with little spectral
%% re\ ddening. Other possibilities include different lines of sight
%% to the X-ray and IR continuum e\ mitting regions such that much of
%% the X-ray-absorbing material does not cover the optical/IR \ source
%% (Risaliti et al. 2000), or absorbing material that is hotter than
%% the dust sublimation\ temperature and so is largely dust-free
%% and/or physically decoupled from the dust (Granato, \ Danese, \&
%% Franceschini 1997; Risaliti, Elvis, \& Nicastro 2001). Optical
%% spectropolarimetry to\ study the scattering material combined with
%% improved constraints on the X-ray-absorbing colu\ mn density for a
%% significant number of these AGNs will allow us to distinguish
%% between these \ various possibilities.''

%% \begin{eqnarray}
%%   \log \left(\frac{[\rm{O \Rmnum{3}}]_{5007}}{\hbeta}\right) &=&
%%   \frac{0.14}{\log \left(\frac{[\rm{O
%%           \Rmnum{2}}]_{3727}}{\hbeta}\right)-1.45}+0.83 \label{eqn:bpt}\\
%%   \log \rm{O_{32}} &=& \frac{1.5}{\log
%%     \rm{R_{23}}-1.7}+2.3 \label{eqn:o32}\\
%%   \rm{O_{32}} &=& \left(\frac{[\rm{O \Rmnum{3}}]_{4959,5007}}{[\rm{O
%%         \Rmnum{2}}]_{3727}}\right) \nonumber\\
%%   \rm{R_{23}} &=& \left(\frac{[\rm{O
%%         \Rmnum{3}}]_{4959,5007}+[\rm{O\Rmnum{2}}]_{3727}}{\hbeta}\right)
%%   \nonumber
%% \end{eqnarray}

%% The \oiii/\hbeta\ and [O \Rmnum{2}] $\lambda 3727$/\hbeta\ ratios
%% of 1.00 and 8.49, respectively, are inconsistent with the all of
%% the shock models in \citet{shull79} and the fast shock models in
%% \citet{dopita96}.

%% -- integrate B-band to get $L_{B,nuc}$ 
%% -- calc SFR from \citet{mcnamara89} with $\psi_{B,nuc} = 1.58  \times  10^{-10} L_{B,nuc}$
%% -- 4000 ang break is shallow, SF fills this feature in
%% -- how much of emission lines the result of AGN ionization?

%% For a galaxy with limited on-going star formation, NUV emission is
%% dominated by stars at the Main Sequence turnoff, subgiants, and
%% some blue horizontal branch stars \citep{XXX}, while the FUV
%% emission is thought to reflect the presence of hot, evolved stars
%% \citep{XXX}. To this end, the \galex\ and \xom\ UV observations are
%% useful for investigating the extent of star formation in \rbs. The
%% \galex\ FUV-NUV color is extremely blue, $-1.19 \pm 0.58$, and
%% inconsistent with UV colors for other star forming galaxies
%% \citep{XXX}, possibly as a result of strong AGN contamination. For
%% $z = 0.354$, \lya\ ($\lambda = 1646$ \AA) falls near the center of
%% the \galex\ FUV filter, which may bias the UV color to the
%% blue. XXX: Why does this matter, and what does Lya tell us?

%% %%%%%%%%%%%%%%%%%%%%%%%%%%%%%%%%%%%%%%%%%%%%%%%%%%%%%%%%%%%
%% \subsubsection{Is the BCG Star Formation or AGN Dominated?}
%% %%%%%%%%%%%%%%%%%%%%%%%%%%%%%%%%%%%%%%%%%%%%%%%%%%%%%%%%%%%

%% A spectral energy distribution (SED) for \rbs\ was constructed by
%% compiling photometry available in the literature. Corrections for
%% evolution, Galactic extinction, and redshift reddening were applied
%% to the photometry using values derived from the relations of
%% \citet{cardelli89} and \citet{poggianti97}. The SED models of
%% \citet{2007A&A...461..445S} were fitted to the data using
%% \chisq\ minimization, and the best-fit models are shown in Figure
%% \ref{fig:sed} along with the photometry. The measurements shown
%% come from apertures $< 5\arcs$ in radius. The
%% \citet{2007A&A...461..445S} models depict the SEDs for the nuclei
%% of starburst galaxies including the effects of dust extinction and
%% compact massive star formation regions. The best-fit models have
%% luminosities and core radii consistent with our measured values:
%% $L_{\rm{opt}} \approx 5 \times 10^{45} ~\lum$ \& $R_{\rm{nuc}}
%% \approx 1$ kpc. The visual dust extinction of the best-fit models
%% is $\av = 2.2$ mag where $0.6 L_{\rm{opt}}$ originates from OB
%% stars embedded in regions with $\nH \sim 100 ~\pcc$. Typical
%% optical extinction in dusty, star-forming nebulae is $\av \sim 1$
%% \citep{voit97}, consistent with the preferred value found from the
%% SED fits.

%% The line ratios for \oii, \oiii, and \hbeta\ were used to further
%% determine if the BCG has significant star formation. For
%% starbursts, there is a well-defined upper limit on the intensity of
%% collisional lines relative to recombination lines, while for AGN no
%% such limit exists because the more energetic photons associated
%% with the AGN can create ever stronger collisional lines. Thus, the
%% relative emission line signature of starbursts and AGN are quite
%% different, and a plot comparing the line ratios will segregate into
%% starburst- and AGN-dominated galaxies. \hbeta\ stellar absorption
%% was accounted for by adding 4.6 \AA\ to the equivalent width
%% \citep[see][for details]{2004MNRAS.350..396L}. The continuum around
%% the measured lines was flat, making the uncertainties associated
%% with the contiuum subtraction insignificant. The emission line
%% diagnostics are shown in the diagrams of Figure \ref{fig:sed}. The
%% curves in Figure \ref{fig:sed} separating starburst- from
%% AGN-dominated galaxies calculated using the relations in
%% \citet{2004MNRAS.350..396L}.  The BCG emission line ratios are
%% similar to those of LINERs \citep{1980A&A....87..152H}. Using the
%% spectral windows 3850-3950 \AA\ (W1) and 4000-4100 \AA\ (W2)
%% \citep{balogh99}, the 4000 \AA\ break is $f(\rm{W2})/f(\rm{W1}) =
%% 1.58 \pm 0.21$, which is shallow relative to other ellipticals,
%% indicating the break has filled-in from on-going star
%% formation. Star forming BCGs are also known to have blue gradients
%% \citep{rafferty06}. However, the $(\myv-\mvi)$ color profile
%% declines moving out from the galaxy center. While the ACS ghosts
%% appear to negligible within 10 kpc, we cannot precisely quantify
%% their affect on the photometry and the color profile.

%% \rbs\ was observed as part of a gravitatioal lensing program, thus
%% the spectrum was not analyzed in detail by \citet{schindler01}.

%% The abundance profile is flat interior of $r \approx 30$ kpc with a
%% value of $\approx 0.5 ~\Zsol$ and declines steadily outward. Cool
%% core clusters typically have centrally peaked abundance profiles
%% with the peak covering a narrow radial range, $r \la 20$ kpc
%% \citep{2001ApJ...551..153D}. Similar to what is seen in clusters
%% like Hydra A or M87, the large core may result from metal transport
%% via the AGN outflow \citep{hydrametal, 2010arXiv1002.0395S}.
%% Extraction of a single spectrum for the region $150 ~\kpc \le R \le
%% 500 ~\kpc$ yields an abundance of $0.27 \pm 0.14 ~\Zsol$ ($\tx =
%% 9.97 \pm 1.0 ~\keV$) indicating the abundance upturn at large radii
%% is the result of poor statistics from the smaller bins.

%% cooling luminosity, \lcool, was estimated as the bolometric
%% luminosity of gas within a radius, \rcool, where the local cooling
%% time is equal to the age of the Universe at $z = 0.354$. We measure
%% $\rcool \approx 166$ kpc, giving

%% The \rxj\ cavity system appears to be complex, and may have been
%% created by intermittent AGN activity of a single precessing SMBH,
%% or from multiple SMBHs with anisotropic spin axes.

%% \citet{2007ApJ...662..808L} suggested that the largest cores may
%% indicate the presence of ultramassive black holes (UMBHs) with
%% masses exceeding $10^{10} ~\msol$.

% LocalWords:  Asiago

%% -- $M_{\rm{bul}} = 1.51 \times 10^{12} ~\msol$ from $R$-band\\
%% -- $M_{\rm{bul}} = 1.72 \times 10^{12} \msol$ from \citet{faber89}\\
%% -- $L_{\myi} = 3.70 ~(\pm 0.07) \times 10^{42} ~\lum$\\

%% Marcus' original text:
%% All simulations were performed with FLASH version 3.0 (Fryxell et
%% al. 2000), a multidimensional AMR hydrodynamics code, which solves the
%% Riemann problem on a Cartesian grid using a directionally split
%% Piecewise����Parabolic Method solver. The refinement criteria are the
%% standard density and pressure, and we allow for four levels of
%% refinement beyond the base grid, corresponding to a minimum cell size
%% of 0.66 kpc, and an effective grid of 1024^3 zones.  X-ray cavities in
%% the ICM are thought to be inflated by a pair of ambipolar jets from an
%% AGN in the central galaxy that inject energy into small regions at
%% their terminal points, which expand until they reach pressure
%% equilibrium with the surrounding ICM (Blandford & Rees 1974). The
%% result is a pair of underdense, hot bubbles on opposite sides of the
%% cluster centre.  In order to produce bubbles, we started the
%% simulation by injecting energy into two small spheres of radius 4.5
%% kpc at distances of 13 kpc from the cluster centre. The gas inside
%% these spheres was heated and expanded similar to a Sedov explosion to
%% form a pair of bubbles in a few Myr, a time much shorter than the rise
%% time of the generated bubbles. The parameters were chosen such that
%% these regions reached a radius of 12 kpc and a density contrast of
%% approximately 0.05 as compared to the surrounding ICM. (see
%% Scannapieco & Bruggen 2009)

%% In order to allow a direct comparison of simulation output with X-ray
%% data, we made use of a newly developed pipeline for post-processing of
%% gridded simulation output. The X-ray-imaging pipeline xim (see also
%% Heinz & Br��ggen 2009) is a publically available set of scripts that
%% automate the creation of simulated X-ray data for a range of
%% satellites. It takes as input the density, temperature and velocity,
%% as well as a large number of parameters, and provides simulated X-ray
%% data in the form of spectral-imaging data cubes.

%% BTA:
%% The calibrated spectrum and measurements are provided in Figure
%% \ref{fig:spec} and Table \ref{tab:optspec}, respectively.  The
%% \hbeta\ \& [O \Rmnum{2}] conversion coefficients were interpolated
%% from the tabulated values given in \citet{2006ApJ...642..775M}
%% using the corrected \rbs\ $B$-band magnitude from HyperLEDA.

%% While the \myi\ ACS ghosts were masked and appear to be negligible
%% within $2\arcs$, we cannot precisely quantify their affect on the
%% photometry and restrict our analysis to $r \le 2\arcs$. The SB
%% profile, color profile, and all fits discussed below are shown in
%% Figure \ref{fig:sbr}. The \myv+\myi\ SB profile was parametrized using
%% the Nuker model \citep{nuker}, and the best-fit model has \chisq(DOF)
%% = 0.027(76) and a break radius of $r_b = 1.6 \pm 0.1$ kpc. In the BCG
%% and gE samples of \citet{2003AJ....125..478L} and
%% \citet{2007ApJ...662..808L}, fewer than 5\% of the more than 200
%% galaxies studied have a core with $r_b > 1$ kpc. Black hole scouring
%% has been proposed as one method for creating such large cores
%% \citep[\eg][]{2001ApJ...563...34M}, and assuming that the stellar mass
%% ejected from the core is equivalent to the final \mbh, the difference
%% between the observed and initial light profiles gives an estimate of
%% \mbh\ \citep[\eg][]{2002MNRAS.331L..51M}. The initial SB profile was
%% approximated with a \sersic\ model fit to the data beyond $r_b$ and
%% then extrapolated inward. For an assumed mass-to-light ratio of 3, the
%% light deficit of $1.9 \times 10^{9} ~\lsol$, gives $\mbh = 5.8 \times
%% 10^{9} ~\msol$, consistent with the range of estimated black hole
%% masses (see Section \ref{sec:accretion}). The color profile,
%% $\myv-\myi$, was fitted with the function $\Delta(\myv-\myi) \log r +
%% b$ where $\Delta(\myv-\myi)$ [mag dex$^{-1}$] is the color gradient,
%% $r$ is radius, and $b$ [mag] is a normalization. The best-fit
%% paramaters are $\Delta(\myv-\myi) = -0.20 \pm 0.02$ and $b = 1.1 \pm
%% 0.01$ for \chisq(DOF) = 0.009(21), consistent with the range of
%% gradients for BCGs in the \citet{rafferty06} sample.

%% The spheroid associated with regions 5 and 7 bears some resemblance to
%% a galaxy being ram pressure stripped on its way through the core. The
%% blue region 5 and red region 7 may be associated with
%% \citep[\eg][]{2007ApJ...671..190S}.

%% \citet{2000AJ....119.2540G} Asiago QSO survey
%% \citet{2000ApJS..129..547B} radio and opt ID of bright X-ray sources
%% \citet{2001A&A...374...92V} quasar, agn catalogue identification
%% \citet{2003A&A...401..927B} BL LAc objects
%% \citet{2003MNRAS.339..913E} 
%% \citep{dunn06}
%% \citep{rafferty06}

%% Meier 99/00:
%% E_{\rm{spin}} &\approx& 1.6 \times 10^{62} ~{\erg}
%% ~\left(\frac{\mbh}{10^9 ~\msol}\right) ~a^2\\
%% P_{\rm{jet}} &=& 1.1 \times 10^{46} ~{\lum} ~\left(\frac{B_p}{10^4
%%   ~\rm{G}}\right)^2 ~\left(\frac{\mbh}{10^9 ~\msol}\right)^2 ~a^2

%% XXX: does \dme\ of 0.25 suggest then that this thing was a quasar
%% during the outburst? do systems like rbs797, ms0725, and herca look
%% like i09 before they go animal? are we seeing the echoes of smbh's
%% spinning down, eg retro to pro? would explain reorient of
%% jets... maybe?

%% \begin{eqnarray}
%%   \alpha &=& \delta\left(\frac{3}{2}-j\right)\\
%%   \beta &=& -\frac{3}{2}j^3+12j^2-10j+7-\gamma\\
%%   \gamma &=& \frac{0.002}{(j-0.65)^2}+\frac{0.1}{(j+0.95)}+\frac{0.002}{(j-0.055)^2}
%% \end{eqnarray}

%% the 0.6-7.0 keV source flux is $2.83 ~(\pm 0.33) \times 10^{-13}
%% ~\flux$ gives a luminosity $1.21 ~(\pm 0.14) \times 10^{44} ~\lum$.

%% \begin{deluxetable}{lcccc}
  \tablewidth{0pt}
  \tabletypesize{}
  \tablecaption{Summary of Optical Line Measurements\label{tab:optspec}}
  \tablehead{\colhead{Line} & \colhead{$\lambda_0^{\rm{rest}}$} & \colhead{EW} & \colhead{$f_{\lambda}$} & \colhead{$f_{\lambda}^{\rm{cont}}$}\\
    \colhead{-} & \colhead{\AA} & \colhead{\AA}& \colhead{$10^{-16} ~\flux$} & \colhead{$10^{-16} ~\flux$}\\
    \colhead{(1)} & \colhead{(2)} & \colhead{(3)} & \colhead{(4)} & \colhead{(5)}}
  \startdata
  $[$O\Rmnum{2}$]$  & 3728 & -166  & 86.3  & 0.48\\
  Ca H              & 3939 & 9.38  & -6.21 & 0.67\\
  Ca K+H$\epsilon$  & 3959 & 1.89  & -1.31 & 0.69\\
  G Band            & 4302 & 4.10  & -3.53 & 0.85\\
  H$\gamma$         & 4340 & -20.6 & 18.7  & 0.91\\
  Fe \Rmnum{1}      & 4372 & -2.18 & 2.03  & 0.93\\
  H$\beta$          & 4860 & -15.4 & 19.0  & 1.08\\
  $[$O\Rmnum{3}$]$  & 4961 & -5.14 & 6.22  & 1.18\\
  $[$O\Rmnum{3}$]$  & 5009 & -16.8 & 20.2  & 1.11
  \enddata
  \tablecomments{
    Corrections for stellar absorption, dust attenuation, and cosmic
    reddening have not been applied.
    Col. (1) Line identification;
    Col. (2) Central rest-frame wavelength of best-fit Gaussian;
    Col. (3) Equivalent width;
    Col. (4) Line flux;
    Col. (5) Continuum flux.}
\end{deluxetable}

%% $[$O\Rmnum{2}$]$  & 5048 & -166  & 25.3 & 86.3  & 0.48\\
%% Ca H              & 5333 & 9.38  & 15.7 & -6.21 & 0.67\\
%% Ca K+H$\epsilon$  & 5361 & 1.89  & 9.28 & -1.31 & 0.69\\
%% G Band            & 5825 & 4.10  & 27.6 & -3.53 & 0.85\\
%% H$\gamma$         & 5876 & -20.6 & 36.2 & 18.7  & 0.91\\
%% Fe \Rmnum{1}      & 5920 & -2.18 & 11.9 & 2.03  & 0.93\\
%% H$\beta$          & 6580 & -15.4 & 28.0 & 19.0  & 1.08\\
%% $[$O\Rmnum{3}$]$  & 6717 & -5.14 & 30.2 & 6.22  & 1.18\\
%% $[$O\Rmnum{3}$]$  & 6782 & -16.8 & 29.9 & 20.2  & 1.11

