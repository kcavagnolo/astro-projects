\newcommand{\iras}{IRAS 09104+4109}
\newcommand{\irs}{IRAS09}
\newcommand{\rxj}{RX J0913.7+4056}
\newcommand{\apj}{ApJ}
\newcommand{\apjs}{ApJS}
\newcommand{\apjl}{ApJL}
\newcommand{\aap}{A{\&}A}
\newcommand{\aaps}{A{\&}AS}
\newcommand{\apss}{Ap{\&}SS}
\newcommand{\mnras}{MNRAS}
\newcommand{\aj}{AJ}
\newcommand{\araa}{ARAA}
\newcommand{\pasp}{PASP}

\documentclass[12pt]{article}
\usepackage{common,graphicx,phase1}
\usepackage[nonamebreak,numbers,sort&compress]{natbib}
\bibliographystyle{plainnat}
\begin{document}

%%%%%%%%%%%%%%
\justification
%%%%%%%%%%%%%%

Half of all galaxy clusters have an X-ray halo with a core cooling
time $\ll \Hn^{-1}$. These short cooling times should result in the
formation of $> 100 ~\Msolpy$ ``cooling flows'' [see 32, for a
  review]. The continuum of X-ray gas temperatures expected in the
cooling flow scenario are not observed [33, 40], and direct evidence
for cooling flows, such as prodigious molecular cloud \& star
formation, are not found in the expected quantities [23, 27, 28].
Feedback from active galactic nuclei (AGN) has emerged as the
consensus solution for regulating star formation and suppressing
cooling of the hot halos of galaxies and clusters [3, 4, 10, 15].
Additionally, the existence of correlations between AGN outburst ages
and host halo cooling times indicates the presence of a finely-tuned
feedback loop [\ie\ 5, 34]. But how AGN feedback energy is
thermalized, specifically on scales the size of the host galaxy, and
directed to gas with the shortest cooling times is still unclear [see
  26, for a review]. Sparse observational evidence suggests weak
shocks, sound waves, and conduction may be responsible [17, 20, 44].

It is well-known that conduction on its own has a minor role in
defining galaxy cluster properties [14, 42]. But, when favorable
magnetic field configurations are imposed by certain
magnetohydrodynamic (MHD) processes [\eg\ 2, 29], conduction is an
efficient heating mechanism. In the presence of subsonic turbulence,
for example from gentle mergers or AGN activity, numerical simulations
have shown MHD processes can effectively boost conduction such that
catastrophic cluster core cooling (\ie\ the cooling flows) is staved
off [21, 30, 35]. One prediction of these simulations is that channels
of preferentially radial magnetic fields are established within the
cluster core. In the case of a cool core cluster, the radial fields
connect regions of vastly different temperatures, thereby amplifying
conductive heating between the regions. If these MHD processes
function on galactic scales, then one may expect to find evidence for
radial, conductively heated regions in cluster cores. This would serve
as additional evidence that conduction has a vital role in
distributing heat across the scales where rapid cooling is taking
place.

To this end, BCG nebulae and optical filaments are a valuable
diagnostic for understanding small-scale heating processes [7, 22, 37,
  43]. But only a handful of BCGs with {\it{radial}} optical filament
systems have been studied in detail [9, 20, 38]. Recently, Fabian et
al. [18] demonstrated that the filaments around NGC 1275 must be
magnetically supported, and Sparks et al. [39] suggested the detection
of \civ, an efficient coolant at $\sim 10^{5}$ K, in one of M87's
filaments results from conductive heating. Observational evidence is
mounting that magnetism and conduction are important in explaining,
and possibly regulating, the thermal state of BCG halos. But, better
constraints are needed, and this requires deep, multiwavelength
observations of cool core BCGs hosting radial optical filaments. One
such object which is ideal for study is the extreme BCG
\iras\ (hereafter \irs; $z=0.4418$).

\irs\ belongs to the population of uncommon low-redshift,
ultraluminous infrared galaxies (ULIRG, $L_{\mathrm{IR}} \ga 10^{12}
~\Lsol$). Unlike nearly all ULIRGs, \irs\ is the BCG in a rich galaxy
cluster with a strong cool core ($\tcool < 0.3$ Gyr) which could
supply the BCG with ample amounts of gas condensing out of the ICM
($\dot{M}_{\mathrm{cool}} > 500 ~\Msolpy$). And unlike most BCGs, 99\%
of \irs's bolometric luminosity is emitted longward of 1 $\mu$m due to
a heavily dust obscured Seyfert-2 [16, 24, 25]. However, while large
amounts of hot dust and optical nebulae are detected in \irs, the
galaxy hosts little cold dust and has no detected PAH features [11,
  31, 36]. In addition, there are six compact spheroids within 50 kpc
of the BCG which may be the bulges of cannibalized companions (shown
in Figure \ref{fig:i09}). However, the BCG nebulae are stationary
relative to the galaxy, suggesting the nebulae were not stripped from
companions [8]. But, the large dust content of the nebulae rules out
the hot ICM as their origin since dust has a short sputtering time in
hot gas [12, 13]. \irs\ also hosts an odd FR-II/FR-I radio source
which has excavated cavities in the galaxy's halo [Figure
  \ref{fig:i09}; 6]. The radio and cavity properties indicate a
supersonic AGN outflow, but, the ionization state and dust fraction of
gas neighboring the cavities indicate the nebulae are not being shock
heated [41]. So, the origin, or more specifically, the physical,
thermal, and emission properties, of the gas reservoir in \irs\ is
mysterious.

Utilizing \hst\ WFPC2 imaging data, Armus et al. [1] pointed-out a
series of red, tenuous, radially-oriented ``whiskers'' surrounding
\irs\ (Figure \ref{fig:i09}). The whiskers avoid the radio jet axis,
which led Armus et al. [1] to suggest the whiskers might be cooling
flow gas illuminated by QSO light escaping through ``cracks'' of the
obscuring material. We suspect the whiskers are parts of giant radial
\halpha\ filaments ($l > 20$ kpc) similar to those seen around other
BCGs. If so, then the morphologies and luminosities of the whiskers
provide a sub-kpc probe of the magnetic and thermal state of BCG gas,
and may provide further observational constraints on the influence of
magnetism and conduction in cluster cores. {\bf{We propose to obtain
    deep, high-resolution near-infrared (NIR) and near-ultraviolet
    (NUV) observations of the \iras\ ``whiskers'' to determine their
    morphologies, composition, thermal \& magnetic states, and to
    model the mechanisms which may be heating the whiskers.}}

If the filament system in \irs\ is comparable to those of other BCGs,
then \halpha\ emission (in the NIR for $z=0.4418$) likely dominates
the whisker flux. For a {\it{WMAP}} cosmology, the archival F814W
\hst\ data indicates the whiskers have widths $w < 450$ pc ($<
0.1\arcs$) and lengths $l > 5700$ pc ($> 1\arcs$), giving aspect
ratios $> 10$. A combination of WFC3/IR and NICMOS/NIC1 observations
will yield the flux measurements and spatial resolution, respectively,
needed to calculate accurate \halpha\ luminosities (\lha) and more
precise physical dimensions ($D_{\mathrm{whk}}$). Assuming a scaling
between \halpha\ luminosity and whisker gas mass, the whisker surface
densities ($\Sigma$) can be derived from $D_{\mathrm{whk}}$ and
\lha. The magnetic field strengths, $B$, needed to explain the whisker
morphologies in the presence of various forces (shear, gravity,
internal turbulence, thermal pressure) are proportional to
$C\Sigma^\alpha$, where $C$ and $\alpha$ are functions of the various
force assumptions. If the whiskers are thin, continuous objects, as in
NGC 1275, then they are at a minimum stable against gravity, and the
implied $B$-field strengths along the whiskers need to be $> 100
~\mu$G. For comparison, typical X-ray halo $B$-fields are of the order
$1 ~\mu$G. If the whisker fields are strong and associated with the
ambient fields, then it suggests they have been amplified via some
process, \eg\ MHD amplification or AGN outflows stretching field
lines. We point out that no whiskers are seen along the jet axis,
suggesting some dynamical interaction between the whiskers and the
AGN.

With a constraint on the whisker magnetic properties, the whisker
thermal relationship with the ambient medium can be addressed. For
example, if the whisker magnetic pressure dominates over the ambient
thermal pressure, then simple contact heating at the whisker surfaces
may be unlikely, and some other process(es), \eg\ conduction along the
$B$-fields, confined cosmic rays, gas phase mixing, massive star
formation, or QSO irradiation, may be needed to explain the emission
properties. We can also compare the whisker field strengths with those
for magnetized relativistic jets. Might the whisker fields be the
result of fields seeded by, or even coupled to on-going, AGN activity?
Or, could the whiskers be glowing threads along magnetic loops
emanating from the nucleus hosting the QSO? The on-going AGN outburst
and (apparent) companion mergers in \irs\ imply that the hot halo is
turbulent. If the whiskers are magnetically dominated, that suggests
the whiskers are coherent bodies, and interaction with the hot halo
will be mostly kinematic. Thus, the whisker age upper limits can be
derived, assuming a mean turbulent gas velocity, such that the
$B$-fields resist whisker destruction by the ambient turbulent shear
flows. Conversely, ages estimates from the whisker emission properties
can be used to constrain the ambient turbulent velocities.

Equipped with measurements of the physical state of the whiskers, NUV
continuum and line flux measurements with WFC3/UVIS and COS/NUV,
respectively, will enable further interpretation of the whiskers'
thermal state. Measurement, or upper limits, for whisker \civ\ line
fluxes using COS/NUV spectroscopy, and NIR inferred densities \&
temperatures, will allow us to explore the parameter space of
conductive heating models, for example with \cloudy\ [19], which can
produce the observed \civ\ emission. The WFC3/UVIS imaging will map
out the UV continuum in the COS/NUV passband, allowing us to constrain
the contribution from star formation and AGN emission to our
measurements for the whiskers. Further, if the whiskers are supported
by strong, ordered $B$-fields, then they should be continuous, thin,
mostly linear structures. However, if the NUV imaging in particular
indicates the presence of knotty and diffuse emission, possibly from
star formation, then this belies the influence of strong fields as the
whisker gas should be mostly homogeneous.

Our proposed observations will also yield supplementary science
regarding:\\
{\bf{1)}} The UV emission line properties of the [O \Rmnum{3}]
dominated ionized plume NE of the BCG.\\
{\bf{2)}} The properties of the isolated filament $\approx 60$ kpc NE
of the BCG.\\
{\bf{3)}} The nature of the six spheroids around \irs. Are they young
rapidly star forming regions? Old, massive star clusters? Stripped
bulges?\\
{\bf{4)}} The finer details of the dusty gas obscuring the
\irs\ QSO.

%%%%%%%%%%%%%%%%%%%%%
\describeobservations
%%%%%%%%%%%%%%%%%%%%%

{\bf{NIR:}} Photometry using the archival F622W and F814W images
indicate the whiskers are bluer than the nucleus and galaxy halo, but
emit the strongest at redder wavelengths. Most of the whisker flux may
be emerging as \halpha\ emission at $\lambda_{\mathrm{obs}} = 9463$
\AA. The whiskers have sizes $< 0.1\arcs$ with other substructures
being $0.2-0.5\arcs$. The WFC3/IR (resolution $0.13\arcs/$pixel) field
of view (FOV) and F098M filter throughput are larger than the
NICMOS/NIC1-F090M in the NIR, but the angular resolution of NIC1
($0.04\arcs/$pixel) is required to resolve the whiskers. {\bf{We
    therefore request 3 orbits for our NIR science objectives: 0.75
    orbits with WFC3/IR for flux measurements and 2.25 orbits with
    NICMOS/NIC1 to resolve structure.}}

For comparison, the F622W and F814W passbands of the archival
observations are shown with the F090M and F098M passbands of the
proposed observations in Figure \ref{fig:hst}.  Our proposed
observations are longward of [O \Rmnum{3}] emission and will not be
affected by the bright [O \Rmnum{3}] BCG nebulae. The \irs\ SED peaks
longward of 2 $\mu$m, the end of the WFC3/IR-F098M passband, and
contamination from hot dust should be minimal. Cold dust contributes
$< 3\%$ of $L_{\mathrm{bol}}$ in \irs\ [11] further maximizing the
detectability of \halpha\ between 0.8-1.1 $\mu$m. The
\halpha\ contribution to the F098M and F090M filters was estimated
using the \hbeta\ equivalent width (EW) of the NE [O \Rmnum{3}] plume
provided in Tran et al. [41]. Assuming a dereddend Balmer decrement of
3, and that 40\% of the whisker fluxes arise from \halpha, the
\halpha(EW) to F090M(EW) \& F098M(EW) ratios are $\approx 15-20\%$,
which is a lower limit given that the whiskers may be
\halpha\ dominated relative to the [O \Rmnum{3}] plume and that they
cover a larger area. The WFC3/IR F098M and NIC1 F090M filters provide
sufficient wavelength coverage and transmission to make
\halpha\ emission a dominant feature in the imaging.

To determine needed exposure times, we treat \irs\ as an extended ($R
= 5\arcs$), Galactic-reddened, redshifted elliptical galaxy with
emission lines at \hbeta\ 4865 \AA, \oiii, and [Fe \Rmnum{7}] 8776
\AA\ (fluxes provided in Tran et al. [41]). The galaxy template was
flux normalized using the SDSS-$z$ detection (0.492 mJy at 0.921
$\mu$m), and standard average backgrounds were added. With the
WFC3/IR-F098M setup, a 900 s exposure (3/5\ths\ the saturation time)
gives a signal-to-noise ratio (SN) of $\sim 200$ for a 0.13 arcsec$^2$
extraction region. This will enable flux measurements to $\pm 0.08$
mag and, assuming the F814W stellar continuum to whisker flux ratio
scales linear into the F098M passband, the whiskers should have
average contrasts with the galaxy of $\approx 1.2$ mag. Using the same
galaxy template, for the NICMOS/NIC1-F090M setup, a 6700 s exposure
(1/20\ths\ of the saturation time) yields a SN of 50 for a 0.13
arcsec$^2$ extraction region. This will be sufficient to resolve
whisker detail at the resolution limit of the NIC1 instrument.

Our science goals in the NIR can be achieved in 3 orbits. At
$+41\mydeg$ declination, one orbit has 57 min of observable time. For
the first orbit, the NIC1 overhead plus setup time is $\approx 7$ m,
and $\approx 5$ min for subsequent orbits. To achieve the 112 min of
NIC1 science exposure time requires 2.25 orbits. Changing to WFC3/IR
in the third orbit requires 10 min for spacecraft maneuvering, setup,
and overhead. A WFC3/IR science exposure of 15 min leaves 17 min in
the third orbit.\\

\noindent{\bf{NUV:}} Imaging BCG and filament UV emission requires the
high throughput and angular resolution of WFC3/UVIS (resolution
$0.04\arcs/$pixel). Unambiguous detection of the \civ\ resonant line
($\lambda_{\mathrm{obs}} = 2233$ \AA) to constrain conductive heating
models for the filaments requires the dispersive resolution and
sensitivity of the COS/NUV spectrograph. {\bf{We therefore request 10
    orbits to achieve our NUV science objectives: 2.4 orbits with
    WFC3/UVIS to map spatial emission, and 7.6 orbits with COS/NUV to,
    at a minimum, place tight upper limits on \civ\ emission.}}

The \civ\ line falls near the peak of the WFC3/UVIS-F218W filter and
is one of the available central wavelengths of the COS/NUV-G225M
grating. To assist our exposure time calculations, we simulated the
expected spectrum of a conductively heated gas slab in \cloudy\ with
the dimensions of the whiskers and density of the NE [O \Rmnum{3}]
plume as a calibrator. Summing over the number of predicted whiskers
and their sky area, we predict a \civ\ flux of $\sim 6\times10^{-14}
~\flux$. The next brightest line within the F218W passband has 25\% of
the \civ\ flux and falls outside of the dispersion windows of the
G225M grating.

Assuming the model fluxes are accurate, a COS/NUV exposure time was
calculated to reach a SN per resolution element of 4 at 2233 \AA,
which should be sufficient to provide $\ga 3\sigma$ line detection at
the dispersion limit of COS/NUV. For the COS/NUV-G225M setup, we set
$\lambda_{\mathrm{cent}} = 2233$ \AA\ with a Galactic-reddened,
redshifted elliptical galaxy with an emission line at 2233 \AA. The
galaxy template was flux normalized using the \galex\ NUV detection
($m = 19.58$ mag) with standard average backgrounds assumed. The
required exposure time is 20160 sec. For the first orbit, the total
COS/NUV overhead and setup time is 9 min with 8 min for subsequent
orbits. The science exposure time then requires 7.6 orbits to
complete. A WFC3/UVIS exposure time was calculated for the
WFC3/UVIS-F218W setup with a goal of attaining a SN of 15 per
pixel. The same galaxy template was used as for the spectroscopy. The
required exposure time is 6732 sec ($\ll 1\%$ of the saturation
time). Alloting for spacecraft maneuvering and 5 min of WFC3/UVIS
overhead \& setup times, the science exposures can be achieved in 2.4
orbits, bringing our total request to 10 orbits.

The requested target field does not contain excessively bright UV
sources that might damage the WFC3/UVIS or COS/NUV instruments. The
COS/NUV Bright Object Protection policies require that count rates not
exceed a global limit of 30000 ct/s/stripe and local limit of 70
ct/s/pixel. The requested COS/NUV exposures yield a global rate of 85
ct/s, a local rate of $\ll 1$ ct/s, and no stripe exceeding 5.5
ct/s. The count rate during target acquisition imaging does not exceed
the 300 ct/s limit in a $9 \times 9$ pixel region around the
target. No brightness-related limits are imposed for WFC3/UVIS.

The NIR observations have the highest priority and would achieve many
of our scientific goals. If the TAC decided not to award the NUV
requests, specifically the COS spectroscopy, our ability to model and
constrain conductive heating of the filaments will be very
limited. However, NIR imaging of the BCG inner regions will further
resolve the structure of gas which dominates $L_{\mathrm{bol}}$ and
enable us to place interesting constraints on the AGN-environment
interaction which will be worthy of publication.

%%%%%%%%%%%
\specialreq
%%%%%%%%%%%

%%%%%%%%%%%%%%%
\coordinatedobs
%%%%%%%%%%%%%%%

%%%%%%%%%%%%%
\duplications
%%%%%%%%%%%%%

%%%%%%%%%%%%%
\pasthstusage
%%%%%%%%%%%%%


\noindent {\bf{References}}

\noindent [1] Armus et al., Ap\&SS, 266:113-118, 1999

\noindent [2] Balbus, ApJ 534:420-427, May 2000

\noindent [3] B\^irzan et al. ApJ, 607:800-809, June 2004

\noindent [4] Bower et al. MNRAS, 390:1399-1410, November 2008

\noindent [5] Cavagnolo et al. ApJ, 683:L107-L110, August 2008

\noindent [6] Cavagnolo et al. In preparation for ApJL, 2010

\noindent [7] Conselice et al. AJ, 122:2281-2300, November 2001

\noindent [8] Crawford \& Vanderriest. MNRAS, 283:1003-1014, December 1996

\noindent [9] Crawford et al. MNRAS, 363:216-222, October 2005

\noindent [10] Croton et al. MNRAS, 365:11-28, January 2006

\noindent [11] Deane \& Trentham. MNRAS, 326:1467-1474, October 2001

\noindent [12] Donahue \& Voit. ApJ, 414:L17-L20, September 1993

\noindent [13] Draine \& Salpeter. ApJ, 231:77-94, July 1979

\noindent [14] Dunn \& Fabian. MNRAS, 385:757-768, April 2008

\noindent [15] Dunn et al. MNRAS, 364:1343-1353, December 2005

\noindent [16] Evans et al. ApJ, 506:205-221, October 1998

\noindent [17] Fabian et al. MNRAS, 366:417-428, February 2006

\noindent [18] Fabian et al. Nature, 454:968-970, August 2008

\noindent [19] Ferland et al. PASP, 110:761-778, July 1998

\noindent [20] Forman et al. ApJ, 665:1057-1066, August 2007

\noindent [21] Guo et al. ApJ, 688:859-874, December 2008

\noindent [22] Hatch et al. MNRAS, 380:33-43, September 2007

\noindent [23] Heckman et al. ApJ, 338:48-77, March 1989

\noindent [24] Hines \& Wills. ApJ, 415:82-+, September 1993

\noindent [25] Kleinmann et al. ApJ, 328:161-169, May 1988

\noindent [26] McNamara \& Nulsen. ARA\&A, 45:117-175, September 2007

\noindent [27] McNamara et al. ApJ, 360:20-29, September 1990

\noindent [28] O'Dea et al. ApJ, 422:467-479, February 1994

\noindent [29] Parrish \& Quataert. ApJ, 677:L9-L12, April 2008

\noindent [30] Parrish et al. ApJ, 703:96-108, September 2009

\noindent [31] Peeters et al. ApJ, 613:986-1003, October 2004

\noindent [32] Peterson \& Fabian. Phys. Rep., 427:1-39, April 2006

\noindent [33] Peterson et al. ApJ, 590:207-224, June 2003

\noindent [34] Rafferty et al. ApJ, 687:899-918, November 2008

\noindent [35] Ruszkowski \& Oh. ArXiv e-prints: 0911.5198, November 2009

\noindent [36] Sargsyan et al. ApJ, 683:114-122, August 2008

\noindent [37] Sparks et al. ApJ, 345:153-162, October 1989

\noindent [38] Sparks et al. ApJ, 607:294-301, May 2004

\noindent [39] Sparks et al. ApJ, 704:L20-L24, October 2009

\noindent [40] Tamura et al. A\&A, 365:L87-L92, January 2001

\noindent [41] Tran et al. AJ, 120:562-574, August 2000

\noindent [42] Voigt \& Fabian. MNRAS, 347:1130-1149, February 2004

\noindent [43] Voit \& Donahue. ApJ, 486:242-+, September 1997

\noindent [44] Voit et al. ApJ, 681:L5-L8, July 2008.

\begin{figure}[htp]
  \begin{center}
    \begin{minipage}{0.48\linewidth}	
      \includegraphics*[width=\textwidth, trim=43.5mm 3.2mm 44.75mm 3.75mm, clip]{whiskers}
    \end{minipage}
    \begin{minipage}{0.47\linewidth}
      \includegraphics*[width=\textwidth, trim=47mm 7mm 49mm 5mm, clip]{xray_sub_cones}
    \end{minipage}
    \caption{{\bf{Left:}} Rest-frame 4910-6650 \AA\ \hst\ image of the
      \irs\ BCG. Green box denotes the NIC1 FOV; cyan arrows highlight
      ``whiskers,'' yellow circles enclose (stellar?) spheroids; green
      dashed line marks AGN axis. {\bf{Right}}: \chandra\ residual
      0.5-10.0 keV X-ray image. Green contour \& dashed line show VLA
      1.4 GHz radio emission and jet axis, respectively; blue contours
      trace rest-frame 3900-4570 \AA\ \hst\ image (the extension is
      the NE [O \Rmnum{3}] plume); red solid lines denote opening
      angle of QSO ionization cone; yellow dashed line indicates dark
      matter halo major-axis.}
    \label{fig:i09}
    \begin{minipage}{0.53\linewidth}
      \includegraphics*[width=\textwidth, trim=29mm 5mm 28mm 5mm, clip]{uvis}
    \end{minipage}
    \begin{minipage}{0.46\linewidth}	
      \includegraphics*[width=\textwidth, trim=32mm 5mm 40mm 5mm, clip]{nic1}
    \end{minipage}
    \caption{Comparison of passbands with relevant emission lines
      labeled. {\bf{Left:}} Proposed NUV imaging and
      spectroscopy. {\bf{Right:}} Archived optical and proposed NIR
      observations.}
    \label{fig:hst}
 \end{center}
\end{figure}

%%%%%%%%%%%%%%
\end{document}
%%%%%%%%%%%%%%

%Conduction and magnetism in the nucleus of the ULIRG IRAS 09104+4109

%% How AGN feedback energy is thermalized in the cool cores of galaxy
%% clusters is of fundamental importance in refining our understanding of
%% structure formation and evolution. Numerical simulations and sparse
%% observational evidence suggest thermal conduction, aided by magnetic
%% fields, may be a key element in solving the puzzle of how AGN suppress
%% the cooling of their hot host halos. One object which may provide
%% insight into the importance of conduction in the core of a galaxy
%% cluster is IRAS 09104+4109. Archival HST WFPC2 images hint at the
%% presence of many bright, tenuous, radial optical filaments emerging
%% from the galaxy. IRAS09 is a prime candidate for the kind of
%% magnetohydrodynamic and thermal processes which would result in
%% radial, conductively heated filaments such as those observed. By
%% imaging IRAS09 at high-resolution in the NIR \& NUV, we will constrain
%% the morphologies, compositions, and magnetic field strengths of the
%% filaments in order to determine if they are magnetically dominated, or
%% if some other processes are responsible for their structure. Utilizing
%% NUV spectroscopy, we will place constraints on the \civ\ emission from
%% the filaments, and, via models based on the NIR and NUV data,
%% determine at what level conduction is responsible for heating the
%% filaments.
