%%%%%%%%%%%%%%%%%%%%%%%%%%%%%%%%%%%%%%%%%%%%%%%%%%%%%%%%%%%%%%%%%%%%%%%%%%
%
%    phase1-GO.tex  (use only for General Observer and Snapshot proposals; 
%                      use phase1-AR.tex for Archival Research and
%                      Theory proposals use phase1-DD.tex for GO/DD proposals).
%
%    HUBBLE SPACE TELESCOPE
%    PHASE I OBSERVING PROPOSAL TEMPLATE 
%     FOR CYCLE 18 (2009)
%
%    Version 1.0, December 01, 2009.
%
%    Guidelines and assistance
%    =========================
%     Cycle 18 Announcement Web Page:
%
%         http://www.stsci.edu/hst/proposing/docs/cycle18announce 
%
%    Please contact the STScI Help Desk if you need assistance with any
%    aspect of proposing for and using HST. Either send e-mail to
%    help@stsci.edu, or call 1-800-544-8125; from outside the United
%    States, call [1] 410-338-1082.
%
%
%%%%%%%%%%%%%%%%%%%%%%%%%%%%%%%%%%%%%%%%%%%%%%%%%%%%%%%%%%%%%%%%%%%%%%%%%%%

% The template begins here. Please do not modify the font size from 12 point.

\documentclass[12pt]{article}
\usepackage{phase1,graphicx}
%\usepackage{psfig}
%\usepackage{epsf}

\begin{document}
%   1. SCIENTIFIC JUSTIFICATION
%       (see Section 9.1 of the Call for Proposals)
%
%
\def\gae{\mathrel{\hbox{\rlap{\hbox{\lower2pt\hbox{$\sim$}}}\hbox{\raise2pt\hbox
{$>$}}}}}

\justification          % Do not delete this command.
% Enter your scientific justification here. 

More than $10^{62}~\rm erg$ of gravitational binding energy is released during the formation of a $10^9~M_\odot$ black hole.  This energetic output in the form of radiation and mechanical winds exceeds the average kinetic energy of the stars in
the black hole's  host galaxy.  If supermassive black holes (SMBH) exist in the nuclei of all massive galaxies, as recent observations imply, they should play an influential role in galaxy formation.  This conjecture is supported by, for example,  the remarkably constant ratio of bulge mass to SMBH mass implied by the slopes of the  scaling relations between bulge luminosity ($L-M_{BH}$), stellar velocity dispersion ($\sigma - M_{BH}$), and black hole mass (Magorrian et al. 1998, Gebhardt et al. 2000, Ferrarese \& Merritt 2000).   Galaxy formation models that incorporate energetic feedback  from SMBHs 
are able to reproduce the luminosity function of galaxies,  the dichotomy of galaxy colors,  and
the scaling relations between bulge mass and black hole mass (Bower et al. 2006).  Direct evidence for AGN feedback is seen in Chandra Observatory X-ray images of buoyant bubbles and shock fronts associated with AGN embedded in hot halos of galaxy clusters (McNamara \& Nulsen 2007). 


It was noted recently by Rafferty et al. (2006) and Lauer et al. (2007) that the slopes of  $L-M_{BH}$ and  $\sigma - M_{BH}$ relations diverge in luminous ellipticals and brightest cluster galaxies (BCGs).  SMBH masses in luminous ellipticals predicted by the $L- M_{BH}$  are several times larger than those predicted by the $\sigma-M_{BH}$ relationship. Furthermore, the local black hole mass function determined with the $\sigma - M_{BH}$ relation implies an upper mass limit to black holes of $\sim 3\times 10^9~ M_\odot$, a value that roughly corresponds to the mass of M87's black hole (Lauer et al. 2007).
In contrast, the $L-M_{BH}$ relation predicts black hole masses  $> 10^{10}~\rm M_\odot$ in the most luminous BCGs.
The most luminous and massive BCGs exceed M87's luminosity and mass by several 
times, implying that their SMBHs may also be considerably more massive than M87's.    The existence of such ``ultramassive" black holes (UMBHs) has been inferred in distant quasars (Vestergaard 2009).   However,  no direct dynamical evidence for an UMBH has been reported.  

If UMBHs exist, their consequences would be extraordinary.   Forming such a large black hole releases an enormous amount of energy.  Most of this energy may be locked up in the
hot intracluster medium in the form of entropy and potential energy.  Excess entropy would
prevent the gas from cooling and forming stars, and may be a partial solution
to the cooling flow problem in clusters.  In fact, the need to stave off cooling flows by AGN feedback implies that some
SMBHs may have grown disproportionately large for the luminosity of their host galaxy.  The existence of
UMBHs would challenge current ideas about how and how quickly nuclear black holes form.   It is already difficult to understand how $10^9~M_\odot$ holes and their hosts formed so quickly after the big bang.  UMBHs would
add to this mystery by demanding very massive black hole seeds and rapid accretion (Natarajan \& Treister 2009).


The logical place to search for UMBHs would be the nuclei of the most luminous BCGs in the Universe.  While such searches have been contemplated (e.g., Lauer et al. 2007), they are extremely difficult. BCGs are rare  and their stellar cores are faint.  The characteristic rise in central stellar velocity dispersion that signals the existence of
a massive central object would require an unreasonably large time allocation with HST.  Searches may become routine later in the decade when 30 meter telescopes with adaptive optics become available.  The only hope in the near term would be to identify BCGs with bright, extended, nuclear line emission that have other indirect evidence for an UMBH.  

{\bf This Proposal} We have identified the BCG with the strongest evidence for an UMBH.   We propose to obtain spectra of its nuclear emission lines to search for centrally-peaked rotation and or velocity dispersion that would
signal its presence.  The BCG is centered in the $z=0.216$ cluster MS0735.6+7421, and has two remarkable properties that suggest it harbors a very massive black hole. First, the BCG has a large, $3.8$ kpc core (break) radius (1.1 arcsec, Fig. 2), which is the largest  break radius reported in the literature (McNamara et al. 2009).  Second, a Chandra X-ray image has revealed a pair of cavities in its hot halo that are roughly 200 kpc (1 arcmin) in diameter (Fig. 1).  The cavities are filled with radio emission and are surrounded by a weak shock front.  The total energy released in $\simeq 10^8$ yr is a remarkable $\sim 10^{62}$ erg.   
The significance of these properties will be discussed in turn. 
%Such a high energetic output is consistent with
%a very large black hole. 

{\bf Core Radius}  The size of a galaxy's core radius is related to its black hole mass.  A correlation exists between core radius and bulge luminosity (Laine et al. 2003).  Galaxies with  $B$-band absolute magnitudes fainter than $\sim -20$ have stellar cusps rather than cores, while luminous galaxies have core radii that scale with galaxy luminosity (C\^ot\'e et al. 2006).  The progression of cusps to cores in the most luminous galaxies can be understood as a consequence of the hierarchical growth of  galaxies and their SMBHs through mergers.   When lower luminosity cusp galaxies merge, 
their SMBHs sink toward the center of the remnant by flinging out stars.  By the time they reach the nucleus they will have flung out roughly their own mass in stars, ``scouring-out" a core or missing light (Fig. 2).  The amount of missing light multiplied by its $M/L$ should, in principle, provide a lower limit to the mass of the SMBH.   Lauer et al. (2007) and Kormendy et al. (2009) have shown that a galaxy's core radius  and missing light correlate with its black hole mass in a small sample of galaxies with dynamically determined black hole masses.  This trend is consistent with the scouring model, although not uniquely so.    In these studies the largest black hole masses are $\sim 10^{9.5}$, and the largest core radii  are
$< 1$ kpc, which is much smaller than in MS0735. Extending this logic to MS0735, McNamara et al. (2009) estimate the mass of its black hole to be
$\sim 7\times 10^{10}~\rm M_\odot$, but naturally with a very large uncertainty.

{\bf Powerful AGN} At $10^{62}$ erg, the BCG hosts the most energetic AGN outburst known (Fig. 1). Its nucleus shows no sign of a point source or quasar activity in the optical/UV/X-ray bands.  Essentially all of its power output is in mechanical form.  If its AGN is powered by accretion, its SMBH must have swallowed $\sim 6\times 10^8~ M_\odot$ of gas over the past $10^8$ yr.  The absence of strong nuclear UV emission suggests it accreted this material in a radiatively inefficient accretion flow, operating at a few percent or less of the Eddington accretion rate.   This interpretation implies that its black hole mass exceeds $10^{10}~ M_\odot$.  Furthermore, while the absolute power output from a black hole does not depend on its mass, the frequency and strength of AGN activity 
correlate with galaxy luminosity, and by inference, its black hole mass (Best et al.  2005).  Therefore, while the evidence for an UMBH mass is circumstantial, it's large core radius and high AGN power make it the most promising candidate to search for an UMBH.  

{\bf Technical Advantage} In order to detect a central mass, we must be able to measure the motions of stars or gas within the radius of influence of the black hole,
$R_g\approx 25~{\rm pc}~M_{\rm BH,9}~\sigma_{400}^{-2}$, where $M_{\rm BH,9}$ is the black hole mass in units of $10^9~M_{\odot}$, and $ \sigma_{400}$ is the assumed stellar velocity dispersion.  For a mass of $\sim 7\times 10^{10}~ M_\odot$, the
black hole's  radius of influence would be
%$R_g \approx GM_{\rm BH}/\sigma_{\rm gal}^2 = 600 ~\rm pc$ corresponding to $0.17$ arcsec (McNamara et al. 2009).
%$R_g\approx 25~{\rm pc}~M_{\rm BH,9}~\sigma_{400}^{-2}$, where $M_{\rm BH,9}$ is the black hole mass in units of $10^9~M_{\odot}$, and $ \sigma_{400}$ is the assumed stellar %velocity dispersion.
1.75 kpc, or $0.5$ arcsec at the distance of MS0735, {\it which is easily resolved with HST}. 
%dispersion of the bulge outside of $R_g$ in units of $200 ~{\rm km~s^{-1}}$,
%which corresponds to $0.17$ arcsec.  
A measurement of the stellar velocity dispersion within $R_g$ would be prohibitively expensive because of the BCG's low central surface brightness.  However, the BCG's bright nebular emission (Fig. 2) should provide a probe of the gravitational field within $R_g$ at a reasonable observational cost. 
The Keplerian circular velocity of a parcel of gas at a distance of 600 pc (0.17 arcsec)  from a $10^{10} ~ M_\odot$ black hole is $270 
~ \rm km~s^{-1}$.  This value rises
to $700 ~\rm km~s^{-1}$ for a $7\times 10^{10}~ M_\odot$ black hole, which will be easily detectable with the proposed observation,
provided the disk is seen nearly edge-on.

{\bf Consequences of this program} If we detect a rise in the nuclear velocity dispersion or rotational motion of the ionized gas, it would not in itself prove that an UMBH exists.
It would show that its existence is plausible.  However, failure to detect a 
rise in amplitude that is consistent with our prediction would place an upper limit on its mass that would in itself be interesting.
A non detection would cast serious doubt that such large black holes
are able to form.   It would also be problematical for the scouring model for massive galaxy formation.  The model predicts
core sizes that are on the same order as the black hole's gravitational radius.  The absence of an UMBH in MS0735 would then require some other mechanism than scouring or, at least, a serious revision of the scouring model. 

MS0735's power source is poorly understood.  
%Outbursts of this magnitude are generally accompanied by
%large supplies of molecular gas.  However, its molecular gas mass lies below $<3\times 10^9~ M_\odot$, and its ionized
%gas mass is smaller yet.  
Powering the outburst by accretion alone requires $\Delta M =(1-\epsilon)E_{AGN}/\epsilon c^2\sim 6\times 10^8~ M_\odot$ ($\epsilon =0.1$)
in $10^8$ yr, which is comparable to the total gas mass available in the galaxy.  Accretion at this level would require remarkably efficient shedding
of the gas's angular momentum.  This angular momentum may now reside in 
a large-scale rotating disk (cf. M87, Ford et al 1994).  Such a disk will be detectable using the proposed HST observations. 
If the mass of the hole is large enough,  the AGN may be powered by Bondi accretion of gas from the hot, X-ray atmosphere (Fig. 1).  
The Bondi accretion rate scales as $\propto \rho_{\rm gas}\times M_{BH}^2$.
A comparison between our black hole mass measurement and
a measurement of the central gas density made using a new, 500 kilosecond image obtained with the Chandra X-ray observatory (McNamara, PI),
 will show whether the Bondi mechanism is plausible.   
%The proposed observations will permit us to search for evidence for ordered motion of the ionized gas within the Bondi radius of
%the BCG, which is comparable to $R_g$ in MS0735. 
%If the black hole accreted   $6\times 10^8~ M_\odot$ in
%the past $10^8$ yr, as X-ray observations imply (McNamara et al. 2009), 

A potential problem with this measurement would be non gravitational gas motions induced by the AGN itself.  Our H$\alpha$ image taken with WIYN (Fig. 2) shows some  evidence
for an interaction with the jet.  However, by and large the H$\alpha$ isophotes follow the stellar isophotes, indicating that the 
ionized gas is relaxed.   Nevertheless, we will control for this possibility by placing slits along and perpendicular to
the radio jet.  Estimating the black hole mass will require modeling the gas motions (e.g., Marconi et al. 2006) and accounting for the stellar
mass within the radius of interest.  Lacking strong nuclear emission from the AGN, the stellar mass will be estimated cleanly using our existing HST $I$-band image and 
the spheroidal density model of van der Marel \& van den Bosch (1998). 

\newpage
\noindent
{\bf References}\\
\noindent 
Best, P.~N., et al.  2005, MNRAS, 362, 25\\
Bower, R.~G., et al. \ 2006, MNRAS, 370, 645 \\
C{\^o}t{\'e}, P., et al.\ 2006, ApJS, 165, 57 \\
Donahue, M., Stocke, J.~T., \& Gioia, I.~M.\ 1992, ApJ, 385, 49 \\
Ferrarese, L., \& Merritt, D.\ 2000, ApJ, 539, L9 \\
Ford, H.~C., et al.\ 1994, ApJ, 435, L27 \\
Gebhardt, K., et al.\ 2000, ApJ, 539, L13 \\
Kormendy, J., \& Bender, R.\ 2009, ApJ, 691, L142 \\
Laine, S., et al. \ 2003, AJ, 125, 478\\
Lauer, T.~R., et al.\ 2007a, ApJ, 662, 808 \\
Marconi, A., et al. \ 2006, A\&A, 448, 921 \\
Magorrian, J., et al.\ 1998, AJ, 115, 2285\\ 
McNamara, B.~R., \& Nulsen, P.~E~J., ARAA, 45, 117 \\
McNamara, B.~R., et al. \ 2009, ApJ, 698, 594 \\
Natarajan, P., \& Treister, E.\ 2009, MNRAS, 393, 838 \\
Rafferty, D.~A., et al. \ 2006, ApJ, 652, 216 \\
van der Marel, R.~P., \& van den Bosch, F.~C.\ 1998, AJ, 116, 2220 \\
Vestergaard, M.\ 2009, arXiv:0904.2615\\

\begin{figure}[ht]
  \begin{center}
    \begin{minipage}{0.47\linewidth}    
      \includegraphics*[width=\textwidth]{fig1}
    \end{minipage}
    \begin{minipage}{0.52\linewidth}    
      \includegraphics*[width=\textwidth]{fig4a}
    \end{minipage}
    \caption{{\it{Left:}} X-ray 
emission (blue), 320 MHz radio emission (red) superposed on an HST I-band image of the 
z=0.21 cluster MS0735.6+7421 (McNamara et al. 2009).  The image is 700 kpc on a 
side.  Giant cavities, each 200 kpc (1 arcmin) in diameter, were excavated by the 
AGN.  The mechanical energy is reliably measured in X-rays by multiplying the gas 
pressure by the volume of the cavities, and from the properties of the surrounding 
shock front.  With a mechanical energy of $10^{62}$ erg, MS0735 is the most energetic 
AGN known. {\it{Right:}} Detailed view of the HST image of the BCG showing the nucleus and
surrounding dust filaments.}
    \begin{minipage}{0.52\linewidth}    
      \includegraphics*[width=\textwidth, trim=15mm 5mm 0mm 10mm, clip]{fig9}
    \end{minipage}
    \begin{minipage}{0.47\linewidth}    
      \includegraphics*[width=\textwidth]{MS0735_Halpha}
    \end{minipage}
    \caption{{\it{Left:}} $I$-band HST surface brightness profile with
      Sersic profile superposed. The Sersic Law overshoots the the
      profile within the break radius.  Beyond the the break radius,
      the light profile closely follows the $R^{1/4}$
      law. The region between the observed profile and Sersic profile within 1 arcsec is the missing light.
      {\it{Right:}} H$\alpha$ + [N II] image of the BCG taken with the WIYN telescope.}
    \label{fig}
  \end{center}
\end{figure}


%%%%%%%%%%%%%%%%%%%%%%%%%%%%%%%%%%%%%%%%%%%%%%%%%%%%%%%%%%%%%%%%%%%%%%%%%%%

%   2. DESCRIPTION OF THE OBSERVATIONS
%       (see Section 9.2 of the Call for Proposals)
%
%
\describeobservations   % Do not delete this command.
% Enter your observing description here.

We will measure the velocity and velocity width profiles of the ionized gas along the major ($-15~\rm degrees$) and minor axes of the BCG, which are
roughly parallel to and perpendicular to the radio jet ($\rm PA_{jet}= - 22$ degrees; Fig. 1).  The observation will be sensitive to black hole masses
 $\gae 7\times 10^9~M_{\odot}$, which lies just above the expected mass from the $L-M_{bh}$ scaling relation.
Therefore, our black hole mass measurement will be sensitive to any significant upward departure from the mean scaling relation,
and will place a strong constraint on the upper mass limit of black holes for what is currently
the best UMBH candidate.  Our program has few risks, because an upper limit on the black hole mass will still be interesting.
But the potential payoff for HST will be high should we detect a $>10^{10}~M_\odot$ black hole, which we believe will be the likely outcome.  
We explored the possibility of using Gemini's integral field spectrograph with adaptive optics but
the system is not operational at such a high declination.  While it might be possible to detect a black hole approaching
$7\times 10^{10}~\rm M_\odot$ from the ground, no available ground based instrumentation would be able to constrain masses
below $\sim 10^{10}~\rm M_\odot$ at the distance of MS0735.  This is science HST was designed to perform.

The nuclear position will be defined by the locations of the radio core at 320 MHz and the peak of the X-ray emission,
which are coincident at RA 07:41:44.595; Dec: +74:14:39.74.  At $z=0.216$, the redshifted H$\alpha$ feature lies at $\lambda 7980$ \AA.  We will 
attempt to obtain a spectrum of the $H\alpha$ + [N II] lines to measure the line centroids and shapes as
a function of position along the slit.  We will use
the G750M grating in first order centered at $\lambda 7795$ \AA\ and the $52\times 0.1$ arcsec slit. 
 This configuration provides a reciprocal dispersion of $0.56$ \AA\ per pixel and an instrumental resolution of $\sim 50~\rm km~s^{-1}$.  Our observing time estimate conservatively assumes the lines will be broadened by $500~\rm km~s^{-1}$,
 which corresponds to $13$ \AA\ FWHM.   The total $H\alpha$ + [N II]  flux from the nebula based on ground based images (Fig. 2)
is $9\times 10^{-15}~\rm erg~s^{-1}~cm^{-2}$ (Donahue et al. 1992).  The emission is strongly peaked in the nucleus. Twenty percent of the flux comes from the inner arcsec, and we expect the flux to be centrally peaked within this radius.  Therefore, for the purpose
of estimating exposure time, we assume that $5\%$ of the total flux, $4.5\times 10^{-16}~\rm erg~s^{-1}~cm^{-2}$, is emerging from the central 0.1 arcsec.  We assume a foreground extinction of $0.023$ mag.  The I-band HST image
shows dust patches near the nucleus, but they are relatively weak and should not strongly affect our analysis (McNamara
et al. 2009).  The BCG has no appreciable unresolved nuclear emission or star formation, so the measurements should
be clean.   We require at least
a $S/N\sim 6$ per resolution element, which corresponds to roughly 5000 photons per resolution element within
the $H\alpha$ + [N II] lines.  The STIS Spectroscopy Exposure Time Calculator estimates an integration time of 11,720 seconds per slit position will be required to achieve this goal.  The slit will cross the nucleus twice, resulting in an excellent spectrum
of the nucleus.   Our proposed setup and the procedures we will use for modeling the line profiles have become standard in this field, and have been successfully used to
measure black hole masses in several galaxies (e.g., NGC 5252, Capetti et al. 2005, A\&A, 431, 465).  Our
proposed $S/N$ should exceed the values used to successfully measure the black hole mass in Centaurus A using
[SIII] lines (Marconi et al. 2006, A\&A, 448, 921).

 
The observation is well within the total count and count rate tolerances of STIS as given in the instrument manual.  
We assume CRSPLIT=2, GAIN=1.  We will require target acquisition exposure and peak up (ACQ/Peak).  MS0735
lies at DEC: +74, which gives it a 60 min. visibility per orbit in 3 gyro mode.  Subtracting overheads (guide star
acquisition, peak up, etc.) we assume net visibility of 50 min per orbit.  Two slit positions at $11, 720$ sec per position
yields 4 orbits per position for a total of 8 orbits.



%%%%%%%%%%%%%%%%%%%%%%%%%%%%%%%%%%%%%%%%%%%%%%%%%%%%%%%%%%%%%%%%%%%%%%%%%%%

%   3. SPECIAL REQUIREMENTS
%       (see Section 9.3 of the Call for Proposals)
%
%
\specialreq             % Do not delete this command.
% Justify your special requirements here, if any.

%%%%%%%%%%%%%%%%%%%%%%%%%%%%%%%%%%%%%%%%%%%%%%%%%%%%%%%%%%%%%%%%%%%%%%%%%%%

%   4. COORDINATED OBSERVATIONS
%       (see Section 9.4 of the Call for Proposals)
%
%
\coordinatedobs          % Do not delete this command.
% Enter your coordinated observing plans here, if any.

%%%%%%%%%%%%%%%%%%%%%%%%%%%%%%%%%%%%%%%%%%%%%%%%%%%%%%%%%%%%%%%%%%%%%%%%%%%

%   5. JUSTIFY DUPLICATIONS
%       (see Section 9.5 of the Call for Proposals)
%
%
\duplications           % Do not delete this command.
% Enter your duplication justifications here, if any.

%%%%%%%%%%%%%%%%%%%%%%%%%%%%%%%%%%%%%%%%%%%%%%%%%%%%%%%%%%%%%%%%%%%%%%%%%%%


%   6. PAST HST USAGE AND CURRENT COMMITMENTS
%       (see Section 9.8 of the Call for Proposals)
%
%
\pasthstusage  % Do not delete this command.
% Enter your Past HST Usage and current commitments for the PI
% and other key investigators.
``The Nuclear Environment of the Galaxy Hosting the Largest Known Radio Outburst," B.R. McNamara PI.  Imaging of the central galaxy in the MS0735
cluster.  See Fig. 1, this proposal.  Data reduced, interpreted, and published in McNamara et al. (2009).
%%%%%%%%%%%%%%%%%%%%%%%%%%%%%%%%%%%%%%%%%%%%%%%%%%%%%%%%%%%%%%%%%%%%%%%%%%%

\end{document}          % End of proposal. Do not delete this line.
                        % Everything after this command is ignored.

