\begin{figure}
  \begin{center}
    \begin{minipage}{0.495\linewidth}
      \includegraphics*[width=\textwidth, trim=27mm 5mm 35mm 10mm, clip]{radiofit.eps}
    \end{minipage}
    \begin{minipage}{0.495\linewidth}
      \includegraphics*[width=\textwidth, trim=27mm 5mm 35mm 10mm, clip]{radiofit_lobes.eps}
    \end{minipage}
    \caption{Best-fit synchrotron models for the radio spectrum of the
      full radio source ({\it{left}}) and just the lobes
      ({\it{right}}). See Section \ref{sec:radio} for description of
      models. Stars denote points excluded in fitting and the upper
      limits shown are $1\sigma$.}
    \label{fig:radio}
  \end{center}
\end{figure}

\begin{figure}
  \begin{center}
    \begin{minipage}{0.495\linewidth}
      \includegraphics*[width=\textwidth]{i09.ps}
    \end{minipage}
    \begin{minipage}{0.495\linewidth}
      \includegraphics*[width=\textwidth]{unsharp.ps}
    \end{minipage}
    \caption{\cxo\ 0.5-10.0 keV X-ray clean images of \irs. Note the
      difference in physical area shown in the two
      panels. {\it{Left:}} Box is 300 kpc on a side and the image is
      unsmoothed. The white dashed curve highlights a faint arc of
      emission and the dashed black arrow in the nucleus denotes a
      skewing of core emission toward the NE. Two regions of lower
      than average surface brightness are also pointed
      out. {\it{Right:}} Unsharp masked image resulting from
      differencing images smoothed by a $2\arcs$ and $5\arcs$
      Gaussian. Blue contours are the 1.4 GHz radio emission discussed
      in Section \ref{sec:radio}, and the dashed black curve is the
      faint arc from the left panel.}
    \label{fig:imgs}
  \end{center}
\end{figure}

\begin{figure}
  \begin{center}
    \begin{minipage}{0.33\linewidth}
      \includegraphics*[width=\textwidth]{resid_betamod.ps}
    \end{minipage}
    \begin{minipage}{0.33\linewidth}
      \includegraphics*[width=\textwidth]{resid_freemod.ps}
    \end{minipage}
    \begin{minipage}{0.33\linewidth}
      \includegraphics*[width=\textwidth]{resid_fixedmod.ps}
    \end{minipage}
    \begin{minipage}{0.33\linewidth}
      \includegraphics*[width=\textwidth]{resid_beta.ps}
    \end{minipage}
    \begin{minipage}{0.33\linewidth}
      \includegraphics*[width=\textwidth]{resid_free.ps}
    \end{minipage}
    \begin{minipage}{0.33\linewidth}
      \includegraphics*[width=\textwidth]{resid_fixed.ps}
    \end{minipage}
    \caption{Top row are various best-fit surface brightness models
      (discussed in Section \ref{sec:sub}), and bottom row are
      residual images from differencing data and model. All real X-ray
      images are from the clean \cxo\ data, cover the energy range
      0.5--7.5 keV, and have units ph \pcmsq\ \ps\ pix$^{-1}$. {\it
        Left column:} Images using model-A: the double $\beta$-model
      fit to the 1D X-ray surface brightness. {\it Middle column:}
      Images using model-B: 2D surface brightness fit with free
      ellipticity, position angle, and centroid. {\it Right column:}
      Images using model-C: 2D surface brightness fit with fixed
      ellipticity, position angle, and centroid.}
    \label{fig:multiresid}
  \end{center}
\end{figure}

\begin{figure}
  \begin{center}
    \begin{minipage}{0.5\linewidth}
      \includegraphics*[width=\textwidth, trim=15mm 10mm 10mm 10mm, clip]{ellpa.ps}
      \caption{Best-fit radial ellipticity ($\epsilon$), position
        angle ($\phi$), and centroid variation ($\Delta C$) from
        $C$[J2000] = (09:13:45.5; +40:56:28.4) for 2D X-ray isophotes
        exterior to the ICM substructures. Mean values are shown as
        horizontal dashed-dotted lines and have values $\epsilon =
        0.14$, $\phi = -76\mydeg$, and $\Delta C = 1.5\arcs$.}
      \label{fig:ellpa}
    \end{minipage}
  \end{center}
\end{figure}

\begin{figure}
  \begin{center}
    \begin{minipage}{0.495\linewidth}
      \includegraphics*[width=\textwidth]{cavs.ps}
    \end{minipage}
    \begin{minipage}{0.495\linewidth}
      \includegraphics*[width=\textwidth]{resid_cones.ps}
    \end{minipage}
    \caption{{\it{Left}}: Residual X-ray image made using surface
      brightness model-B. Green contours trace 1.4 GHz radio emission
      above $3\sigma_{RMS}$, and regions of interest are
      labeled. {\it{Right}}: Same residual image as left panel with
      gray-scale biased to lower values. Green contour traces
      $3\sigma_{RMS}$ 1.4 GHz radio emission, and the dashed green
      line shows the jet axis. Blue contours trace
      $\lambda_{\mathrm{rest}} \approx 3900-6650$ \AA\ emission as
      seen with \hst. The red dashed and solid lines show the mean
      direction and opening angle, respectively, of the nuclear UV
      scattering bicone (see H99 for discussion). Yellow dashed line
      represents the approximate semi-major axis of the galaxy cluster
      \citep[see][]{1988ApJ...328..161K}. The EEx lies exactly along
      the axis of beamed nuclear radiation which is also coincident
      with the nuclear radio spur.}
    \label{fig:resid}
  \end{center}
\end{figure}

\begin{figure}
  \begin{center}
    \begin{minipage}{0.9\linewidth}
      \includegraphics*[width=\textwidth, trim=6mm 0mm 6mm 6mm, clip]{iras09_nhfro.eps}
      \caption{Gallery of radial ICM profiles of temperature (\tx),
        abundance ($Z$), surface brightness (SB), gas density
        (\nelec), total gas pressure ($P$), entropy ($K$), cooling
        time (\tcool), and enclosed X-ray luminosity (\lx). Vertical
        black dashed lines mark the approximate end-points of
        cavities. Horizontal dashed line in \tcool\ profile marks
        $\Hn^{-1}$ at $z=0.4418$. For \lx\ profile, dashed line marks
        \lcool, and dashed-dotted line marks total \pcav.}
      \label{fig:gallery}
    \end{minipage}
  \end{center}
\end{figure}

\begin{figure}
  \begin{center}
    \begin{minipage}{0.495\linewidth}
      \includegraphics*[width=\textwidth, trim=30mm 5mm 35mm 10mm, clip]{tsyn_k.eps}
    \end{minipage}
    \begin{minipage}{0.495\linewidth}
      \includegraphics*[width=\textwidth, trim=30mm 5mm 35mm 10mm, clip]{tsyn_k_lobe.eps}
    \end{minipage}
    \caption{Synchrotron age (\tsync) as a function of $k$, the ratio
      of lobe energy in non-radiating particles to that in
      relativistic electrons, for three values of $\Phi$, the volume
      filling factor of the radiating particle
      population. {\it{Left:}} Ages for full radio source;
      {\it{Right:}} ages for lobes only.}
    \label{fig:tsync}
  \end{center}
\end{figure}

\begin{figure}
  \begin{center}
    \begin{minipage}{0.5\linewidth}
      \includegraphics*[width=\textwidth]{hardness.ps}
    \end{minipage}
    \caption{Hardness ratio map of \irs. Green contours trace the
      highest and lowest significance regions of the continuous 1.4
      GHz radio emission. White ellipse is the 90\% EEF source region
      and dashed white wedges are background regions. The areas with
      the largest $HR$ are coincident with the central source and the
      termination point of the northern radio jet.}
    \label{fig:hardness}
  \end{center}
\end{figure}

\begin{figure}
  \begin{center}
    \begin{minipage}{\linewidth}
      \includegraphics*[angle=270, width=\textwidth, trim=0mm 20mm 0mm 0mm, clip]{nuc.ps}
    \end{minipage}
    \vspace{0.25cm}
    \caption{Background-subtracted nuclear spectrum and best-fit model
      for the 1999 (red) \& 2009 (black) \cxo\ data and 1998
      \bepposax\ PDS data (blue). Data has been binned to $3\sigma$
      significance. The significant flux difference below 1.3 keV is a
      result of the greater effective area of the ACIS-S3 CCD in 1999
      versus ACIS-I3 in 2009.}
    \label{fig:nucspec}
  \end{center}
\end{figure}

\begin{figure}
  \begin{center}
    \begin{minipage}{0.5\linewidth}
      \includegraphics*[width=\textwidth]{exspec.ps}
    \end{minipage}
    \caption{Residual X-ray image using model-B after dividing by
      $10^{-7}$ and scaling each pixel by a power of 1000 to enhance
      substructure contrast. The outermost contour of continuous 1.4
      GHz radio emission is overlaid for reference. Solid red lines
      bound spectral extraction regions for each excess, and nearest
      red dashed wedges bound associated background regions.}
    \label{fig:exspec}
  \end{center}
\end{figure}

\begin{figure}
  \begin{center}
    \begin{minipage}{\linewidth}
      \includegraphics*[width=\textwidth, trim=0mm 0mm 0mm 0mm, clip]{opt_xray.ps}
    \end{minipage}
    \caption{Exposure-corrected 0.5--7.5 keV X-ray residual image
      zoomed-in on the EEx with units ph
      \pcmsq\ \ps\ pix$^{-1}$. Green contour is 1.4 GHz radio emission
      at $3\sigma_{\rm{rms}}$, blue contours are optical emission
      ($\lambda_{\mathrm{rest}} \approx 3900-6650$ \AA) in 10
      log-space steps from 0.5-50 ct s$^{-1}$, and cyan contours are
      4.8 GHz radio emission beginning at $3\sigma_{\rm{rms}}$. The
      optical plume-like structure to the northeast of the galaxy is a
      bright [O \Rmnum{3}] emission nebula
      \citep[see][]{1996MNRAS.283.1003C}. The red vector denotes the
      best-fit emission axis for the UV scattering bi-cone, and the
      red wedge marks its extent and spans the $3\sigma$ opening angle
      (see H99). The dashed, white region is the EEx spectral
      extraction region, note that it was defined without considering
      the location of any other emission features. The residual
      nuclear X-ray emission extends exclusively to the north of the
      nucleus and is where nearly all of the \feka\ emission
      originates.}
    \label{fig:eex}
  \end{center}
\end{figure}

\begin{figure}
  \begin{center}
    \begin{minipage}{\linewidth}
      \includegraphics*[angle=270, width=\textwidth, trim=0mm 20mm 0mm 0mm, clip]{qso_ec.ps}
    \end{minipage}
    \caption{Background-subtracted EEx spectrum binned to
      $3\sigma$ significance. The black points are data and the
      solid black line is the normalized \cloudy\ model for a
      quasar irradiated nebula and ICM.}
    \label{fig:qso}
  \end{center}
\end{figure}

\begin{figure}
  \begin{center}
    \begin{minipage}{0.5\linewidth}
      \includegraphics*[width=\textwidth]{f09.eps}
    \end{minipage}
    \caption{Figure taken from \citet{2009MNRAS.394L..89F} showing
      absorbing column density versus the effective Eddington ratio of
      AGN in the 9-month \swift/BAT survey. \irs\ is denoted by a star
      in the upper right of the figure, and it's general location is
      shown with a dark shaded region. The solid curved line denotes
      \leff\ for a Galactic dust grain abundance, and the meaning of
      the different symbols is not relevant to our
      discussion. \citet{2009MNRAS.394L..89F} divide the
      \leff-\nh\ plane into three regions: 1) an area where dusty,
      obscuring clouds near the nucleus of the AGN host galaxy are
      ``long-lived'' meaning they are not expelled; 2) at low column
      densities, the presence of dust lanes in the host galaxy
      periphery can explain measured absorption values; 3) in the
      ``forbidden'' region, dusty clouds with a sight-line to the
      nucleus experience a super-Eddington AGN and are likely
      short-lived from possibly being expelled from the nucleus or
      being broken apart.}
    \label{fig:f09}
  \end{center}
\end{figure}
