%%%%%%%%%%%%%%%%%%%
% Custom commands %
%%%%%%%%%%%%%%%%%%%

\newcommand{\mytitle}{Radiative and Mechanical Feedback of the Quasar in \iras}
\newcommand{\mystitle}{Feedback in \iras}
\newcommand{\iras}{IRAS 09104+4109}
\newcommand{\irs}{IRAS09}
\newcommand{\rxj}{RX J0913.7+4056}
\newcommand{\radec}{R.A.(J2000) $=09^h 13^m 45^s.5$, Dec.(J2000) $=+40^{\mydeg} 56^{\arcm} 28^{\arcs}$}
\newcommand{\ellradec}{09^h 13^m 45^s.5, +40^{\mydeg} 56^{\arcm} 28^{\arcs}}
\newcommand{\tsync}{\ensuremath{t_{\mathrm{sync}}}}
\newcommand{\refl}{\ensuremath{r_{\mathrm{refl}}}}
\newcommand{\fekaew}{\ensuremath{\mathrm{EW}_{\mathrm{K}\alpha}}}
\newcommand{\cltx}{\ensuremath{T_{\mathrm{cl}}}}
\newcommand{\leff}{\ensuremath{\lambda_{\mathrm{Edd}}}}
\newcommand{\mnine}{\ensuremath{\mathrm{M}_9}}
\newcommand{\lqso}{\ensuremath{\lbol^{\mathrm{QSO}}}}

%%%%%%%%%%
% Header %
%%%%%%%%%%

\documentclass[useAMS,usenatbib]{mn2e}
\usepackage{graphicx, here, common, longtable, ifthen, amsmath,
amssymb, natbib, lscape, subfigure, mathptmx, url, times, array}
\usepackage[abs]{overpic}
\usepackage[pagebackref,
  pdftitle={\mytitle},
  pdfauthor={Dr. Kenneth W. Cavagnolo},
  pdfsubject={ApJ},
  pdfkeywords={},
  pdfproducer={LaTeX with hyperref},
  pdfcreator={LaTeX with hyperref}
  pdfdisplaydoctitle=true,
  colorlinks=true,
  citecolor=blue,
  linkcolor=blue,
  urlcolor=blue]{hyperref}

\title[\mystitle]{\mytitle}

\author[Cavagnolo et al.]{K. Cavagnolo$^{1}$\thanks{Email:
    kcavagno@uwaterloo.ca}, M. Donahue$^{2}$, B. McNamara$^{1,3,4}$,
  G. M. Voit$^{2}$, and M. Sun$^{3}$\\
$^{1}$Department of Physics and Astronomy, University of Waterloo,
Waterloo, ON N2L 3G1, Canada.\\
$^{2}$Department of Physics and Astronomy, Michigan State University,
East Lansing, MI, 48824-2320, USA.\\
$^{3}$Perimeter Institute for Theoretical Physics, 31 Caroline St. N.,
Waterloo, Ontario, N2L 2Y5, Canada\\
$^{4}$Harvard-Smithsonian Center for Astrophysics, 60 Garden Street,
Cambridge, MA 02138, USA\\
$^{5}$Department of Astronomy, University of Virginia,
Charlottesville, VA, 22904, USA.}

\begin{document}

\date{Accepted 2010 Month DD. Received 2010 Month DD; in original form 2010 Month DD}

\pagerange{\pageref{firstpage}--\pageref{lastpage}} \pubyear{2010}

\maketitle

\label{firstpage}

%%%%%%%%%%%%
% Abstract %
%%%%%%%%%%%%

\begin{abstract}
%%   We present a detailed study of the enigmatic hyperluminous infrared
%%   BCG \iras\ using deep Chandra X-ray data. The X-ray observations
%%   reveal AGN excavated cavities in the galaxy's halo, gas uplifted by
%%   the AGN outflows, and non-thermal emission at the termination points
%%   of the AGN jets. We also demonstrate that the properties of the
%%   nuclear point source are consistent with an AGN embedded in a
%%   moderately Compton-thick medium. Our analysis focuses on discussing
%%   IRAS09 as an object at the turning point between the
%%   radiatively-dominated and mechanically-dominated modes of AGN
%%   feedback.  We highlight how IRAS 09104+4109 is an ideal test case of
%%   a very short-lived but highly active stage of cluster and central
%%   galaxy formation.
\end{abstract}

%%%%%%%%%%%%
% Keywords %
%%%%%%%%%%%%

\begin{keywords}
cooling flows -- galaxies: clusters: general -- galaxies:
  clusters: individual (\iras)
\end{keywords}

%%%%%%%%%%%%%%%%%%%%%%
\section{Introduction}
\label{sec:intro}
%%%%%%%%%%%%%%%%%%%%%%

\iras\ (hereafter, \irs) is an uncommon low-redshift ($z = 0.4418$)
ultraluminous infrared galaxy (ULIRG; $L_{\mathrm{IR}} > 10^{12}
~\lsol$). Unlike most ULIRGs, \irs\ is the brightest cluster galaxy
(BCG) in a rich galaxy cluster, but unlike most BCGs, \irs\ is a
Seyfert-2 with 99\% of the bolometric luminosity emerging longward of
1 \mymicron\ due to a heavily-obscured ($\nhobs > 10^{24} ~\pcmsq$)
quasar \citep{1988ApJ...328..161K, 1993ApJ...415...82H,
  1994ApJ...436L..51F, 1998ApJ...506..205E, 2000A&A...353..910F,
  2001MNRAS.321L..15I}. The present analysis shows the host cluster,
\rxj, has a mass $\sim 10^{15} ~\msol$, a central cooling time $\tcool
< 1$ Gyr, and a spectroscopically determined (non-grating) mass
deposition rate $\mdot > 100 ~\msolpy$. Even at $0.1 ~\mdot$, a gas
reservoir of more than $10^{10} ~\msol$ should condense out of the
intracluster medium (ICM) onto the BCG in $\sim 1$ Gyr. However, while
$\sim 10^9 ~\msol$ of hot dust \citep{1997A&A...318L...1T} and
multiple bright optical nebulae are detected in \irs, unlike other
Seyfert-2 galaxies, \irs\ has $< 10^{10} ~\msol$ of CO determined
H$_2$ gas \citep{1998ApJ...506..205E}, $< 10^8 ~\msol$ of cold dust
\citep{2001MNRAS.326.1467D}, and no detected polycyclic aromatic
hydrocarbon emission or silicate absorption features
\citep{2004ApJ...613..986P, 2008ApJ...683..114S}. Further, near-IR
spectroscopy reveals that \irs\ emits $\sim 10^{42 \dash 43} ~\lum$ in
\halpha\ \citep{1996MNRAS.283.1003C, 1998ApJ...506..205E}. If the CO
upper limits are correct, then compared to other BCGs, the
\irs\ emission line properties indicate a molecular gas poor galaxy
\citep{2001MNRAS.328..762E}, accentuating the lack of strong
indicators of cold, dusty, molecular gas around such a powerful quasar
(QSO).

The \irs\ gas-to-dust ratio ($< 300$) is suspiciously small compared
to other cooling flow BCGs (typically $> 500$). The answer to this
riddle may be that the QSO in \irs\ is driving a powerful
outflow. Integral field spectroscopy presented by
\citet{1996MNRAS.283.1003C} indicates the presence of a $> 1000
~\kmps$ emission line outflow coincident with the nucleus. The CO
observations presented in \citet{1998ApJ...506..205E} were not
sensitive to high-velocities, $> 300 ~\kmps$, the regime where
expelled molecular gas may reside. We demonstrate in this paper that
the powerful BCG radio source is driving a supersonic plasma outflow
with mechanical power $\sim 10^{44} ~\lum$. We also show, via
detection of an X-ray excess 13-26 kpc NE of the nucleus, radiation
from the $\sim 10^{47} ~\lum$ QSO is escaping the nucleus and
interacting with the ICM in the same region as strongly photoionized
\& polarized optical emission are detected.

Based on the radiative, mechanical, and ostensible gas-poorness
properties of \irs, we suggest the system is undergoing a rare phase
of BCG assembly where radiative and mechanical feedback, in an
approximately 100:1 ratio, are simultaneously conspiring to quench
cooling within and around the host galaxy. Numerical simulations
suggest such a phase of galaxy assembly should exist, though no
conclusive examples have been found. Cosmological simulations
typically put radiation- and kinetic-dominated feedback into a
distinct early-time quasar-mode \citep[\eg][]{2005Natur.435..629S} and
a late-time radio-mode \citep[\eg][]{croton06, bower08},
respectively. The quasar-mode of feedback is expected to be brief,
expelling large quantities of molecular gas from the host galaxy
\citep{2006ApJ...642L.107N}, while the prolonged, intermittent
radio-mode heats the host environment, regulating cooling for the rest
of the galaxy's life \citep[see][for a review]{mcnamrev}. Examples of
mechanical feedback are many \citep[\eg][]{perseus1, ms0735}, and the
best evidence of AGN feedback expelling molecular gas comes from
low-luminosity, low-redshift early-type galaxies
\citep[\eg][]{2009ApJ...690.1672S}, but, to our knowledge, \irs\ is
the only system where both are observed {\bf{XXX: Is this true?}}.

In this paper, we present the argument that \irs\ is undergoing a
crossover between quasar-mode and radio-mode feedback. We explore the
possibility that the QSO in \irs\ has expelled, and/or sufficiently
heated, the galactic gas reservoir which provides fuel for the
supermassive black hole (SMBH), and that the associated jets are
suppressing cooling of the X-ray halo from which the galactic
reservoir would be re-supplied. Further, we confirm the conclusion of
\citet{1996MNRAS.283.1003C} that the BCG nebular velocities and gas
masses are inconsistent with a scenario where the gas reservoir was
deposited by stripping or tidal interactions. Based on the AGN
outburst energetics, it is shown that Bondi accretion is unlikely to
be responsible for fueling the AGN. We thus conclude that the BCG
reached its current state predominately through the influence of
cold-mode accretion and a cooling flow. However, this conclusion is
limited by the need for deeper sub-mm observations to probe molecular
gas mass scales of $\sim 10^9 ~\msol$ and velocities $> 300 ~\kmps$.

Reduction of X-ray and radio data is discussed in Section
\ref{sec:obs}. ICM global (Section \ref{sec:global}) and radial
(Section \ref{sec:rad}) properties are then analyzed. Details
regarding the ICM cavities, SMBH fueling, and QSO heating of the ICM
are given in Sections \ref{sec:cavs}, \ref{sec:fuel}, and
\ref{sec:excess}, respectively. The complex nuclear source is
discussed in Section \ref{sec:centsrc}. We provide interpretation of
the results throughout and a supply a brief summary in Section
\ref{sec:summ}. \LCDM\ For our assumed cosmography, the cluster
redshift of $z = 0.4418$ corresponds to $\approx 9.1$ Gyr for the age
of the Universe, $\da \approx 5.72$ kpc arcsec$^{-1}$, and $\dl
\approx 2.45~\Gpc$. All errors are 90\% confidence unless stated
otherwise.

%%%%%%%%%%%%%%%%%%%%%%%%%%%%%%%%%%%%%%%%%
\section{Observations and Data Reduction}
\label{sec:obs}
%%%%%%%%%%%%%%%%%%%%%%%%%%%%%%%%%%%%%%%%%

For all spectral fits, the Galactic absorbing column density was fixed
at $\nhgal = 1.58 \times 10^{20} ~\pcmsq$ \citep{lab}. Unless
otherwise noted, spectral fitting was performed with the
\chisq\ statistic in \xspec\ 12.4.0 \citep{xspec} using an absorbed
single-temperature \mekal\ model \citep{mekal1, mekal2} with abundance
as a free parameter (\citealt{ag89} solar abundance distribution
assumed) over the energy range 0.7-7.0 keV.  For all calculations
involving ICM gas, we assume a mean molecular weight of $\mu = 0.597$
and adiabatic index $\gamma = 5/3$.

%%%%%%%%%%%%%%%%%%%%%
\subsection{\chandra}
\label{sec:xray}
%%%%%%%%%%%%%%%%%%%%%%

A 77.2 ks observation of \irs\ was taken on 2009 January 09 with the
ACIS-I instrument (\dataset [ADS/Sa.CXO#Obs/10445] {ObsID 10445}; PI
Cavagnolo). The 9 ks archival \chandra\ observation of \irs\ from 1999
November 03 taken with the ACIS-S array was included in our analysis
(\dataset [ADS/Sa.CXO#Obs/00509] {ObsID 509}; PI Fabian). Both
datasets were reprocessed and reduced using \ciao\ and
\caldb\ versions 4.2. X-ray events were selected using \asca\ grades,
and corrections for the ACIS gain change, charge transfer
inefficiency, and degraded quantum efficiency were applied. Point
sources were located and excluded using {\textsc{wavdetect}} and
visual inspection. Light curves from a source free region of each
observation were created for a front-illuminated and back-illuminated
CCD and compared to look for flares. Time intervals which fell outside
$20\%$ of the mean background count rate were excluded. After flare
exclusion, the final exposure times for ObsID 509 and 10445 were 7 ks
and 76 ks, respectively.

For imaging analysis, the flare-clean events files were reprojected to
a common tangent point and summed. The astrometry of the ObsID 509
dataset was improved using a new aspect solution created with the
\ciao\ tool {\textsc{reproject\_aspect}} and the positions of several
field sources. The positional uncertainty introduced by the strong
angle dependence of the off-axis \chandra\ point-spread function (PSF)
was minimized by using sources closest to the aim-point. After
correction of the astrometry, the positional accuracy between both
observations was comparable to the resolution limit of the ACIS
detectors. We refer to the final point source free, flare-clean,
exposure-corrected images as the ``clean'' images. In Figure
\ref{fig:imgs} are the 0.5-10.0 keV mosaiced clean image of \rxj, a
zoom-in of the core region harboring \irs, and photons in the energy
range 4.35-4.50 keV associated with the \feka\ fluorescence line from
the nucleus (discussed in Section \ref{sec:centsrc}). Unless stated
otherwise, the X-ray analysis in this paper relates to the
\chandra\ data only -- \xmm\ and \bepposax\ data were re-analyzed for
the purposes of cross checking our \chandra\ analysis.

%%%%%%%%%%%%%%%%%
\subsection{\xmm}
\label{sec:xmm}
%%%%%%%%%%%%%%%%%

\xmm\ observed \irs\ on 2003 April 23 for 14 ks with the EPIC PN and
MOS detectors (ObsID 0147671001; PI Fiore). Data was reprocessed using
SAS version 7.1 and CCF release 258. Events files were created using
the tools {\textsc{EMCHAIN}} and {\textsc{EPCHAIN}} for patterns
0-4. Light curves were extracted from the energy range 10-12 keV for
the full field after \chandra\ identified point sources and cluster
emission were removed. After flare exclusion, the effective exposure
times for PN and MOS were 10 ks and 12 ks, respectively. A source
spectrum grouped to 20 counts per energy channel was extracted from a
region centered on the X-ray peak and extending to \rf\ (defined in
Section \ref{sec:global}). A background spectrum was extracted from a
source-free region with an area equal to the source region. Instrument
responses were generated with the tools {\textsc{ARFGEN}} and
{\textsc{RMFGEN}}. The \xmm\ data is utilized in Section
\ref{sec:centsrc} to check our results for the nuclear source against
the analysis presented in \citet{2007A&A...473...85P}.

%%%%%%%%%%%%%%%%%%%%%%
\subsection{\bepposax}
\label{sec:beppo}
%%%%%%%%%%%%%%%%%%%%%%

Conclusions reached in previous studies regarding the nature of the
\irs\ nuclear absorber have relied on the \bepposax\ hard X-ray
detection discussed by \citet{2000A&A...353..910F}. Here, we repeat
and confirm that analysis in order to compare our results in Section
\ref{sec:centsrc} against \citet{2000A&A...353..910F}. We retrieved
and re-analyzed the \bepposax\ PDS X-ray data (15-220 keV) taken 1998
April 18 (ObsCode 50273002; PI Franceschini). The data was reduced and
analyzed with \saxdas\ version 2.3.1 using the calibration data and
cookbook available from
HEASARC\footnote{http://heasarc.nasa.gov/docs/sax/shp\_software.html}. PDS
data was accumulated, screened for good time intervals, and then a
light curve was extracted to look for spikes caused by fluorescence of
only one of the four PDS crystal scintillators. No spikes were
detected.

A PDS total spectrum was extracted from the on-axis data, and
background subtraction was performed using the variable rise time
threshold. The epoch appropriate response function used in fitting was
selected from the HEASARC database. Following the HEASARC cookbook,
LECS and MECS spectra were also generated for the \rf\ region with a
background spectrum taken from an annulus outside \rf. We measured a
PDS 15-80 keV count rate of $0.106 \pm 0.055$ ct \ps. Fitting the PDS
spectrum over the energy range 20-200 keV with an absorbed power-law
having fixed spectral index of $\Gamma = 1.7$ yields fluxes of $f_{10
  \dash 200} = 2.09^{+1.95}_{-1.95} \times 10^{-11} ~\flux$ and $f_{20
  \dash 100} = 1.10^{+1.57}_{-1.63} \times 10^{-11} ~\flux$. These
values are consistent with the results presented in
\citet{2000A&A...353..910F}.

%%%%%%%%%%%%%%%%%%%%%%%%%%%%%%%%%
\subsection{\integral\ \& \swift}
\label{sec:integral}
%%%%%%%%%%%%%%%%%%%%%%%%%%%%%%%%%

In the following section, we show that, \integral\ and \swift\ data do
not reveal any hard X-ray sources at the location of \irs, but the
upper limits are consistent with the \bepposax\ detection. Between the
beginning of 2005 and end of 2006, \irs\ was within the
\integral\ field of view (FOV) during 85 pointings. For 81 pointings,
data was collected with the IBIS Soft Gamma-Ray Imager (ISGRI;
$E_{\mathrm{sens}}$ = 15 keV-1 MeV), and for 79 pointings data was
collected with detector-1 of the Joint European X-ray Monitor (JEM-X;
$E_{\mathrm{sens}}$ = 3-35 keV). Datasets were reduced using
\osa\ version 8.0 and version 8.0.1 of the Instrument
Characteristics. For each instrument, mosaiced images of intensity,
significance, variance, and exposure were generated from the
background-subtracted images of each pointing. The combined ISGRI and
JEM-X effective exposure times are 200 ks and 210 ks, respectively.

Versions 1 and 30 of the \integral\ Reference Catalogue were used for
source detection. The \osa\ source detection routines did not locate
any $5\sigma$ sources in the ISGRI and JEM-X mosaiced
images. Additional visual inspection of the mosaiced ISGRI and JEM-X
images did not reveal any features which might suggest emission from a
source. The ISGRI and JEM-X instrument responses have a strong energy
dependence, thus, upper limits calculated using only the variance
images (\ie\ assuming uniform sensitivity) will systematically
underestimate the flux limit. To account for this variation, flux
upper limits were derived by integrating the ISGRI and JEM-X response
matrix functions (RMFs) over a specified energy range and weighting by
an assumed spectral shape.

We assumed the \irs\ $E> 10$ keV spectrum goes as $S_{\nu} =
\nu^{-1.7}$ with no high-energy cut-off. Between 10-35 keV and 20-100
keV, we derive $3\sigma$ upper limits of $f_{10 \dash 35} = 1.28
\times 10^{-12} ~\flux$ and $f_{20 \dash 100} = 5.70 \times 10^{-11}
~\flux$, respectively. The \integral\ $1\sigma$ 20-100 keV flux limit
is narrowly higher than the \bepposax\ 20-100 keV PDS measured
flux. The \integral\ 20-100 keV upper limit is also consistent with a
$z = 0.442$ source which would not be detected in the IBIS
Extragalactic AGN Survey \citep{2006ApJ...636L..65B}.

As a check of this result, the \swift-BAT archive was searched for
sources. No sources within $5\mydeg$ of \irs\ were detected in the 22
month \swift-BAT survey \citep{2010ApJS..186..378T}. The \swift-BAT
survey has a 14-195 keV $4.8\sigma$ detection limit of $2.2 \times
10^{-11} ~\flux$, which is 14\% higher than the 14-195 keV \irs\ flux
expected based on the \bepposax\ detection. Assuming the
\integral\ and \swift-BAT upper limits are representative of a
$1\mydeg$ region around \irs\ (\ie\ the FWHM PDS FOV), the lack of
detected hard X-ray sources near \irs\ suggests that the PDS detection
did not originate from a brighter off-axis source, assuming the source
is/was not transient or a one-off event.

%%%%%%%%%%%%%%%%
\subsection{VLA}
\label{sec:vla}
%%%%%%%%%%%%%%%%

Between the years 1986 and 2000, \irs\ was observed at multiple
frequencies with varying resolutions using the VLA radio
observatory. Continuum mode observations were taken from the VLA
archive and reduced using version 3.0.0 of the Common Astronomy
Software Applications (\casa). Flagging of bad data was performed
using a combination of \casa's {\textsc{flagdata}} tool in
{\textsc{rfi}} mode and manual inspection. Radio images were generated
by Fourier transforming, cleaning, self-calibrating, and restoring
individual radio observations. The additional steps of phase and
amplitude self-calibration were included to increase the dynamic range
and sensitivity of the radio maps. All sources within the primary beam
and first side-lobe detected with fluxes $> 5\sigma_{\mathrm{rms}}$
were imaged to further maximize the sensitivity of the radio maps.

Resolved radio emission associated with \irs\ is detected at 1.4 GHz,
5 GHz, and 8.4 GHz, while a $3\sigma$ upper limit of $0.84$ mJy is
established at 14.9 GHz. Fluxes for unresolved emission at 74 MHz, 151
MHz, and 325 MHz were retrieved from the VLA Low-Frequency Sky Survey
\citep[VLSS;][]{vlss}, 7C Survey \citep{1999MNRAS.306...31R}, and
Westerbork Northern Sky Survey \citep[WENSS;][]{1997A&AS..124..259R},
respectively. No formal detection is found in VLSS, however, an
overdensity of emission at the location of \irs\ is evident. For
completeness, we measured a flux for the potential source, but
excluded the value during fitting of the radio spectrum. The combined
1.4 GHz image reveals the most extended structure, and thus our
discussion regarding radio morphology is guided using this
frequency. An analysis of 1.4 and 5 GHz VLA data is also presented in
\citet[][hereafter H93]{1993ApJ...415...82H}.

The resolution of the combined 1.4 GHz dataset is $0.37\arcs$
pixel$^{-1}$ with an average beam shape of $1.26\arcs \times
1.15\arcs$ inclined at $-37.16\mydeg$ from north and an off-source rms
noise of $0.027$ mJy beam$^{-1}$. The deconvolved, integrated 1.4 GHz
flux of the continuous extended structure coincident with \irs, and
having $S_{\nu} \ga 3\sigma_{\mathrm{rms}}$, is $14.0 \pm 0.51$ mJy. A
significant spur of radio emission northeast of the nucleus is
detected with flux $0.21 \pm 0.07$ mJy. Radio contours were generated
beginning at 3 times the rms noise and moving up in 6 log-space steps
to the peak intensity of 4.7 mJy beam$^{-1}$. These are the contours
referenced in all following discussion of the radio source morphology
and its interaction with the X-ray gas.

To investigate properties of the radio source, we fitted the radio
spectrum between 151 MHz and 8.4 GHz for the full radio source (lobes,
jets, \& core) with the well-known KP \citep{1962SvA.....6..317K,
  pach}, JP \citep{1973A&A....26..423J}, and CI
\citep{1987MNRAS.225..335H} synchrotron models. The models vary by
assumption of pitch-angle distribution and number of electron
injections. The models were fitted to the radio spectrum using the
code of \citet{2005ApJ...624..656W}, which is based on the method of
\citet{1991ApJ...383..554C}. The JP model (single electron injection,
randomized but isotropic pitch-angle distribution) yields the best fit
with \chisq(DOF)$ = 0.491(3)$, a break frequency of $\nu_B = 12.9 \pm
1.0$ GHz, and a low-frequency ($\nu < 2$ GHz) spectral index of
$\alpha = -1.10 \pm 0.09$. The radio spectrum and best-fit models are
shown in Figure \ref{fig:radio}. The bolometric radio luminosity was
approximated by integrating under the JP curve between $\nu_1 = 10$
MHz and $\nu_2 = 10,000$ MHz, giving $\lrad = 1.09\times10^{42}~\lum$.

Assuming inverse-Compton (IC) scattering and synchrotron emission are
the dominant radiative mechanisms of the radio source, the time since
acceleration for an isotropic particle population is given by
\citet{2001AJ....122.1172S} as
\begin{equation}
  \tsync = 1590 \left(\frac{B^{1/2}}{B^2 + B_{\mathrm{CMB}}^2}\right)~
  \left[\nu_{\mathrm{B}} (1+z)\right]^{-1/2} ~\Myr
\end{equation}
where $B$ [\mg] is magnetic field strength, $B_{\mathrm{CMB}} =
3.25(1+z)^2$ [\mg] is a correction for IC losses to the cosmic
microwave background, $\nu_{\mathrm{B}}$ [GHz] is the radio spectrum
break, and $z$ is the dimensionless source redshift. We assume that
$B$ can be approximated by the equipartition magnetic field strength,
$B_{\mathrm{eq}}$, derived from the minimum energy density condition
\citep{1980ARA&A..18..165M},
\begin{equation}
  B_{\mathrm{eq}} = \left[\frac{6 \pi ~c_{12}(\alpha,\nu_1, \nu_2)
      ~\lrad ~(1+k)}{V \Phi}\right]^{2/7} ~\mathrm{\mg}
\end{equation}
where $c_{12}(\alpha,\nu_1,\nu_2)$ is a dimensionless constant derived
in \citet{pach}, \lrad\ [$\lum$] is the integrated radio luminosity
from $\nu_1$ to $\nu_2$, $k$ is the dimensionless ratio of lobe energy
in non-radiating particles to that in relativistic electrons, $V$
[\cc] is the radio source volume, and $\Phi$ is a dimensionless
radiating population volume filling factor. Synchrotron age curves as
a function of $k$ and $\Phi$ are given in Figure \ref{fig:radio}. The
various combinations of $k$ and $\Phi$ give $B_{\mathrm{eq}} \approx 4
\dash 56 ~\mg$ and associated synchrotron ages in the range $\approx 1
\dash 12$ Myr. Our synchrotron ages do not account for energy lost to
adiabatic expansion of the radio plasma and should be considered upper
limits \citep{1968ARA&A...6..321S}.

%%%%%%%%%%%%%%%%%%%%%%%%%%%%%%%
\section{Global ICM Properties}
\label{sec:global}
%%%%%%%%%%%%%%%%%%%%%%%%%%%%%%%

Our analysis begins at the cluster scale with the integrated
properties of the \rxj\ ICM hosting \irs. We define the mean cluster
temperature, \cltx, as the ICM temperature within a core-excised
aperture extending to $R_{\Delta_c}$, the radius at which the average
cluster density is $\Delta_c$ times the critical density for a
spatially flat Universe. We chose $\Delta_c = 500$ and used the
relations from \cite{2002A&A...389....1A} to calculate
$R_{\Delta_c}$. \irs\ has a luminous, cool core which is not
representative of \cltx, thus, the convention of
\citet{2007ApJ...668..772M} was followed and emission inside $0.15
~\rf$ was excised. Source spectra were extracted from the region $0.15
\dash 1.0~\rf$ and background spectra were extracted from reprocessed
\caldb\ blank-sky backgrounds (see Section \ref{sec:rad}). Because
\cltx\ and $R_{\Delta_c}$ are correlated in the adopted definitions,
they were recursively determined until three consecutive iterations
produced \cltx\ values which agreed within the 68\% confidence
intervals. We measure $\cltx = 7.54^{+1.76}_{-1.15}$ keV corresponding
to $\rf = 1.16^{+0.27}_{-0.19}~\Mpc$. Measurements for a variety of
$R_{\Delta_c}$ apertures are summarized in Table \ref{tab:specfits}.

The cluster gas and gravitational masses were derived using the
deprojected radial electron density and temperature profiles presented
in Section \ref{sec:rad}. Electron gas density, $\nelec$, was
converted to total gas density as $n_g = 1.92 \nelec \mu \mH$ where
\mH\ [g] is the mass of hydrogen. The gas density profile was fitted
with a $\beta$-model \citep{betamodel}, and the temperature profile
was fitted with the 3D-$T(r)$ model of \citet{2006ApJ...640..691V} to
ensure continuity and smoothness of the radial log-space derivatives
when solving the hydrostatic equilibrium equation. Total gas mass was
calculated by assuming spherical symmetry and integrating the best-fit
$\beta$-model out to \rt, giving $\mgas(r<\rt) = 7.99 ~(\pm 0.65)
\times 10^{13} ~\msol$. The gravitating mass was derived by solving
the hydrostatic equilibrium equation using the analytic density and
temperature profiles. We calculate $\mgrav(r<\rt) = 7.22 ~(\pm 1.44)
\times 10^{14} ~\msol$. The ratio of gas mass to gravitating mass is
$0.11 \pm 0.02$. The gas and gravitating mass errors were estimated
from 10,000 Monte Carlo realizations of the measured density and
temperature profiles and their associated uncertainties.

With the exception of the strange ULIRG/QSO BCG at its heart, \rxj\ is
a typical, relaxed massive cluster. None of the integrated X-ray
cluster properties suggest the system has undergone a recent major
merger or cluster-scale AGN outburst which may have dramatically
disrupted the cluster. The lack of a detected radio halo also suggests
no recent merger activity, previous powerful AGN outbursts, and
possibly no turbulent motions in the core
\citep{2008SSRv..134...93F}. In terms of the galaxy cluster
population, \rxj\ resides toward the high-end of the mass distribution
with a luminosity-temperature ratio and gas fraction consistent with
flux-limited and representative cluster samples \citep{hiflugcs2,
  2009A&A...498..361P}. Adjusted for differences in assumed cosmology,
within the uncertainties our measurements agree with prior
\irs\ studies \citep[\eg][]{2000MNRAS.315..269A}.

%%%%%%%%%%%%%%%%%%%%%%%%%%%%%%%
\section{Radial ICM Properties}
\label{sec:rad}
%%%%%%%%%%%%%%%%%%%%%%%%%%%%%%%

We now discuss the finer structure of the ICM pervading \rxj\ via
radial ICM properties. Consistent with the global analysis, the
nuclear source was excluded from all radial analysis (see comments for
Table \ref{tab:specfits}). Temperature and abundance profiles were
created using circular annuli centered on the cluster X-ray peak and
containing 2.5K and 5K source counts per annulus, respectively. A
deprojected temperature profile was generated using the
\textsc{deproj} method in \xspec. We use the projected profile in all
analysis as it does not significantly differ from the deprojected
profile. Spectra were grouped to 25 source counts per energy
channel. \caldb\ blank-sky backgrounds were reprocessed and
reprojected to match each observation, and then normalized for
variations of the hard-particle background using the ratio of
blank-sky and observation 9.5-12 keV count rates. Following the method
outlined in \citet{2005ApJ...628..655V}, a fixed background component
was included during spectral analysis to account for the
spatially-varying Galactic foreground \citep[see][for more
  detail]{xrayband}. The temperature and abundance profiles are shown
in the top row of Figure \ref{fig:gallery}.

A surface brightness profile was extracted using concentric $1\arcs$
wide circular annuli centered on the cluster X-ray peak. From the
surface brightness and temperature profiles, a deprojected electron
density profile was derived using the \citet{kriss83} technique
\citep[see][for more detail]{accept}. Errors for the density profile
were estimated from $10,000$ Monte Carlo bootstrap resamplings of the
original surface brightness profile. The surface brightness and
electron gas density profiles are shown in the second row of Figure
\ref{fig:gallery}.

Total gas pressure ($p = 2.4 \tx \nelec$), entropy ($K =
\tx\nelec^{-2/3}$), cooling time ($\tcool = 3n\tx~[2\nelec \nH
  \Lambda(T,Z)]^{-1}$), and enclosed X-ray luminosity profiles were
also created. These profiles are presented in the bottom two rows of
Figure \ref{fig:gallery}. Uncertainties for each profile were
calculated by propagating the individual parameter errors and then
summing in quadrature. The cooling functions, $\Lambda(T,Z)$, used to
calculate cooling times were derived from the best-fit spectral model
for each annulus of the temperature profile and interpolated onto the
grid of the higher resolution density profile. We also fitted the
entropy model $K(r) = \kna +\khun (r/100 ~\kpc)^{\alpha}$ to the
entropy profile and found best-fit values of $\kna = 12.6 \pm 2.9
~\ent$, $\khun = 139 \pm 8 ~\ent$, and $\alpha = 1.71 \pm 0.10$.
After masking out all substructure and the central $2\arcs$, a grouped
spectrum for the central 20 kpc was fitted with a thermal model plus
cooling flow component. The best-fit model had a mass deposition rate
of $\mdot = 206^{+87}_{-65} ~\msol$ for upper and lower temperatures
of 5.43 keV and 0.65 keV, respectively, with abundance $0.51 ~\Zsol$.

The \rxj\ ICM structure is typical of the cool core class of galaxy
clusters, with a temperature profile that rises with increasing radius
and an entropy profile with a relatively small, flattened core. There
are no resolved discontinuities in the temperature, density, or
pressure profiles to suggest the presence of a shock or cold front.
The entropy profile is consistent with the cool core population as a
whole \citep{accept}, and, in particular, with the population of $\kna
< 30 ~\ent$ clusters that have radio-loud AGN and star formation in
the BCG \citep{haradent}. The $\kna \la 30 ~\ent$ scale also defines
an entropy regime in which thermal electron conduction in cluster
cores inefficiently suppresses widespread environmental cooling
\citep{conduction}. Therefore, core sub-systems, such as parcels of
gas uplifted by an AGN, ram pressure stripped gas, or filamentary
thermal instabilities, should be long-lived.

Using the \rxj\ radial density structure, the ram pressure stripping
analysis presented in \citet{a1664}
%% Elaborate on this a little more - it's a little too cryptic. The
%% next sentence does not follow at all, so help us readers out a
%% little.
was performed to investigate how much of the BCG gas reservoir could
be deposited by the 6 compact spheroids ($r < 2.5$ kpc;
$M_{\mathrm{V}} > -17$ mag) within a projected 80 kpc of the BCG
\citep{1996AJ....111..649S, 1999Ap&SS.266..113A}. We estimated a
cluster velocity dispersion of $\sigma_{\mathrm{gal}} \approx 1109 \pm
259 ~\kmps$ using the relations from \citet{2000MNRAS.318..715X}. The
ICM density dictates that only $10^{11 \dash 12} ~\msol$ spirals will
be efficiently stripped if they are traveling at $> 700 ~\kmps$ inside
of 37 kpc. Our analysis indicates even if 6 gas-rich galaxies
($M_{\mathrm{gas}} = 2 \dash 5 \times 10^8 ~\msol$) were stripped
bare, they still contribute 5-10 times too little gas to explain the
reservoir in \irs.
%% Then the stripping analysis is kind of unnecessary -- the extreme
%% approximation (requiring zero knowledge of velocities and
%% densities) that ALL of the gas-rich galaxies' gasses were removed
%% still doesn't explain the reservoir. But why limit the BCG's source
%% of gas to those within 80 kpc?
If the spheroids are instead tidally-truncated compact ellipticals
similar to M32, it is odd that the mean projected separation among
them is $30 \pm 15$ kpc, 8-15 times the expected tidal radius of each
object.
%% I'm not sure I understand this argument. What kinds of timescales
%% are you assuming?

If the spheroids have interacted with each other, or the BCG,
to the point of tidal truncation, then to have such large separations
requires huge relative velocities. Yet, the BCG nebulae which may have
been stripped from the spheroids have small velocities relative to the
BCG \citep{1996MNRAS.283.1003C}. But the relaxation and dynamical
friction time scales in the core are $\sim 10^{8-9}$ yr, 10-100 times
the free-fall time, so if the gas is old enough to have slowed, then
it should have fallen into the BCG, unless the gas has high angular
momentum or is in a stable orbit.

If mergers are responsible for the accumulation of cold gas in the
core, it is odd that the radio source is very linear. Bulk motion and
turbulence induced by mergers would disrupt the radio plasma
\citep[\eg][]{2009A&A...495..721S, 2010arXiv1002.0395S}, indicating
that the core has been mostly undisturbed for $> 12$
Myr. Alternatively, mergers could have been very gentle so as not to
stir the core gas, but then how was gas stripped from the galaxies?
We thus conclude that the gas reservoir likely did not result from
mergers, but most likely from a cooling flow.

%% The counter argument to this is: how did the gas get dusty then? If
%% the gas is ex-cooling flow gas, it was recently at X-ray emitting
%% temperatures, and such gas has a very short sputtering time. So if the
%% gas came from a hot ICM, the dust must have come from somewhere
%% else...

%%%%%%%%%%%%%%%%%%%%%%%%%%%
\section{ICM Cavity System}
\label{sec:cavs}
%%%%%%%%%%%%%%%%%%%%%%%%%%%

To aid investigation of ICM substructure, a residual X-ray image was
created by subtracting a surface brightness model from the mosaiced
\chandra\ clean image. The \chandra\ clean image was binned by a
factor of 2 and the surface brightness isophotes fitted with
ellipses. The geometric parameters ellipticity ($\epsilon$), position
angle ($\phi$), and centroid $(X_0,Y_0)$ were initially free to vary,
but the best-fit values for each isophote converged to mean values of
$\epsilon = 0.52 \pm 0.02$, $\phi = 72.5\mydeg \pm 1.7\mydeg,$ and
$(X_0,Y_0) = \ellradec$ (J2000). These values were then fixed in the
fitting routine to eliminate isophotal twisting which results from
statistical variation of the best-fit values for each radial step.

A surface brightness model normalized to the parent image was
constructed from the best-fit isophote ellipses and subtracted from
the parent image. The resulting residual image is shown in Figure
\ref{fig:resid}. The faint surface brightness decrements NE and SW of
the nucleus in the parent image are resolved into cylindrical voids in
the residual image. The void and radio jet morphologies closely trace
each other, confirming they share a common origin in the AGN
outburst. Based on a 1994 \rosat\ HRI observation,
\citet{1995MNRAS.274L..63F} suggested the presence of a ``hole'' in
the core of \rxj\ which was attributed to absorption by a $\mdot >
1000 ~\msolpy$ cooling flow. Using the new \chandra\ observation as a
guide, we find the hole is wedged between the nuclear source and the
SE cavity. The true cavities are not resolved in the 1994 HRI image,
and a longer HRI observation from 1995 did not resolve the
cavities. Cavities are a well-known feature of groups and clusters
\citep[\eg][]{birzan04, 2005MNRAS.364.1343D}, but currently, \irs\ is
the highest redshift object where cavities have been directly
imaged. In addition, \irs\ is thus far the only example of a QSO
system with an unambiguous cavity detection.

The AGN outburst energetics were investigated by calculating the time
averaged cavity energy, \pcav\ \citep[see][for a
  review]{mcnamrev}. Note that \pcav\ does not account for energy
which may be in unresolved shocks, and that \pcav\ is assumed to be a
good estimate of the physical quantity jet power, \pjet. For the
following calculations the cavities were treated as symmetric
cylinders. Properties of the individual cavities are listed in Table
\ref{tab:cylcavities}.

%% Need to show how pV->E->pcav

The $4pV$ work to create the cavities was calculated by integrating
the high-resolution pressure profile over the surface of each
cylinder. The radio source morphology, spectrum, and age suggest the
jets were recently, or still are, being fed by the central
source. Thus, we assumed the cavities are being excavated on a
timescale dictated by the ambient gas sound speed,
$t_{\mathrm{sonic}}$ \citep[see][]{birzan04}, and are not buoyant
structures. We set the distance of the cavity as its leading edge when
calculating $t_{\mathrm{sonic}}$, not the cavity mid-point, as is
common in many other studies. A systematic uncertainty of 10\% was
included to account for error in the cavity geometric parameters and
from assuming $\Delta (kT_X) = 0$ over the length of the cavities. We
calculate a total cavity power of $\pcav = 3.41 (\pm 0.82) \times
10^{44} ~\lum$. Radio luminosity has been shown to also be a
reasonable surrogate for measuring total jet power
\citep{birzan08}. Using the \citet{pjet} \pjet-\prad\ scaling
relations, the \irs\ 1.4 GHz \& 325 MHz powers correspond to $\pjet
\sim 2 \dash 6 \times 10^{44} ~\lum$, in agreement with the X-ray
determined value.

The lower limit for $t_{\mathrm{sonic}}$ is 42 Myr, but the radio
source is less than 12 Myr old. For the radio plasma to still be
synchrotron-loud throughout the jet requires constant injection from
the AGN, which is ruled out by the radio spectral properties, or that
the AGN outflow is supersonic. Compared to the ICM sound speed, the
velocity required to reach the end of the radio jet in 12 Myrs
suggests $M \sim 4$. If the outflow is supersonic, then local gas
shocking may have occurred. However, the X-ray analysis and nebular
properties indicate no shocks are present \citep{1996MNRAS.283.1003C,
  2000AJ....120..562T}, so the estimates presented next are poorly
constrained. We estimated the outburst energy including shocks to be
$\approx 10^{46} ~\lum$ by setting $t_{\mathrm{sonic}} = 12$ Myr and
multiplying up by a factor of $M^3$ since $\Delta \pcav = \Delta p /
\Delta t$ and $\Delta p \propto M^3$.

Within the formal uncertainties, the AGN ourburst energy is on the
order of a few times $10^{44} ~\lum$ with a maximal value of $10^{46}
~\lum$. Compared with other systems hosting cavities, \irs\ resides
between the middle and upper-end of the cavity power distribution. The
AGN outburst is powerful, but there is nothing unusual about the
energetics given the cluster mass and ICM properties.

Of interest is how the AGN energetics compare to the cooling rate of
the host X-ray halo. The cooling radius was set at the radius where
the ICM cooling time is equal to $H_0^{-1}$ at the redshift of
\irs. We calculate $R_{\mathrm{cool}} = 128$ kpc, and measure an
unabsorbed bolometric luminosity within this radius of
$L_{\mathrm{cool}} = 1.61^{+0.25}_{-0.20} \times 10^{45} ~\lum$. If
all of the cavity energy is thermalized over $4\pi$ sr, then $\approx
20\%$ of the energy radiated away by gas within $R_{\mathrm{cool}}$ is
offset by the ongoing AGN outburst. This highly optimistic scenario
implies that five such AGN outbursts will significantly suppress
cooling of the gaseous halo.

%%%%%%%%%%%%%%%%%%%%%%%%%%%%%%%%%%
\section{Fueling the AGN Outburst}
\label{sec:fuel}
%%%%%%%%%%%%%%%%%%%%%%%%%%%%%%%%%%

An estimate of black hole mass, \mbh, is key to investigating what
powers an AGN outburst. There are a variety of \mbh\ estimators, many
of which rely on near-IR, far-IR, and optical emission line
diagnostics \citep{marconihunt03, 2007MNRAS.379..711G,
  2008ApJ...673..703M}. However, the IR and emission line properties
of \irs\ may not be representative of the bulge stellar component, but
hot dust and complex nebulae. Therefore, we calculated \mbh\ using a
variety of relations and took their weighted mean. Using the corrected
$B$-band magnitude from HyperLeda \citep{hyperleda}, we calculated a
BCG stellar velocity dispersion of $\sigma_s = 293 \pm 6 ~\kmps$ from
the \citet{1976ApJ...204..668F} relation. Inserting this value into
the \citet{2002ApJ...574..740T} relation gives $\mbh = 0.63 (\pm 0.05)
~\mnine$ where $\mnine = 10^9 ~\msol$. Using the
\citet{2007MNRAS.379..711G} relations which relate the absolute
$[B,R,K]$-band magnitudes to \mbh, we find $\mbh = 0.64 \dash 4.1
~\mnine$. The final weighted mean value is $\mbh = 1.05 \pm 0.17
~\mnine$.

The gravitational binding energy of the material accreting onto the
SMBH is transported outward via relativistic jets. Assuming this
conversion has some efficiency, $\epsilon$, then the energy deposited
in cavities by the jets implies an accretion mass expressed as $\macc
= \hcav/(\epsilon c^2)$ with a time-averaged mass accretion rate of
$\dmacc = \macc/t_{\mathrm{sonic}}$. Assuming $\epsilon = 0.1$, the
AGN outburst resulted from the accretion of $3.19 ~(\pm 0.35) \times
10^{6} ~\msol$ at a rate of $0.06 \pm 0.01 ~\msolpy$. The mass-energy
going into the SMBH and not the jets is expressed as $\Delta \mbh =
(1-\epsilon) \macc$ with a time-averaged rate of $\dmbh = \Delta
\mbh/t_{\mathrm{sonic}}$. The black hole mass thus grew by $2.87 ~(\pm
0.31) \times 10^{6} ~\msol$ at a rate of $0.05 \pm 0.01 ~\msolpy$.

Assuming spherical symmetry, the accretion flow feeding the SMBH can
be characterized in terms of the Eddington (Eqn. \ref{eqn:edd}) and
Bondi (Eqn. \ref{eqn:bon}) accretion rates,
\begin{eqnarray}
  \dmedd &=& \frac{2.2}{\epsilon} \left(\frac{\mbh}{10^9~\msol}\right)
  ~\msolpy  \label{eqn:edd}\\
  \dmbon &=& 0.013 ~K_{\mathrm{Bon}}^{-3/2} \left(\frac{\mbh}{10^9
    ~\msol}\right)^{2} ~\msolpy \label{eqn:bon}
\end{eqnarray}
where $K_{\mathrm{Bon}}$ [\ent] is the mean entropy of gas within the
Bondi radius. The Eddington rate defines the maximal inflow rate of
gas not expelled by radiation pressure, as where the Bondi rate
approximates the quantity of hot ambient gas captured by the SMBH,
\ie\ hot-mode accretion. Assuming $\epsilon = 0.1$ and
$K_{\mathrm{Bon}} = \kna$, the derived \mbh\ gives $\dmedd \approx 23
~\msolpy$ and $\dmbon \approx 3.2 \times 10^{-4} ~\msolpy$. Thus, the
Eddington and Bondi ratios for the accretion event which powered the
AGN outburst are $\dmacc/\dmedd \approx 0.003$ and $\dmacc/\dmbon
\approx 300$.

The Bondi radius for \irs\ is unresolved ($\rbon = 9$ pc), and
$K_{\mathrm{Bon}}$ is likely less than \kna. For a Bondi ratio of at
least unity, $K_{\mathrm{Bon}}$ must be $\le 0.34 ~\ent$, lower than
is measured for even galactic coronae \citep{coronae}. In terms of
entropy, $\tcool \propto K^{3/2} ~\tx^{-1}$ \citep{d06}. Assuming gas
close to \rbon\ is no cooler than 0.5 keV, the accreting material will
have $\tcool < 6$ Myr, a factor of 60 below the shortest ICM cooling
time and 1/4 the free-fall time in the core. But this creates the
problem that gas close to \rbon\ is disconnected from cooling at
larger radii, breaking the feedback loop
\citep{2006NewA...12...38S}. If instead cold-mode accretion dominates,
then the gas which becomes fuel for the AGN is distributed in the BCG
halo, $r \sim 1 \dash 30$ kpc, and falls into the BCG as a result of
cooling \citep{pizzolato05, 2010arXiv1003.4181P}. Indeed, \hst\ WFPC2
images reveal radial filaments and gaseous substructure down to the
resolution-limit within 30 kpc of \irs\ \citep{1999Ap&SS.266..113A}.
This may indicate cooling, overdense regions similar to the cold blobs
expected in cold-mode accretion. The process of cold-mode accretion is
more consistent with the nature of \irs, and though Bondi accretion
cannot be ruled out, it does not seem viable, and may not be feasible
in general for cool core BCGs \citep{minaspin}.

Unlike most clusters with cavities where AGN mechanical feedback
energetically dominates, \irs\ is radiatively-dominated by a heavily
obscured QSO with a bolometric luminosity of $\lqso \approx 1 \times
10^{47} ~\lum$ (see Section \ref{sec:centsrc}). Assuming the mass
accretion powering the QSO goes as $\dmacc = \lqso/(\epsilon c^2)$
with $\epsilon = 0.1$, then $\dmacc \approx 19 ~\msolpy$. Hence, the
Eddington ratio for the accretion powering the QSO is $\dmacc/\dmedd
\approx 0.80$. \citet[][hereafter F09]{2009MNRAS.394L..89F}
demonstrated that AGN radiation pressure has a significant influence
on the mean density of dusty material in the host galaxy. In the
formalism of F09, the \irs\ AGN has an effective Eddington ratio of
$\leff = \lqso (1.38 \times 10^{38} ~\mbh)^{-1} \approx 0.72$. In the
\nhi-\leff\ plane presented by F09, \irs\ resides very near the
boundary between the regions where AGN obscuring clouds are long-lived
and where dusty clouds experience the effects of a super-Eddington
AGN, \ie\ where clouds are efficiently expelled. We point out that
these estimates are at the mercy of our choice for \mbh. If \mbh\ is
$\ge 5 ~\mnine$, the Eddington ratios will be $< 0.2$, as where $\mbh
< 1 ~\mnine$ implies Eddington ratios $> 1$.

%%%%%%%%%%%%%%%%%%%%%%%%%%%%%%%%%%%%%%%%%%%%%
\section{ICM Substructure \& QSO Irradiation}
\label{sec:excess}
%%%%%%%%%%%%%%%%%%%%%%%%%%%%%%%%%%%%%%%%%%%%%

In Figure \ref{fig:resid}, three regions of X-ray emission in excess
of the best-fit surface brightness model are highlighted. The regions
are illustrative and approximate the constant surface brightness
contours used to define the spectral extraction regions. Each region
is denoted by its location relative to the nucleus: northern excess
(NEx), eastern excess (EEx), and western excess (WEx). The NEx and WEx
appear to be part of a tenuous, arc-like filament which may be gas
displaced by the NW radio jet. Similar features are found around other
cavity systems \citep[\eg][]{2009ApJ...697L..95B}, but the data is
insufficient to determine if this is the case for \irs, hence we treat
them as separate structures.

Spectral analysis was performed on each region. A background spectrum
was extracted from regions neighboring the excesses which did not show
enhanced emission in the residual image. The backgrounds were scaled
to correct for differences in sky area. For each region, the ungrouped
source and background spectra were differenced in \xspec\ to create a
residual spectrum. To avoid systematically cooler best-fit
temperatures resulting from low count rates
\citep{1989ApJ...342.1207N}, the modified Cash statistic
\citep{1979ApJ...228..939C} was used during fitting. The low
signal-to-noise ratio (SN) of each spectrum precluded setting metal
abundance as a free parameter when fitting a thermal model. The three
excesses reside within the two central annuli of the abundance
profile, thus abundance was fixed at $0.51 ~\Zsol$. Varying the fixed
abundance by $\pm 0.2 ~\Zsol$ changed the output temperatures and
normalizations within the statistical uncertainties when $0.51 ~\Zsol$
was assumed. A fixed thermal component scaled to sky area representing
the ICM emission for the coincident annulus was included in the
fitting.

The NEx residual spectrum had low-SN which resulted in poor resolution
of spectral features. The NEx thermal model had an unconstrained
temperature of $\sim 7$ keV, similarly the power-law model had an
unconstrained spectral index of $\Gamma \sim 1.9$. The northern radio
jet terminates in the NEx region, and the hardness ratio map (see
Section \ref{sec:centsrc} and Figure \ref{fig:resid}) shows a possible
hot spot in this same area. The NEx may result from non-thermal
emission in the hot spot, but we cannot confirm this
spectroscopically.

The EEx and WEx residual spectra have characteristics consistent with
thermal emission. The best-fit values for the thermal models are given
in Table \ref{tab:excess}. The EEx spectrum has prominent features at
$E < 2$ keV which were poorly fit by a single-component thermal
model. The thermal \feka\ complex was also poorly fit because of an
obvious asymmetry toward lower energies. To reconcile the poor fit,
three Gaussians were included in the fitting. Comparison of fit
statistics and goodness of fits determined from 10,000 Monte Carlo
simulations of the best-fit spectra suggest the model with the
Gaussians is preferred. The strength of the features relative to the
continuum suggest a thermal origin is unlikely, and that the features
may be emission line blends.

\citet[][hereafter H99]{1999ApJ...512..145H} suggested the AGN which
produced the large-scale radio jets has been reoriented within the
last few Myrs, resulting in a new beaming direction close to the line
of sight and at roughly a right angle to the previous beaming
axis. The new AGN axis suggested by H99 is coincident with the EEx,
the $3\sigma$ radio spur northeast of the radio core, a UV ionization
cone (H93), an ionized optical nebula \citep{1996MNRAS.283.1003C,
  1999Ap&SS.266..113A}, and highly polarized diffuse optical emission
\citep{2000AJ....120..562T}. A schematic of these respective features
is shown in Figure \ref{fig:resid}. \citet{2010MNRAS.402.1561R} show
that the QSO in H1821+643, which is a factor of 10 more powerful than
\irs, is capable of photoionizing gas up to 30 kpc from the nucleus,
and we suspected a similar process may be occurring in \irs.

To test this hypothesis, reflection and diffuse spectra were simulated
for the nebula and ICM coincident with the EEx using
\cloudy\ \citep{cloudy}. The nebular gas density and ionization state
were taken from \citet{2000AJ....120..562T}, while the initial ICM
temperature, density, and abundance were set at 3 keV, 0.04 \pcc, \&
0.51 \Zsol, respectively. No Ca or Fe lines are detected from the
nebula coincident with the EEx, but strong Mg, Ne, and O lines are
\citep{2000AJ....120..562T}, possibly as a result of metal depletion
onto dust grains \citep[\eg][]{1993ApJ...414L..17D}. Thus, a metal
depleted, grain-rich, 12 kpc thick nebular slab was placed 15 kpc from
an attenuated $\Gamma = 1.7$ power law source with power $1 \times
10^{47} ~\lum$. Likewise, a $17 ~\kpc \times 16 ~\kpc$ ICM slab was
placed 19 kpc from the same source. The QSO radiation was attenuated
using a 15 kpc column of density 0.06 \pcc, abundance 0.51 \Zsol, and
temperature 3 keV. The output models were summed, folded through the
\chandra\ responses using \xspec, and fitted to the observed EEx
spectrum (shown in Figure \ref{fig:qso}).

In the energy range 0.1-10.0 keV, the nebula emission lines which
exceed the thermal line emission originate from Si, Cl, O, F, K, Ne,
Co, Na, \& Fe and occur as blends around redshifted 0.4, 0.6, 0.9, \&
1.6 keV. The energies and strengths of these blends are in good
agreement with the EEx spectrum. Further, the \feka\ emission from the
nebula is 100 times fainter than that from the ICM, and the observed
asymmetry of the EEx \feka\ emission results from the 6.4 keV
\feka\ photoionized line of the ICM. We conclude that the QSO is
responsible for the nature of the EEx, but higher resolution radio
observations are necessary to discern between beamed AGN radiation and
QSO radiation escaping the obscured nucleus.

An alternative explanation for the EEx is that parcels of low entropy
gas have been uplifted from deeper within the core along the new AGN
beaming axis. Scattered UV emission 32 kpc from the core places a
minimum lifetime of 73 kyr for the new beaming direction (H93), and,
assuming saturated heat flux across the EEx surface, the evaporation
time is exceedingly short, $< 1$ Myr. These short timescales suggest
the EEx was transported to the present location at $> 20$ times the
ambient sound speed, or $v \sim 0.06 \dash 0.1c$, well below typical
jet bulk flow velocities. The radio spur has a 1.4 GHz luminosity of
$\approx 3 \times 10^{39} ~\lum$, suggesting an associated 1 Myr old
jet would have $\sim 10^{57}$ erg of kinetic energy, which is
sufficient to lift $\sim 10^{10} ~\msol$ to a distance of 19
kpc. Uplift is feasible, particularly if the uplifted gas is
magnetically-shielded, which will stave-off conduction.

%%%%%%%%%%%%%%%%%%%%%%%%%%%%%%%%
\section{Nucleus X-ray Emission}
\label{sec:centsrc}
%%%%%%%%%%%%%%%%%%%%%%%%%%%%%%%%

Coordinates for the nuclear source were determined using the
\ciao\ tool {\tt wavdetect} and confirmed with a hardness ratio map
calculated as $HR = f(2.0 \dash 9.0 ~\keV) / f(0.5 \dash 2.0 ~\keV)$,
where $f$ is the flux in the denoted energy band. The $HR$ map and
spectral extraction regions are shown in Figure \ref{fig:nucleus}. A
source extraction region was defined using the 90\% enclosed energy
fraction (EEF) of the normalized \chandra\ PSF specific to the nuclear
source median photon energy and off-axis position. The elliptical
source region had an effective radius of $1.16\arcs$. A segmented
elliptical annulus with the same central coordinates, ellipticity, and
position angle as the source region, but having 5 times the area, was
used for the background region. The background annulus was broken into
segments to avoid the regions of excess X-ray emission.

Source and background spectra were created using the \ciao\ tool {\tt
  psextract}. The source spectrum was grouped to have 20 counts per
energy channel. Presented in Figure \ref{fig:nucleus} are the 1999 and
2009 background-subtracted \chandra\ spectra and the best-fit
models. The spectra have been corrected for soft Galactic emission and
ICM emission. The significant flux difference below 1.3 keV is a
result of the greater effective area of the ACIS-S3 CCD in 1999 versus
ACIS-I3 in 2009. Approximately 72\% of the 2009 spectrum (hereafter,
SP09) is from the source, with a count rate of $1.63 ~(\pm 0.06)
\times 10^{-2}$ ct \ps\ in the 0.5-9.0 keV band. For the 1999 spectrum
(hereafter, SP99), 67\% is source flux, with a count rate of $2.71
~(\pm 0.26) \times 10^{-2}$ ct \ps\ in the 0.5-9.0 keV band.

Previous studies have shown the nuclear spectrum is best modeled as
Compton reflection from cold matter with a $\sim 1$ keV equivalent
width (EW) \feka\ fluorescence line at rest-frame 6.4 keV
\citep{2000AJ....120..562T, 2001MNRAS.321L..15I}. Illumination of the
AGN-facing side of a metal-rich circumnuclear torus is the favored
explanation for the origin of \feka\ fluorescence
\citep{1994ApJ...420L..57K}. We confirm the conclusion of
\citet{2001MNRAS.321L..15I} that the contribution from thermal ICM
\feka\ emission to the line feature around 4.4 keV is negligible. The
SP09 also reveals prominent features around 0.8 keV and 1.3 keV
superposed on the reflection continuum. The relative strength and
location of these features suggest they are emission line blends.

The SP99 and SP09 were fitted separately in \xspec\ over the energy
range 0.5-7.0 keV with an absorbed \pexrav\ model \citep{pexrav} plus
three Gaussians. The disk-reflection geometry employed in the
\pexrav\ model is not ideal for fitting reflection from a
Compton-thick torus \citep{2009MNRAS.397.1549M}, but no other suitable
\xspec\ model is currently available. Hence, only the
\pexrav\ reflection component was fitted and no high energy cut-off
for the power law was used. Fitting separate SP99 and SP09 models
allowed for source variation in the decade between observations,
however $\Gamma$ was poorly constrained for SP99 and thus fixed at the
SP09 value. Using constraints from \citet{2000AJ....120..562T}, the
model parameters for reflector abundance and source inclination were
fixed at $1.0 ~\Zsol$ and $i = 50\mydeg$, respectively. Setting
abundance as a free parameter did not statistically improve the
fits. The best-fit model parameters are presented in Table
\ref{tab:nucspec}. The unabsorbed 2-10 keV {\it{reflected}} flux is
$4.24^{+0.57}_{-0.55} \times 10^{-13} ~\flux$ corresponding to a
rest-frame $L_{2-10} = 1.57^{+0.19}_{-0.19} \times 10^{44}
~\lum$. Adjusted for cosmology, this agrees with the measurement from
\citet{2001MNRAS.321L..15I}. Since we have used a pure reflection
model, the intrinsic QSO luminosity can only be estimated as
$(\kappa/\eta) L_{2-10}$ where $\kappa = 40$ is a bolometric
correction factor \citep{2007MNRAS.381.1235V} and $\eta = 0.06$ is the
reflector albedo \citep{2009MNRAS.397.1549M}. This gives $\lqso = 1.05
(\pm 0.13) \times 10^{47} ~\lum$.

The \fekaew, is a valuable diagnostic for probing the environment of
an AGN \citep[see][for a review]{2000PASP..112.1145F}. Our best-fit
\fekaew\ values agree with previous observations and models which show
that $\fekaew \ga 0.5$ keV is correlated with $\Gamma \ga 1.7$ and
reflecting column densities $\nhref \sim 10^{24} ~\pcmsq$
\citep{1996MNRAS.280..823M, 1997ApJ...477..602N, 1999MNRAS.303L..11Z,
  2005A&A...444..119G}. Our best-fit values for the
\irs\ \fekaew\ also agree with previous measurements, which varied
between $0.2 \dash 2$ keV with $\approx 0.5$ keV preferred
\citep{2000A&A...353..910F, 2001MNRAS.321L..15I,
  2007A&A...473...85P}. The large uncertainties and inhomogeneous
spectral analyses of literature values prevents us from determining if
\fekaew\ has varied since 1998. However, jointly fitting the
integrated spectrum within \rf\ with our SP09 and ICM models using the
\chandra, \xmm, and \bepposax\ data yielded satisfactory fits with
\fekaew\ $< 0.9$ keV for all datasets.

Fitting the SP09 with a solar abundance thermal component in place of
the two low-energy Gaussians yielded a statistically worse fit. The
model systematically underestimated the 1-1.5 keV flux and
overestimated the 2-4 keV flux. Leaving the thermal component
abundance as a free parameter resulted in $0.1 ~\Zsol$, \ie\ the
thermal component tended toward a featureless skewed-Gaussian. Strong
Mg, Ne, S, and Si K$\alpha$ fluorescence lines at $E < 3.0$ keV can be
present in reflection spectra \citep{1991MNRAS.249..352G}, as can Fe
L-shell lines from photoionized gas \citep{1990ApJ...362...90B}. We
conclude that the soft X-ray emission modeled using the Gaussians is
likely a combination of emission line blends and low-level thermal
continuum, whether the thermal component is nuclear or ambient in
origin is unclear.

%% You might say more explicitly here that you not only check for
%% acceptable agreement with the Beppo Sax spectrum, but you include
%% it in a joint fit - that's not clear from the text here.

Previous studies suggested the \bepposax\ PDS detection resulted
primarily from transmission of hard X-rays through an obscuring screen
with $\nhobs > 10^{24} ~\pcmsq$. Extrapolating our SP09 model out to
10-80 keV reveals statistically acceptable agreement with the PDS data
(see Figure \ref{fig:resid}). The 10-200 keV flux for the SP09 model
is $f_{10 \dash 200} = 8.15^{+0.21}_{-0.19} \times 10^{-12} ~\flux$,
which is not significantly different from $f_{10 \dash 200}$ measured
with \bepposax. Adding a power-law component with a $\nhobs = 3 \times
10^{24} ~\pcmsq$ absorber at the \irs\ redshift to the SP09 model
lowers \chisq\ but with no statistical improvement to the fit. If
transmitted hard X-ray emission fell within the passband used for
analysis of the reflection spectrum, then lower values of $\Gamma$
would result, and the extrapolated hard X-ray flux would
increase. However, for $\Gamma \ge 1.7$, column densities $> 3 \times
10^{24} ~\pcmsq$ are sufficient to suppress significant transmitted
emission below our 7 keV analysis cut-off, indicating the SP09 model
should not have an artificially low $\Gamma$. That we find no need for
an additional hard X-ray component does not contradict the conclusion
that \irs\ harbors a Compton-thick AGN. The measured \fekaew\ suggests
reflecting column densities of $\nhref \sim 1 \dash 5 \times 10^{24}
~\pcmsq$ \citep{1993MNRAS.263..314L, 2005A&A...444..119G}, assuming
the nuclear material is mostly homogeneous, \ie\ $\nhref \approx
\nhobs$, our results are consistent with the presence of a heavily
obscured AGN.

%%%%%%%%%%%%%%%%%
\section{Summary}
\label{sec:summ}
%%%%%%%%%%%%%%%%%

We have presented analysis of the ULIRG \iras\ and the host galaxy
cluster \rxj\ using a new 75 ks \chandra\ X-ray observation. The
results presented in this paper are as follows:
\begin{itemize}
\item The \rxj\ ICM global and radial properties reveal no signs of
  shocks, cold fronts, deviations from hydrostatic equilibrium, or the
  like, to suggest disruption by a major merger or cluster-scale AGN
  outburst. \rxj\ is an unremarkable, massive, cool core object with a
  12 \ent\ core entropy, $M \propto T^{3/2}$, and mean gas fraction of
  0.11.
\item Using the \rxj\ gas density profile, we have determined that the
  measured BCG gas mass is at odds with the amount of gas which can be
  efficiently removed from cluster members via the process of ram
  pressure stripping. We also find that tidal stripping is an unlikely
  explanation for the nature of spheroids found in the cluster core,
  further suggesting that subcluster mergers are an unlikely source
  for the gas reservoir in \irs. The highly linear radio source
  morphology further indicates that merger activity has ceased for at
  least the last 12 Myr.
\item We have discovered cavities in the X-ray halo of \irs\ which
  indicate an AGN outburst with mechanical energy of at least $3.41
  \times 10^{44} ~\lum$, a hundredth of the QSO radiative
  energy. Comparison of cavity sound speed age and radio source age
  indicate the outflow is supersonic with $M = 2.3$, suggesting the
  total kinetic energy may be $\sim 10^{46} ~\lum$.
\item We have determined that cold-mode accretion is a preferable
  explanation for the nature of \irs. We also show that the QSO has a
  high effective Eddington luminosity of $\approx 0.9$, indicating
  radiation pressure should be expelling dusty clouds from \irs.
\item We have resolved substructure in the X-ray halo coincident with
  the [O \Rmnum{3}] nebula NE of the nucleus. The X-ray emission
  properties of this region are well-fit by a model where the nebula
  and ICM are irradiated by the QSO.
\item The nuclear X-ray source is well-fit by a reflection model
  resulting from a $\Gamma=1.7$ power law source. The
  \fekaew\ indicates a reflecting column density, and presumably
  obscuring column density, of $> 10^{24} \pcmsq$. We also show that
  the \bepposax\ detected hard X-ray emission results from reflected
  nuclear emission.
\end{itemize}

\irs\ simultaneously shows the characteristics of a system in quasar-
and radio-mode. The photoionized region NE of the nucleus, high \leff,
low gas-to-dust ratio, and strong nuclear optical outflow all suggest
the central 30 kpc of \rxj\ is dominated by radiative feedback from
the QSO in \irs. At the same time, the X-ray cavities and radio
outflow show an AGN actively suppressing cooling of the ICM. \irs\ may
be a local example of how massive galaxies at higher redshifts evolve
from quasar-mode into radio-mode. 

%%%%%%%%%%%%%%%%%%%%%%%%%%%
\section*{Acknowledgements}
%%%%%%%%%%%%%%%%%%%%%%%%%%%

KWC \& MD were supported by SAO grant GO9-0143X, and MD acknowledges
support through NASA LTSA grant NASA NNG-05GD82G. KWC \& BRM thank the
Natural Sciences and Engineering Research Council of Canada for
support. KWC thanks Alastair Edge \& Niayesh Afshordi for helpful
insight, and Guillaume Belanger \& Roland Walter for advice regarding
\integral\ data analysis. The \cxo\ Center is operated by the
Smithsonian Astrophysical Observatory for and on behalf of NASA under
contract NAS8-03060. The VLA (Very Large Array) is a facility of the
National Radio Astronomy Observatory (NRAO), which is a facility of
the National Science Foundation operated under cooperative agreement
by Associated Universities, Inc. This research has made use of: data
obtained from the Chandra Data Archive, the Chandra Source Catalog,
software provided by the Chandra X-ray Center (CXC), the NASA/IPAC
Extragalactic Database, and the NASA Astrophysics Data System.

%%%%%%%%%%%%%%%%
% Bibliography %
%%%%%%%%%%%%%%%%

\bibliographystyle{mn2e}
\bibliography{cavagnolo}

%%%%%%%%%%%%%%%%%%%%%%%
% Figures  and Tables %
%%%%%%%%%%%%%%%%%%%%%%%

\clearpage
\onecolumn
\begin{table*}
\caption{\sc Summary of Global ICM Spectral Fits. \label{tab:specfits}}
\begin{tabular}{lcccccccccc}
\hline
\hline
Region & $R_{\mathrm{in}}$ & $R_{\mathrm{out}}$  & \tx & \lbol & $Z$ & \redchisq & D.O.F. & \% Source & $\eta$ & Ct. Rate\\
- & kpc & kpc & keV & $10^{44}$ erg s$^{-1}$ & $Z_{\sun}$ & - & - & - & $10^{-4}$ cm$^{-5}$ & ct s$^{-1}$\\
(1) & (2) & (3) & (4) & (5) & (6) & (7) & (8) & (9) & (10) & (11)\\
\hline
$R_{500-\mathrm{Core}}$ & 174 & 1160 & 7.54$^{+1.76}_{-1.15}$  & 6.90$^{+0.61}_{-0.59}$  & 0.38$^{+0.31}_{-0.17}$  & 1.01 & 277 &  27 & 8.24   $^{+6\%}_{-6\%}$  & 0.063\\
$R_{1000-\mathrm{Core}}$ & 174 & 820 & 6.80$^{+1.14}_{-0.88}$  & 6.17$^{+0.41}_{-0.57}$  & 0.38$\dagger$ & 1.05 & 219 &  38 & 7.90   $^{+3\%}_{-3\%}$  & 0.058\\
$R_{2500-\mathrm{Core}}$ & 174 & 519 & 7.18$^{+1.25}_{-0.93}$  & 5.18$^{+0.41}_{-0.38}$  & 0.38$\dagger$ & 1.06 & 150 &  56 & 6.48   $^{+3\%}_{-3\%}$  & 0.048\\
$R_{500}$ & 13 & 1160 & 5.61$^{+0.32}_{-0.30}$  & 24.8$^{+2.9}_{-2.5}$  & 0.43$^{+0.09}_{-0.08}$  & 0.78 & 357 &  54 & 34.1  $^{+2\%}_{-2\%}$  & 0.237\\
$R_{1000}$ & 13 & 820 & 5.49$^{+0.28}_{-0.26}$  & 24.2$^{+2.6}_{-2.4}$  & 0.40$^{+0.08}_{-0.08}$  & 0.80 & 306 &  68 & 33.8  $^{+2\%}_{-2\%}$  & 0.232\\
$R_{2500}$ & 13 & 519 & 5.50$^{+0.27}_{-0.25}$  & 23.1$^{+2.5}_{-2.0}$  & 0.39$^{+0.07}_{-0.07}$  & 0.82 & 265 &  83 & 32.4  $^{+2\%}_{-2\%}$  & 0.222\\
$R_{\mathrm{cool}}$ & 13 & 128 & 4.94$^{+0.24}_{-0.22}$  & 16.1$^{+2.5}_{-2.0}$  & 0.42$^{+0.08}_{-0.08}$  & 0.89 & 208 &  98 & 22.9  $^{+3\%}_{-3\%}$  & 0.155\\
\hline
\end{tabular}
\begin{quote}
A dagger ($\dagger$) indicates core-excised regions fit with $Z$ fixed
at the iteratively determined value for
$R_{500-\mathrm{Core}}$. Bolometric luminosities were determined using
a diagonalized response function over the energy range 0.01-100.0 keV
with 5000 linearly spaced energy channels. Col. (1) Spectral
extraction region; Col. (2) Inner radius; Col. (3) Outer radius;
Col. (4) Gas temperature; Col. (5) Unabsorbed bolometric luminosity;
Col. (6) Gas abundance; Col. (7) Reduced \chisq; Col. (8) Degrees of
freedom; Col. (9) Percentage of emission attributable to source;
Col. (10) Model normalization; Col. (11) Background-subtracted count
rate.
\end{quote}
\end{table*}

%% $R_{200-\mathrm{Core}}$ & 174 & 1835 & 10.22$^{+5.38}_{-2.65}$ & 7.95$^{+1.50}_{-1.80}$  & 0.38$\dagger$ & 1.13 & 429 &  15 & 8.50   $^{+4\%}_{-4\%}$  & 0.066\\
%% $R_{5000-\mathrm{Core}}$ & 174 & 367 & 6.40$^{+1.01}_{-0.80}$  & 3.80$^{+0.44}_{-0.26}$  & 0.38$\dagger$ & 1.01 & 104 &  67 & 4.99   $^{+4\%}_{-3\%}$  & 0.036\\
%% $R_{7500-\mathrm{Core}}$ & 174 & 300 & 6.52$^{+1.16}_{-0.89}$  & 2.92$^{+0.23}_{-0.22}$  & 0.38$\dagger$ & 1.24 &  73 &  74 & 3.81   $^{+4\%}_{-4\%}$  & 0.027\\
%% $R_{200}$ & 13 & 1835 & 6.02$^{+0.45}_{-0.40}$  & 25.6$^{+2.7}_{-2.4}$  & 0.41$^{+0.10}_{-0.11}$  & 1.00 & 498 &  34 & 34.4  $^{+2\%}_{-2\%}$  & 0.240\\
%% $R_{5000}$ & 13 & 367 & 5.34$^{+0.25}_{-0.23}$  & 21.8$^{+2.1}_{-1.9}$  & 0.39$^{+0.07}_{-0.07}$  & 0.80 & 241 &  90 & 30.9  $^{+2\%}_{-2\%}$  & 0.210\\
%% $R_{7500}$ & 13 & 300 & 5.30$^{+0.24}_{-0.24}$  & 20.9$^{+1.7}_{-1.7}$  & 0.40$^{+0.07}_{-0.07}$  & 0.86 & 233 &  93 & 29.6  $^{+2\%}_{-2\%}$  & 0.202\\

\begin{table*}
  \begin{center}
    \caption{\sc Summary of X-ray Excesses Spectral Fits.\label{tab:excess}}
    \begin{tabular}{lccccccc}
      \hline
      \hline
      Region & \tx & $\eta$ & $E_{\mathrm{G}}$ & $\sigma_{\mathrm{G}}$ & $\eta_{\mathrm{G}}$ & Cash & DOF\\
      - & keV & $10^{-5}$ cm$^{-5}$ & keV & keV & $10^{-6}~\pcmsq~\ps$ & - & -\\
      (1) & (2) & (3) & (4) & (5) & (6) & (7) & (8)\\
      \hline
      Eastern excess     & 3.03$^{+1.19}_{-0.74}$ & $5.80^{+1.07}_{-0.97}$ & -                  & -                    & -                & 524 & 430\\
      Eastern excess     & 3.68$^{+3.34}_{-1.58}$ & $2.73^{+0.98}_{-0.94}$ & [0.89, 1.42, 4.23] & [0.04, 0.16, 3.6E-4] & [1.2, 2.0, 0.16] & 384 & 430\\
      Eastern excess bgd & 3.92$^{+0.35}_{-0.31}$ & $39.9^{+0.18}_{-0.17}$ & -                  & -                    & -                & 471 & 430\\
      Lower-NW excess & 2.55$^{+2.61}_{-0.98}$ & $0.66^{+0.11}_{-0.07}$ & -                  & -                    & -                & 387 & 430\\
      \hline
    \end{tabular}
    \begin{quote}
      Metal abundance was fixed at $0.5 ~\Zsol$ for all fits.
      Col. (1) Extraction region; Col. (2) Thermal gas temperature;
      Col. (3) Model normalization; Col. (4) Gaussian central
      energies; Col. (5) Gaussian dispersions; Col. (6) Gaussian
      normalizations; Col. (7) Modified Cash statistic; Col. (8)
      Degrees of freedom.
    \end{quote}
  \end{center}
\end{table*}

\begin{table*}
  \caption{\sc Summary of Cavity Properties.\label{tab:cylcavities}}
  \begin{tabular}{lcccccc}
    \hline
    \hline
    Cavity & $r$ & $l$ & \tsonic & $pV$ & \ecav & \pcav\\
    -- & kpc & kpc & $10^6$ yr & $10^{58}$ ergs & $10^{59}$ ergs & $10^{44}$ ergs s$^{-1}$\\
    (1) & (2) & (3) & (4) & (5) & (6) & (7)\\
    \hline
    NW & 6.40 & 58.3 & ${50.5 \pm 7.6}$ & ${5.78 \pm 1.07}$ & ${2.31 \pm 0.43}$ & ${1.45 \pm 0.35}$\\
    SE & 6.81 & 64.0 & ${55.4 \pm 8.4}$ & ${6.99 \pm 1.29}$ & ${2.80 \pm 0.52}$ & ${1.60 \pm 0.38}$\\
    \hline
  \end{tabular}
  \begin{quote}
    Col. (1) Cavity location; Col. (2) Radius of excavated cylinder;
    Col. (3) Length of excavated cylinder; Col. (4) Sound speed age;
    Col. (5) $pV$ work; Col. (6) Cavity energy; Col. (7) Cavity power.
  \end{quote}
\end{table*}

\begin{table*}
  \begin{center}
    \caption{\sc Summary of Nuclear Source Spectral Fits.\label{tab:nucspec}}
    \begin{tabular}{lccc}
      \hline
      \hline
      Component & Parameter & SP09 & SP99\\
      (1) & (2) & (3) & (4)\\
      \hline
      \pexrav\  & $\Gamma$              & $1.71^{+0.23}_{-0.65}$                & fixed to SP09\\
      -         & $\eta_{\mathrm{P}}$   & $8.07^{+0.64}_{-0.62}\times10^{-4}$   & $8.46^{+2.08}_{-2.12} \times 10^{-4}$\\
      \gauss\ 1 & $E_{\mathrm{G}}$      & $0.73^{+0.05}_{-0.24}$                & $0.61^{+0.10}_{-0.05}$\\
      -         & $\sigma_{\mathrm{G}}$ & $85^{+197}_{-53}$                     & $97^{+150}_{-97}$\\
      -         & $\eta_{\mathrm{G}}$   & $8.14^{+3.74}_{-5.82} \times 10^{-6}$ & $1.65^{+1.52}_{-1.00} \times 10^{-5}$\\
      \gauss\ 2 & $E_{\mathrm{G}}$      & $1.16^{+0.19}_{-0.33}$                & $0.90^{+0.17}_{-0.90}$\\
      -         & $\sigma_{\mathrm{G}}$ & $383^{+610}_{-166}$                   & $506^{+314}_{-262}$\\
      -         & $\eta_{\mathrm{G}}$   & $1.03^{+3.22}_{-0.48} \times 10^{-5}$ & $1.48^{+2.68}_{-1.16} \times 10^{-5}$\\
      \gauss\ 3 & $E_{\mathrm{G}}$      & $4.45^{+0.04}_{-0.04}$                & $4.46^{+0.04}_{-0.07}$\\
      -         & $\sigma_{\mathrm{G}}$ & $45^{+60}_{-45}$                      & $31^{+94}_{-31}$\\
      -         & $\eta_{\mathrm{G}}$   & $2.67^{+0.91}_{-0.86} \times 10^{-6}$ & $6.45^{+4.17}_{-3.69} \times 10^{-6}$\\
      -         & EW$^{\mathrm{corr}}_{\mathrm{K}\alpha}$ & $531^{+211}_{-218}$ & $1210^{+720}_{-710}$\\
      Statistic & \chisq                & 79.0                                  & 7.9\\
      -         & DOF                   & 74                                    & 15\\
      \hline
    \end{tabular}
    \begin{quote}
      \feka\ equivalent widths have been corrected for redshift. Units for
      parameters: $\Gamma$ is dimensionless, $\eta_{\mathrm{P}}$ is in ph
      keV$^{-1}$ cm$^{-2}$ s$^{-1}$, $E_{\mathrm{G}}$ are in keV,
      $\sigma_{\mathrm{G}}$ are in eV, $\eta_{\mathrm{G}}$ are in ph
      cm$^{-2}$ s$^{-1}$, EW$_{\mathrm{corr}}$ are in eV. Col. (1)
      \xspec\ model name; Col. (2) Model parameters; Col. (3) Values for
      2009 \cxo\ spectrum; Col. (4) Values for 1999 \cxo\ spectrum.
    \end{quote}
  \end{center}
\end{table*}

\clearpage
\begin{figure}[htp]
  \begin{center}
    \begin{minipage}[htp]{0.9\linewidth}
      \includegraphics*[width=\textwidth, trim=15mm 10mm 10mm 10mm, clip]{beta.eps}
      \caption{Surface brightness profiles for clusters requiring a
        $\beta$-model fit for deprojection (discussed in
        \S\ref{sec:beta}). The best-fit $\beta$-model for each cluster
        is overplotted as a dashed line. The discrepancy between the
        data and best-fit model for some clusters results from the
        presence of a compact X-ray source at the center of the
        cluster. These cases are discussed in Appendix
        \ref{app:beta}.}
      \label{fig:betamods}
    \end{minipage}
  \end{center}
\end{figure}
\clearpage
\begin{figure}[htp]
  \begin{center}
    \begin{minipage}[htp]{0.9\linewidth}
      \includegraphics*[width=\textwidth, trim=5mm 0mm 5mm 5mm, clip]{itplflat_rat.eps}
      \caption{Ratio of best-fit \kna\ for the two treatments of
        central temperature interpolation (see \S\ref{sec:temppr}):
        (1) temperature is free to decline across the central density
        bins ($\Delta T_{center} \ne 0$), and (2) the temperature
        across the central density bins is isothermal ($\Delta
        T_{center} = 0$). Filled black squares are clusters for which
        the \kna\ ratio is inconsistent with unity.}
      \label{fig:kcomp}
    \end{minipage}
  \end{center}
\end{figure}
\clearpage
\begin{figure}[htp]
  \begin{center}
    \begin{minipage}[htp]{0.9\linewidth}
      \includegraphics*[width=\textwidth, trim=5mm 0mm 5mm 5mm, clip]{k0res.eps}
      \caption{Best-fit \kna\ vs. redshift. Some clusters have
        \kna\ error bars smaller than the point. The clusters with
        upper-limits ({\it{black points with downward arrows}}) are:
        A2151, AS0405, MS 0116.3-0115, and RX J1347.5-1145. The
        numerically labeled clusters are: (1) M87, (2) Centaurus
        Cluster, (3) RBS 533, (4) HCG 42, (5) HCG 62, (6) SS2B153, (7)
        A1991, (8) MACS0744.8+3927, and (9) CL J1226.9+3332. For
        CLJ1226, \cite{2007ApJ...659.1125M} found best-fit $\kna = 132
        \pm 24 \ent$ which is not significantly different from our
        value of $\kna = 166 \pm 45 \ent$. The lack of $\kna < 10
        \ent$ clusters at $z > 0.1$ is most likely the result of
        insufficient angular resolution (see \S\ref{sec:angres}).}
      \label{fig:k0res}
    \end{minipage}
  \end{center}
\end{figure}
\clearpage
\begin{center}
  \begin{figure}[htp]
    \begin{minipage}[htp]{0.5\linewidth}
      \includegraphics*[width=\textwidth, trim=28mm 7mm 30mm 17mm, clip]{curvk0.eps}
    \end{minipage}
    \begin{minipage}[htp]{0.5\linewidth}
      \includegraphics*[width=\textwidth, trim=28mm 7mm 30mm 17mm, clip]{nbins_k0.eps}
    \end{minipage}
    \begin{minipage}[htp]{0.5\linewidth}
      \includegraphics*[width=\textwidth, trim=28mm 7mm 30mm 17mm, clip]{texpk0.eps}
    \end{minipage}
    \begin{minipage}[htp]{0.5\linewidth}
      \includegraphics*[width=\textwidth, trim=28mm 7mm 30mm 17mm, clip]{ntxbins_k0.eps}
    \end{minipage}
    \caption{Plots of possible systematics versus best-fit \kna.
      {\it{Top left:}} Best-fit \kna\ plotted versus average curvature
      of the corresponding entropy profile (see eq. \ref{eqn:avgcurv})
      There is no trend between these two quantities suggesting that
      \kna\ is not heavily influenced by the total shape of the
      entropy profile. {\it{Top right:}} Best-fit \kna\ plotted versus
      number of bins in the entropy profile which were used during
      fitting. Again, no trend is found. {\it{Bottom left:}} Best-fit
      \kna\ plotted versus the total used exposure time for each
      cluster. No trend is found. {\it{Bottom right:}} Best-fit
      \kna\ plotted versus the number of bins in the temperature
      profile for each cluster. As expected, fewer $\Tx(r)$ does not
      correlate with \kna.}
    \label{fig:sys}
  \end{figure}
\end{center}
\clearpage
\begin{center}
  \begin{figure}[htp]
    \begin{minipage}[htp]{0.5\linewidth}
      \includegraphics*[width=\textwidth, trim=28mm 7mm 30mm 17mm, clip]{splots_allt.eps}
    \end{minipage}
    \begin{minipage}[htp]{0.5\linewidth}
      \includegraphics*[width=\textwidth, trim=28mm 7mm 30mm 17mm, clip]{splots_tle4.eps}
    \end{minipage}
    \begin{minipage}[htp]{0.5\linewidth}
      \includegraphics*[width=\textwidth, trim=28mm 7mm 30mm 17mm, clip]{splots_gt4tle8.eps}
    \end{minipage}
    \begin{minipage}[htp]{0.5\linewidth}
      \includegraphics*[width=\textwidth, trim=28mm 7mm 30mm 17mm, clip]{splots_tgt8.eps}
    \end{minipage}
    \caption{Composite plots of entropy profiles for varying cluster
      temperature ranges. Profiles are color-coded based on average
      cluster temperature. Units of the color bars are keV. The solid
      line is the pure-cooling model of \cite{voitbryan}, the dashed
      line is the mean profile for clusters with $\kna \le 50 \ent$,
      and the dashed-dotted line is the mean profile for clusters with
      $\kna > 50 \ent$. {\it{Top left:}} This panel contains all the
      entropy profiles in our study. {\it{Top right:}} Clusters with
      $kT_X < 4$ keV. {\it{Bottom left:}} Clusters with $4\keV < kT_X
      < 8\keV$. {\it{Bottom right:}} Clusters with $kT_X > 8$
      keV. Note that while the dispersion of core entropy for each
      temperature range is large, as the $kT_X$ range increases so to
      does the mean core entropy.}
    \label{fig:splots}
  \end{figure}
\end{center}
\clearpage
\begin{figure}[htp]
  \begin{center}
    \begin{minipage}[htp]{0.9\linewidth}
      \includegraphics*[width=\textwidth, trim=20mm 10mm 10mm 10mm, clip]{k0hist.eps}
      \caption{{\it{Top panel:}} Histogram of best-fit \kna\ for all
        the clusters in \accept. Bin widths are 0.15 in log space.
        {\it{Bottom panel:}} Cumulative distribution of \kna\ values
        for the full sample. The distinct bimodality in \kna\ is
        present in both distributions, which would not be seen if it
        were an artifact of the histogram binning. A KMM test finds
        the \kna\ distribution cannot arise from a simple unimodal
        Gaussian.}
      \label{fig:k0hist}
    \end{minipage}
  \end{center}
\end{figure}
\clearpage
\begin{figure}[htp]
  \begin{center}
    \begin{minipage}[htp]{0.9\linewidth}
      \includegraphics*[width=\textwidth, trim=20mm 10mm 10mm 10mm, clip]{hifl_k0hist.eps}
      \caption{{\it{Top panel:}} Histogram of best-fit \kna\ values
        for the primary \hifl\ sample. Bin widths are 0.15 in log
        space.  {\it{Bottom panel:}} Cumulative distribution of
        best-fit \kna\ values. The distinct bimodality seen in the
        full \accept\ sample (Fig. \ref{fig:k0hist}) is also present
        in the \hifl\ subsample and shares the same gap between the
        low-entropy peak at 10-20 \ent\ and the high-entropy peak at
        100-200 \ent. That bimodality is present in both samples is
        strong evidence it is not a result of an unknown archival
        bias.}
      \label{fig:hiflk0}
    \end{minipage}
  \end{center}
\end{figure}
\clearpage
\begin{figure}[htp]
  \begin{center}
    \begin{minipage}[htp]{0.8\linewidth}
      \includegraphics*[width=\textwidth, trim=20mm 10mm 10mm 10mm, clip]{t0.eps}
    \end{minipage}
    \begin{minipage}[htp]{0.8\linewidth}
      \includegraphics*[width=\textwidth, trim=20mm 10mm 10mm 10mm, clip]{k0cool.eps}
    \end{minipage}
    \caption{{\it{Top panel:}} Log-binned histogram and cumulative
      distribution of best-fit core cooling times, $t_{c0}$
      (eqn. \ref{eqn:tc0}), for all the clusters in \accept. Histogram
      bin widths are 0.2 in log space. {\it{Bottom panel:}} Log-binned
      histogram and cumulative distribution of core cooling times
      calculated from best-fit \kna\ values, $t_{c0}(\kna)$
      (eqn. \ref{eqn:tck0}), for all the clusters in
      \accept. Histogram bin widths are 0.2 in log space. The
      bimodality we observe in the \kna\ distribution is also present
      in best-fit $t_{c0}$. However, the gaps between the two
      populations of $t_{c0}$ and $t_{c0}(\kna)$ differ by $\sim 0.3$
      Gyrs which may be an artifact of the binning.}
    \label{fig:t0}
  \end{center}
\end{figure}



%%%%%%%%%%%%%%%%%%%%
% End the document %
%%%%%%%%%%%%%%%%%%%%

\label{lastpage}
\end{document}

%% -- radio morphology might consistent with the ``X-shaped'' class of
%% radio galaxies \citep{1992ersf.meet..307L} where the opposing jet in
%% \irs\ has been beamed away
%% -- can the morph be explained through simple precession of smbh spin
%% axis?
%% -- based on polarization and radio measurements, reorientation angle
%% is $50-60\mydeg$
%% -- the synchrotron age of the radio source and lifetime from uv
%% scattering precludes simple precession of the smbh spin axis as the
%% precession rate required is much larger than is expected from models
%% \citep{precess}
%% -- can explain x-shape using the backflow scenario
%% \citep{1984MNRAS.210..929L, 1995ApJ...449...93W, 2002A&A...394...39C,
%%   2005ApJ...622..149K}. But this is an unlikely explanation
%% since... are wings aligned with galaxy minor axis? are there hot
%% spots? uplifted gas, scattering cone, and optical filaments make
%% compelling case for new jet.
%% -- could be a ``spin-flip'' system \citep{2002Sci...297.1310M} where
%% the smbh in \irs\ merged with a smaller smbh and the spin axis was
%% realigned through the transfer of angular momentum
%% -- this scenario makes more sense since \irs\ is at the center of
%% cluster where many mergers have taken place recently
%% -- smbh mergers are lengthy and difficult processes, specifically in
%% gas-rich systems where spin realignment may not be dramatic
%% \citep{2007ApJ...661L.147B}
%% -- but, if this is a common process in systems similar to \irs,
%% especially at higher redshift, then the implications are that agn
%% feedback is capable of interacting with a much larger volume of the
%% ambient environment
%% -- if nothing else, multi smbhs, when they are qsos, could more
%% readily destory an obscuring cocoon?
%% AGN feedback implementation in models
%% - Quasar mode (Kauffmann \& Haehnelt 2000)
%% - Powerful outflow evacutes gas and quenches star formation
%% immediately (not clear required by observations)
%% - Tremonti et al 2007 find high velocity outflows in post starburst
%% (also post- AGN) systems
%% - Radio mode (Croton et al. 2006)
%% - Heating of hot gas envelope of galaxies stops further gas cooling
%% - Very successful qualitatively
%% - Particularly effective when have hot gas as working surface (Dekel,
%% Cattaneo)
%% - P. Hopkins et al. 2006-present
%% - good to have deep X-ray hi-res spectroscopy to directly determine
%% the mass deposition rate from the X-ray halo

%% \rbon &=& 32 \left(\frac{\tx}{\keV}\right) \left(\frac{\mbh}{10^9
%%   ~\msol}\right) ~\pc\\
%% \dmbon &=& 0.01 \left(\frac{\nelec}{\pcc}\right)
%% \left(\frac{\tx}{\keV}\right)^{-3/2} \left(\frac{\mbh}{10^9
%%   ~\msol}\right)^{2} ~\msolpy \label{eqn:edd}\\

%% \begin{equation}
%%   \mgrav(< r) = -\frac{r \tx(r)}{G \mu \mH}\left[\frac{d\ln
%%       \tx(r)}{d\ln r} - \frac{d\ln n_g(r)}{d\ln r}\right]
%% \end{equation}
%% where (all cgs) $k$ is the Boltzmann constant, $G$ is the
%% gravitational constant, \mH\ is hydrogen mass, $\tx(r)$ and
%% $\nelec(r)$ are the radial X-ray gas temperature and gas density
%% profiles, respectively.

%% Potentially emitting species were first identified using the
%% ATOMDB\footnote{http://cxc.harvard.edu/atomdb/WebGUIDE/}. The
%% strongest lines within 0.1 keV of the 0.91 keV feature belong to Mg
%% \Rmnum{11} and Fe \Rmnum{21}, while for the 1.31 keV feature Si
%% \Rmnum{13} is the strongest. The emission around 4.4 keV broadening
%% the thermal rest-frame \feka\ 6.7 and 6.9 keV complex is consistent
%% with the rest-frame 6.4 keV \feka\ line.

%%%%%%%%%%%%%%%%%%%%%%%%%%%%%%%%%%%%%%%%%%%%%
%%%%%%%%%%%%%%%%%%%%%%%%%%%%%%%%%%%%%%%%%%%%%
Megan's v1 comments

It will take me a little longer to really digest your intricate
arguments about feeding the AGN and the illumination spectra, but I
really do appreciate the intellectual efforts here!  I think the paper
has basically what it needs except for a real introduction and an
abstract. The current introduction jumps immediately into details
without giving context to a reader without cluster/AGN expertise
(think graduate students/ non-X-ray astronomers etc.). The next step
may be to write a draft of the abstract, with specifics about your
conclusions, and with that hovering over the intro, that will allow
you to write a more accessible intro, ending with something like ``To
investigate these questions, we present a case study of X-ray and
radio observations of an unusual brightest cluster galaxy and ULIRG,
IRAS 09...''

What's currently in the introduction belongs in a ``Review of known
properties of IRAS 09'' and complete with a really brief summary of
what we're going to claim about the source.

Input from the other authors will help refine & clarify your arguments
about the AGN - shall we pass this by Mark next?  (After a draft
abstract and intro are in place.)  He'd also have some good
suggestions about where to start, as well as applying some AGN
expertise to your AGN analysis.

Anyway, here's a start - see comments (more) and edits (few). I'll try
to read through it more carefully soon, but since I'm recommending
some macro re-organization, I figured that you'd like to hear about
sooner than details. In the AGN category I do not feel expert enough
to make canonical pronouncements :) but I'll definitely have more
questions -- some of them may be more naive than others, so have
patience on that.
%%%%%%%%%%%%%%%%%%%%%%%%%%%%%%%%%%%%%%%%%%%%%
%%%%%%%%%%%%%%%%%%%%%%%%%%%%%%%%%%%%%%%%%%%%%
