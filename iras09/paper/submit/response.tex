\documentclass[11pt]{article}
\setlength{\topmargin}{-.3in}
\setlength{\oddsidemargin}{-0.1in}
\setlength{\evensidemargin}{-0.1in}
\setlength{\textwidth}{6.7in}
\setlength{\headheight}{0in}
\setlength{\headsep}{0in}
\setlength{\topskip}{0.5in}
\setlength{\textheight}{9.25in}
\setlength{\parindent}{0.0in}
\setlength{\parskip}{1em}
\usepackage{common,graphicx,hyperref,epsfig}

\pagestyle{empty}

\begin{document}

Dear Editor:

Below is our reply to the referee's report for MN-10-1576-MJ. The
referee's comments are {\it{italicized}} and our replies are not. The
revised manuscript contains substantial revisions beyond those
requested by the referee, though we still address the comments in
order.

\hrulefill

{\it{``Unfortunately, a decrement or an excess over a particular model
    is only meaningful if that model is itself physically meaningful,
    and in the present work, the authors provide no evidence to
    support the physicality of the particular model which is fitted.
    By inspection of the residual image (Fig 5), what is immediately
    apparent is that the excess is oriented roughly east-west, while
    the decrements are oriented in an almost exactly perpendicular
    direction. This is exactly what is expected when fitting a model
    which does not have the correct shape. (An example would be
    fitting such a model to an image which has a single excess, in
    which case the normalization, and ellipse orientation would likely
    be distorted by trying to fit this feature, producing ostensible
    deficits elsewhere). In light of this, I am left deeply skeptical
    of the authors' interpretations of these features.''}}

We appreciate the discussion that follows this comment and in the
revised manuscript we now address the surface brightness models used
for fitting and the statistical robustness of the cavity
detections. In the revised text, we present three different models
which vary in ellipticity and discuss the effect on residual
substructure accordingly. We also discuss the insignificant influence
of the excesses on the modeling and emphasize that an essential
component in claiming the cavities and excesses are real structures is
their one-to-one correlation with emission features across multiple
wavebands. We have chosen not to adopt the suggestion of the referee
to perform MARX simulations since it introduces more uncertainties
than addresses (e.g. accuracy of the input 2D surface brightness and
temperature models).

\hrulefill

{\it{``Nevertheless, I am prepared to accept that the image is
    asymmetric, as suggested by Fig 1 (right panel; although the
    absence of a colour-scale for this image, and the use of
    smoothing, makes it hard to assess the formal statistical
    significance of the asymmetry).''}}

Discussion of the significance of each substructure has been added to
the text, and color bars and labels now appear in the associated
images.

\hrulefill

{\it{``Morphological disturbances are actually fairly commonplace in
    the centres of normal galaxy clusters (which, according to the
    authors, this is) due to, for example, ``cold fronts'', cavities
    (such as are postulated here), or (possibly) shocks. Since the
    nature of the asymmetries is essential for the results in this
    paper, the problem for the authors is to establish that the only
    viable explanation for the morphology of the gas is an excess in
    the east-west direction, and cavities perpendicular to it. Could
    the morphology be explained in terms of: only excess(es), only
    cavities, or possibly disturbances similar to cold fronts?  The
    question of cold fronts, in particular, is summarily dismissed by
    the authors in Section 4, where they argue that there is an
    absence of evidence of temperature or density discontinuities in
    the radial profiles, and in a 2D analysis.  However: a) the 2D
    analysis is neither explained, nor the results of that analysis
    presented in this paper (hence the text discussing it should
    either be removed, or considerably expanded, and figures
    presented). b) the absence of evidence is not evidence of
    absence. Are the data also consistent with the asymmetries being
    (at least in part) due to cold fronts?''}}

Mention of the 2D analysis has been dropped since it adds no value to
the discussion because the data is insufficient to constrain the
presence of shocks or cold fronts.

\hrulefill

{\it{``Next, even if we take at face value the residual plot shown in
    Fig 5, there remains the question of the statistical significance
    of the features which are shown. I appreciate that there is
    something of an industry of visually identifying and interpreting
    features in galaxy cluster images, without quantifying the formal
    statistical significance of those features.  This is, in my
    opinion, only acceptable insofar as the S/N is high enough for
    features to be unambiguously evident in minimally processed
    images, so that the computation of a formal significance is
    somewhat redundant. This is self-evidently not the case in this
    paper; the S/N appears low and the ``cavities'' are only visually
    identified by apparently stretching the colour-scale on a residual
    image. Coupled with the scientific importance of finding clear
    cavities in a system such as IRAS 09104+4109, an ``unambiguous''
    detection demands a more thorough, quantitative analysis.  There
    are a number of ways this could be done, but I would suggest at
    minimum a Monte-Carlo analysis, in which realistic images are
    generated from a smooth, elliptical model, but with excesses
    similar to the purported features (but no cavities). These should
    be processed identically to the real data, and any spurious
    cavities assessed and characterized. Since the authors are using
    visual inspection to identify cavities, for this to be done
    objectively, a clear set of criteria will need to be adopted to
    explain what constitutes a cavity, and how its size and position
    are determined. A similar kind of analysis could be carried out to
    investigate the excess(es).''}}

Discussion of the significance of each substructure has been added to
the text.

\hrulefill

{\it{``The derived jet mechanical power is obtained from the
    properties of the cavities that are identified by the
    authors. However, the error-bars are determined entirely from the
    uncertainties on the spectroscopic properties of the ambient gas
    (eg pressure, sound speed), rather than the position and geometry
    of the cavities. I suspect this latter to be a major source of
    uncertainty, and so it will need to be folded in. Perhaps the
    authors could achieve this by Monte-Carlo simulations similar to
    those outlined above, in which cavities are added to the model
    before simulation, and the errors on the geometry and location are
    assessed. (This could also allow the authors to place limits on as
    yet undetected cavities, that could affect the jet power
    estimation).''}}

A systematic uncertainty of 10\% is added to the volume estimates to
account for unknown projection effects. Utilizing the suggested MARX
Monte Carlo approach to assessing the volume uncertainties introduces
more questions than it addresses. Given that the referee is skeptical
of the cavity detections as they are, placing constraints on
undetected cavities seems an unnecessary exercise.

\hrulefill

{\it{``The discussion of the analysis in this section was lacking in
    some necessary detail. How are the different regions chosen and
    characterized?  In particular, where are the background regions?
    Performing a background subtraction, such as described, makes
    sense only if the ``source'' region is expected to have the same
    spectrum as the ``background'', with an additional component
    added. If it is actually hot gas at a different temperature, what
    you would obtain by doing a subtraction is unclear. Since the
    authors already have background spectral models for everywhere on
    the CCD (used in Sect 4), why don't they employ these and fit the
    data in the source and background regions separately? This would
    considerably aid in interpreting these data, and assessing whether
    we're just dealing with temperature asymmetries. I'm assuming the
    fits shown in Table 3 are ``background subtracted''; which
    background was used for the ``EEx bgd''?''}}

A figure has been added showing the excess source and background
regions and the text explains how they were chosen. Utilizing the deep
backgrounds for spectral analysis does not address the nature of the
excesses because cluster emission will not be removed.

\hrulefill

{\it{``The authors discuss some models as being a ``poor fit''. What
    does this mean; How was it assessed; How poor was the fit? Some
    models are apparently ``preferred'' over others; apparently Monte
    Carlo simulations were used to ascertain this, but these are not
    explained, nor the significance of the difference
    quantified. Additionally, what is the impact of varying the
    metallicity? Figures would be especially helpful here: The authors
    should overlay on Fig 6 (right panel) the best-fitting model
    without the additional Gaussian lines. They should also decompose
    their preferred model into its different components: the ICM
    component, and the Gaussian lines, and provide a comparison
    between this parameterization and the cloudy model they adopt.''}}

Plots of the spectra for each excess are not revealing because of the
low-SN, and each of the numerous questions given above is now
addressed in the text. The models are decomposed into their components
as shown in Table 3. The validity of the base CLOUDY model can be
found by comparing the values in Table 3, plots in Figure 6, and text
in Section 6.2.

\hrulefill

{\it{``The ordering of the material was confusing. Section 3 uses
    results from Section 4; Section 6 uses results from Section 7. I
    could see no reason why these respective sections could not be
    re-ordered to aid with clarity.''}}

The structure of the paper has been extensively changed.

\hrulefill

{\it{``Section 3: I was very puzzled at this section, which seemed
    rather unconnected to anything else in the paper. I can understand
    the authors wishing to assess whether the cluster is ``ordinary''
    by asking if it obeys standard scaling relations, and in that
    sense some of the material here is relevant. However, this is
    never done in this section; rather they compute the gravitating
    mass and gas mass within R200 by extrapolating fits within 0.2
    R200 to the temperature and density profile, while the global
    parameters from the scaling relations are computed within R500!
    They at no point explain why we should be interested in the
    tedious tabulation of various global parameters given in Table 1,
    and don't use them elsewhere in the paper. Therefore, it was
    unclear how this material led the authors to the conclusion ``RX
    J0913.7+4056 appears to be a typical massive, relaxed galaxy
    cluster."}}

The discussion has been expanded, R200 has been changed to R500, the
table has been removed, and the section has been moved to the end of
the paper. It is vital to demonstrate that the host cluster is rather
normal and hence this discussion remains in the manuscript.

\hrulefill

{\it{``Section 4: The deprojection method for computing the gas
    density needs to be summarized, in addition to the citation to
    Cavagnolo 2009. Having read the description in Cavagnolo 2009, I
    am still puzzled as to how the emissivity is incorporated
    self-consistently into this calculation. Surely eta (r) is
    determined from the projected temperature/ abundance profiles, so
    doesn't this assume that the projected and deprojected
    temperature/ abundance profiles are identical? How do errors on
    the metallicity/ abundance profile propagate into eta, and hence
    onto the gas density profile? How are the errors on the gas
    density computed? The authors should provide details of how the
    Monte Carlo was performed.''}}

Discussion of the widely used Kriss et al. deprojection method is
lengthy and hence remains only referenced. Answers to all the
questions above are provided in the referred publication Cavagnolo et
al. 2009 and references therein.

\hrulefill

{\it{``The quality of the fit to the entropy profile is not
    discussed. The best-fitting model should be overlaid on Fig 3. If
    the system is morphologically disturbed in the core, that means
    that the deprojection which is implicit in the derivation of the
    entropy is not going to be correct, and the gas entropy within,
    say, 30 kpc, will not be correctly derived. How much will this
    affect the fit presented in this section?''}}

The \chisq\ and DOF are now noted in the text. Morphological
disturbances are only relevant if gas over a broad range of entropy
are co-spatial and thus averaged when we assume azimuthal
symmetry. Again, these issues are discussed in-detail in Cavagnolo et
al. 2009 and references therein.

\end{document}
