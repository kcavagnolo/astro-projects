%%%%%%%%%%%%%%%%%%%
% Custom commands %
%%%%%%%%%%%%%%%%%%%

\newcommand{\mykeywords}{cooling flows -- galaxies: clusters:
  galaxies: individual (\iras): clusters: individual (\rxj)}
\newcommand{\mytitle}{Direct Evidence of Quasar Radiative and
  Mechanical Feedback in \iras}
\newcommand{\mystitle}{Feedback in \iras}
\newcommand{\iras}{IRAS 09104+4109}
\newcommand{\irs}{IRAS09}
\newcommand{\rxj}{RX J0913.7+4056}
\newcommand{\tsync}{\ensuremath{t_{\mathrm{sync}}}}
\newcommand{\refl}{\ensuremath{r_{\mathrm{refl}}}}
\newcommand{\fekaew}{\ensuremath{\mathrm{EW}_{\mathrm{K}\alpha}}}
\newcommand{\cltx}{\ensuremath{T_{\mathrm{cl}}}}
\newcommand{\leff}{\ensuremath{\lambda_{\mathrm{Edd}}}}
\newcommand{\lqso}{\ensuremath{\lbol^{\mathrm{QSO}}}}
\newcommand{\esens}{\ensuremath{E_{\rm{sens}}}}

%%%%%%%%%%
% Header %
%%%%%%%%%%

\documentclass[useAMS,usenatbib]{mn2e}
\usepackage{graphicx, here, common, longtable, ifthen, amsmath,
amssymb, natbib, lscape, subfigure, mathptmx, url, times, array}
\usepackage[abs]{overpic}
\usepackage[pagebackref,
  pdftitle={\mytitle},
  pdfauthor={Dr. Kenneth W. Cavagnolo},
  pdfsubject={ApJ},
  pdfkeywords={},
  pdfproducer={LaTeX with hyperref},
  pdfcreator={LaTeX with hyperref}
  pdfdisplaydoctitle=true,
  colorlinks=true,
  citecolor=blue,
  linkcolor=blue,
  urlcolor=blue]{hyperref}

\title[\mystitle]{\mytitle}

\author[Cavagnolo et al.]{K. W. Cavagnolo$^{1}$\thanks{Email:
    kcavagno@uwaterloo.ca}, M. Donahue$^{2}$, B. R. McNamara$^{1,3,4}$,
  G. M. Voit$^{2}$, and M. Sun$^{5}$\\
  $^{1}$Department of Physics and Astronomy, University of Waterloo,
  Waterloo, ON N2L 3G1, Canada.\\
  $^{2}$Department of Physics and Astronomy, Michigan State University,
  East Lansing, MI, 48824-2320, USA.\\
  $^{3}$Perimeter Institute for Theoretical Physics, 31 Caroline St. N.,
  Waterloo, Ontario, N2L 2Y5, Canada\\
  $^{4}$Harvard-Smithsonian Center for Astrophysics, 60 Garden Street,
  Cambridge, MA 02138, USA\\
  $^{5}$Department of Astronomy, University of Virginia,
  Charlottesville, VA, 22904, USA.}

\begin{document}

\date{Accepted (2010 Month Day). Received (2010 Month Day); in
  original form (2010 Month Day)}

\pagerange{\pageref{firstpage}--\pageref{lastpage}} \pubyear{2010}

\maketitle

\label{firstpage}

%%%%%%%%%%%%
% Abstract %
%%%%%%%%%%%%

\begin{abstract}
  We present a detailed study of the hyperluminous infrared brightest
  cluster galaxy \iras\ using a deep \chandra\ X-ray observation. %% The
%%   X-ray observations reveal interaction between the galaxy's halo and
%%   the mechanical outflow \& radiative emission of the central AGN. We
%%   show that the properties of the \iras\ nucleus are consistent with
%%   reflected emission from an AGN embedded in a moderately
%%   Compton-thick medium. The observations indicate that \iras\ may be
%%   in transition from a radiatively-dominated to a
%%   mechanically-dominated mode of AGN feedback. We highlight why
%%   \iras\ is an ideal test case of a very short-lived but highly active
%%   stage of galaxy cluster and central galaxy formation.
\end{abstract}

%%%%%%%%%%%%
% Keywords %
%%%%%%%%%%%%

\begin{keywords}
  \mykeywords
\end{keywords}

%%%%%%%%%%%%%%%%%%%%%%
\section{Introduction}
\label{sec:intro}
%%%%%%%%%%%%%%%%%%%%%%

\iras\ (hereafter, \irs) is an uncommon low-redshift ($z = 0.4418$)
ultraluminous infrared galaxy (ULIRG; $L_{\mathrm{IR}} > 10^{12}
~\lsol$). Unlike most ULIRGs, \irs\ is the brightest cluster galaxy
(BCG) in a rich galaxy cluster, but unlike most BCGs, \irs\ is a
Seyfert-2 with 99\% of the bolometric luminosity emerging longward of
1 \mymicron\ due to a heavily-obscured quasar (QSO) with a luminosity
$\sim 10^{47} ~\lum$ \citep{1988ApJ...328..161K, 1993ApJ...415...82H,
  1994ApJ...436L..51F, 1998ApJ...506..205E, 2000A&A...353..910F,
  2001MNRAS.321L..15I}.

Data reduction is discussed in Section \ref{sec:obs}. Properties of
the ICM are analyzed in Sections \ref{sec:global} and
\ref{sec:rad}. Analysis of the ICM cavities, SMBH fueling, and QSO
irradiation of the ICM are given in Sections \ref{sec:cavs},
\ref{sec:fuel}, and \ref{sec:excess}, respectively. The complex
nuclear source is discussed in Section
\ref{sec:centsrc}. Interpretation of the results is given in Section
\ref{sec:evo}, with a brief summary in Section \ref{sec:summ}. \LCDM,
for which a redshift of $z = 0.4418$ corresponds to $\approx 9.1$ Gyr
for the age of the Universe, $\da \approx 5.72$ kpc arcsec$^{-1}$, and
$\dl \approx 2.45~\Gpc$.

%%%%%%%%%%%%%%%%%%%%%%%%%%%%%%%%%%%%%%%%%
\section{Observations and Data Reduction}
\label{sec:obs}
%%%%%%%%%%%%%%%%%%%%%%%%%%%%%%%%%%%%%%%%%

Unless stated otherwise, all spectral fits were performed over the
energy range 0.7-7.0 keV with the \chisq\ statistic in \xspec\ 12.4
\citep{xspec} using an absorbed, single-temperature \mekal\ model
\citep{mekal1} with abundance as a free parameter (\citealt{ag89}
distribution) and quoted uncertainties of 90\% confidence. All
spectral models had the Galactic absorbing column density fixed at
$\nhgal = 1.58 \times 10^{20} ~\pcmsq$ \citep{lab}. For the ICM, we
assumed a mean molecular weight of $\mu = 0.597$ and adiabatic index
$\gamma = 5/3$.

%%%%%%%%%%%%%%%%%%%%%
\subsection{\chandra}
\label{sec:xray}
%%%%%%%%%%%%%%%%%%%%%%

A 77.2 ks observation of \irs\ was taken on 2009 January 09 with the
ACIS-I instrument (\dataset [ADS/Sa.CXO#Obs/10445] {ObsID 10445}; PI
Cavagnolo). The 9 ks archival \chandra\ observation of \irs\ from 1999
November 03 taken with the ACIS-S array was included in our analysis
(\dataset [ADS/Sa.CXO#Obs/00509] {ObsID 509}; PI Fabian). Both
datasets were reprocessed and reduced using \ciao\ and
\caldb\ versions 4.2. X-ray events were selected using \asca\ grades,
and corrections for the ACIS gain change, charge transfer
inefficiency, and degraded quantum efficiency were applied. Point
sources were located and excluded using {\textsc{wavdetect}} and
visual inspection. Light curves from a source free region of each
observation were created for a front-illuminated and back-illuminated
CCD and compared to look for flares. Time intervals which fell outside
$20\%$ of the mean background count rate were excluded. After flare
exclusion, the final exposure times for ObsID 509 and 10445 were 7 ks
and 76 ks, respectively.

For imaging analysis, the flare-clean events files were reprojected to
a common tangent point and summed. The astrometry of the ObsID 509
dataset was improved using a new aspect solution created with the
\ciao\ tool {\textsc{reproject\_aspect}} and the positions of several
field sources. After astrometry correction, the positional accuracy
between both observations was comparable to the resolution limit of
the ACIS detectors. We refer to the final point source free,
flare-clean, exposure-corrected images as the ``clean'' images. In
Figure \ref{fig:imgs} are the 0.5-10.0 keV mosaiced clean image of
\rxj, a zoom-in of the core region harboring \irs, and photons in the
energy range 4.35-4.50 keV associated with the \feka\ fluorescence
line from the nucleus (discussed in Section \ref{sec:centsrc}). Unless
stated otherwise, the X-ray analysis in this paper relates to the
\chandra\ data only.

%%%%%%%%%%%%%%%%%
\subsection{\xmm}
\label{sec:xmm}
%%%%%%%%%%%%%%%%%

\xmm\ data is utilized in Section \ref{sec:centsrc} to check our
results for the nuclear source against the analysis presented in
\citet{2007A&A...473...85P}. \xmm\ observed \irs\ on 2003 April 23 for
14 ks with the EPIC PN and MOS detectors (ObsID 0147671001; PI
Fiore). Data was reprocessed using SAS version 7.1 and CCF release
258. Events files were created using the tools {\textsc{EMCHAIN}} and
{\textsc{EPCHAIN}} for patterns 0-4. Light curves were extracted from
the energy range 10-12 keV for the full field after cluster emission
and \chandra\ identified point sources were removed. After flare
exclusion, the effective exposure times for PN and MOS were 10 ks and
12 ks, respectively. A source spectrum grouped to 20 counts per energy
channel was extracted from a region centered on the X-ray peak and
extending to \rf\ (defined in Section \ref{sec:global}). A background
spectrum was extracted from a source-free region with an area equal to
the source region.

%%%%%%%%%%%%%%%%%%%%%%
\subsection{\bepposax}
\label{sec:beppo}
%%%%%%%%%%%%%%%%%%%%%%

Conclusions reached in previous studies regarding the nature of the
\irs\ nuclear absorber have relied on the \bepposax\ hard X-ray
detection discussed by \citet{2000A&A...353..910F}. Here, we repeat
and confirm that analysis in order to compare our results in Section
\ref{sec:centsrc} against \citet{2000A&A...353..910F}. We retrieved
and re-analyzed the \bepposax\ data taken 1998 April 18 (ObsCode
50273002; PI Franceschini). The data was reduced and analyzed with
\saxdas\ version 2.3.1 using the calibration data, cookbook, and epoch
appropriate response functions available from
HEASARC\footnote{http://heasarc.nasa.gov/docs/sax}.

Data from the PDS instrument (\esens\ = 15-300 keV) was accumulated,
screened for good time intervals, and then a light curve was
extracted. No significant variations of the light curve were
detected. A PDS total spectrum was extracted from the on-axis data,
and background subtraction was performed using the variable rise time
threshold. Using data from the LECS (\esens\ = 0.1-10 keV) and MECS
(\esens\ = 1-10 keV) instruments, spectra were extracted from the
region enclosed by \rf\ with a background spectrum taken from an
annulus outside \rf. We measure a PDS 15-80 keV count rate of $0.106
\pm 0.055$ ct \ps. Fitting the PDS spectrum over the energy range
20-200 keV with an absorbed power-law having fixed spectral index of
$\Gamma = 1.7$ yielded fluxes of $f_{10 \dash 200} =
2.09^{+1.95}_{-1.95} \times 10^{-11} ~\flux$ and $f_{20 \dash 100} =
1.10^{+1.57}_{-1.63} \times 10^{-11} ~\flux$. The count rate and
fluxes are consistent with the results presented in
\citet{2000A&A...353..910F}.

%%%%%%%%%%%%%%%%%%%%%%%%%%%%%%%%%
\subsection{\integral\ \& \swift}
\label{sec:integral}
%%%%%%%%%%%%%%%%%%%%%%%%%%%%%%%%%

In this section we show that \integral\ and \swift\ data do not reveal
any hard X-ray sources in the vicinity of \irs, but that the flux
upper limits are consistent with the \bepposax\ PDS
detection. Assuming the upper limits are representative of an $\approx
1\mydeg$ region around \irs\ (\ie\ the full-width half maximum PDS
field of view), the lack of detected hard X-ray sources near
\irs\ suggests that the PDS detection did not originate from a
brighter off-axis source, assuming the source is/was not transient or
a one-off event.

Between 2005--2007, \irs\ was within the \integral\ field of view
during 85 pointings. Data was collected with the ISGRI (\esens\ = 15
keV-1 MeV) and JEM-X (\esens\ = 3-35 keV) instruments for 81 and 79
pointings, respectively. Datasets were reduced using \osa\ version 8.0
and version 8.0.1 of the Instrument Characteristics. For each
instrument, mosaiced images of intensity, significance, variance, and
exposure were generated from the background-subtracted images of each
pointing. The combined ISGRI and JEM-X effective exposure times were
200 ks and 210 ks, respectively. Versions 1 and 30 of the
\integral\ Reference Catalogue were used for source detection. The
\osa\ source detection routines did not locate any $5\sigma$ sources
in the ISGRI and JEM-X mosaiced images. Additional visual inspection
of the images did not reveal any features which might suggest emission
from a source.

The ISGRI and JEM-X instrument responses have a strong energy
dependence, thus, upper limits calculated using only the variance
images (\ie\ assuming uniform sensitivity) will systematically
underestimate the flux limit. To account for this variation, flux
upper limits were derived by integrating the ISGRI and JEM-X responses
over a specified energy range and weighting by an assumed spectral
shape. We assumed the \irs\ $E> 10$ keV spectrum goes as $S_{\nu} =
\nu^{-1.7}$ with no high-energy cut-off. Between 10-35 keV and 20-100
keV, we derive $3\sigma$ upper limits of $f_{10 \dash 35} = 1.28
\times 10^{-12} ~\flux$ and $f_{20 \dash 100} = 5.70 \times 10^{-11}
~\flux$, respectively. The \integral\ $1\sigma$ 20-100 keV flux limit
is narrowly higher than the \bepposax\ 20-100 keV PDS measured flux,
and is consistent with a $z = 0.44$ source which would not be detected
in the IBIS Extragalactic AGN Survey \citep{2006ApJ...636L..65B}. As a
check of our analysis, the 22 month \swift-BAT (\esens\ = 15-150 keV)
survey \citep{2010ApJS..186..378T} was searched for sources within
$5\mydeg$ of \irs. The survey has a 14-195 keV $4.8\sigma$ detection
limit of $2.2 \times 10^{-11} ~\flux$, 14\% higher than the 14-195 keV
\irs\ flux expected based on the \bepposax\ detection, and consistent
with the \integral\ upper limits, no sources were found.

%%%%%%%%%%%%%%%%
\subsection{VLA}
\label{sec:vla}
%%%%%%%%%%%%%%%%

From 1986 to 2000, \irs\ was observed with VLA at multiple radio
frequencies and resolutions. \citet[][hereafter
  H93]{1993ApJ...415...82H} also present analysis of 1.4 and 5 GHz VLA
observations. Continuum mode observations were taken from the VLA
archive and reduced using version 3.0 of the Common Astronomy Software
Applications (\casa). Flagging of bad data was performed using a
combination of \casa's {\textsc{flagdata}} tool in {\textsc{rfi}} mode
and manual inspection. Radio images were generated by Fourier
transforming, cleaning, self-calibrating, and restoring individual
radio observations. The additional steps of phase and amplitude
self-calibration were included to increase the dynamic range and
sensitivity of the radio maps. All sources within the primary beam and
first side-lobe detected with fluxes $> 5\sigma_{\mathrm{rms}}$ were
imaged to further maximize the sensitivity of the radio maps.

Resolved radio emission associated with \irs\ is detected at 1.4 GHz,
5.0 GHz, and 8.4 GHz, while a $3\sigma$ upper limit of $0.84$ mJy is
established at 14.9 GHz. Fluxes for unresolved emission at 74 MHz, 151
MHz, and 325 MHz were retrieved from VLSS \citep{vlss}, 7C Survey
\citep{1999MNRAS.306...31R}, and WENSS \citep{1997A&AS..124..259R},
respectively. No formal detection is found in VLSS, however, an
overdensity of emission at the location of \irs\ is evident. For
completeness, we measured a flux for the potential source, but
excluded the value during fitting of the radio spectrum.

The combined 1.4 GHz image reveals the most extended structure, and
thus our discussion regarding radio morphology is guided using this
frequency. The deconvolved, integrated 1.4 GHz flux of the continuous
extended structure coincident with \irs, and having $S_{\nu} \ga
3\sigma_{\mathrm{rms}}$, is $14.0 \pm 0.51$ mJy. A significant spur of
radio emission northeast of the nucleus is detected with flux $0.21
\pm 0.07$ mJy. Radio contours were generated beginning at 3 times the
rms noise and moving up in 6 log-space steps to the peak intensity of
4.7 mJy beam$^{-1}$. These are the contours referenced in all
following discussion of the radio source morphology and its
interaction with the X-ray gas.

Below we determine properties of the radio source for the purpose of
constraining the AGN outflow properties which created X-ray cavities
in the \rxj\ halo (see Section \ref{sec:cavs}). The radio spectrum was
fitted between 151 MHz and 8.4 GHz for the full radio source (lobes,
jets, \& core) with the well-known KP \citep{1962SvA.....6..317K,
  pach}, JP \citep{1973A&A....26..423J}, and CI
\citep{1987MNRAS.225..335H} synchrotron models. The models primarily
vary in their assumptions regarding the electron pitch-angle
distribution and number of injections. The models were fitted to the
radio spectrum using the code of \citet{2005ApJ...624..656W}, which is
based on the method of \citet{1991ApJ...383..554C}. The JP model
(single electron injection, randomized but isotropic pitch-angle
distribution) yields the best fit with \chisq(DOF)$ = 0.491(3)$, a
break frequency of $\nu_B = 12.9 \pm 1.0$ GHz, and a low-frequency
($\nu < 2$ GHz) spectral index of $\alpha = -1.10 \pm 0.09$. The
bolometric radio luminosity was approximated by integrating under the
JP curve between $\nu_1 = 10$ MHz and $\nu_2 =$ 10,000 MHz, giving
$\lrad = 1.09 \times 10^{42}~\lum$. The radio spectrum and best-fit
models are shown in Figure \ref{fig:radio}.

Assuming inverse-Compton (IC) scattering and synchrotron emission are
the dominant radiative mechanisms of the radio source, the time since
acceleration for an isotropic particle population is given by
\citet{2001AJ....122.1172S} as
\begin{equation}
  \tsync = 1590 \left(\frac{B^{1/2}}{B^2 + B_{\mathrm{CMB}}^2}\right)~
  \left[\nu_{\mathrm{B}} (1+z)\right]^{-1/2} ~\Myr
\end{equation}
where $B$ [\mg] is magnetic field strength, $B_{\mathrm{CMB}} =
3.25(1+z)^2$ [\mg] is a correction for IC losses to the cosmic
microwave background, $\nu_{\mathrm{B}}$ [GHz] is the radio spectrum
break frequency, and $z$ is the dimensionless source redshift. Note
that this form for \tsync\ neglects energy lost to adiabatic expansion
of the radio plasma \citep{1968ARA&A...6..321S}. We assume that $B$ is
not significantly different from the equipartition magnetic field
strength, $B_{\mathrm{eq}}$, which is derived from the minimum energy
density condition as \citep{1980ARA&A..18..165M}
\begin{equation}
  B_{\mathrm{eq}} = \left[\frac{6 \pi ~c_{12}(\alpha,\nu_1, \nu_2)
      ~\lrad ~(1+k)}{V \Phi}\right]^{2/7} ~\mathrm{\mg}
\end{equation}
where $c_{12}(\alpha,\nu_1,\nu_2)$ is a dimensionless constant derived
in \citet{pach}, \lrad\ [$\lum$] is the integrated radio luminosity
from $\nu_1$ to $\nu_2$, $k$ is the dimensionless ratio of lobe energy
in non-radiating particles to that in relativistic electrons, $V$
[\cc] is the radio source volume, and $\Phi$ is a dimensionless
radiating population volume filling factor. Synchrotron age as a
function of $k$ and $\Phi$ for the full radio source is shown in
Figure \ref{fig:radio}. For various combinations of $k$ and $\Phi$,
$B_{\mathrm{eq}} \approx 4 \dash 57 ~\mg$, with associated synchrotron
ages in the range $\approx 1 \dash 12$ Myr. Repeating the above
analysis using only radio lobe emission at 1.4 GHz, 5.0 GHz, and an
8.4 GHz upper limit, reveals \tsync\ could be as high as $30$ Myr as a
result of a significantly steeper spectral index and lower break
frequency.

%%%%%%%%%%%%%%%%%%%%%%%%%%%%%%%
\section{Global ICM Properties}
\label{sec:global}
%%%%%%%%%%%%%%%%%%%%%%%%%%%%%%%

Our analysis begins at the cluster scale with the integrated
properties of the \rxj\ ICM hosting \irs. We define the mean cluster
temperature, \cltx, as the ICM temperature within a core-excised
aperture extending to $R_{\Delta_c}$, the radius at which the average
cluster density is $\Delta_c$ times the critical density for a
spatially flat Universe. We chose $\Delta_c = 500$ and used the
relations from \cite{2002A&A...389....1A} to calculate
$R_{\Delta_c}$. \rxj\ has a luminous, cool core and complex nucleus
which are not representative of \cltx, thus, the convention of
\citet{2007ApJ...668..772M} was followed and emission inside $0.15
~\rf$ was excised. Source spectra were extracted from the region $0.15
\dash 1.0~\rf$ and background spectra were extracted from reprocessed
\caldb\ blank-sky backgrounds (see Section \ref{sec:rad}). Because
\cltx\ and $R_{\Delta_c}$ are correlated in the adopted definitions,
they were recursively determined until three consecutive iterations
produced \cltx\ values which agreed within the 68\% confidence
intervals. We measure $\cltx = 7.54^{+1.76}_{-1.15}$ keV corresponding
to $\rf = 1.16^{+0.27}_{-0.19}~\Mpc$. Measurements for a variety of
$R_{\Delta_c}$ apertures are summarized in Table \ref{tab:specfits}.
The BCG nucleus emits strong \feka\ emission which affects the
spectral fitting, and for any aperture including the core, the nucleus
was excluded using a region twice the size of the \chandra\ PSF 90\%
EEF (see Section \ref{sec:centsrc} for details).

The cluster gas and gravitational masses were derived using the
deprojected radial electron density and temperature profiles presented
in Section \ref{sec:rad}. Electron gas density, $\nelec$, was
converted to total gas density as $n_g = 1.92 \nelec \mu \mH$ where
\mH\ [g] is the mass of hydrogen. The gas density profile was fitted
with a $\beta$-model \citep{betamodel}, and the temperature profile
was fitted with the 3D-$T(r)$ model of \citet{2006ApJ...640..691V} to
ensure continuity and smoothness of the radial log-space derivatives
when solving the hydrostatic equilibrium equation. Total gas mass was
calculated by assuming spherical symmetry and integrating the best-fit
$\beta$-model out to \rt, giving $\mgas(r<\rt) = 7.99 ~(\pm 0.65)
\times 10^{13} ~\msol$. The gravitating mass was derived by solving
the hydrostatic equilibrium equation using the analytic density and
temperature profiles. We calculate $\mgrav(r<\rt) = 7.22 ~(\pm 1.44)
\times 10^{14} ~\msol$, giving a ratio of gas mass to gravitating mass
of $0.11 \pm 0.02$. The gas and gravitating mass errors were estimated
from 10,000 Monte Carlo realizations of the measured density and
temperature profiles and their associated uncertainties.

In terms of the galaxy cluster population, \rxj\ resides toward the
high-end of the mass distribution with a luminosity-temperature ratio
and gas fraction consistent with flux-limited and representative
cluster samples \citep{hiflugcs2, 2009A&A...498..361P}. Adjusted for
differences in assumed cosmology, our global measurements agree with
prior studies of \irs\ \citep[\eg][]{2000MNRAS.315..269A}. With the
exception of the strange BCG at its heart, \rxj\ appears to be a
typical massive, relaxed galaxy cluster. None of the integrated X-ray
cluster properties suggest the system has undergone a recent major
merger or cluster-scale AGN outburst which may have dramatically
disrupted the ICM. The lack of a detected radio halo also suggests no
recent merger activity, previous powerful AGN outbursts, and possibly
no turbulent motions in the core \citep[\eg][]{2008SSRv..134...93F}.

%%%%%%%%%%%%%%%%%%%%%%%%%%%%%%%
\section{Radial ICM Properties}
\label{sec:rad}
%%%%%%%%%%%%%%%%%%%%%%%%%%%%%%%

Now we discuss the finer global structure of \rxj\ via radial ICM
properties. Consistent with the analysis of Section \ref{sec:global},
the BCG nucleus was excluded from all radial analysis. Temperature
(\tx) and abundance ($Z$) profiles were created using circular annuli
centered on the cluster X-ray peak and containing 2.5K and 5K source
counts per annulus, respectively. A deprojected temperature profile
was generated using the \textsc{deproj} method in \xspec. We use the
projected profile in all analysis as it does not significantly differ
from the deprojected profile. Spectra were grouped to 25 source counts
per energy channel. \caldb\ blank-sky backgrounds were reprocessed and
reprojected to match each observation, and then normalized for
variations of the hard-particle background using the ratio of
blank-sky and observation 9.5-12 keV count rates. Following the method
outlined in \citet{2005ApJ...628..655V}, a fixed background component
was included during spectral analysis to account for the
spatially-varying Galactic foreground \citep[see][for more
  detail]{xrayband}. The temperature and abundance profiles are shown
in the top row of Figure \ref{fig:gallery}. After masking out all
X-ray substructure (see Section \ref{sec:excess}) and the central
$2\arcs$, a grouped spectrum for the central 20 kpc was fitted with a
thermal model plus a cooling flow component. The best-fit model had a
mass deposition rate of $\mdot = 206^{+87}_{-65} ~\msol$ for upper and
lower temperatures of 5.43 keV and 0.65 keV, respectively, with
abundance $0.51 ~\Zsol$.

A surface brightness (SB) profile was extracted using concentric
$1\arcs$ wide circular annuli centered on the cluster X-ray peak. From
the SB and temperature profiles, a deprojected electron density
(\nelec) profile was derived using the \citet{kriss83} technique
\citep[see][for more detail]{accept}. Errors for the density profile
were estimated from 10,000 Monte Carlo bootstrap resamplings of the SB
profile. The SB and electron gas density profiles are shown in the
second row of Figure \ref{fig:gallery}.

Total gas pressure ($P = 2.4 \tx \nelec$), entropy ($K =
\tx\nelec^{-2/3}$), cooling time ($\tcool = 3n\tx~[2\nelec \nH
  \Lambda(T,Z)]^{-1}$), and enclosed X-ray luminosity (\lx) profiles
were also created. These profiles are presented in the bottom two rows
of Figure \ref{fig:gallery}. Uncertainties for each profile were
calculated by propagating the individual parameter errors and then
summing in quadrature. The cooling functions, $\Lambda(T,Z)$, used to
calculate cooling times were derived from the best-fit spectral model
for each annulus of the temperature profile and interpolated onto the
grid of the higher resolution density profile. The function $K(r) =
\kna +\khun (r/100 ~\kpc)^{\alpha}$ was fitted to the entropy profile,
giving best-fit values of $\kna = 12.6 \pm 2.9 ~\ent$, $\khun = 139
\pm 8 ~\ent$, and $\alpha = 1.71 \pm 0.10$.

The \rxj\ ICM structure is typical of the cool core class of galaxy
clusters, with a temperature profile that rises with increasing radius
and an entropy profile with a relatively small, flattened core. There
are no resolved discontinuities in the \tx, \nelec, or $P$ profiles to
suggest the presence of a shock or cold front. Additional 2D analysis
using the weighted Voronoi tesselation and contour binning methods of
\citet{wvt} and \citet{2006MNRAS.371..829S}, respectively, also did
not reveal any significant temperature or abundance substructure. The
entropy profile is consistent with the cool core population as a whole
\citep{accept}, and, in particular, with the population of $\kna < 30
~\ent$ clusters that have radio-loud AGN and star formation in the BCG
\citep{haradent, rafferty08}. The $\kna \la 30 ~\ent$ scale also
defines an entropy regime in which thermal electron conduction in
cluster cores is too inefficient to suppress widespread environmental
cooling \citep{conduction}. Therefore, cooling subsystems, like gas
ram pressure stripped from cluster members or ICM thermal
instabilities, should be long-lived. There is an abundance of cool,
gaseous substructure surrounding \irs, and in Section \ref{sec:excess}
we discuss the relation of this structure to the AGN and QSO.

%%%%%%%%%%%%%%%%%%%%%%%%%%%
\section{ICM Cavity System}
\label{sec:cavs}
%%%%%%%%%%%%%%%%%%%%%%%%%%%

To aid investigation of ICM substructure, a residual X-ray image was
created by subtracting a SB model for the ICM from the mosaiced
\chandra\ clean image. The \chandra\ clean image was binned by a
factor of 2 and the SB isophotes were fitted using the \iraf\ tool
\textsc{ellipse}. The geometric parameters ellipticity ($\epsilon$),
position angle ($\phi$), and centroid ($C$) were initially free to
vary, but the best-fit values for each isophote converged to mean
values of $\epsilon = 0.14$, $\phi = -76\mydeg$, and $C$ [J2000] =
(09:13:45.5; +40:56:28.4). These values were fixed in the fitting
routine to eliminate the isophotal twisting resulting from statistical
variation of the best-fit values for each radial step. A SB model was
constructed from the best-fit model and subtracted from the clean
image. The resulting residual image is shown in Figure
\ref{fig:resid}.

The faint SB decrements NW and SE of the nucleus in the clean image
are resolved into cylindrical voids in the residual image. The void
and radio jet morphologies closely trace each other, confirming they
share a common origin in the AGN outburst. Cavities are a well-known
phenomenon, but currently, \irs\ is the highest redshift object where
cavities have been directly imaged. In addition, \irs\ is thus far the
only example of a QSO-dominated system with an unambiguous cavity
detection. Using a 1994 \rosat\ HRI observation,
\citet{1995MNRAS.274L..63F} found a ``hole'' in the core of
\rxj\ which they attributed to absorption by a $> 1000 ~\msolpy$
cooling flow. When juxtaposed with the \chandra\ residual image, the
``hole'' is clearly not associated with the real cavities, which are
not resolved in the 1994 HRI. Neither the cavities nor the ``hole''
are seen in a longer 1995 HRI follow-up observation.

The AGN outburst energetics were investigated using properties of the
cavities \citep[see][for a review]{mcnamrev}. Cavity volumes, $V$,
were calculated by approximating each void in the X-ray image with a
right circular cylinder projected onto the plane of the sky along the
cylinder radial axis. The lengthwise axis of the cylinders were
assumed to lie in a plane perpendicular to the line of sight that
passes through the central AGN. The energy in each cavity, $\ecav =
\gamma PV/(\gamma-1)$, was estimated by assuming the contents are a
relativistic plasma ($\gamma = 4/3$), and then integrating the total
gas pressure, $P$, over the surface of each cylinder. The radio source
morphology, spectrum, and age suggest the jets were recently, or still
are, being fed by the central AGN. Thus, we assumed the cavities were
created on a timescale dictated by the ambient gas sound speed,
\tsonic\ \citep[see][]{birzan04}, and the distance the AGN outflow has
traveled to create each cavity was set to the cylinder length, not the
midpoints, as is common. The power of each cavity is thus $\pcav =
\ecav/\tsonic$. Cavity power is often assumed to be a good estimate of
the physical quantity jet power, \pjet, but note that neither accounts
for energy which may be imparted to shocks. Properties of the
individual cavities are listed in Table \ref{tab:cylcavities}.

The total cavity energy and power are estimated at $\ecav = 5.11 ~(\pm
1.33) \times 10^{59}$ erg and $\pcav = 3.05 ~(\pm 1.03) \times 10^{44}
~\lum$, respectively, with a mean $\pcav = 1.52 ~(\pm 0.11) \times
10^{44} ~\lum$. Radio power has been shown to be a reasonable
surrogate for estimating mean jet power \citep{birzan08}. Thus, we
checked the \pcav\ calculation using the Cavagnolo et al. (in
preparation) \pjet-\prad\ 1.4 GHz and 200-400 MHz scaling
relations. The relations give $\pjet \approx 2 \dash 6 \times 10^{44}
~\lum$, in agreement with the X-ray measurements. Compared with other
systems hosting cavities, \irs\ resides between the middle and
upper-end of the cavity power distribution. The AGN outburst is
powerful, but there is nothing unusual about the energetics or the
radiative efficiency (\prad/\pjet) given the cluster mass and ICM
properties.

Of interest is how the AGN energetics compare to the cooling rate of
the host X-ray halo. The cooling radius was set at the radius where
the ICM cooling time is equal to $H_0^{-1}$ at the redshift of
\irs. We calculate $R_{\mathrm{cool}} = 128$ kpc, and measure an
unabsorbed bolometric luminosity within this radius of
$L_{\mathrm{cool}} = 1.61^{+0.25}_{-0.20} \times 10^{45} ~\lum$. If
all of the cavity energy is thermalized over $4\pi$ sr, then $\approx
20\%$ of the energy radiated away by gas within $R_{\mathrm{cool}}$ is
replaced by energy coming from the AGN. Assuming the mean ICM cooling
rate does not vary significantly on a timescale of $\sim 1$ Gyr, this
highly optimistic scenario implies that 5 similar power AGN outbursts
will significantly suppress cooling of the cluster halo.

Our calculations neglect the influence of shocks, but the synchrotron
age and cavity age are useful in addressing this issue. If \tsync\ is
an accurate measure of the radio source age, then the age discrepency
$\tsonic \ga 42$ Myr versus $\tsync \la 30$ Myr implies the AGN
outflow is supersonic, otherwise the radio-loud plasma will radiate
away all its energy and be radio-silent prior to reaching the end of
the observed jet. The implication being that some amount of energy may
have gone into gas shocking. Recall, however, that no shocks are
detected in the X-ray analysis, and note that the properties of
\irs\ nebulae are inconsistent with excitation due to shocks
\citep{1996MNRAS.283.1003C, 2000AJ....120..562T}. But, the nebular
regions studied are $\ga 20$ kpc from the jet axis, and may not be
indicative of gas dynamics close to the outflow. Regardless, the
energy in shocks was crudely estimated by setting $t_{\mathrm{sonic}}
= 20$ Myr and adjusting \pcav\ by the Mach number: $\Delta \pcav =
\Delta P / \Delta t$ and $\Delta P \propto M^3$. Relative to the ICM
sound speed, the velocity needed to reach the end of the radio jet in
20 Myrs requires a Mach number of $M \approx 2.5$, which brings the
outburst power up to $\approx 1 \times 10^{46} ~\lum$ ($\ecav \sim
10^{61}$ erg for a 20 Myr duration). Within the formal uncertainties,
the AGN outburst power is on the order of a few times $10^{44} ~\lum$,
with the possibility of being as large as $10^{46} ~\lum$.

%%%%%%%%%%%%%%%%%%%%%%%%%%%%%%%%%%
\section{Fueling the AGN Outburst}
\label{sec:fuel}
%%%%%%%%%%%%%%%%%%%%%%%%%%%%%%%%%%

An estimate of black hole mass, \mbh, is key to investigating what
powers an AGN outburst. There are a variety of \mbh\ estimators, many
of which rely on infrared or optical emission line measurements. The
integrated IR and emission line properties of \irs, however, are
dominated by hot dust and complex nebulae, and may not be
representative of \mbh. Therefore, \mbh\ was constrained using a
variety of relations and the weighted mean was used in subsequent
calculations. Using a stellar velocity dispersion of $\sigma_s = 293
\pm 6 ~\kmps$, determined from the \citet{1976ApJ...204..668F}
relation and the corrected $B$-band magnitude from HyperLeda
\citep{hyperleda}, the \citet{2002ApJ...574..740T} relation gives
$\mbh = 0.63 ~(\pm 0.05) \times 10^9 ~\msol$. The
\citet{2007MNRAS.379..711G} relations relating absolute $[B,R,K]$-band
magnitudes to \mbh\ predict $\mbh = 0.5 \dash 5.0 \times 10^9
~\msol$. The weighted mean \mbh\ value we adopt is $1.05 ~(\pm 0.17)
\times 10^9 ~\msol$.

Some of the gravitational binding energy of the material accreting
onto the SMBH is transported outward via relativistic jets. Assuming
this conversion has an efficiency $\epsilon$, the energy deposited in
cavities by the jets implies an accretion mass expressed as $\macc =
\ecav/(\epsilon c^2)$ with a time-averaged mass accretion rate of
$\dmacc = \macc/t_{\mathrm{sonic}}$. Setting $\epsilon = 0.1$, the AGN
outburst resulted from the accretion of $2.86 ~(\pm 0.75) \times
10^{6} ~\msol$ of matter at a rate of $0.054 \pm 0.004 ~\msolpy$. If
the accretion flow feeding the SMBH is spherically symmetric, it can
be characterized in terms of the Eddington (Eqn. \ref{eqn:edd}) and
Bondi (Eqn. \ref{eqn:bon}) accretion rates,
\begin{eqnarray}
  \dmedd &=& \frac{2.2}{\epsilon} \left(\frac{\mbh}{10^9~\msol}\right)
  ~\msolpy  \label{eqn:edd}\\
  \dmbon &=& 0.013 ~K_{\mathrm{Bon}}^{-3/2} \left(\frac{\mbh}{10^9
    ~\msol}\right)^{2} ~\msolpy \label{eqn:bon}
\end{eqnarray}
where $K_{\mathrm{Bon}}$ [\ent] is the mean entropy of gas within the
Bondi radius. The Eddington rate defines the maximal inflow rate of
gas not expelled by radiation pressure, as where the Bondi rate
approximates the quantity of hot, ambient gas captured by the
SMBH. Assuming $K_{\mathrm{Bon}} = \kna$, the derived \mbh\ gives
$\dmedd \approx 23 ~\msolpy$ and $\dmbon \approx 3.2 \times 10^{-4}
~\msolpy$. Thus, the Eddington and Bondi ratios for the accretion
event which powered the AGN outburst are $\dmacc/\dmedd \approx 0.002$
and $\dmacc/\dmbon \approx 300$.

The Bondi radius for \irs\ is unresolved ($\rbon = 9$ pc), and
$K_{\mathrm{Bon}}$ is likely less than \kna. But, for a Bondi ratio of
at least unity, $K_{\mathrm{Bon}}$ must be $\la 0.36 ~\ent$, lower
than is measured for even galactic coronae \citep{coronae}. In terms
of entropy, $\tcool \propto K^{3/2} ~\tx^{-1}$ \citep{d06}. Assuming
gas close to \rbon\ is no cooler than 0.3 keV, the accreting material
will have $\tcool \la 10$ Myr, a factor of 100 below the shortest ICM
cooling time and 1/5 the free-fall time in the core. But this creates
the problem that gas close to \rbon\ is disconnected from cooling at
larger radii, breaking the feedback loop
\citep{2006NewA...12...38S}. If instead cold-mode accretion dominates,
then the gas which becomes fuel for the AGN is distributed in the BCG
halo and falls into the BCG as a result of cooling \citep{pizzolato05,
  2010arXiv1003.4181P}. Indeed, radial filaments and gaseous
substructure within 30 kpc of \irs\ are seen down to the
resolution-limit of \hst\ \citep{1999Ap&SS.266..113A}. This may
indicate the presence of cooling, overdense regions similar to the
cold blobs expected in cold-mode accretion. Though Bondi accretion
cannot be ruled out, it does not seem viable, and the process of
cold-mode accretion appears to be more consistent with the nature of
\irs.

%%%%%%%%%%%%%%%%%%%%%%%%%%%%%%%%%%%%
\section{QSO Irradiation of the ICM}
\label{sec:excess}
%%%%%%%%%%%%%%%%%%%%%%%%%%%%%%%%%%%%

In Figure \ref{fig:resid}, three regions of X-ray emission in excess
of the best-fit SB model are highlighted. The regions are illustrative
and approximate the constant SB contours used to define the spectral
extraction regions. Each region is denoted by its location relative to
the nucleus: northern excess (NEx), eastern excess (EEx), and western
excess (WEx). The NEx and WEx appear to be part of a tenuous, arc-like
filament which may be gas displaced by the NW radio jet. In other
clusters, structures similar to the NEx-WEx are found to be cool rims
of gas \citep[\eg][]{2009ApJ...697L..95B}, but the \chandra\ data is
insufficient to determine if this is the case for \irs, hence we treat
the NEx and WEx as separate structures.

Spectral analysis was performed on each region. A background spectrum
was extracted from regions neighboring the excesses which did not show
enhanced emission in the residual image. The backgrounds were scaled
to correct for differences in sky area. For each region, the ungrouped
source and background spectra were differenced in \xspec\ to create a
residual spectrum. To avoid systematically cooler best-fit
temperatures resulting from low count rates
\citep{1989ApJ...342.1207N}, the modified Cash statistic
\citep{1979ApJ...228..939C} was used during fitting. The low
signal-to-noise ratio (SN) of each spectrum precluded setting metal
abundance as a free parameter when fitting a thermal model. Since the
three excesses reside within the two central annuli of the abundance
profile, the abundance was fixed at $0.51 ~\Zsol$. The best-fit values
for the spectral models are given in Table \ref{tab:excess}.

The NEx residual spectrum had low-SN which resulted in poor resolution
of spectral features. Thus, the thermal model had an unconstrained
temperature of $\sim 7$ keV, and similarly the power-law model had an
unconstrained spectral index of $\Gamma \sim 1.9$. The northern radio
jet terminates in the NEx region, and the hardness ratio map (see
Section \ref{sec:centsrc} and Figure \ref{fig:resid}) shows a possible
hot spot in this same area. The NEx may result from non-thermal
emission in the hot spot, but we cannot confirm this
spectroscopically.

The WEx has a residual spectrum consistent with thermal emission, but
the EEx spectrum has prominent features at $E < 2$ keV which were
poorly fit by a single-component thermal model. The EEx thermal
\feka\ complex was also poorly fit because of an obvious asymmetry
toward lower energies. To reconcile the poor fit, three Gaussians were
added to the EEx model. Comparison of fit statistics and goodness of
fits determined from 10,000 Monte Carlo simulations of the best-fit
spectra suggest the model with the Gaussians is preferred. The
strength of the features relative to the continuum suggest a thermal
origin is unlikely, and that the features may be emission line
blends. Below we discuss the EEx exclusively.

H93 and \citet[][hereafter H99]{1999ApJ...512..145H} suggest the AGN
which produced the large-scale jets has been reoriented within the
last few Myrs, resulting in a new beaming direction close to the line
of sight and at roughly a right angle to the previous beaming
axis. Interestingly, the new AGN axis suggested by H99 is coincident
with the EEx, the radio spur northeast of the radio core, a cone of UV
ionization, an ionized optical nebula, and highly polarized diffuse
optical emission. These respective features are outlined in Figure
\ref{fig:resid}. \citet{2010MNRAS.402.1561R} demonstrate that the QSO
in H1821+643, which is 2 times more luminous than \irs, is capable of
photoionizing gas up to 30 kpc from the nucleus, and we suspect a
similar process may be occurring in \irs.

To test this hypothesis, reflection and diffuse spectra were simulated
for the nebula and ICM coincident with the EEx using
\cloudy\ \citep{cloudy}. The nebular gas density and ionization state
were taken from \citet{2000AJ....120..562T}, while the initial ICM
temperature, density, and abundance were set at 3 keV, 0.04 \pcc, \&
0.51 \Zsol, respectively. No Ca or Fe lines are detected from the
nebula coincident with the EEx, but strong Mg, Ne, and O lines are
\citep{2000AJ....120..562T}, possibly as a result of metal depletion
onto dust grains \citep[\eg][]{1993ApJ...414L..17D}. Thus, a metal
depleted, grain-rich, 12 kpc thick nebular slab was placed 15 kpc from
an attenuated $\Gamma = 1.7$ power law source with power $1 \times
10^{47} ~\lum$. Likewise, a $17 ~\kpc \times 16 ~\kpc$ ICM slab was
placed 19 kpc from the same source. The QSO radiation was attenuated
using a 15 kpc column of density 0.06 \pcc, abundance 0.51 \Zsol, and
temperature 3 keV. The output models were summed, folded through the
\chandra\ responses using \xspec, and fitted to the observed EEx
spectrum (shown in Figure \ref{fig:qso}).

In the energy range 0.1-10.0 keV, the nebula emission lines which
exceed the thermal line emission originate from Si, Cl, O, F, K, Ne,
Co, Na, \& Fe and occur as blends around redshifted 0.4, 0.6, 0.9, \&
1.6 keV. The energies and strengths of these blends are in good
agreement with the EEx spectrum. Further, the \feka\ emission from the
nebula is 100 times fainter than that from the ICM, and the observed
asymmetry of the EEx \feka\ emission results from the 6.4 keV
\feka\ photoionized line of the ICM. It is clear that beamed QSO
radiation is responsible for the nature of the EEx, but if the
photoionization equates to heating of the gas is unclear. Nonetheless,
we have demonstrated that QSO radiation is capable of impacting cool
gas over a large angle which is far from the nucleus.

One alternative explanation for the EEx is that it is low entropy gas
uplifted from deeper within the core along the new AGN beaming
axis. Scattered UV emission 32 kpc from the core places a minimum
lifetime of 73 kyr for the new beaming direction (H93), and, assuming
saturated heat flux across the EEx surface, the evaporation time is
exceedingly short, $< 1$ Myr. These short timescales suggest the EEx
was transported to the present location at $> 20$ times the ambient
sound speed, or $v \sim 0.06 \dash 0.1c$, well below typical jet bulk
flow velocities. The radio spur has a 1.4 GHz luminosity of $\approx 3
\times 10^{39} ~\lum$, suggesting an associated 1 Myr old jet would
have $\sim 10^{57}$ erg of kinetic energy, which is sufficient to lift
$\sim 10^{10} ~\msol$ to a distance of 19 kpc. It appears uplift is
feasible, particularly if the gas is magnetically-shielded and
conduction is staved-off.

%%%%%%%%%%%%%%%%%%%%%%%%%%%%%%%%
\section{Nucleus X-ray Emission}
\label{sec:centsrc}
%%%%%%%%%%%%%%%%%%%%%%%%%%%%%%%%

The centroid and extent of the nuclear X-ray source were determined
using the \ciao\ tool {\tt wavdetect} and the \chandra\ PSF. Each was
confirmed with a hardness ratio map calculated as $HR = f(2.0 \dash
9.0 ~\keV) / f(0.5 \dash 2.0 ~\keV)$, where $f$ is the flux in the
denoted energy band. A source extraction region was defined using the
90\% enclosed energy fraction (EEF) of the normalized \chandra\ PSF
specific to the nuclear source median photon energy and off-axis
position. The elliptical source region had an effective radius of
$1.16\arcs$. A segmented elliptical annulus with the same central
coordinates, ellipticity, and position angle as the source region, but
having 5 times the area, was used for the background region. The
background annulus was broken into segments to avoid the regions of
excess X-ray emission discussed in Section \ref{sec:excess}. The $HR$
map and extraction regions are shown in Figure \ref{fig:nucleus}.

Source and background spectra were created using the \ciao\ tool {\tt
  psextract}. The source spectrum was grouped to have 20 counts per
energy channel. The background-subtracted \chandra\ spectra and
best-fit models are presented in Figure \ref{fig:nucleus}. The
significant flux difference below 1.3 keV is a result of the greater
effective area of the ACIS-S3 CCD in 1999 versus ACIS-I3 in
2009. Approximately 72\% of the 2009 spectrum (hereafter, SP09) is
from the source, with a count rate of $1.63 ~(\pm 0.06) \times
10^{-2}$ ct \ps\ in the 0.5-9.0 keV band. For the 1999 spectrum
(hereafter, SP99), 67\% is source flux, with a 0.5-9.0 keV count rate
of $2.71 ~(\pm 0.26) \times 10^{-2}$ ct \ps. We confirm the findings
of \citet{2001MNRAS.321L..15I} that the ICM thermal
\feka\ contribution to the nuclear spectrum is negligible, and that
prominent, blended, line-like features around 0.8 keV and 1.3 keV are
superposed on the continuum.

Previous studies have shown the nuclear spectrum is best modeled as
Compton reflection from cold matter with a strong \feka\ fluorescence
line ($E_{\rm{rest}} = 6.4$ keV). The SP99 and SP09 were fitted
separately in \xspec\ over the energy range 0.5-7.0 keV with an
absorbed \pexrav\ model \citep{pexrav} plus three Gaussians. The
disk-reflection geometry employed in the \pexrav\ model is not ideal
for fitting reflection from a Compton-thick torus
\citep{2009MNRAS.397.1549M}, but no other suitable \xspec\ model is
currently available. Hence, only the \pexrav\ reflection component was
fitted and no high energy cut-off for the power law was used. Fitting
separate SP99 and SP09 models allowed for source variation in the
decade between observations, however $\Gamma$ was poorly constrained
for SP99 and thus fixed at the SP09 value. Using constraints from
\citet{2000AJ....120..562T}, the model parameters for reflector
abundance and source inclination were fixed at $1.0 ~\Zsol$ and $i =
50\mydeg$, respectively. Setting abundance as a free parameter did not
statistically improve the fits. The best-fit model parameters are
presented in Table \ref{tab:nucspec}.

Using a solar abundance thermal component in place of the two
low-energy Gaussians yielded a statistically worse fit. The model
systematically underestimated the 1-1.5 keV flux and overestimated the
2-4 keV flux. Leaving the thermal component abundance as a free
parameter resulted in $0.1 ~\Zsol$, \ie\ the thermal component tended
toward a featureless skewed-Gaussian. Strong Mg, Ne, S, and Si
K$\alpha$ fluorescence lines at $E < 3.0$ keV can be present in
reflection spectra \citep{1991MNRAS.249..352G}, as can Fe L-shell
lines from photoionized gas \citep{1990ApJ...362...90B}. We conclude
that the soft X-ray emission modeled using the Gaussians is likely a
combination of emission line blends and low-level thermal continuum,
whether the thermal component is nuclear or ambient in origin is
unclear.

The unabsorbed 2-10 keV {\it{reflected}} flux is $4.24^{+0.57}_{-0.55}
\times 10^{-13} ~\flux$ corresponding to a rest-frame $L_{2-10} =
1.57^{+0.19}_{-0.19} \times 10^{44} ~\lum$. Adjusted for cosmology,
this agrees with the measurement from
\citet{2001MNRAS.321L..15I}. Since we have used a pure reflection
model, the intrinsic QSO luminosity can only be estimated as
$(\kappa/\eta) L_{2-10}$ where $\kappa = 40$ is a bolometric
correction factor \citep{2007MNRAS.381.1235V} and $\eta = 0.06$ is the
reflector albedo \citep{2009MNRAS.397.1549M}. This gives $\lqso = 1.05
~(\pm 0.13) \times 10^{47} ~\lum$.

As a check for acceptable agreement between our models and results of
prior studies, the \rf\ \chandra, \xmm, and \bepposax\ spectra were
jointly fitted with our best-fit nucleus and ICM models
simultaneously. There were no significant differences between our
models and those of \citet{2000A&A...353..910F},
\citet{2001MNRAS.321L..15I}, and the \citet{2007A&A...473...85P}
reflection model. The value for the \feka\ equivalent width (\fekaew),
which is a valuable diagnostic for probing the environment of an AGN
\citep[see][for a review]{2000PASP..112.1145F}, also agrees with
previous measurements which found $\fekaew \la 1$ keV. The large
uncertainties associated with the individual SP99, \xmm, and
\bepposax\ \fekaew\ values prevents us from determining if
\fekaew\ has varied since 1998.

Our results are consistent with models and observations which show
that $\fekaew \ga 0.5$ keV is correlated with $\Gamma \ga 1.7$ and
reflecting column densities $\nhref \sim 10^{24} ~\pcmsq$
\citep{1996MNRAS.280..823M, 1997ApJ...477..602N, 1999MNRAS.303L..11Z,
  2005A&A...444..119G}. Previous studies suggested the \bepposax\ PDS
detection resulted primarily from transmission of hard X-rays through
an obscuring screen with $\nhobs > 10^{24} ~\pcmsq$. Extrapolating our
best-fit model out to 10-80 keV reveals statistically acceptable
agreement with the PDS data (see Figure \ref{fig:resid}). The 10-200
keV model flux is $f_{10 \dash 200} = 8.15^{+0.21}_{-0.19} \times
10^{-12} ~\flux$, which is not significantly different from the $f_{10
  \dash 200}$ measured with \bepposax. Addition of a second power-law
component ($\Gamma = 1.7$) absorbed by a $\nhobs = 3 \times 10^{24}
~\pcmsq$ screen at the \irs\ redshift to the model lowered \chisq\ but
with no statistical improvement to the fit. If transmitted hard X-ray
emission falls within the passband used for spectral analysis,
$\Gamma$ would be artificially lowered, and the extrapolated hard
X-ray flux thus increases. However, for $\Gamma \ge 1.7$, column
densities $> 3 \times 10^{24} ~\pcmsq$ are sufficient to suppress
significant transmitted emission below the 7 keV spectral analysis
cut-off, indicating the best-fit model should not have an artificially
low $\Gamma$.

That we find no need for an additional hard X-ray component does not
contradict the well-founded conclusion that \irs\ harbors a
Compton-thick QSO. On the contrary, the measured \fekaew\ suggests
reflecting column densities of $\nhref \sim 1 \dash 5 \times 10^{24}
~\pcmsq$ \citep{1993MNRAS.263..314L, 2005A&A...444..119G,
  2010arXiv1005.3253C}. Assuming the density of material surrounding
the QSO is mostly homogeneous, \ie\ $\nhref \approx \nhobs$, our
results are consistent with the presence of a moderately Compton-thick
screen.

%%%%%%%%%%%%%%%%%%%%%%%%%%%%%%%%%%%%%%%%%
\section{Evolution of the feedback mode?}
\label{sec:evo}
%%%%%%%%%%%%%%%%%%%%%%%%%%%%%%%%%%%%%%%%%

Cosmological simulations typically segregate radiatively- and
mechanically-dominated feedback into a distinct early-time quasar-mode
\citep[\eg][]{2005Natur.435..629S} and a late-time radio-mode
\citep[\eg][]{croton06}, respectively. These modes of feedback are
used to ensure that SMBH-host galaxy co-evolution is in accordance
with observations \citep[\eg][]{magorrian}. The quasar-mode is
expected to be brief, expelling large quantities of cold gas from the
host galaxy \citep{2006ApJ...642L.107N}; whereas the radio-mode is
prolonged \& intermittent, heating the extended hot halo such that
future cooling is regulated \citep[see][for a review]{mcnamrev}. There
are numerous examples of systems dominated by mechanical feedback
\citep[\eg][]{perseus1, ms0735}, and there are indications that many
high-redshift galaxies are dominated by QSO feedback \citep[see][for a
  review]{2005ARA&A..43..769V}, but in a unified feedback model, there
must be a transition from the dominance of one mode to the other, and
we suspect \irs\ is one such example. But aside from the obvious AGN
mechanical feedback and irradiation of the ICM by the QSO, are there
other indications of quasar-mode feedback in the host galaxy which
would strengthen the case?

The H$_2$ mass of \irs\ is $< 10^{10} ~\msol$
\citep{1998ApJ...506..205E}, there is $< 10^8 ~\msol$ of cold dust
\citep{2001MNRAS.326.1467D}, no polycyclic aromatic hydrocarbon or
silicate absorption features are detected \citep{2004ApJ...613..986P,
  2008ApJ...683..114S}, and the hot dust mass is $\sim 10^9 ~\msol$
\citep{1997A&A...318L...1T}. Additionally, the \halpha\ luminosity
exceeds $10^{42} ~\lum$ \citep{1996MNRAS.283.1003C,
  1998ApJ...506..205E}. Based on these measurements, and compared with
other BCGs \citep[\eg][]{2001MNRAS.328..762E}, \irs\ appears to be
gas-poor with a low gas-to-dust ratio. All told, the lack of strong
molecular gas indicators around such a powerful QSO is odd. One
possible explanation for the discrepancy is that the QSO is driving
gas out of the galaxy via non-relativistic winds, radiation pressure,
or a combination of both \citep[\eg][]{2010MNRAS.401....7H}. Indeed,
integral field spectroscopy indicates the presence of a $> 1000
~\kmps$ emission line outflow coincident with the nucleus
\citep{1996MNRAS.283.1003C}. Further, the CO observations used to
infer the H$_2$ mass \citep{1998ApJ...506..205E} were not sensitive to
velocities $> 300 ~\kmps$, the regime where molecular gas being
expelled at high-velocities ($\sim 1000 \kmps$) by QSO winds might be
detected.  Given that rapid \& extensive dust formation is also
expected in such QSO winds \citep{2002ApJ...567L.107E}, this may
further explain the extreme dust richness of \irs.

\citet[][hereafter F09]{2009MNRAS.394L..89F} show that QSO/AGN
radiation pressure has a significant influence on dusty material in
the host galaxy. F09 define the effective Eddington ratio for dusty
gas to be $\leff = \lqso (1.38 \times 10^{38} ~\mbh)^{-1}$, and for
\irs, this has a value of $\approx 0.72$. F09 also present a plane for
\nhobs-\leff\ which is divided into regions where obscuring clouds are
either long-lived or experience the effects of a super-Eddington AGN,
\ie\ where clouds are efficiently expelled. In this plane,
\irs\ resides near the boundary of the two regions, close enough that
it is reasonable to suspect that the massive reservoir of dust-laden
gas in the galaxy is being heated, ionized, or accelerated away from
the QSO by radiation pressure. These conclusions are, however, at the
mercy of our choice for \mbh, \ie\ if $\mbh \ge 5 \times 10^9 ~\msol$,
then $\leff < 0.2$, as where $\mbh < 1 \times 10^9 ~\msol$ implies
$\leff > 1$.

All three of the primary channels for QSO/AGN feedback to influence
its environment (jets, winds, radiation) are conceivably active in
\irs, and it may be that these processes are simultaneously conspiring
to quench cooling within and around the host galaxy. But are we really
witnessing the evolution from one dominant form of feedback to
another? The mass accretion rate required to power the QSO is $\dmacc
= \lqso/(0.1 c^2) \approx 20 ~\msolpy$, 300 times larger than
\dmacc\ needed to power the jets. At this rate, the black hole mass
will double in $\approx 50$ Myr, and for the Magorrian relation to
hold, $> 10^{11} ~\msol$ of stars would need to form, $> 10\%$ of the
current bulge mass. This seems unlikely, and thus the current period
of QSO activity should be fleeting, giving way to a sustained period
of sub-Eddington accretion which can readily power an AGN but not its
hulking brother the QSO. The emergence of jets may have signaled this
change. Further, H93 and H99 discuss in detail that the misalignment
between the large-scale radio jets and beamed nuclear radiation may be
correlated with evolution of the radio source from a \frii\ to \fri,
and that the jet axis realignment must have transpired in less than a
few Myrs. The origin of the misalignment cannot be determined with any
certainty, but one can speculate.

In the following discussion we assume the axes of an AGN jet and SMBH
spin are one in the same. One simple explanation is that there are
multiple SMBHs in the nucleus, each with their own accretion system
but isotropic spin axes. Hence there is no misalignment, we are seeing
emission from two seperate systems: one turning-on, one
turning-off. Another explanation is that a single SMBH has undergone a
``spin-flip'' \citep{2002Sci...297.1310M} when a smaller black hole
merged with it. BCGs are known for cannibalism, and 6 suspected
companion galaxies reside within a projected 80 kpc of the BCG
\citep{1996AJ....111..649S, 1999Ap&SS.266..113A}, so it is not out of
the question that one or more mergers have taken place in the last few
Myrs. However, black hole mergers are lengthy (few Gyrs) and difficult
processes, and the spin axes of the merging black holes may naturally
align when in a gas-rich environment \citep{2007ApJ...661L.147B}.

Based on the spin evolution framework of \citet{2010arXiv1004.1166G},
there is yet another intriguing explanation which does not
specifically require mergers to alter the SMBH spin
axis. \citet{2010arXiv1004.1166G} suggest that evolution of a black
hole spin state from retrograde to prograde relative to accreting
matter is specifically correlated with a transition of the radio
source from a powerful \frii\ to a low-power \fri. During the process
of retrograde spin-down, the black hole must pass through a state
where the spin is $\approx 0$. At this point, if there is an
asymmetric accretion flow exceeding a \mbh-dependent critical \dmacc,
the spin axis can be dramatically reoriented on timescales of a few
Myrs (Cavagnolo et al., in preparation), possibly giving rise to the
type of jet-beamed radiation misalignment observed in \irs. This
process may also accompany quenching of the cold gas reservoir in the
host galaxy, hastening the transition from quasar-dominated to
radio-dominated feedback.

%%%%%%%%%%%%%%%%%
\section{Summary}
\label{sec:summ}
%%%%%%%%%%%%%%%%%

In this paper we have shown, through a new \chandra\ X-ray
observation, that the AGN/QSO in \iras\ is interacting with the ICM of
\rxj\ through both the mechanical and raditive feedback channels. The
results presented in this paper are as follows:
\begin{itemize}
\item The \rxj\ ICM global and radial properties reveal no signs of
  shocks, cold fronts, a radio halo, or deviations from hydrostatic
  equilibrium to suggest disruption by a major merger or prior
  cluster-scale AGN outburst. \rxj\ is an unremarkable, massive,
  cool-core galaxy cluster with a 12 \ent\ core entropy, mass that
  scales as $\cltx^{3/2}$, and a mean cluster gas fraction of 0.11.
\item We have discovered cavities in the X-ray halo of \irs\ which
  indicate an AGN outburst with total mechanical power of at least $3
  \times 10^{44} ~\lum$ and total energy output of $6 \times 10^{59}$
  erg. Comparison of the cavity sound speed ages and radio source ages
  indicate the outflow may be supersonic, and if significant gas
  shocking has occurred, the total kinetic power of the AGN may be
  $\sim 10^{46} ~\lum$.
\item Core ICM properties and the mass accretion rate required to
  power the AGN outburst suggest that fuel for the AGN was likley not
  accreted directly from the hot ICM, \ie\ via the Bondi mechanism,
  but rather attained through cold-mode accretion. This conclusion is
  consistent with the sub-kpc structure of the BCG halo and the X-ray
  properties of the nucleus.
\item Detection of an X-ray excess 13-26 kpc NE of the nucleus
  indicates beamed radiation from the $\sim 10^{47} ~\lum$ QSO is
  escaping the nucleus and interacting with the ICM in the same region
  as a strongly photoionized nebulae. The X-ray emission properties of
  this region are well-fit by a model where the nebula and ICM are
  being irradiated by the QSO.
\item The nuclear X-ray source is well-fit by a reflection-dominated
  model where the power law source is obscured by moderately
  Compton-thick material. The width of the nuclear \feka\ fluorescence
  emission line indicates a reflecting column density, and presumably
  obscuring column density, of $> 10^{24} \pcmsq$. We also show that
  the \bepposax\ PDS detection of hard X-ray emission can also be
  explained by reflected emission.
\item Based on the ostensible \irs\ gas-poorness, nuclear emission
  line outflow, high effective Eddington QSO luminosity which can
  expel clouds, and misalignment of the large-scale radio jets \&
  beamed radiation from the nucleus, we suggest that \irs\ is evolving
  from a radiation-dominated mode of feedback to a kinetic-dominated
  mode. Among other possible explanations, we speculate that the
  observed properties of \irs\ may be related to the process of SMBH
  spin evolution. \irs\ may be a local example of how massive galaxies
  at higher redshifts evolve from quasar-mode into radio-mode.
\end{itemize}

%%%%%%%%%%%%%%%%%%%%%%%%%%%
\section*{Acknowledgements}
%%%%%%%%%%%%%%%%%%%%%%%%%%%

KWC and MD were supported by SAO grant GO9-0143X, and MD and GMV
acknowledges support through NASA LTSA grant NASA NNG-05GD82G. KWC and
BRM thank the Natural Sciences and Engineering Research Council of
Canada for support. KWC thanks Alastair Edge \& Niayesh Afshordi for
helpful insight, and Guillaume Belanger \& Roland Walter for advice
regarding \integral\ data analysis.

%%%%%%%%%%%%%%%%
% Bibliography %
%%%%%%%%%%%%%%%%

\bibliographystyle{mn2e}
\bibliography{cavagnolo}

%%%%%%%%%%%%%%%%%%%%%%%
% Figures  and Tables %
%%%%%%%%%%%%%%%%%%%%%%%

\clearpage
\onecolumn
\begin{deluxetable}{lccccccccc}
\tablewidth{0pt}
\tabletypesize{\scriptsize}
\tablecaption{Summary of Global Spectral Properties\label{tab:specfits}}
\tablehead{\colhead{Region} & \colhead{$R_{in}$} & \colhead{$R_{out}$ } & \colhead{$N_{HI}$} & \colhead{$T_{X}$} & \colhead{$Z$} & \colhead{redshift} & \colhead{$\chi^2_{red.}$} & \colhead{D.O.F.} & \colhead{\% Source}\\
\colhead{ } & \colhead{kpc} & \colhead{kpc} & \colhead{$10^{20}$ cm$^{-2}$} & \colhead{keV} & \colhead{$Z_{\sun}$} & \colhead{ } & \colhead{ } & \colhead{ } & \colhead{ }\\
\colhead{{(1)}} & \colhead{{(2)}} & \colhead{{(3)}} & \colhead{{(4)}} & \colhead{{(5)}} & \colhead{{(6)}} & \colhead{{(7)}} & \colhead{{(8)}} & \colhead{{(9)}} & \colhead{{(10)}}
}
\startdata
$R_{500-core}$ & 251 & 1675 & 2.86$^{+2.46}_{-2.75}$  & 13.26$^{+6.21}_{-2.95}$  & 0.54$^{+0.28}_{-0.26}$  & 0.3605$^{+0.0235}_{-0.0162}$  & 1.10 & 368 &  19\\
$R_{1000-core}$ & 251 & 1184 & 3.13$^{+2.31}_{-2.33}$  & 11.20$^{+3.11}_{-1.97}$  & 0.51$^{+0.21}_{-0.21}$  & 0.3639$^{+0.0201}_{-0.0156}$  & 1.08 & 292 &  25\\
$R_{2500-core}$ & 251 & 749 & 1.90$^{+2.30}_{-1.90}$ & 10.66$^{+2.20}_{-1.65}$  & 0.60$^{+0.22}_{-0.19}$  & 0.3611$^{+0.0143}_{-0.0127}$  & 1.01 & 219 &  37\\
$R_{5000-core}$ & 251 & 529 & 3.22$^{+2.75}_{-2.58}$  & 8.80$^{+1.87}_{-1.31}$  & 0.46$^{+0.19}_{-0.18}$  & 0.3556$^{+0.0134}_{-0.0119}$  & 1.11 & 170 &  50\\
$R_{7500-core}$ & 251 & 432 & 2.70$^{+3.02}_{-2.70}$ & 9.85$^{+2.80}_{-1.90}$  & 0.35$^{+0.22}_{-0.21}$  & 0.3632$^{+0.0240}_{-0.0231}$  & 1.06 & 138 &  57\\
$R_{500}$ & \nodata & 1675 & 3.22$^{+1.09}_{-1.02}$  & 7.28$^{+0.50}_{-0.45}$  & 0.41$^{+0.06}_{-0.06}$  & 0.3563$^{+0.0053}_{-0.0044}$  & 0.95 & 535 &  39\\
$R_{1000}$ & \nodata & 1184 & 3.25$^{+1.00}_{-0.93}$  & 7.05$^{+0.40}_{-0.39}$  & 0.40$^{+0.05}_{-0.05}$  & 0.3573$^{+0.0043}_{-0.0040}$  & 0.91 & 488 &  51\\
$R_{2500}$ & \nodata & 749 & 2.97$^{+0.87}_{-0.97}$  & 6.88$^{+0.38}_{-0.33}$  & 0.41$^{+0.05}_{-0.05}$  & 0.3558$^{+0.0026}_{-0.0046}$  & 0.88 & 442 &  70\\
$R_{5000}$ & \nodata & 529 & 3.10$^{+0.90}_{-0.96}$  & 6.66$^{+0.34}_{-0.31}$  & 0.40$^{+0.05}_{-0.05}$  & 0.3560$^{+0.0028}_{-0.0047}$  & 0.85 & 418 &  81\\
$R_{7500}$ & \nodata & 432 & 3.17$^{+0.97}_{-1.04}$  & 6.61$^{+0.38}_{-0.31}$  & 0.39$^{+0.05}_{-0.05}$  & 0.3550$^{+0.0037}_{-0.0047}$  & 0.86 & 410 &  85\\
\hline
$R_{500-core}$ & 251 & 1675 & 3.18$^{+2.46}_{-2.61}$  & 12.63$^{+5.19}_{-2.54}$  & 0.53$^{+0.27}_{-0.26}$  & 0.3540 & 1.10 & 369 &  19\\
$R_{1000-core}$ & 251 & 1184 & 3.40$^{+2.19}_{-2.22}$  & 10.79$^{+2.69}_{-1.70}$  & 0.50$^{+0.20}_{-0.21}$  & 0.3540 & 1.08 & 293 &  25\\
$R_{2500-core}$ & 251 & 749 & 2.19$^{+2.20}_{-2.15}$  & 10.33$^{+2.08}_{-1.50}$  & 0.60$^{+0.21}_{-0.20}$  & 0.3540 & 1.01 & 220 &  37\\
$R_{5000-core}$ & 251 & 529 & 3.25$^{+2.63}_{-2.49}$  & 8.76$^{+1.73}_{-1.30}$  & 0.46$^{+0.18}_{-0.17}$  & 0.3540 & 1.10 & 171 &  50\\
$R_{7500-core}$ & 251 & 432 & 2.99$^{+2.94}_{-2.79}$  & 9.56$^{+2.67}_{-1.74}$  & 0.34$^{+0.21}_{-0.21}$  & 0.3540 & 1.05 & 139 &  57\\
$R_{500}$ & \nodata & 1675 & 3.25$^{+1.01}_{-1.01}$  & 7.25$^{+0.46}_{-0.42}$  & 0.41$^{+0.06}_{-0.06}$  & 0.3540 & 0.95 & 536 &  39\\
$R_{1000}$ & \nodata & 1184 & 3.31$^{+0.99}_{-0.98}$  & 7.00$^{+0.40}_{-0.37}$  & 0.40$^{+0.05}_{-0.05}$  & 0.3540 & 0.91 & 489 &  51\\
$R_{2500}$ & \nodata & 749 & 2.95$^{+0.97}_{-0.95}$  & 6.87$^{+0.37}_{-0.33}$  & 0.41$^{+0.06}_{-0.05}$  & 0.3540 & 0.88 & 443 &  70\\
$R_{5000}$ & \nodata & 529 & 3.10$^{+0.98}_{-0.95}$  & 6.65$^{+0.34}_{-0.32}$  & 0.40$^{+0.05}_{-0.05}$  & 0.3540 & 0.85 & 419 &  81\\
$R_{7500}$ & \nodata & 432 & 3.17$^{+0.99}_{-0.97}$  & 6.60$^{+0.34}_{-0.31}$  & 0.39$^{+0.05}_{-0.05}$  & 0.3540 & 0.86 & 411 &  85\\
\hline
$R_{500-core}$ & 251 & 1675 & 2.22 & 14.16$^{+3.68}_{-2.43}$  & 0.54$^{+0.30}_{-0.28}$  & 0.3626$^{+0.0215}_{-0.0219}$  & 1.09 & 369 &  19\\
$R_{1000-core}$ & 251 & 1184 & 2.22 & 11.91$^{+2.15}_{-1.52}$  & 0.52$^{+0.22}_{-0.22}$  & 0.3658$^{+0.0226}_{-0.0160}$  & 1.08 & 293 &  25\\
$R_{2500-core}$ & 251 & 749 & 2.22 & 10.46$^{+1.49}_{-1.15}$  & 0.60$^{+0.20}_{-0.19}$  & 0.3613$^{+0.0144}_{-0.0132}$  & 1.01 & 220 &  37\\
$R_{5000-core}$ & 251 & 529 & 2.22 & 9.26$^{+1.27}_{-1.02}$  & 0.46$^{+0.19}_{-0.18}$  & 0.3560$^{+0.0140}_{-0.0061}$  & 1.11 & 171 &  50\\
$R_{7500-core}$ & 251 & 432 & 2.22 & 10.15$^{+1.81}_{-1.36}$  & 0.35$^{+0.22}_{-0.20}$  & 0.3647$^{+0.0123}_{-0.0231}$  & 1.05 & 139 &  57\\
$R_{500}$ & \nodata & 1675 & 2.22 & 7.63$^{+0.33}_{-0.30}$  & 0.41$^{+0.06}_{-0.06}$  & 0.3567$^{+0.0048}_{-0.0034}$  & 0.96 & 536 &  39\\
$R_{1000}$ & \nodata & 1184 & 2.22 & 7.38$^{+0.29}_{-0.29}$  & 0.40$^{+0.06}_{-0.05}$  & 0.3586$^{+0.0027}_{-0.0067}$  & 0.91 & 489 &  51\\
$R_{2500}$ & \nodata & 749 & 2.22 & 7.09$^{+0.26}_{-0.23}$  & 0.41$^{+0.06}_{-0.05}$  & 0.3556$^{+0.0031}_{-0.0045}$  & 0.88 & 443 &  70\\
$R_{5000}$ & \nodata & 529 & 2.22 & 6.90$^{+0.23}_{-0.23}$  & 0.40$^{+0.05}_{-0.05}$  & 0.3560$^{+0.0027}_{-0.0048}$  & 0.85 & 419 &  81\\
$R_{7500}$ & \nodata & 432 & 2.22 & 6.86$^{+0.24}_{-0.22}$  & 0.39$^{+0.05}_{-0.05}$  & 0.3558$^{+0.0020}_{-0.0046}$  & 0.87 & 411 &  85\\
\hline
$R_{500-core}$ & 251 & 1675 & 2.22 & 13.80$^{+3.08}_{-2.21}$  & 0.53$^{+0.28}_{-0.28}$  & 0.3540 & 1.09 & 370 &  19\\
$R_{1000-core}$ & 251 & 1184 & 2.22 & 11.64$^{+1.09}_{-1.44}$  & 0.50$^{+0.22}_{-0.22}$  & 0.3540 & 1.08 & 294 &  25\\
$R_{2500-core}$ & 251 & 749 & 2.22 & 10.31$^{+1.38}_{-1.09}$  & 0.60$^{+0.20}_{-0.20}$  & 0.3540 & 1.01 & 221 &  37\\
$R_{5000-core}$ & 251 & 529 & 2.22 & 9.22$^{+1.21}_{-0.97}$  & 0.47$^{+0.19}_{-0.18}$  & 0.3540 & 1.10 & 172 &  50\\
$R_{7500-core}$ & 251 & 432 & 2.22 & 10.01$^{+1.77}_{-1.33}$  & 0.34$^{+0.21}_{-0.22}$  & 0.3540 & 1.05 & 140 &  57\\
$R_{500}$ & \nodata & 1675 & 2.22 & 7.60$^{+0.32}_{-0.30}$  & 0.41$^{+0.06}_{-0.06}$  & 0.3540 & 0.96 & 537 &  39\\
$R_{1000}$ & \nodata & 1184 & 2.22 & 7.34$^{+0.28}_{-0.26}$  & 0.40$^{+0.06}_{-0.05}$  & 0.3540 & 0.92 & 490 &  51\\
$R_{2500}$ & \nodata & 749 & 2.22 & 7.08$^{+0.25}_{-0.23}$  & 0.42$^{+0.05}_{-0.06}$  & 0.3540 & 0.88 & 444 &  70\\
$R_{5000}$ & \nodata & 529 & 2.22 & 6.88$^{+0.23}_{-0.22}$  & 0.40$^{+0.05}_{-0.05}$  & 0.3540 & 0.85 & 420 &  81\\
$R_{7500}$ & \nodata & 432 & 2.22 & 6.85$^{+0.24}_{-0.22}$  & 0.39$^{+0.05}_{-0.05}$  & 0.3540 & 0.87 & 412 &  85\\
\hline
NE-Arm & \nodata & \nodata & 2.22 & 5.00$^{+1.08}_{-0.78}$  & 1.23$^{+1.28}_{-0.79}$  & 0.3540 & 0.19 & 300 &  98\\
NW-Arm & \nodata & \nodata & 2.22 & 5.73$^{+1.28}_{-0.98}$  & 0.54$^{+0.55}_{-0.48}$  & 0.3540 & 0.26 & 184 &  99\\
SE-Arm & \nodata & \nodata & 2.22 & 5.49$^{+1.24}_{-0.80}$  & 0.36$^{+0.41}_{-0.36}$  & 0.3540 & 0.16 & 268 &  99\\
SW-Arm & \nodata & \nodata & 2.22 & 5.99$^{+1.41}_{-0.98}$  & 1.04$^{+0.73}_{-0.49}$  & 0.3540 & 0.14 & 338 &  99\\
\hline
\enddata
\tablecomments{Col. (1) Name of region used for spectral extraction; col. (2) inner radius of extraction region; col. (3) outer radius of extraction region; col. (4) absorbing, Galactic neutral hydrogen column density; col. (5) best-fit temperature; col. (6) best-fit metallicity; col. (7) best-fit redshift; col. (8) reduced \chisq\ for best-fit model; col. (9) degrees of freedom for best-fit model; col. (10) percentage of emission attributable to source.}
\end{deluxetable}

\begin{table*}
  \begin{center}
    \caption{\sc Summary of Cavity Properties.\label{tab:cylcavities}}
    \begin{tabular}{lcccccc}
      \hline
      \hline
      Cavity & $r$ & $l$ & \tsonic & $pV$ & \ecav & \pcav\\
      -- & kpc & kpc & $10^6$ yr & $10^{58}$ ergs & $10^{59}$ ergs & $10^{44}$ ergs s$^{-1}$\\
      (1) & (2) & (3) & (4) & (5) & (6) & (7)\\
      \hline
      Northwest & 6.40 & 58.3 & ${50.5 \pm 7.6}$ & ${5.78 \pm 1.07}$ & ${2.31 \pm 0.43}$ & ${1.45 \pm 0.35}$\\
      Southeast & 6.81 & 64.0 & ${55.4 \pm 8.4}$ & ${6.99 \pm 1.29}$ & ${2.80 \pm 0.52}$ & ${1.60 \pm 0.38}$\\
      \hline
    \end{tabular}
    \begin{quote}
      Col. (1) Cavity location; Col. (2) Radius of excavated cylinder;
      Col. (3) Length of excavated cylinder; Col. (4) Sound speed age;
      Col. (5) $pV$ work; Col. (6) Cavity energy; Col. (7) Cavity power.
    \end{quote}
  \end{center}
\end{table*}

\begin{table*}
  \begin{center}
    \caption{\sc Summary of X-ray Excesses Spectral Fits.\label{tab:excess}}
    \begin{tabular}{lccccccc}
      \hline
      \hline
      Region & \tx & $\eta$ & $E_{\mathrm{G}}$ & $\sigma_{\mathrm{G}}$ & $\eta_{\mathrm{G}}$ & Cash & DOF\\
      - & keV & $10^{-5}$ cm$^{-5}$ & keV & keV & $10^{-6}~\pcmsq~\ps$ & - & -\\
      (1) & (2) & (3) & (4) & (5) & (6) & (7) & (8)\\
      \hline
      Eastern excess     & 3.03$^{+1.19}_{-0.74}$ & $5.80^{+1.07}_{-0.97}$ & -                  & -                    & -                & 524 & 430\\
      Eastern excess     & 3.68$^{+3.34}_{-1.58}$ & $2.73^{+0.98}_{-0.94}$ & [0.89, 1.42, 4.23] & [0.04, 0.16, 3.6E-4] & [1.2, 2.0, 0.16] & 384 & 430\\
      Eastern excess bgd & 3.92$^{+0.35}_{-0.31}$ & $39.9^{+0.18}_{-0.17}$ & -                  & -                    & -                & 471 & 430\\
      Lower-NW excess & 2.55$^{+2.61}_{-0.98}$ & $0.66^{+0.11}_{-0.07}$ & -                  & -                    & -                & 387 & 430\\
      \hline
    \end{tabular}
    \begin{quote}
      Metal abundance was fixed at $0.5 ~\Zsol$ for all fits.
      Col. (1) Extraction region; Col. (2) Thermal gas temperature;
      Col. (3) Model normalization; Col. (4) Gaussian central
      energies; Col. (5) Gaussian dispersions; Col. (6) Gaussian
      normalizations; Col. (7) Modified Cash statistic; Col. (8)
      Degrees of freedom.
    \end{quote}
  \end{center}
\end{table*}

\begin{table*}
  \begin{center}
    \caption{\sc Summary of Nuclear Source Spectral Fits.\label{tab:nucspec}}
    \begin{tabular}{lccc}
      \hline
      \hline
      Component & Parameter & SP09 & SP99\\
      (1) & (2) & (3) & (4)\\
      \hline
      \pexrav\  & $\Gamma$              & $1.71^{+0.23}_{-0.65}$                & fixed to SP09\\
      -         & $\eta_{\mathrm{P}}$   & $8.07^{+0.64}_{-0.62}\times10^{-4}$   & $8.46^{+2.08}_{-2.12} \times 10^{-4}$\\
      \gauss\ 1 & $E_{\mathrm{G}}$      & $0.73^{+0.05}_{-0.24}$                & $0.61^{+0.10}_{-0.05}$\\
      -         & $\sigma_{\mathrm{G}}$ & $85^{+197}_{-53}$                     & $97^{+150}_{-97}$\\
      -         & $\eta_{\mathrm{G}}$   & $8.14^{+3.74}_{-5.82} \times 10^{-6}$ & $1.65^{+1.52}_{-1.00} \times 10^{-5}$\\
      \gauss\ 2 & $E_{\mathrm{G}}$      & $1.16^{+0.19}_{-0.33}$                & $0.90^{+0.17}_{-0.90}$\\
      -         & $\sigma_{\mathrm{G}}$ & $383^{+610}_{-166}$                   & $506^{+314}_{-262}$\\
      -         & $\eta_{\mathrm{G}}$   & $1.03^{+3.22}_{-0.48} \times 10^{-5}$ & $1.48^{+2.68}_{-1.16} \times 10^{-5}$\\
      \gauss\ 3 & $E_{\mathrm{G}}$      & $4.45^{+0.04}_{-0.04}$                & $4.46^{+0.04}_{-0.07}$\\
      -         & $\sigma_{\mathrm{G}}$ & $45^{+60}_{-45}$                      & $31^{+94}_{-31}$\\
      -         & $\eta_{\mathrm{G}}$   & $2.67^{+0.91}_{-0.86} \times 10^{-6}$ & $6.45^{+4.17}_{-3.69} \times 10^{-6}$\\
      -         & EW$^{\mathrm{corr}}_{\mathrm{K}\alpha}$ & $531^{+211}_{-218}$ & $1210^{+720}_{-710}$\\
      Statistic & \chisq                & 79.0                                  & 7.9\\
      -         & DOF                   & 74                                    & 15\\
      \hline
    \end{tabular}
    \begin{quote}
      \feka\ equivalent widths have been corrected for redshift. Units for
      parameters: $\Gamma$ is dimensionless, $\eta_{\mathrm{P}}$ is in ph
      keV$^{-1}$ cm$^{-2}$ s$^{-1}$, $E_{\mathrm{G}}$ are in keV,
      $\sigma_{\mathrm{G}}$ are in eV, $\eta_{\mathrm{G}}$ are in ph
      cm$^{-2}$ s$^{-1}$, EW$_{\mathrm{corr}}$ are in eV. Col. (1)
      \xspec\ model name; Col. (2) Model parameters; Col. (3) Values for
      2009 \cxo\ spectrum; Col. (4) Values for 1999 \cxo\ spectrum.
    \end{quote}
  \end{center}
\end{table*}

\clearpage
\begin{figure}
  \begin{center}
    \begin{minipage}{\linewidth}
      \includegraphics*[width=\textwidth, trim=0mm 0mm 0mm 0mm, clip]{rbs797.ps}
    \end{minipage}
    \caption{Fluxed, unsmoothed 0.7--2.0 keV clean image of \rbs\ in
      units of ph \pcmsq\ \ps\ pix$^{-1}$. Image is $\approx 250$ kpc
      on a side and coordinates are J2000 epoch. Black contours in the
      nucleus are 2.5--9.0 keV X-ray emission of the nuclear point
      source; the outer contour approximately traces the 90\% enclosed
      energy fraction (EEF) of the \cxo\ point spread function. The
      dashed green ellipse is centered on the nuclear point source,
      encloses both cavities, and was drawn by-eye to pass through the
      X-ray ridge/rims.}
    \label{fig:img}
  \end{center}
\end{figure}

\begin{figure}
  \begin{center}
    \begin{minipage}{0.495\linewidth}
      \includegraphics*[width=\textwidth, trim=0mm 0mm 0mm 0mm, clip]{325.ps}
    \end{minipage}
   \begin{minipage}{0.495\linewidth}
      \includegraphics*[width=\textwidth, trim=0mm 0mm 0mm 0mm, clip]{8.4.ps}
   \end{minipage}
   \begin{minipage}{0.495\linewidth}
      \includegraphics*[width=\textwidth, trim=0mm 0mm 0mm 0mm, clip]{1.4.ps}
    \end{minipage}
    \begin{minipage}{0.495\linewidth}
      \includegraphics*[width=\textwidth, trim=0mm 0mm 0mm 0mm, clip]{4.8.ps}
    \end{minipage}
     \caption{Radio images of \rbs\ overlaid with black contours
       tracing ICM X-ray emission. Images are in mJy beam$^{-1}$ with
       intensity beginning at $3\sigma_{\rm{rms}}$ and ending at the
       peak flux, and are arranged by decreasing size of the
       significant, projected radio structure. X-ray contours are from
       $2.3 \times 10^{-6}$ to $1.3 \times 10^{-7}$ ph
       \pcmsq\ \ps\ pix$^{-1}$ in 12 square-root steps. {\it{Clockwise
           from top left}}: 325 MHz \vla\ A-array, 8.4 GHz
       \vla\ D-array, 4.8 GHz \vla\ A-array, and 1.4 GHz
       \vla\ A-array.}
    \label{fig:composite}
  \end{center}
\end{figure}

\begin{figure}
  \begin{center}
    \begin{minipage}{0.495\linewidth}
      \includegraphics*[width=\textwidth, trim=0mm 0mm 0mm 0mm, clip]{sub_inner.ps}
    \end{minipage}
    \begin{minipage}{0.495\linewidth}
      \includegraphics*[width=\textwidth, trim=0mm 0mm 0mm 0mm, clip]{sub_outer.ps}
    \end{minipage}
    \caption{Red text point-out regions of interest discussed in
      Section \ref{sec:cavities}. {\it{Left:}} Residual 0.3-10.0 keV
      X-ray image smoothed with $1\arcs$ Gaussian. Yellow contours are
      1.4 GHz emission (\vla\ A-array), orange contours are 4.8 GHz
      emission (\vla\ A-array), orange vector is 4.8 GHz jet axis, and
      red ellipses outline definite cavities. {\it{Bottom:}} Residual
      0.3-10.0 keV X-ray image smoothed with $3\arcs$ Gaussian. Green
      contours are 325 MHz emission (\vla\ A-array), blue contours are
      8.4 GHz emission (\vla\ D-array), and orange vector is 4.8 GHz
      jet axis.}
    \label{fig:subxray}
  \end{center}
\end{figure}

\begin{figure}
  \begin{center}
    \begin{minipage}{\linewidth}
      \includegraphics*[width=\textwidth]{r797_nhfro.eps}
      \caption{Gallery of radial ICM profiles. Vertical black dashed
        lines mark the approximate end-points of both
        cavities. Horizontal dashed line on cooling time profile marks
        age of the Universe at redshift of \rbs. For X-ray luminosity
        profile, dashed line marks \lcool, and dashed-dotted line
        marks \pcav.}
      \label{fig:gallery}
    \end{minipage}
  \end{center}
\end{figure}

\begin{figure}
  \begin{center}
    \begin{minipage}{\linewidth}
      \setlength\fboxsep{0pt}
      \setlength\fboxrule{0.5pt}
      \fbox{\includegraphics*[width=\textwidth]{cav_config.eps}}
    \end{minipage}
    \caption{Cartoon of possible cavity configurations. Arrows denote
      direction of AGN outflow, ellipses outline cavities, \rlos\ is
      line-of-sight cavity depth, and $z$ is the height of a cavity's
      center above the plane of the sky. {\it{Left:}} Cavities which
      are symmetric about the plane of the sky, have $z=0$, and are
      inflating perpendicular to the line-of-sight. {\it{Right:}}
      Cavities which are larger than left panel, have non-zero $z$,
      and are inflating along an axis close to our line-of-sight.}
    \label{fig:config}
  \end{center}
\end{figure}

\begin{figure}
  \begin{center}
    \begin{minipage}{0.495\linewidth}
      \includegraphics*[width=\textwidth, trim=25mm 0mm 40mm 10mm, clip]{edec.eps}
    \end{minipage}
    \begin{minipage}{0.495\linewidth}
      \includegraphics*[width=\textwidth, trim=25mm 0mm 40mm 10mm, clip]{wdec.eps}
    \end{minipage}
    \caption{Surface brightness decrement as a function of height
      above the plane of the sky for a variety of cavity radii. Each
      curve is labeled with the corresponding \rlos. The curves
      furthest to the left are for the minimum \rlos\ needed to
      reproduce $y_{\rm{min}}$, \ie\ the case of $z = 0$, and the
      horizontal dashed line denotes the minimum decrement for each
      cavity. {\it{Left}} Cavity E1; {\it{Right}} Cavity W1.}
    \label{fig:decs}
  \end{center}
\end{figure}


\begin{figure}
  \begin{center}
    \begin{minipage}{\linewidth}
      \includegraphics*[width=\textwidth, trim=15mm 5mm 5mm 10mm, clip]{pannorm.eps}
      \caption{Histograms of normalized surface brightness variation
        in wedges of a $2.5\arcs$ wide annulus centered on the X-ray
        peak and passing through the cavity midpoints. {\it{Left:}}
        $36\mydeg$ wedges; {\it{Middle:}} $14.4\mydeg$ wedges;
        {\it{Right:}} $7.2\mydeg$ wedges. The depth of the cavities
        and prominence of the rims can be clearly seen in this plot.}
      \label{fig:pannorm}
    \end{minipage}
  \end{center}
\end{figure}

\begin{figure}
  \begin{center}
    \begin{minipage}{0.5\linewidth}
      \includegraphics*[width=\textwidth, angle=-90]{nucspec.ps}
    \end{minipage}
    \caption{X-ray spectrum of nuclear point source. Black denotes
      year 2000 \cxo\ data (points) and best-fit model (line), and red
      denotes year 2007 \cxo\ data (points) and best-fit model (line).
      The residuals of the fit for both datasets are given below.}
    \label{fig:nucspec}
  \end{center}
\end{figure}

\begin{figure}
  \begin{center}
    \begin{minipage}{\linewidth}
      \includegraphics*[width=\textwidth, trim=10mm 5mm 10mm 10mm, clip]{radiofit.eps}
    \end{minipage}
    \caption{Best-fit continuous injection (CI) synchrotron model to
      the nuclear 1.4 GHz, 4.8 GHz, and 7.0 keV X-ray emission. The
      two triangles are \galex\ UV fluxes showing the emission is
      boosted above the power-law attributable to the nucleus.}
    \label{fig:sync}
    \end{center}
\end{figure}

\begin{figure}
  \begin{center}
    \begin{minipage}{\linewidth}
      \includegraphics*[width=\textwidth, trim=0mm 0mm 0mm 0mm, clip]{rbs797_opt.ps}
    \end{minipage}
    \caption{\hst\ \myi+\myv\ image of the \rbs\ BCG with units e$^-$
      s$^{-1}$. Green, dashed contour is the \cxo\ 90\% EEF. Emission
      features discussed in the text are labeled.}
    \label{fig:hst}
  \end{center}
\end{figure}

\begin{figure}
  \begin{center}
    \begin{minipage}{0.495\linewidth}
      \includegraphics*[width=\textwidth, trim=0mm 0mm 0mm 0mm, clip]{suboptcolor.ps}
    \end{minipage}
    \begin{minipage}{0.495\linewidth}
      \includegraphics*[width=\textwidth, trim=0mm 0mm 0mm 0mm, clip]{suboptrad.ps}
    \end{minipage}
    \caption{{\it{Left:}} Residual \hst\ \myv\ image. White regions
      (numbered 1--8) are areas with greatest color difference with
      \rbs\ halo. {\it{Right:}} Residual \hst\ \myi\ image. Green
      contours are 4.8 GHz radio emission down to
      $1\sigma_{\rm{rms}}$, white dashed circle has radius $2\arcs$,
      edge of ACS ghost is show in yellow, and southern whiskers are
      numbered 9--11 with corresponding white lines.}
    \label{fig:subopt}
  \end{center}
\end{figure}


%%%%%%%%%%%%%%%%%%%%
% End the document %
%%%%%%%%%%%%%%%%%%%%

\label{lastpage}
\end{document}
