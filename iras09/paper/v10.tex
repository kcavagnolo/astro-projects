%%%%%%%%%%%%%%%%%%%
% Custom commands %
%%%%%%%%%%%%%%%%%%%

\newcommand{\mykeywords}{cooling flows -- galaxies: clusters:
  galaxies: individual (\inine): clusters: individual (\rxj)}
\newcommand{\mytitle}{Mechanical and Radiative Feedback from the Quasar in \inine}
\newcommand{\mystitle}{Feedback in \inine}
\newcommand{\inine}{IRAS 09104+4109}
\newcommand{\irs}{I09}
\newcommand{\rxj}{RX J0913.7+4056}
\newcommand{\tsync}{\ensuremath{t_{\rm{sync}}}}
\newcommand{\refl}{\ensuremath{r_{\rm{refl}}}}
\newcommand{\fekaew}{\ensuremath{\rm{EW}_{\rm{K}\alpha}}}
\newcommand{\cltx}{\ensuremath{T_{\rm{cl}}}}
\newcommand{\leff}{\ensuremath{\lambda_{\rm{Edd}}}}
\newcommand{\lqso}{\ensuremath{L_{\rm{QSO}}}}
\newcommand{\esens}{\ensuremath{E_{\rm{sens}}}}
\newcommand{\nflux}{\ensuremath{\rm{ph} ~\rm{cm}^{-2} ~\rm{s}^{-1} ~\rm{pix}^{-1}}}

%%%%%%%%%%
% Header %
%%%%%%%%%%

\documentclass[useAMS,usenatbib]{mn2e}
\usepackage{color, graphicx, here, common, ifthen, amsmath, amssymb, natbib,
  mathptmx, url, times, array}
\usepackage[pagebackref,
  pdftitle={\mytitle},
  pdfauthor={Dr. Kenneth W. Cavagnolo},
  pdfsubject={ApJ},
  pdfkeywords={},
  pdfproducer={LaTeX with hyperref},
  pdfcreator={LaTeX with hyperref}
  pdfdisplaydoctitle=true,
  colorlinks=true,
  citecolor=blue,
  linkcolor=blue,
  urlcolor=blue]{hyperref}

\title[\mystitle]{\mytitle}

\author[Cavagnolo et al.]{K. W. Cavagnolo$^{1,2}$\thanks{Email:
    kcavagno@uwaterloo.ca}, M. Donahue$^{3}$,
  B. R. McNamara$^{1,4,5}$, G. M. Voit$^{3}$, and M. Sun$^{6}$\\
  $^{1}$University of Waterloo, Waterloo, ON, Canada.\\
  $^{2}$UNS, CNRS UMR 6202 Cassiop\'{e}e, Observatoire de la C\^{o}te d'Azur, Nice, France.\\
  $^{3}$Michigan State University, East Lansing, MI, USA.\\
  $^{4}$Perimeter Institute for Theoretical Physics, Waterloo, ON, Canada.\\
  $^{5}$Harvard-Smithsonian Center for Astrophysics, Cambridge, MA, USA.\\
  $^{6}$University of Virginia, Charlottesville, VA, USA.}

\begin{document}

\date{Accepted (2010 Month Day). Received (2010 Month Day); in
  original form (2010 Month Day)}

\pagerange{\pageref{firstpage}--\pageref{lastpage}} \pubyear{2010}

\maketitle

\label{firstpage}

%%%%%%%%%%%%
% Abstract %
%%%%%%%%%%%%

\begin{abstract}
  Using $\approx 77$ ks of new data from the Chandra X-ray
  Observatory, we present a detailed study of the AGN activity
  associated with the ultraluminous infrared brightest cluster galaxy
  (BCG) \inine. Our X-ray analysis reveals the obscured AGN in
  \inine\ has excavated cavities in the host galaxy cluster
  intracluster medium (ICM) and is beaming intense radiation into the
  halo which has photoionized the ICM and a galactic nebula. Using the
  cavity properties, we estimate the mechanical AGN power output at
  $\approx 3 \times 10^{44} ~\lum$, while the nuclear X-ray emission
  suggests a radiative AGN power output in excess of $5 \times 10^{46}
  ~\lum$, \ie\ a radiative to mechanical ratio of $\approx 200:1$. In
  addition to the direct evidence of mechanical and radiative feedback
  impacting the extended \inine\ halo, we argue that the relative
  dust-richness and gas poorness of \inine\ indicate galactic-scale
  radiative feedback is significantly influencing the host
  galaxy. From these results, we speculate that \inine\ may be the
  first example of a system simultaneously undergoing what is often
  termed ``quasar-mode'' and ``radio-mode'' feedback. Using the
  misalignment between the beamed nuclear radiation and the
  large-scale jets as a constraint, we explore a range of possible
  explanations for the nature of \inine, such as a brief period of
  supercritical accretion and associated evolution of black hole
  spin. We go on to suggest that \inine\ may be a local example of how
  massive, high-redshift galaxies transition from a
  radiatively-dominated to a mechanically-dominated form of feedback.
\end{abstract}

%%%%%%%%%%%%
% Keywords %
%%%%%%%%%%%%

\begin{keywords}
  \mykeywords
\end{keywords}

%%%%%%%%%%%%%%%%%%%%%%
\section{Introduction}
\label{sec:intro}
%%%%%%%%%%%%%%%%%%%%%%

The discovery of tight correlations between galaxy stellar bulge
properties and the mass of their centrally located supermassive black
holes (SMBHs) indicate that the two co-evolve
\citep[\eg][]{1995ARA&A..33..581K, magorrian, 2000ApJ...539L...9F,
  2000ApJ...539L..13G, 2001ApJ...563L..11G}. It has been suggested
that galaxy-galaxy interactions, along with feedback from active
galactic nuclei (AGN), form the foundation for SMBH-host galaxy
co-evolution \citep[\eg][]{1995MNRAS.276..663B, 1998A&A...331L...1S,
  2000MNRAS.311..576K, 2001MNRAS.324..757G}. The most massive galaxies
in the universe, \eg\ brightest cluster galaxies (BCGs), are a unique
population in the co-evolution framework as their properties are also
correlated with the galaxy cluster, or group, in which they reside
\citep[\eg][]{1984ApJ...276...38J, 1998ApJ...502..141D}. Thus, BCGs
are especially valuable for investigating galaxy formation and
evolution processes in a cosmological context, and in this paper we
discuss one rare BCG, \inine\ (hereafter \irs), which may be providing
vital clues about these processes.

Galaxy formation models typically segregate AGN feedback into a
distinct early-time, radiatively-dominated quasar (QSO) mode
\citep[\eg][]{2005Natur.435..629S, 2006ApJS..163....1H} and a
late-time, mechanically-dominated radio mode \citep[\eg][]{croton06,
  bower06}. During the quasar mode of feedback, it is believed that
quasar radiation couples to gas within the host galaxy and drives
strong winds which deprive the SMBH of additional fuel, regulating
growth of black hole mass \citep[\eg][]{2005ApJ...630..705H,
  2005Natur.433..604D}. This phase is expected to be short-lived,
resulting in the expulsion of gas from the host galaxy and temporarily
quenching star formation \citep[\eg][]{2006ApJ...642L.107N,
  2008ApJ...686..219M}. Direct evidence of radiative AGN feedback has
been elusive \citep[see][for a review]{2005ARA&A..43..769V}, with only
a handful of systems \citep[\eg][]{2008A&A...492...81P,
  2010A&A...518L.155F} and low-redshift ellipticals
\citep{2009ApJ...690.1672S} providing the strongest evidence to date
that quasar feedback influences the host galaxy in the ways models
predict.

At later times, when the nuclear accretion rate is sub-Eddington and
quasar activity has ceased, SMBH launched jets regulate the growth of
galaxy mass through prolonged and intermittent mechanical heating of a
galaxy's gaseous halo \citep[\eg][]{2005MNRAS.363....2K,
  2006MNRAS.368....2D}. Direct evidence of mechanical AGN feedback is
seen in the form of cavities, sound waves, and shocks found in the
X-ray halos of many massive galaxies, particularly in the relatively
dense intracluster medium (ICM) surrounding most BCGs
\citep[\eg][]{perseus1, hydraa0, 2001ApJ...554..261C,
  2007ApJ...665.1057F, 2008MNRAS.390L..93S}. Encouragingly, studies
have shown AGN supply enough energy to offset most of the host halo
radiative losses \citep[\eg][]{perseus2, birzan04, dunn06} and that
AGN activity is closely correlated with cluster core conditions
\citep[\eg][]{haradent, rafferty08}, both indicative of the feedback
loop required in models.

Though models divide feedback into two generic modes, they still form
a unified schema \citep[\eg][]{sijacki07} which predicts a continuous
distribution of AGN luminosities
\citep[\eg][]{2009ApJ...698.1550H}. However, the association of the
modes, and whether they interact, is still poorly understood, partly
from a lack of observational constraints. This paper presents
observational evidence of simultaneous mechanical and radiative
feedback from the AGN in \irs, perhaps implying that it is, using the
nomenclature of quasar-mode/radio-mode models, a ``transition''
object.

\irs\ is an uncommon, low-redshift ($z = 0.4418$) ultraluminous
infrared galaxy (ULIRG; $L_{\rm{IR}} \sim 10^{13} ~\lsol$) residing at
the center of the cool core galaxy cluster \rxj. Unlike most ULIRGs,
\irs\ is the BCG in a rich galaxy cluster, but unlike most BCGs, the
spectral energy distribution of \irs\ is dominated by a heavily
obscured quasar: it has the optical spectrum of a Seyfert-2 with most
of the bolometric luminosity emerging longward of 1 $\mu$m
\citep{1988ApJ...328..161K, 1993ApJ...415...82H, 1994ApJ...436L..51F,
  1998ApJ...506..205E, 2000A&A...353..910F,
  2001MNRAS.321L..15I}. \irs\ also hosts a pair of highly-linear,
$\approx 60$ kpc long jets which are dramatically misaligned from the
beaming direction of the quasar \citep[][hereafter H93 and H99,
  respectively]{1993ApJ...415...82H, 1999ApJ...512..145H}. Using data
from the Chandra\ X-ray Observatory (\cxo), we present the discovery
of 1) X-ray halo cavities associated with the \irs\ jets, and 2) an
X-ray excess lying along the misaligned quasar beaming direction which
may be associated with an irradiated nebula in the \irs\ halo. From
these discoveries, and using results in the lengthy \irs\ literature,
we go on to suggest that \irs\ may be cycling between the dominant
mode of feedback, providing a local example of how massive galaxies at
higher redshifts evolve from quasar-mode into radio-mode.

The observations and data reduction are discussed in Section
\ref{sec:obs}. The AGN-ICM interaction is discussed in Section
\ref{sec:sub}, analysis of ICM cavities, the \irs\ nucleus, and X-ray
halo excesses are given in Sections \ref{sec:cavs}, \ref{sec:nucleus},
and \ref{sec:rad}. Interpretation of the results are given in Section
\ref{sec:interp}, and a brief summary concludes the paper in Section
\ref{sec:summ}. \LCDM, for which a redshift of $z = 0.4418$
corresponds to $\approx 9.1$ Gyr for the age of the Universe, $\da
\approx 5.72$ kpc arcsec$^{-1}$, and $\dl \approx 2.45~\Gpc$. The ICM
mean molecular weight and adiabatic index are assumed to be $\mu =
0.597$ and $\gamma = 5/3$, respectively.

%%%%%%%%%%%%%%%%%%%%%%%%%%%%%%%%%%%%%%%%%
\section{Observations and Data Reduction}
\label{sec:obs}
%%%%%%%%%%%%%%%%%%%%%%%%%%%%%%%%%%%%%%%%%

%%%%%%%%%%%%%%%%%%%%%
\subsection{X-ray}
\label{sec:xray}
%%%%%%%%%%%%%%%%%%%%%

A 77.2 ks \cxo\ observation of \irs\ was taken in January 2009 with
the ACIS-I instrument (\dataset [ADS/Sa.CXO#Obs/10445] {ObsID 10445}),
and a 9 ks ACIS-S observation was taken in November 1999 (\dataset
[ADS/Sa.CXO#Obs/00509] {ObsID 509}). Both datasets were reprocessed
and reduced using \ciao\ and \caldb\ versions 4.2. X-ray events were
selected using \asca\ grades, and corrections for the ACIS gain
change, charge transfer inefficiency, and degraded quantum efficiency
were applied. The pileup percentage for both observations across the
full bandpass is $< 4\%$, low enough that no correction or spectral
modeling was applied. Point sources were located and excluded using
{\textsc{wavdetect}} and verified by visual inspection. Light curves
were extracted from a source-free region of each observation to look
for flares, and time intervals with $> 20\%$ of the mean background
count rate were excluded. After flare exclusion, the final combined
exposure time is 83 ks.

The point source free, flare-free events files were reprojected to a
common tangent point, summed, and used for imaging analysis. The
astrometry of the ObsID 509 dataset was improved using an aspect
solution created with the \ciao\ tool {\textsc{reproject\_aspect}} and
the positions of several field sources. After astrometry correction,
the positional accuracy between both observations improved by $\approx
0.4\arcs$ and was comparable to the resolution limit of the ACIS
detectors ($\approx 0.492\arcs$ pix$^{-1}$). We refer to the final
point source free, flare-free, exposure-corrected images as the
``clean'' images.

Unless stated otherwise, the X-ray spectral analysis in this paper was
performed over the energy range 0.7--7.0 keV with the
\chisq\ statistic in \xspec\ 12.4 \citep{xspec} using an absorbed,
single-component \mekal\ model \citep{mekal1} with metal abundance as
a free parameter (Solar distribution of \citealt{ag89}) and 90\%
confidence intervals. All spectral models had the Galactic absorbing
column density fixed at $\nhgal = 1.58 \times 10^{20} ~\pcmsq$
\citep{lab}. The \irs\ nucleus emits strong \feka\ emission which
affects spectral fitting, and for analysis of the ICM, the nucleus was
excluded using a region twice the size of the 90\% \cxo\ point-spread
function (PSF) enclosed energy fraction (EEF; see Section
\ref{sec:nucleus}).

%%%%%%%%%%%%%%%%%%
\subsection{Radio}
\label{sec:radio}
%%%%%%%%%%%%%%%%%%

Between 1986 and 2000, \irs\ was observed with the Very Large Array
(\vla) at multiple radio frequencies and resolutions. Archival
\vla\ continuum observations were reduced using the Common Astronomy
Software Applications (\casa) version 3.0. Flagging of bad data was
performed using a combination of \casa's {\textsc{flagdata}} tool and
manual inspection. Radio images were generated by Fourier
transforming, cleaning, self-calibrating, and restoring individual
radio observations. The additional steps of phase and amplitude
self-calibration were included to increase the dynamic range and
sensitivity of the radio maps. All sources within the primary beam and
first side-lobe with fluxes $\ge 5\sigma_{\rm{rms}}$ were imaged to
minimize the final image noise.

Resolved emission associated with \irs\ is detected at 1.4 GHz, 5.0
GHz, and 8.4 GHz, while a $3\sigma$ upper limit of $0.84$ mJy is
established at 14.9 GHz. The combined 1.4 GHz image is the deepest and
reveals the most extended structure, thus our discussion regarding
radio morphology is guided using this frequency. The deconvolved,
integrated 1.4 GHz flux of the continuous extended structure
coincident with \irs, and having $S_{\nu} \ga 3\sigma_{\rm{rms}}$, is
$14.0 \pm 0.5$ mJy. A significant spur of radio emission extending
northeast from the nucleus is also detected with flux $0.21 \pm 0.07$
mJy.

Fluxes for unresolved emission at 74 MHz, 151 MHz, and 325 MHz were
retrieved from VLSS \citep{vlss}, 7C Survey
\citep{1999MNRAS.306...31R}, and WENSS \citep{1997A&AS..124..259R},
respectively. No formal detection is found in VLSS, however, an
overdensity of emission at the location of \irs\ is evident. For
completeness, we measured a flux for the potential source, but
excluded the value during fitting of the radio spectrum. The radio
spectrum from 151--8400 MHz for the full radio source (lobes, jets,
and core) was fitted with the KP \citep{1962SvA.....6..317K, pach}, JP
\citep{1973A&A....26..423J}, and CI \citep{1987MNRAS.225..335H}
synchrotron models using the code of \citet{2005ApJ...624..656W},
which is based on the method of \citet{1991ApJ...383..554C}. The
models primarily differ in their electron pitch-angle distribution and
number of particle injection assumptions. The JP model (single
electron injection, randomized but isotropic pitch-angle distribution)
yields the best fit with \chisq(DOF)$ = 4.91(3)$, a break frequency of
$\nu_B = 12.9 \pm 1.0$ GHz, and a low-frequency ($\nu < 2$ GHz)
spectral index of $\alpha = -1.10 \pm 0.09$. The bolometric radio
luminosity was approximated by integrating under the JP curve between
$\nu_1 = 10$ MHz and $\nu_2 = 10,000$ MHz, giving $\lrad = 1.09 \times
10^{42}~\lum$. Repeating the above analysis using only radio lobe
emission at 1.4 GHz, 5.0 GHz, and an 8.4 GHz upper limit gives a
poorly constrained best-fit JP model with break frequency $\nu_B = 1.3
\pm 1.1$ GHz and spectral index $\alpha = -1.65 \pm 0.30$. The radio
spectra and best-fit models are shown in Figure \ref{fig:radio}.

%%%%%%%%%%%%%%%%%%%%%%%%%%%%%%%
\section{AGN-Halo Interaction}
\label{sec:sub}
%%%%%%%%%%%%%%%%%%%%%%%%%%%%%%%

Unsmoothed and unsharp masked 0.5--10.0 keV X-ray images of \irs\ are
shown in Figure \ref{fig:imgs}. Several systematic features are
present in the unprocessed image which become pronounced structures in
the unsharp masked image: 1) an arc of emission northwest of the
nucleus, 2) a surface brightness depression directly southeast of the
arc, 3) a northeast skewing of the nuclear emission, and 4) an
edge-like feature southeast of the core with a faint surface
brightness depression below it. All of the structures lie along the
radio jet axis suggesting interaction between the AGN and ICM.

To better reveal ICM substructure, the clean X-ray image was
adaptively smoothed with a $1\arcs$ Gaussian and residual images were
made by subtracting various smooth surface brightness models. A
primary concern with residual imaging is that differences in the model
and source ellipticities can create artifacts reminiscent of cavities,
making it difficult to interpret the physical significance of faint
residual structures. Thus, in Figure \ref{fig:multiresid}, we present
three residual images created from three models with varying degrees
of ellipticity.

The first residual image was created by fitting the 1D radial X-ray
surface brightness profile with a double $\beta$-model
\citep{betamodel} and subtracting its azimuthally symmetric 2D analog
(model-A) from the clean image. The central source was excluded during
fitting, hence its prominence in the residual image. For the second
and third residual images, the 2D X-ray surface brightness isophotes
of the clean image were fitted using the \iraf\ tool
\textsc{ellipse}. The second residual image results from a model which
had ellipticity ($\epsilon$), position angle ($\phi$), and centroid
($C$) as free parameters (model-B). The model for the third residual
image (model-C) had $\epsilon$, $\phi$, and $C$ fixed to their mean
values exterior to the substructure (see Figure \ref{fig:ellpa}).

In all three images, the surface brightness depressions northwest and
southeast of the nucleus are resolved into cavities, and excesses to
the north, east, and west of the core are evident. The presence of
excesses raises the concern that they have influenced the isophotal
fitting. We explored this possibility by masking the excesses during
all phases of fitting and found no significant differences between the
resulting residual images and those shown in Figure
\ref{fig:multiresid}. We also tried removing the excesses from the
X-ray images and repeating the analysis, but again found no
significant differences in the resulting images. The cavities and
excesses persist in spite of our best efforts to remove them and we
conclude that these are real features of \irs.

To assess the statistical significance of the substructures, we used
two methods, the results of which are given in Table
\ref{tab:sig}. For method-1, we measured the ratio of the surface
brightness of each substructure to the value of the model-B isophotes
over the same radial range \citep[see][for a similar
  approach]{hydraa}. In method-2, we compared the 0.7--7.0 keV
exposure-corrected surface brightness in $5\mydeg$ wedges around an
annulus centered on the nucleus with width equal to the radial size of
each structure. On average, the substructures have significance of
$5.3\sigma$ with all meeting the formal detection requirement of being
$>3\sigma$.

Shown in more detail in Figure \ref{fig:resid} is the model-B residual
image overlaid with radio contours. Though the substructures are
relatively weak, the nice correlation between the X-ray and radio
morphologies strongly indicates they originate from a jet-halo
interaction. The northern and western excesses (NEx and WEx,
respectively) appear to form a tenuous, arc-like filament of gas
possibly displaced by the northwest radio jet, and several knots of
significant radio emission coincide with the NEx-WEx base. The region
between the NEx and nucleus is not of uniform depth, but it lies
perfectly along the northwest jet forming a continuous cavity, and the
same is true of the region along the southeast jet. At the base of the
southeast jet, the radio emission narrows at the same location an arm
of X-ray emission impinges on the jet, forming what appears to be a
bottleneck. A detailed analysis of the cavities is provided in Section
\ref{sec:cavs}.

There are additional deficits east of the EEx and south of the WEx
which lie along the nuclear beaming axis but are further from the core
then the detected cavities. The extra decrements are unassociated with
any radio emission and their physical significance is
unclear. \citet{1995MNRAS.274L..63F} previously reported detecting a
``hole'' in a \rosat/HRI image of \irs, but, when overlaid on the
\cxo\ residual image, the hole is not associated with the cavities,
and neither the cavities nor the hole are seen in a longer follow-up
HRI observation.

H93 and H99 suggest the AGN which produced the large-scale jets has
been reoriented such that a new beaming direction roughly orthogonal
to the previous beaming axis was established no less than 70,000 years
ago. Interestingly, the new AGN axis suggested by H93/H99 is
coincident with the eastern excess (EEx), the radio spur emanating
northeast from the nucleus, a cone of scattered UV radiation, an
ionized optical nebula, and highly polarized optical emission, all of
which are highlighted in Figure \ref{fig:resid}. Thus, we suspect that
beamed nuclear radiation is interacting with the X-ray halo producing
the EEx, a possibility discussed in Sections \ref{sec:nucleus} and
\ref{sec:rad}.

%%%%%%%%%%%%%%%%%%%%%%%%%%%%%%%%%%%%%%%%%%%%%%%%%%
\section{ICM Cavities and AGN Mechanical Feedback}
\label{sec:cavs}
%%%%%%%%%%%%%%%%%%%%%%%%%%%%%%%%%%%%%%%%%%%%%%%%%%

AGN induced X-ray cavities are a well-known feature of many massive
galaxies and provide a reliable method for estimating the $PV$ work,
and hence mechanical energy output, expended by an AGN \citep[see][for
  a review]{mcnamrev}. \irs\ is unique in that it is currently the
only quasar-dominated system (and highest redshift object) with a
cavity detection. Below, we first quantify the radial properties of
the ICM (a requisite for cavity analysis) and then present discussion
of the outburst energetics.

%%%%%%%%%%%%%%%%%%%%%%%%%%%%%%%%
\subsection{ICM Radial Profiles}
%%%%%%%%%%%%%%%%%%%%%%%%%%%%%%%%

The radial profiles discussed in this section are shown in Figure
\ref{fig:gallery}. Radial temperature (\tx) and abundance ($Z$)
profiles were created by extracting spectra from circular annuli
centered on the cluster X-ray peak with each annulus containing 2500
and 5000 source counts, respectively. Background spectra were
extracted from reprocessed \caldb\ blank-sky backgrounds matched to
each observation and renormalized using the ratio of blank-sky and
observation 9.5--12 keV flux for an off-axis, source-free region. A
fixed background component was included during spectral analysis to
account for the spatially-varying Galactic foreground (see
\citealt{2005ApJ...628..655V} and \citealt{xrayband} for
method). Source spectra were grouped to 25 counts per energy channel
and fit using the model described in Section \ref{sec:xray}. A
deprojected temperature profile was generated using the
\textsc{projct} model in \xspec, but it does not significantly differ
from the projected profile which is used in all subsequent analysis.

A gas density profile was derived from a radial surface brightness
profile extracted from concentric $1\arcs$ wide circular annuli
centered on the cluster X-ray peak. Using the surface brightness and
temperature profiles, the deprojected electron density (\nelec)
profile was calculated using the technique of \citet{kriss83} which
incorporates spectral count rates and best-fit normalizations to
account for variations of gas emissivity \citep[see][for method
  details]{accept}. Gas density errors were estimated from 10,000
Monte Carlo simulations of the surface brightness profile. Total gas
pressure was calculated as $P = n \tx$, where $n = 2.3 \nH$ and $\nH =
\nelec/1.2$ for a fully ionized plasma. For completeness, profiles of
entropy ($K = \tx\nelec^{-2/3}$), cooling time ($\tcool =
3n\tx~[2\nelec \nH \Lambda(T,Z)]^{-1}$), and enclosed X-ray luminosity
(\lx) were also produced. The $\Lambda(T,Z)$ in \tcool\ are cooling
functions derived from the best-fit spectral model for each annulus of
the temperature profile and were linearly interpolated onto the grid
of the higher resolution density profile. The uncertainties for each
profile were propagated from the individual parameters and summed in
quadrature.

%%%%%%%%%%%%%%%%%%%%%%%%%%%%%%%%
\subsection{Outburst Energetics}
%%%%%%%%%%%%%%%%%%%%%%%%%%%%%%%%

Cavity volumes, $V$, were calculated by approximating each void in the
X-ray image with a right circular cylinder projected onto the plane of
the sky along the cylinder radial axis. The lengthwise axis of the
cylinders were assumed to lie in a plane that passes through the
central AGN and is perpendicular to our line-of-sight. A systematic
uncertainty of 10\% was assigned to the cavity volumes to account for
unknown projection effects. The energy in each cavity, $\ecav = \gamma
PV/(\gamma-1)$, was estimated by assuming the contents are a
relativistic plasma ($\gamma = 4/3$), and then integrating the total
ICM pressure over the surface of each cylinder. We assume the cavity
ages are equal to the time required for the jets to displace gas at
the ambient sound speed \citep{birzan04}:
\begin{equation}
  \tsonic = D\sqrt\frac{\mu \mH}{\gamma \tx}
\end{equation}
where $\gamma$ and $\mu$ are the ICM adiabatic index and mean
molecular weight, \mH\ [g] is the mass of hydrogen, and $D$ [cm] is
the distance the AGN outflow has traveled to create each cavity, which
was set to the cylinder lengths and not their midpoints, as is
common. The AGN power required to make each cavity is thus $\pcav =
\ecav/\tsonic$. Properties of the individual cavities are listed in
Table \ref{tab:cylcavities}.

The total cavity energy and power are estimated at $\ecav = 5.11 ~(\pm
1.33) \times 10^{59}$ erg and $\pcav = 3.05 ~(\pm 1.03) \times 10^{44}
~\lum$, respectively. Radio power has been shown to be a reasonable
surrogate for estimating mean jet power \citep{birzan08}, though with
considerable scatter. Thus, we checked the \pcav\ calculation using
the \citet{pjet} \pcav-\lrad\ scaling relations which give $\pcav
\approx 2 \dash 6 \times 10^{44} ~\lum$, in agreement with the X-ray
measurements, suggesting there is nothing unusual about the jet
radiative efficiency. Compared with other systems hosting cavities
\citep[\eg][]{birzan04, dunn08}, \irs\ is not unusually powerful,
residing near the middle of the cavity power distribution.

Since AGN have been implicated as a key component in regulating
late-time galaxy growth, of interest is a comparison of the AGN energy
output and the cooling rate of the host X-ray halo. The cooling radius
of the halo was set at the radius where the ICM cooling time is equal
to $H_0^{-1}$ at the redshift of \irs. We calculate $R_{\rm{cool}} =
128$ kpc, and measure an unabsorbed bolometric luminosity within this
radius of $L_{\rm{cool}} = 1.61^{+0.25}_{-0.20} \times 10^{45}
~\lum$. If all of the cavity energy is thermalized over $4\pi$ sr,
then $\approx 20\%$ of the energy radiated away by gas within
$R_{\rm{cool}}$ is replaced by energy in the observed
outburst. Assuming the mean ICM cooling rate does not vary
significantly on a timescale of $\sim 1$ Gyr, this idealized scenario
implies that 5 similar power AGN outbursts will balance the cooling
losses of the cluster halo. Similar results are found for many other
BCGs \citep[\eg][]{rafferty06}, again highlighting that the
\irs\ outburst mechanical properties are not atypical.

%%%%%%%%%%%%%%%%%%%%%%%%%%%%%%%%%%%%%%%%
\subsection{Constraints on Shock Energy}
%%%%%%%%%%%%%%%%%%%%%%%%%%%%%%%%%%%%%%%%

The energetics calculations given above are often assumed to be a good
estimate of the physical quantity jet power, \pjet. However, neither
\pcav\ nor \pjet\ account for energy which may be imparted to shocks.
But, the radio source synchrotron age (\tsync) and cavity ages are
useful in addressing this issue. Assuming \tsync\ is an accurate
measure of a cavity system's dynamical age, if \tsync\ is less than
\tsonic, one might infer that an AGN outflow is strongly driven,
possibly supersonic. Were it not, the radio-loud plasma should radiate
away much of its energy (neglecting re-acceleration) and be mostly
radio-quiet prior to reaching the end of an observed jet. The
implication being that during cavity creation, some jet kinetic energy
may be imparted to shocks, making \ecav\ a lower limit on the total
AGN energy output.

We first point-out that there are no resolved discontinuities in the
radial ICM temperature, density, or pressure profiles to suggest the
presence of large-scale shocks. Other authors have also noted the
\irs\ emission line properties are inconsistent with shock excitation
\citep{1996MNRAS.283.1003C, 2000AJ....120..562T}. But, detecting weak,
small-scale shocks in the X-ray sometimes requires very high
signal-to-noise (SN), and the nebular regions used to study shock
excitation are far from the jets and may not be indicative of gas
dynamics close to the outflow.

The radio source synchrotron age was constrained using the radio
spectra presented in Section \ref{sec:radio}. Assuming inverse-Compton
scattering and synchrotron emission are the dominant radiative
mechanisms, the time since last acceleration for an isotropic particle
population is \citep{2001AJ....122.1172S}
\begin{equation}
  \tsync = 1590 \left(\frac{B^{1/2}}{B^2 + B_{\rm{CMB}}^2}\right)~
  \left[\nu_{\rm{B}} (1+z)\right]^{-1/2} ~\Myr
\end{equation}
where $B$ [\mg] is magnetic field strength, $B_{\rm{CMB}} =
3.25(1+z)^2$ [\mg] is a correction for inverse-Compton losses to the
cosmic microwave background, $\nu_{\rm{B}}$ [GHz] is the radio
spectrum break frequency, and $z$ is the dimensionless source
redshift. Note that energy losses to adiabatic expansion have been
neglected \citep{1968ARA&A...6..321S}. We assume $B$ is not
significantly different from the equipartition magnetic field
strength, $B_{\rm{eq}}$ \citep[see][regarding the validity of this
  assumption]{birzan08}, which is derived from the minimum energy
density condition as \citep{1980ARA&A..18..165M}
\begin{equation}
  B_{\rm{eq}} = \left[\frac{6 \pi ~c_{12}(\alpha,\nu_1, \nu_2)
      ~\lrad ~(1+k)}{V \Phi}\right]^{2/7} ~\rm{\mg}
\end{equation}
where $c_{12}(\alpha,\nu_1,\nu_2)$ is a dimensionless constant
\citep{pach}, \lrad\ [$\lum$] is the integrated radio luminosity from
$\nu_1$ to $\nu_2$, $k$ is the dimensionless ratio of lobe energy in
non-radiating particles to that in relativistic electrons, $V$ [\cc]
is the radio source volume, and $\Phi$ is a dimensionless radiating
population volume filling factor. Synchrotron age as a function of $k$
and $\Phi$ for the full source, and lobes only, are shown in Figure
\ref{fig:tsync}. For various combinations of $k$ and $\Phi$, the ages
range from $\approx 2 \dash 30$ Myr, not atypical of other BCG radio
sources \citep[\eg][]{birzan08}.

A comparison of the radio source and cavity ages reveals $\tsonic \ga
42$ Myr and $\tsync \la 30$ Myr, the situation where a supersonic
outflow may be present. Relative to the ICM sound speed, the velocity
needed to reach the end of the radio jet in 30 Myrs requires a Mach
number of $\approx 1.7$. The energy possibly channeled into shocks was
crudely estimated by setting $t_{\rm{sonic}} = 30$ Myr and adjusting
the pressure at the end of the cavities by the Mach number, $\Delta
\pcav = \Delta P / \Delta t$ and $\Delta P \propto M^3$, giving $\pjet
\sim 10^{45} ~\lum$ and $\ecav \sim 10^{60}$ erg for a 30 Myr
duration. These are large values for an AGN outburst, but the inferred
Mach number is not outside the range of observed values in other
systems \citep[\eg][]{2005ApJ...635..894F, ms0735, hydraa,
  2009MNRAS.395.1999C}. If the midpoints of the cavities are instead
used for calculating cavity ages, then $\tsonic \approx 30$ Myr and
the outflow does not need to be supersonic. Given the uncertainties in
determination of the cavity and radio source ages, there exists a
reasonable possibility of weak ICM shocking.

%%%%%%%%%%%%%%%%%%%%%%%%%%%
\section{The \irs\ Nucleus}
\label{sec:nucleus}
%%%%%%%%%%%%%%%%%%%%%%%%%%%

At the heart of the AGN outburst is the complex nucleus of
\irs. Below, we analyze the nucleus for the later purpose of
investigating the irradiation of the ICM and to help constrain which
processes may be fueling the AGN activity.

%%%%%%%%%%%%%%%%%%%%%%%%%%%%%%%%
\subsection{Spectral Properties}
%%%%%%%%%%%%%%%%%%%%%%%%%%%%%%%%

The centroid and extent of the nuclear X-ray source were determined
with the \ciao\ tool {\tt wavdetect} and confirmed with visual
inspection of a hardness ratio map calculated as $HR = f(2.0 \dash 9.0
~\keV) / f(0.5 \dash 2.0 ~\keV)$, where $f$ is the flux in the denoted
energy band. Comparison of surface brightness profiles for the nuclear
source and normalized \cxo\ PSF specific to the nuclear source median
photon energy and off-axis position confirm the source is
point-like. A source extraction region was defined using the 90\% EEF
of the PSF (effective radius of $1.16\arcs$), and an identical
segmented elliptical annulus with 5 times the area of the source
region was used for extracting a background spectrum. The background
annulus was broken into segments to avoid the regions of excess X-ray
emission discussed in Section \ref{sec:sub}. The $HR$ map and
extraction regions are shown in Figure \ref{fig:hardness}.

For each \cxo\ observation, source and background spectra were
extracted using the \ciao\ tool {\tt psextract} and grouped to have 20
counts per energy channel. Previous studies have shown that the
nucleus is obscured by a Compton thick screen, and that the X-ray
emission is in fact reflected quasar radiation with a strong
\feka\ fluorescence line ($E_{\rm{rest}} = 6.4$ keV) component
\citep[\eg][]{2000A&A...353..910F, 2001MNRAS.321L..15I}. Thus, the
1999 and 2009 spectra (hereafter SP99 and SP09, respectively) were
fitted separately in \xspec\ over the energy range 0.5--7.0 keV with
an absorbed \pexrav\ model \citep{pexrav} plus three Gaussians to
account for the \feka\ line and two additional line-like features
around 0.8 keV and 1.3 keV \citep[see
  also][]{2001MNRAS.321L..15I}. Note that the iridium edge feature of
the ACIS detector is at $\approx 2.0$ keV, far from any of the
observed emission features.

The disk-reflection geometry employed in the \pexrav\ model is not
ideal for fitting reflection from a Compton-thick torus
\citep{2009MNRAS.397.1549M}, but no other suitable \xspec\ model is
currently available. Hence, only the reflection component of the
\pexrav\ model was used and the power law had no high energy
cut-off. Fitting separate SP99 and SP09 models allowed for source
variation in the decade between observations, however $\Gamma$ was
poorly constrained for SP99 and thus fixed at the SP09 value. Using
constraints from \citet{1997A&A...318L...1T} and
\citet{2000AJ....120..562T}, the model parameters for reflector
abundance and source inclination were fixed at $1.0 ~\Zsol$ and $i =
50\mydeg$, respectively. The best-fit model parameters are given in
Table \ref{tab:nucspec}, and the background-subtracted spectra
overplotted with the best-fit models are presented in Figure
\ref{fig:nucspec}.

The 2--10 keV {\it{reflected}} flux of our best-fit model without
Galactic absorption is $4.24^{+0.57}_{-0.55} \times 10^{-13} ~\flux$,
corresponding to a 2--10 keV rest-frame luminosity
$L^{\rm{refl}}_{2-10} = 1.57^{+0.19}_{-0.19} \times 10^{44} ~\lum$ and
bolometric (0.01--100.0 keV) luminosity $L_{\rm{bol}}^{\rm{refl}} =
4.20^{+0.49}_{-0.47} \times 10^{45} ~\lum$. Since the reflection
component is the only directly measured quantity, the intrinsic quasar
luminosity, \lqso, can only be inferred. If the reflector scattering
albedo is $\eta \la 0.1$, a reasonable assumption for systems with
properties similar to \irs\ \citep{2009MNRAS.397.1549M}, and less than
half of the reflector solid angle is exposed to our line of sight,
then the intrinsic luminosity may be more than 20 times the calculated
bolometric value. Thus, we hereafter assume $\lqso \approx 8 \times
10^{46} ~\lum$, consistent with \irs's 0.3--70$\mu$m luminosity of
$\approx 5 \times 10^{46} ~\lum$ which is attributed to dust
reprocessing of quasar radiation \citep[][H99]{1988ApJ...328..161K}.

Strong Mg, Ne, S, and Si K$\alpha$ fluorescence lines at $E < 3.0$
keV, in addition to photoionized Fe L-shell lines, can be present in
reflection spectra \citep{1990ApJ...362...90B,
  1991MNRAS.249..352G}. Given that the nucleus is extremely luminous
and that the spectrum is reflection-dominated, we find it likely that
the soft X-ray emission fitted by the two separate Gaussians
represents some combination of these emission lines. Using a solar
abundance thermal component in place of the two low-energy Gaussians
yielded a statistically worse fit: the model systematically
underestimated the 1--1.5 keV flux and overestimated the 2--4 keV
flux. Leaving the thermal component abundance as a free parameter
resulted in $0.1 ~\Zsol$, \ie\ the thermal component tended toward a
featureless, skewed-Gaussian.

Our measurement of the \feka\ line equivalent width (\fekaew) agrees
with previous studies which found $\fekaew \la 1$ keV
\citep{2000A&A...353..910F, 2001MNRAS.321L..15I, 2007A&A...473...85P},
but the large uncertainties prevent us from determining if
\fekaew\ has varied since 1998. Our \fekaew\ measurement is also
consistent with AGN models and observations which show that $\fekaew
\ga 0.5$ keV is correlated with $\Gamma \ga 1.7$, and implies that the
column density of the reflecting material is $\nhref \sim 10^{24}
~\pcmsq$ \citep{1996MNRAS.280..823M, 1997ApJ...477..602N,
  1999MNRAS.303L..11Z, 2000PASP..112.1145F, 2005A&A...444..119G}. The
high inferred \nhref\ is relevant to the properties of the nuclear
obscurer, which we discuss next.

%%%%%%%%%%%%%%%%%%%%%%%%%%%%%%%%
\subsection{Nuclear Obscuration}
%%%%%%%%%%%%%%%%%%%%%%%%%%%%%%%%

Conclusions reached in previous studies regarding the \irs\ nucleus
have relied on a \bepposax\ hard X-ray detection which has been
interpreted as emission transmitted through a moderately Compton-thick
obscuring screen, \ie\ $\nhobs > 10^{24} ~\pcmsq$
\citep{2000A&A...353..910F, 2001MNRAS.321L..15I}. This result has been
questioned by \citet{2007A&A...473...85P}, and we confirmed the
original \bepposax\ analysis of \citet{2000A&A...353..910F} by
analyzing the archival \bepposax\ \irs\ observation with
\saxdas\ version 2.3.1 using the calibration data, cookbook, and epoch
appropriate response functions available from
HEASARC\footnote{http://heasarc.nasa.gov/docs/sax}. In agreement with
their analysis, we measure a 15--80 keV count rate of $0.106 \pm
0.055$ ct \ps\ for the PDS hard X-ray instrument, and estimate 10--200
keV and 20--100 keV fluxes of $f_{10 \dash 200} = 2.09^{+1.95}_{-1.95}
\times 10^{-11} ~\flux$ and $f_{20 \dash 100} = 1.10^{+1.57}_{-1.63}
\times 10^{-11} ~\flux$.

Extrapolating the best-fit model for the \cxo\ data out to 10--80 keV
reveals statistically acceptable agreement with the PDS data (see
Figure \ref{fig:resid}). The best-fit model 10--200 keV flux is $f_{10
  \dash 200} = 8.15^{+0.21}_{-0.19} \times 10^{-12} ~\flux$, which is
not formally different from the \bepposax\ 10--200 keV flux. However,
if transmitted hard X-ray emission is present below the 7.0 keV
cut-off selected for spectral analysis, the best-fit power-law
component might be artificially shallower than its intrinsic value, and
the hard X-ray flux will thus be overestimated. We tested this
possibility by simulating absorbed power-law spectra and found that
for $\Gamma \ge 1.7$, column densities $> 3 \times 10^{24} ~\pcmsq$
are sufficient to suppress emission below 7 keV. Consistent with this
result, addition of an absorbed, power-law component to the modeling
of the real data lowered \chisq\ (best-fit $\nhobs = 3 \times 10^{24}
~\pcmsq$ and $\Gamma = 1.7$), but with no improvement to the goodness
of fit derived from 10,000 Monte Carlo simulations.

That we find no need for an additional hard X-ray component does not
contradict the well-founded conclusion that \irs\ harbors a
Compton-thick quasar. On the contrary, the measured \fekaew\ suggests
reflecting column densities of $\nhref \sim 1 \dash 5 \times 10^{24}
~\pcmsq$ \citep{1993MNRAS.263..314L, 2005A&A...444..119G,
  2010arXiv1005.3253C}. Assuming the obscuring nuclear material is
nearly homogeneous, then $\nhref \approx \nhobs$ and our results are
consistent with the presence of a moderately Compton-thick obscuring
screen.

The \bepposax/PDS full-width half maximum field is quite large
($\approx 1\mydeg$), raising the concern that the hard X-ray emission
is not associated with \irs. The X-ray instruments on-board
\integral\ and \swift\ provide a check of this possibility. \irs\ was
not detected in the \integral/IBIS Extragalactic AGN Survey
\citep{2006ApJ...636L..65B}, and our re-analysis of archival
\integral\ data yielded a 20--100 keV $3\sigma$ upper limit of $f_{20
  \dash 100} = 5.70 \times 10^{-11} ~\flux$, higher than the 20--100
keV \bepposax/PDS measured flux and consistent with a $z=0.44$ source
which would not be detected in the IBIS Survey. \irs\ was also not
detected in the 22 month \swift/BAT Survey \citep{2010ApJS..186..378T}
which has a 14--195 keV $4.8\sigma$ detection limit of $2.2 \times
10^{-11} ~\flux$, higher than the \bepposax\ 14--195 keV
flux. Assuming the \integral\ and \swift\ upper limits are valid for
an $\approx 1\mydeg$ region around \irs, the lack of detected hard
X-ray sources suggests the \bepposax\ detection did not originate from
a {\it{brighter}} off-axis source.

%%%%%%%%%%%%%%%%%%%%%%%%%%%%%%%%%%%%%%%%%%%%%%%%%
\section{ICM Excesses and AGN Radiative Feedback}
\label{sec:rad}
%%%%%%%%%%%%%%%%%%%%%%%%%%%%%%%%%%%%%%%%%%%%%%%%%

%%%%%%%%%%%%%%%%%%%%%%%%%%%%%%%%%%%%%%%%%%
\subsection{Spectral Analysis of Excesses}
%%%%%%%%%%%%%%%%%%%%%%%%%%%%%%%%%%%%%%%%%%

Having constrained the properties of the nucleus and mechanical
outflow, it is now possible to analyze the NEx, WEx, and EEx ICM X-ray
excesses (see Section \ref{sec:sub}) and evaluate their association
with AGN feedback. A source spectrum was extracted for each excess,
and a background spectrum was extracted from regions neighboring each
excess which show minimal enhanced emission in the residual image
(regions shown in Figure \ref{fig:exspec}). The source regions were
selected by moving out from the peak of each excess and finding the
isocontour one pixel interior to where the residual normalized flux
declined by more than a factor of two. This is an arbitrary choice
made to ensure the ``core'' of the excesses was analyzed. For each
region, the ungrouped source and background spectra were differenced
within \xspec\ and fitted with \xspec's modified Cash statistic
\citep{1979ApJ...228..939C}, appropriate for low-count,
background-subtracted spectra \citep[see \xspec\ Manual Appendix B
  and][]{1989ApJ...342.1207N}. Metal abundance was fixed at $0.5
~\Zsol$ because the low-SN of each residual spectrum prohibited
setting it as a free parameter. Varying the fixed abundance by $\pm
0.2 ~\Zsol$ changed the output temperatures and normalizations within
the statistical uncertainties when $0.5 ~\Zsol$ was assumed. The
best-fit spectral models are given in Table \ref{tab:excess}.

Analysis of the NEx residual spectrum was inconclusive due to
extremely low-SN. However, the northern radio jet terminates in the
NEx region, and the hardness ratio map shows a possible hot spot in
this same area. It is possible the NEx results from the presence of a
very hot thermal or non-thermal phase associated with the radio
lobe. The WEx has a residual spectrum consistent with thermal emission
from gas cooler than its surroundings, and is rather unremarkable
apart from its unclear association with the outflow and the NEx.

For the EEx spectrum, no combination of thermal models was able to
reproduce the flux associated with prominent emission features at $E <
2$ keV, resulting in large systematic trends in the fit statistic
residuals. The EEx thermal \feka\ complex was also poorly fit because
of an obvious asymmetry toward lower energies. To address these
spectral features, the EEx was modeled with three Gaussians and a
single thermal component. Comparison of \chisq\ goodness of fits
determined from 10,000 Monte Carlo simulations for the best-fit models
with and without the Gaussians indicate the model with the Gaussians
is preferred. In the next section we discuss the EEx further.

%%%%%%%%%%%%%%%%%%%%%%%%%%%%%%%%%%%%%%%%%%%
\subsection{Quasar Irradiation of the Halo}
%%%%%%%%%%%%%%%%%%%%%%%%%%%%%%%%%%%%%%%%%%%

As pointed-out in Section \ref{sec:sub}, the EEx is coincident with a
number of multiwavelength emission features and the nuclear beaming
direction, indicating the EEx may arise from quasar irradiation of the
halo. \citet{2010MNRAS.402.1561R} have demonstrated that the quasar in
H1821+643 is capable of photoionizing gas up to 30 kpc from the
nucleus, and we suspect a similar process is occurring in \irs. To
test this hypothesis, we adopted the approach of
\citet{2010MNRAS.402.1561R} by simulating reflection and diffuse
spectra for the nebula and ICM coincident with the EEx using
\cloudy\ \citep{cloudy}. The nebular gas density and ionization state
were taken from \citet{2000AJ....120..562T}, while the initial ICM
temperature, density, and abundance were set at 3 keV, 0.04 \pcc, and
0.5 \Zsol, respectively. No Ca or Fe lines are detected from the
nebula coincident with the EEx, possibly as a result of metal
depletion onto dust grains \citep[\eg][]{1993ApJ...414L..17D}, while
strong Mg, Ne, and O lines are \citep{2000AJ....120..562T}. Thus, a
metal depleted, grain-rich, 12 kpc thick nebular slab was placed 15
kpc from an attenuated $\Gamma = 1.7$ power law source with power $8
\times 10^{46} ~\lum$ (see Section \ref{sec:nucleus}). Likewise, a $17
~\kpc \times 16 ~\kpc$ ICM slab was placed 19 kpc from the same
source. The quasar radiation was attenuated using a 15 kpc column of
density 0.06 \pcc, abundance 0.5 \Zsol, and temperature 3 keV. The
output models were summed, folded through the \cxo\ responses using
\xspec, and normalized to the observed EEx spectrum (shown in Figure
\ref{fig:qso}).

In the energy range 0.1--10.0 keV, the nebular emission lines which
exceed thermal line emission originate from Si, Cl, O, F, K, Ne, Co,
Na, and Fe and occur as blends around redshifted 0.4, 0.6, 0.9, and
1.6 keV. The energies and strengths of these blends nicely coincide
with the $E < 2$ keV emission humps in the EEx spectrum. Further, the
\feka\ emission from the nebula is 100 times fainter than that from
the ICM, and the observed asymmetry of the EEx \feka\ emission results
from the 6.4 keV \feka\ photoionized line of the ICM. The consistency
of the irradiation model with the observed EEx spectrum, and the
coincidence with other emission features like the radio spur, strongly
suggests that beamed quasar radiation is indeed responsible for the
nature of the EEx. Whether this irradiation translates to heating of
the gas is unclear given the limitations of the data.

%%%%%%%%%%%%%%%%%%%%%%%%
\section{Interpretation}
\label{sec:interp}
%%%%%%%%%%%%%%%%%%%%%%%%

We have presented evidence that the AGN/QSO in \irs\ is interacting
with its surroundings through the radiative and mechanical feedback
channels, and this remarkable galaxy may be providing important clues
about how AGN feedback fundamentally operates. But, because \irs\ is a
very unique BCG, before we can interpret this object in a broader
context, it is important to determine if the host cluster appears
exceptional in any way, check for corroborating evidence of radiative
quasar feedback, explore how the feedback may have been fueled, and
attempt to understand why the feedback mode may be evolving. In this
section, we explore all of these issues. Hereafter, we assume that
relativistic AGN emission, \eg\ jets and beamed radiation, emerge
along the spin axis of a SMBH.

%%%%%%%%%%%%%%%%%%%%%%%%%%%%%%%%%%%%
\subsection{Host Cluster Properties}
%%%%%%%%%%%%%%%%%%%%%%%%%%%%%%%%%%%%

The mean cluster temperature, \cltx, was defined as the ICM
temperature within the core-excised aperture $0.15 \dash 1.0~\rf$
where \rf\ is the radius where the average cluster density is 500
times the critical density for a spatially flat Universe. The cluster
cool core was excluded using a region $0.15 ~\rf$ in size
\citep{2007ApJ...668..772M}. A source spectrum and weighted responses
were created for the core-excised aperture, and a background spectrum
was extracted from the reprocessed \caldb\ blank-sky backgrounds. The
relations from \cite{2002A&A...389....1A} were used to iteratively
calculate \rf\ until three consecutive iterations produced
\cltx\ values which agreed within the 68\% confidence intervals. We
measure $\cltx = 7.54^{+1.76}_{-1.15}$ keV corresponding to $\rf =
1.16^{+0.27}_{-0.19}~\Mpc$, and the ratio of the hard-band (2.0--7.0
keV) to broadband (0.7--7.0 keV) temperature is consistent with unity,
implying the cluster may not be a merger system
\citep[\eg][]{xrayband}.

The cluster gas and gravitational masses were derived assuming
hydrostatic equilibrium and using the deprojected radial electron
density and temperature profiles. Because the radial profiles extend
to $\approx 0.3 \rf$, the mass calculations below include significant
extrapolation, and thus may be lower limits. Electron gas density was
converted to total gas density as $n_g = 1.92 \nelec \mu \mH$. To
ensure continuity and smoothness of the radial log-space derivatives,
the density profile was fitted with a double $\beta$-model, and the
temperature profile was fitted with the 3D-$T(r)$ model of
\citet{2006ApJ...640..691V}. The total gas mass within \rf\ is
$\approx 5.2 ~(\pm 0.4) \times 10^{13} ~\msol$, and the total
gravitating mass is $\approx 4.7 ~(\pm 0.9) \times 10^{14}
~\msol$. The gas to gravitating mass ratio is thus $0.11 \pm
0.02$. Errors were estimated from 10,000 Monte Carlo realizations of
the density and temperature profiles.

The \irs\ host cluster, \rxj, has a temperature, luminosity, and gas
fraction consistent with flux-limited and representative cluster
samples \citep{hiflugcs2, 2009A&A...498..361P}. Fitting the function
$K(r) = \kna +\khun (r/100 ~\kpc)^{\alpha}$ to the entropy profile
reveals best-fit values of $\kna = 12.6 \pm 2.9 ~\ent$, $\khun = 139
\pm 8 ~\ent$, and $\alpha = 1.71 \pm 0.10$ for \chisq(DOF)$ =
3.0(49)$, typical of cool core clusters and the population of $\kna <
30 ~\ent$ clusters that have a radio-loud AGN and multiphase gas
associated with the BCG \citep{haradent, rafferty08, accept,
  2009MNRAS.395..764S}. If one ignores the strange BCG at its heart,
\rxj\ appears to be morphologically and spectroscopically a typical
massive, relaxed galaxy cluster. With no reason to suspect the host
cluster is unique in anyway, it appears the peculiar nature of
\irs\ does not arise from the galaxy residing in a special cluster.


%%%%%%%%%%%%%%%%%%%%%%%%%%%%%%%%%%%%%%%%%%%%%%%%%%%%%%
\subsection{Supplementary Evidence of Quasar Feedback}
%%%%%%%%%%%%%%%%%%%%%%%%%%%%%%%%%%%%%%%%%%%%%%%%%%%%%%

AGN have three primary channels for interacting with their environment
-- jets, non-relativistic winds, and radiation pressure -- and it is
suspected these channels are active at different phases in a galaxy's
evolution. The jets from \irs\ are clearly impacting the X-ray halo,
and below we argue that winds and radiation may be affecting gas
within the galaxy, indicating that all three AGN channels may be
simultaneously active, further pointing to \irs\ as a rare transition
object.

\irs's status as a ULIRG, and its $> 10^{42} ~\lum$
\halpha\ luminosity, would suggest that the galaxy hosts $> 10^{10}
~\msol$ of cold gas \citep[\eg][]{1988ApJ...325...74S, edge01}. On the
contrary, the H$_2$ mass of \irs\ is $\approx 3 \times 10^{9} ~\msol$,
there is $< 5 \times 10^7 ~\msol$ of cold dust, the hot dust mass is
$\approx 1 \times 10^9 ~\msol$, and no polycyclic aromatic hydrocarbon
or silicate absorption features are detected in the galaxy's IR
spectrum, though the huge IR luminosity may dilute their signal
\citep{1997A&A...318L...1T, 2004ApJ...613..986P, 2008ApJ...683..114S,
  2010arXiv1009.2040C}. In spite of being a ULIRG BCG in a cool core
cluster, \irs\ appears to be relatively gas-poor with a low
gas-to-dust ratio. One possible explanation for the apparent lack of
cold gas is that outflows driven by non-relativistic winds are
breaking apart the gas reservoirs \citep[\eg][]{2010MNRAS.401....7H},
and, because rapid and extensive dust formation is expected in such
winds \citep{2002ApJ...567L.107E}, may explain the \irs\ dust
richness. Integral field spectroscopy indicates the presence of a $>
1000 ~\kmps$ emission line outflow coincident with the \irs\ nucleus
\citep{1996MNRAS.283.1003C}, and the CO upper limit of
\citet{1998ApJ...506..205E} and nominal detection by
\citet{2010arXiv1009.2040C} do not exclude the existence of a
high-velocity ($> 1500 ~\kmps$) component which could result from
small-scale quasar wind-driven shocks
\citep[\eg][]{2010A&A...518L.155F}. The quasar model of
\citet{2005ApJ...619...60L} predicts $\approx 5\%$ of \lqso\ goes into
strong winds, so for \irs, the quasar wind power may be $\sim 10^{45}
~\lum$, and, assuming the quasar lifetime is less than the cavity
ages, the total wind energy may be $\sim 10^{60}$ erg. If the
\irs\ intergalactic medium (IGM) gas mass is $< 10^{10} ~\msol$ and
its temperature range is $< 6000$ K, the IGM thermal energy is well
below the $10^{60}$ erg wind energy, making it possible for
small-scale strong shocks to be driven through the galaxy, coupling
ambient gas reservoirs to the winds and potentially destroying
accretion fuel.

Pressure from the quasar radiation field will also influence gas in
the galaxy, and in systems like \irs\ where the dust content is high,
the pressure may be more efficiently absorbed because the quasar
Eddington luminosity is effectively lowered: $\leff = \lqso (1.38
\times 10^{38} ~\mbh)^{-1} ~\erg^{-1} ~\s ~\msol$
\citep[\eg][]{1993ApJ...402..441L}. In Figure \ref{fig:f09} is a
diagram presented in \citet[][hereafter F09]{2009MNRAS.394L..89F}
showing obscuring column density as a function of \leff\ for AGN host
galaxies selected from the 9-month \swift/BAT Survey. The diagram is
divided into regions where dusty obscuring clouds are either
long-lived, expelled, or appear as dust lanes (see F09 for
details). In the formalism of F09, \irs\ has $\leff \approx 0.55$, and
in the \nhobs-\leff\ plane, resides toward the far-side of the
long-lived region nearer the expulsion region than most other systems
of similar \nhobs, suggesting gas with a sight line to the quasar may
be subject to forces capable of accelerating galactic gas clouds away
from the quasar, potentially depriving the quasar of additional
fuel. However, this conclusion is at the mercy of our choice for
\mbh\ and assumed \lqso, \ie\ if $\mbh \ge 4 \times 10^9 ~\msol$ and
$\lqso \sim 4 \times 10^{46} ~\lum$, then $\leff < 0.08$, pushing
\irs\ deeper into the long-lived region of the \nhobs-\leff\ plane,
though still with a higher \leff\ than AGN of similar \nhobs.

%%%%%%%%%%%%%%%%%%%%%%%%%%%%%%%%%
\subsection{Fueling the Feedback}
%%%%%%%%%%%%%%%%%%%%%%%%%%%%%%%%%

Assuming the nucleus hosts a single SMBH, an estimate of its mass,
\mbh, is needed prior to investigating how the AGN feedback is
fueled. We note that black hole mass estimates for BCGs are
non-trivial and using scaling relations calibrated for lower mass
systems may result in \mbh\ being underestimated
\citep[\eg][]{2007ApJ...662..808L, 2009ApJ...690..537D}. We calculated
\mbh\ using the \citet{2007MNRAS.379..711G} relations which rely on
host galaxy $[B,R,K]$-band magnitudes. These magnitudes were collected
from HyperLeda, SDSS, and 2MASS and corrected for extinction,
redshift, and evolution using the relations of \citet{cardelli89} and
\citet{poggianti97}. We find a \mbh\ range of $0.5 \dash 5.2 \times
10^9 ~\msol$ and adopt the weighted mean value of $1.1^{+4.1}_{-0.5}
\times 10^9 ~\msol$ where the uncertainties span the lowest and
highest $1\sigma$ values of the individual estimates. The Eddington
accretion rate, which is the maximal inflow rate of gas not expelled
by radiation pressure, for a black hole of this mass is
\begin{equation}
  \dmedd = \frac{2.2}{\epsilon} \left(\frac{\mbh}{10^9~\msol}\right)
  \approx 23^{+91}_{-12} ~\msolpy
\end{equation}
where we use an accretion disk radiative efficiency of $\epsilon =
0.1$. Hereafter, we define normalized accretion rates as $\lmdot
\equiv \dmacc/\dmedd$.

If mass accretion alone is the dominant power source for the feedback,
and not, for example, SMBH spin \citep{2002NewAR..46..247M, minaspin},
then \pcav\ and \lqso\ are directly related to the gravitational
binding energy released by matter accreting onto the SMBH. The
aggregate cavity energy implies a total accreted mass of $\macc =
\ecav/(\epsilon c^2)$ with a mean accretion rate of $\dmacc =
\macc/t_{\rm{sonic}}$ where $\epsilon$ is a mass-energy conversion
efficiency. Using the widely accepted $\epsilon = 0.1$
\citep{2002apa..book.....F}, the mechanical outflow may have resulted
from the accretion of $2.9 ~(\pm 0.7) \times 10^{6} ~\msol$ of matter
at a rate of $0.054 \pm 0.004 ~\msolpy$ or $\lmdot \approx 0.002$. On
the other hand, the mass accretion rate required to power the quasar
is $\dmaccqso = \lqso/(\epsilon c^2) \approx 14 ~\msolpy$ or $\lmdot
\approx 0.6$.

The origin of gas driving the nuclear activity cannot be precisely
known, but how the gas is accreted can be constrained. If the
accretion flow feeding the SMBH is composed of the hot ICM, it can be
characterized in terms of the Bondi accretion rate
\begin{equation}
  \dmbon = 0.013 ~\kbon^{-3/2} \left(\frac{\mbh}{10^9
    ~\msol}\right)^{2} \approx 3.2^{+17.9}_{-2.5} \times 10^{-4}
  ~\msolpy
\end{equation}
where $\kbon$ [\ent] is the mean entropy of gas within the Bondi
radius and we have assumed $\kbon = \kna$. Considering only the
demands of the jets, the Bondi ratio of such an accretion flow is
$\dmacc/\dmbon \approx 200^{+900}_{-100}$, disturbingly large and
implying highly efficient hot gas accretion. However, the Bondi radius
for \irs\ is unresolved, and $\kbon$ is likely less than
\kna. Assuming gas close to \rbon\ is no cooler than one-third the
central temperature, \ie\ $\approx 1$ keV, and that the black hole
mass is underestimated, \ie\ $\mbh = 5.2 \times 10^9 ~\msol$, for a
Bondi ratio near unity, $\kbon$ needs to be less than 3 \ent, 4 times
lower than \kna. In terms of entropy, gas cooling time is $\propto
K^{3/2} ~\tx^{-1}$ \citep{d06}, which for $K \sim 3 ~\ent$ suggests
the accreting material will have $\tcool \approx 100$ Myr, a factor of
$\approx 3.5$ below the shortest ICM cooling time and of order the
core free-fall time ($t_{\rm{ff}}$). But this creates the problems
that the gas should fragment and form stars (since $\tcool \sim
t_{\rm{ff}}$), and is disconnected from cooling at larger radii,
breaking the feedback loop \citep{2006NewA...12...38S}.

If instead cold-mode accretion dominates \citep{pizzolato05}, then the
gas which becomes fuel for the AGN is distributed in the BCG halo and
migrates to the bottom of the galaxy potential in the form of cold
blobs and filaments \citep{2010MNRAS.408..961P}. The cluster core
hosting \irs\ resides in the $\kna \la 30 ~\ent$ regime in which
thermal electron conduction is too inefficient to suppress widespread
environmental cooling \citep{conduction}, allowing such cooling
subsystems to form and be long-lived. Indeed, radial filaments and
gaseous substructure within 30 kpc of \irs\ are seen down to the
resolution-limit of \hst\ \citep{1999Ap&SS.266..113A}. This may
indicate the presence of cooling, overdense regions similar to those
expected in the cold-mode accretion model. Though Bondi accretion
cannot be ruled out, it does not seem viable and the process of
cold-mode accretion appears to be more consistent with the properties
of \irs's halo. These results are not in conflict with prevailing
models of AGN fueling and suggest the mechanism which fueled the
outburst need not be exotic.

%%%%%%%%%%%%%%%%%%%%%%%%%%%%%%%%%%%%%%%%%%%%
\subsection{Evolution of the Feedback Mode?}
%%%%%%%%%%%%%%%%%%%%%%%%%%%%%%%%%%%%%%%%%%%%

Using galaxy formation models as a template, if \irs\ is an example of
a transition object, then its nature may originate from evolution of
the nuclear accretion properties. In the advection-dominated accretion
flow model \citep{adaf}, for any given \mbh, the competition between
AGN radiative and mechanical output may hinge on a critical accretion
rate, \dmcrit, which dictates when accretion disk winds suppress the
formation of jets. When $\lmdot > \dmcrit$, accretion power is
primarily radiated away, and when $\lmdot < \dmcrit$, accretion power
is mostly removed in a mechanical outflow
\citep{1997ApJ...489..865E}. The observational constraints on
\dmcrit\ are loose, but studies of Galactic X-ray binaries and
clusters with cavities indicate $\dmcrit \sim 10^{-2}$
\citep[\eg][]{2003MNRAS.344...60G, 2008NewAR..51..733N, minaspin}. Not
surprisingly, the normalized accretion rate needed to power the
\irs\ quasar ($\lmdot \sim 1$) is well above \dmcrit, while the
accretion rate for the jets ($\lmdot \sim 10^{-3}$) is well below
it. But, H93 argue that radio properties of the jets and nucleus
indicate the misalignment between the beamed nuclear emission and the
large-scale jets likely transpired no more than $\sim 1$ Myr ago but
no less than $\sim 0.1$ Myr. So, if the nature of \irs\ is related to
evolution of its accretion flow properties, a relatively rapid
thousand-fold change in the normalized accretion rate may be hard to
explain.

Interestingly, the beaming misalignment may provide a constraint on
the scenario which led to such a dramatic change. The nuclear emission
indicates an active quasar, so it may be that during a prior period of
highly sub-Eddington accretion when the jets were launched, the black
hole may have received a supercritical inflow of gas \citep[a
  hypothesized fueling mechanism for high-redshift quasars,
  \eg][]{2005ApJ...633..624V} that sent the accretion rate over
\dmcrit\ and turned it into a quasar, subsequently quenching the
jets. For a slowly-spinning black hole, an asymmetric supercritical
accretion event can potentially produce dramatic changes in SMBH
angular momentum and the orientation of its spin axis (Cavagnolo et
al. 2010, in preparation), thereby producing the type of jet-beamed
radiation misalignment observed in \irs. Furthermore, this scenario
fits nicely with the AGN spin-evolution model of \citet{gesspin} which
predicts the process of retrograde accretion will 1) produce SMBH's
with low angular momentum and 2) coincides with the suppression of
jets by disk winds, similar to what we speculate may be occurring in
\irs.

Another possible explanation for the properties of \irs\ is that there
are multiple SMBHs in the nucleus (testable with upcoming {\it{VLBA}}
observation), each with its own accretion system and beaming axis: one
quasar system, one jetted system. If there is only one SMBH and the
jets and quasar are operating simultaneously, which we cannot
definitively rule-out, it would be inconsistent with the
widely-accepted picture of jets being associated with thick-disks and
quasars with thin-disks. We also cannot neglect that the black hole
spin axis might have been altered by a succession of misaligned
accretion disks \citep[\eg][]{2005MNRAS.363...49K} or a merger induced
``spin-flip'' \citep{2002Sci...297.1310M}. The latter is problematic,
however, since SMBH mergers may eject the final black hole from its
host galaxy \citep[\eg][]{2007ApJ...659L...5C}, and merging black hole
spin axes may naturally align in gas-rich environments
\citep{2007ApJ...661L.147B}. Further, the jets are highly-linear and
the ICM appears quite relaxed, and since mergers have been implicated
in producing long-lived X-ray substructures like cold fronts or shocks
\citep{2007PhR...443....1M} and inducing ICM bulk motions that result
in deformed radio sources \citep{2009A&A...495..721S} --
characteristics \irs\ does not possess -- in general, the influence of
mergers on \irs\ is questionable. There are, however, several objects
within a projected 80 kpc of the nucleus which may be companion galaxy
remnants \citep{1996AJ....111..649S, 1999Ap&SS.266..113A}, so one or
more mergers having taken place in the last few Myrs cannot be ruled
out.

%%%%%%%%%%%%%%%%%
\section{Summary}
\label{sec:summ}
%%%%%%%%%%%%%%%%%

In this paper we have shown that the quasar in \inine\ is interacting
with its environment through both the mechanical and radiative
feedback pathways. For the first time, we have a direct measurement of
the radiative to mechanical feedback ratio in a single system, and may
be peaking into the process of how massive galaxies at higher
redshifts evolve from quasar-mode into radio-mode. If \irs\ is typical
of transition objects, our results suggest that as these systems
evolve they may cycle between quasar-mode/radio-mode before becoming
\fri's, or, that \fri's occasionally become \frii's when enough gas is
accreted. It is essential to note that the radiative and mechanical
properties of \irs\ are not atypical, and, taken individually,
\irs\ appears to be: a normal BCG, a normal obscured quasar, and a
normal radio galaxy. \irs\ just so happens to be all these things at
once. The key results in this paper are:
\begin{itemize}
\item ICM cavities have been discovered which indicate an AGN outburst
  with total mechanical power and energy of $\pcav \approx 3 \times
  10^{44} ~\lum$ and $\ecav \approx 5 \times 10^{59}$ erg,
  respectively. Comparison of the radio source synchrotron age and
  cavity dynamical ages indicate that the AGN outflow may be
  supersonic, leading us to speculate that weak ICM shocking may have
  occurred, potentially driving the total AGN power and energy output
  up to $\sim 10^{45} ~\lum$ and $10^{60}$ erg, respectively.
\item Detection and modeling of an X-ray excess northeast of the
  nucleus, which is co-spatial with the quasar beaming direction and
  several multiwavelength emission features associated with the
  nucleus, strongly indicates quasar radiation is interacting with the
  ICM and a nebula in the same region.
\item The mass accretion rates required to power the AGN suggest that
  the fuel feeding the SMBH was likely not accreted directly from the
  hot ICM, \ie\ via the Bondi mechanism, but rather obtained through
  the accretion of cold blobs of gas surrounding the BCG, \ie\ via
  cold-mode accretion.
\item We argue that the presence of a quasar, interaction of the X-ray
  halo and jets, ostensible \irs\ gas-poorness, nuclear emission line
  outflow, high effective Eddington quasar luminosity, and
  misalignment of the large-scale radio jets and beamed radiation from
  the nucleus, suggest that \irs\ is cycling between a
  kinetic-dominated mode of feedback to a radiation-dominated
  mode. Among several possible explanations for the nature of \irs, we
  speculate that the feedback mode may be evolving because of a rapid,
  supercritical accretion event (of unknown origin) which drove the
  nuclear accretion rate above a critical rate which controls the
  competition between disk winds and jets.
\end{itemize}

%%%%%%%%%%%%%%%%%%%%%%%%%%
\section*{Acknowledgments}
%%%%%%%%%%%%%%%%%%%%%%%%%%

Support for this work was provided by the National Aeronautics and
Space Administration through Chandra Award Number GO9-0143X issued by
the Chandra X-ray Observatory Center, which is operated by the
Smithsonian Astrophysical Observatory for and on behalf of the
National Aeronautics Space Administration under contract
NAS8-03060. MD and GMV acknowledge support through NASA LTSA grant
NASA NNG-05GD82G. BRM thanks the Natural Sciences and Engineering
Research Council of Canada for support, and KWC acknowledges financial
support from L'Agence Nationale de la Recherche (ANR) through grant
ANR-09-JCJC-0001-01. KWC thanks Niayesh Afshordi, Alastair Edge, and
Kazushi Iwasawa for helpful discussions, and Guillaume Belanger and
Roland Walter for advice regarding \integral\ data analysis.

%%%%%%%%%%%%%%%%
% Bibliography %
%%%%%%%%%%%%%%%%

\bibliographystyle{mn2e}
\bibliography{cavagnolo}

%%%%%%%%%%%%%%%%%%%%%%%
% Figures  and Tables %
%%%%%%%%%%%%%%%%%%%%%%%

\clearpage
\onecolumn
\begin{table*}
  \caption{\sc Summary of X-ray Substructure Significance Tests.\label{tab:sig}}
  \begin{tabular}{lccc}
    \hline
    \hline
    Structure & Method 1 & Method 2\\
    (1) & (2) & (3)\\
    \hline
    NEx       & $1.54 \pm 0.09$ & $1.57 \pm 0.12$\\
    EEx       & $1.39 \pm 0.05$ & $1.56 \pm 0.08$\\
    WEx       & $1.47 \pm 0.11$ & $1.51 \pm 0.06$\\
    NW Cavity & $0.78 \pm 0.08$ & $0.84 \pm 0.04$\\
    SE Cavity & $0.71 \pm 0.10$ & $0.76 \pm 0.06$\\
    \hline
  \end{tabular}
  \begin{quote}
    Details of test methods are given in Section
    \ref{sec:sub}. Col. (1) Structure identification; Col. (2) ratio
    of structure surface brightness relative to surface brightness of
    model B at same radius; Col. (3) ratio of structure surface
    brightness to ICM surface brightness in an annulus centered on the
    cluster and having the same radial width as the structure.
  \end{quote}
\end{table*}

\begin{table*}
  \caption{\sc Summary of Cavity Properties.\label{tab:cylcavities}}
  \begin{tabular}{lcccccc}
    \hline
    \hline
    Cavity & $r$ & $l$ & \tsonic & $pV$ & \ecav & \pcav\\
    -- & kpc & kpc & $10^6$ yr & $10^{58}$ ergs & $10^{59}$ ergs & $10^{44}$ ergs s$^{-1}$\\
    (1) & (2) & (3) & (4) & (5) & (6) & (7)\\
    \hline
    NW & 6.40 & 58.3 & ${50.5 \pm 7.6}$ & ${5.78 \pm 1.07}$ & ${2.31 \pm 0.43}$ & ${1.45 \pm 0.35}$\\
    SE & 6.81 & 64.0 & ${55.4 \pm 8.4}$ & ${6.99 \pm 1.29}$ & ${2.80 \pm 0.52}$ & ${1.60 \pm 0.38}$\\
    \hline
  \end{tabular}
  \begin{quote}
    Col. (1) Cavity location; Col. (2) Radius of excavated cylinder;
    Col. (3) Length of excavated cylinder; Col. (4) Sound speed age;
    Col. (5) $pV$ work; Col. (6) Cavity energy; Col. (7) Cavity power.
  \end{quote}
\end{table*}

\begin{table*}
  \begin{center}
    \caption{\sc Summary of X-ray Excesses Spectral Fits.\label{tab:excess}}
    \begin{tabular}{lccccccc}
      \hline
      \hline
      Region & \tx & $\eta$ & $E_{\mathrm{G}}$ & $\sigma_{\mathrm{G}}$ & $\eta_{\mathrm{G}}$ & Cash & DOF\\
      - & keV & $10^{-5}$ cm$^{-5}$ & keV & keV & $10^{-6}~\pcmsq~\ps$ & - & -\\
      (1) & (2) & (3) & (4) & (5) & (6) & (7) & (8)\\
      \hline
      Eastern excess     & 3.03$^{+1.19}_{-0.74}$ & $5.80^{+1.07}_{-0.97}$ & -                  & -                    & -                & 524 & 430\\
      Eastern excess     & 3.68$^{+3.34}_{-1.58}$ & $2.73^{+0.98}_{-0.94}$ & [0.89, 1.42, 4.23] & [0.04, 0.16, 3.6E-4] & [1.2, 2.0, 0.16] & 384 & 430\\
      Eastern excess bgd & 3.92$^{+0.35}_{-0.31}$ & $39.9^{+0.18}_{-0.17}$ & -                  & -                    & -                & 471 & 430\\
      Lower-NW excess & 2.55$^{+2.61}_{-0.98}$ & $0.66^{+0.11}_{-0.07}$ & -                  & -                    & -                & 387 & 430\\
      \hline
    \end{tabular}
    \begin{quote}
      Metal abundance was fixed at $0.5 ~\Zsol$ for all fits.
      Col. (1) Extraction region; Col. (2) Thermal gas temperature;
      Col. (3) Model normalization; Col. (4) Gaussian central
      energies; Col. (5) Gaussian dispersions; Col. (6) Gaussian
      normalizations; Col. (7) Modified Cash statistic; Col. (8)
      Degrees of freedom.
    \end{quote}
  \end{center}
\end{table*}

\begin{table*}
  \begin{center}
    \caption{\sc Summary of Nuclear Source Spectral Fits.\label{tab:nucspec}}
    \begin{tabular}{lccc}
      \hline
      \hline
      Component & Parameter & SP09 & SP99\\
      (1) & (2) & (3) & (4)\\
      \hline
      \pexrav\  & $\Gamma$              & $1.71^{+0.23}_{-0.65}$                & fixed to SP09\\
      -         & $\eta_{\mathrm{P}}$   & $8.07^{+0.64}_{-0.62}\times10^{-4}$   & $8.46^{+2.08}_{-2.12} \times 10^{-4}$\\
      \gauss\ 1 & $E_{\mathrm{G}}$      & $0.73^{+0.05}_{-0.24}$                & $0.61^{+0.10}_{-0.05}$\\
      -         & $\sigma_{\mathrm{G}}$ & $85^{+197}_{-53}$                     & $97^{+150}_{-97}$\\
      -         & $\eta_{\mathrm{G}}$   & $8.14^{+3.74}_{-5.82} \times 10^{-6}$ & $1.65^{+1.52}_{-1.00} \times 10^{-5}$\\
      \gauss\ 2 & $E_{\mathrm{G}}$      & $1.16^{+0.19}_{-0.33}$                & $0.90^{+0.17}_{-0.90}$\\
      -         & $\sigma_{\mathrm{G}}$ & $383^{+610}_{-166}$                   & $506^{+314}_{-262}$\\
      -         & $\eta_{\mathrm{G}}$   & $1.03^{+3.22}_{-0.48} \times 10^{-5}$ & $1.48^{+2.68}_{-1.16} \times 10^{-5}$\\
      \gauss\ 3 & $E_{\mathrm{G}}$      & $4.45^{+0.04}_{-0.04}$                & $4.46^{+0.04}_{-0.07}$\\
      -         & $\sigma_{\mathrm{G}}$ & $45^{+60}_{-45}$                      & $31^{+94}_{-31}$\\
      -         & $\eta_{\mathrm{G}}$   & $2.67^{+0.91}_{-0.86} \times 10^{-6}$ & $6.45^{+4.17}_{-3.69} \times 10^{-6}$\\
      -         & EW$^{\mathrm{corr}}_{\mathrm{K}\alpha}$ & $531^{+211}_{-218}$ & $1210^{+720}_{-710}$\\
      Statistic & \chisq                & 79.0                                  & 7.9\\
      -         & DOF                   & 74                                    & 15\\
      \hline
    \end{tabular}
    \begin{quote}
      \feka\ equivalent widths have been corrected for redshift. Units for
      parameters: $\Gamma$ is dimensionless, $\eta_{\mathrm{P}}$ is in ph
      keV$^{-1}$ cm$^{-2}$ s$^{-1}$, $E_{\mathrm{G}}$ are in keV,
      $\sigma_{\mathrm{G}}$ are in eV, $\eta_{\mathrm{G}}$ are in ph
      cm$^{-2}$ s$^{-1}$, EW$_{\mathrm{corr}}$ are in eV. Col. (1)
      \xspec\ model name; Col. (2) Model parameters; Col. (3) Values for
      2009 \cxo\ spectrum; Col. (4) Values for 1999 \cxo\ spectrum.
    \end{quote}
  \end{center}
\end{table*}

\clearpage
\begin{figure}[htp]
  \begin{center}
    \begin{minipage}[htp]{0.9\linewidth}
      \includegraphics*[width=\textwidth, trim=15mm 10mm 10mm 10mm, clip]{beta.eps}
      \caption{Surface brightness profiles for clusters requiring a
        $\beta$-model fit for deprojection (discussed in
        \S\ref{sec:beta}). The best-fit $\beta$-model for each cluster
        is overplotted as a dashed line. The discrepancy between the
        data and best-fit model for some clusters results from the
        presence of a compact X-ray source at the center of the
        cluster. These cases are discussed in Appendix
        \ref{app:beta}.}
      \label{fig:betamods}
    \end{minipage}
  \end{center}
\end{figure}
\clearpage
\begin{figure}[htp]
  \begin{center}
    \begin{minipage}[htp]{0.9\linewidth}
      \includegraphics*[width=\textwidth, trim=5mm 0mm 5mm 5mm, clip]{itplflat_rat.eps}
      \caption{Ratio of best-fit \kna\ for the two treatments of
        central temperature interpolation (see \S\ref{sec:temppr}):
        (1) temperature is free to decline across the central density
        bins ($\Delta T_{center} \ne 0$), and (2) the temperature
        across the central density bins is isothermal ($\Delta
        T_{center} = 0$). Filled black squares are clusters for which
        the \kna\ ratio is inconsistent with unity.}
      \label{fig:kcomp}
    \end{minipage}
  \end{center}
\end{figure}
\clearpage
\begin{figure}[htp]
  \begin{center}
    \begin{minipage}[htp]{0.9\linewidth}
      \includegraphics*[width=\textwidth, trim=5mm 0mm 5mm 5mm, clip]{k0res.eps}
      \caption{Best-fit \kna\ vs. redshift. Some clusters have
        \kna\ error bars smaller than the point. The clusters with
        upper-limits ({\it{black points with downward arrows}}) are:
        A2151, AS0405, MS 0116.3-0115, and RX J1347.5-1145. The
        numerically labeled clusters are: (1) M87, (2) Centaurus
        Cluster, (3) RBS 533, (4) HCG 42, (5) HCG 62, (6) SS2B153, (7)
        A1991, (8) MACS0744.8+3927, and (9) CL J1226.9+3332. For
        CLJ1226, \cite{2007ApJ...659.1125M} found best-fit $\kna = 132
        \pm 24 \ent$ which is not significantly different from our
        value of $\kna = 166 \pm 45 \ent$. The lack of $\kna < 10
        \ent$ clusters at $z > 0.1$ is most likely the result of
        insufficient angular resolution (see \S\ref{sec:angres}).}
      \label{fig:k0res}
    \end{minipage}
  \end{center}
\end{figure}
\clearpage
\begin{center}
  \begin{figure}[htp]
    \begin{minipage}[htp]{0.5\linewidth}
      \includegraphics*[width=\textwidth, trim=28mm 7mm 30mm 17mm, clip]{curvk0.eps}
    \end{minipage}
    \begin{minipage}[htp]{0.5\linewidth}
      \includegraphics*[width=\textwidth, trim=28mm 7mm 30mm 17mm, clip]{nbins_k0.eps}
    \end{minipage}
    \begin{minipage}[htp]{0.5\linewidth}
      \includegraphics*[width=\textwidth, trim=28mm 7mm 30mm 17mm, clip]{texpk0.eps}
    \end{minipage}
    \begin{minipage}[htp]{0.5\linewidth}
      \includegraphics*[width=\textwidth, trim=28mm 7mm 30mm 17mm, clip]{ntxbins_k0.eps}
    \end{minipage}
    \caption{Plots of possible systematics versus best-fit \kna.
      {\it{Top left:}} Best-fit \kna\ plotted versus average curvature
      of the corresponding entropy profile (see eq. \ref{eqn:avgcurv})
      There is no trend between these two quantities suggesting that
      \kna\ is not heavily influenced by the total shape of the
      entropy profile. {\it{Top right:}} Best-fit \kna\ plotted versus
      number of bins in the entropy profile which were used during
      fitting. Again, no trend is found. {\it{Bottom left:}} Best-fit
      \kna\ plotted versus the total used exposure time for each
      cluster. No trend is found. {\it{Bottom right:}} Best-fit
      \kna\ plotted versus the number of bins in the temperature
      profile for each cluster. As expected, fewer $\Tx(r)$ does not
      correlate with \kna.}
    \label{fig:sys}
  \end{figure}
\end{center}
\clearpage
\begin{center}
  \begin{figure}[htp]
    \begin{minipage}[htp]{0.5\linewidth}
      \includegraphics*[width=\textwidth, trim=28mm 7mm 30mm 17mm, clip]{splots_allt.eps}
    \end{minipage}
    \begin{minipage}[htp]{0.5\linewidth}
      \includegraphics*[width=\textwidth, trim=28mm 7mm 30mm 17mm, clip]{splots_tle4.eps}
    \end{minipage}
    \begin{minipage}[htp]{0.5\linewidth}
      \includegraphics*[width=\textwidth, trim=28mm 7mm 30mm 17mm, clip]{splots_gt4tle8.eps}
    \end{minipage}
    \begin{minipage}[htp]{0.5\linewidth}
      \includegraphics*[width=\textwidth, trim=28mm 7mm 30mm 17mm, clip]{splots_tgt8.eps}
    \end{minipage}
    \caption{Composite plots of entropy profiles for varying cluster
      temperature ranges. Profiles are color-coded based on average
      cluster temperature. Units of the color bars are keV. The solid
      line is the pure-cooling model of \cite{voitbryan}, the dashed
      line is the mean profile for clusters with $\kna \le 50 \ent$,
      and the dashed-dotted line is the mean profile for clusters with
      $\kna > 50 \ent$. {\it{Top left:}} This panel contains all the
      entropy profiles in our study. {\it{Top right:}} Clusters with
      $kT_X < 4$ keV. {\it{Bottom left:}} Clusters with $4\keV < kT_X
      < 8\keV$. {\it{Bottom right:}} Clusters with $kT_X > 8$
      keV. Note that while the dispersion of core entropy for each
      temperature range is large, as the $kT_X$ range increases so to
      does the mean core entropy.}
    \label{fig:splots}
  \end{figure}
\end{center}
\clearpage
\begin{figure}[htp]
  \begin{center}
    \begin{minipage}[htp]{0.9\linewidth}
      \includegraphics*[width=\textwidth, trim=20mm 10mm 10mm 10mm, clip]{k0hist.eps}
      \caption{{\it{Top panel:}} Histogram of best-fit \kna\ for all
        the clusters in \accept. Bin widths are 0.15 in log space.
        {\it{Bottom panel:}} Cumulative distribution of \kna\ values
        for the full sample. The distinct bimodality in \kna\ is
        present in both distributions, which would not be seen if it
        were an artifact of the histogram binning. A KMM test finds
        the \kna\ distribution cannot arise from a simple unimodal
        Gaussian.}
      \label{fig:k0hist}
    \end{minipage}
  \end{center}
\end{figure}
\clearpage
\begin{figure}[htp]
  \begin{center}
    \begin{minipage}[htp]{0.9\linewidth}
      \includegraphics*[width=\textwidth, trim=20mm 10mm 10mm 10mm, clip]{hifl_k0hist.eps}
      \caption{{\it{Top panel:}} Histogram of best-fit \kna\ values
        for the primary \hifl\ sample. Bin widths are 0.15 in log
        space.  {\it{Bottom panel:}} Cumulative distribution of
        best-fit \kna\ values. The distinct bimodality seen in the
        full \accept\ sample (Fig. \ref{fig:k0hist}) is also present
        in the \hifl\ subsample and shares the same gap between the
        low-entropy peak at 10-20 \ent\ and the high-entropy peak at
        100-200 \ent. That bimodality is present in both samples is
        strong evidence it is not a result of an unknown archival
        bias.}
      \label{fig:hiflk0}
    \end{minipage}
  \end{center}
\end{figure}
\clearpage
\begin{figure}[htp]
  \begin{center}
    \begin{minipage}[htp]{0.8\linewidth}
      \includegraphics*[width=\textwidth, trim=20mm 10mm 10mm 10mm, clip]{t0.eps}
    \end{minipage}
    \begin{minipage}[htp]{0.8\linewidth}
      \includegraphics*[width=\textwidth, trim=20mm 10mm 10mm 10mm, clip]{k0cool.eps}
    \end{minipage}
    \caption{{\it{Top panel:}} Log-binned histogram and cumulative
      distribution of best-fit core cooling times, $t_{c0}$
      (eqn. \ref{eqn:tc0}), for all the clusters in \accept. Histogram
      bin widths are 0.2 in log space. {\it{Bottom panel:}} Log-binned
      histogram and cumulative distribution of core cooling times
      calculated from best-fit \kna\ values, $t_{c0}(\kna)$
      (eqn. \ref{eqn:tck0}), for all the clusters in
      \accept. Histogram bin widths are 0.2 in log space. The
      bimodality we observe in the \kna\ distribution is also present
      in best-fit $t_{c0}$. However, the gaps between the two
      populations of $t_{c0}$ and $t_{c0}(\kna)$ differ by $\sim 0.3$
      Gyrs which may be an artifact of the binning.}
    \label{fig:t0}
  \end{center}
\end{figure}



%%%%%%%%%%%%%%%%%%%%
% End the document %
%%%%%%%%%%%%%%%%%%%%

\label{lastpage}
\end{document}
