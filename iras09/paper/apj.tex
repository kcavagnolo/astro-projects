%%%%%%%%%%%%%%%%%%%
% Custom commands %
%%%%%%%%%%%%%%%%%%%

\newcommand{\mytitle}{AGN Heating and Gas Uplift via the Obscured Radio-Quiet
Quasar in \iras}
\newcommand{\mystitle}{AGN Feedback in \iras}
\newcommand{\iras}{IRAS 09104+4109}
\newcommand{\rxj}{RX J0913.7+4056}
\newcommand{\radec}{R.A.(J2000) $=09^h 13^m 45^s.5$, Dec.(J2000) $=+40^{\mydeg} 56^{\arcm} 28^{\arcs}$}
\newcommand{\ellradec}{09^h 13^m 45^s.5, +40^{\mydeg} 56^{\arcm} 28^{\arcs}}

%%%%%%%%%%
% Header %
%%%%%%%%%%

\documentclass{emulateapj}
\usepackage{apjfonts,graphicx,here,common,longtable,ifthen,amsmath,amssymb,natbib,lscape,subfigure}
%% \usepackage[pagebackref,
%%   pdftitle={TBD},
%%   pdfauthor={Dr. Kenneth W. Cavagnolo},
%%   pdfsubject={ApJ},
%%   pdfkeywords={},
%%   pdfproducer={LaTeX with hyperref},
%%   pdfcreator={LaTeX}
%%   pdfdisplaydoctitle=true,
%%   colorlinks=true,
%%   citecolor=blue,
%%   linkcolor=blue,
%%   urlcolor=blue]{hyperref}
\bibliographystyle{apj}
\begin{document}
\title{\mytitle}
\shorttitle{\mystitle}
\author{
  K. W. Cavagnolo\altaffilmark{1,6},
  M. Donahue\altaffilmark{2},
  B. R. McNamara\altaffilmark{1,3,4},\\
  G. M. Voit\altaffilmark{2},
  and M. Sun\altaffilmark{5},
}
\altaffiltext{1}{Department of Physics and Astronomy, University of
  Waterloo, Waterloo, ON N2L 3G1, Canada.}
\altaffiltext{2}{Michigan State University, Department of Physics and
  Astronomy, East Lansing, MI, 48824-2320}
\altaffiltext{3}{Perimeter Institute for Theoretical Physics, 31
  Caroline Street N, Waterloo, ON N2L 2Y5, Canada.}
\altaffiltext{4}{Harvard-Smithsonian Center for Astrophysics, 60
  Garden Street, Cambridge, MA 01238, USA.}
\altaffiltext{5}{University of Virginia, Department of Astronomy,
  Charlottesville, VA, 22904}
\altaffiltext{6}{kcavagno@uwaterloo.ca}
\shortauthors{K. W. Cavagnolo et al.}
\journalinfo{}
\slugcomment{For submission to MNRAS}

%%%%%%%%%%%%
% Abstract %
%%%%%%%%%%%%

\begin{abstract}
We report on the new \chandra\ X-ray Observations 
\end{abstract}

%%%%%%%%%%%%
% Keywords %
%%%%%%%%%%%%

\keywords{cooling flows -- galaxies: clusters: general -- galaxies:
  clusters: individual (\iras)}

%%%%%%%%%%%%%%%%%%%%%%
\section{Introduction}
\label{sec:intro}
%%%%%%%%%%%%%%%%%%%%%%

\citet{1988ApJ...328..161K}: Kleinmann, id'd as sey2 cD most ir lum
and coind w/ rad, 99\% longwa of 1micron\\

\citet{1993ApJ...415...82H}: Hines, VLA and ground-base polarimetry
show steep radio spec in jet and lobes (old source), but young new
beam\\

\citet{1994ApJ...436L..51F}: Fabian, ASCA FeKalpha detection, thermal
ICM\\

\citet{1995MNRAS.274L..63F}: Fabian, ROSAT, suggested hole at center
of cluster associated with massive cooling flow, hole is nothing
(follow-up obs, see notes)\\

\citet{1996MNRAS.283.1003C}: Crawfrod, Optical integral field
spectroscopy, nebulae is static\\

\citet{1996AJ....111..649S}: Soifer, near-IR, six stripped bulges in
cD envelope at MK~ -20->-24\\

\citet{1997A&A...318L...1T}: Taniguchi, mid-ir spec and torus
properties, awesome graphic\\

\citet{1998ApJ...506..205E}: Evans, radio, dust covering factor of
>90\%, $<10^{10} h^{-2}$ \Msol H2 from CO ul\\

\citet{1999Ap&SS.266..113A}: Armus, hst, image oiii filament sitting
in uv cone, whiskers on cD in red-light\\

\citet{1999ApJ...512..145H}: Hines, HST, polarimetry proves new
beaming direction of AGN\\

\citet{2000A&A...353..910F}: Franceschini, Beppo-Sax, buried type-2
QSO, 1000 msol cf\\

\citet{2000MNRAS.315..269A}: Allen, ASCA, global properties of \iras\\

\citet{2000AJ....120..562T}: Tran, inclination angle of disk, extended
emission from shredded members, not cf or expelled by qso, no shocks
in nucleus\\

\citet{2001MNRAS.321L..15I}: Iwasawa, Chandra, cold matter model
pexrav\\

\citet{2001MNRAS.326.1467D}: Deane, cold dust contrib <3\% of bolo
lum\\

``The absence of cold dust in a powerful AGN like this is certainly
unusual. High-redshift optical quasars often have cold dust (McMahon
et al. 1999), so one might expect an infrared quasar like IRAS
P0910414109 to have some too, particularly since the X-ray
measurements (Iwasawa et al. 2001) tell us that plenty of cold gas is
present. The solution to this apparent paradox is presumably tied up
in the way that gas condenses out of the hot X-ray phase in a cooling
flow. If gas condenses in such a way that there is no new star
formation, dust generation will be inefficient -- carbon atoms
associated with this cooling gas are more likely to interact with four
hydrogen atoms and become methane rather than with numerous other
carbon atoms and become dust grains. This could explain the presence
of large amounts of cold X-ray absorbing gas (Iwasawa et al. 2001) and
no cold dust. If the cooling flow is also responsible for fuelling the
AGN, any star formation that happens along the way must be confined to
regions very close to the AGN. Red giants and supernovae associated
with this star formation could generate the hot dust seen by IRAS.''

\citet{2004ApJ...613..986P}: Peeters, no PAHs detection\\

\citet{2007A&A...473...85P}: Piconcelli, XMM, changing-look, confirm
reflectio domination of iwasawa\\

\citet{2008ApJ...683..114S}: Sargsyan, spitzer, no PAHs\\

The paper is structured as follows: \WMAP\ For the cluster redshift of
$z=0.4418$, the age of the Universe was $\approx 9.1$ Gyr and the
corresponding angular diameter distance is $\approx 5.72$ kpc
arcsec$^{-1}$. All errors are quoted at the 68\% level unless stated
otherwise.

%%%%%%%%%%%%%%%%%%%%%%%%%%%%%%%%%%%%%%%%
\section{Observations and Data Analysis}
\label{sec:obs}
%%%%%%%%%%%%%%%%%%%%%%%%%%%%%%%%%%%%%%%%

%%%%%%%%%%%%%%%%%%
\subsection{X-ray}
\label{sec:xray}
%%%%%%%%%%%%%%%%%%

A 77.2 ksec observation of \iras\ was taken on January 9\th, 2009 with
the Advanced CCD Imaging Spectrometer I-array (ACIS-I) ($\approx
0.492\arcs$ pix$^{-1}$) on-board the \cxo\ (ObsID 10445; PI
Cavagnolo). The archival \chandra\ observation of \iras\ from November
3\rd, 1999 taken with the ACIS-S array was included in our X-ray
analysis (ObsID 509; PI Fabian). Both datasets were reprocessed and
reduced using the newest \chandra\ Interactive Analysis of
Observations software (\ciao) version 4.1 and Calibration Database
(\caldb) version 4.1.1. Events were selected based on \asca\ grades
and corrections for the ACIS gain change, charge transfer
inefficiency, and degraded quantum efficicency were applied. The new
dataset did not require removal of the ACIS cosmic-ray induced
afterglow nor creation of a new bad pixel file. However, these two
additional reduction steps were applied to the 1999
observation. Lightcurves from a source free region of each observation
were created for a front-illuminated (FI) and back-illuminated (BI)
CCD. The lightcurves were used to determine if any time intervals
during the observations were contaminated by background flares. Six
time intervals in the new observation fell outside $20\%$ of the mean
background count rate and were excluded, yielding a final exposure
time of 76.2 ksec. Comparison of the FI and BI lightcurves for the
archival observation revealed significant contamination from a
hard-flare, and a soft-flare, which reduced the usable exposure time
from $\approx 9$ ksec to $\approx 6$ ksec.

To take advantage of both observations for imaging analysis, the
level-2, flare-clean events files were reprojected to a common tangent
point and summed. A similar step was taken for the individual exposure
maps except that the exposure maps were averaged. The combined events
file was then normalized by the average exposure map. After merging
the datasets, it was apparent that the known aspect offset found in
some data from earlier than 2004 was present in ObsID 509. To improve
the astrometry of the ObsID 509 dataset, a new aspect solution was
created using the \ciao\ tool {\textsc{reproject\_aspect}} and the
positions of several field sources. To minimize the positional
uncertainty introduced by the strong angle dependance of the off-axis
\chandra\ PSF, non-target X-ray sources closest to the aimpoint were
selected for creation of the new aspect solution. After correction of
the astrometry, the positional accuracy between both observations was
$\approx 0.7\arcs$, \ie\ uncertainties comparable to the resolution
limit of the ACIS detectors.

%%%%%%%%%%%%%%%%%%
\subsection{Radio}
\label{sec:radio}
%%%%%%%%%%%%%%%%%%

From 1986 to 2000, \iras\ has been observed at multiple frequencies at
varying resolutions using the Very Large Array (VLA) radio
observatory. Using archival VLA data, resolved radio emission is
detected at 1.4 GHz, 5 GHz, and 8.4 GHz, while a
$3-\sigma_{mathrm{RMS}}$ upper limit of $0.84$ mJy is established at
14.9 GHz. Fluxes for unresolved emission at 151 and 325 MHz were
retrieved from \citet{1999MNRAS.306...31R} and
\citet{1997A&AS..124..259R}, respectively. The combined 1.4 GHz data
reveals the most extended structure, and thus we focus discussion
regarding radio morphology using this frequency only. An analysis of
1.4 and 5 GHz VLA data is also presented in
\citet{1993ApJ...415...82H}.

Continuum mode observations were taken from the VLA archive and
reduced using the new Common Astronomy Software Applications package
(\casa\footnote{\url{http://casa.nrao.edu/}}) developed and supported
by the National Radio Astronomy Observatory. The data reduction steps
in \casa\ are similar to those using \aips: the data was Fourier
transformed, cleaned, self-calibrated, and restored. Flux density
calibration was performed using the radio source 3C 286, while phase
calibration was performed with 3C 147. The additional step of phase
and amplitude self-calibration increases the dynamic range and
sensitivity of the radio maps, yielding greater detail of low
significance emission. All significant sources ($\ga 3\sigma_{RMS}$)
within the primary beam and first sidelobe were imaged to further
maximize the sensitivity of the radio map.

The angular resolution of the combined 1.4 GHz dataset is $0.37\arcs$
pixel$^{-1}$ with an average beam shape of $1.26\arcs \times
1.15\arcs$ with $\theta = 37.16\mydeg$. Using the \casa\ tool
{\textsc{viewer}} we measure an off-source RMS noise of $\approx
27.3~\mu$Jy beam$^{-1}$. The deconvolved, integrated 1.4 GHz flux of
the continuous extended structure coincident with \iras, and having
$S_{\nu} \ga 3\sigma_{RMS}$, is $14.0 \pm 0.51$ mJy. These values are
consistent with the results presented in
\citet{1993ApJ...415...82H}. Contours of the radio emission
surrounding \iras\ were generated beginning at 3 times the RMS noise
and moving up in 6 log-space steps to the peak intensity of 4.7 mJy
beam$^{-1}$. These are the contours which will be referenced in all
following discussion of the radio source morphology and its
interaction with the X-ray gas. The radio contours are shown in Figure
\ref{fig:resid}.

We fit the radio spectrum between 178 MHz and 8.4 GHz for the full
radio source (jets+core) with the well-known synchrotron models of
Kardashev \& Pacholczyk \citep[KP][]{1962SvA.....6..317K, pach},
\citet[JP][]{1973A&A....26..423J}, and
\citet[CI][]{1987MNRAS.225..335H}. The models vary by assumption of
pitch-angle distribution and number of electron injections. These
three synchrotron models were fitted to the radio spectrum using the
code of \citet{2005ApJ...624..656W}, which is based on the method and
code of \citet{1991ApJ...383..554C}. The JP model (single electron
injection, randomized but isotropic pitch-angle distribution) yields
the best fit with $\chisq = 0.491$ for 2 degrees of freedom, a break
frequency of $\nu_B = 12.9 \pm 1.2$ GHz, and low-frequency spectral
index of $\alpha_{\mathrm{JP}} = -1.00 \pm 0.49$. The radio spectrum
and best-fit models are shown in Figure \ref{fig:radiospec}. The
bolometric radio luminosity was calculated by integrating under the JP
curve between 10 MHz and $10,000$ MHz, giving $\lrad =
1.50\times10^{42}~\lum$.

%%%%%%%%%%%%%%%%%%%%%%%%%%%%%%%%
\section{Results and Discussion}
\label{sec:rd}
%%%%%%%%%%%%%%%%%%%%%%%%%%%%%%%%

%%%%%%%%%%%%%%%%%%%%%%%%%%%%%%%%%%
\subsection{Global ICM Properties}
\label{sec:global}
%%%%%%%%%%%%%%%%%%%%%%%%%%%%%%%%%%

We define the mean cluster temperature, $T_{\mathrm{cluster}}$, to be
the ICM temperature within the core-excised aperture of $R_{500}$
(defined below). $T_{\mathrm{cluster}}$ and $R_{500}$ are linked in
the adopted calculation, and so they were determined recursively. We
used the convention of \citet{2007ApJ...668..772M} and set the core
radius to $0.15 R_{500}$. $R_{\Delta_c}$ is defined as the radius at
which the average cluster density is $\Delta_c$ times the critical
density for a spatially flat Universe, $\rho_c=3H(z)^2/8\pi G$. To
calculate $R_{\Delta_c}$, we used the relation from
\cite{2002A&A...389....1A}:
\begin{eqnarray}
  R_{\Delta_c} &=& 2.71~\beta_T^{1/2} \Delta_z^{-1/2} (1+z)^{-3/2} \left(\frac{kT_{\mathrm{cluster}}}{10 \keV}\right)^{1/2} \Mpc~h_{70}^{-1} \nonumber \\
  \Delta_z &=& \frac{\Delta_c \Omega_M}{18\pi^2\Omega_z} \nonumber \\
  \Omega_z &=& \frac{\Omega_M (1+z)^3}{[\Omega_M (1+z)^3]+[(1-\Omega_M-\Omega_{\Lambda})(1+z)^2]+\Omega_{\Lambda}} \nonumber
\end{eqnarray}
where $kT_{\mathrm{cluster}}$ is the mean cluster temperature in keV,
$\Delta_c$ is the assumed density contrast of the cluster at
$R_{\Delta_c}$, and $\beta_T$ is a numerically determined,
cosmology-independent ($\lesssim \pm 20\%$) normalization for the
virial relation $GM/2R = \beta_TkT_{\mathrm{cluster}}$. We use
$\beta_T = 1.05$ taken from \cite{1996ApJ...469..494E}.

A spectrum was extracted within the region $0.15~R_{500} < R <
R_{500}$ using a first guess temperature of 7 keV. The spectrum was
fitted with a single-component thermal plasma model (\mekal) keeping
the Galactic column density in the direction of \iras\ fixed at the
weighted-mean value from the LAB survey \citep[$\nhi =
  1.58\times10^{20}~\pcmsq$;][]{lab}. Gas metal abundance was left as
a free parameter using the solar standard abundance distribution of
\citet{grsa}. The new best-fit temperature was used to recalculate
$R_{500}$ and the process was repeated until three consecutive
iterations produced \tx\ values which differed by $\leq 1\sigma$.

With this method, we measure a cluster temperature of
$9.98^{+3.13}_{-2.02}$ keV, which corresponds to $R_{500} =
1.34^{+0.47}_{-0.19}~\Mpc~h_{70}^{-1}$, and a cluster metallicity of
$0.38^{+0.31}_{-0.24}~\Zsol$. The bolometric luminosity in this
aperture was determined using a diagonalized response function over
the energy range 0.01-100.0 keV with 3000 linearly spaced energy
channels, yielding $L_{\mathrm{bol}} =
6.25^{+1.08}_{-0.94}\times10^{44}~\lum h_{70}^{-1}$. Additional
spectral fits for a variety of apertures are summaraized in Table
\ref{tab:specfits}.

Integrating the gas mass profile presented in the next section, we
measure $M_{\mathrm{gas}} =
X.XX^{+X.XX}_{-X.XX}\times10^{XX}~\Msol$. To derive a total
gravitating mass for \rxj, hydrostatic equilibrium was assumed:
\begin{equation}
  M_{\mathrm{grav}}(< r) = -\frac{k T(r) r}{G \mu m_p}\left[\frac{d\ln
      T(r)}{d\ln r} - \frac{d\ln \nelec(r)}{d\ln r}\right]
\end{equation}
where $k$ is the Boltzmann constant, $G$ is the gravitational
constant, $m_{\mathrm{p}}$ is proton mass, ICM mean molecular weight
$\mu = 0.597$, $T(r)$ and $\nelec(r)$ are radial temperature and
electron gas density profiles presented in the next section. We derive
$M_{\mathrm{grav}} = X.XX^{+X.XX}_{-X.XX}\times10^{XX}~\Msol$

%%%%%%%%%%%%%%%%%%%%%%%%%%%%%%%%
\subsection{Radial ICM Profiles}
\label{sec:rad}
%%%%%%%%%%%%%%%%%%%%%%%%%%%%%%%%

An azimuthally averaged ICM temperature profile was created using
circular annuli centered on the cluster X-ray peak and containing a
minimum of $5,000$ source counts per annulus. Spectra were binned to
contain a minimum of 25 counts per energy channel.  Blank-sky
backgrounds within the \caldb\ were reprocessed and reprojected to
match each observation, and then normalized for variations of the
hard-particle background using the ratio of blank-sky and observation
9.5-12 keV count rates. Using the method outlined in
\citet{2005ApJ...628..655V}, we included a fixed background component
in our spectral analysis to account for the spatially-varying soft
Galactic background \citep[see also][for more
  detail]{xrayband}. Spectra were fitted with \xspec\ 12.4.0
\citep{xspec} using an absorbed, single-temperature \mekal\ model
\citep{mekal1, mekal2} over the energy range 0.7-7.0 \keV. The
Galactic \nhi\ from the LAB survey was used and kept fixed (see
Section \ref{sec:global}). \citet{ag89} abundance normalizations were
also used. The fit statistic was $\chi^2$. The temperature profile and
associated abundance profile are shown shown in the top row of Figure
\ref{fig:gallery}.

A radial surface brightness profile was extracted from the point
source clean, flare clean, exposure corrected events files with the
90\% enclosed energy fraction (EEF) region of the nuclear point source
excluded ($r_{exclude} = 1.16\arcs$). Concentric circular annuli with
a width of 2 ACIS pixels ($\approx 1\arcs$) and centered on the cD
X-ray point source were used. Using the surface brightness and
temperature profiles, a deprojected radial electron density was
derived using the technique outlined in \citet{kriss83}. Errors for
the density profile were estimated using $5,000$ Monte Carlo bootstrap
resamplings of the original surface brightness profile. The surface
brightness and density profiles are shown in the second row of Figure
\ref{fig:gallery}. Gas pressure ($p = \tx \nelec$), entropy ($K =
\tx\nelec^{-2/3}$), cooling time ($\tcool = 3n\tx~[2\nelec \nH
  \Lambda(T,Z)]^{-1}$), and gas mass ($M_{\mathrm{gas}} =
\int_{r=0}^{r=R_{200}} 4 \pi r^3 \nelec dr/3$) profiles were then
calculated. These profiles are shown in the bottom two rows of Figure
\ref{fig:gallery}.

There are no resolved discontinuities in the radial temperature
structure to suggest the presence of a shock or cold front. But, the
density (and hence entropy) and pressure profiles do show significant
discontinuities at the leading edge of the cavity system. This likely
results from the surface brightness enhancement caused by cool, dense
gas associated with the northern cavity (see Section
\ref{sec:uplift}). If there is an unresolved shock, the
Rankine-Hugoniot pressure and density jump conditions imply a Mach
number between 1.2-1.5.

The ICM in \rxj, and surrounding \iras, is consistent with the cool
core class of galaxy clusters.  We fit the radial entropy model $K(r)
= \kna +\khun (r/100\kpc)^{\alpha}$ to the entropy profile and found
best-fit values of $\kna = 22.5 \pm 1.7~\ent$, $\khun = 123 \pm 6
~\ent$, and $\alpha = 1.94 \pm 0.12$. These values are consistent with
the cool core population as a whole \citep{accept}, and, in
particular, a $\kna < 30 ~\ent$ places \rxj\ with the population of
``active'' clusters that typically have radio-loud AGN and star
formation in the BCG \citep{haradent, 2008ApJ...687..899R}. The $\kna
< 30 ~\ent$ regime is also interesting because this is the entropy
level at which cluster cores become conducive to the formation of
thermal instabilities \citep{conduction}, \eg\ in low-entropy gas
clouds compressed and uplifted by AGN activity.

%%%%%%%%%%%%%%%%%%%%%%%%%%%%%%%%%%%%%%%%%%%%%%%
\subsection{ICM Substructure and Cavity System}
\label{sec:cavs}
%%%%%%%%%%%%%%%%%%%%%%%%%%%%%%%%%%%%%%%%%%%%%%%

To aid investigation of substructure in the X-ray emission, a residual
image for \iras\ was created. Utilizing the flare-clean,
exposure-corrected, combined \chandra\ X-ray image, we fit the surface
brightness isophotes with ellipses. Initially, the geometric
parameters ellipticity, centroid, and position angle were varied, but
the best-fit values for each isophote converged to mean values of
$\epsilon = 0.52 \pm 0.02$, $\phi = 72.5\mydeg \pm 2.7\mydeg,$ and
$(X_0,Y_0) = \ellradec$. These values were then given to the fitting
routine as fixed parameters to eliminate isophotal twisting which
results from the statistical variation of best-fit values at each step
of the fitting. A surface brightness model normalized to match the
parent image was then constructed from the best-fit isophote
ellipses. The surface brightness model was subtracted from the parent
image resulting in a residual image. Shown in the left pane of Figure
\ref{fig:resid} is the parent X-ray image, and in the right pane is
the residual image. Overlaid on the residual image are the 1.4 GHz
contours discussed in Section \ref{sec:radio}.

In the exposure-corrected X-ray image (left panel in Figure
\ref{fig:resid}), subtle depressions in the X-ray surface brightness
are present northeast and southwest of the central point source. The
residual X-ray image (right panel in Figure \ref{fig:resid}) further
reveals the structure of these depressions, and with the 1.4 GHz radio
emission overlaid, the connection between the depressions and the AGN
outflow is obvious. Excavation of cavities via AGN outflows is an oft
observed process \citep[\eg][]{perseus1, schindler01,
  2003ApJ...596..190M, ms0735, hydraa, 2007ApJ...665.1129K,
  2009ApJ...697L..95B}, but to-date, \iras\ is the highest redshift
object where cavities have been imaged, in addition to being the only
QSO system with an unambiguous cavity detection. Though a cavity
detection has been suggested in the QSO H1821+643
\citep{2009arXiv0911.2339R}. Unlike typical cavity systems where the
X-ray gas is preferentially displaced around a radio lobe, usually
creating an approximately spherical bubble-like shape, the voids in
\iras\ are located along the base of the radio jets with no obvious
bubble-like structure around the lobes. In fact, the lobes have
uplifted cool, enriched gas from the nuclear region (see Section
\ref{sec:uplift}). Because of the void morphologies, we assume the
cavities are cylindrical rather than ellipsoidal.

%%
across the board 10\% assumed for uncertainty in geometric parameters
pV is measured by integrating over cavity surface
tcs includes 10\% sys uncert because of assuming $\Delta T_X = 0$ over the cavity length
%%

The energetics of the AGN outburst were investigated by following the
standard procedure of calculating cavity enthalpy under the assumption
that the voids and environment are in pressure equilibrium
\citep[][for a review]{rafferty06, mcnamrev}. Properties of the
individual cavities are listed in Table \ref{tab:cylcavities}. The
cavity volumes were calculated using a by-eye measurement of the
projected cross-section. The $pV$ work to create the cavities was then
calculated using a high-resolution, interpolated pressure profile and
integrating this profile over the surface of each cylinder. Cavity
enthalpy, $H_{\mathrm{cav}} = pV[\gamma/(\gamma-1)]$, was then calculated
assuming the cavity contents are a relativistic plasma with $\gamma =
4/3$. We have assumed the cavities were excavated over a timescale
determined by the sound speed of the gas,
\begin{equation}
  t_{\mathrm{sonic}} = D \left(\frac{\gamma \tx}{\mu \mH}\right)^{-1/2} \nonumber
\end{equation}
where $D$ is the distance to the leading edge of the cavity,
$\gamma=5/3$ for the ICM, $\mu=0.62$ is the mean molecular weight of
the ICM, and \mH\ is the mass of hydrogen. Under these assumptions, we
calculate a total cavity power $\pcav = 3.35(\pm0.80)\times10^{44}
~\lum$. Compared with other systems with cavities, for example
Centaurus A ($\pcav \sim 7.4\times10^{42} ~\lum$), Hydra A ($\pcav
\sim 4\times10^{44} ~\lum$), or MS 0735.6+7421 ($\pcav \sim
7\times10^{45} ~\lum$), \iras\ resides between the middle and
upper-end of the cavity power distribution.

How does the cavity power compare with the cooling luminosity of the
X-ray halo which the AGN is interacting with? Setting the cooling
radius to $r_{\mathrm{cool}} = 116$ kpc, the radius at which the
azimuthally averaged radial gas cooling time is less than the age of
the Universe at the redshift of \iras, we measure a bolometric
(0.01-100.0 keV) cooling luminosity of $L_{\mathrm{bol}}(<
r_{\mathrm{cool}}) = 1.54(\pm 0.XX)\times10^{45} ~\lum$. Correcting
for differences in cosmology, our value of $L_{\mathrm{bol}}(<
r_{\mathrm{cool}})$ is in general agreement with previous measurements
\citep{1994ApJ...436L..51F, 1995MNRAS.274L..63F,
1998MNRAS.297L..57A}. If all of the cavity power is thermalized over
$4\pi$ steradians as heat, then our numbers indicate $\approx 50\%$ of
the energy radiated away by the X-ray atmosphere is offset by the heat
returned in the ongoing AGN outburst. This highly optimistic scenario
then requires only two such AGN outbursts to shut-off cooling of the
gaseous halo surrounding
\iras.

Implications?

%%%%%%%%%%%%%%%%%%%%%%%%%
\subsection{Uplifted Gas}
\label{sec:uplift}
%%%%%%%%%%%%%%%%%%%%%%%%%

In the middle panel of Figure \ref{fig:resid} three white ellipses
highlight regions of excess X-ray emission relative to the best-fit
surface brightness model. The ellipses are for illustrative purpose
and approximate the contours of constant surface brightness used to
define the regions. For clarity, we reference each region by its
relative cardinal location to the core: western cloud (WC), northern
cloud (NC), and eastern cloud (EC).

For each region, a source spectrum and weighted responses were
extracted. Local background spectra were extracted from neighboring
regions which did not show enhanced emission in the subtracted
image. The local backgrounds were scaled to correct for differences in
sky area. For each region, the ungrouped source and background spectra
were loaded into \xspec\ and the residual source spectrum was fit with
a \mekal\ model using the modified Cash statistic
\citep{1979ApJ...228..939C}, the appropriate statistic for low count
data to avoid systematically cooler temperatures
\citep{1989ApJ...342.1207N, 2007A&A...462..429B}. The low
signal-to-noise (SN) ratio of each spectrum precluded leaving metal
abundance as a free parameter, \eg\ no L-shell or K-shell features are
resolved. Since the NC and WC regions coincide with the second annulus
of the global radial temperature profile, while the EC is coincident
with the first annulus, we fixed the metal abundance at the
weighted-mean value for these two regions: $0.61~\Zsol$. Increasing or
decreasing the fixed abundance changes the best-fit temperatures and
normalizations within the statistical uncertainties. A fixed
background thermal model representing the best-fit global spectrum
from the coincident annulus (scaled to sky area) was also included in
the fitting.

A summary of the best-fit models are presented in Table
\ref{tab:specfits}. The best-fit NC thermal temperature was $\sim 6$
keV, but it was unconstrained. We then fit the NC spectrum with an
absorbed power-law which had best-fit parameters $\Gamma = 1.74 (\pm
0.2$ and $\eta = 4.40\times10^{-6}$ photons keV$^{-1}$ cm$^{-2}$
s$^{-1}$. However, comparison of fit statistics and goodness of fits
determined from $10,000$ Monte Carlo simulations of the best-fit
spectra did not suggest if one model was preferrable to the other. The
low SN prevented determination of important spectral features which
can distinguish between a hot ($E \ga 5$ keV) thermal plasma and a
power law, \eg\ the high-energy cut-off and presence of Fe K$\alpha$
and Fe L-shell lines. Hence we could not distinguish a second
component power-law and a second thermal component.

However, the EC and WC residual spectra were consistent with a thermal
model. Comparison of the best-fit models for the EC and WC regions
with the ambient medium shows that both regions are cooler and denser
(\ie\ lower entropy) than their surroundings. One possible explanation
is that the gas in these two regions has been displaced from deeper
within the core where lower entropy gas resides.

-- are E and W blobs different?\\

-- lifetime of cold blobs\\

The EC and WC reside in ambient mediums which are hotter and of
comparable density. Along with the relatively large surface areas of
the clouds, if conduction is not too heavily supressed, then the
clouds will evaporate. Assuming the heat flux across the cloud-ICM
interface is saturated \citep{1977ApJ...211..135C}, then the heat flux
is
\begin{equation}
  q_{\mathrm{sat}} = f_{\mathrm{C}} \left(\frac{2 \tx}{\pi \melec}\right)^{1/2} \nelec \tx
  \label{eqn:q}
\end{equation}
where $q_{\mathrm{sat}}$ has units keV \cmsq\ \ps, $f_{\mathrm{C}}$ is
a conduction suppresion factor due to magnetic fields, \melec\ is the
electron mass in keV $c^{-2}$, \tx\ is ambient temperature in keV, and
\nelec\ is ambient density in \pcc. Typical values of $f_{\mathrm{C}}$
for the ICM are 0.2-0.4 \citep{2001ApJ...554..561M,
  2001ApJ...562L.129N}, and we use $f_{\mathrm{C}}=0.4$. The total
heat transfer across the cloud-ICM interface is then
\begin{equation}
  Q = 4 \pi r^2 q_{\mathrm{sat}}
  \label{eqn:Q}
\end{equation}
where $Q$ has units erg \ps, and $r$ is the radius of the cloud in
cm. The enthalpy of the gas cloud which must be replaced is
\begin{equation}
  H = \frac{\gamma p V}{\gamma-1}
  \label{eqn:H}
\end{equation}
where $H$ has units of erg, $p$ is cloud pressure in erg \pcc, $V$ is
cloud volume in \cc, and we assume $\gamma = 5/3$ for the ratio of
specific heat capacities. In terms of Equations \ref{eqn:Q} and
\ref{eqn:H}, the evaporation time scale of the clouds is
\begin{equation}
  t_{\mathrm{evap}} = \frac{H}{3.16\times10^{13}~Q}
\end{equation}
where $t_{\mathrm{evap}}$ has units Myr.

Assuming conduction will completely erase the low entropy clouds, the
remaining EC lifetime is $t_{\mathrm{evap}} \la 1.4$ Myr, and the WC
lifetime is $t_{\mathrm{evap}} \la 0.5$ Myr. These are the time scales
required to evaporate the observed clouds, which says nothing of gas
which may have previously been heated and then mixed with the ambient
medium. Assume that: (1) the EC is young and composed of gas dredged
up by the new jet; (2) the WC is older and composed of gas dredged up
and moved aside by the old jet. Further assuming that the initial
state of the gas in the EC and WC was the same, then as a result of
the WC having been conductively heated for longer, the comparable gas
entropies but differing evaporation time scales make sense as only the
lowest entropy gas in the WC should be present and observeable.

-- star formation in WC?\\
SFR from radio obs: \citep{1992ARA&A..30..575C}
\citet{2009MNRAS.397.1101G} show the radio SFR relation in
\citet{2003ApJ...586..794B} applies to $L > L_{\star}$ galaxies

relating radio synchrotron luminosity to massive ($M \ga 5 \Msol$)
star formation rate, $\Psi$, can be written in terms of observed flux
density as
\begin{equation}
  \frac{\Psi}{\Msolpy} =
  \frac{0.066}{(1+z)}~\left(\frac{D_L}{\Mpc}\right)^2~\left[\frac{(1+z)\nu_0}{1.4
      \GHz}\right]^{\alpha}~\left(\frac{S_{\nu}}{\jy}\right) \nonumber
\end{equation}
where $z$ is source redshift, $S_{\nu}$ is the radio flux density due
to synchrotron emission, $\nu_0$ is the observing frequency in GHz,
$D_L$ is the source luminosity distance in Mpc, and $\alpha$ is the
radio spectral index. We assume $S_{\nu} \propto \nu^{-\alpha}$ with
$\alpha = 1.4$ taken from \citet{1993ApJ...415...82H}.

-- uv filament?\\
-- no enhancement to south, why?\\

%%%%%%%%%%%%%%%%%%%%%%%%%%%
\subsection{Central Source}
\label{sec:centsrc}
%%%%%%%%%%%%%%%%%%%%%%%%%%%

%% Iwa01: ``The soft X-ray emission seen in the nuclear spectrum is
%% probably caused by photoionized gas in the inner nucleus. The highly
%% polarized biconical reflection nebula imaged by the Hubble Space
%% Telescope (HST) (Hines et al. 1999) is, however, not a likely source,
%% as scattering by dust rather than electrons appears to be a dominant
%% mechanism to induce the high polarization (Tran et al.  2000). The
%% inner wall of the obscuring torus is exposed to the intense radiation
%% from the primary source and thus expected to be highly
%% ionized. Provided the optical depth of the ionized gas is small, the
%% Fe L emission bump can be very strong (e.g. Band et al.  1990). A
%% sub-pc scale ionized disc has indeed been imaged with the Very Large
%% Baseline Array (VLBA) in the nucleus of the nearby Compton-thick
%% Seyfert 2 galaxy NGC 1068 (Gallimore, Baum & O'Dea 1997), and they
%% predicted strong soft X-ray emission lines based on a photoionization
%% computation. In this case, the luminosity of the emission from the
%% photoionized gas should be larger than observed ($~3x10^42 h^-2 erg
%% s^-1$ in the 0.5-2 keV band), as significant absorption is likely to
%% occur during the escape from the nuclear region.''

The cD envelope and nuclear region of \iras\ is an extremely active
location. In the central $\sim10$ kpc, there is a bright extended
region with an associated point source. Previous work has suggested
the nuclear quasar is ``changing-look'' \citep{2003MNRAS.342..422M}.

The location of the nuclear point source was determined using a
hardness ratio map, $HR = f(2.0-8.0~\keV)/f(0.3-2.0~\keV)$, shown in
Figure \ref{fig:hardness}. For both \chandra\ observations, an image
of the normalized point-spread function (PSF) for the median photon
energy ($E \approx 1.25$ kev) and location ($\theta_{off-axis} =
2.27\arcm$) of the hard point source was created following standard
\ciao\ procedures. An elliptical region enclosing 90\% of the PSF
energy fraction was then generated. The resultant region had an
effective radius of $1.16\arcs$. A partially elliptical annular
background region with the same central coordinates, ellipticity, and
position angle as the 90\% EEF region, but having 5 times the area,
was created. The background annulus is broken into segments to avoid
the regions of uplifted gas. Source and background spectra were then
extracted using the \ciao\ tool {\tt psextract}. The source spectrum
was grouped to have 15 counts per energy channel. Approximately XX\%
of the 2009 spectrum is from the source, background subtracted source
count rate of $X.XX\times10^{XX} \pm X.XX\times10^{XX}$ ct \ps,
yielding $\approx XXX$ source counts. For the 1999 spectrum, XX\% is
source, with a background-subtracted count rate of $X.XX\times10^{XX}
\pm X.XX\times10^{XX}$ ct \ps, resulting in $\approx XXX$ source
counts. For both spectra, the number of source counts exceeds the low
count regime, thus grouped spectra and the \chisq\ statistic were used
during fitting.

In addition to the \chandra\ data, we retrieved and re-analyzed the
\bepposax\ PDS X-ray data (15-220 keV) taken April 18\ths, 1998
(ObsCode 50273002; PI Franceschini). The hard X-ray data from PDS
provides an important constraint on the behavior of the power-law
spectral component. The data was reduced and analyzed with
\saxdas\ v2.3.1 using the calibration data and cookbook available via
HEASARC\footnote{\url{http://heasarc.nasa.gov/docs/sax/shp\_software.html}}. Our
measured PDS 15-80 keV count rate, $0.106 \pm 0.055$ ct \ps, is
consistent with published values \citep{2000A&A...353..910F}.

The nuclear spectrum is complex and varying with time. Shown in Figure
\ref{fig:nucspec} are the 1999 and 2009 background-subtracted
\chandra\ spectra extracted from the regions shown in Figure
\ref{fig:hardness}. The spectra have been corrected for both soft
Galactic emission and azimuthally averaged ICM emission from the
central annulus of the temperature profile. A soft-excess is evident
at $E < 1$ keV, and there may be evolution of the hard component at $E
> 5$ keV. Because of these complexities, our discussion is divided
into two following sections.

%%%%%%%%%%%%%%%%%%%%%%%%%%%%%%%%%%%%%
\subsubsection{Hard X-ray Componenet}
\label{sec:hard}
%%%%%%%%%%%%%%%%%%%%%%%%%%%%%%%%%%%%%

%%%%%%%%%%%%%%%%%%%%%%%%%%%%%%%%%
\subsubsection{Soft X-ray Excess}
\label{sec:soft}
%%%%%%%%%%%%%%%%%%%%%%%%%%%%%%%%%

The best-fit parameters are listed in Table \ref{tab:nucspec}. The
spectrum and best-fit model are shown in Figure \ref{fig:nucspec}.

\absori\ model for an ionized absorber \citep{1992ApJ...395..275D,
  1995ApJ...438L..63Z}

\mekal\ model for thermal emission \citep{mekal1, mekal2}

\pexrav\ model for reflected emission from cold matter \citep{pexrav}

\warmabs\ model, which is an implementation of the photoionization
code provided in \xstar\ \citep{xstar1, xstar2}.

background region:
2009 data:
see reflected emission and k-alpha in bgd region
based on this, expand extraction region

models considered
2009 data:

- wabs(pexrav+gaussian)
- good fit at kalpha line (4.46 keV, EW 446 eV) and E > 4 keV with
gamma=2.0, nh at galactic, fe fixed at 2.0 sol,
- not enough emission at < 2 kev, systematic underestimate at Fe-L
hump, specifically at 1.25 keV
- needs Nh lower than galactic
- too much flux at 2-4 keV 
- poor fit (chisq 1.31/57 dof)
- thawing Fe gives better agreement, but with > 40 sol (chisq 1.23/56 dof)

- wabs(plcabs+gaussian)
- powerlaw observed through dense, cold matter
- absorber density goes to zero
- poor fit

- wabs(pexrav+gaussian+zwabs(powerlaw))
- absorber density goes to zero
- same problems as just pexrav model


1999 data
- wabs(pexrav+gauss+apec)
- frozen to be bf model from 2009 data
- left zwabs free and kalpha eqw
- poor fit chisq 1.63/8 dof
- kalpha eqw 719 eV (1.06 keV eqw red-corr)

soft excess:
L-shell emission
\citet{1990ApJ...362...90B}



-- discuss best-fit model\\
-- interpretation of best-fit params\\
-- nature of central region\\
-- changing-look AGN?\\
-- Very strong OIII lines typical of high-ionization suggest an AGN
does not power the emission line spectrum.

%%%%%%%%%%%%%%%%%%%%%%%%%%%%%%%%%%%%%%%%%%%%%%%%%%%%%%%
\subsection{New Epoch of AGN Outburst, Star Formation?}
\label{sec:centsrc}
%%%%%%%%%%%%%%%%%%%%%%%%%%%%%%%%%%%%%%%%%%%%%%%%%%%%%%%

TBD

%%%%%%%%%%%%%%%%%%%%%%%%%%%%%%%%%
\section{Summary \& Conclusions}
\label{sec:summary}
%%%%%%%%%%%%%%%%%%%%%%%%%%%%%%%%%

TBD

%%%%%%%%%%%%%%%%%
\acknowledgements
%%%%%%%%%%%%%%%%%

This work was supported by CXO grant A00-0000A. The \cxo\ Center is
operated by the Smithsonian Astrophysical Observatory for and on
behalf of NASA under contract NAS8-03060. The VLA (Very Large Array)
is a facility of the National Radio Astronomy Observatory (NRAO). The
NRAO is a facility of the National Science Foundation operated under
cooperative agreement by Associated Universities, Inc. This research
has made use of software provided by the Chandra X-ray Center in the
application packages \ciao, \chips, and \sherpa. This research has
made use of the NASA/IPAC Extragalactic Database which is operated by
the Jet Propulsion Laboratory, California Institute of Technology,
under contract with NASA. This research has also made use of NASA's
Astrophysics Data System. Some software was obtained from the High
Energy Astrophysics Science Archive Research Center, provided by
NASA's Goddard Space Flight Center.

%%%%%%%%%%%%%%
% Facilities %
%%%%%%%%%%%%%%

{\it Facilities:} \facility{Beppo-SAX}, \facility{CXO}, \facility{VLA}

%%%%%%%%%%%%%%%%
% Bibliography %
%%%%%%%%%%%%%%%%

\bibliography{cavagnolo}

%%%%%%%%%%%%%%%%%%%%%%%
% Figures  and Tables %
%%%%%%%%%%%%%%%%%%%%%%%

\clearpage
\begin{deluxetable}{lccccccccc}
\tablewidth{0pt}
\tabletypesize{\scriptsize}
\tablecaption{Summary of Global Spectral Properties\label{tab:specfits}}
\tablehead{\colhead{Region} & \colhead{$R_{in}$} & \colhead{$R_{out}$ } & \colhead{$N_{HI}$} & \colhead{$T_{X}$} & \colhead{$Z$} & \colhead{redshift} & \colhead{$\chi^2_{red.}$} & \colhead{D.O.F.} & \colhead{\% Source}\\
\colhead{ } & \colhead{kpc} & \colhead{kpc} & \colhead{$10^{20}$ cm$^{-2}$} & \colhead{keV} & \colhead{$Z_{\sun}$} & \colhead{ } & \colhead{ } & \colhead{ } & \colhead{ }\\
\colhead{{(1)}} & \colhead{{(2)}} & \colhead{{(3)}} & \colhead{{(4)}} & \colhead{{(5)}} & \colhead{{(6)}} & \colhead{{(7)}} & \colhead{{(8)}} & \colhead{{(9)}} & \colhead{{(10)}}
}
\startdata
$R_{500-core}$ & 251 & 1675 & 2.86$^{+2.46}_{-2.75}$  & 13.26$^{+6.21}_{-2.95}$  & 0.54$^{+0.28}_{-0.26}$  & 0.3605$^{+0.0235}_{-0.0162}$  & 1.10 & 368 &  19\\
$R_{1000-core}$ & 251 & 1184 & 3.13$^{+2.31}_{-2.33}$  & 11.20$^{+3.11}_{-1.97}$  & 0.51$^{+0.21}_{-0.21}$  & 0.3639$^{+0.0201}_{-0.0156}$  & 1.08 & 292 &  25\\
$R_{2500-core}$ & 251 & 749 & 1.90$^{+2.30}_{-1.90}$ & 10.66$^{+2.20}_{-1.65}$  & 0.60$^{+0.22}_{-0.19}$  & 0.3611$^{+0.0143}_{-0.0127}$  & 1.01 & 219 &  37\\
$R_{5000-core}$ & 251 & 529 & 3.22$^{+2.75}_{-2.58}$  & 8.80$^{+1.87}_{-1.31}$  & 0.46$^{+0.19}_{-0.18}$  & 0.3556$^{+0.0134}_{-0.0119}$  & 1.11 & 170 &  50\\
$R_{7500-core}$ & 251 & 432 & 2.70$^{+3.02}_{-2.70}$ & 9.85$^{+2.80}_{-1.90}$  & 0.35$^{+0.22}_{-0.21}$  & 0.3632$^{+0.0240}_{-0.0231}$  & 1.06 & 138 &  57\\
$R_{500}$ & \nodata & 1675 & 3.22$^{+1.09}_{-1.02}$  & 7.28$^{+0.50}_{-0.45}$  & 0.41$^{+0.06}_{-0.06}$  & 0.3563$^{+0.0053}_{-0.0044}$  & 0.95 & 535 &  39\\
$R_{1000}$ & \nodata & 1184 & 3.25$^{+1.00}_{-0.93}$  & 7.05$^{+0.40}_{-0.39}$  & 0.40$^{+0.05}_{-0.05}$  & 0.3573$^{+0.0043}_{-0.0040}$  & 0.91 & 488 &  51\\
$R_{2500}$ & \nodata & 749 & 2.97$^{+0.87}_{-0.97}$  & 6.88$^{+0.38}_{-0.33}$  & 0.41$^{+0.05}_{-0.05}$  & 0.3558$^{+0.0026}_{-0.0046}$  & 0.88 & 442 &  70\\
$R_{5000}$ & \nodata & 529 & 3.10$^{+0.90}_{-0.96}$  & 6.66$^{+0.34}_{-0.31}$  & 0.40$^{+0.05}_{-0.05}$  & 0.3560$^{+0.0028}_{-0.0047}$  & 0.85 & 418 &  81\\
$R_{7500}$ & \nodata & 432 & 3.17$^{+0.97}_{-1.04}$  & 6.61$^{+0.38}_{-0.31}$  & 0.39$^{+0.05}_{-0.05}$  & 0.3550$^{+0.0037}_{-0.0047}$  & 0.86 & 410 &  85\\
\hline
$R_{500-core}$ & 251 & 1675 & 3.18$^{+2.46}_{-2.61}$  & 12.63$^{+5.19}_{-2.54}$  & 0.53$^{+0.27}_{-0.26}$  & 0.3540 & 1.10 & 369 &  19\\
$R_{1000-core}$ & 251 & 1184 & 3.40$^{+2.19}_{-2.22}$  & 10.79$^{+2.69}_{-1.70}$  & 0.50$^{+0.20}_{-0.21}$  & 0.3540 & 1.08 & 293 &  25\\
$R_{2500-core}$ & 251 & 749 & 2.19$^{+2.20}_{-2.15}$  & 10.33$^{+2.08}_{-1.50}$  & 0.60$^{+0.21}_{-0.20}$  & 0.3540 & 1.01 & 220 &  37\\
$R_{5000-core}$ & 251 & 529 & 3.25$^{+2.63}_{-2.49}$  & 8.76$^{+1.73}_{-1.30}$  & 0.46$^{+0.18}_{-0.17}$  & 0.3540 & 1.10 & 171 &  50\\
$R_{7500-core}$ & 251 & 432 & 2.99$^{+2.94}_{-2.79}$  & 9.56$^{+2.67}_{-1.74}$  & 0.34$^{+0.21}_{-0.21}$  & 0.3540 & 1.05 & 139 &  57\\
$R_{500}$ & \nodata & 1675 & 3.25$^{+1.01}_{-1.01}$  & 7.25$^{+0.46}_{-0.42}$  & 0.41$^{+0.06}_{-0.06}$  & 0.3540 & 0.95 & 536 &  39\\
$R_{1000}$ & \nodata & 1184 & 3.31$^{+0.99}_{-0.98}$  & 7.00$^{+0.40}_{-0.37}$  & 0.40$^{+0.05}_{-0.05}$  & 0.3540 & 0.91 & 489 &  51\\
$R_{2500}$ & \nodata & 749 & 2.95$^{+0.97}_{-0.95}$  & 6.87$^{+0.37}_{-0.33}$  & 0.41$^{+0.06}_{-0.05}$  & 0.3540 & 0.88 & 443 &  70\\
$R_{5000}$ & \nodata & 529 & 3.10$^{+0.98}_{-0.95}$  & 6.65$^{+0.34}_{-0.32}$  & 0.40$^{+0.05}_{-0.05}$  & 0.3540 & 0.85 & 419 &  81\\
$R_{7500}$ & \nodata & 432 & 3.17$^{+0.99}_{-0.97}$  & 6.60$^{+0.34}_{-0.31}$  & 0.39$^{+0.05}_{-0.05}$  & 0.3540 & 0.86 & 411 &  85\\
\hline
$R_{500-core}$ & 251 & 1675 & 2.22 & 14.16$^{+3.68}_{-2.43}$  & 0.54$^{+0.30}_{-0.28}$  & 0.3626$^{+0.0215}_{-0.0219}$  & 1.09 & 369 &  19\\
$R_{1000-core}$ & 251 & 1184 & 2.22 & 11.91$^{+2.15}_{-1.52}$  & 0.52$^{+0.22}_{-0.22}$  & 0.3658$^{+0.0226}_{-0.0160}$  & 1.08 & 293 &  25\\
$R_{2500-core}$ & 251 & 749 & 2.22 & 10.46$^{+1.49}_{-1.15}$  & 0.60$^{+0.20}_{-0.19}$  & 0.3613$^{+0.0144}_{-0.0132}$  & 1.01 & 220 &  37\\
$R_{5000-core}$ & 251 & 529 & 2.22 & 9.26$^{+1.27}_{-1.02}$  & 0.46$^{+0.19}_{-0.18}$  & 0.3560$^{+0.0140}_{-0.0061}$  & 1.11 & 171 &  50\\
$R_{7500-core}$ & 251 & 432 & 2.22 & 10.15$^{+1.81}_{-1.36}$  & 0.35$^{+0.22}_{-0.20}$  & 0.3647$^{+0.0123}_{-0.0231}$  & 1.05 & 139 &  57\\
$R_{500}$ & \nodata & 1675 & 2.22 & 7.63$^{+0.33}_{-0.30}$  & 0.41$^{+0.06}_{-0.06}$  & 0.3567$^{+0.0048}_{-0.0034}$  & 0.96 & 536 &  39\\
$R_{1000}$ & \nodata & 1184 & 2.22 & 7.38$^{+0.29}_{-0.29}$  & 0.40$^{+0.06}_{-0.05}$  & 0.3586$^{+0.0027}_{-0.0067}$  & 0.91 & 489 &  51\\
$R_{2500}$ & \nodata & 749 & 2.22 & 7.09$^{+0.26}_{-0.23}$  & 0.41$^{+0.06}_{-0.05}$  & 0.3556$^{+0.0031}_{-0.0045}$  & 0.88 & 443 &  70\\
$R_{5000}$ & \nodata & 529 & 2.22 & 6.90$^{+0.23}_{-0.23}$  & 0.40$^{+0.05}_{-0.05}$  & 0.3560$^{+0.0027}_{-0.0048}$  & 0.85 & 419 &  81\\
$R_{7500}$ & \nodata & 432 & 2.22 & 6.86$^{+0.24}_{-0.22}$  & 0.39$^{+0.05}_{-0.05}$  & 0.3558$^{+0.0020}_{-0.0046}$  & 0.87 & 411 &  85\\
\hline
$R_{500-core}$ & 251 & 1675 & 2.22 & 13.80$^{+3.08}_{-2.21}$  & 0.53$^{+0.28}_{-0.28}$  & 0.3540 & 1.09 & 370 &  19\\
$R_{1000-core}$ & 251 & 1184 & 2.22 & 11.64$^{+1.09}_{-1.44}$  & 0.50$^{+0.22}_{-0.22}$  & 0.3540 & 1.08 & 294 &  25\\
$R_{2500-core}$ & 251 & 749 & 2.22 & 10.31$^{+1.38}_{-1.09}$  & 0.60$^{+0.20}_{-0.20}$  & 0.3540 & 1.01 & 221 &  37\\
$R_{5000-core}$ & 251 & 529 & 2.22 & 9.22$^{+1.21}_{-0.97}$  & 0.47$^{+0.19}_{-0.18}$  & 0.3540 & 1.10 & 172 &  50\\
$R_{7500-core}$ & 251 & 432 & 2.22 & 10.01$^{+1.77}_{-1.33}$  & 0.34$^{+0.21}_{-0.22}$  & 0.3540 & 1.05 & 140 &  57\\
$R_{500}$ & \nodata & 1675 & 2.22 & 7.60$^{+0.32}_{-0.30}$  & 0.41$^{+0.06}_{-0.06}$  & 0.3540 & 0.96 & 537 &  39\\
$R_{1000}$ & \nodata & 1184 & 2.22 & 7.34$^{+0.28}_{-0.26}$  & 0.40$^{+0.06}_{-0.05}$  & 0.3540 & 0.92 & 490 &  51\\
$R_{2500}$ & \nodata & 749 & 2.22 & 7.08$^{+0.25}_{-0.23}$  & 0.42$^{+0.05}_{-0.06}$  & 0.3540 & 0.88 & 444 &  70\\
$R_{5000}$ & \nodata & 529 & 2.22 & 6.88$^{+0.23}_{-0.22}$  & 0.40$^{+0.05}_{-0.05}$  & 0.3540 & 0.85 & 420 &  81\\
$R_{7500}$ & \nodata & 432 & 2.22 & 6.85$^{+0.24}_{-0.22}$  & 0.39$^{+0.05}_{-0.05}$  & 0.3540 & 0.87 & 412 &  85\\
\hline
NE-Arm & \nodata & \nodata & 2.22 & 5.00$^{+1.08}_{-0.78}$  & 1.23$^{+1.28}_{-0.79}$  & 0.3540 & 0.19 & 300 &  98\\
NW-Arm & \nodata & \nodata & 2.22 & 5.73$^{+1.28}_{-0.98}$  & 0.54$^{+0.55}_{-0.48}$  & 0.3540 & 0.26 & 184 &  99\\
SE-Arm & \nodata & \nodata & 2.22 & 5.49$^{+1.24}_{-0.80}$  & 0.36$^{+0.41}_{-0.36}$  & 0.3540 & 0.16 & 268 &  99\\
SW-Arm & \nodata & \nodata & 2.22 & 5.99$^{+1.41}_{-0.98}$  & 1.04$^{+0.73}_{-0.49}$  & 0.3540 & 0.14 & 338 &  99\\
\hline
\enddata
\tablecomments{Col. (1) Name of region used for spectral extraction; col. (2) inner radius of extraction region; col. (3) outer radius of extraction region; col. (4) absorbing, Galactic neutral hydrogen column density; col. (5) best-fit temperature; col. (6) best-fit metallicity; col. (7) best-fit redshift; col. (8) reduced \chisq\ for best-fit model; col. (9) degrees of freedom for best-fit model; col. (10) percentage of emission attributable to source.}
\end{deluxetable}

%\begin{table*}
  \begin{center}
    \caption{\sc Summary of Cavity Properties.\label{tab:cylcavities}}
    \begin{tabular}{lcccccc}
      \hline
      \hline
      Cavity & $r$ & $l$ & \tsonic & $pV$ & \ecav & \pcav\\
      -- & kpc & kpc & $10^6$ yr & $10^{58}$ ergs & $10^{59}$ ergs & $10^{44}$ ergs s$^{-1}$\\
      (1) & (2) & (3) & (4) & (5) & (6) & (7)\\
      \hline
      Northwest & 6.40 & 58.3 & ${50.5 \pm 7.6}$ & ${5.78 \pm 1.07}$ & ${2.31 \pm 0.43}$ & ${1.45 \pm 0.35}$\\
      Southeast & 6.81 & 64.0 & ${55.4 \pm 8.4}$ & ${6.99 \pm 1.29}$ & ${2.80 \pm 0.52}$ & ${1.60 \pm 0.38}$\\
      \hline
    \end{tabular}
    \begin{quote}
      Col. (1) Cavity location; Col. (2) Radius of excavated cylinder;
      Col. (3) Length of excavated cylinder; Col. (4) Sound speed age;
      Col. (5) $pV$ work; Col. (6) Cavity energy; Col. (7) Cavity power.
    \end{quote}
  \end{center}
\end{table*}

%\begin{table*}
  \begin{center}
    \caption{\sc Summary of Nuclear Source Spectral Fits.\label{tab:nucspec}}
    \begin{tabular}{lccc}
      \hline
      \hline
      Component & Parameter & SP09 & SP99\\
      (1) & (2) & (3) & (4)\\
      \hline
      \pexrav\  & $\Gamma$              & $1.71^{+0.23}_{-0.65}$                & fixed to SP09\\
      -         & $\eta_{\mathrm{P}}$   & $8.07^{+0.64}_{-0.62}\times10^{-4}$   & $8.46^{+2.08}_{-2.12} \times 10^{-4}$\\
      \gauss\ 1 & $E_{\mathrm{G}}$      & $0.73^{+0.05}_{-0.24}$                & $0.61^{+0.10}_{-0.05}$\\
      -         & $\sigma_{\mathrm{G}}$ & $85^{+197}_{-53}$                     & $97^{+150}_{-97}$\\
      -         & $\eta_{\mathrm{G}}$   & $8.14^{+3.74}_{-5.82} \times 10^{-6}$ & $1.65^{+1.52}_{-1.00} \times 10^{-5}$\\
      \gauss\ 2 & $E_{\mathrm{G}}$      & $1.16^{+0.19}_{-0.33}$                & $0.90^{+0.17}_{-0.90}$\\
      -         & $\sigma_{\mathrm{G}}$ & $383^{+610}_{-166}$                   & $506^{+314}_{-262}$\\
      -         & $\eta_{\mathrm{G}}$   & $1.03^{+3.22}_{-0.48} \times 10^{-5}$ & $1.48^{+2.68}_{-1.16} \times 10^{-5}$\\
      \gauss\ 3 & $E_{\mathrm{G}}$      & $4.45^{+0.04}_{-0.04}$                & $4.46^{+0.04}_{-0.07}$\\
      -         & $\sigma_{\mathrm{G}}$ & $45^{+60}_{-45}$                      & $31^{+94}_{-31}$\\
      -         & $\eta_{\mathrm{G}}$   & $2.67^{+0.91}_{-0.86} \times 10^{-6}$ & $6.45^{+4.17}_{-3.69} \times 10^{-6}$\\
      -         & EW$^{\mathrm{corr}}_{\mathrm{K}\alpha}$ & $531^{+211}_{-218}$ & $1210^{+720}_{-710}$\\
      Statistic & \chisq                & 79.0                                  & 7.9\\
      -         & DOF                   & 74                                    & 15\\
      \hline
    \end{tabular}
    \begin{quote}
      \feka\ equivalent widths have been corrected for redshift. Units for
      parameters: $\Gamma$ is dimensionless, $\eta_{\mathrm{P}}$ is in ph
      keV$^{-1}$ cm$^{-2}$ s$^{-1}$, $E_{\mathrm{G}}$ are in keV,
      $\sigma_{\mathrm{G}}$ are in eV, $\eta_{\mathrm{G}}$ are in ph
      cm$^{-2}$ s$^{-1}$, EW$_{\mathrm{corr}}$ are in eV. Col. (1)
      \xspec\ model name; Col. (2) Model parameters; Col. (3) Values for
      2009 \cxo\ spectrum; Col. (4) Values for 1999 \cxo\ spectrum.
    \end{quote}
  \end{center}
\end{table*}

%\begin{figure}
  \begin{center}
    \begin{minipage}{\linewidth}
      \includegraphics*[width=\textwidth, trim=0mm 0mm 0mm 0mm, clip]{rbs797.ps}
    \end{minipage}
    \caption{Fluxed, unsmoothed 0.7--2.0 keV clean image of \rbs\ in
      units of ph \pcmsq\ \ps\ pix$^{-1}$. Image is $\approx 250$ kpc
      on a side and coordinates are J2000 epoch. Black contours in the
      nucleus are 2.5--9.0 keV X-ray emission of the nuclear point
      source; the outer contour approximately traces the 90\% enclosed
      energy fraction (EEF) of the \cxo\ point spread function. The
      dashed green ellipse is centered on the nuclear point source,
      encloses both cavities, and was drawn by-eye to pass through the
      X-ray ridge/rims.}
    \label{fig:img}
  \end{center}
\end{figure}

\begin{figure}
  \begin{center}
    \begin{minipage}{0.495\linewidth}
      \includegraphics*[width=\textwidth, trim=0mm 0mm 0mm 0mm, clip]{325.ps}
    \end{minipage}
   \begin{minipage}{0.495\linewidth}
      \includegraphics*[width=\textwidth, trim=0mm 0mm 0mm 0mm, clip]{8.4.ps}
   \end{minipage}
   \begin{minipage}{0.495\linewidth}
      \includegraphics*[width=\textwidth, trim=0mm 0mm 0mm 0mm, clip]{1.4.ps}
    \end{minipage}
    \begin{minipage}{0.495\linewidth}
      \includegraphics*[width=\textwidth, trim=0mm 0mm 0mm 0mm, clip]{4.8.ps}
    \end{minipage}
     \caption{Radio images of \rbs\ overlaid with black contours
       tracing ICM X-ray emission. Images are in mJy beam$^{-1}$ with
       intensity beginning at $3\sigma_{\rm{rms}}$ and ending at the
       peak flux, and are arranged by decreasing size of the
       significant, projected radio structure. X-ray contours are from
       $2.3 \times 10^{-6}$ to $1.3 \times 10^{-7}$ ph
       \pcmsq\ \ps\ pix$^{-1}$ in 12 square-root steps. {\it{Clockwise
           from top left}}: 325 MHz \vla\ A-array, 8.4 GHz
       \vla\ D-array, 4.8 GHz \vla\ A-array, and 1.4 GHz
       \vla\ A-array.}
    \label{fig:composite}
  \end{center}
\end{figure}

\begin{figure}
  \begin{center}
    \begin{minipage}{0.495\linewidth}
      \includegraphics*[width=\textwidth, trim=0mm 0mm 0mm 0mm, clip]{sub_inner.ps}
    \end{minipage}
    \begin{minipage}{0.495\linewidth}
      \includegraphics*[width=\textwidth, trim=0mm 0mm 0mm 0mm, clip]{sub_outer.ps}
    \end{minipage}
    \caption{Red text point-out regions of interest discussed in
      Section \ref{sec:cavities}. {\it{Left:}} Residual 0.3-10.0 keV
      X-ray image smoothed with $1\arcs$ Gaussian. Yellow contours are
      1.4 GHz emission (\vla\ A-array), orange contours are 4.8 GHz
      emission (\vla\ A-array), orange vector is 4.8 GHz jet axis, and
      red ellipses outline definite cavities. {\it{Bottom:}} Residual
      0.3-10.0 keV X-ray image smoothed with $3\arcs$ Gaussian. Green
      contours are 325 MHz emission (\vla\ A-array), blue contours are
      8.4 GHz emission (\vla\ D-array), and orange vector is 4.8 GHz
      jet axis.}
    \label{fig:subxray}
  \end{center}
\end{figure}

\begin{figure}
  \begin{center}
    \begin{minipage}{\linewidth}
      \includegraphics*[width=\textwidth]{r797_nhfro.eps}
      \caption{Gallery of radial ICM profiles. Vertical black dashed
        lines mark the approximate end-points of both
        cavities. Horizontal dashed line on cooling time profile marks
        age of the Universe at redshift of \rbs. For X-ray luminosity
        profile, dashed line marks \lcool, and dashed-dotted line
        marks \pcav.}
      \label{fig:gallery}
    \end{minipage}
  \end{center}
\end{figure}

\begin{figure}
  \begin{center}
    \begin{minipage}{\linewidth}
      \setlength\fboxsep{0pt}
      \setlength\fboxrule{0.5pt}
      \fbox{\includegraphics*[width=\textwidth]{cav_config.eps}}
    \end{minipage}
    \caption{Cartoon of possible cavity configurations. Arrows denote
      direction of AGN outflow, ellipses outline cavities, \rlos\ is
      line-of-sight cavity depth, and $z$ is the height of a cavity's
      center above the plane of the sky. {\it{Left:}} Cavities which
      are symmetric about the plane of the sky, have $z=0$, and are
      inflating perpendicular to the line-of-sight. {\it{Right:}}
      Cavities which are larger than left panel, have non-zero $z$,
      and are inflating along an axis close to our line-of-sight.}
    \label{fig:config}
  \end{center}
\end{figure}

\begin{figure}
  \begin{center}
    \begin{minipage}{0.495\linewidth}
      \includegraphics*[width=\textwidth, trim=25mm 0mm 40mm 10mm, clip]{edec.eps}
    \end{minipage}
    \begin{minipage}{0.495\linewidth}
      \includegraphics*[width=\textwidth, trim=25mm 0mm 40mm 10mm, clip]{wdec.eps}
    \end{minipage}
    \caption{Surface brightness decrement as a function of height
      above the plane of the sky for a variety of cavity radii. Each
      curve is labeled with the corresponding \rlos. The curves
      furthest to the left are for the minimum \rlos\ needed to
      reproduce $y_{\rm{min}}$, \ie\ the case of $z = 0$, and the
      horizontal dashed line denotes the minimum decrement for each
      cavity. {\it{Left}} Cavity E1; {\it{Right}} Cavity W1.}
    \label{fig:decs}
  \end{center}
\end{figure}


\begin{figure}
  \begin{center}
    \begin{minipage}{\linewidth}
      \includegraphics*[width=\textwidth, trim=15mm 5mm 5mm 10mm, clip]{pannorm.eps}
      \caption{Histograms of normalized surface brightness variation
        in wedges of a $2.5\arcs$ wide annulus centered on the X-ray
        peak and passing through the cavity midpoints. {\it{Left:}}
        $36\mydeg$ wedges; {\it{Middle:}} $14.4\mydeg$ wedges;
        {\it{Right:}} $7.2\mydeg$ wedges. The depth of the cavities
        and prominence of the rims can be clearly seen in this plot.}
      \label{fig:pannorm}
    \end{minipage}
  \end{center}
\end{figure}

\begin{figure}
  \begin{center}
    \begin{minipage}{0.5\linewidth}
      \includegraphics*[width=\textwidth, angle=-90]{nucspec.ps}
    \end{minipage}
    \caption{X-ray spectrum of nuclear point source. Black denotes
      year 2000 \cxo\ data (points) and best-fit model (line), and red
      denotes year 2007 \cxo\ data (points) and best-fit model (line).
      The residuals of the fit for both datasets are given below.}
    \label{fig:nucspec}
  \end{center}
\end{figure}

\begin{figure}
  \begin{center}
    \begin{minipage}{\linewidth}
      \includegraphics*[width=\textwidth, trim=10mm 5mm 10mm 10mm, clip]{radiofit.eps}
    \end{minipage}
    \caption{Best-fit continuous injection (CI) synchrotron model to
      the nuclear 1.4 GHz, 4.8 GHz, and 7.0 keV X-ray emission. The
      two triangles are \galex\ UV fluxes showing the emission is
      boosted above the power-law attributable to the nucleus.}
    \label{fig:sync}
    \end{center}
\end{figure}

\begin{figure}
  \begin{center}
    \begin{minipage}{\linewidth}
      \includegraphics*[width=\textwidth, trim=0mm 0mm 0mm 0mm, clip]{rbs797_opt.ps}
    \end{minipage}
    \caption{\hst\ \myi+\myv\ image of the \rbs\ BCG with units e$^-$
      s$^{-1}$. Green, dashed contour is the \cxo\ 90\% EEF. Emission
      features discussed in the text are labeled.}
    \label{fig:hst}
  \end{center}
\end{figure}

\begin{figure}
  \begin{center}
    \begin{minipage}{0.495\linewidth}
      \includegraphics*[width=\textwidth, trim=0mm 0mm 0mm 0mm, clip]{suboptcolor.ps}
    \end{minipage}
    \begin{minipage}{0.495\linewidth}
      \includegraphics*[width=\textwidth, trim=0mm 0mm 0mm 0mm, clip]{suboptrad.ps}
    \end{minipage}
    \caption{{\it{Left:}} Residual \hst\ \myv\ image. White regions
      (numbered 1--8) are areas with greatest color difference with
      \rbs\ halo. {\it{Right:}} Residual \hst\ \myi\ image. Green
      contours are 4.8 GHz radio emission down to
      $1\sigma_{\rm{rms}}$, white dashed circle has radius $2\arcs$,
      edge of ACS ghost is show in yellow, and southern whiskers are
      numbered 9--11 with corresponding white lines.}
    \label{fig:subopt}
  \end{center}
\end{figure}


%%%%%%%%%%%%%%%%%%%%
% End the document %
%%%%%%%%%%%%%%%%%%%%

\end{document}
