%%%%%%%%%%%%%%%%%%%
% Custom commands %
%%%%%%%%%%%%%%%%%%%

\newcommand{\mykeywords}{cooling flows -- galaxies: clusters:
  galaxies: individual (\inine): clusters: individual (\rxj)}
\newcommand{\mytitle}{Direct Evidence of Mechanical and Radiative AGN
  Feedback in \inine}
\newcommand{\mystitle}{Feedback in \inine}
\newcommand{\inine}{IRAS 09104+4109}
\newcommand{\irs}{I09}
\newcommand{\rxj}{RX J0913.7+4056}
\newcommand{\tsync}{\ensuremath{t_{\mathrm{sync}}}}
\newcommand{\refl}{\ensuremath{r_{\mathrm{refl}}}}
\newcommand{\fekaew}{\ensuremath{\mathrm{EW}_{\mathrm{K}\alpha}}}
\newcommand{\cltx}{\ensuremath{T_{\mathrm{cl}}}}
\newcommand{\leff}{\ensuremath{\lambda_{\mathrm{Edd}}}}
\newcommand{\lqso}{\ensuremath{\lbol^{\mathrm{QSO}}}}
\newcommand{\esens}{\ensuremath{E_{\rm{sens}}}}

%%%%%%%%%%
% Header %
%%%%%%%%%%

\documentclass[useAMS,usenatbib]{mn2e}
\usepackage{graphicx, here, common, longtable, ifthen, amsmath,
amssymb, natbib, lscape, subfigure, mathptmx, url, times, array}
\usepackage[abs]{overpic}
\usepackage[pagebackref,
  pdftitle={\mytitle},
  pdfauthor={Dr. Kenneth W. Cavagnolo},
  pdfsubject={ApJ},
  pdfkeywords={},
  pdfproducer={LaTeX with hyperref},
  pdfcreator={LaTeX with hyperref}
  pdfdisplaydoctitle=true,
  colorlinks=true,
  citecolor=blue,
  linkcolor=blue,
  urlcolor=blue]{hyperref}

\title[\mystitle]{\mytitle}

\author[Cavagnolo et al.]{K. W. Cavagnolo$^{1,2,3}$\thanks{Email:
    kcavagno@uwaterloo.ca}, M. Donahue$^{2}$, G. M. Voit$^{2}$,
  B. R. McNamara$^{1,4,5}$, and M. Sun$^{6}$\\
  $^{1}$University of Waterloo, Waterloo, ON, Canada.\\
  $^{2}$Michigan State University, East Lansing, MI, USA.\\
  $^{3}$Observatoire de la C\^ote d'Azur, Nice, PACA, France.\\
  $^{4}$Perimeter Institute for Theoretical Physics, Waterloo, ON, Canada.\\
  $^{5}$Harvard-Smithsonian Center for Astrophysics, Cambridge, MA, USA.\\
  $^{6}$University of Virginia, Charlottesville, VA, USA.}

\begin{document}

\date{Accepted (2010 Month Day). Received (2010 Month Day); in
  original form (2010 Month Day)}

\pagerange{\pageref{firstpage}--\pageref{lastpage}} \pubyear{2010}

\maketitle

\label{firstpage}

%%%%%%%%%%%%
% Abstract %
%%%%%%%%%%%%

\begin{abstract}
  Using observations from the \chandra\ X-ray Observatory, we present
  a detailed study of the ultraluminous infrared brightest cluster
  galaxy \inine\ and the surrounding intracluster medium (ICM) of the
  host galaxy cluster \rxj. The X-ray data reveals ICM cavities formed
  by $\sim 10^{44} ~\lum$ jets from the $> 4 \times 10^{46} ~\lum$
  buried quasar in \inine, and excess X-ray emission near the core
  which is well-modeled as illumination of the ICM and a galactic
  nebula by the quasar. Consistent with previous studies, we show that
  the X-ray properties of the \inine\ nucleus are dominated by
  reflected emission from a quasar embedded in a moderately
  Compton-thick medium. Our interpretation of the observations suggest
  \inine\ may be a local example of how higher-redshift galaxies
  transition from a radiatively-dominated (\ie\ quasar-mode) to a
  mechanically-dominated (\ie\ radio-mode) form of feedback.
\end{abstract}

%%%%%%%%%%%%
% Keywords %
%%%%%%%%%%%%

\begin{keywords}
  \mykeywords
\end{keywords}

%%%%%%%%%%%%%%%%%%%%%%
\section{Introduction}
\label{sec:intro}
%%%%%%%%%%%%%%%%%%%%%%

The discovery of tight correlations between various properties of a
galaxy and the mass of its centrally located supermassive black hole
(SMBH) strongly indicate that the two co-evolve
\citep[\eg][]{1995ARA&A..33..581K, magorrian, 2000ApJ...539L...9F,
  2000ApJ...539L..13G, 2001ApJ...563L..11G}. It has been suggested
that galaxy mergers and interactions, along with feedback from an
active galactic nucleus (AGN), form the foundation for SMBH-host
galaxy co-evolution \citep[\eg][]{1995MNRAS.276..663B,
  1998A&A...331L...1S, 2000MNRAS.311..576K, 2001MNRAS.324..757G}. The
most massive galaxies in the universe, \eg\ brightest cluster galaxies
(BCGs), are a unique population in the co-evolution framework as their
properties are also correlated with the galaxy cluster, or group, in
which they reside \citep[\eg][]{1984ApJ...276...38J,
  1998ApJ...502..141D}. Thus, BCGs are especially valuable for
investigating galaxy formation and evolution processes in a
cosmological context, and in this paper we discuss one rare BCG,
\inine\ (hereafter \irs), which may be providing vital clues about
these processes.

Galaxy formation models typically segregate AGN feedback into a
distinct early-time, radiatively-dominated quasar mode
\citep[\eg][]{2005Natur.435..629S, 2006ApJS..163....1H} and a
late-time, mechanically-dominated radio mode \citep[\eg][]{croton06,
  bower06}. During the quasar mode of feedback, it is believed that
quasar radiation couples to nearby gas and drives strong winds which
deprive the SMBH of additional fuel, regulating growth of black hole
mass \citep[\eg][]{2005ApJ...630..705H, 2005Natur.433..604D}. This
phase is expected to be short-lived, resulting in the expulsion of gas
from the host galaxy and temporarily quenching star formation
\citep[\eg][]{2006ApJ...642L.107N, 2008ApJ...686..219M}. Direct
evidence of radiative AGN feedback has been elusive \citep[see][for a
  review]{2005ARA&A..43..769V}, with only a handful of systems
\citep[\eg][]{2008A&A...492...81P, 2010arXiv1006.1655F} and
low-redshift ellipticals \citep{2009ApJ...690.1672S} providing the
strongest evidence to date that quasar feedback influences the host
galaxy in the ways models predict.

At later times, when quasar activity has faded and radio mode feedback
takes over, SMBH launched jets regulate the growth of galaxy mass
through prolonged and intermittent mechanical heating of a galaxy's
gaseous halo \citep{2005MNRAS.363....2K, 2006MNRAS.368....2D,
  mcnamrev}. Direct evidence of mechanical AGN feedback is seen in the
form of cavities and shocks found in the X-ray halos of many massive
galaxies \citep[\eg][]{2001ApJ...554..261C, 2007ApJ...665.1057F},
particularly in the relatively dense ICM surrounding most BCGs
\citep[\eg][]{perseus1, 2000ApJ...534L.135M}. Encouragingly, these
cavities and shocks have been shown to contain enough energy to offset
much of the halo cooling \citep{perseus2, birzan04, dunn06,
  rafferty06}. However, how cavity energy is thermalized remains
unclear \citep{2009arXiv0910.3691M, 2008ASPC..386..343D}.

Models breakdown AGN feedback into two generic modes for simplicity,
but they still form a unified schema \citep[\eg][]{sijacki07} which
predicts a continuous distribution of AGN luminosities
\citep[\eg][]{2009ApJ...698.1550H}. However, the association of the
two generic AGN modes, and whether they interact, is still poorly
understood, partly from a lack of observational constraints. This
paper presents evidence that the AGN in \irs\ is producing both
mechanical and radiative feedback, perhaps implying that it is
transitioning between the dominant mode of feedback, providing a local
example of how massive galaxies at higher redshifts evolve from
quasar-mode into radio-mode.

\irs\ is an uncommon, low-redshift ($z = 0.4418$) ultraluminous
infrared galaxy (ULIRG; $L_{\mathrm{IR}} \sim 10^{13} ~\lsol$). Unlike
most ULIRGs, \irs\ is the BCG in a rich galaxy cluster, but unlike
most BCGs, the spectral energy distribution (SED) of \irs\ is
dominated by a heavily obscured quasar: it has the optical spectrum of
a Seyfert-2 with most of the bolometric luminosity emerging longward
of 1 $\mu$m \citep{1988ApJ...328..161K, 1993ApJ...415...82H,
  1994ApJ...436L..51F, 1998ApJ...506..205E, 2000A&A...353..910F,
  2001MNRAS.321L..15I}. Another interesting feature of \irs\ is the
nearly orthogonal misalignment between the beaming directions of the
radio jets and nuclear radiation \citep[][hereafter
  H99]{1999ApJ...512..145H}. Using X-ray data from the \chandra\ X-ray
Observatory (\cxo), we present the discovery of jet induced cavities
in the X-ray halo surrounding \irs, and an unambiguous detection of
quasar radiation interacting with cold gas in the galaxy halo. The
analysis and discussion that follows focuses primarily on placing
\irs\ into the quasar-mode/radio-mode AGN feedback paradigm and
interpreting the misalignment between the radiative and mechanical
outflows in terms of this paradigm.

Data reduction is discussed in Section \ref{sec:obs}. Properties of
the ICM are analyzed in Sections \ref{sec:global} and
\ref{sec:rad}. The complex nuclear source is discussed in Section
\ref{sec:centsrc}. Analysis of the quasar irradiation of the ICM, ICM
cavities, and SMBH fueling are given in Sections \ref{sec:excess},
\ref{sec:cavs}, and \ref{sec:fuel}, respectively. Interpretation of
the results are given in Section \ref{sec:evo}, with a brief summary
in Section \ref{sec:summ}. \LCDM, for which a redshift of $z = 0.4418$
corresponds to $\approx 9.1$ Gyr for the age of the Universe, $\da
\approx 5.72$ kpc arcsec$^{-1}$, and $\dl \approx 2.45~\Gpc$.

%%%%%%%%%%%%%%%%%%%%%%
\section{Observations}
\label{sec:obs}
%%%%%%%%%%%%%%%%%%%%%%

%%%%%%%%%%%%%%%%%%
\subsection{X-ray}
\label{sec:xray}
%%%%%%%%%%%%%%%%%%

Unless stated otherwise, all X-ray spectral analysis was performed
over the energy range 0.7--7.0 keV with the \chisq\ statistic in
\xspec\ 12.4 \citep{xspec} using an absorbed, single-temperature
\mekal\ model \citep{mekal1} with gas metal abundance as a free
parameter (\citealt{ag89} Solar distribution adopted) and quoted
uncertainties of 90\% confidence. All spectral models had the Galactic
absorbing column density fixed at $\nhgal = 1.58 \times 10^{20}
~\pcmsq$ \citep{lab}. The ICM mean molecular weight and adiabatic
index are assumed to be $\mu = 0.597$ and $\gamma = 5/3$,
respectively. The \irs\ nucleus emits strong \feka\ emission which
affects spectral fitting, and for analysis of the ICM, the nucleus was
excluded using a region twice the size of the \cxo\ PSF 90\% EEF (see
Section \ref{sec:centsrc}). The X-ray analysis in this paper relates
to \cxo\ data only.

%%%%%%%%%%%%%%%%%%%%%%%%
\subsubsection{\cxo}
%%%%%%%%%%%%%%%%%%%%%%%%

A 77.2 ks \cxo\ observation of \irs\ was taken in January 2009 with
the ACIS-I instrument (\dataset [ADS/Sa.CXO#Obs/10445] {ObsID 10445}),
and a 9 ks ACIS-S observation was taken in November 1999 (\dataset
[ADS/Sa.CXO#Obs/00509] {ObsID 509}; PI Fabian). Both datasets were
reprocessed and reduced using \ciao\ and \caldb\ versions 4.2. X-ray
events were selected using \asca\ grades, and corrections for the ACIS
gain change, charge transfer inefficiency, and degraded quantum
efficiency were applied. Point sources were located and excluded using
{\textsc{wavdetect}} and verified by visual inspection. Light curves
were extracted from a source-free region of each observation to look
for flares, and time intervals with $> 20\%$ of the mean background
count rate were excluded. After flare exclusion, the final combined
exposure time is 83 ks.

For imaging analysis, the flare-free events files were reprojected to
a common tangent point and summed. The astrometry of the ObsID 509
dataset was improved using a new aspect solution created with the
\ciao\ tool {\textsc{reproject\_aspect}} and the positions of several
field sources. After astrometry correction, the positional accuracy
between both observations improved by $\approx 0.4\arcs$ and was
comparable to the resolution limit of the ACIS detectors ($\approx
0.492\arcs$ pix$^{-1}$). We refer to the final point source free,
flare-free, exposure-corrected images as the ``clean'' images. In
Figure \ref{fig:imgs} are the 0.5--10.0 keV mosaiced clean image of
\rxj, a zoom-in of the core region harboring \irs, and photons in the
energy range 4.35--4.50 keV associated with the \feka\ fluorescence
line from the nucleus (discussed in Section \ref{sec:centsrc}).

%%%%%%%%%%%%%%%%%%%%%%%%%
\subsubsection{\bepposax}
%%%%%%%%%%%%%%%%%%%%%%%%%

Conclusions reached in previous studies regarding the nature of the
obscuring material in the nucleus of \irs\ have relied on the
\bepposax\ hard X-ray detection discussed by
\citet{2000A&A...353..910F}. Here, we repeat and confirm that analysis
in order to compare our results in Section \ref{sec:centsrc} against
\citet{2000A&A...353..910F}. We retrieved and re-analyzed the
\bepposax\ data taken April 1998 (ObsCode 50273002; PI
Franceschini). After reprocessing, we measure a PDS 15--80 keV count
rate of $0.106 \pm 0.055$ ct \ps. Fitting the PDS spectrum over the
energy range 20--200 keV with an absorbed power-law having fixed
spectral index of $\Gamma = 1.7$ yielded fluxes of $f_{10 \dash 200} =
2.09^{+1.95}_{-1.95} \times 10^{-11} ~\flux$ and $f_{20 \dash 100} =
1.10^{+1.57}_{-1.63} \times 10^{-11} ~\flux$, in agreement with
\citet{2000A&A...353..910F}.

%%%%%%%%%%%%%%%%%%%%%%%%%%%%%%%%%%%%%
\subsubsection{\integral\ and \swift}
%%%%%%%%%%%%%%%%%%%%%%%%%%%%%%%%%%%%%

\irs\ was not detected in the \integral\ IBIS Extragalactic AGN Survey
\citep[][\esens\ = 20--100 keV]{2006ApJ...636L..65B}. Our re-analysis
of archival \integral\ ISGRI and JEM-X data resulted in $3\sigma$
upper limits between 10--35 keV and 20--100 keV of $f_{10 \dash 35} =
1.28 \times 10^{-12} ~\flux$ and $f_{20 \dash 100} = 5.70 \times
10^{-11} ~\flux$, higher than the \bepposax\ 20--100 keV PDS measured
flux, and consistent with the IBIS Survey non-detection for a $z =
0.44$ source. \irs\ was also not detected in the 22 month \swift-BAT
Survey \citep[][\esens\ = 15--150 keV]{2010ApJS..186..378T}. The
\swift-BAT Survey has a 14--195 keV $4.8\sigma$ detection limit of
$2.2 \times 10^{-11} ~\flux$, higher than the 14--195 keV \irs\ flux
expected based on the \bepposax\ detection. Assuming the upper limits
are representative of an $\approx 1\mydeg$ region around
\irs\ (\ie\ the full-width half maximum PDS field of view), the lack
of detected hard X-ray sources near \irs\ suggests that the PDS
detection did not originate from a brighter off-axis source, assuming
the source is not transient or was a one-time event.

%%%%%%%%%%%%%%%%%%
\subsection{Radio}
\label{sec:radio}
%%%%%%%%%%%%%%%%%%

\subsubsection{Data Reduction}

Between 1986 and 2000, \irs\ was observed with the Very Large Array
(\vla) at multiple radio frequencies and resolutions \citep[see
  also][for 1.4 and 5 GHz analysis; hereafter
  H93]{1993ApJ...415...82H}. Continuum mode observations were taken
from the \vla\ archive and reduced using version 3.0 of the Common
Astronomy Software Applications (\casa). Flagging of bad data was
performed using a combination of \casa's {\textsc{flagdata}} tool in
{\textsc{rfi}} mode and manual inspection. Radio images were generated
by Fourier transforming, cleaning, self-calibrating, and restoring
individual radio observations. The additional steps of phase and
amplitude self-calibration were included to increase the dynamic range
and sensitivity of the radio maps. All sources within the primary beam
and first side-lobe detected with fluxes $\ge 5\sigma_{\mathrm{rms}}$
were imaged to further maximize the sensitivity of the radio maps.

Resolved radio emission associated with \irs\ is detected at 1.4 GHz,
5.0 GHz, and 8.4 GHz, while a $3\sigma$ upper limit of $0.84$ mJy is
established at 14.9 GHz. Fluxes for unresolved emission at 74 MHz, 151
MHz, and 325 MHz were retrieved from VLSS \citep{vlss}, 7C Survey
\citep{1999MNRAS.306...31R}, and WENSS \citep{1997A&AS..124..259R},
respectively. No formal detection is found in VLSS, however, an
overdensity of emission at the location of \irs\ is evident. For
completeness, we measured a flux for the potential source, but
excluded the value during fitting of the radio spectrum.

The combined 1.4 GHz image is the deepest and reveals the most
extended structure, thus our discussion regarding radio morphology is
guided using this frequency. The deconvolved, integrated 1.4 GHz flux
of the continuous extended structure coincident with \irs, and having
$S_{\nu} \ga 3\sigma_{\mathrm{rms}}$, is $14.0 \pm 0.51$ mJy. A
significant spur of radio emission extending northeast from the
nucleus is detected with flux $0.21 \pm 0.07$ mJy. Radio contours were
generated beginning at 3 times the rms noise and moving up in 6
log-space steps to the peak intensity of 4.7 mJy beam$^{-1}$. These
are the contours referenced in all following discussion of the radio
source morphology and its interaction with the X-ray gas.

\subsubsection{Source Analysis}

Below we discuss properties of the radio source for the later purpose
of better understanding the AGN outflow which created cavities in the
X-ray halo around \irs\ (see Section \ref{sec:cavs}). The radio
spectrum was fitted between 151 MHz and 8.4 GHz for the full radio
source (lobes, jets, and core) with the well-known KP
\citep{1962SvA.....6..317K, pach}, JP \citep{1973A&A....26..423J}, and
CI \citep{1987MNRAS.225..335H} synchrotron models. The models
primarily vary in their assumptions regarding the electron pitch-angle
distribution and number of particle injections. The models were fitted
to the radio spectrum using the code of \citet{2005ApJ...624..656W},
which is based on the method of \citet{1991ApJ...383..554C}. The JP
model (single electron injection, randomized but isotropic pitch-angle
distribution) yields the best fit with \chisq(DOF)$ = 4.91(3)$, a
break frequency of $\nu_B = 12.9 \pm 1.0$ GHz, and a low-frequency
($\nu < 2$ GHz) spectral index of $\alpha = -1.10 \pm 0.09$. The
bolometric radio luminosity was approximated by integrating under the
JP curve between $\nu_1 = 10$ MHz and $\nu_2 =$ 10,000 MHz, giving
$\lrad = 1.09 \times 10^{42}~\lum$. The radio spectrum and best-fit
models are shown in Figure \ref{fig:radio}.

Assuming inverse-Compton (IC) scattering and synchrotron emission are
the dominant radiative mechanisms of the radio source, the time since
acceleration for an isotropic particle population is given by
\citet{2001AJ....122.1172S} as
\begin{equation}
  \tsync = 1590 \left(\frac{B^{1/2}}{B^2 + B_{\mathrm{CMB}}^2}\right)~
  \left[\nu_{\mathrm{B}} (1+z)\right]^{-1/2} ~\Myr
\end{equation}
where $B$ [\mg] is magnetic field strength, $B_{\mathrm{CMB}} =
3.25(1+z)^2$ [\mg] is a correction for IC losses to the cosmic
microwave background, $\nu_{\mathrm{B}}$ [GHz] is the radio spectrum
break frequency, and $z$ is the dimensionless source redshift. Note
that this form for \tsync\ neglects energy lost to adiabatic expansion
of the radio plasma \citep{1968ARA&A...6..321S}. We assume that $B$ is
not significantly different from the equipartition magnetic field
strength, $B_{\mathrm{eq}}$ \citep[see][for thorough discussion of the
  validity and shortcomings of this assumption]{birzan08}, which is
derived from the minimum energy density condition as
\citep{1980ARA&A..18..165M}
\begin{equation}
  B_{\mathrm{eq}} = \left[\frac{6 \pi ~c_{12}(\alpha,\nu_1, \nu_2)
      ~\lrad ~(1+k)}{V \Phi}\right]^{2/7} ~\mathrm{\mg}
\end{equation}
where $c_{12}(\alpha,\nu_1,\nu_2)$ is a dimensionless constant
\citep{pach}, \lrad\ [$\lum$] is the integrated radio luminosity from
$\nu_1$ to $\nu_2$, $k$ is the dimensionless ratio of lobe energy in
non-radiating particles to that in relativistic electrons, $V$ [\cc]
is the radio source volume, and $\Phi$ is a dimensionless radiating
population volume filling factor. Synchrotron age as a function of $k$
and $\Phi$ for the full radio source is shown in Figure
\ref{fig:radio}. For various combinations of $k$ and $\Phi$,
$B_{\mathrm{eq}} \approx 4 \dash 57 ~\mg$, with associated synchrotron
ages in the range $\approx 1 \dash 12$ Myr, typical of other BCG radio
sources \citep[\eg][]{birzan08}. Repeating the above analysis using
only radio lobe emission at 1.4 GHz, 5.0 GHz, and an 8.4 GHz upper
limit reveals $\tsync \sim 30$ Myr resulting from a significantly
steeper spectral index and lower break frequency.

%%%%%%%%%%%%%%%%%%%%%%%%%%%%%%%
\section{Global ICM Properties}
\label{sec:global}
%%%%%%%%%%%%%%%%%%%%%%%%%%%%%%%

Our analysis begins at the cluster scale with the integrated
properties of the \rxj\ ICM hosting \irs. We define the mean cluster
temperature, \cltx, as the ICM temperature within a core-excised
aperture extending to $R_{\Delta_c}$, the radius at which the average
cluster density is $\Delta_c$ times the critical density for a
spatially flat Universe. We chose $\Delta_c = 500$ and used the
relations from \cite{2002A&A...389....1A} to calculate
$R_{\Delta_c}$. \rxj\ has a luminous, cool core and complex nucleus
(see Section \ref{sec:centsrc}) which are not representative of \cltx,
thus, the convention of \citet{2007ApJ...668..772M} was followed and
emission inside $0.15 ~\rf$ was excised. Source spectra were extracted
from the region $0.15 \dash 1.0~\rf$ and background spectra were
extracted from reprocessed \caldb\ blank-sky backgrounds (see Section
\ref{sec:rad}). Because \cltx\ and $R_{\Delta_c}$ are correlated in
the adopted definitions, they were recursively determined until three
consecutive iterations produced \cltx\ values which agreed within the
68\% confidence intervals. We measure $\cltx = 7.54^{+1.76}_{-1.15}$
keV corresponding to $\rf = 1.16^{+0.27}_{-0.19}~\Mpc$. Measurements
for other $R_{\Delta_c}$ apertures are given in Table
\ref{tab:specfits}.

The cluster gas and gravitational masses were derived using the
deprojected radial electron density and temperature profiles presented
in Section \ref{sec:rad}. Because the radial temperature and density
profiles extend to $\approx 0.2 \rt$, the gravitating and gas mass
calculations below include significant extrapolation, and thus may be
lower limits, though their ratio should not be significantly
different. Electron gas density, $\nelec$, was converted to total gas
density as $n_g = 1.92 \nelec \mu \mH$ where \mH\ [g] is the mass of
hydrogen. The gas density profile was fitted with a $\beta$-model
\citep{betamodel}, and the temperature profile was fitted with the
3D-$T(r)$ model of \citet{2006ApJ...640..691V} to ensure continuity
and smoothness of the radial log-space derivatives when solving the
hydrostatic equilibrium equation. Total gas mass was calculated by
assuming spherical symmetry and integrating the best-fit $\beta$-model
out to \rt, giving $\mgas(r<\rt) = 7.99 ~(\pm 0.65) \times 10^{13}
~\msol$. The gravitating mass was derived by solving the hydrostatic
equilibrium equation using the analytic density and temperature
profiles. We calculate $\mgrav(r<\rt) = 7.22 ~(\pm 1.44) \times
10^{14} ~\msol$, giving a ratio of gas mass to gravitating mass of
$0.11 \pm 0.02$. The gas and gravitating mass errors were estimated
from 10,000 Monte Carlo realizations of the measured density and
temperature profiles and their associated uncertainties.

With the exception of the strange BCG at its heart, \rxj\ appears to
be a typical massive, relaxed galaxy cluster. The integrated X-ray
cluster properties give no indication to suspect the system has
experienced a major event which may have dramatically disrupted the
ICM. \rxj\ has a temperature, luminosity, and gas fraction consistent
with flux-limited and representative cluster samples \citep{hiflugcs2,
  2009A&A...498..361P}, implying that the peculiar nature of
\irs\ does not arise from the galaxy residing in a special or atypical
cluster. Adjusted for differences in assumed cosmology, our global
measurements agree with prior studies of
\irs\ \citep[\eg][]{2000MNRAS.315..269A}.

%%%%%%%%%%%%%%%%%%%%%%%%%%%%%%%
\section{Radial ICM Properties}
\label{sec:rad}
%%%%%%%%%%%%%%%%%%%%%%%%%%%%%%%

Now we discuss the finer global structure of \rxj\ via radial ICM
profiles. Temperature (\tx) and abundance ($Z$) profiles were created
using circular annuli centered on the cluster X-ray peak and
containing 2500 and 5000 source counts per annulus,
respectively. Spectra were grouped to 25 source counts per energy
channel. \caldb\ blank-sky backgrounds were reprocessed and
reprojected to match each observation, and then normalized for
variations of the hard-particle background using the ratio of
blank-sky and observation 9.5--12 keV count rates. Following the
method outlined in \citet{2005ApJ...628..655V}, a fixed background
component was included during spectral analysis to account for the
spatially-varying Galactic foreground \citep[see][for more
  detail]{xrayband}. A deprojected temperature profile was generated
using the \textsc{deproj} model in \xspec, but it does not
significantly differ from the projected profile. Thus, the projected
profile is used in all analysis. The temperature and abundance
profiles are shown in the top row of Figure \ref{fig:gallery}.

After masking out all X-ray substructure (see Section
\ref{sec:excess}) and the central $2\arcs$, a grouped spectrum for the
central 20 kpc was fitted with a thermal model plus a cooling flow
component. The best-fit model has a mass deposition rate of $\mdot =
206^{+87}_{-65} ~\msol$ for upper and lower temperatures of 5.43 keV
and 0.65 keV, respectively, with abundance $0.5 ~\Zsol$. The factor of
3 temperature difference implies there may be interesting soft X-ray
ionization lines, but currently there is no grating X-ray spectroscopy
of \irs.

A surface brightness (SB) profile was extracted using concentric
$1\arcs$ wide circular annuli centered on the cluster X-ray peak. From
the SB and temperature profiles, a deprojected electron density
(\nelec) profile was derived using the \citet{kriss83} technique
\citep[see][for more detail]{accept}. Errors for the density profile
were estimated from 10,000 Monte Carlo bootstrap resamplings of the SB
profile. The SB and \nelec\ profiles are shown in the second row of
Figure \ref{fig:gallery}.

Total gas pressure ($P = n \tx$), entropy ($K = \tx\nelec^{-2/3}$),
cooling time ($\tcool = 3n\tx~[2\nelec \nH \Lambda(T,Z)]^{-1}$), and
enclosed X-ray luminosity (\lx) profiles were also created, where $n =
2.3 \nH$, $\nH = \nelec/1.2$, and $\Lambda(T,Z)$ is a cooling
function. These profiles are presented in the bottom two rows of
Figure \ref{fig:gallery}. Uncertainties for each profile were
calculated by propagating the individual parameter errors and then
summing in quadrature. The cooling functions were derived from the
best-fit spectral model for each annulus of the temperature profile
and linearly interpolated onto the grid of the higher resolution
density profile. The function $K(r) = \kna +\khun (r/100
~\kpc)^{\alpha}$ was fitted to the entropy profile, giving best-fit
values of $\kna = 12.6 \pm 2.9 ~\ent$, $\khun = 139 \pm 8 ~\ent$, and
$\alpha = 1.71 \pm 0.10$.

There are no resolved discontinuities in the \tx, \nelec, or $P$
profiles to suggest the presence of a shock or cold front. Additional
2D analysis using the weighted Voronoi tessellation and contour
binning methods of \citet{wvt} and \citet{2006MNRAS.371..829S},
respectively, also did not reveal any significant temperature or
abundance substructure. The ICM structure is typical of the cool core
class of galaxy clusters \citep[\eg][]{accept, 2009MNRAS.395..764S}
and the population of $\kna < 30 ~\ent$ clusters that have radio-loud
AGN and star formation in the BCG \citep{haradent, rafferty08}. Again,
\rxj\ appears to be a perfectly ordinary cluster, sans the presence of
\irs.

We point out that the $\kna \la 30 ~\ent$ scale defines an interesting
entropy regime in which thermal electron conduction in cluster cores
is suspected of being too inefficient to suppress widespread
environmental cooling \citep{conduction}. Therefore, cooling
subsystems, like gas ram pressure stripped from cluster members or ICM
thermal instabilities, should be long-lived. Consistent with this
picture, there is an abundance of cool, gaseous substructure
surrounding \irs, and in Section \ref{sec:fuel} we discuss the
possible relation of this structure to the AGN activity.

%%%%%%%%%%%%%%%%%%%%%%%%%%%%%%%%%%
\section{\irs\ Nucleus Properties}
\label{sec:centsrc}
%%%%%%%%%%%%%%%%%%%%%%%%%%%%%%%%%%

\subsection{Spectral Analysis}

The central \irs\ X-ray source excluded from the global and radial
analyses is analyzed in detail in this section. The centroid and
extent of the source were determined using the \ciao\ tool {\tt
  wavdetect} and the \cxo\ PSF. Each was confirmed with a hardness
ratio map, shown in Figure \ref{fig:nucleus}, calculated as $HR =
f(2.0 \dash 9.0 ~\keV) / f(0.5 \dash 2.0 ~\keV)$, where $f$ is the
flux in the denoted energy band. The source is point-like, so a source
extraction region was defined using the 90\% enclosed energy fraction
(EEF) of the normalized \cxo\ PSF specific to the nuclear source
median photon energy and off-axis position. The elliptical source
region had an effective radius of $1.16\arcs$. A segmented elliptical
annulus with the same central coordinates, ellipticity, and position
angle as the source region, but having 5 times the area, was used for
the background region. The background annulus was broken into segments
to avoid the regions of excess X-ray emission discussed in Section
\ref{sec:excess}. The $HR$ map and extraction regions are shown in
Figure \ref{fig:nucleus}. Source and background spectra were created
using the \ciao\ tool {\tt psextract} for each \cxo\ observation and
grouped to have 20 counts per energy channel. Approximately 72\% of
the 2009 spectrum (hereafter SP09) is from the source, with a count
rate of $1.63 ~(\pm 0.06) \times 10^{-2}$ ct \ps\ in the 0.5--9.0 keV
band. For the 1999 spectrum (hereafter SP99), 67\% is source flux,
with a 0.5--9.0 keV count rate of $2.71 ~(\pm 0.26) \times 10^{-2}$ ct
\ps.

Previous studies have shown the nuclear spectrum is best modeled as
Compton reflection from cold matter with a strong \feka\ fluorescence
line at $E_{\rm{rest}} = 6.4$ keV
\citep[\eg][]{2001MNRAS.321L..15I}. SP99 and SP09 were fitted
separately in \xspec\ over the energy range 0.5--7.0 keV with an
absorbed \pexrav\ model \citep{pexrav} plus three Gaussians to account
for the \feka\ line and two additional line-like features around 0.8
keV and 1.3 keV \citep[see also][]{2001MNRAS.321L..15I}. The
disk-reflection geometry employed in the \pexrav\ model is not ideal
for fitting reflection from a Compton-thick torus
\citep{2009MNRAS.397.1549M}, but no other suitable \xspec\ model is
currently available. Hence, only the reflection component of the
\pexrav\ model was used and the power law had no high energy
cut-off. Fitting separate SP99 and SP09 models also allowed for source
variation in the decade between observations, however $\Gamma$ was
poorly constrained for SP99 and thus fixed at the SP09 value. Using
constraints from \citet{1997A&A...318L...1T} and
\citet{2000AJ....120..562T}, the model parameters for reflector
abundance and source inclination were fixed at $1.0 ~\Zsol$ and $i =
50\mydeg$, respectively. The best-fit model parameters are given in
Table \ref{tab:nucspec}, and the background-subtracted spectra
overplotted with the best-fit models are presented in Figure
\ref{fig:nucleus}.

\subsection{Emission Lines}

Strong Mg, Ne, S, and Si K$\alpha$ fluorescence lines at $E < 3.0$ keV
can be present in reflection spectra \citep{1991MNRAS.249..352G}, as
can Fe L-shell lines from photoionized gas
\citep{1990ApJ...362...90B}. Given that the QSO is extremely luminous
and that the spectrum is reflection-dominated, we find it likely that
the soft X-ray emission fitted by the two separate Gaussians
represents some combination of these emission lines. Using a solar
abundance thermal component in place of the two low-energy Gaussians
yielded a statistically worse fit. The model systematically
underestimated the 1--1.5 keV flux and overestimated the 2--4 keV
flux. Leaving the thermal component abundance as a free parameter
resulted in $0.1 ~\Zsol$, \ie\ the thermal component tended toward a
featureless, skewed-Gaussian.

The equivalent width of the \feka\ line (\fekaew) is a valuable
diagnostic for probing the environment of an AGN \citep[see][for a
  review]{2000PASP..112.1145F}. Our measurement of \fekaew\ agrees
with previous studies which found $\fekaew \la 1$ keV
\citep{2000A&A...353..910F, 2001MNRAS.321L..15I, 2007A&A...473...85P},
but the large uncertainties prevent us from determining if
\fekaew\ has varied since 1998. Our results are also consistent with
models and observations which show that $\fekaew \ga 0.5$ keV is
correlated with $\Gamma \ga 1.7$ and reflecting column densities
$\nhref \sim 10^{24} ~\pcmsq$ \citep{1996MNRAS.280..823M,
  1997ApJ...477..602N, 1999MNRAS.303L..11Z, 2005A&A...444..119G}.

\subsection{Obscuring Screen}

Previous studies of \irs\ suggested the \bepposax\ PDS detection
resulted primarily from transmission of hard X-rays through an
obscuring screen with a column density $\nhobs > 10^{24}
~\pcmsq$. Extrapolating our best-fit model out to 10--80 keV reveals
statistically acceptable agreement with the PDS data (see Figure
\ref{fig:resid}), indicating no transmitted component is
necessary. The 10--200 keV flux of our best-fit reflection-only model
is $f_{10 \dash 200} = 8.15^{+0.21}_{-0.19} \times 10^{-12} ~\flux$,
which is not significantly different from the 10--200 keV flux
measured with \bepposax.

If transmitted hard X-ray emission is present in the 0.5--7.0 keV
range selected for spectral analysis, the best-fit power-law component
will be artifically shallower than its intrinsic value, and the
reflected hard X-ray flux will thus be overestimated. However, we
tested this possibility using simulated spectra and found that for
$\Gamma \ge 1.7$, column densities $> 3 \times 10^{24} ~\pcmsq$ are
sufficient to suppress significant transmitted emission below our 7
keV spectral analysis cut-off, indicating the best-fit model should
not have an artificially low $\Gamma$. Consistent with this result,
addition of an absorbed, power-law component to the modeling lowered
\chisq\ (best-fit $\nhobs = 3 \times 10^{24} ~\pcmsq$ and $\Gamma =
1.7$), but with no improvement to the goodness of fit derived from
10,000 Monte Carlo simulations.

That we find no need for an additional hard X-ray component does not
contradict the well-founded conclusion that \irs\ harbors a
Compton-thick quasar. On the contrary, the measured \fekaew\ suggests
reflecting column densities of $\nhref \sim 1 \dash 5 \times 10^{24}
~\pcmsq$ \citep{1993MNRAS.263..314L, 2005A&A...444..119G,
  2010arXiv1005.3253C}. Assuming the density of material surrounding
the quasar is mostly homogeneous, \ie\ $\nhref \approx \nhobs$, our
results are consistent with the presence of a moderately Compton-thick
obscuring screen.

\subsection{Intrinsic Quasar Luminosity}

The 2--10 keV {\it{reflected}} flux of our best-fit model without
Galactic absorption is $4.24^{+0.57}_{-0.55} \times 10^{-13} ~\flux$,
corresponding to a rest-frame luminosity $L^{\rm{refl}}_{2-10} =
1.57^{+0.19}_{-0.19} \times 10^{44} ~\lum$ and bolometric (0.01--100.0
keV) luminosity $L_{\rm{bol}}^{\rm{refl}} = 4.20^{+0.49}_{-0.47}
\times 10^{45} ~\lum$. Since the reflection component is the only
directly measured quantity, the intrinsic quasar luminosity, \lqso,
can only be infered. If the reflector scattering albedo is $\eta \la
0.1$, a reasonable assumption for systems with properties like
\irs\ \citep{2009MNRAS.397.1549M}, then $\lqso \ga 4 \times 10^{46}
~\lum$. Since the reflector solid angle exposed to our line of sight
is highly uncertain, the true luminosity may be more than twice this
value, possibly $\sim 10^{47} ~\lum$. These values are consistent with
\irs's 0.3--70$\mu$m luminosity of $\approx 5 \times 10^{46} ~\lum$
\citep[][H99]{1988ApJ...328..161K} which is attributed to dust
reprocessing of quasar radiation. Hereafter, we assume that $\lqso
\approx 8 \times 10^{46} ~\lum$.

%%%%%%%%%%%%%%%%%%%%%%%%%%%%%%%%%%%
\section{Quasar Radiative Feedback}
\label{sec:excess}
%%%%%%%%%%%%%%%%%%%%%%%%%%%%%%%%%%%

\subsection{ICM X-ray Excesses}

Shown in Figure \ref{fig:resid} is a residual X-ray image of the ICM
which reveals three regions with X-ray emission in excess of the
best-fit ICM SB model (residual imaging discussed in Section
\ref{sec:cavs}). Each region is named according to its location
relative to the nucleus: northern excess (NEx), eastern excess (EEx),
and western excess (WEx). Below, we discuss the relation of each
excess with energy released by the AGN.

A source spectrum was created for each region, and a background
spectrum was extracted from regions neighboring each excess which did
not show enhanced emission in the residual image. For each region, the
ungrouped source and background spectra were differenced within
\xspec\ and fit with \xspec's modified Cash statistic
\citep{1979ApJ...228..939C}, appropriate for low-count,
background-subtracted spectra \citep[see \xspec\ Manual Appendix B
  and][]{1989ApJ...342.1207N, 2004A&A...423...75V,
  2007A&A...462..429B, 2007ApJ...666..835B,
  2009A&A...503...35E}. During spectral analysis, metal abundance was
fixed at $0.5 ~\Zsol$ because the low signal-to-noise (SN) of each
residual spectrum prohibited setting it as a free parameter. The
best-fit spectral models are given in Table \ref{tab:excess}.

Analysis of the NEx spectrum was inconclusive due to extremely low SN,
and the WEx has a residual spectrum consistent with thermal
emission. The northern radio jet terminates in the NEx region, and the
hardness ratio map (see Figure \ref{fig:resid}) shows a possible hot
spot in this same area. The NEx may result from the presence of a very
hot thermal phase in the hot spot, or be simply non-thermal
emission. Alternatively, the NEx and WEx may be a tenuous, arc-like
filament of cool gas displaced by the NW radio jet. Unfortunately the
data limits our ability to test these hypotheses spectroscopically.

The EEx spectrum was poorly fit by any combination of thermal models
because of prominent emission features at $E < 2$ keV. The EEx thermal
\feka\ complex was also poorly fit because of an obvious asymmetry
toward lower energies. To reconcile the poor fit, three Gaussians were
added to the EEx model. Comparison of goodness of fits determined from
10,000 Monte Carlo simulations of the best-fit spectra suggest the
model with the Gaussians is preferred. Below, we further discuss the
EEx exclusively.

\subsection{Quasar Irradiation of the ICM and a Nebula}

H93 and H99 suggest the AGN which produced the large-scale jets has
been reoriented within the last few Myrs, resulting in a new beaming
direction close to the line of sight and at roughly a right angle to
the previous beaming axis. Interestingly, the new AGN axis suggested
by H99 is coincident with the EEx, the radio spur northeast of the
radio core, a cone of UV ionization, an ionized optical nebula, and
highly polarized diffuse optical emission. These respective features
are outlined in Figure \ref{fig:resid}. \citet{2010MNRAS.402.1561R}
demonstrate that the quasar in H1821+643 is capable of photoionizing
gas up to 30 kpc from the nucleus, and we suspect a similar process
may be occurring in \irs.

To test this hypothesis, reflection and diffuse spectra were simulated
for the nebula and ICM coincident with the EEx using
\cloudy\ \citep{cloudy}. The nebular gas density and ionization state
were taken from \citet{2000AJ....120..562T}, while the initial ICM
temperature, density, and abundance were set at 3 keV, 0.04 \pcc, and
0.5 \Zsol, respectively. No Ca or Fe lines are detected from the
nebula coincident with the EEx, possibly as a result of metal
depletion onto dust grains \citep[\eg][]{1993ApJ...414L..17D}, while
strong Mg, Ne, and O lines are \citep{2000AJ....120..562T}. Thus, a
metal depleted, grain-rich, 12 kpc thick nebular slab was placed 15
kpc from an attenuated $\Gamma = 1.7$ power law source with power $8
\times 10^{46} ~\lum$. Likewise, a $17 ~\kpc \times 16 ~\kpc$ ICM slab
was placed 19 kpc from the same source. The quasar radiation was
attenuated using a 15 kpc column of density 0.06 \pcc, abundance 0.5
\Zsol, and temperature 3 keV. The output models were summed, folded
through the \cxo\ responses using \xspec, and normalized to the
observed EEx spectrum (shown in Figure \ref{fig:qso}).

In the energy range 0.1--10.0 keV, the nebula emission lines which
exceed thermal line emission originate from Si, Cl, O, F, K, Ne, Co,
Na, and Fe and occur as blends around redshifted 0.4, 0.6, 0.9, and
1.6 keV. The energies and strengths of these blends are in good
agreement with the EEx spectrum. Further, the \feka\ emission from the
nebula is 100 times fainter than that from the ICM, and the observed
asymmetry of the EEx \feka\ emission results from the 6.4 keV
\feka\ photoionized line of the ICM. The consistency of the
irradiation model with the observed EEx spectrum, and the coincidence
with other emission features like the radio spur, strongly suggests
that beamed quasar radiation is responsible for the nature of the
EEx. However, whether the irradiation equates to heating of the gas is
unclear.

%%%%%%%%%%%%%%%%%%%%%%%%%%%%%%%%%
\section{AGN Mechanical Feedback}
\label{sec:cavs}
%%%%%%%%%%%%%%%%%%%%%%%%%%%%%%%%%

\subsection{Discovery of ICM Cavities}

The \irs\ X-ray and radio emission morphologies are suggestive of
interaction between the halo and jets. To aid investigation of any
relationship, a residual X-ray image was created by subtracting a SB
model for the ICM from the \cxo\ clean image. The \cxo\ clean image
was binned by a factor of 2 and the SB isophotes were fitted using the
\iraf\ tool \textsc{ellipse}. The geometric parameters ellipticity
($\epsilon$), position angle ($\phi$), and centroid ($C$) were
initially free to vary, but the best-fit values for each isophote
converged to mean values of $\epsilon = 0.14$, $\phi = -76\mydeg$, and
$C$ [J2000] = (09:13:45.5; +40:56:28.4). These values were fixed in
the fitting routine to eliminate the isophotal twisting resulting from
statistical variation of the best-fit values for each radial step. The
SB model was subtracted from the clean image, resulting in the
residual image shown in Figure \ref{fig:resid}.

The faint SB decrements NW and SE of the nucleus in the clean image
are resolved into cylindrical voids in the residual image. The void
and radio jet morphologies closely trace each other, confirming they
share a common origin in the AGN outburst. Cavities are a well-known
phenomenon, but currently, \irs\ is the highest redshift object where
cavities have been directly imaged. In addition, \irs\ is thus far the
only example of a quasar-dominated system with an unambiguous cavity
detection. Using a 1994 \rosat\ HRI observation,
\citet{1995MNRAS.274L..63F} found a ``hole'' in the core of
\rxj\ which they attributed to absorption by a $> 1000 ~\msolpy$
cooling flow. Neither the cavities nor the hole are seen in a longer
1995 \rosat\ HRI observation, and when juxtaposed with the
\cxo\ residual image, the hole in the 1994 \rosat\ observation is not
associated with the cavities.

\subsection{Outburst Energetics}

The AGN outburst energetics were investigated using properties of the
cavities \citep[see][for a review]{mcnamrev}. Cavity volumes, $V$,
were calculated by approximating each void in the X-ray image with a
right circular cylinder projected onto the plane of the sky along the
cylinder radial axis. The lengthwise axis of the cylinders were
assumed to lie in a plane perpendicular to the line of sight that
passes through the central AGN. The energy in each cavity, $\ecav =
\gamma PV/(\gamma-1)$, was estimated by assuming the contents are a
relativistic plasma ($\gamma = 4/3$), and then integrating the total
gas pressure, $P$, over the surface of each cylinder. The radio source
morphology, spectrum, and age suggest the jets were recently being fed
by the central AGN. Thus, we assumed the cavities were created on a
timescale dictated by the ambient gas sound speed,
\tsonic\ \citep[see][]{birzan04}. The distance the AGN outflow has
traveled to create each cavity was set to the cylinder length, not the
midpoints, as is common. The power of each cavity is thus $\pcav =
\ecav/\tsonic$. Cavity power is often assumed to be a good estimate of
the physical quantity jet power, \pjet, but note that neither accounts
for energy which may be imparted to shocks (discussed
below). Properties of the individual cavities are listed in Table
\ref{tab:cylcavities}.

The total cavity energy and power are estimated at $\ecav = 5.11 ~(\pm
1.33) \times 10^{59}$ erg and $\pcav = 3.05 ~(\pm 1.03) \times 10^{44}
~\lum$, respectively. Compared with other systems hosting cavities
\citep[\eg][]{birzan04, dunn08}, \irs\ resides between the middle and
upper-end of the cavity power distribution. Radio power, \prad, has
been shown to be a reasonable surrogate for estimating mean jet power
\citep{birzan08}, though with considerable scatter. Thus, we checked
the \pcav\ calculation using the \citet{pjet} \pjet-\prad\ scaling
relations. The relations give $\pjet \approx 2 \dash 6 \times 10^{44}
~\lum$, in agreement with the X-ray measurements, suggesting there is
nothing unusual about the ratio of radio to jet power, \ie\ the
implied jet radiative efficiency. The AGN outburst appears to be quite
ordinary.

Of interest is how the energy deposited in the cavities compares to
the cooling rate of the host X-ray halo. The cooling radius was set at
the radius where the ICM cooling time is equal to $H_0^{-1}$ at the
redshift of \irs. We calculate $R_{\mathrm{cool}} = 128$ kpc, and
measure an unabsorbed bolometric luminosity within this radius of
$L_{\mathrm{cool}} = 1.61^{+0.25}_{-0.20} \times 10^{45} ~\lum$. If
all of the cavity energy is thermalized over $4\pi$ sr, then $\approx
20\%$ of the energy radiated away by gas within $R_{\mathrm{cool}}$ is
replaced by energy in the observed mechanical outflow. Assuming the
mean ICM cooling rate does not vary significantly on a timescale of
$\sim 1$ Gyr, this idealized scenario implies that 5 similar power AGN
outbursts will balance the cooling losses of the cluster halo. These
are not atypical results for an AGN outburst
\citep[\eg][]{rafferty06}.

\subsection{Constraints on Shock Energy}

The \pcav\ estimates neglect the influence of shocks, but the
synchrotron age and cavity age are useful in addressing this issue. If
\tsync\ is an accurate measure of the radio source age, then the age
discrepancy $\tsonic \ga 42$ Myr versus $\tsync \la 30$ Myr implies
the AGN outflow is supersonic. Were it not, the radio-loud plasma will
radiate away much of its energy and be mostly radio-quiet prior to
reaching the end of the observed jet (neglecting re-acceleration). The
implication being that during the cavity creation, some amount of jet
kinetic energy may have been imparted to shocks.

Recall that no shocks are detected in the X-ray analysis, and note
that the properties of \irs\ optical emission-line nebulae are
inconsistent with excitation due to shocks \citep{1996MNRAS.283.1003C,
  2000AJ....120..562T}. But, the nebular regions studied are $\ga 20$
kpc from the jet axis, and may not be indicative of gas dynamics close
to the outflow. Regardless, the energy in possible shocks was crudely
estimated by setting $t_{\mathrm{sonic}} = 30$ Myr and adjusting
\pcav\ by the Mach number: $\Delta \pcav = \Delta P / \Delta t$ and
$\Delta P \propto M^3$. Relative to the ICM sound speed, the velocity
needed to reach the end of the radio jet in 30 Myrs requires a Mach
number of $\approx 1.7$, which brings the outburst power up to
$\approx 3 \times 10^{45} ~\lum$ ($\ecav \approx 2 \times 10^{60}$ erg
for a 30 Myr duration). Within the formal uncertainties, the AGN
outburst power is on the order of a few times $10^{44} ~\lum$, with
the possibility of being as large as $10^{45} ~\lum$.

%%%%%%%%%%%%%%%%%%%%%%%%%%%%%%
\section{Fueling the Feedback}
\label{sec:fuel}
%%%%%%%%%%%%%%%%%%%%%%%%%%%%%%

\subsection{Powering the QSO and Jets}

An estimate of black hole mass, \mbh, is relevant to investigating AGN
activity. However, measuring \mbh\ for a BCG is non-trivial
\citep[\eg][]{2009ApJ...690..537D}, and \mbh-host galaxy relations
calibrated using lower mass galaxies may not be relevant above $10^9
~\msol$ \citep{2007ApJ...662..808L}. Acknowledging these difficulties
which may yield underestimated \mbh, we used the relations in
\citet{2002ApJ...574..740T} and \citet{2007MNRAS.379..711G} which
relate \mbh\ with a host galaxy's stellar velocity dispersion and
absolute $[B,R,K]$-band magnitudes, respectively. The \irs\ velocity
dispersion, $\sigma_s = 293 \pm 6 ~\kmps$, was determined from the
\citet{1976ApJ...204..668F} relation, and magnitudes were taken from
HyperLeda\footnote{http://leda.univ-lyon1.fr/},
SDSS\footnote{http://www.sdss.org/}, and
2MASS\footnote{http://www.ipac.caltech.edu/2mass/} and corrected using
the relations of \citet{cardelli89} and \citet{poggianti97}. The
\citet{2007MNRAS.379..711G} relations give $\mbh \approx 0.6 \dash 4.0
\times 10^9 ~\msol$ and the \citet{2002ApJ...574..740T} relation gives
$\mbh \approx 0.6 \times 10^9 ~\msol$. We adopt the weighted mean
value of $1.05 ~(\pm 0.17) \times 10^9 ~\msol$ where the uncertainty
is the error of the mean. The Eddington accretion rate, which is the
maximal inflow rate of gas not expelled by radiation pressure, for a
black hole of this mass is
\begin{equation}
  \dmedd = \frac{2.2}{\epsilon} \left(\frac{\mbh}{10^9~\msol}\right)
  \approx 23 ~\msolpy
\end{equation}
where $\epsilon = 0.1$ is the accretion disk radiative efficiency.

If mass accretion is the dominant power source for the jets
\citep{1984RvMP...56..255B}, and not, for example, SMBH spin
\citep{2002NewAR..46..247M}, then jet energy represents some fraction
of the gravitational binding energy of material accreting onto the
SMBH. Assuming the mass-energy conversion has some efficiency
$\epsilon$, the cumulative cavity energy implies a total accreted mass
of $\macc = \ecav/(\epsilon c^2)$ with a mean accretion rate of
$\dmacc = \macc/t_{\mathrm{sonic}}$. Setting $\epsilon = 0.1$, the
mechanical outflow resulted from the accretion of at least $2.86 ~(\pm
0.75) \times 10^{6} ~\msol$ of matter at a rate of $0.054 \pm 0.004
~\msolpy$. For comparison, the mass accretion rate required to power
the quasar is $\dmaccqso = \lqso/(\epsilon c^2) \approx 14 ~\msolpy
\approx 0.6 \dmedd$, dwarfing the mass accretion rate needed to power
the jets. However, if \mbh\ grows as $(1-\epsilon) \dmaccqso$, it will
double in $\approx 80$ Myr, and for \irs\ to adhere to the Magorrian
relation, $> 10^{11} ~\msol$ of stars would need to form, $> 10\%$ of
the current bulge mass. This seems unlikely, so while the jets can
subsist on a sub-Eddington accretion rate, the current period of
quasar activity cannot, and thus is likely fleeting.

\subsection{Hot and Cold Gas Accretion}

The origin of the gas driving the nuclear activity cannot be precisely
known, but how it is accreted can be constrained. If the accretion
flow feeding the SMBH is composed of the hot ICM, it can be
characterized in terms of the Bondi accretion rate, which represents
the idealized spherical accretion of gas onto a compact object at the
ambient sound speed,
\begin{equation}
  \dmbon = 0.013 ~K_{\mathrm{Bon}}^{-3/2} \left(\frac{\mbh}{10^9
    ~\msol}\right)^{2} \approx 3.2 \times 10^{-4} ~\msolpy
\end{equation}
where $K_{\mathrm{Bon}}$ [\ent] is the mean entropy of gas within the
Bondi radius and $K_{\mathrm{Bon}} = \kna$ was assumed. Only
considering the demands of the jets, the Bondi ratio of such an
accretion flow is $\dmacc/\dmbon \approx 300$, disturbingly large and
implying highly efficient hot gas accretion. The Bondi radius for
\irs\ is unresolved, and $K_{\mathrm{Bon}}$ is likely less than
\kna. But, for a Bondi ratio of at least unity, even assuming gas
close to \rbon\ is no cooler than 0.65 keV and $\mbh = 4 \times 10^9
~\msol$, $K_{\mathrm{Bon}}$ must be $\la 2 ~\ent$, 6 times lower than
\kna, on the order of galactic coronae \citep{coronae}. In terms of
entropy, $\tcool \propto K^{3/2} ~\tx^{-1}$ \citep{d06}, which
suggests the accreting material will have $\tcool \la 100$ Myr, a
factor of $\approx 3.5$ below the shortest ICM cooling time and of
order the core free-fall (ff) time. But this creates the problems that
the gas should fragment and form stars (since $\tcool \sim
t_{\rm{ff}}$), and is disconnected from cooling at larger radii,
breaking the feedback loop \citep{2006NewA...12...38S}.

If cold-mode accretion \citep{pizzolato05} dominates instead, then the
gas which becomes fuel for the AGN is distributed in the BCG halo and
migrates to the bottom of the galaxy potential in the form of cold
blobs and filaments \citep{2010arXiv1003.4181P,
  2010arXiv1007.3512P}. Indeed, radial filaments and gaseous
substructure within 30 kpc of \irs\ are seen down to the
resolution-limit of \hst\ \citep{1999Ap&SS.266..113A}. This may
indicate the presence of cooling, overdense regions similar to those
expected in the cold-mode accretion model. Though Bondi accretion
cannot be ruled out, it does not seem viable \citep[which may be true
  in general, ][]{minaspin} and the process of cold-mode accretion
appears to be more consistent with the nature of \irs.

%%%%%%%%%%%%%%%%%%%%%%%%%%%%%%%%%%%%%%%%%
\section{Evolution of the Feedback Mode?}
\label{sec:evo}
%%%%%%%%%%%%%%%%%%%%%%%%%%%%%%%%%%%%%%%%%

AGN have three primary channels for interacting with their environment
-- jets, non-relativistic winds, and radiation pressure -- and it is
suspected these channels are active at different phases in a galaxy's
evolution. The jets from \irs\ are clearly impacting the X-ray halo,
and below we argue that winds and radiation may be affecting gas
within the galaxy. It is plausible that all three AGN channels are
simultaneously active in \irs, further pointing to \irs\ as a rare
transition object in the AGN feedback paradigm. In this section we
also discuss the quasar misalignment, and suggest reasons it may be
related to evolution of the feedback mode.

\subsection{Non-Relativistic Winds}

\irs's status as a ULIRG, and its $> 10^{42} ~\lum$
\halpha\ luminosity \citep{1996MNRAS.283.1003C, 1998ApJ...506..205E},
would suggest that the galaxy hosts $> 10^{10} ~\msol$ of cold gas
\citep[\eg][]{1988ApJ...325...74S, edge01}. But, the cold H$_2$ mass
of \irs\ is $< 10^{10} ~\msol$ \citep{1998ApJ...506..205E}, there is
$< 10^8 ~\msol$ of cold dust \citep{2001MNRAS.326.1467D}, the hot dust
mass is $\sim 10^9 ~\msol$ \citep{1997A&A...318L...1T}, and no
polycyclic aromatic hydrocarbon \citep{1997A&A...318L...1T,
  2008ApJ...683..114S} or silicate absorption features
\citep{2004ApJ...613..986P} are detected in the galaxy's IR
spectrum. Though the lack of PAHs may not be salient since the huge IR
luminosity may dilute their signal. In spite of being a ULIRG BCG in a
cool core cluster, \irs\ appears to be relatively gas-poor with a low
gas-to-dust ratio.

One possible explanation for the apparent lack of cold gas in \irs\ is
that non-relativistic winds are driving outflows which have broken
apart the gas reservoirs \citep[\eg][]{2010MNRAS.401....7H}. Rapid and
extensive dust formation is expected in such winds
\citep{2002ApJ...567L.107E}, and this could in part also explain the
dust richness of \irs. Outflows of varying speeds are found in many
AGN host galaxies \citep[see][for a review]{2003ARA&A..41..117C}, and
indeed, integral field spectroscopy indicates the presence of a $>
1000 ~\kmps$ emission line outflow coincident with the \irs\ nucleus
\citep{1996MNRAS.283.1003C}. Further, the CO upper limits found by
\citet{1998ApJ...506..205E} do not exclude the existence of a
high-velocity ($> 1500 ~\kmps$) component which could result from
strong shocks driven by winds \citep[\eg][]{2010arXiv1006.1655F}. The
model of \citet{2005ApJ...619...60L} predicts a quasar driven gas
outflow will have a kinetic power of $L_{\rm{kin}} \approx 0.05
\lqso$, with a total energy $E_{\rm{kin}} \approx L_{\rm{kin}}
t_{\rm{QSO}}$ where $t_{\rm{QSO}}$ is the quasar lifetime. For
simplicity, we assume $t_{\rm{QSO}}$ is less than the age of the AGN
outburst (see Section \ref{sec:cavs}), giving $L_{\rm{kin}} \sim
10^{45} ~\lum$ and $E_{\rm{kin}} \sim 10^{60}$ erg. If this energy
exceeds the thermal energy of the \irs\ intergalactic medium, then it
is possible highly supersonic, small-scale shocks are being driven
through the galaxy, thereby coupling ambient gas to the winds
and amplifying their influence.

\subsection{Radiation}

Using a flux-limited sample of AGN selected from the \swift-BAT
catalog, \citet[][hereafter F09]{2009MNRAS.394L..89F} argue that
radiation pressure has a significant influence on dusty gas clouds in
a galaxy hosting an AGN. The F09 study compares the measured obscuring
column density of each host galaxy with its effective Eddington ratio,
$\leff = \lqso (1.38 \times 10^{38} ~\mbh)^{-1} ~\erg^{-1} ~\s
~\msol$, which accounts for the effects of dust in decreasing the
standard Eddington luminosity of a compact object
\citep[\eg][]{1993ApJ...402..441L}. F09 present a plot of
\nhobs-\leff\ which is divided into regions where obscuring clouds are
either long-lived, expelled, or appear as dust lanes. 

In the formalism of F09, \irs\ has $\leff \approx 0.55$, and in the
\nhobs-\leff\ plane, resides toward the far-side of the long-lived
region nearer the expulsion region than most other systems of similar
\nhobs. We find it reasonable to suspect that gas with a sight line to
the quasar is being accelerated away from the quasar by radiation
pressure, and is possibly being heated as well. These conclusions are,
however, at the mercy of our choice for \mbh\ and assumed \lqso. For
example, if $\mbh \ge 4 \times 10^9 ~\msol$ and $\lqso \sim 4 \times
10^{46} ~\lum$, then $\leff < 0.08$, pushing \irs\ deeper into the
long-lived region of the \nhobs-\leff\ plane, though still with a
higher \leff\ than AGN of similar \nhobs.

\subsection{Origin of Quasar Misalignment}

H93 and H99 discuss in detail that the misalignment between the
large-scale radio jets and beamed nuclear radiation may be correlated
with evolution of the radio source from a \frii\ to \fri, and that the
jet axis realignment must have transpired in less than a few Myrs. One
simple explanation for the misalignment is that there are multiple
SMBHs in the nucleus, each with its own accretion system and beaming
axis. We could be seeing emission from two separate systems: one in
quasar-mode, another in radio-mode. If multiple nuclei are a common
feature of massive galaxies, which tend to experience a higher
incidence of galaxy mergers and thus may be more likely to have more
than one central SMBH, the implication is that a single AGN may not be
wholly responsible for the evolution of the galaxy, and models should
incorporate this property.

Alternatively, the misalignment may arise from a single SMBH having
undergone a ``spin-flip'' \citep{2002Sci...297.1310M} which is
hypothesized to occur when two black holes merge and the final black
hole spin axis (believed to be synonymous with the radiation beaming
axis) is drastically reoriented
\citep[\eg][]{2010ApJ...717L..37H}. However, black hole mergers may be
lengthy (few Gyrs) and difficult processes which produce high-velocity
kicks sufficient to eject the final black hole from even a massive
galaxy \citep[\eg][]{2007ApJ...659L...5C}. In addition, the spin axes
of merging black holes may naturally align when in a relatively
gas-rich environment like \irs's \citep{2007ApJ...661L.147B}. There
are, however, 6 spheroids within a projected 80 kpc of \irs\ which may
be companion galaxy remnants \citep{1996AJ....111..649S,
  1999Ap&SS.266..113A}, so there is the possibility that one or more
mergers has taken place in the last few Myrs.

However, if there have been recent mergers, it is odd that the
\irs\ radio source is very linear and highly-structured, and that the
ICM appears to be so relaxed. Mergers have been implicated in
producing long-lived X-ray substructures such as cold fronts and
shocks \citep[see][for a review]{2007PhR...443....1M} and inducing ICM
bulk motions and turbulence that disrupt the flow of jet plasma,
resulting in deformed radio sources \citep[\eg][]{2009A&A...495..721S,
  2010arXiv1002.0395S}. Yet \irs\ possesses none of these
characteristics. It appears then, that if the misalignment arose via a
merger, it either occurred prior to jet emergence or completely avoided
the jets. The merger must also have been gentle so as not to stir the
ICM.

The spin evolution framework of \citet[][hereafter GES]{gesspin}
provides another intriguing explanation which does not necessarily
require mergers but instead relies on mass accretion. The GES model
suggests that the evolution of a black hole's spin state is
specifically correlated with a transition of the associated radio
source from a powerful \frii\ to lower-power \fri. In the GES model,
during the process of retrograde accretion induced spin-down, a black
hole should pass through a state where the spin is $\approx 0$. At
this point, if there is an asymmetric accretion flow exceeding a
\mbh-dependent critical \dmacc, the spin axis can be dramatically
reoriented on relatively short timescales (Cavagnolo et al. 2010, in
preparation), possibly giving rise to the type of beamed jet-radiation
misalignment observed in \irs.

%%%%%%%%%%%%%%%%%
\section{Summary}
\label{sec:summ}
%%%%%%%%%%%%%%%%%

In this paper we have shown that the QSO/AGN in \inine\ is interacting
with its environment through both the mechanical and radiative
feedback pathways. For the first time, we have a direct measurement of
the radiative to mechanical feedback ratio in a single system, and may
be peaking into the process of how massive galaxies at higher
redshifts evolve from quasar-mode into radio-mode. It must be noted
that, taken individually, \irs\ appears to be a normal BCG, a normal
obscured quasar, a normal radio galaxy, residing in a normal
cluster. It just so happens \irs\ is all these things at once. The
individual results of this paper are as follows:
\begin{itemize}
\item Global and radial ICM analyses of the cluster hosting \irs\ do
  not reveal any X-ray properties or substructure to suggest \rxj\ is
  anything but an unremarkable, massive, cool-core galaxy cluster.
\item The nuclear X-ray source of \irs\ is well-described as reflected
  emission from cold matter illuminated by a $> 4 \times 10^{46}
  ~\lum$ quasar obscured by moderately Compton-thick ($> 10^{24}
  ~\pcmsq$) matter.
\item Detection and modeling of an X-ray excess NE of the nucleus
  indicates beamed quasar radiation is interacting with the ICM and a
  strongly photoionized nebulae in this same region.
\item Cavities have been discovered in the X-ray halo of \irs\ which
  indicate the AGN outflow has a total mechanical power of $\approx 3
  \times 10^{44} ~\lum$, which may be $\sim 10^{45} ~\lum$ if
  significant gas shocking has occurred, and total energy output of
  $\approx 5 \times 10^{59}$ erg.
\item The mass accretion rates required to power the QSO/AGN suggest
  that the fuel feeding the SMBH was likely not accreted directly from
  the hot ICM, \ie\ via the Bondi mechanism, but rather attained
  through the accretion of cold blobs of gas, \ie\ via cold-mode
  accretion.
\item Based on the presence of a quasar, interaction of the X-ray halo
  and jets, ostensible \irs\ gas-poorness, nuclear emission line
  outflow, high effective Eddington quasar luminosity, and
  misalignment of the large-scale radio jets and beamed radiation from
  the nucleus, we suggest that \irs\ is evolving from a
  radiation-dominated mode of feedback to a kinetic-dominated
  mode. Among other possible explanations, we speculate that the
  observed properties of \irs\ may be related to the process of SMBH
  spin evolution.
\end{itemize}

%%%%%%%%%%%%%%%%%%%%%%%%%%
\section*{Acknowledgments}
%%%%%%%%%%%%%%%%%%%%%%%%%%

K.W.C. and M.D. were supported by SAO grant GO9-0143X, and M.D. and
G.M.V.  acknowledges support through NASA LTSA grant NASA
NNG-05GD82G. K.W.C. and B.R.M. thank the Natural Sciences and
Engineering Research Council of Canada for support. K.W.C. thanks
Alastair Edge and Niayesh Afshordi for helpful insight, and Guillaume
Belanger and Roland Walter for advice regarding \integral\ data
analysis.

%%%%%%%%%%%%%%%%
% Bibliography %
%%%%%%%%%%%%%%%%

\bibliographystyle{mn2e}
\bibliography{cavagnolo}

%%%%%%%%%%%%%%%%%%%%%%%
% Figures  and Tables %
%%%%%%%%%%%%%%%%%%%%%%%

\clearpage
\onecolumn
\begin{deluxetable}{lccccccccc}
\tablewidth{0pt}
\tabletypesize{\scriptsize}
\tablecaption{Summary of Global Spectral Properties\label{tab:specfits}}
\tablehead{\colhead{Region} & \colhead{$R_{in}$} & \colhead{$R_{out}$ } & \colhead{$N_{HI}$} & \colhead{$T_{X}$} & \colhead{$Z$} & \colhead{redshift} & \colhead{$\chi^2_{red.}$} & \colhead{D.O.F.} & \colhead{\% Source}\\
\colhead{ } & \colhead{kpc} & \colhead{kpc} & \colhead{$10^{20}$ cm$^{-2}$} & \colhead{keV} & \colhead{$Z_{\sun}$} & \colhead{ } & \colhead{ } & \colhead{ } & \colhead{ }\\
\colhead{{(1)}} & \colhead{{(2)}} & \colhead{{(3)}} & \colhead{{(4)}} & \colhead{{(5)}} & \colhead{{(6)}} & \colhead{{(7)}} & \colhead{{(8)}} & \colhead{{(9)}} & \colhead{{(10)}}
}
\startdata
$R_{500-core}$ & 251 & 1675 & 2.86$^{+2.46}_{-2.75}$  & 13.26$^{+6.21}_{-2.95}$  & 0.54$^{+0.28}_{-0.26}$  & 0.3605$^{+0.0235}_{-0.0162}$  & 1.10 & 368 &  19\\
$R_{1000-core}$ & 251 & 1184 & 3.13$^{+2.31}_{-2.33}$  & 11.20$^{+3.11}_{-1.97}$  & 0.51$^{+0.21}_{-0.21}$  & 0.3639$^{+0.0201}_{-0.0156}$  & 1.08 & 292 &  25\\
$R_{2500-core}$ & 251 & 749 & 1.90$^{+2.30}_{-1.90}$ & 10.66$^{+2.20}_{-1.65}$  & 0.60$^{+0.22}_{-0.19}$  & 0.3611$^{+0.0143}_{-0.0127}$  & 1.01 & 219 &  37\\
$R_{5000-core}$ & 251 & 529 & 3.22$^{+2.75}_{-2.58}$  & 8.80$^{+1.87}_{-1.31}$  & 0.46$^{+0.19}_{-0.18}$  & 0.3556$^{+0.0134}_{-0.0119}$  & 1.11 & 170 &  50\\
$R_{7500-core}$ & 251 & 432 & 2.70$^{+3.02}_{-2.70}$ & 9.85$^{+2.80}_{-1.90}$  & 0.35$^{+0.22}_{-0.21}$  & 0.3632$^{+0.0240}_{-0.0231}$  & 1.06 & 138 &  57\\
$R_{500}$ & \nodata & 1675 & 3.22$^{+1.09}_{-1.02}$  & 7.28$^{+0.50}_{-0.45}$  & 0.41$^{+0.06}_{-0.06}$  & 0.3563$^{+0.0053}_{-0.0044}$  & 0.95 & 535 &  39\\
$R_{1000}$ & \nodata & 1184 & 3.25$^{+1.00}_{-0.93}$  & 7.05$^{+0.40}_{-0.39}$  & 0.40$^{+0.05}_{-0.05}$  & 0.3573$^{+0.0043}_{-0.0040}$  & 0.91 & 488 &  51\\
$R_{2500}$ & \nodata & 749 & 2.97$^{+0.87}_{-0.97}$  & 6.88$^{+0.38}_{-0.33}$  & 0.41$^{+0.05}_{-0.05}$  & 0.3558$^{+0.0026}_{-0.0046}$  & 0.88 & 442 &  70\\
$R_{5000}$ & \nodata & 529 & 3.10$^{+0.90}_{-0.96}$  & 6.66$^{+0.34}_{-0.31}$  & 0.40$^{+0.05}_{-0.05}$  & 0.3560$^{+0.0028}_{-0.0047}$  & 0.85 & 418 &  81\\
$R_{7500}$ & \nodata & 432 & 3.17$^{+0.97}_{-1.04}$  & 6.61$^{+0.38}_{-0.31}$  & 0.39$^{+0.05}_{-0.05}$  & 0.3550$^{+0.0037}_{-0.0047}$  & 0.86 & 410 &  85\\
\hline
$R_{500-core}$ & 251 & 1675 & 3.18$^{+2.46}_{-2.61}$  & 12.63$^{+5.19}_{-2.54}$  & 0.53$^{+0.27}_{-0.26}$  & 0.3540 & 1.10 & 369 &  19\\
$R_{1000-core}$ & 251 & 1184 & 3.40$^{+2.19}_{-2.22}$  & 10.79$^{+2.69}_{-1.70}$  & 0.50$^{+0.20}_{-0.21}$  & 0.3540 & 1.08 & 293 &  25\\
$R_{2500-core}$ & 251 & 749 & 2.19$^{+2.20}_{-2.15}$  & 10.33$^{+2.08}_{-1.50}$  & 0.60$^{+0.21}_{-0.20}$  & 0.3540 & 1.01 & 220 &  37\\
$R_{5000-core}$ & 251 & 529 & 3.25$^{+2.63}_{-2.49}$  & 8.76$^{+1.73}_{-1.30}$  & 0.46$^{+0.18}_{-0.17}$  & 0.3540 & 1.10 & 171 &  50\\
$R_{7500-core}$ & 251 & 432 & 2.99$^{+2.94}_{-2.79}$  & 9.56$^{+2.67}_{-1.74}$  & 0.34$^{+0.21}_{-0.21}$  & 0.3540 & 1.05 & 139 &  57\\
$R_{500}$ & \nodata & 1675 & 3.25$^{+1.01}_{-1.01}$  & 7.25$^{+0.46}_{-0.42}$  & 0.41$^{+0.06}_{-0.06}$  & 0.3540 & 0.95 & 536 &  39\\
$R_{1000}$ & \nodata & 1184 & 3.31$^{+0.99}_{-0.98}$  & 7.00$^{+0.40}_{-0.37}$  & 0.40$^{+0.05}_{-0.05}$  & 0.3540 & 0.91 & 489 &  51\\
$R_{2500}$ & \nodata & 749 & 2.95$^{+0.97}_{-0.95}$  & 6.87$^{+0.37}_{-0.33}$  & 0.41$^{+0.06}_{-0.05}$  & 0.3540 & 0.88 & 443 &  70\\
$R_{5000}$ & \nodata & 529 & 3.10$^{+0.98}_{-0.95}$  & 6.65$^{+0.34}_{-0.32}$  & 0.40$^{+0.05}_{-0.05}$  & 0.3540 & 0.85 & 419 &  81\\
$R_{7500}$ & \nodata & 432 & 3.17$^{+0.99}_{-0.97}$  & 6.60$^{+0.34}_{-0.31}$  & 0.39$^{+0.05}_{-0.05}$  & 0.3540 & 0.86 & 411 &  85\\
\hline
$R_{500-core}$ & 251 & 1675 & 2.22 & 14.16$^{+3.68}_{-2.43}$  & 0.54$^{+0.30}_{-0.28}$  & 0.3626$^{+0.0215}_{-0.0219}$  & 1.09 & 369 &  19\\
$R_{1000-core}$ & 251 & 1184 & 2.22 & 11.91$^{+2.15}_{-1.52}$  & 0.52$^{+0.22}_{-0.22}$  & 0.3658$^{+0.0226}_{-0.0160}$  & 1.08 & 293 &  25\\
$R_{2500-core}$ & 251 & 749 & 2.22 & 10.46$^{+1.49}_{-1.15}$  & 0.60$^{+0.20}_{-0.19}$  & 0.3613$^{+0.0144}_{-0.0132}$  & 1.01 & 220 &  37\\
$R_{5000-core}$ & 251 & 529 & 2.22 & 9.26$^{+1.27}_{-1.02}$  & 0.46$^{+0.19}_{-0.18}$  & 0.3560$^{+0.0140}_{-0.0061}$  & 1.11 & 171 &  50\\
$R_{7500-core}$ & 251 & 432 & 2.22 & 10.15$^{+1.81}_{-1.36}$  & 0.35$^{+0.22}_{-0.20}$  & 0.3647$^{+0.0123}_{-0.0231}$  & 1.05 & 139 &  57\\
$R_{500}$ & \nodata & 1675 & 2.22 & 7.63$^{+0.33}_{-0.30}$  & 0.41$^{+0.06}_{-0.06}$  & 0.3567$^{+0.0048}_{-0.0034}$  & 0.96 & 536 &  39\\
$R_{1000}$ & \nodata & 1184 & 2.22 & 7.38$^{+0.29}_{-0.29}$  & 0.40$^{+0.06}_{-0.05}$  & 0.3586$^{+0.0027}_{-0.0067}$  & 0.91 & 489 &  51\\
$R_{2500}$ & \nodata & 749 & 2.22 & 7.09$^{+0.26}_{-0.23}$  & 0.41$^{+0.06}_{-0.05}$  & 0.3556$^{+0.0031}_{-0.0045}$  & 0.88 & 443 &  70\\
$R_{5000}$ & \nodata & 529 & 2.22 & 6.90$^{+0.23}_{-0.23}$  & 0.40$^{+0.05}_{-0.05}$  & 0.3560$^{+0.0027}_{-0.0048}$  & 0.85 & 419 &  81\\
$R_{7500}$ & \nodata & 432 & 2.22 & 6.86$^{+0.24}_{-0.22}$  & 0.39$^{+0.05}_{-0.05}$  & 0.3558$^{+0.0020}_{-0.0046}$  & 0.87 & 411 &  85\\
\hline
$R_{500-core}$ & 251 & 1675 & 2.22 & 13.80$^{+3.08}_{-2.21}$  & 0.53$^{+0.28}_{-0.28}$  & 0.3540 & 1.09 & 370 &  19\\
$R_{1000-core}$ & 251 & 1184 & 2.22 & 11.64$^{+1.09}_{-1.44}$  & 0.50$^{+0.22}_{-0.22}$  & 0.3540 & 1.08 & 294 &  25\\
$R_{2500-core}$ & 251 & 749 & 2.22 & 10.31$^{+1.38}_{-1.09}$  & 0.60$^{+0.20}_{-0.20}$  & 0.3540 & 1.01 & 221 &  37\\
$R_{5000-core}$ & 251 & 529 & 2.22 & 9.22$^{+1.21}_{-0.97}$  & 0.47$^{+0.19}_{-0.18}$  & 0.3540 & 1.10 & 172 &  50\\
$R_{7500-core}$ & 251 & 432 & 2.22 & 10.01$^{+1.77}_{-1.33}$  & 0.34$^{+0.21}_{-0.22}$  & 0.3540 & 1.05 & 140 &  57\\
$R_{500}$ & \nodata & 1675 & 2.22 & 7.60$^{+0.32}_{-0.30}$  & 0.41$^{+0.06}_{-0.06}$  & 0.3540 & 0.96 & 537 &  39\\
$R_{1000}$ & \nodata & 1184 & 2.22 & 7.34$^{+0.28}_{-0.26}$  & 0.40$^{+0.06}_{-0.05}$  & 0.3540 & 0.92 & 490 &  51\\
$R_{2500}$ & \nodata & 749 & 2.22 & 7.08$^{+0.25}_{-0.23}$  & 0.42$^{+0.05}_{-0.06}$  & 0.3540 & 0.88 & 444 &  70\\
$R_{5000}$ & \nodata & 529 & 2.22 & 6.88$^{+0.23}_{-0.22}$  & 0.40$^{+0.05}_{-0.05}$  & 0.3540 & 0.85 & 420 &  81\\
$R_{7500}$ & \nodata & 432 & 2.22 & 6.85$^{+0.24}_{-0.22}$  & 0.39$^{+0.05}_{-0.05}$  & 0.3540 & 0.87 & 412 &  85\\
\hline
NE-Arm & \nodata & \nodata & 2.22 & 5.00$^{+1.08}_{-0.78}$  & 1.23$^{+1.28}_{-0.79}$  & 0.3540 & 0.19 & 300 &  98\\
NW-Arm & \nodata & \nodata & 2.22 & 5.73$^{+1.28}_{-0.98}$  & 0.54$^{+0.55}_{-0.48}$  & 0.3540 & 0.26 & 184 &  99\\
SE-Arm & \nodata & \nodata & 2.22 & 5.49$^{+1.24}_{-0.80}$  & 0.36$^{+0.41}_{-0.36}$  & 0.3540 & 0.16 & 268 &  99\\
SW-Arm & \nodata & \nodata & 2.22 & 5.99$^{+1.41}_{-0.98}$  & 1.04$^{+0.73}_{-0.49}$  & 0.3540 & 0.14 & 338 &  99\\
\hline
\enddata
\tablecomments{Col. (1) Name of region used for spectral extraction; col. (2) inner radius of extraction region; col. (3) outer radius of extraction region; col. (4) absorbing, Galactic neutral hydrogen column density; col. (5) best-fit temperature; col. (6) best-fit metallicity; col. (7) best-fit redshift; col. (8) reduced \chisq\ for best-fit model; col. (9) degrees of freedom for best-fit model; col. (10) percentage of emission attributable to source.}
\end{deluxetable}

\begin{table*}
  \begin{center}
    \caption{\sc Summary of Nuclear Source Spectral Fits.\label{tab:nucspec}}
    \begin{tabular}{lccc}
      \hline
      \hline
      Component & Parameter & SP09 & SP99\\
      (1) & (2) & (3) & (4)\\
      \hline
      \pexrav\  & $\Gamma$              & $1.71^{+0.23}_{-0.65}$                & fixed to SP09\\
      -         & $\eta_{\mathrm{P}}$   & $8.07^{+0.64}_{-0.62}\times10^{-4}$   & $8.46^{+2.08}_{-2.12} \times 10^{-4}$\\
      \gauss\ 1 & $E_{\mathrm{G}}$      & $0.73^{+0.05}_{-0.24}$                & $0.61^{+0.10}_{-0.05}$\\
      -         & $\sigma_{\mathrm{G}}$ & $85^{+197}_{-53}$                     & $97^{+150}_{-97}$\\
      -         & $\eta_{\mathrm{G}}$   & $8.14^{+3.74}_{-5.82} \times 10^{-6}$ & $1.65^{+1.52}_{-1.00} \times 10^{-5}$\\
      \gauss\ 2 & $E_{\mathrm{G}}$      & $1.16^{+0.19}_{-0.33}$                & $0.90^{+0.17}_{-0.90}$\\
      -         & $\sigma_{\mathrm{G}}$ & $383^{+610}_{-166}$                   & $506^{+314}_{-262}$\\
      -         & $\eta_{\mathrm{G}}$   & $1.03^{+3.22}_{-0.48} \times 10^{-5}$ & $1.48^{+2.68}_{-1.16} \times 10^{-5}$\\
      \gauss\ 3 & $E_{\mathrm{G}}$      & $4.45^{+0.04}_{-0.04}$                & $4.46^{+0.04}_{-0.07}$\\
      -         & $\sigma_{\mathrm{G}}$ & $45^{+60}_{-45}$                      & $31^{+94}_{-31}$\\
      -         & $\eta_{\mathrm{G}}$   & $2.67^{+0.91}_{-0.86} \times 10^{-6}$ & $6.45^{+4.17}_{-3.69} \times 10^{-6}$\\
      -         & EW$^{\mathrm{corr}}_{\mathrm{K}\alpha}$ & $531^{+211}_{-218}$ & $1210^{+720}_{-710}$\\
      Statistic & \chisq                & 79.0                                  & 7.9\\
      -         & DOF                   & 74                                    & 15\\
      \hline
    \end{tabular}
    \begin{quote}
      \feka\ equivalent widths have been corrected for redshift. Units for
      parameters: $\Gamma$ is dimensionless, $\eta_{\mathrm{P}}$ is in ph
      keV$^{-1}$ cm$^{-2}$ s$^{-1}$, $E_{\mathrm{G}}$ are in keV,
      $\sigma_{\mathrm{G}}$ are in eV, $\eta_{\mathrm{G}}$ are in ph
      cm$^{-2}$ s$^{-1}$, EW$_{\mathrm{corr}}$ are in eV. Col. (1)
      \xspec\ model name; Col. (2) Model parameters; Col. (3) Values for
      2009 \cxo\ spectrum; Col. (4) Values for 1999 \cxo\ spectrum.
    \end{quote}
  \end{center}
\end{table*}

\begin{table*}
  \begin{center}
    \caption{\sc Summary of X-ray Excesses Spectral Fits.\label{tab:excess}}
    \begin{tabular}{lccccccc}
      \hline
      \hline
      Region & \tx & $\eta$ & $E_{\mathrm{G}}$ & $\sigma_{\mathrm{G}}$ & $\eta_{\mathrm{G}}$ & Cash & DOF\\
      - & keV & $10^{-5}$ cm$^{-5}$ & keV & keV & $10^{-6}~\pcmsq~\ps$ & - & -\\
      (1) & (2) & (3) & (4) & (5) & (6) & (7) & (8)\\
      \hline
      Eastern excess     & 3.03$^{+1.19}_{-0.74}$ & $5.80^{+1.07}_{-0.97}$ & -                  & -                    & -                & 524 & 430\\
      Eastern excess     & 3.68$^{+3.34}_{-1.58}$ & $2.73^{+0.98}_{-0.94}$ & [0.89, 1.42, 4.23] & [0.04, 0.16, 3.6E-4] & [1.2, 2.0, 0.16] & 384 & 430\\
      Eastern excess bgd & 3.92$^{+0.35}_{-0.31}$ & $39.9^{+0.18}_{-0.17}$ & -                  & -                    & -                & 471 & 430\\
      Lower-NW excess & 2.55$^{+2.61}_{-0.98}$ & $0.66^{+0.11}_{-0.07}$ & -                  & -                    & -                & 387 & 430\\
      \hline
    \end{tabular}
    \begin{quote}
      Metal abundance was fixed at $0.5 ~\Zsol$ for all fits.
      Col. (1) Extraction region; Col. (2) Thermal gas temperature;
      Col. (3) Model normalization; Col. (4) Gaussian central
      energies; Col. (5) Gaussian dispersions; Col. (6) Gaussian
      normalizations; Col. (7) Modified Cash statistic; Col. (8)
      Degrees of freedom.
    \end{quote}
  \end{center}
\end{table*}

\begin{table*}
  \begin{center}
    \caption{\sc Summary of Cavity Properties.\label{tab:cylcavities}}
    \begin{tabular}{lcccccc}
      \hline
      \hline
      Cavity & $r$ & $l$ & \tsonic & $pV$ & \ecav & \pcav\\
      -- & kpc & kpc & $10^6$ yr & $10^{58}$ ergs & $10^{59}$ ergs & $10^{44}$ ergs s$^{-1}$\\
      (1) & (2) & (3) & (4) & (5) & (6) & (7)\\
      \hline
      Northwest & 6.40 & 58.3 & ${50.5 \pm 7.6}$ & ${5.78 \pm 1.07}$ & ${2.31 \pm 0.43}$ & ${1.45 \pm 0.35}$\\
      Southeast & 6.81 & 64.0 & ${55.4 \pm 8.4}$ & ${6.99 \pm 1.29}$ & ${2.80 \pm 0.52}$ & ${1.60 \pm 0.38}$\\
      \hline
    \end{tabular}
    \begin{quote}
      Col. (1) Cavity location; Col. (2) Radius of excavated cylinder;
      Col. (3) Length of excavated cylinder; Col. (4) Sound speed age;
      Col. (5) $pV$ work; Col. (6) Cavity energy; Col. (7) Cavity power.
    \end{quote}
  \end{center}
\end{table*}

\clearpage
\begin{figure}
  \begin{center}
    \begin{minipage}{\linewidth}
      \includegraphics*[width=\textwidth, trim=0mm 0mm 0mm 0mm, clip]{rbs797.ps}
    \end{minipage}
    \caption{Fluxed, unsmoothed 0.7--2.0 keV clean image of \rbs\ in
      units of ph \pcmsq\ \ps\ pix$^{-1}$. Image is $\approx 250$ kpc
      on a side and coordinates are J2000 epoch. Black contours in the
      nucleus are 2.5--9.0 keV X-ray emission of the nuclear point
      source; the outer contour approximately traces the 90\% enclosed
      energy fraction (EEF) of the \cxo\ point spread function. The
      dashed green ellipse is centered on the nuclear point source,
      encloses both cavities, and was drawn by-eye to pass through the
      X-ray ridge/rims.}
    \label{fig:img}
  \end{center}
\end{figure}

\begin{figure}
  \begin{center}
    \begin{minipage}{0.495\linewidth}
      \includegraphics*[width=\textwidth, trim=0mm 0mm 0mm 0mm, clip]{325.ps}
    \end{minipage}
   \begin{minipage}{0.495\linewidth}
      \includegraphics*[width=\textwidth, trim=0mm 0mm 0mm 0mm, clip]{8.4.ps}
   \end{minipage}
   \begin{minipage}{0.495\linewidth}
      \includegraphics*[width=\textwidth, trim=0mm 0mm 0mm 0mm, clip]{1.4.ps}
    \end{minipage}
    \begin{minipage}{0.495\linewidth}
      \includegraphics*[width=\textwidth, trim=0mm 0mm 0mm 0mm, clip]{4.8.ps}
    \end{minipage}
     \caption{Radio images of \rbs\ overlaid with black contours
       tracing ICM X-ray emission. Images are in mJy beam$^{-1}$ with
       intensity beginning at $3\sigma_{\rm{rms}}$ and ending at the
       peak flux, and are arranged by decreasing size of the
       significant, projected radio structure. X-ray contours are from
       $2.3 \times 10^{-6}$ to $1.3 \times 10^{-7}$ ph
       \pcmsq\ \ps\ pix$^{-1}$ in 12 square-root steps. {\it{Clockwise
           from top left}}: 325 MHz \vla\ A-array, 8.4 GHz
       \vla\ D-array, 4.8 GHz \vla\ A-array, and 1.4 GHz
       \vla\ A-array.}
    \label{fig:composite}
  \end{center}
\end{figure}

\begin{figure}
  \begin{center}
    \begin{minipage}{0.495\linewidth}
      \includegraphics*[width=\textwidth, trim=0mm 0mm 0mm 0mm, clip]{sub_inner.ps}
    \end{minipage}
    \begin{minipage}{0.495\linewidth}
      \includegraphics*[width=\textwidth, trim=0mm 0mm 0mm 0mm, clip]{sub_outer.ps}
    \end{minipage}
    \caption{Red text point-out regions of interest discussed in
      Section \ref{sec:cavities}. {\it{Left:}} Residual 0.3-10.0 keV
      X-ray image smoothed with $1\arcs$ Gaussian. Yellow contours are
      1.4 GHz emission (\vla\ A-array), orange contours are 4.8 GHz
      emission (\vla\ A-array), orange vector is 4.8 GHz jet axis, and
      red ellipses outline definite cavities. {\it{Bottom:}} Residual
      0.3-10.0 keV X-ray image smoothed with $3\arcs$ Gaussian. Green
      contours are 325 MHz emission (\vla\ A-array), blue contours are
      8.4 GHz emission (\vla\ D-array), and orange vector is 4.8 GHz
      jet axis.}
    \label{fig:subxray}
  \end{center}
\end{figure}

\begin{figure}
  \begin{center}
    \begin{minipage}{\linewidth}
      \includegraphics*[width=\textwidth]{r797_nhfro.eps}
      \caption{Gallery of radial ICM profiles. Vertical black dashed
        lines mark the approximate end-points of both
        cavities. Horizontal dashed line on cooling time profile marks
        age of the Universe at redshift of \rbs. For X-ray luminosity
        profile, dashed line marks \lcool, and dashed-dotted line
        marks \pcav.}
      \label{fig:gallery}
    \end{minipage}
  \end{center}
\end{figure}

\begin{figure}
  \begin{center}
    \begin{minipage}{\linewidth}
      \setlength\fboxsep{0pt}
      \setlength\fboxrule{0.5pt}
      \fbox{\includegraphics*[width=\textwidth]{cav_config.eps}}
    \end{minipage}
    \caption{Cartoon of possible cavity configurations. Arrows denote
      direction of AGN outflow, ellipses outline cavities, \rlos\ is
      line-of-sight cavity depth, and $z$ is the height of a cavity's
      center above the plane of the sky. {\it{Left:}} Cavities which
      are symmetric about the plane of the sky, have $z=0$, and are
      inflating perpendicular to the line-of-sight. {\it{Right:}}
      Cavities which are larger than left panel, have non-zero $z$,
      and are inflating along an axis close to our line-of-sight.}
    \label{fig:config}
  \end{center}
\end{figure}

\begin{figure}
  \begin{center}
    \begin{minipage}{0.495\linewidth}
      \includegraphics*[width=\textwidth, trim=25mm 0mm 40mm 10mm, clip]{edec.eps}
    \end{minipage}
    \begin{minipage}{0.495\linewidth}
      \includegraphics*[width=\textwidth, trim=25mm 0mm 40mm 10mm, clip]{wdec.eps}
    \end{minipage}
    \caption{Surface brightness decrement as a function of height
      above the plane of the sky for a variety of cavity radii. Each
      curve is labeled with the corresponding \rlos. The curves
      furthest to the left are for the minimum \rlos\ needed to
      reproduce $y_{\rm{min}}$, \ie\ the case of $z = 0$, and the
      horizontal dashed line denotes the minimum decrement for each
      cavity. {\it{Left}} Cavity E1; {\it{Right}} Cavity W1.}
    \label{fig:decs}
  \end{center}
\end{figure}


\begin{figure}
  \begin{center}
    \begin{minipage}{\linewidth}
      \includegraphics*[width=\textwidth, trim=15mm 5mm 5mm 10mm, clip]{pannorm.eps}
      \caption{Histograms of normalized surface brightness variation
        in wedges of a $2.5\arcs$ wide annulus centered on the X-ray
        peak and passing through the cavity midpoints. {\it{Left:}}
        $36\mydeg$ wedges; {\it{Middle:}} $14.4\mydeg$ wedges;
        {\it{Right:}} $7.2\mydeg$ wedges. The depth of the cavities
        and prominence of the rims can be clearly seen in this plot.}
      \label{fig:pannorm}
    \end{minipage}
  \end{center}
\end{figure}

\begin{figure}
  \begin{center}
    \begin{minipage}{0.5\linewidth}
      \includegraphics*[width=\textwidth, angle=-90]{nucspec.ps}
    \end{minipage}
    \caption{X-ray spectrum of nuclear point source. Black denotes
      year 2000 \cxo\ data (points) and best-fit model (line), and red
      denotes year 2007 \cxo\ data (points) and best-fit model (line).
      The residuals of the fit for both datasets are given below.}
    \label{fig:nucspec}
  \end{center}
\end{figure}

\begin{figure}
  \begin{center}
    \begin{minipage}{\linewidth}
      \includegraphics*[width=\textwidth, trim=10mm 5mm 10mm 10mm, clip]{radiofit.eps}
    \end{minipage}
    \caption{Best-fit continuous injection (CI) synchrotron model to
      the nuclear 1.4 GHz, 4.8 GHz, and 7.0 keV X-ray emission. The
      two triangles are \galex\ UV fluxes showing the emission is
      boosted above the power-law attributable to the nucleus.}
    \label{fig:sync}
    \end{center}
\end{figure}

\begin{figure}
  \begin{center}
    \begin{minipage}{\linewidth}
      \includegraphics*[width=\textwidth, trim=0mm 0mm 0mm 0mm, clip]{rbs797_opt.ps}
    \end{minipage}
    \caption{\hst\ \myi+\myv\ image of the \rbs\ BCG with units e$^-$
      s$^{-1}$. Green, dashed contour is the \cxo\ 90\% EEF. Emission
      features discussed in the text are labeled.}
    \label{fig:hst}
  \end{center}
\end{figure}

\begin{figure}
  \begin{center}
    \begin{minipage}{0.495\linewidth}
      \includegraphics*[width=\textwidth, trim=0mm 0mm 0mm 0mm, clip]{suboptcolor.ps}
    \end{minipage}
    \begin{minipage}{0.495\linewidth}
      \includegraphics*[width=\textwidth, trim=0mm 0mm 0mm 0mm, clip]{suboptrad.ps}
    \end{minipage}
    \caption{{\it{Left:}} Residual \hst\ \myv\ image. White regions
      (numbered 1--8) are areas with greatest color difference with
      \rbs\ halo. {\it{Right:}} Residual \hst\ \myi\ image. Green
      contours are 4.8 GHz radio emission down to
      $1\sigma_{\rm{rms}}$, white dashed circle has radius $2\arcs$,
      edge of ACS ghost is show in yellow, and southern whiskers are
      numbered 9--11 with corresponding white lines.}
    \label{fig:subopt}
  \end{center}
\end{figure}


%%%%%%%%%%%%%%%%%%%%
% End the document %
%%%%%%%%%%%%%%%%%%%%

\label{lastpage}
\end{document}

%% The data was reduced and analyzed with \saxdas\ version 2.3.1 using
%% the calibration data, cookbook, and epoch appropriate response
%% functions available from
%% HEASARC\footnote{http://heasarc.nasa.gov/docs/sax}. Data from the
%% PDS instrument (\esens\ = 15--300 keV) was accumulated, screened
%% for good time intervals, and then a light curve was extracted. No
%% significant variations of the light curve were detected. A PDS
%% total spectrum was extracted from the on-axis data, and background
%% subtraction was performed using the variable rise time
%% threshold. We

%% Gas consumption by star formation also cannot be neglected: the
%% \irs\ near-UV flux measured by \galex\ is $\approx 47 ~\mu$Jy,
%% which the relations of \citet{kennicutt2} predict may be associated
%% with $\approx 45 ~\msolpy$ of star formation. However, the
%% \galex\ emission is unresolved, and scattered UV emission may
%% account for much of this flux (H99).

%% Taking in the extensive literature and observations of \irs, we posit
%% the following evolutionary scenario for the galaxy:
%% \begin{enumerate}
%% \item Through an unknown set of circumstances, \irs\ becomes a typical
%%   radio-quiet quasar.
%% \item Encased in a Compton-thick obscuring shell, radiation pressure
%%   thins the shell, eventually allowing a pair of jets to punch through
%%   and escape the nucleus.
%% \item At the same time, the massive accretion flow driving the quasar
%%   spins down the SMBH, and asymmetry in the flow eventually alters the
%%   SMBH spin axis.
%% \item After the accretion disk has stabilized, the beaming of
%%   radiation is reestablished, and is later followed by the birth of
%%   new jets, temporarily giving the appearance of a misalignment
%%   between the radiation and mechanical outflows.
%% \item As the mass in the accretion flow is exhausted, the quasar fades
%%   and a stereotypical BCG radio galaxy emerges.
%% \end{enumerate}
%% It should not be ignored that \irs\ is a type-2 quasar, which has long
%% been conjectured as an early stage in the AGN lifecycle
%% \citep[\eg][]{1999MNRAS.308L..39F}.

%% \subsection{Gas Uplift?}

%% One alternative explanation for the EEx is that it is low entropy gas
%% uplifted from deeper within the core along the new AGN beaming
%% axis. Scattered UV emission 32 kpc from the core places a minimum
%% lifetime of 73 kyr for the new beaming direction (H93), and, assuming
%% saturated heat flux across the EEx surface, the evaporation time is
%% exceedingly short, $< 1$ Myr. These short timescales suggest gas would
%% need to be transported to the present location at $> 20$ times the
%% ambient sound speed, or $v \sim 0.06 \dash 0.1c$, well below typical
%% jet bulk flow velocities. Using the jet power-radio power scaling
%% relations in \citet{pjet}, the 1.4 GHz luminosity of the radio spur
%% ($\approx 3 \times 10^{39} ~\lum$), suggests that a 1 Myr old jet may
%% possess $\sim 10^{57}$ erg of kinetic energy. This is sufficient to
%% lift $\sim 10^{9} ~\msol$ to a distance of 19 kpc, so it appears
%% uplift is a feasible process, particularly if the jet has
%% magnetically-shielded the gas and conduction is staved-off.
