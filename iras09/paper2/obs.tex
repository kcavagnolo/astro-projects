\begin{deluxetable}{ccccccc}
  \tablecaption{Summary of Radio Observations.\label{tab:obs}}
  \tablecolumns{7}
  \tabletypesize{}
  \tablewidth{0pt}
  \tablehead{
    \colhead{Array (Config.)} & \colhead{Freq.} & \colhead{Bandwidth} & \colhead{Time} & \colhead{\srms (Theo.)} & \colhead{Peak}  & \colhead{Beam}\\
    \colhead{-}               & \colhead{GHz}   & \colhead{MHz}       & \colhead{hr}   & \colhead{\murms}        & \colhead{\mrms} & \colhead{-}\\
    \colhead{(1)}             & \colhead{(2)}   & \colhead{(3)}       & \colhead{(4)}  & \colhead{(5)}           & \colhead{(6)}   & \colhead{(7)}}
  \startdata
  \vlba\ (-) & 1.4  & 16 & 7.1 & 48 (49)  & 00 & 10.3$\times$7.9 mas\\
  \vlba\ (-) & 5.0  & 64 & 0.0 & 00 (24)  & 00 & 0.00$\times$0.00 mas\\
  \vla\ (A)  & 1.5  & 86 & 4.1 & 28 (23)  & 00 & 1.26$\arcs\times$1.15$\arcs$\\
  \vla\ (A)  & 5.0  & 86 & 2.0 & 31 (26)  & 00 & 0.41$\arcs\times$0.34$\arcs$\\
  \vla\ (A)  & 8.3  & 86 & 0.5 & 00 (41)  & 00 & 0.34$\arcs\times$0.21$\arcs$\\
  \vla\ (C)  & 15.0 & 86 & 0.2 & 00 (246) & 00 & 1.46$\arcs\times$1.36$\arcs$
  \enddata
  \tablecomments{
    Col. (1) Radio array, and its configurtation, used for observation;
    Col. (2) Approximate frequency of observation;
    Col. (3) Total observing bandwidth;
    Col. (4) Total on-source integration time;
    Col. (5) Measured rms noise of final radio map and theoretical thermal noise;
    Col. (6) Peak intensity in final radio map;
    Col. (7) Beam size given as [major$\times$minor] axis.
  }
\end{deluxetable}
