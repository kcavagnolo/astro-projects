%%%%%%%%%%%%%%%%%%%
% Custom commands %
%%%%%%%%%%%%%%%%%%%

\newcommand{\inine}{IRAS 09104+4109}
\newcommand{\irs}{I09}
\newcommand{\rxj}{RX J0913.7+4056}
\newcommand{\tsync}{\ensuremath{t_{\rm{sync}}}}
\newcommand{\mykeywords}{galaxies: active -- cooling flows --
  galaxies: clusters: galaxies: individual (\inine) -- techniques:
  interferometric}
\newcommand{\mytitle}{A New Beaming Axis of the Quasar in \inine}
\newcommand{\mystitle}{New \inine\ QSO Beaming Axis}

%%%%%%%%%%
% Header %
%%%%%%%%%%

\documentclass[11pt, preprint]{aastex}
\usepackage{common,graphicx,longtable,overpic}
%\documentclass[iop]{emulateapj-rtx4}
%\usepackage{apjfonts,common,graphicx,longtable,overpic}
\usepackage[pagebackref,
  pdftitle={\mytitle},
  pdfauthor={K. W. Cavagnolo},
  pdfkeywords={\mykeywords},
  pdfsubject={Astrophysical Journal},
  pdfproducer={ps2pdf},
  pdfcreator={LaTeX with hyperref}
  pdfdisplaydoctitle=true,
  colorlinks=true,
  citecolor=blue,
  linkcolor=blue,
  urlcolor=blue]{hyperref}
\bibliographystyle{apj}

\slugcomment{Submitted to ApJ}
\shorttitle{\mystitle}
\shortauthors{Cavagnolo \etal}

\begin{document}
\title{\mytitle}
\author{
K. Cavagnolo\altaffilmark{1},
C. Ferrari\altaffilmark{1},
E. Momjian\altaffilmark{2},
J. Wrobel\altaffilmark{2},\\
A. Edge\altaffilmark{3},
and D. Hines\altaffilmark{4}}
\altaffiltext{1}{UNS, CNRS UMR 6202 Cassiop\'{e}e, Observatoire de la
  C\^{o}te d'Azur, B.P. 4229, F-06304 Nice CEDEX 4,
  France.}
\altaffiltext{2}{National Radio Astronomy Observatory, P.O. Box O,
  1003 Lopezville Road, Socorro, NM 87801-0387, USA.}
\altaffiltext{3}{Institute for Computational Cosmology, Department of
  Physics, Durham University, South Road, Durham DH1 3LE, UK.}
\altaffiltext{4}{Space Telescope Science Institute, 3700 San Martin
  Drive, Baltimore, MD 21218, USA.}
\email{cavagnolo@oca.edu}

%%%%%%%%%%%%
% Abstract %
%%%%%%%%%%%%

\begin{abstract}
  We present analysis of the sub-kiloparsec structure of the nuclear
  region around the obscured quasar in \inine\ using phase-referenced
  21 cm and 6 cm Very Long Baseline Array radio observations. The
  observations reveal the nucleus is dominated by an unresolved ($<
  100$ pc) core and a clumpy, one-sided jet system with a projected
  size of $\approx 4$ kpc. The jet properties can be reproduced if the
  quasar beaming axis is inclined $< 40\mydeg$ from our line-of-sight
  and the mean jet velocities are $\approx 0.3c$. \inine\ also hosts a
  highly-linear set of $\approx 60$ kpc large-scale jets which our
  analysis indicates cannot be continuous extensions, either through
  precession or bending by an ambient medium, of the nuclear jets. We
  conclude that the hypothesis presented in
  \citet{1999ApJ...512..145H} that the large-scale jets are no longer
  being fed by the nucleus and a new beaming axis has been established
  between 0.07--0.5 kyr ago is the scenario most consistent with the
  data. XXX: What's the prestige? Say something about BCG assembly.
\end{abstract}

%%%%%%%%%%%%
% Keywords %
%%%%%%%%%%%%

\keywords{\mykeywords}

%%%%%%%%%%%%%%%%%%%%%%
\section{Introduction}
\label{sec:intro}
%%%%%%%%%%%%%%%%%%%%%%

-- introduce i09

\object{\inine} (hereafter \irs; $z = 0.4418$) is the peculiar
brightest cluster galaxy (BCG) residing in the cool core galaxy
cluster \object{\rxj}.

enigmatic galaxy: hosts obscured qso, large-scale jets, misaligned
beaming/jet axes

highly-linear, $\approx 60$ kpc long jets misaligned from the beaming
direction of the quasar \citep[][hereafter H93 and H99,
  respectively]{1993ApJ...415...82H, 1999ApJ...512..145H}.

A detailed analysis of \irs\ is presented in \citep{iras09} suggesting
it's a radio-mode/quasar-mode crossover object

Using data from the Chandra\ X-ray Observatory (\cxo)...
   X-ray halo cavities associated with the \irs\ jets,
   irradiated icm/nebula in along misaligned quasar beaming axis
   we suggest \irs\ may be cycling between the dominant mode of feedback
   local example of high-z galaxies evolve

-- explain why important
   close-up look at bcg assembly which occurs at much higher redshifts
   need to learn about processes in details to aide modeling and further
   understanding

-- explain why nuclear props are salient
   first, verify misalignment (e.g. confirm interp)
   study qso feedback
   constrain accretion fueling?

-- what did we find and how
   Very Long Baseline Array (\vlba)
   indeed misaligned
   one-sided jet consistent with h93/99 analysis
   no signs of multinucleation
   accretion?

-- what do we conclude
   not sure right now...

Reduction of the \vlba\ radio data is discussed in Section
\ref{sec:obs}, observational findings are presented in Section
\ref{sec:res}, and our interpretation concludes the paper in Section
\ref{sec:dis}. \LCDM, for which $z = 0.4418$ corresponds to $\da
\approx 5.72$ kpc arcsec$^{-1}$ and $\dl \approx 2.45~\Gpc$. All
errors are quoted at 68\% confidence. The unit [mas] indicates
milliarcseconds, and flux density measurements for a particular radio
band are denoted using the band's approximate frequency.

%%%%%%%%%%%%%%%%%%%%%%%%%%%%%%%%%%%%%%%%%
\section{Observations and Data Reduction}
\label{sec:obs}
%%%%%%%%%%%%%%%%%%%%%%%%%%%%%%%%%%%%%%%%%

\vlba\ \dataset[ADS/NRAO.VLBA\#BH0110]{L-band} (21 cm; 1.4 GHz) and
\dataset[ADS/NRAO.VLBA\#BC0195]{C-band} (6 cm; 5.0 GHz) observations
of \irs\ were obtained in March 2003 and January 2011,
respectively. Both datasets were taken in 2-bit, dual circular
polarization using recording modes of $2\times8$ MHz sub-bands (IFs)
per polarization (128 Mbps) for L-band, and $4\times16$ MHz IFs per
polarization (512 Mbps) for C-band. Correlation was performed at NRAO
in Socorro, New Mexico using 2.1 s integrations. For both
observations, the fringe calibrator was 4C39.25 ($3.18\mydeg$ offset;
$S_{1.4} = 2$ Jy; $S_{5.0} = 000$ Jy). Nodded referencing to calibrate
the phases, rates, and delays was performed in L-band with the source
J0908+4150 ($1.32\mydeg$ offset; $S_{1.4} = 0.2$ Jy) and J0923+4125
($1.90\mydeg$ offset; $S_{5.0} = 000$ Jy) in C-band.

The data was reduced using the Astronomical Image Processing System
\vlba\ data calibration pipeline
\citep[\aips\ v31DEC11;][]{2005ASPC..340..613S} and the Common
Astronomy Software Applications (\casa\ v3.1). Flagging of bad data
was performed via manual inspection of the $uv$ measurements and using
automated tasks within \casa's {\textsc{flagdata}} tool. Ionospheric
and Earth orientation corrections were applied using, respectively,
the total electron content models from
JPL\footnote{\url{http://cddis.nasa.gov}} and geodetic solutions from
USNO\footnote{\url{http://gemini.gsfc.nasa.gov}}. The amplitude scale
was set using {\it{a priori}} corrections based on the measured system
temperatures and gains. \vlba\ sampler errors were removed, and phases
and delays were refined, by correcting for instrument-induced shifts
and accounting for antenna parallactic angle effects. Residual fringe
rates and delays were removed using antenna-based global fringe
fitting. Self-calibration of amplitudes and phases was performed for
the calibrators, and the solutions were applied to \irs.

As a check of the \vlba\ results, and to make comparisons of the
large- and small-scale \irs\ radio structure, archival Very Large
Array (\vla) \dataset[ADS/NRAO.VLA\#AH0388]{L-band},
\dataset[ADS/NRAO.VLA\#AT0211]{C-band},
\dataset[ADS/NRAO.VLA\#AH0388]{X-band} (3.6 cm; 8.3 GHz), and
\dataset[ADS/NRAO.VLA\#AH0244]{U-band} (2 cm; 15.0 GHz) observations
were also reduced and analyzed. Data editing and calibration were
performed using standard methods outlined in the
\aips\ Cookbook\footnote{\url{http://www.aips.nrao.edu/cook.html}}. Radio
images were generated by Fourier transforming, cleaning,
self-calibrating (amplitude and phase), and restoring the individual
radio observations. A summary of the radio observations are given in
Table \ref{tab:obs}.

%%%%%%%%%%%%%%%%%
\section{Results}
\label{sec:res}
%%%%%%%%%%%%%%%%%

Presented in Figure \ref{fig:high} are the highest resolution
\vlba\ 1.4 GHz and 5.0 GHz images of the \irs\ nucleus. The images
were generated by averaging down to one frequency channel and
deconvolving an area twice the size of the primary beam using uniform
weights and cells one-fifth the size of the synthesized beam. The
\irs\ nuclear region is composed of a resolved arm emerging from the
north side of an unresolved core. The extension is $\approx 140$ pc
long and may be a small-scale jet with no apparent southern
counterpart. The \irs\ nucleus is $\approx 220$ pc north-south and
$\approx 70$ pc east-west. A source is formally detected 180 pc east
of the core in the 1.4 GHz image, but it has no 5.0 GHz counterpart,
and may be an imaging artifact. Approximately 2--4 kpc north of the
nucleus in the direction of the extension, a cluster of clean
components appeared in the analysis, suggesting additional radio
structure.

To probe for faint, extended emission, lower resolution \vlba\ images
were made from the averaged data by weighting-down long baselines with
a 30\% Gaussian taper beginning at 10 M$\lambda$ in the $uv$ plane and
then cleaning with natural weights and cells 20 times the maximum
resolution. Shown in Figure \ref{fig:comp} are intensity contours from
the low resolution \vlba\ images overlaid on the \vla\ 4.8 GHz radio
image. The low resolution \vlba\ images reveal a chain of unresolved
knots (labeled 1--4) which are coincident with the elongation in the
\vla\ image and lie precisely along the direction defined by the
nucleus' northern extension. The fluxes and spectral indices of the
core, extension, and knots are given in Table \ref{tab:fluxes}. The
total knot+nucleus fluxes ($> 3\srms$) measured with the \vlba\ are
$S_{1.4} = 5.48 \pm 0.15$ mJy and $S_{5.0} = 000 \pm 000$ mJy, not
significantly different from the \vla\ measurements of the same
region: $S_{1.4} = 5.75 \pm 0.84$ mJy and $S_{5.0} = 000 \pm 000$
mJy. Neglecting emission on angular scales larger than the restoring
beams or fainter than the rms noise, it appears the dominant emitting
structures detected with the \vla\ are resolved with the \vlba. In
none of the images where the central 000 kpc are resolved is emission
detected south of the nucleus or in the direction of the large-scale
radio jets.

%%%%%%%%%%%%%%%%%%%%
\section{Discussion}
\label{sec:dis}
%%%%%%%%%%%%%%%%%%%%

%%%%%%%%%%%%%%%%%%%%%%%%%%%%%%%%%%%%%%%%%%%%
\subsection{Jet Spectral Properties and Age}
\label{sec:age}
%%%%%%%%%%%%%%%%%%%%%%%%%%%%%%%%%%%%%%%%%%%%

Before the age of the jet can be estimated, the properties of the
nuclear radio spectrum need to be known. The spectrum was modeled with
the well-known KP \citep{1962SvA.....6..317K, pach}, JP
\citep{1973A&A....26..423J}, and CI \citep{1987MNRAS.225..335H}
synchrotron models by fitting between 1.4--14.9 GHz with the
{\textsc{radiospec}} code of \citet{2009arXiv0912.2317N} which
determines the Bayesian posterior distribution and evidence for each
model. The fits for each model showed little variation, thus only the
results for the CI model, which assumes XXX electron injections and a
XXX pitch-angle distribution, are shown in Figure \ref{fig:radio}. The
maximum likelihood parameter set has a break frequency $\log \nu_B =
10.4 \pm 0.84$, spectral index of $\alpha = 1.03 \pm 0.19$, and a
likelihood ratio test statistic of $\ln \mathcal{L} =
-29.28$. Integrating under the CI curve between $\nu_1 = 0.01$ GHz and
$\nu_2 = 100$ GHz yields a luminosity of $\lrad = 5.8 \times 10^{41}
~\lum$.

The jet synchrotron age was estimated using the maximum-likelihood
parameters. Assuming inverse-Compton scattering and synchrotron
emission are the dominant radiative mechanisms, and that
re-acceleration is negligible, the age of an isotropic particle
population is \citep{2001AJ....122.1172S}:
\begin{equation}
  \tsync = 1590 \left(\frac{B^{1/2}}{B^2 + B_{\rm{CMB}}^2}\right)~
  \left[\nu_{\rm{B}} (1+z)\right]^{-1/2} ~\Myr
\end{equation}
where $B$ [\mg] is magnetic field strength, $B_{\rm{CMB}} =
3.25(1+z)^2$ [\mg] is a correction for inverse-Compton losses to the
cosmic microwave background, $\nu_{\rm{B}}$ [GHz] is the radio
spectrum break frequency, and $z$ is the dimensionless source
redshift. The magnetic field is assumed to not significantly differ
from the equipartition magnetic field strength
\citep{1980ARA&A..18..165M}:
\begin{equation}
  B_{\rm{eq}} = \left[\frac{6 \pi ~c_{12}(\alpha,\nu_1, \nu_2)
      ~\lrad ~(1+k)}{V \Phi}\right]^{2/7} ~\rm{\mg}
\end{equation}
where $c_{12}(\alpha,\nu_1,\nu_2)$ is a dimensionless constant
\citep{pach}, \lrad\ [$\lum$] is the integrated radio luminosity from
$\nu_1$ to $\nu_2$, $k$ is the dimensionless ratio of energy in
non-radiating particles to that in relativistic electrons, $V$ [\cc]
is the radio source volume, and $\Phi$ is a dimensionless radiating
population volume filling factor. Synchrotron age as a function of $k$
and $\Phi$ is shown in Figure \ref{fig:tsync}. We find an upper limit
to \tsync\ of $\approx 0.5$ Myr.

%%%%%%%%%%%%%%%%%%%%%%%%%%%%%%%%%
\subsection{Relativistic Beaming}
\label{sec:beam}
%%%%%%%%%%%%%%%%%%%%%%%%%%%%%%%%%

The nuclear radio morphology strongly indicates a one-sided jet system
resulting from Doppler boosting in a modestly relativistic jet
(\ie\ $v/c \equiv \beta \ga 0.1$) oriented close to the line-of-sight
\citep[see][for a review]{1995PASP..107..803U}. The flux density ratio
of the approaching ($S_a$) and receding ($S_r$) jets are useful for
constraining the jet velocity and beaming angle
\citep{1999ARA&A..37..409M}:
\begin{equation}
  \frac{S_a}{S_r} = \left(\frac{1+\beta \cos \theta}{1-\beta \cos
    \theta}\right)^{k+\alpha}
  \label{eqn:bar}
\end{equation}
where $\theta$ is the angle formed by the jet axis with the
line-of-sight, $\alpha$ is the radio spectral index ($d\log
S_{\nu}/d\log \nu$) for an assumed $S \propto \nu^{-\alpha}$, and $k$
is believed to vary between 2--3 depending upon the clumpiness of the
jet plasma \citep[\eg][]{1979Natur.277..182S}. We calculate a lower
limit for the $\beta \cos \theta$ term by assuming the receding jet
has a peak flux less than $3\srms$, making Equation \ref{eqn:bar} read
\begin{equation}
  \beta \cos \theta >
  \frac{(S_a/3\srms)^{1/(k+\alpha)}-1}{(S_a/3\srms)^{1/(k+\alpha)}+1}
  \label{eqn:beam}
\end{equation}
for which $\alpha = 1.0$ and $k = 3$ yields $\beta \cos \theta > 000$,
regardless of whether $S_a^{1.4} = 0.70 ~\mrms$ or $S_a^{5.0} = 000
~\mrms$ is used.

The $\beta$-$\theta$ degeneracy in Equation \ref{eqn:beam} can be
broken by applying constraints presented in published studies of the
\irs\ circumnuclear material's angle of inclination and assuming the
jet axis is orthogonal to the plane of this material. The analyses
presented in \citet{1996ApJ...460L..11G}, \citet{1997A&A...318L...1T},
and \citet{2000AJ....120..562T} of the \irs\ nuclear infrared emission
believed to originate from an extended dusty torus determined that
$\theta$ may be between $45 \dash 60\mydeg$, while the study of
nuclear UV emission and extended polarized UV emission by
\citet{1999ApJ...512..145H} found $\theta \approx 34 \dash
41\mydeg$. A plot of jet velocity as a function of $\theta$ over this
range of values is shown in Figure \ref{fig:beam}. For any given
$\theta$ value, there is a minimum value of $\beta$ which will
reproduce the observed radio emission properties, and this limit is
shown as a solid black curve in Figure \ref{fig:beam}. At face value,
Figure \ref{fig:beam} indicates the jet velocity must be $\ga 0.23c$
for $\theta > 34\mydeg$.

\citet{1999ApJ...512..145H} argue that in order to reproduce the
radial extent of scattered nuclear UV emission, the beaming direction
of the nucleus must have persisted for longer than 70 kyr. The
one-sided jet is consistent with the beaming direction proposed by
\citet{1999ApJ...512..145H}, and thus the UV timescale and jet length
can be used to derive an upper limit to the jet velocity. The radio
observations indicate the full-extent of the jet has been detected, in
which case the location of knot 4 relative to the nucleus is
approximately equal to the projected jet length, $d_{\rm{proj}} =
3.64$ kpc, and is related to the deprojected jet length as
$l_{\rm{jet}} = d_{\rm{proj}} \csc \theta$. The velocity needed to
reach knot 4 in some time $t$ is then simply $\beta =
l_{\rm{jet}}/t$. Assuming $t$ is greater than 70 kyr produces a region
of excluded velocities shown in Figure \ref{fig:beam} as a light-grey
band.



As $\theta$ increases, $l_{\rm{jet}}$ approaches the projected value
and hence the velocity needed to reach that distance also
decreases. But the velocity cannot arbitrarily drop because then the
effect of Doppler beaming diminishes and the observed jet would fade
into the noise. Thus, where the set of curves cross each other (shown
with filled circles) defines a minimum and maximum $\theta$ which can
reproduce the observed beaming and jet length. Note that 70 kyr is a
lower limit for the beaming age, and higher age values demand lower
jet velocities at any given $\theta$ further restricting $\theta$ to
values less $30\mydeg$. Assuming knot 4 defines the extent of the jet,
from this argument we can conclude that the viewing angle must be less
than $40\mydeg$.

%%%%%%%%%%%%%%%%%%
\subsection{Stuff}
\label{sec:stuff}
%%%%%%%%%%%%%%%%%%

The southern large-scale jet has the appearance of fading-out, or in,
depending on interpretation. If the velocity of the jet drops at that
point, for example if it impinges on a high density region (which it
appears to from the X-ray) that's where beaming would cease. This
would be consistent with our beaming analysis were it not for the
significantly different spectral indices between the radio core and
the large-scale jets. The flat core and steep jets/lobes suggest the
two are disconnected. Further, the highly linear large-scale jet
structure is inconsistent with the interpretation that the outflow is
no longer relativistic so close to the launch point: the jets have the
morphology of a fast flow driven through the ICM (like a firehose),
not a slow flow following the pressure anisotropies of the ICM.

-- minimum age set by equating v=1.0c and traversing ljet, what is it?
-- so, what does all this mean?
-- are the angle of inclination for the large-scale jets and the
beaming of the nucleus significantly different?
-- do our results demand a dramatic change in beaming axis?
-- can I restrict the timescale any further?
-- multiple sources?
-- coincidence with other large-scale struc
-- definitely new beaming?
-- the jets aren't bent indicating little interaction with an
   ambient medium... which weird because it's a friggin' cool core
   cluster
-- could all this be explained via jet precession
   \citep{1982ApJ...262..478G}
-- does radio core lum and optical nuc lum fit with other low-z qso's
   \citep{1984AJ.....89.1658G}?
-- nice pc to kpc analysis of 3C 120 \citep{1987ApJ...316..546W}
-- how fast are jets into the ICM, and how far do they go at these speeds?
-- is this a slow wide jet or fast narrow jet?
-- also, what does upper limits to k say about entrainment?
-- lrad suggests XXX kinetic energy from \citet{pjet} pjet-prad
   scaling relation.

XXX: How do I know that the nucleus and large-scale jets are not part
of one beaming system? Jets bend and change direction? Nuclei stop and
restart.

-- for theta less than 40deg, ljet > 6 kpc

%%%%%%%%%%%%%%%%%
\acknowledgements
%%%%%%%%%%%%%%%%%

K.W.C. and C.F. acknowledge financial support from L'Agence Nationale
de la Recherche through grant ANR-09-JCJC-0001-01. The National Radio
Astronomy Observatory is a facility of the National Science Foundation
operated under cooperative agreement by Associated Universities,
Inc. This research has made use of NASA's Astrophysics Data System
Bibliographic Services, and the NASA/IPAC Extragalactic Database (NED)
which is operated by the Jet Propulsion Laboratory, California
Institute of Technology, under contract with the National Aeronautics
and Space Administration.

{\it Facilities:} \facility{VLA}, \facility{VLBA}

%%%%%%%%%%%%%%%%
% Bibliography %
%%%%%%%%%%%%%%%%

\bibliography{cavagnolo}

%%%%%%%%%%%%%%%%%%%%%%
% Figures and Tables %
%%%%%%%%%%%%%%%%%%%%%%

\clearpage
\begin{deluxetable}{ccccccc}
  \tablecaption{Summary of Radio Observations.\label{tab:obs}}
  \tablecolumns{7}
  \tabletypesize{}
  \tablewidth{0pt}
  \tablehead{
    \colhead{Array (Config.)} & \colhead{Freq.} & \colhead{Bandwidth} & \colhead{Time} & \colhead{\srms (Theo.)} & \colhead{Peak}  & \colhead{Beam}\\
    \colhead{-}               & \colhead{GHz}   & \colhead{MHz}       & \colhead{hr}   & \colhead{\murms}        & \colhead{\mrms} & \colhead{-}\\
    \colhead{(1)}             & \colhead{(2)}   & \colhead{(3)}       & \colhead{(4)}  & \colhead{(5)}           & \colhead{(6)}   & \colhead{(7)}}
  \startdata
  \vlba\ (-) & 1.4  & 16 & 7.1 & 48 (49)  & 00 & 10.3$\times$7.9 mas\\
  \vlba\ (-) & 5.0  & 64 & 0.0 & 00 (24)  & 00 & 0.00$\times$0.00 mas\\
  \vla\ (A)  & 1.5  & 86 & 4.1 & 28 (23)  & 00 & 1.26$\arcs\times$1.15$\arcs$\\
  \vla\ (A)  & 5.0  & 86 & 2.0 & 31 (26)  & 00 & 0.41$\arcs\times$0.34$\arcs$\\
  \vla\ (A)  & 8.3  & 86 & 0.5 & 00 (41)  & 00 & 0.34$\arcs\times$0.21$\arcs$\\
  \vla\ (C)  & 15.0 & 86 & 0.2 & 00 (246) & 00 & 1.46$\arcs\times$1.36$\arcs$
  \enddata
  \tablecomments{
    Col. (1) Radio array, and its configurtation, used for observation;
    Col. (2) Approximate frequency of observation;
    Col. (3) Total observing bandwidth;
    Col. (4) Total on-source integration time;
    Col. (5) Measured rms noise of final radio map and theoretical thermal noise;
    Col. (6) Peak intensity in final radio map;
    Col. (7) Beam size given as [major$\times$minor] axis.
  }
\end{deluxetable}

\begin{deluxetable}{lcccc}
  \tablecaption{Measured Radio Flux Densities.\label{tab:fluxes}}
  \tablecolumns{5}
  \tabletypesize{}
  \tablewidth{0pt}
  \tablehead{
    \colhead{Array} & \colhead{Feature} & \colhead{$S_{1.4}$} & \colhead{$S_{5.0}$} & \colhead{$\alpha$}\\
    \colhead{-}     & \colhead{-}       & \colhead{mJy}       & \colhead{mJy}       & \colhead{-}\\
    \colhead{(1)}   & \colhead{(2)}     & \colhead{(3)}       & \colhead{(4)}       & \colhead{(5)}}
  \startdata
  \vlba\  & Nucleus & $0.86 \pm 0.07$ & $0.00 \pm 0.00$ & $0.00 \pm 0.00$\\
  \nodata & Knot 1  & $0.98 \pm 0.07$ & $0.00 \pm 0.00$ & $0.00 \pm 0.00$\\
  \nodata & Knot 2  & $1.58 \pm 0.09$ & $0.00 \pm 0.00$ & $0.00 \pm 0.00$\\
  \nodata & Knot 3  & $0.82 \pm 0.07$ & $0.00 \pm 0.00$ & $0.00 \pm 0.00$\\
  \nodata & Knot 4  & $1.24 \pm 0.02$ & $0.00 \pm 0.00$ & $0.00 \pm 0.00$\\
  \nodata & Total   & $5.48 \pm 0.15$ & $0.00 \pm 0.00$ & $0.00 \pm 0.00$\\
  \vla\   & Nucleus & unresolved      & $1.25 \pm 0.28$ & \nodata\\
  \nodata & Jet     & unresolved      & $0.39 \pm 0.11$ & \nodata\\
  \nodata & Total   & $5.75 \pm 0.84$ & $1.64 \pm 0.30$ & $1.00 \pm 0.14$
  \enddata
  \tablecomments{
    Col. (1) Radio array used for measurement;
    Col. (2) Identity of image feature;
    Col. (3) Flux density at 1.4 GHz;
    Col. (4) Flux density at 5.0 GHz;
    Col. (5) Power-law spectral index between $S_{1.4}$ and $S_{5.0}$.
  }
\end{deluxetable}

%\clearpage
\begin{figure}[htp]
  \begin{center}
    \begin{minipage}[htp]{0.9\linewidth}
      \includegraphics*[width=\textwidth, trim=15mm 10mm 10mm 10mm, clip]{beta.eps}
      \caption{Surface brightness profiles for clusters requiring a
        $\beta$-model fit for deprojection (discussed in
        \S\ref{sec:beta}). The best-fit $\beta$-model for each cluster
        is overplotted as a dashed line. The discrepancy between the
        data and best-fit model for some clusters results from the
        presence of a compact X-ray source at the center of the
        cluster. These cases are discussed in Appendix
        \ref{app:beta}.}
      \label{fig:betamods}
    \end{minipage}
  \end{center}
\end{figure}
\clearpage
\begin{figure}[htp]
  \begin{center}
    \begin{minipage}[htp]{0.9\linewidth}
      \includegraphics*[width=\textwidth, trim=5mm 0mm 5mm 5mm, clip]{itplflat_rat.eps}
      \caption{Ratio of best-fit \kna\ for the two treatments of
        central temperature interpolation (see \S\ref{sec:temppr}):
        (1) temperature is free to decline across the central density
        bins ($\Delta T_{center} \ne 0$), and (2) the temperature
        across the central density bins is isothermal ($\Delta
        T_{center} = 0$). Filled black squares are clusters for which
        the \kna\ ratio is inconsistent with unity.}
      \label{fig:kcomp}
    \end{minipage}
  \end{center}
\end{figure}
\clearpage
\begin{figure}[htp]
  \begin{center}
    \begin{minipage}[htp]{0.9\linewidth}
      \includegraphics*[width=\textwidth, trim=5mm 0mm 5mm 5mm, clip]{k0res.eps}
      \caption{Best-fit \kna\ vs. redshift. Some clusters have
        \kna\ error bars smaller than the point. The clusters with
        upper-limits ({\it{black points with downward arrows}}) are:
        A2151, AS0405, MS 0116.3-0115, and RX J1347.5-1145. The
        numerically labeled clusters are: (1) M87, (2) Centaurus
        Cluster, (3) RBS 533, (4) HCG 42, (5) HCG 62, (6) SS2B153, (7)
        A1991, (8) MACS0744.8+3927, and (9) CL J1226.9+3332. For
        CLJ1226, \cite{2007ApJ...659.1125M} found best-fit $\kna = 132
        \pm 24 \ent$ which is not significantly different from our
        value of $\kna = 166 \pm 45 \ent$. The lack of $\kna < 10
        \ent$ clusters at $z > 0.1$ is most likely the result of
        insufficient angular resolution (see \S\ref{sec:angres}).}
      \label{fig:k0res}
    \end{minipage}
  \end{center}
\end{figure}
\clearpage
\begin{center}
  \begin{figure}[htp]
    \begin{minipage}[htp]{0.5\linewidth}
      \includegraphics*[width=\textwidth, trim=28mm 7mm 30mm 17mm, clip]{curvk0.eps}
    \end{minipage}
    \begin{minipage}[htp]{0.5\linewidth}
      \includegraphics*[width=\textwidth, trim=28mm 7mm 30mm 17mm, clip]{nbins_k0.eps}
    \end{minipage}
    \begin{minipage}[htp]{0.5\linewidth}
      \includegraphics*[width=\textwidth, trim=28mm 7mm 30mm 17mm, clip]{texpk0.eps}
    \end{minipage}
    \begin{minipage}[htp]{0.5\linewidth}
      \includegraphics*[width=\textwidth, trim=28mm 7mm 30mm 17mm, clip]{ntxbins_k0.eps}
    \end{minipage}
    \caption{Plots of possible systematics versus best-fit \kna.
      {\it{Top left:}} Best-fit \kna\ plotted versus average curvature
      of the corresponding entropy profile (see eq. \ref{eqn:avgcurv})
      There is no trend between these two quantities suggesting that
      \kna\ is not heavily influenced by the total shape of the
      entropy profile. {\it{Top right:}} Best-fit \kna\ plotted versus
      number of bins in the entropy profile which were used during
      fitting. Again, no trend is found. {\it{Bottom left:}} Best-fit
      \kna\ plotted versus the total used exposure time for each
      cluster. No trend is found. {\it{Bottom right:}} Best-fit
      \kna\ plotted versus the number of bins in the temperature
      profile for each cluster. As expected, fewer $\Tx(r)$ does not
      correlate with \kna.}
    \label{fig:sys}
  \end{figure}
\end{center}
\clearpage
\begin{center}
  \begin{figure}[htp]
    \begin{minipage}[htp]{0.5\linewidth}
      \includegraphics*[width=\textwidth, trim=28mm 7mm 30mm 17mm, clip]{splots_allt.eps}
    \end{minipage}
    \begin{minipage}[htp]{0.5\linewidth}
      \includegraphics*[width=\textwidth, trim=28mm 7mm 30mm 17mm, clip]{splots_tle4.eps}
    \end{minipage}
    \begin{minipage}[htp]{0.5\linewidth}
      \includegraphics*[width=\textwidth, trim=28mm 7mm 30mm 17mm, clip]{splots_gt4tle8.eps}
    \end{minipage}
    \begin{minipage}[htp]{0.5\linewidth}
      \includegraphics*[width=\textwidth, trim=28mm 7mm 30mm 17mm, clip]{splots_tgt8.eps}
    \end{minipage}
    \caption{Composite plots of entropy profiles for varying cluster
      temperature ranges. Profiles are color-coded based on average
      cluster temperature. Units of the color bars are keV. The solid
      line is the pure-cooling model of \cite{voitbryan}, the dashed
      line is the mean profile for clusters with $\kna \le 50 \ent$,
      and the dashed-dotted line is the mean profile for clusters with
      $\kna > 50 \ent$. {\it{Top left:}} This panel contains all the
      entropy profiles in our study. {\it{Top right:}} Clusters with
      $kT_X < 4$ keV. {\it{Bottom left:}} Clusters with $4\keV < kT_X
      < 8\keV$. {\it{Bottom right:}} Clusters with $kT_X > 8$
      keV. Note that while the dispersion of core entropy for each
      temperature range is large, as the $kT_X$ range increases so to
      does the mean core entropy.}
    \label{fig:splots}
  \end{figure}
\end{center}
\clearpage
\begin{figure}[htp]
  \begin{center}
    \begin{minipage}[htp]{0.9\linewidth}
      \includegraphics*[width=\textwidth, trim=20mm 10mm 10mm 10mm, clip]{k0hist.eps}
      \caption{{\it{Top panel:}} Histogram of best-fit \kna\ for all
        the clusters in \accept. Bin widths are 0.15 in log space.
        {\it{Bottom panel:}} Cumulative distribution of \kna\ values
        for the full sample. The distinct bimodality in \kna\ is
        present in both distributions, which would not be seen if it
        were an artifact of the histogram binning. A KMM test finds
        the \kna\ distribution cannot arise from a simple unimodal
        Gaussian.}
      \label{fig:k0hist}
    \end{minipage}
  \end{center}
\end{figure}
\clearpage
\begin{figure}[htp]
  \begin{center}
    \begin{minipage}[htp]{0.9\linewidth}
      \includegraphics*[width=\textwidth, trim=20mm 10mm 10mm 10mm, clip]{hifl_k0hist.eps}
      \caption{{\it{Top panel:}} Histogram of best-fit \kna\ values
        for the primary \hifl\ sample. Bin widths are 0.15 in log
        space.  {\it{Bottom panel:}} Cumulative distribution of
        best-fit \kna\ values. The distinct bimodality seen in the
        full \accept\ sample (Fig. \ref{fig:k0hist}) is also present
        in the \hifl\ subsample and shares the same gap between the
        low-entropy peak at 10-20 \ent\ and the high-entropy peak at
        100-200 \ent. That bimodality is present in both samples is
        strong evidence it is not a result of an unknown archival
        bias.}
      \label{fig:hiflk0}
    \end{minipage}
  \end{center}
\end{figure}
\clearpage
\begin{figure}[htp]
  \begin{center}
    \begin{minipage}[htp]{0.8\linewidth}
      \includegraphics*[width=\textwidth, trim=20mm 10mm 10mm 10mm, clip]{t0.eps}
    \end{minipage}
    \begin{minipage}[htp]{0.8\linewidth}
      \includegraphics*[width=\textwidth, trim=20mm 10mm 10mm 10mm, clip]{k0cool.eps}
    \end{minipage}
    \caption{{\it{Top panel:}} Log-binned histogram and cumulative
      distribution of best-fit core cooling times, $t_{c0}$
      (eqn. \ref{eqn:tc0}), for all the clusters in \accept. Histogram
      bin widths are 0.2 in log space. {\it{Bottom panel:}} Log-binned
      histogram and cumulative distribution of core cooling times
      calculated from best-fit \kna\ values, $t_{c0}(\kna)$
      (eqn. \ref{eqn:tck0}), for all the clusters in
      \accept. Histogram bin widths are 0.2 in log space. The
      bimodality we observe in the \kna\ distribution is also present
      in best-fit $t_{c0}$. However, the gaps between the two
      populations of $t_{c0}$ and $t_{c0}(\kna)$ differ by $\sim 0.3$
      Gyrs which may be an artifact of the binning.}
    \label{fig:t0}
  \end{center}
\end{figure}



%%%%%%%%%%%%%%%%%%%%
% End the document %
%%%%%%%%%%%%%%%%%%%%

\end{document}

Due to the lack of self-calibration, there are imaging artifacts
present in the final images, but the most significant sources are all
associated with \irs\ radio emission seen in lower resolution Very
Large Array (\vla) images.

\dataset[ADS/NRAO.VLA\#AK0454]{4-band} (400 cm; 75 MHz),
\dataset[ADS/NRAO.VLA\#AK0454]{P-band} (90 cm; 333 MHz),
NRL Observing Guide\footnote{\url{http://lwa.nrl.navy.mil/tutorial/}}

For each \vla\ image, the region enclosing the nucleus seen in the
\vlba\ images was identified and fluxes were measured.
vla observations at 4.8, 8.4, 14.9:
14.9 GHz $0.61 \pm 0.21$
8.4 GHz $0.76 \pm 0.12$
