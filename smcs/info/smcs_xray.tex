\documentclass{aastex}
\bibliographystyle{apj}
\usepackage{common,graphicx}
\begin{document}

Archival \chandra\ X-ray observations of each cluster were reduced
using \ciao\ and \caldb\ versions 4.3. Cosmic rays were removed from
the data using \asca\ grades and {\textsc{vfaint}} filtering. New
level-2 event files were generated by applying corrections for the
time-dependent gain change, charge transfer inefficiency, and degraded
quantum efficiency of the ACIS detectors. The afterglow and dead area
corrections were also applied. Time intervals affected by background
flares exceeding 20\% of the mean background count rate were removed
using light curve filtering. Point sources were identified and removed
via visual inspection and use of the \ciao\ tool
{\textsc{wavdetect}}. Each flare-free, point source-free datasets was
then exposure corrected. For objects with multiple observations, the
normalized images were reprojected to a common tangent point.

The X-ray emission center of each cluster was determined using
0.5--2.0 keV normalized images and the two methods described below. In
method 1, the images were convolved with a $\sigma = 4\arcs$ Gaussian
and the location of the maximum surface brightness within an aperture
enclosing the entire cluster was taken as the center. For method 2,
the normalized images were first decomposed into multipole components,
$P_m$, using the method of \citet{1995ApJ...452..522B}:
\begin{eqnarray}
  P_m &=& \frac{(a_m^2+b_m^2)}{2_m^2 R^{2m}}\\
  a_m(R) &=& \int_{R' \le R} S(x')(R')^m \cos(m\phi') ~d^2x'\\
  b_m(R) &=& \int_{R' \le R} S(x')(R')^m \sin(m\phi') ~d^2x'\\
  P_0 &=& (a_0 \ln R)^2
\end{eqnarray}
where $R$ is the aperture radius, $S(x')$ is the X-ray surface
brightness of pixel $x'$, $\phi'$ is..., and $P_0$ is the total
intensity within the aperture. The cluster center was then defined as
the location within the aperture where the ratio of the dipole moment
to the total intensity, $P_{10} \equiv P_1/P_0$, is a
minimum. Minimization was performed using the code of \citet{vv08}
which uses a self-annealing algorithm based on the metropolis
minimization routine of \citet{1992nrfa.book.....P}. The results are
given in Table \ref{tab:centers}.

\begin{deluxetable}{ccccc}
  \tablecolumns{5}
  \tablewidth{0pc}
  \tablecaption{Cluster X-ray Emission Centers.\label{tab:centers}}
  \tablehead{
    \colhead{Cluster} & \multicolumn{2}{c}{Local Maximum} & \multicolumn{2}{c}{$P_{10}$ Minimization}\\
    \colhead{-}       & \colhead{R.A.} & \colhead{Dec.}   & \colhead{R.A.} & \colhead{Dec.}\\
    \colhead{(1)}     & \colhead{(2)}  & \colhead{(3)}    & \colhead{(4)}  & \colhead{(5)}}
  \startdata
  Abell 0068 & 00:37:06.608 & +09:09:28.34 & 00:37:04.913 & +09:09:58.84\\
  Abell 0267 & 01:52:42.383 & +01:00:48.54 & 01:52:41.990 & +01:00:31.81\\
  Abell 0611 & 08:00:56.853 & +36:03:23.67 & 08:00:56.812 & +36:03:24.16\\
  Abell 0665 & 08:30:59.066 & +65:50:22.03 & 08:30:58.193 & +65:50:49.60\\
  Abell 0697 & 08:42:57.671 & +36:21:58.64 & 08:43:00.604 & +36:21:45.84\\
  Abell 0773 & 09:17:53.142 & +51:43:41.41 & 09:17:51.394 & +51:43:33.56\\
  Abell 0963 & 10:17:03.662 & +39:02:49.29 & 10:17:03.578 & +39:02:50.27\\
  Abell 1576 & 12:36:58.128 & +63:11:13.88 & 12:36:58.128 & +63:11:14.37\\
  Abell 1758 & 13:32:38.271 & +50:33:44.74 & 13:32:46.946 & +50:32:22.60\\
  Abell 1763 & 13:35:16.479 & +41:00:02.69 & 13:35:17.522 & +40:59:56.79\\
  Abell 1914 & 14:26:03.143 & +37:49:26.85 & 14:26:03.060 & +37:49:27.34\\
  Abell 2218 & 16:35:50.620 & +66:12:36.14 & 16:35:52.897 & +66:12:31.71\\
  Abell 2261 & 17:22:27.254 & +32:07:58.11 & 17:22:23.768 & +32:07:52.67\\
  ZWCL 3146  & 10:23:39.649 & +04:11:10.87 & 10:23:42.708 & +04:11:24.16
  \enddata
\end{deluxetable}



\bibliography{cavagnolo}

\end{document}

b) X-ray surface brightness distribution [for comparison with the 2d
WL mass map - e.g. for comparison of multipole moments]

c) Mass profiles derived from the temperature maps, except for A2111
where that is not possible
