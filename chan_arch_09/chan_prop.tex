%Work needed to be done:

%-- ABUNDANCE --

%-- Extract new abundance profiles with more counts/bin. The code is
%finished and ready to use the up-to-date CIAO 4.0 and CalDB. To be
%rigorous the data needs to be updated with the new calibration (if we
%dare tread there) and shoved through the pipeline. I did a quick check
%with the current database of how many clusters we can expect to have
%after applying a minimum criterium of four annuli with 5K
%counts/annulus and that leaves at least 200 clusters.

%-- Fit abundance profiles with some function which allows for
%peakiness, central decrement, width of abundance ``core,'' and
%describes gradient at large radii. I have plenty of robust profile
%fitting code which can do this, it's a super simple task.

%-- 2D maps; The code to do this is all finished and gives the option
%to use any of multiple algorithms in making maps: contour binning, a
%variety of adaptive binning techniques, weighted voronoi tesselation
%(WVT). Having the choice different algorithms at our disposal is
%useful because different algorithms excel in different SN regimes:
%contour binning-> high SN, adaptive binning-> mid-SN, WVT-> low-SN.
%The maps which need to be made and analyzed are:
%	+ temperature
%	+ abundance
%	+ density
%	+ pressure
%	+ entropy
%	+ hardness ratio (maps and profiles, see note below)
%About the hardness ratio maps: I just finished refereeing a paper by
%Henning who is in the Burns and Hallman group at UC-Boulder. They make
%some interesting predictions about the lingering signature major
%mergers have on the hardness ratio profiles of the ICM. I don't expect
%the paper will be on arXiv by the deadline, but it makes some
%predictions which we could test in a week or two and which would be
%paper worthy.

%-- If we want to make meaningful comparisons with the outputs of
%simualtions (which I think is a big selling point of this proposal),
%we'll need someone with sims that have chemical enrichment and AGN
%feedback. Rasera et al. is a good one, and I haven't dug around to
%find others, I assumed this is something we could say we're going to
%do, and then fill-in the blank of ``with who'' later on.

%-- CONDUCTION --

%As for the conduction stuff, which may or may not be worthy of going
%in the proposal, but which I think is worth investigating either way,
%here is the list of to-do:

%A sidenote first: I recognize that Voigt & Fabian pretty much covered
%this ground, but their sample was all K0 < 30 clusters. And it's also
%worth pointing out that the systems where they found conduction was
%probably an efficient heating mechanism, those 2 (3? I can't remember)
%had the highest K0 of their sample. In addition, one cluster had no
%Halpha emission and the other had a piddly little source.

%-- calculate effective conductivity profiles (ala Voigt & Fabian)
%-- calculate implied suppression factor profiles (ala Sanderson)
%-- calculate minimum conductive efficiency (ala Rafferty in our
%conduction paper)

%-- SCALING --

%I have the entropy scaling work underway, it just needs to be finished
%and written-up. I want to write this paper, and I don't think it needs
%to go into the proposal as something we're going to do.If we want to
%tackle the topic of other scaling relations, then we'll need to:

%-- Derive LX for small fixed overdensity (r2500. r1000) apertures and
%then extrapolate out to larger fixed overdensity apertures (r500
%etc). Do we want to incorporate ROSAT and/or XMM data when possible?
%That adds the new beast of reducing and analyzing the XMM and ROSAT
%data, which I think is a good idea, but I'd need to adapt the pipeline
%to do this.

%-- Derive masses and YX

%-- Calc deviation from LX-TX for ACCEPT clusters

%%%%%%%%
% Header
%%%%%%%%

%\documentclass[letterpaper,11pt,twocolumn]{article}
\documentclass[letterpaper,11pt]{article}
\usepackage{common,graphicx}
%\usepackage[nonamebreak,numbers,square, sort&compress,comma]{natbib}
%\bibliographystyle{abbrv}
%\bibliographystyle{apj}
\setlength{\textwidth}{6.5in} 
\setlength{\textheight}{9in}
\setlength{\topmargin}{-0.0625in} 
\setlength{\oddsidemargin}{0in}
\setlength{\evensidemargin}{0in} 
\setlength{\headheight}{0in}
\setlength{\headsep}{0in} 
\setlength{\hoffset}{0in}
\setlength{\voffset}{0in}
\makeatletter
\renewcommand{\section}{\@startsection%
{section}{1}{0mm}{-\baselineskip}%
{0.5\baselineskip}{\normalfont\Large\bfseries}}%
\makeatother
\begin{document}
\pagestyle{plain}
\pagenumbering{arabic}

%%%%%%%
% Title
%%%%%%%

\begin{center} 
\bfseries\uppercase{An Archival Study of Galaxy Cluster Metal
  Enrichment, Thermal Conduction, and Scaling Relations}
\end{center}

%%%%%%%
% Intro
%%%%%%%
We recently concluded a multi-year \chandra\ archival project where we
analyzed the X-ray emission from the intracluster medium (ICM) for an
archive-limited sample of $352$ galaxy clusters. In total, we analyzed
459 ACIS-S/I observations with a combined exposure time of 12.5
Msec. The goals of the archival work were to develop a better
understanding of the galaxy cluster relaxation process and to quantify
the entropy state of cluster cores with a specific focus on how
entropy relates to feedback processes like star formation and
outbursts from active galactic nuclei (AGN). The results from our
archival work have been presented in seven referred publications thus
far [3-7, 9, 21]. We have also made the all results from our work
available to the public via the ``Archive of \chandra\ Cluster Entropy
Profiles Tables'' (\accept)
Database\footnote{http://www.pa.msu.edu/astro/MC2/accept/}.

While most of the goals of the original project have been achieved,
additional interesting scientific questions have arisen which, because
of our extensive experience, we are well-positioned to study with more
depth. Using an archive-limited sample of galaxy cluster observations,
the topics we are proposing for study are: 1) a detailed investigation
of ICM metal distributions and using those distributions as probes of
the AGN feedback mechanism, supernovae enrichment, and redshift
evolution of global cluster abundance; 2) in direct support of ongoing
theoretical work, a study to investigate the conduction properties of
the ICM for a broad range of clusters and discerning how those
properties relate to the processes of star formation and AGN feedback;
3) a study of the accuracy, scatter, and redshift evolution of
mass-observable scaling relations.

%%%%%%%%%%%
% Abundance
%%%%%%%%%%%
\large
\begin{center}
\noindent{\bf{ICM Metal Distributions}}
\end{center}
\normalsize

The heavy element content of the hot, diffuse gas pervaiding the space
between galaxies in a cluster, the ICM, is a partial record of the
integrated star formation history of cluster galaxies and how those
metals were mixed into the ICM. There are a multitude of pathways for
metals to migrate from their parent star into the ICM: ram-pressure
stripping of gas previously bound to a galaxy as it travels through
the ICM, tidal stripping of gas via galaxy-galaxy interactions, gas
outflow via galactic superwinds, gas ejection from intragalaxy stars,
and gas outflows powered by AGN outbursts which uplift enriched gas
residing in the cluster core. Limitations of the current generation of
X-ray instrumentation restrict direct observation of transport
processes like ram-pressure stripping or galactic winds to mostly
nearby objects like Coma or Virgo, and in rare instances spectacular
objects like ESO 137-001 [20]. However, simulations have shown that
the interaction between AGN jets, and subsequent buoyant bubbles which
form in the ICM, result in bulk gas motions which result in observable
alterations of the ICM metal distribution [16-18]. The AGN-ICM
interaction has also been infered for several clusters with AGN
induced ICM cavities [\eg\ 14], and directly observed in Hydra A
[\eg\ 12].

Though AGN induced mixing of metals into the ICM is not considered to
be an important contributor to global ICM enrichment, the process is
unique and interesting because it operates over a large range of
physical scales and over long time periods. Thus resulting in changes
of the bulk properties of the ICM metal distribution which are
observationally accessible with \chandra. Hence, by observing the
general characteristics of ICM metal distributions (\ie\ shape,
normalization, and ``peakiness'') for a large sample of galaxy
clusters and making additional comparisons with other cluster
properties (\eg\ gas entropy distribution, properties of X-ray
cavities, radio emission, outburst energetics, \etc), the metals
become a valuable tracer of the AGN feedback process. At some level,
the metals directly reflect ICM processes and properties like
diffusion, convection, turbulence, and viscosity. Results from the
work we are proposing can be used to place interesting constraints on
these processes and how they included in simulations. These
constraints can then be used in the next generation of simulations
which seek to produce galaxy and cluster enrichment properties which
are consistent with observations. It is therefore important to
undertake a detailed study of the ICM metal distribution for a large
sample of galaxy clusters utilizing the high-quality data found in the
\chandra\ archive. {\bf{We propose an archival study to better
    understand how AGN feedback impacts the ICM metal distribution in
    galaxy clusters.}}

We are well positioned to undertake a large archival study thanks to
our well-tested, batch-style reduction and analysis pipeline which was
written during completion of the \accept\ database. For our proposed
archival study we will measure global abundances (both with and
without the core region), radial abundance profiles, and when
reasonable signal-to-noise (SN) allows, we will also generate 2D maps.
We will then model the behavior of the radial abundance profiles by
fitting a simple function to each profile to look for trends or
correlations among the best-fit parameters and scientifically
interesting quantities such as core entropy, central abundance, the
steepness of the abundance gradient beyond the core. The modeling will
enable us to reduce the scatter in interesting relations which may
otherwise be washed out by the noise which would result from using
values taken directly from the observational analysis. Looking for
relationships among other ICM properties and metal distributions will
help form a better understanding of, for example, the efficiency of
metal transport in the ICM, the radial extent of ICM enrichment from
AGN transport, limitations on turbulent mixing, and how all of these
properties depend on cluster core environment. We will also
investigate the central abundance ``dip'' which has been seen in many
cool core clusters [19], and determine if it is feature related to a
physical mechanism in many cool core clusters (AGN excavation of
central region? Drop-out due to star formation?), or if it results
from the limitations of the data and/or the analysis process. Our
results can also be compared with results from simulations which
include feedback and chemical enrichment [\eg\ 8, 16]. We can also
investigate the controversial claim of redshift evolution in the
global abundance of galaxy clusters and/or the properties of their
abundance profiles [2, 13].

The analysis which was used during creation of the \accept\ database
will be updated to shift the focus from creating high-quality
temperature profiles to creating high-quality abundance profiles. The
quality increase will be accomplished by decreasing the uncertainties
of the best-fit spectroscopically determined abundance measurements
through improving the SN in each spectrum (\ie\ increasing the number
of counts per radial bin). In addition, we will explore using more
than a simple single-component temperature model during spectral
fitting. This can include more complex spectral models which have
multiple temperature components or variable metal species
abundances. The more complex spectral models will also give us the
ability, for spectra with good signal-to-noise, to use the relative
abundances of various metal species as a probe of the enrichment rate
from various types of supernovae. Based on the size of the
\accept\ database and the expected minimum criteria for deriving
high-quality abundance profiles (presumably a minimum of 4 radial
annuli with 5000 counts per region), we expect to have $\ga 200$
clusters in the archive-limited sample which will cover a broad range
of redshifts, luminoisities, and morphologies.

A study this large will also produce interesting comparisons of
analysis methodologies and \chandra\ calibrations. We will explore the
subtle systematic differences between various spectral models
(\eg\ MeKaL, APEC, GDEM, \etc) and how they impact abundance
measurements in addition to quantifying the role of multitemperature
gas in biasing best-fit values. The results in \accept\ were derived
using version 3.4.0 of the \chandra\ Calibration Database
(\caldb). With a new large study, we will update the analysis to use
the new \caldb\ version 4.1.1, which includes significant changes from
v3.4.0. Through comparison of results in \accept\ with the results of
a new study, we will be able to quantify how the major corrections to
the \chandra\ effective area impact existing cluster studies.

%%%%%%%%%%%%
% Conduction
%%%%%%%%%%%%
\large
\begin{center}
\noindent{\bf{ICM Thermal Conduction}}
\end{center}
\normalsize

Using measurements of \kna\ in addition to observations of
\halpha\ emission, radio emission, and blue light from BCGs, [21]
suggested that star formation in BCGs proceeds only if a multiphase
medium can form as the result of thermal conduction being less
efficient than radiative cooling. Moreover, one of the most intriguing
results from [6] is the bimodal core entropy distribution
(Fig. \ref{fig:hist}). [10] and [21] further propose that electron
thermal conduction is a possible explanation for bimodality in the
\kna\ distribution. The argument proceeds from writing the Field
length, the length scale below which a space of thermal instabilities
will be smoothed [11] ($\lambda_F$), as a function of entropy alone:
$\lambda_F \propto f_c^{1/2}K^{3/2}$, where $K$ is gas entropy and
$f_c$ is a suppression factor set by magnetic field strength and
structure [10]. Shown in Fig. \ref{fig:cond} is a plot of a few
\accept\ radial entropy profiles color coded for the presence of
\halpha\ emission from the BCG and/or a BCG with an inwardly
increasing blue color gradient. Red entropy profiles have neither
\halpha\ or a blue gradient, green have only \halpha, and blue have
both. The dashed lines indicate the value of the Field length for a
given entropy and suppression factor. Below the dashed lines, the gas
at each length scale represents a subsystem which are thermally
unstable. From the figure it appears that a transition occurs from
clusters with high \kna\ which cannot host thermal instabilities (red
profiles), to those which can (green profiles), to those which are
actively forming stars (blue profiles).

Using hydrodynamical simulations, \chandra\ and NSF funded theoretical
investigation of the coupling between AGN feedback, thermal
conduction, and BCG star formation is already underway at MSU. The
theory work is the first time detailed simulations have focused on the
relationship between the transition of gas in clusters to a multiphase
medium and conduction-related processes in the ICM. Our existing
observational work has already provided a strong set of constraints
for the theoretical study, however, additional constraints, such as
the radial behavior of the infered magnetic suppression factor and the
effective conductivity, are necessary to form a better understanding
of the smaller-scale processes which manifest in larger scale
structure (\eg\ the influence of magnetic fields).

{\bf{We propose an archival investigation into the conduction
    properties of the ICM in galaxy clusters.}} The archival work will
be in direct support of the ongoing \chandra\ funded theoretical
work. As part of the study we will place specific emphasis on studying
clusters having a \kna\ which places them in the transition region
between clusters with and without signs of thermal instability. The
data needed to undertake an archival study is already in the
\accept\ database. We will use the existing data to calculate radial
profiles of the implied suppression factor, $f_c \propto r^2 K^{-3}$,
which is an indicator for which clusters are capable of being
stabilized by thermal conduction, \ie\ $f_c = 1$. As a corrolary, the
minimum conductive efficiency, $\epsilon_{min} = \kappa T (\Lambda n_e
n_H r^2)^{-1}$, and effective conductivity, $\kappa_{eff} \propto
(\nabla{T} r^2)^{-1} \Lambda n_e n_H$, will be calculated for each
cluster. A comparison of these quantities and profiles with cluster
cooling luminosity profiles and the results from the simulations will
give insight on the dependance of thermal heat transport on radial gas
structure, and give guidance on how the radius at which conduction is
unable to prevent cooling (\eg\ the Field length) depends on various
properties of global cluster structure.

%%%%%%%%%%%%%%%
% Standing work
%%%%%%%%%%%%%%%
\large
\begin{center}
\noindent{\bf{ICM Scaling Relations}}
\end{center}
\normalsize

The archival studies we have completed and are proposing yield an
extensive amount of extra information which can be used in other
important investigations. Cluster scaling relations is one such topic
which has been the subject of exhaustive study by many groups, and
which we can further explore with our archival work. A key challenge
in using galaxy clusters as cosmological probes lies in obtaining
reliable mass estimates. A cornerstone assumption of cluster mass
estimation is that the ICM is in hydrostatic equilibrium and that
departures from equilibrium are small. However, solving the equation
of hydrostatic equilibrium requires detailed gas density and
temperature measurements, which can be exceedingly difficult to
measure for high redshift or intrinsically faint clusters. To
circumvent this problem, observationalists have traditionally infered
cluster masses by utilizing scaling relations between mass and
observables like luminosity or temperature. When $T_X$, or at the very
worst, $L_X$, are the only quantities which can be reliably measured
for a cluster, mass can still be calculated via a scaling relation.

-- L-T slope, evolution, intrinsic scatter\\
-- archival sample is wildly hetergeneous, are there advantages?\\
       + coverage of z and tx (mass)

It is also possible to use the scatter in the $L_X-M$ relation to
better understand the unknown biases in X-ray flux-limited samples
which are used for cosmological studies.\\

-- use yx as a proxy for mass and study lx-m relation\\

The scope of our proposed and prior archival work allows us to further
probe the accuracy, scatter within, and redshift evolution of
mass-observable scaling relations such as $L_X-T_X$, $Y_X-T_X$,
$M_{gas}-T_X$, and $M_{grav}-T_X$ for a sample which is larger than
any used previously.\\

-- also study deviations from self-sim evo (e.g. does scatter evolve with redshift?)\\

-- entropy probes non-grav physics\\
-- look at deviations from relations with mean temperature, yx, thbr, substructure\\
-- K-scaling relations? work is done, just need to write.

%%%%%%%%%%%%%%
% ACCEPT Dbase
%%%%%%%%%%%%%%
\large
\begin{center}
\noindent{\bf{Expansion of Interactive \accept\ Database}}
\end{center}
\normalsize

An important component of our prior archival work was making the
results, data, and analysis code available to the entire research
community via the \accept\ online database. As with our previous work,
all results of the proposed archival study will be added to the
\accept\ database so that the observational and theoretical
communities will have access to our work and the richness of the
\chandra\ archive. The database is currently a blend of \ascii\ files,
binary \fits\ tables, useful figures/graphics, and has limited
interactive functionality. With a new archival study we will convert
the existing hybrid database into a proper \sql\ database. This
conversion will allow us to maintain the ``flat'' look of the site,
which provides ease of use, but at the same time allowing us to
implement boolean operations directly from the site. This change will
circumvent the need for users to download data and manipulate it with
their own tools. With an \sql\ database and overlying \perl\ and
\python\ tools, custom output files (both \ascii\ and \fits) and
on-the-fly figure generation will also be available through the
site. Owing to limited storage space and bandwidth, currently lacking
from the site is access to fully reduced data products, such as
flare-free, point source clean events files, which would be valuable
to users (such as theorticians) who are unfamiliar with \chandra\ data
reduction but have use for reduced data products. With a portion of
the requested funds, we will to make data and results available to
users as quickly as possible by adding large capacity, high-speed data
storage (via a SAS RAID array) and high-bandwidth distribution
capabilities (via a dedicated fibre channel). Simple, rapid
dissemination of finished work to the largest audience possible
benefits both us and the CXC by expanding the scope of our proposed
and completed archival work.

%%%%%
% Bib
%%%%%
\large
\begin{center}
\noindent{\bf{References}}
\end{center}
\normalsize
\noindent
1. Balestra et al. 2007, A\&A, 462, 429\\
2. Burns et al. 2008, ApJ, 675, 1125\\
3. Cavagnolo 2008, PhD thesis, MSU\\
4. Cavagnolo et al. 2008, ApJ, 683, L107\\
5. Cavagnolo et al. 2008, ApJ, 682, 821\\
6. Cavagnolo et al. 2009, arXiv 0902.1802\\
7. Cavagnolo et al. 2009, In prep. for ApJ\\
8. De Grandi et al. 2001, ApJ, 551, 153\\
9. Donahue et al. 2006 ApJ, 643, 730\\
10. Donahue et al. 2005, ApJ, 630, L13\\
11. Field 1965, ApJ, 142, 531\\
12. Kirkpatrick et al. 2009, In prep. for ApJ\\
13. Maughan et al. 2008, ApJS, 174, 117\\
14. McCarthy et al. 2008, MNRAS, 386, 1309\\
15. McNamara et al. 2007, ARA\&A, 45, 117\\
16. Rafferty et al. 2008, ApJ, 687, 899\\
17. Rasera et al. 2008, ApJ, 689, 825\\
18. Rebusco et al. 2006, MNRAS, 372, 1840\\
18. Roediger et al. 2007, MNRAS, 375, 15\\
19. Sanderson 2009, ArXiv 0902.1747\\
20. Sun et al. 2007, ApJ, 671, 190\\
21. Voit et al. 2008, ApJ, 681, L5
%XX. Voigt et al. 2004, MNRAS, 347, 1130

%%%%%%%%%
% Figures
%%%%%%%%%
\begin{center}
%\begin{figure}[htp]
%  \includegraphics*[width=\columnwidth, trim=28mm 5mm 30mm 17mm, clip]{allsplots}
%  \caption{}
%  \label{fig:splot}
%\end{figure}
\begin{figure}[htp]
  \includegraphics*[width=\columnwidth, trim=28mm 5mm 30mm 17mm, clip]{k0hist}
  \caption{Histogran of best-fit core entropy, \kna, for the
    current \accept\ database. The vertical line is drawn at
    $\kna=30 \ent$.}
  \label{fig:hist}
\end{figure}
\begin{figure}[htp]
  \includegraphics*[width=\columnwidth, trim=0mm 0mm 0mm 0mm, clip]{cond1}
  \caption{Gas entropy plotted versus radius: no \halpha\ and no
    BCG blue gradient (red), only \halpha\ (green), both
    \halpha\ and BCG blue gradient (blue). Dashed lines are Field
    length for given entropy and supression factor.}
  \label{fig:cond}
\end{figure}
\end{center}

%%%%%%%%%%%%%%%%%%%%%%%
% Budget and prev progs
%%%%%%%%%%%%%%%%%%%%%%%
\onecolumn
\large
\begin{center}
\noindent{\bf{Proposed Budget}}
\end{center}
\normalsize
\noindent

\large
\begin{center}
\noindent{\bf{Previous \chandra\ Programs for PI Donahue, Co-I Cavagnolo, and Co-I Voit}}
\end{center}
\normalsize
\noindent
Cycle 10-GO program 10800695\\
Cycle 10-TH program 10800396\\
Cycle 6-GO program 06800721\\
Cycle 6-AR program 06800364\\
Cycle 4-AR program 04800840\\
Cycle 4-GO program 04800327\\
Cycle 2 GO Program 02800376\\
Cycle 1-GO program 01800448

\end{document}
