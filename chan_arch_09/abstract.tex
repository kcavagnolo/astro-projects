\documentclass[letterpaper,12pt]{article}
\usepackage{graphics,graphicx}
\setlength{\textwidth}{6.5in} 
\setlength{\textheight}{9in}
\setlength{\topmargin}{-0.0625in} 
\setlength{\oddsidemargin}{0in}
\setlength{\evensidemargin}{0in} 
\setlength{\headheight}{0in}
\setlength{\headsep}{0in} 
\setlength{\hoffset}{0in}
\setlength{\voffset}{0in}
\makeatletter
\renewcommand{\section}{\@startsection%
{section}{1}{0mm}{-\baselineskip}%
{0.5\baselineskip}{\normalfont\Large\bfseries}}%
\makeatother
\begin{document}
\pagestyle{plain}
\pagenumbering{arabic}
\begin{center}
\bfseries\uppercase{The Hyperluminous Infrared Galaxy IRAS 09104+4109: An Extreme Brightest Cluster Galaxy}
\end{center}
\section{Abstract}
We propose a detailed study of the hyperluminous infrared BCG IRAS
09104+4109. This BCG is in the rich cluster MACS J0913.7+4056, and
likely hosts a ``changing-look'' AGN and the highest redshift AGN
blown bubbles known to date. The environment of the BCG is best
described as extreme, with cannibalized companion galaxies, the most
powerful radio source of any IRAS object, and an AGN which has
established a new beaming direction in the last 70 kyrs. Understanding
the relationship of the BCG, AGN, and ICM in this peculiar and unique
object will aide in developing better models for coupling together
galaxy formation, AGN feedback, and large scale cluster
environment. IRAS 09104+4109 is an ideal test case of a very
short-lived but highly active stage of cluster and central galaxy
formation.
\end{document}
