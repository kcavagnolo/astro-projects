\newcommand{\iras}{IRAS 09104+4109}
\newcommand{\irs}{IRAS09}
\newcommand{\rxj}{RX J0913.7+4056}
\documentclass[11pt]{article}
\usepackage{proposal,graphicx}

%%%%%%%%%%%%%%%%%%%%%%

\begin{document}

%%%%%%%%%%%%%%%%%%%%%%

\picoveleta

%%%%%%%%%%%%%%%%%%%%%%

\title{Constraining the Cold Gas Mass in RBS 797, One of the Most Powerful AGN Outbursts}

%%%%%%%%%%%%%%%%%%%%%%

\exgalco

%%%%%%%%%%%%%%%%%%%%%%

\abstract{A maximum of nine lines is reserved for the abstract.}

%%%%%%%%%%%%%%%%%%%%%%%%%%%%%%%%%%%%%%%%%%%%%%%%%%%%%%%%%%%%%%%%
% NOTE: telescope time includes all overheads, like in the Time
% Estimator
%%%%%%%%%%%%%%%%%%%%%%%%%%%%%%%%%%%%%%%%%%%%%%%%%%%%%%%%%%%%%%%%

\hours{XX}

%%%%%%%%%%%%%%%%%%%%%%%%%%%%%%%%%%%%%%%%%%%%%%%%%%%%%%%%%%%%%%%%%%%
% Receivers (remove the '%' sign when using the receiver) argument:
% telescope time, hours, requested for this receiver
%%%%%%%%%%%%%%%%%%%%%%%%%%%%%%%%%%%%%%%%%%%%%%%%%%%%%%%%%%%%%%%%%%%

\emir{XX}

%%%%%%%%%%%%%%%%%%%%%%%%%%%%%%%%%%%%%%%%%%%%%%%%%%%%%%%%%%%%%%%%%%%%%%
% Sidereal time intervals: {start hour}{end hour}{number of intervals}
%%%%%%%%%%%%%%%%%%%%%%%%%%%%%%%%%%%%%%%%%%%%%%%%%%%%%%%%%%%%%%%%%%%%%% 

%\lsta{11}{11}{10}
%\lstb{23}{23}{10}

%%%%%%%%%%%%%%%%%%%%%%
% Special requirements 
%%%%%%%%%%%%%%%%%%%%%%

\remoteobserving
%\LargeProgram
%\pooledobserving
%\serviceobserving
%\polarimeter

%%%%%%%%%%%%%%%%%%%%%%

\constraints{None}

%%%%%%%%%%%%%%%%%%%%%%

\pifirstname{Kenneth}
\pilastname{Cavagnolo}
\piinstitute{University of Waterloo; Dept. of Physics \& Astronomy}
\pistreetandnumber{200 University Avenue West}
\pizipandcity{Waterloo, Ontario, Canada N2L 3G1}
\picountry{Canada}
\piphone{(+001) 519888456735074}
\pifax{(+001) 5197468115}
\piemail{kcavagno@uwaterloo.ca}

%%%%%%%%%%%%%%%%%%%%%%

\coauthor{Chiara Ferrari}{Observatoire de la C\^ote d'Azur}{France}
\coauthor{Myriam Gitti}{INAF-OAB}{Italy}
\coauthor{Brian McNamara}{University of Waterloo}{Canada}
\coauthor{Paul Nulsen}{Harvard-Smithsonian Center for Astrophysics}{U.S.A.}
\coauthor{David Rafferty}{Leiden Observatory}{Netherlands}
\coauthor{Michael Wise}{University of Amsterdam}{Netherlands}

%%%%%%%%%%%%%%%%%%%%%%

\observer{Cavagnolo}

%%%%%%%%%%%%%%%%%%%%%%

\epoch{J2000.0}  % J2000 epoch coordinates preferred
\sourcelist{RBS 797 & 09:47:12.5 & +76:23:12 & 0.354}

%%%%%%%%%%%%%%%%%%%%%%

\frontpage

%%%%%%%%%%%%%%%%%%%%%%%%%%%%%%%%%%%%%
%%%%%%%%%%%%%%%%%%%%%%%%%%%%%%%%%%%%%
%--- EMIR -------- EMIR -------- EMIR
%%%%%%%%%%%%%%%%%%%%%%%%%%%%%%%%%%%%%
%%%%%%%%%%%%%%%%%%%%%%%%%%%%%%%%%%%%%
% Enter up to 2 \EMIR parameters macros for each setup each macro MUST
% have 9 parameters (pairs {} of curly brackets) :
% setup - band - species - transition - GHz - Ta* - rms - width -backend 

\EMIRparameters{1}{E0}{CO}{1-0}{85.1}{143.2}{XX}{100.0}{W}
\EMIRparameters{1}{E1}{CO}{2-1}{170.2}{311.1}{XX}{100.0}{W}
\EMIRparameters{2}{E2}{CO}{3-2}{255.3}{370.4}{XX}{100.0}{W}

% EMIR observing parameters and telescope time requested, in hours.
% enter one macro for each EMIR setup, if maps:
% setup - dRA - .dDec.obsMode - switchMode - .pwv - .hours - remark

\EMIRmap{1}{}{}{none}{PSw}{7}{X}{Dual-band observation with E1}
\EMIRmap{1}{}{}{none}{PSw}{7}{X}{Dual-band observation with E0}
\EMIRmap{2}{}{}{none}{PSw}{7}{X}{Single-band observation}

%%%%%%%%%%%%%%%%%%%%%%%%%%%%%%%%%%%%%%%%%%%%%%%%%%%%%%%%
% The following command will produce the Technical Sheet
%%%%%%%%%%%%%%%%%%%%%%%%%%%%%%%%%%%%%%%%%%%%%%%%%%%%%%%%

\techsheetPV

%%%%%%%%%%%%%%%%%%%%%%

\maketitle

%%%%%%%%%%%%%%%%%%%%%%

\proposalhistory{None.}

%%%%%%%%%%%%%%%%%%%%%%

intro:
  cooling flows
  agn feedback/cavs
  smbh feeding on cool gas make natural feedback loop
  cavities give info about mass accretion
  use to place constraints on mass and spin of smbhs
  accretion:
    grav binding energy released by infall matter
    low Edd -- matter forms disk, and is advected in or blown away
    high Edd -- quasar
  spin:
    accretion powered, but more power per unit mass
    jets launched by rotation of hole, disk, or both
  what's the impact of these studies?
  CO->H2 mass provides highest detectable limit of ``cold'' gas content
  to test the two models, need gas poor systems with lots of energy coming out, e.g. ms0735
  for example, r797

case of r797:
  deep cavs
  not much dust
  not much Halpha
  lack of stars
  weird radio
  > 10^45 e/s

what we want, deep CO obs:
  why?
  what will we do with it?
  how?
  is it possible?
  show calculations
  explain setup and time requests

%%%%%%%%%%%%%%%%%
{\bf{References}}
%%%%%%%%%%%%%%%%%

\begin{figure}[htp]
  \begin{center}
    \begin{minipage}{0.485\linewidth}    
      \includegraphics*[width=\textwidth, trim=66mm 21mm 66mm 21mm, clip]{xray_r797}
    \end{minipage}
    \begin{minipage}{0.505\linewidth}
      \includegraphics*[width=\textwidth, trim=60mm 17mm 60mm 15mm, clip]{beam_r797}
    \end{minipage}
    \caption{}
    \label{fig:r797}
    \begin{minipage}{\linewidth}
      \includegraphics*[width=\textwidth, trim=28mm 10mm 8mm 50mm, clip]{r797_iram_mh2.eps}
    \end{minipage}
    \caption{EMIR CO line intensity, rms noise, and H$_2$ mass as a
      function of position switched observing time. Solid lines are
      $I_{\mathrm{CO}}$ $3\sigma$ upper limits; dashed-dot lines are
      $T_{\mathrm{rms}}$ specific to elevation \&
      $\nu_{\mathrm{obs}}$; dashed lines are M(H$_2$) $3\sigma$ upper
      limits; downward arrows are $3\sigma$ M(H$_2$) upper limits from
      Evans et al. 1998 (adjusted to our cosmology and
      I$_{\mathrm{CO}}$:M(H$_2$) assumptions). The CO(1-0) and CO(2-1)
      calculations include the 16K and 6K $T_{\mathrm{sys}}$
      increases, respectively, from use of the E0/E1 dichroic. All
      calculations assumed: 7 mm of precipitable water vapor, $\Delta
      v_{\mathrm{res}} = 50$ km s$^{-1}$,
      $v^{\mathrm{CO}}_{\mathrm{FWHM}} = 300$ km s$^{-1}$, and backend
      efficiency of 0.87.}
    \label{fig:feas}
  \end{center}
\end{figure}

%%%%%%%%%%%%%%
\end{document}
%%%%%%%%%%%%%%
