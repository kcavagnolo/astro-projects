\newcommand{\trms}{\ensuremath{T_{\mathrm{rms}}}}
\newcommand{\htoo}{\ensuremath{\mathrm{H}_{2}}}
\newcommand{\mht}{\ensuremath{\mathrm{M(H}_{2})}}
\newcommand{\vco}{\ensuremath{\Delta v^{\mathrm{CO}}_{\mathrm{FWHM}}}}
\newcommand{\cooz}{\ensuremath{\mathrm{CO} (1-0)}}
\newcommand{\coto}{\ensuremath{\mathrm{CO} (2-1)}}
\newcommand{\cott}{\ensuremath{\mathrm{CO} (3-2)}}
\newcommand{\lcp}{\ensuremath{L^{\prime}_{\mathrm{CO}}}}
\newcommand{\flux}{\ensuremath{\mathrm{erg~cm^{-2}~s^{-1}}}}
\newcommand{\lum}{\ensuremath{\mathrm{erg~s^{-1}}}}
\newcommand{\msol}{\ensuremath{\mathrm{M}_{\odot}}}
\newcommand{\msolpy}{\ensuremath{\mathrm{M}_{\odot}~\mathrm{yr}^{-1}}}
\newcommand{\lsol}{\ensuremath{\mathrm{L}_{\odot}}}
\newcommand{\iras}{IRAS 09104+4109}
\newcommand{\irs}{IRAS09}
\newcommand{\rxj}{RX J0913.7+4056}
\documentclass[11pt]{article}
\usepackage{proposal,graphicx}

%%%%%%%%%%%%%%%%%%%%%%

\begin{document}

%%%%%%%%%%%%%%%%%%%%%%

\picoveleta

%%%%%%%%%%%%%%%%%%%%%%

\title{Searching for Molecular Gas in IRAS 09104+4109}

%%%%%%%%%%%%%%%%%%%%%%

\exgalco

%%%%%%%%%%%%%%%%%%%%%%

\abstract{NEEDS ABSTRACTNEEDS ABSTRACTNEEDS ABSTRACTNEEDS
  ABSTRACTNEEDS ABSTRACTNEEDS ABSTRACTNEEDS ABSTRACTNEEDS
  ABSTRACTNEEDS ABSTRACTNEEDS ABSTRACTNEEDS ABSTRACTNEEDS
  ABSTRACTNEEDS ABSTRACTNEEDS ABSTRACTNEEDS ABSTRACTNEEDS
  ABSTRACTNEEDS ABSTRACTNEEDS ABSTRACTNEEDS ABSTRACTNEEDS
  ABSTRACTNEEDS ABSTRACTNEEDS ABSTRACTNEEDS ABSTRACTNEEDS
  ABSTRACTNEEDS ABSTRACTNEEDS ABSTRACTNEEDS ABSTRACTNEEDS
  ABSTRACTNEEDS ABSTRACTNEEDS ABSTRACTNEEDS ABSTRACT All your base are
  belong to us?}

%%%%%%%%%%%%%%%%%%%%%%%%%%%%%%%%%%%%%%%%%%%%%%%%%%%%%%%%%%%%%%%%
% NOTE: telescope time includes all overheads, like in the Time
% Estimator
%%%%%%%%%%%%%%%%%%%%%%%%%%%%%%%%%%%%%%%%%%%%%%%%%%%%%%%%%%%%%%%%

\hours{16}

%%%%%%%%%%%%%%%%%%%%%%%%%%%%%%%%%%%%%%%%%%%%%%%%%%%%%%%%%%%%%%%%%%%
% Receivers (remove the '%' sign when using the receiver) argument:
% telescope time, hours, requested for this receiver
%%%%%%%%%%%%%%%%%%%%%%%%%%%%%%%%%%%%%%%%%%%%%%%%%%%%%%%%%%%%%%%%%%%

\emir{16}

%%%%%%%%%%%%%%%%%%%%%%%%%%%%%%%%%%%%%%%%%%%%%%%%%%%%%%%%%%%%%%%%%%%%%%
% Sidereal time intervals: {start hour}{end hour}{number of intervals}
%%%%%%%%%%%%%%%%%%%%%%%%%%%%%%%%%%%%%%%%%%%%%%%%%%%%%%%%%%%%%%%%%%%%%% 

%\lsta{11}{11}{10}
%\lstb{23}{23}{10}

%%%%%%%%%%%%%%%%%%%%%%
% Special requirements 
%%%%%%%%%%%%%%%%%%%%%%

%\remoteobserving
%\LargeProgram
%\pooledobserving
%\serviceobserving
%\polarimeter

%%%%%%%%%%%%%%%%%%%%%%

\constraints{None}

%%%%%%%%%%%%%%%%%%%%%%

\pifirstname{Kenneth}
\pilastname{Cavagnolo}
\piinstitute{University of Waterloo; Dept. of Physics \& Astronomy}
\pistreetandnumber{200 University Avenue West}
\pizipandcity{Waterloo, Ontario, Canada N2L 3G1}
\picountry{Canada}
\piphone{(+001) 519888456735074}
\pifax{(+001) 5197468115}
\piemail{kcavagno@uwaterloo.ca}

%%%%%%%%%%%%%%%%%%%%%%

\coauthor{Megan Donahue}{Michigan State University}{U.S.A.}
\coauthor{Alastair Edge}{Durham University}{U.K.}
\coauthor{Chiara Ferrari}{Observatoire de la C\^ote d'Azur}{France}
%\coauthor{Brian McNamara}{University of Waterloo}{Canada}

%%%%%%%%%%%%%%%%%%%%%%

\observer{PI Cavagnolo}

%%%%%%%%%%%%%%%%%%%%%%

\epoch{J2000.0}  % J2000 epoch coordinates preferred
\sourcelist{IRAS09104 & 09:13:45.5 & +40:56:28 & 0.4418}

%%%%%%%%%%%%%%%%%%%%%%

\frontpage

%%%%%%%%%%%%%%%%%%%%%%%%%%%%%%%%%%%%%
%%%%%%%%%%%%%%%%%%%%%%%%%%%%%%%%%%%%%
%--- EMIR -------- EMIR -------- EMIR
%%%%%%%%%%%%%%%%%%%%%%%%%%%%%%%%%%%%%
%%%%%%%%%%%%%%%%%%%%%%%%%%%%%%%%%%%%%
% Enter up to 2 \EMIR parameters macros for each setup each macro MUST
% have 9 parameters (pairs {} of curly brackets) :
% setup - band - species - transition - GHz - Ta* - rms - width -backend 

\EMIRparameters{1}{E0}{CO}{1-0}{79.9}{162.7}{0.420}{100.0}{W}
\EMIRparameters{1}{E1}{CO}{2-1}{159.9}{276.5}{0.514}{100.0}{W}
\EMIRparameters{2}{E2}{CO}{3-2}{239.8}{488.1}{0.741}{100.0}{W}

% EMIR observing parameters and telescope time requested, in hours.
% enter one macro for each EMIR setup, if maps:
% setup - dRA - .dDec.obsMode - switchMode - .pwv - .hours - remark

\EMIRmap{1}{}{}{none}{PSw}{7}{8}{Dual-band observation with E1}
\EMIRmap{1}{}{}{none}{PSw}{7}{8}{Dual-band observation with E0}
\EMIRmap{2}{}{}{none}{PSw}{7}{8}{Single-band observation}

%%%%%%%%%%%%%%%%%%%%%%%%%%%%%%%%%%%%%%%%%%%%%%%%%%%%%%%%
% The following command will produce the Technical Sheet
%%%%%%%%%%%%%%%%%%%%%%%%%%%%%%%%%%%%%%%%%%%%%%%%%%%%%%%%

\techsheetPV

%%%%%%%%%%%%%%%%%%%%%%

\maketitle

%%%%%%%%%%%%%%%%%%%%%%%%%%%%%%%%%%%%%%%%%%%%%%%%%%%%%%%%%%%%%%%%%
% Scientific justification (do not neglect this part ...)  max. 2
% pages of text (4 pages for large programs) plus 2 pages of Figs.,
% Tables and Refs.
%%%%%%%%%%%%%%%%%%%%%%%%%%%%%%%%%%%%%%%%%%%%%%%%%%%%%%%%%%%%%%%%%

\proposalhistory{None.}

%%%%%%%%%%%%%%%%%%%%%%

{\bf{SCIENCE JUSTIFICATION:}} The object \iras\ (hereafter, \irs;
shown in Figure \ref{fig:i09}) is an uncommon low-redshift ($z =
0.4418$) hyperluminous infrared galaxy (HyLIRG; $L_{\mathrm{IR}} >
10^{13} ~\lsol$). Unlike nearly all HyLIRGs, \irs\ is the brightest
cluster galaxy (BCG) in the $\sim 10^{15} ~\msol$ galaxy cluster
\rxj. In the core region of \rxj, the cooling time of the X-ray
luminous intracluster medium (ICM) is $< 0.3$ Gyr, and the inferred
ICM mass condensation rate is $> 500 ~\msolpy$. At such a prodigious
cooling rate, in 1 Gyr the reservoir of gas which should condense out
of the ICM onto the BCG exceeds $10^{11} ~\msol$. This is the
well-known ``cooling flow'' scenario where hot gas cools through a
continuum of temperatures, flows into a reservoir within the central
dominant galaxy (cD), and is converted to molecular clouds and stars
(see Peterson \& Fabian 2006 for a review). Equally well-known, is
that observations of cDs in cooling flow clusters indicate the true
mass deposition rates are much smaller, on average $< 50 ~\msol$, with
most of the gas localized to the cD and residing in emission line
nebulae, sparse molecular gas, and few stars (Heckman et al. 1989,
McNamara et al. 1990, O'Dea et al. 1994, Edge 2001, Edge \& Frayer
2003, Jaffe et al. 2005, Egami et al. 2006, Sanders et al. 2010). A
finely-tuned feedback loop involving active galactic nuclei (AGN) has
emerged as the favored explanation of why X-ray halo cooling is
suppressed in systems like \irs\ (B\^irzan et al. 2004, Bower et
al. 2008, Croton et al. 2006, Dunn et al. 2005).

\irs\ seems to fit nicely within the AGN feedback regulation model:
there is a Seyfert-2 quasar (QSO) in the galaxy which is radiating $>
5 \times 10^{46} ~\lum$ of energy (Iwasawa et al. 2001) and depositing
$> 5 \times 10^{44} ~\lum$ of energy into its environment via jets
(Cavagnolo et al. 2010). But the normalcy ends there. Unlike most
BCGs, 99\% of \irs's bolometric luminosity is emitted longward of 1
$\mu$m due to QSO radiation being reprocessed by obscuring material
having $N_H > 10^{24}$ cm$^{-2}$ and a covering factor near unity
(Kleinmann, et al. 1988, Hines and Wills. 1993, Cavagnolo et
al. 2010). While $\sim 3 \times 10^8 ~\msol$ of hot dust ($>200$ K)
and extensive optical nebulae are detected in \irs, $< 10^8 ~\msol$ of
cold dust ($< 100$ K) and no PAH features are detected (Deane \&
Trentham 2001, Peeters et al. 2004, Sargaysan et al. 2008). Moreover,
Evans et al. 1998 did not detect \coto\ or \cott\ lines, implying a
\htoo\ mass $< 8 \times 10^{10} ~\msol$. While the \irs\ gas-to-dust
ratio is suspiciously small ($< 270$) compared to other cooling flow
BCGs (typically $> 500$), possibly indicating influence from the AGN,
the cold dust and \htoo\ upper limits were derived from data of
insufficient quality to reveal if the system is truly molecular gas
poor. However, near-IR spectroscopy shows that \irs\ emits $> 10^{42}
~\lum$ in H$\alpha$ (Evans et al. 1998). If \irs\ has $\sim 1-2 \times
10^{10} ~\msol$ of molecular gas, then compared to other BCGs, the
H$\alpha$ luminosity indicates \irs\ is certainly molecular gas poor
(Edge 2001), exacerbating the fact that strong indicators of cold,
dusty nuclear gas are expected for a QSO system like \irs.

But, to investigate the gas content further requires pushing to lower
mass limits with deeper observations. We have analyzed sufficiently
deep {\it{Chandra}} X-ray data to study the large-scale cooling of the
ICM hosting \irs, and have applied for deep {\it{HST}} near-IR and
near-UV high-resolution imaging \& spectroscopy. Our observational
campaign also includes a request for deep sub-mm data using
{\it{SCUBA-2}} (instrument PI Fich at Univ. Waterloo) to probe for
cold, dusty gas. But a key component of this effort which is missing
is to place tighter constraints on the molecular gas mass of using CO
observations. {\bf{Thus, we are requesting 16 hours on the IRAM 30 m
    using the new EMIR receiver to constrain, or detect, the suspected
    paucity of CO emission from \iras.}}  Our time request is
specifically intended to reach a \htoo\ mass limit of $\approx 1.5
\times 10^{10} ~\msol$ (discussed in next section), which is
sufficient to determine if \irs\ is molecular gas poor. The EMIR
receiver also allows us to detect high-velocity CO emission ($\vco >
600$ km s$^{-1}$) which could have gone undetected in the previous
IRAM 30 m observations presented in Evans et al. 1998, and might mask
the presence of $10^{11} ~\msol$ of \htoo.

From a broad perspective, it may be that \irs\ is undergoing a rare
phase of BCG assembly where radiative and mechanical feedback are
simultaneously conspiring to quench cooling within and around the host
galaxy. Numerical simulations have suggested such a phase should
exist, though no conclusive examples have been found. Cosmological
simulations typically put radiation- and kinetic-dominated feedback
into a distinct early-time quasar-mode ({\it{e.g.}} Springel et
al. 2005) and a late-time radio-mode ({\it{e.g.}} Croton et al. 2006),
respectively. The quasar-mode of feedback is expected to be brief,
expelling large quantities of molecular gas from the host galaxy
(Narayanan et al. 2006), while the prolonged, intermittent radio-mode
heats the host environment, regulating cooling for the rest of the
galaxy's life (see McNamara \& Nulsen 2007 for a review). At odds with
the models, CO observations of high-redshift quasars reveal they
harbor ample supplies of gas (Bertoldi et al. 2003), and the best
evidence of AGN feedback expelling molecular gas has come from
observations of $z < 0.1$ early-type galaxies ({\it{e.g.}} Schawinski
et al. 2009).

\irs\ may be our nearby window into how the crossover between
quasar-mode and radio-mode feedback functions in cDs. With deeper CO
data, we can explore the possibility that the AGN in \irs\ has
expelled and/or heated the gas reservoir from which it is fueled. If
deeper CO observations indicate the molecular gas mass is $< 10^{10}
~\msol$, then in the context of the quasar/radio-mode model, the
abundance of hot dust relative to molecular gas makes sense. As QSO
radiation interacts with dusty gas within the host galaxy, the gas is
heated, destroying the molecular component and boosting the IR
luminosity (e.g. Ciotti et al. 2010). On the other hand, if a $10^{11}
~\msol$ high-velocity component were to be detected, that may indicate
a very powerful outflow, or the gas could be associated with a cooling
flow.

As just mentioned, our requested observations will also help address
the {\it{origin}} of the gas reservoir in \irs, which is a bit of a
mystery. There are six compact spheroids within 50 kpc of the BCG
which may be the bulges of cannibalized companions (circled in Figure
\ref{fig:i09}). However, the bright BCG nebulae are stationary
relative to the galaxy, suggesting they were not stripped from
companions (Crawford \& Vanderriest 1996). But, the large dust content
of the nebulae likely rules out the hot ICM as the origin since dust
has a short sputtering time in hot gas (Donahue \& Voit 1993). The
velocity structure of the CO lines will enable us to constrain the
kinematics of the molecular gas. For example, if the gas originated
from a cooling flow, then it should have a velocity dispersion
consistent with the galactic potential. On the other hand, if the gas
was stripped from companions, or is flowing out at high velocities,
then the velocity structure will be different. Very narrow lines would
also suggest velocities consistent with the bright [O III] nebulae,
suggesting they are dynamically related.

%%%%%%%%%%%%%%%%%%%%%%%%%%%%
%%%%%%%%%%%%%%%%%%%%%%%%%%%%

{\bf{REQUEST FOR OBSERVATIONS:}} \irs\ was observed in June 1994 with
the IRAM 30 m by Evans et al. 1998. However, the bandwidth and
sensitivity of those observations, $\ll 1$ GHz \& $\approx 4$ mK,
respectively, yields uninteresting upper limits for \mht. Installation
of the new EMIR receiver on the IRAM 30 m allows us to reach $\trms <
1$ mK and provides the bandwidth (4-8 GHz) necessary to detect $\vco
\leq 800$ km s$^{-1}$ while still probing mass scales $< 10^{11}
~\msol$. We are interested in observations of the \cooz, \coto,
\cott\ transitions at redshifted 79.9 GHz, 159.9 GHz, and 239.8 GHz,
respectively. The associated half power beam widths overlaid on an
{\it{HST}} image of \irs\ is shown in Figure \ref{fig:i09}.

The transitions of interest fall in the E0, E1, and E2 bands of
EMIR. The incoming beams can be split with a dichroic to reduce the
requested observing time. For our requested observations, the possible
combinations of bands are E0/E2 and E0/E1. Using Table 2 of the EMIR
Users Guide, and assuming 1\% loss of signal equals a 4 K increase of
the receiver temperature, the E0/E2 increase is 30 K and the E0/E1
increase is 22 K. We therefore select the dual-band mode for our E0
and E1 observations and observe E2 on its own.

Equations 1-4 were used to calculate $3\sigma$ upper limits for
\mht. The quantities rms noise, $T_{\mathrm{rms}}$, and system
temperature, $T_{\mathrm{sys}}$, were computed from the relations
provided in IRAM memo 2009-1 using the telescope efficiencies and beam
widths from the EMIR Users Manual and the EMIR Commissioning
Report. The relations for \lcp\ and $S_{\mathrm{CO}}$ come from
Solomon et al. 1997 and Salom\'e \& Combes 2008, respectively. We have
assumed $\vco = 500$ km s$^{-1}$, $v_{\mathrm{res}} = 100$ km
s$^{-1}$, and a precipitable water vapor of 7 mm for all
calculations. The brightness temperature, $T_{\mathrm{rms}}$, and
\mht\ upper limits are plotted as a function of exposure time in
Figure \ref{fig:feas}. For our requested exposure times of 8 hrs, the
\cooz, \coto, and \cott\ mass limits (in $10^{10}$ \msol) are 7.52,
2.30, and 1.47, respectively. These limits are sufficient to achieve
our outlined science objectives.

The right panel of Figure \ref{fig:i09} shows that the six spheroids
of what may be cannibalized companions lie within all the beams. There
are three additional bright elliptical galaxies which fall in the
159.9 GHz beam, and five more bright ellipticals in the 79.9 GHz
beam. There may be two spiral galaxies in the field, one in the 159.9
GHz beam and one in the 79.9 GHz beam. For a false detection of $\mht
\sim 10^{10} ~\msol$, each source in the 239.8 GHz beam needs to
contribute $1.7 \times 10^9 ~\msol$, in the 159.9 GHz beam $1.1 \times
10^9 ~\msol$, and $0.7 \times 10^9 ~\msol$ in the 79.9 GHz beam. These
are the gas masses of extremely gas-rich spirals, or ellipticals with
$K$-band magnitudes $> -20$ mag, neither of which is true for any of
the field sources. We conclude contamination from background sources
in our observations will be insignificant.
\begin{eqnarray}
  \mathrm{CO ~brightness ~temperature,} ~I_{\mathrm{CO}} &<& 3
  ~T_{\mathrm{rms}} ~\vco
  (\vco/v_{\mathrm{res}})^{-1/2} ~\mathrm{[K ~km
      ~s^{-1}]}\\
  \mathrm{Integrated ~CO ~intensity,} ~S_{\mathrm{CO}} &<& 6.8
  ~(1+z)^{-1/2} ~I_{\mathrm{CO}} ~\mathrm{[Jy ~km ~s^{-1}]}\\
  \mathrm{CO ~line ~luminosity,} ~\lcp &<& 3.3 \times 10^7
  ~S_{\mathrm{CO}} ~\nu_{\mathrm{obs}}^{-2} ~D_L^2 ~(1+z)^{-3}
  ~\mathrm{[K ~km ~s^{-1} ~pc^{-2}]}\\
  \mathrm{Molecular mass,} ~\mht &<& 4.6 ~\lcp\ ~[\msol].
\end{eqnarray}

\clearpage
\noindent {\bf{REFERENCES}}\\
B\^irzan, et al. ApJ, 607:800-809, 2004.\\
Bertoldi, et al. A\&A, 409:47, 2003.\\
Bower, et al. MNRAS, 390:1399-1410, 2008.\\
Cavagnolo, et al. ApJ, in prep., 2010\\
Ciotti, et al. arXiv 2010, eprint arXiv:1003.0578\\
Crawford \& Vanderriest. MNRAS, 283:1003-1014, 1996.\\
Croton, et al. MNRAS, 365:11-28, 2006.\\
Deane \& Trentham. MNRAS, 326:14671474, 2001.\\
Donahue \& Voit. ApJ, 414:L17L20, 1993.\\
Dunn, et al. MNRAS, 364:1343-1353, 2005.\\
Edge. MNRAS, 328:762-782, 2001.\\
Edge \& Frayer. ApJ, 594:13-17, 2003.\\
Egami, et al. ApJ, 647:922-933, 2006.\\
Evans, et al. ApJ, 506:205221, 1998\\
Heckman, et al. ApJ, 338:48-77, 1989.\\
Hines and Wills. ApJ, 415:82+, 1993.\\
Iwasawa, et al. MNRAS, 321:15-19, 2001.\\
Jaffe, et al. MNRAS, 360:748-762, 2005.\\
Kleinmann, et al. ApJ, 328:161169, 1988\\
McNamara, et al. ApJ, 360:20-29, 1990.\\
McNamara \& Nulsen. ARA\&A, 45:117-175, 2007.\\
Narayanan, et al. ApJ, 642:107, 2006.\\
O'Dea, et al. ApJ, 422:467-479, 1994.\\
Peeters, et al. ApJ, 613:9861003, 2004.\\
Peterson \& Fabian. Phys. Rep., 427:1-39, 2006.\\
Salom\'e \& Combes A\&A, 489:101-104, 2008.\\
Sanders, et al. MNRAS, 402:127-144, 2010.\\
Sargsyan, et al. ApJ, 683:114122, 2008.\\
Schawinski, et al. ApJ, 690:1672-1680, 2009.\\
Solomon, et al. ApJ, 478:144 1997.\\
Springel, et al. MNRAS, 361:776+, 2005.

\begin{figure}[htp]
  \begin{center}
    \begin{minipage}{0.495\linewidth}    
      \includegraphics*[width=\textwidth, trim=45mm 5mm 47mm 6mm, clip]{whiskers}
    \end{minipage}
    \begin{minipage}{0.495\linewidth}
      \includegraphics*[width=\textwidth, trim=59mm 18mm 52mm 10mm, clip]{beam}
    \end{minipage}
    \caption{{\bf{Left:}} HST $I$-band image of the \irs\ BCG. Cyan
      arrows highlight ``whiskers'' of possible cold gas filaments;
      yellow circles enclose (stellar?) spheroids at the same redshift
      of \irs; green dashed line marks on-going AGN outburst
      axis. {\bf{Right}}: HST $V$-band image of \irs\ BCG. Black
      circles denote IRAM 30 m half power beam widths for the proposed
      observing frequencies; for reference, yellow circles from the
      left panel are shown.}
    \label{fig:i09}
    \begin{minipage}{\linewidth}
      \includegraphics*[width=\textwidth, trim=28mm 10mm 9mm 50mm, clip]{iram_mh2.eps}
    \end{minipage}
    \caption{EMIR CO brightness temperature, rms noise, and
      \htoo\ mass as a function of position switched observing
      time. Solid lines are $I_{\mathrm{CO}}$ $3\sigma$ upper limits;
      dashed-dot lines are $T_{\mathrm{rms}}$ specific to elevation \&
      $\nu_{\mathrm{obs}}$; dashed lines are \mht\ $3\sigma$ upper
      limits; downward arrows are $3\sigma$ \mht\ upper limits from
      Evans, et al. 1998 (adjusted to our cosmology and
      $I_{\mathrm{CO}}$:\mht\ assumptions). The CO(1-0) and CO(2-1)
      calculations include the 16 K and 6 K $T_{\mathrm{sys}}$
      increases, respectively, from use of the E0/E1 dichroic. All
      calculations assumed 7 mm of precipitable water vapor, $\Delta
      v_{\mathrm{res}} = 100$ km s$^{-1}$, and $\vco = 500$ km
      s$^{-1}$.}
    \label{fig:feas}
  \end{center}
\end{figure}

%%%%%%%%%%%%%%
\end{document}
%%%%%%%%%%%%%%

%% Additionally, the existence of correlations between AGN outburst
%% ages and host halo cooling times indicates the presence of a
%% finely-tuned feedback loop [{\it{i.e.}} Cavagnolo, et al. 2008,
%% Rafferty, et al. 2008]. But how AGN feedback energy is thermalized,
%% specifically on scales the size of the host galaxy, and directed to
%% gas with the shortest cooling times is still unclear [see McNamara
%% \& Nulsen. 2007, for a review]. Sparse observational evidence
%% suggests weak shocks, sound waves, and conduction may be
%% responsible [Fabian, et al. 2006, Forman, et al. 2007, Voit et
%% al. 2008].

%% Comparions between Perseus and IRAS09:
%%   -- similar Mbh (0.3e9)
%%   -- similar mech power (5d44)
%%   -- similar optical line emi (few d42)
%%   -- similar Mdot (~500 Msol)
%%   -- BUT, Perseus has 5d7 Msol dust, I09 has 2d8
%%   -- I09 has almost 10 times the dust content, but is undetected in
%%      CO. The big difference between I09 and Per, a QSO in I09.
%%   -- so is the lack of mol gas a result of the QSO?
%%   -- need a tight upper limit, or CO detection, to determine this.
%%   -- is this perseus at z=0.4?

%% J10, 115.2 GHz, red 79.9, E090
%% J21, 230.5 GHz, red 159.9, E150
%% J32, 345.7 GHz, red 239.8, E230
%% J43, 461.0 GHz, red 319.7, E300

%% The dichroics are needed for dual-band observations with EMIR.
%% **** on average, increases sys temp by 10-15 K ****
%% can combine:
%% E090 \& E150
%% E090 \& E230
%% E150 \& E330

%% E0/E2: 79.9+239.8, 2.5\% and 5\%
%% E0/E1: 79.9+159.9, 4\% and 1.5\%
%% E1/E3: 159.9+319.7, 3.5\% and 2.5\%
